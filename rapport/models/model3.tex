% MODEL 3

One drawback of reserve requirement is that they may result in highly volatile energy prices when the system is strained. High prices indicate scarcity and the need for investment in capacity. They are therefore desirable, because they trigger investment. However, in systems with limited elasticity, energy price spikes result in enormous price volatility, which increases the risk of investment. Operating reserve demand curves have been proposed as an approach for achieving high energy prices in conditions of scarcity through prices spikes in that are more frequent but less elevated. This is achieved by expressing reserve requirement through a demand function, rather than a hard constraint that needs to be satisfied. Our third model implement the economic dispatch with an operating reserve demand curve and we will see if this model correctly fits the Belgium market.
But first and foremost, let's build the operating reserve demand curve. It's centered around the fixed reserve requirement of previous models and has a width which is equal to $20 \%$ of the national peak demand. The curve is depicted on figure [\ref{fig:ReserveCurve}]. We see that the breakpoint is around $16$, then the slope is about $-1.83$, to finally reach the x-axis at $2700$ MW. The figure [\ref{fig:ReserveUtility}] shows the utility cost of the reserve (integration of the marginal utility).

\begin{minipage}{0.495\textwidth} 
\begin{figure}[H]
    \centering
    \setlength\fheight{3cm}
    \setlength\fwidth{0.8\textwidth}
    % This file was created by matlab2tikz.
% Minimal pgfplots version: 1.3
%
%The latest updates can be retrieved from
%  http://www.mathworks.com/matlabcentral/fileexchange/22022-matlab2tikz
%where you can also make suggestions and rate matlab2tikz.
%
\definecolor{mycolor1}{rgb}{0.84706,0.16078,0.00000}%
\definecolor{mycolor2}{rgb}{0.87059,0.49020,0.00000}%
%
\begin{tikzpicture}

\begin{axis}[%
width=\fwidth,
height=\fheight,
at={(0\fwidth,0\fheight)},
scale only axis,
separate axis lines,
every outer x axis line/.append style={black},
every x tick label/.append style={font=\color{black}},
xmin=0,
xmax=100,
xtick={0,16,50,100},
xlabel={Reserve [MW]},
xmajorgrids,
every outer y axis line/.append style={black},
every y tick label/.append style={font=\color{black}},
ymin=4860,
ymax=5030,
ytick={5000},
ylabel={Marginal Utility [euro/MWh]},
ymajorgrids
]
\addplot [color=mycolor1,solid,line width=2.0pt,forget plot]
  table[row sep=crcr]{%
0	5000\\
16.0972929999998	5000\\
};
\addplot [color=mycolor2,dash pattern=on 1pt off 3pt on 3pt off 3pt,line width=2.0pt,forget plot]
  table[row sep=crcr]{%
16.0972929999998	4860.27302946636\\
16.0972929999998	5000\\
};
\addplot [color=mycolor1,solid,line width=2.0pt,forget plot]
  table[row sep=crcr]{%
16.0972929999998	5000\\
17.0972929999998	4998.13697372622\\
18.0972929999998	4996.27394745244\\
19.0972929999998	4994.41092117865\\
20.0972929999998	4992.54789490487\\
21.0972929999998	4990.68486863109\\
22.0972929999998	4988.82184235731\\
23.0972929999998	4986.9588160833\\
24.0972929999998	4985.09578980974\\
25.0972929999998	4983.23276353596\\
26.0972929999998	4981.36973726218\\
27.0972929999998	4979.5067109884\\
28.0972929999998	4977.64368471462\\
29.0972929999998	4975.78065844084\\
30.0972929999998	4973.91763216705\\
31.0972929999998	4972.05460589327\\
32.0972929999998	4970.19157961949\\
33.0972929999998	4968.32855334571\\
34.0972929999998	4966.46552707193\\
35.0972929999998	4964.60250079814\\
36.0972929999998	4962.73947452436\\
37.0972929999998	4960.87644825058\\
38.0972929999998	4959.0134219768\\
39.0972929999998	4957.15039570302\\
40.0972929999998	4955.28736942924\\
41.0972929999998	4953.42434315545\\
42.0972929999998	4951.56131688167\\
43.0972929999998	4949.69829060789\\
44.0972929999998	4947.83526433411\\
45.0972929999998	4945.97223806033\\
46.0972929999998	4944.10921178654\\
47.0972929999998	4942.24618551276\\
48.0972929999998	4940.38315923898\\
49.0972929999998	4938.5201329652\\
50.0972929999998	4936.65710669142\\
51.0972929999998	4934.79408041764\\
52.0972929999998	4932.93105414385\\
53.0972929999998	4931.06802787007\\
54.0972929999998	4929.20500159629\\
55.0972929999998	4927.34197532251\\
56.0972929999998	4925.47894904873\\
57.0972929999998	4923.61592277494\\
58.0972929999998	4921.75289650116\\
59.0972929999998	4919.88987022738\\
60.0972929999998	4918.0268439536\\
61.0972929999998	4916.16381767982\\
62.0972929999998	4914.30079140603\\
63.0972929999998	4912.43776513225\\
64.0972929999998	4910.57473885847\\
65.0972929999998	4908.71171258469\\
66.0972929999998	4906.84868631091\\
};
\addplot [color=mycolor1,dashed,line width=2.0pt,forget plot]
  table[row sep=crcr]{%
66.0972929999998	4906.84868631091\\
67.0972929999998	4904.98566003712\\
68.0972929999998	4903.12263376334\\
69.0972929999998	4901.25960748956\\
70.0972929999998	4899.39658121578\\
71.0972929999998	4897.533554942\\
72.0972929999998	4895.67052866822\\
73.0972929999998	4893.80750239443\\
74.0972929999998	4891.94447612065\\
75.0972929999998	4890.08144984687\\
76.0972929999998	4888.21842357309\\
77.0972929999998	4886.35539729931\\
78.0972929999998	4884.49237102552\\
79.0972929999998	4882.62934475174\\
80.0972929999998	4880.76631847796\\
81.0972929999998	4878.90329220418\\
82.0972929999998	4877.0402659304\\
83.0972929999998	4875.1772396662\\
84.0972929999998	4873.31421338283\\
85.0972929999998	4871.45118710905\\
86.0972929999998	4869.58816083527\\
87.0972929999998	4867.72513456149\\
88.0972929999998	4865.86210828771\\
89.0972929999998	4863.99908201392\\
90.0972929999998	4862.13605574014\\
91.0972929999998	4860.27302946636\\
};
\end{axis}
\end{tikzpicture}%
    \caption{Marginal utility of reserves}
    \label{fig:ReserveCurve}
\end{figure}
\end{minipage}
\begin{minipage}{0.495\textwidth} 
\begin{figure}[H]
    \centering
    \setlength\fheight{3cm}
    \setlength\fwidth{0.8\textwidth}
    % This file was created by matlab2tikz.
% Minimal pgfplots version: 1.3
%
%The latest updates can be retrieved from
%  http://www.mathworks.com/matlabcentral/fileexchange/22022-matlab2tikz
%where you can also make suggestions and rate matlab2tikz.
%
\definecolor{mycolor1}{rgb}{0.04314,0.51765,0.78039}%
\definecolor{mycolor2}{rgb}{0.84706,0.16078,0.00000}%
%
\begin{tikzpicture}

\begin{axis}[%
width=\fwidth,
height=\fheight,
at={(0\fwidth,0\fheight)},
scale only axis,
separate axis lines,
every outer x axis line/.append style={black},
every x tick label/.append style={font=\color{black}},
xmin=0,
xmax=3200,
xlabel={Reserve [MW]},
xtick={0,1000,2000,2700},
xmajorgrids,
every outer y axis line/.append style={black},
every y tick label/.append style={font=\color{black}},
ymin=0,
ymax=7000000,
ylabel={Utility [\euro]},
ymajorgrids
]
\addplot [color=mycolor2,solid,line width=2.0pt,forget plot]
  table[row sep=crcr]{%
0	0\\
16.0972929999998	80486.464999999\\
};
\addplot [color=mycolor2,solid,line width=2.0pt,forget plot]
  table[row sep=crcr]{%
16.0972929999998	80728.841331323\\
17.0972929999998	85727.9098181865\\
18.0972929999998	90725.1152787758\\
19.0972929999998	95720.4577130913\\
20.0972929999998	100713.937121133\\
21.0972929999998	105705.553502901\\
22.0972929999998	110695.306858395\\
23.0972929999998	115683.197187616\\
24.0972929999998	120669.224490562\\
25.0972929999998	125653.388767235\\
26.0972929999998	130635.690017634\\
27.0972929999998	135616.12824176\\
28.0972929999998	140594.703439611\\
29.0972929999998	145571.415611189\\
30.0972929999998	150546.264756493\\
31.0972929999998	155519.250875523\\
32.0972929999998	160490.373968279\\
33.0972929999998	165459.634034762\\
34.0972929999998	170427.031074971\\
35.0972929999998	175392.565088906\\
36.0972929999998	180356.236076567\\
37.0972929999998	185318.044037954\\
38.0972929999998	190277.988973068\\
39.0972929999998	195236.070881908\\
40.0972929999998	200192.289764474\\
41.0972929999998	205146.645620767\\
42.0972929999998	210099.138450785\\
43.0972929999998	215049.76825453\\
44.0972929999998	219998.53503001\\
45.0972929999998	224945.438783198\\
46.0972929999998	229890.479508121\\
47.0972929999998	234833.657206771\\
48.0972929999998	239774.971879147\\
49.0972929999998	244714.423525249\\
50.0972929999998	249652.012145077\\
51.0972929999998	254587.737738632\\
52.0972929999998	259521.600305913\\
53.0972929999998	264453.59984692\\
54.0972929999998	269383.736361653\\
55.0972929999998	274312.009850112\\
56.0972929999998	279238.420312298\\
57.0972929999998	284162.96774821\\
58.0972929999998	289085.652157848\\
59.0972929999998	294006.473541212\\
60.0972929999998	298925.431898303\\
61.0972929999998	303842.527229119\\
62.0972929999998	308757.759533662\\
63.0972929999998	313671.128811931\\
64.0972929999998	318582.635063927\\
65.0972929999998	323492.278289648\\
66.0972929999998	328400.058489096\\
67.0972929999998	333305.97566227\\
68.0972929999998	338210.02980917\\
69.0972929999998	343112.220929797\\
70.0972929999998	348012.549024149\\
71.0972929999998	352911.014092228\\
72.0972929999998	357807.616134033\\
73.0972929999998	362702.355149565\\
74.0972929999998	367595.231138822\\
75.0972929999998	372486.244101806\\
76.0972929999998	377375.394038516\\
77.0972929999998	382262.680948952\\
78.0972929999998	387148.104833115\\
79.0972929999998	392031.665691003\\
80.0972929999998	396913.363522618\\
81.0972929999998	401793.198327959\\
82.0972929999998	406671.170107026\\
83.0972929999998	411547.27885982\\
84.0972929999998	416421.52458634\\
85.0972929999998	421293.907286586\\
86.0972929999998	426164.426960558\\
87.0972929999998	431033.083608256\\
88.0972929999998	435899.877229681\\
89.0972929999998	440764.807824832\\
90.0972929999998	445627.875393709\\
91.0972929999998	450489.079936312\\
92.0972929999998	455348.421452641\\
93.0972929999998	460205.899942697\\
94.0972929999998	465061.515406479\\
95.0972929999998	469915.267843987\\
96.0972929999998	474767.157255221\\
97.0972929999998	479617.183640182\\
98.0972929999998	484465.346998869\\
99.0972929999998	489311.647331282\\
100.097293	494156.084637421\\
101.097293	498998.658917286\\
102.097293	503839.370170878\\
103.097293	508678.218398196\\
104.097293	513515.20359924\\
105.097293	518350.32577401\\
106.097293	523183.584922507\\
107.097293	528014.98104473\\
108.097293	532844.514140678\\
109.097293	537672.184210354\\
110.097293	542497.991253755\\
111.097293	547321.935270883\\
112.097293	552144.016261737\\
113.097293	556964.234226317\\
114.097293	561782.589164623\\
115.097293	566599.081076655\\
116.097293	571413.709962414\\
117.097293	576226.475821899\\
118.097293	581037.37865511\\
119.097293	585846.418462047\\
120.097293	590653.595242711\\
121.097293	595458.908997101\\
122.097293	600262.359725217\\
123.097293	605063.947427059\\
124.097293	609863.672102628\\
125.097293	614661.533751922\\
126.097293	619457.532374943\\
127.097293	624251.66797169\\
128.097293	629043.940542164\\
129.097293	633834.350086363\\
130.097293	638622.896604289\\
131.097293	643409.580095941\\
132.097293	648194.400561319\\
133.097293	652977.358000423\\
134.097293	657758.452413254\\
135.097293	662537.683799811\\
136.097293	667315.052160094\\
137.097293	672090.557494103\\
138.097293	676864.199801839\\
139.097293	681635.9790833\\
140.097293	686405.895338488\\
141.097293	691173.948567403\\
142.097293	695940.138770043\\
143.097293	700704.46594641\\
144.097293	705466.930096502\\
145.097293	710227.531220321\\
146.097293	714986.269317867\\
147.097293	719743.144389138\\
148.097293	724498.156434136\\
149.097293	729251.30545286\\
150.097293	734002.59144531\\
151.097293	738752.014411486\\
152.097293	743499.574351389\\
153.097293	748245.271265017\\
154.097293	752989.105152373\\
155.097293	757731.076013454\\
156.097293	762471.183848261\\
157.097293	767209.428656795\\
158.097293	771945.810439055\\
159.097293	776680.329195041\\
160.097293	781412.984924753\\
161.097293	786143.777628192\\
162.097293	790872.707305356\\
163.097293	795599.773956247\\
164.097293	800324.977580864\\
165.097293	805048.318179208\\
166.097293	809769.795751277\\
167.097293	814489.410297073\\
168.097293	819207.161816595\\
169.097293	823923.050309844\\
170.097293	828637.075776818\\
171.097293	833349.238217519\\
172.097293	838059.537631946\\
173.097293	842767.974020099\\
174.097293	847474.547381978\\
175.097293	852179.257717584\\
176.097293	856882.105026916\\
177.097293	861583.089309974\\
178.097293	866282.210566758\\
179.097293	870979.468797268\\
180.097293	875674.864001505\\
181.097293	880368.396179468\\
182.097293	885060.065331157\\
183.097293	889749.871456572\\
184.097293	894437.814555714\\
185.097293	899123.894628582\\
186.097293	903808.111675175\\
187.097293	908490.465695496\\
188.097293	913170.956689542\\
189.097293	917849.584657315\\
190.097293	922526.349598814\\
191.097293	927201.251514039\\
192.097293	931874.29040299\\
193.097293	936545.466265667\\
194.097293	941214.779102071\\
195.097293	945882.228912201\\
196.097293	950547.815696057\\
197.097293	955211.53945364\\
198.097293	959873.400184948\\
199.097293	964533.397889983\\
200.097293	969191.532568744\\
201.097293	973847.804221231\\
202.097293	978502.212847445\\
203.097293	983154.758447384\\
204.097293	987805.44102105\\
205.097293	992454.260568442\\
206.097293	997101.217089561\\
207.097293	1001746.31058441\\
208.097293	1006389.54105298\\
209.097293	1011030.90849527\\
210.097293	1015670.4129113\\
211.097293	1020308.05430105\\
212.097293	1024943.83266452\\
213.097293	1029577.74800172\\
214.097293	1034209.80031265\\
215.097293	1038839.98959731\\
216.097293	1043468.31585569\\
217.097293	1048094.77908779\\
218.097293	1052719.37929363\\
219.097293	1057342.11647318\\
220.097293	1061962.99062647\\
221.097293	1066582.00175348\\
222.097293	1071199.14985422\\
223.097293	1075814.43492868\\
224.097293	1080427.85697687\\
225.097293	1085039.41599879\\
226.097293	1089649.11199443\\
227.097293	1094256.9449638\\
228.097293	1098862.9149069\\
229.097293	1103467.02182372\\
230.097293	1108069.26571427\\
231.097293	1112669.64657854\\
232.097293	1117268.16441654\\
233.097293	1121864.81922827\\
234.097293	1126459.61101372\\
235.097293	1131052.5397729\\
236.097293	1135643.6055058\\
237.097293	1140232.80821243\\
238.097293	1144820.14789279\\
239.097293	1149405.62454687\\
240.097293	1153989.23817468\\
241.097293	1158570.98877622\\
242.097293	1163150.87635148\\
243.097293	1167728.90090047\\
244.097293	1172305.06242319\\
245.097293	1176879.36091963\\
246.097293	1181451.79638979\\
247.097293	1186022.36883369\\
248.097293	1190591.07825131\\
249.097293	1195157.92464265\\
250.097293	1199722.90800772\\
251.097293	1204286.02834652\\
252.097293	1208847.28565905\\
253.097293	1213406.6799453\\
254.097293	1217964.21120527\\
255.097293	1222519.87943898\\
256.097293	1227073.68464641\\
257.097293	1231625.62682756\\
258.097293	1236175.70598244\\
259.097293	1240723.92211105\\
260.097293	1245270.27521339\\
261.097293	1249814.76528945\\
262.097293	1254357.39233923\\
263.097293	1258898.15636274\\
264.097293	1263437.05735998\\
265.097293	1267974.09533095\\
266.097293	1272509.27027564\\
267.097293	1277042.58219406\\
268.097293	1281574.0310862\\
269.097293	1286103.61695207\\
270.097293	1290631.33979167\\
271.097293	1295157.19960499\\
272.097293	1299681.19639204\\
273.097293	1304203.33015281\\
274.097293	1308723.60088732\\
275.097293	1313242.00859554\\
276.097293	1317758.5532775\\
277.097293	1322273.23493318\\
278.097293	1326786.05356258\\
279.097293	1331297.00916571\\
280.097293	1335806.10174257\\
281.097293	1340313.33129316\\
282.097293	1344818.69781747\\
283.097293	1349322.20131551\\
284.097293	1353823.84178727\\
285.097293	1358323.61923276\\
286.097293	1362821.53365197\\
287.097293	1367317.58504492\\
288.097293	1371811.77341158\\
289.097293	1376304.09875198\\
290.097293	1380794.5610661\\
291.097293	1385283.16035395\\
292.097293	1389769.89661552\\
293.097293	1394254.76985082\\
294.097293	1398737.78005984\\
295.097293	1403218.9272426\\
296.097293	1407698.21139907\\
297.097293	1412175.63252928\\
298.097293	1416651.19063321\\
299.097293	1421124.88571087\\
300.097293	1425596.71776225\\
301.097293	1430066.68678736\\
302.097293	1434534.79278619\\
303.097293	1439001.03575875\\
304.097293	1443465.41570504\\
305.097293	1447927.93262506\\
306.097293	1452388.5865188\\
307.097293	1456847.37738626\\
308.097293	1461304.30522746\\
309.097293	1465759.37004237\\
310.097293	1470212.57183102\\
311.097293	1474663.91059339\\
312.097293	1479113.38632949\\
313.097293	1483560.99903931\\
314.097293	1488006.74872286\\
315.097293	1492450.63538014\\
316.097293	1496892.65901114\\
317.097293	1501332.81961587\\
318.097293	1505771.11719432\\
319.097293	1510207.5517465\\
320.097293	1514642.12327241\\
321.097293	1519074.83177204\\
322.097293	1523505.6772454\\
323.097293	1527934.65969249\\
324.097293	1532361.7791133\\
325.097293	1536787.03550784\\
326.097293	1541210.4288761\\
327.097293	1545631.9592181\\
328.097293	1550051.62653381\\
329.097293	1554469.43082326\\
330.097293	1558885.37208643\\
331.097293	1563299.45032332\\
332.097293	1567711.66553394\\
333.097293	1572122.01771829\\
334.097293	1576530.50687637\\
335.097293	1580937.13300817\\
336.097293	1585341.89611369\\
337.097293	1589744.79619294\\
338.097293	1594145.83324592\\
339.097293	1598545.00727263\\
340.097293	1602942.31827306\\
341.097293	1607337.76624722\\
342.097293	1611731.3511951\\
343.097293	1616123.07311671\\
344.097293	1620512.93201205\\
345.097293	1624900.92788111\\
346.097293	1629287.0607239\\
347.097293	1633671.33054042\\
348.097293	1638053.73733066\\
349.097293	1642434.28109463\\
350.097293	1646812.96183232\\
351.097293	1651189.77954374\\
352.097293	1655564.73422889\\
353.097293	1659937.82588776\\
354.097293	1664309.05452036\\
355.097293	1668678.42012668\\
356.097293	1673045.92270673\\
357.097293	1677411.56226051\\
358.097293	1681775.33878801\\
359.097293	1686137.25228924\\
360.097293	1690497.3027642\\
361.097293	1694855.49021288\\
362.097293	1699211.81463529\\
363.097293	1703566.27603142\\
364.097293	1707918.87440128\\
365.097293	1712269.60974487\\
366.097293	1716618.48206218\\
367.097293	1720965.49135322\\
368.097293	1725310.63761799\\
369.097293	1729653.92085648\\
370.097293	1733995.3410687\\
371.097293	1738334.89825464\\
372.097293	1742672.59241431\\
373.097293	1747008.42354771\\
374.097293	1751342.39165483\\
375.097293	1755674.49673568\\
376.097293	1760004.73879026\\
377.097293	1764333.11781856\\
378.097293	1768659.63382059\\
379.097293	1772984.28679634\\
380.097293	1777307.07674582\\
381.097293	1781628.00366903\\
382.097293	1785947.06756596\\
383.097293	1790264.26843662\\
384.097293	1794579.60628101\\
385.097293	1798893.08109912\\
386.097293	1803204.69289096\\
387.097293	1807514.44165652\\
388.097293	1811822.32739581\\
389.097293	1816128.35010883\\
390.097293	1820432.50979557\\
391.097293	1824734.80645604\\
392.097293	1829035.24009023\\
393.097293	1833333.81069815\\
394.097293	1837630.5182798\\
395.097293	1841925.36283517\\
396.097293	1846218.34436427\\
397.097293	1850509.4628671\\
398.097293	1854798.71834365\\
399.097293	1859086.11079393\\
400.097293	1863371.64021793\\
401.097293	1867655.30661567\\
402.097293	1871937.10998712\\
403.097293	1876217.05033231\\
404.097293	1880495.12765122\\
405.097293	1884771.34194385\\
406.097293	1889045.69321021\\
407.097293	1893318.1814503\\
408.097293	1897588.80666412\\
409.097293	1901857.56885166\\
410.097293	1906124.46801292\\
411.097293	1910389.50414792\\
412.097293	1914652.67725664\\
413.097293	1918913.98733908\\
414.097293	1923173.43439525\\
415.097293	1927431.01842515\\
416.097293	1931686.73942877\\
417.097293	1935940.59740613\\
418.097293	1940192.5923572\\
419.097293	1944442.724282\\
420.097293	1948690.99318053\\
421.097293	1952937.39905279\\
422.097293	1957181.94189877\\
423.097293	1961424.62171848\\
424.097293	1965665.43851191\\
425.097293	1969904.39227907\\
426.097293	1974141.48301996\\
427.097293	1978376.71073457\\
428.097293	1982610.07542291\\
429.097293	1986841.57708497\\
430.097293	1991071.21572077\\
431.097293	1995298.99133028\\
432.097293	1999524.90391353\\
433.097293	2003748.9534705\\
434.097293	2007971.14000119\\
435.097293	2012191.46350562\\
436.097293	2016409.92398376\\
437.097293	2020626.52143564\\
438.097293	2024841.25586124\\
439.097293	2029054.12726057\\
440.097293	2033265.13563362\\
441.097293	2037474.2809804\\
442.097293	2041681.56330091\\
443.097293	2045886.98259514\\
444.097293	2050090.5388631\\
445.097293	2054292.23210478\\
446.097293	2058492.06232019\\
447.097293	2062690.02950933\\
448.097293	2066886.13367219\\
449.097293	2071080.37480878\\
450.097293	2075272.7529191\\
451.097293	2079463.26800314\\
452.097293	2083651.9200609\\
453.097293	2087838.7090924\\
454.097293	2092023.63509762\\
455.097293	2096206.69807657\\
456.097293	2100387.89802924\\
457.097293	2104567.23495564\\
458.097293	2108744.70885576\\
459.097293	2112920.31972962\\
460.097293	2117094.06757719\\
461.097293	2121265.9523985\\
462.097293	2125435.97419353\\
463.097293	2129604.13296228\\
464.097293	2133770.42870477\\
465.097293	2137934.86142098\\
466.097293	2142097.43111091\\
467.097293	2146258.13777457\\
468.097293	2150416.98141196\\
469.097293	2154573.96202307\\
470.097293	2158729.07960791\\
471.097293	2162882.33416648\\
472.097293	2167033.72569877\\
473.097293	2171183.25420479\\
474.097293	2175330.91968453\\
475.097293	2179476.72213801\\
476.097293	2183620.6615652\\
477.097293	2187762.73796613\\
478.097293	2191902.95134078\\
479.097293	2196041.30168915\\
480.097293	2200177.78901125\\
481.097293	2204312.41330708\\
482.097293	2208445.17457664\\
483.097293	2212576.07281992\\
484.097293	2216705.10803692\\
485.097293	2220832.28022766\\
486.097293	2224957.58939212\\
487.097293	2229081.0355303\\
488.097293	2233202.61864221\\
489.097293	2237322.33872785\\
490.097293	2241440.19578722\\
491.097293	2245556.18982031\\
492.097293	2249670.32082712\\
493.097293	2253782.58880767\\
494.097293	2257892.99376194\\
495.097293	2262001.53568993\\
496.097293	2266108.21459165\\
497.097293	2270213.0304671\\
498.097293	2274315.98331628\\
499.097293	2278417.07313918\\
500.097293	2282516.2999358\\
501.097293	2286613.66370615\\
502.097293	2290709.16445023\\
503.097293	2294802.80216804\\
504.097293	2298894.57685957\\
505.097293	2302984.48852483\\
506.097293	2307072.53716381\\
507.097293	2311158.72277652\\
508.097293	2315243.04536296\\
509.097293	2319325.50492312\\
510.097293	2323406.10145701\\
511.097293	2327484.83496462\\
512.097293	2331561.70544596\\
513.097293	2335636.71290103\\
514.097293	2339709.85732982\\
515.097293	2343781.13873235\\
516.097293	2347850.55710859\\
517.097293	2351918.11245856\\
518.097293	2355983.80478226\\
519.097293	2360047.63407969\\
520.097293	2364109.60035084\\
521.097293	2368169.70359571\\
522.097293	2372227.94381432\\
523.097293	2376284.32100665\\
524.097293	2380338.8351727\\
525.097293	2384391.48631248\\
526.097293	2388442.27442599\\
527.097293	2392491.19951323\\
528.097293	2396538.26157419\\
529.097293	2400583.46060887\\
530.097293	2404626.79661729\\
531.097293	2408668.26959943\\
532.097293	2412707.87955529\\
533.097293	2416745.62648488\\
534.097293	2420781.5103882\\
535.097293	2424815.53126525\\
536.097293	2428847.68911602\\
537.097293	2432877.98394051\\
538.097293	2436906.41573874\\
539.097293	2440932.98451068\\
540.097293	2444957.69025636\\
541.097293	2448980.53297576\\
542.097293	2453001.51266889\\
543.097293	2457020.62933574\\
544.097293	2461037.88297632\\
545.097293	2465053.27359063\\
546.097293	2469066.80117866\\
547.097293	2473078.46574042\\
548.097293	2477088.26727591\\
549.097293	2481096.20578512\\
550.097293	2485102.28126805\\
551.097293	2489106.49372472\\
552.097293	2493108.84315511\\
553.097293	2497109.32955922\\
554.097293	2501107.95293707\\
555.097293	2505104.71328863\\
556.097293	2509099.61061393\\
557.097293	2513092.64491295\\
558.097293	2517083.8161857\\
559.097293	2521073.12443217\\
560.097293	2525060.56965237\\
561.097293	2529046.1518463\\
562.097293	2533029.87101395\\
563.097293	2537011.72715533\\
564.097293	2540991.72027043\\
565.097293	2544969.85035926\\
566.097293	2548946.11742182\\
567.097293	2552920.5214581\\
568.097293	2556893.06246811\\
569.097293	2560863.74045184\\
570.097293	2564832.55540931\\
571.097293	2568799.50734049\\
572.097293	2572764.59624541\\
573.097293	2576727.82212405\\
574.097293	2580689.18497642\\
575.097293	2584648.68480251\\
576.097293	2588606.32160233\\
577.097293	2592562.09537587\\
578.097293	2596516.00612314\\
579.097293	2600468.05384414\\
580.097293	2604418.23853887\\
581.097293	2608366.56020732\\
582.097293	2612313.01884949\\
583.097293	2616257.61446539\\
584.097293	2620200.34705502\\
585.097293	2624141.21661838\\
586.097293	2628080.22315546\\
587.097293	2632017.36666627\\
588.097293	2635952.6471508\\
589.097293	2639886.06460906\\
590.097293	2643817.61904105\\
591.097293	2647747.31044676\\
592.097293	2651675.1388262\\
593.097293	2655601.10417936\\
594.097293	2659525.20650625\\
595.097293	2663447.44580687\\
596.097293	2667367.82208121\\
597.097293	2671286.33532928\\
598.097293	2675202.98555108\\
599.097293	2679117.7727466\\
600.097293	2683030.69691585\\
601.097293	2686941.75805883\\
602.097293	2690850.95617553\\
603.097293	2694758.29126595\\
604.097293	2698663.76333011\\
605.097293	2702567.37236799\\
606.097293	2706469.11837959\\
607.097293	2710369.00136492\\
608.097293	2714267.02132398\\
609.097293	2718163.17825677\\
610.097293	2722057.47216328\\
611.097293	2725949.90304351\\
612.097293	2729840.47089747\\
613.097293	2733729.17572516\\
614.097293	2737616.01752658\\
615.097293	2741500.99630172\\
616.097293	2745384.11205059\\
617.097293	2749265.36477318\\
618.097293	2753144.7544695\\
619.097293	2757022.28113955\\
620.097293	2760897.94478332\\
621.097293	2764771.74540082\\
622.097293	2768643.68299205\\
623.097293	2772513.757557\\
624.097293	2776381.96909567\\
625.097293	2780248.31760808\\
626.097293	2784112.80309421\\
627.097293	2787975.42555406\\
628.097293	2791836.18498765\\
629.097293	2795695.08139496\\
630.097293	2799552.11477599\\
631.097293	2803407.28513075\\
632.097293	2807260.59245924\\
633.097293	2811112.03676145\\
634.097293	2814961.61803739\\
635.097293	2818809.33628706\\
636.097293	2822655.19151045\\
637.097293	2826499.18370757\\
638.097293	2830341.31287841\\
639.097293	2834181.57902298\\
640.097293	2838019.98214128\\
641.097293	2841856.5222333\\
642.097293	2845691.19929905\\
643.097293	2849524.01333853\\
644.097293	2853354.96435173\\
645.097293	2857184.05233866\\
646.097293	2861011.27729931\\
647.097293	2864836.63923369\\
648.097293	2868660.1381418\\
649.097293	2872481.77402363\\
650.097293	2876301.54687919\\
651.097293	2880119.45670848\\
652.097293	2883935.50351149\\
653.097293	2887749.68728823\\
654.097293	2891562.00803869\\
655.097293	2895372.46576288\\
656.097293	2899181.0604608\\
657.097293	2902987.79213244\\
658.097293	2906792.66077781\\
659.097293	2910595.66639691\\
660.097293	2914396.80898973\\
661.097293	2918196.08855627\\
662.097293	2921993.50509655\\
663.097293	2925789.05861055\\
664.097293	2929582.74909827\\
665.097293	2933374.57655973\\
666.097293	2937164.54099491\\
667.097293	2940952.64240381\\
668.097293	2944738.88078644\\
669.097293	2948523.2561428\\
670.097293	2952305.76847288\\
671.097293	2956086.41777669\\
672.097293	2959865.20405423\\
673.097293	2963642.12730549\\
674.097293	2967417.18753048\\
675.097293	2971190.38472919\\
676.097293	2974961.71890163\\
677.097293	2978731.1900478\\
678.097293	2982498.79816769\\
679.097293	2986264.54326131\\
680.097293	2990028.42532866\\
681.097293	2993790.44436973\\
682.097293	2997550.60038453\\
683.097293	3001308.89337305\\
684.097293	3005065.32333531\\
685.097293	3008819.89027128\\
686.097293	3012572.59418099\\
687.097293	3016323.43506441\\
688.097293	3020072.41292157\\
689.097293	3023819.52775245\\
690.097293	3027564.77955706\\
691.097293	3031308.16833539\\
692.097293	3035049.69408745\\
693.097293	3038789.35681324\\
694.097293	3042527.15651275\\
695.097293	3046263.09318599\\
696.097293	3049997.16683296\\
697.097293	3053729.37745365\\
698.097293	3057459.72504807\\
699.097293	3061188.20961621\\
700.097293	3064914.83115808\\
701.097293	3068639.58967368\\
702.097293	3072362.485163\\
703.097293	3076083.51762605\\
704.097293	3079802.68706282\\
705.097293	3083519.99347332\\
706.097293	3087235.43685755\\
707.097293	3090949.01721551\\
708.097293	3094660.73454719\\
709.097293	3098370.58885259\\
710.097293	3102078.58013172\\
711.097293	3105784.70838458\\
712.097293	3109488.97361117\\
713.097293	3113191.37581148\\
714.097293	3116891.91498551\\
715.097293	3120590.59113328\\
716.097293	3124287.40425477\\
717.097293	3127982.35434998\\
718.097293	3131675.44141893\\
719.097293	3135366.66546159\\
720.097293	3139056.02647799\\
721.097293	3142743.52446811\\
722.097293	3146429.15943196\\
723.097293	3150112.93136953\\
724.097293	3153794.84028083\\
725.097293	3157474.88616585\\
726.097293	3161153.06902461\\
727.097293	3164829.38885708\\
728.097293	3168503.84566329\\
729.097293	3172176.43944322\\
730.097293	3175847.17019687\\
731.097293	3179516.03792426\\
732.097293	3183183.04262537\\
733.097293	3186848.1843002\\
734.097293	3190511.46294876\\
735.097293	3194172.87857105\\
736.097293	3197832.43116707\\
737.097293	3201490.12073681\\
738.097293	3205145.94728027\\
739.097293	3208799.91079746\\
740.097293	3212452.01128838\\
741.097293	3216102.24875303\\
742.097293	3219750.6231914\\
743.097293	3223397.1346035\\
744.097293	3227041.78298932\\
745.097293	3230684.56834887\\
746.097293	3234325.49068215\\
747.097293	3237964.54998915\\
748.097293	3241601.74626988\\
749.097293	3245237.07952433\\
750.097293	3248870.54975251\\
751.097293	3252502.15695442\\
752.097293	3256131.90113005\\
753.097293	3259759.78227941\\
754.097293	3263385.8004025\\
755.097293	3267009.95549931\\
756.097293	3270632.24756985\\
757.097293	3274252.67661412\\
758.097293	3277871.24263211\\
759.097293	3281487.94562382\\
760.097293	3285102.78558927\\
761.097293	3288715.76252844\\
762.097293	3292326.87644133\\
763.097293	3295936.12732795\\
764.097293	3299543.5151883\\
765.097293	3303149.04002238\\
766.097293	3306752.70183018\\
767.097293	3310354.5006117\\
768.097293	3313954.43636695\\
769.097293	3317552.50909593\\
770.097293	3321148.71879864\\
771.097293	3324743.06547507\\
772.097293	3328335.54912523\\
773.097293	3331926.16974911\\
774.097293	3335514.92734672\\
775.097293	3339101.82191806\\
776.097293	3342686.85346312\\
777.097293	3346270.02198191\\
778.097293	3349851.32747443\\
779.097293	3353430.76994067\\
780.097293	3357008.34938064\\
781.097293	3360584.06579433\\
782.097293	3364157.91918175\\
783.097293	3367729.90954289\\
784.097293	3371300.03687777\\
785.097293	3374868.30118637\\
786.097293	3378434.70246869\\
787.097293	3381999.24072474\\
788.097293	3385561.91595452\\
789.097293	3389122.72815802\\
790.097293	3392681.67733525\\
791.097293	3396238.76348621\\
792.097293	3399793.98661089\\
793.097293	3403347.3467093\\
794.097293	3406898.84378143\\
795.097293	3410448.47782729\\
796.097293	3413996.24884688\\
797.097293	3417542.1568402\\
798.097293	3421086.20180723\\
799.097293	3424628.383748\\
800.097293	3428168.70266249\\
801.097293	3431707.15855071\\
802.097293	3435243.75141265\\
803.097293	3438778.48124833\\
804.097293	3442311.34805772\\
805.097293	3445842.35184085\\
806.097293	3449371.49259769\\
807.097293	3452898.77032827\\
808.097293	3456424.18503257\\
809.097293	3459947.7367106\\
810.097293	3463469.42536235\\
811.097293	3466989.25098783\\
812.097293	3470507.21358704\\
813.097293	3474023.31315997\\
814.097293	3477537.54970663\\
815.097293	3481049.92322702\\
816.097293	3484560.43372113\\
817.097293	3488069.08118897\\
818.097293	3491575.86563053\\
819.097293	3495080.78704582\\
820.097293	3498583.84543484\\
821.097293	3502085.04079758\\
822.097293	3505584.37313405\\
823.097293	3509081.84244424\\
824.097293	3512577.44872816\\
825.097293	3516071.19198581\\
826.097293	3519563.07221718\\
827.097293	3523053.08942228\\
828.097293	3526541.24360111\\
829.097293	3530027.53475366\\
830.097293	3533511.96287994\\
831.097293	3536994.52797995\\
832.097293	3540475.23005368\\
833.097293	3543954.06910113\\
834.097293	3547431.04512232\\
835.097293	3550906.15811723\\
836.097293	3554379.40808586\\
837.097293	3557850.79502822\\
838.097293	3561320.31894431\\
839.097293	3564787.97983413\\
840.097293	3568253.77769767\\
841.097293	3571717.71253493\\
842.097293	3575179.78434593\\
843.097293	3578639.99313065\\
844.097293	3582098.33888909\\
845.097293	3585554.82162126\\
846.097293	3589009.44132716\\
847.097293	3592462.19800679\\
848.097293	3595913.09166014\\
849.097293	3599362.12228721\\
850.097293	3602809.28988802\\
851.097293	3606254.59446254\\
852.097293	3609698.0360108\\
853.097293	3613139.61453278\\
854.097293	3616579.33002849\\
855.097293	3620017.18249792\\
856.097293	3623453.17194108\\
857.097293	3626887.29835797\\
858.097293	3630319.56174858\\
859.097293	3633749.96211292\\
860.097293	3637178.49945099\\
861.097293	3640605.17376278\\
862.097293	3644029.98504829\\
863.097293	3647452.93330754\\
864.097293	3650874.01854051\\
865.097293	3654293.2407472\\
866.097293	3657710.59992763\\
867.097293	3661126.09608177\\
868.097293	3664539.72920965\\
869.097293	3667951.49931125\\
870.097293	3671361.40638658\\
871.097293	3674769.45043563\\
872.097293	3678175.63145841\\
873.097293	3681579.94945492\\
874.097293	3684982.40442515\\
875.097293	3688382.99636911\\
876.097293	3691781.72528679\\
877.097293	3695178.5911782\\
878.097293	3698573.59404334\\
879.097293	3701966.7338822\\
880.097293	3705358.01069479\\
881.097293	3708747.42448111\\
882.097293	3712134.97524115\\
883.097293	3715520.66297492\\
884.097293	3718904.48768241\\
885.097293	3722286.44936363\\
886.097293	3725666.54801858\\
887.097293	3729044.78364725\\
888.097293	3732421.15624965\\
889.097293	3735795.66582578\\
890.097293	3739168.31237563\\
891.097293	3742539.09589921\\
892.097293	3745908.01639651\\
893.097293	3749275.07386754\\
894.097293	3752640.2683123\\
895.097293	3756003.59973078\\
896.097293	3759365.06812299\\
897.097293	3762724.67348892\\
898.097293	3766082.41582858\\
899.097293	3769438.29514197\\
900.097293	3772792.31142909\\
901.097293	3776144.46468992\\
902.097293	3779494.75492449\\
903.097293	3782843.18213278\\
904.097293	3786189.7463148\\
905.097293	3789534.44747055\\
906.097293	3792877.28560002\\
907.097293	3796218.26070322\\
908.097293	3799557.37278014\\
909.097293	3802894.62183079\\
910.097293	3806230.00785516\\
911.097293	3809563.53085327\\
912.097293	3812895.19082509\\
913.097293	3816224.98777065\\
914.097293	3819552.92168993\\
915.097293	3822878.99258294\\
916.097293	3826203.20044967\\
917.097293	3829525.54529013\\
918.097293	3832846.02710432\\
919.097293	3836164.64589223\\
920.097293	3839481.40165387\\
921.097293	3842796.29438923\\
922.097293	3846109.32409832\\
923.097293	3849420.49078114\\
924.097293	3852729.79443768\\
925.097293	3856037.23506795\\
926.097293	3859342.81267194\\
927.097293	3862646.52724967\\
928.097293	3865948.37880111\\
929.097293	3869248.36732629\\
930.097293	3872546.49282519\\
931.097293	3875842.75529782\\
932.097293	3879137.15474417\\
933.097293	3882429.69116425\\
934.097293	3885720.36455805\\
935.097293	3889009.17492558\\
936.097293	3892296.12226684\\
937.097293	3895581.20658182\\
938.097293	3898864.42787053\\
939.097293	3902145.78613297\\
940.097293	3905425.28136913\\
941.097293	3908702.91357902\\
942.097293	3911978.68276264\\
943.097293	3915252.58891998\\
944.097293	3918524.63205105\\
945.097293	3921794.81215584\\
946.097293	3925063.12923436\\
947.097293	3928329.5832866\\
948.097293	3931594.17431258\\
949.097293	3934856.90231228\\
950.097293	3938117.7672857\\
951.097293	3941376.76923285\\
952.097293	3944633.90815373\\
953.097293	3947889.18404833\\
954.097293	3951142.59691666\\
955.097293	3954394.14675872\\
956.097293	3957643.8335745\\
957.097293	3960891.65736401\\
958.097293	3964137.61812724\\
959.097293	3967381.7158642\\
960.097293	3970623.95057489\\
961.097293	3973864.3222593\\
962.097293	3977102.83091744\\
963.097293	3980339.47654931\\
964.097293	3983574.2591549\\
965.097293	3986807.17873421\\
966.097293	3990038.23528726\\
967.097293	3993267.42881403\\
968.097293	3996494.75931453\\
969.097293	3999720.22678875\\
970.097293	4002943.8312367\\
971.097293	4006165.57265837\\
972.097293	4009385.45105377\\
973.097293	4012603.4664229\\
974.097293	4015819.61876576\\
975.097293	4019033.90808234\\
976.097293	4022246.33437264\\
977.097293	4025456.89763668\\
978.097293	4028665.59787443\\
979.097293	4031872.43508592\\
980.097293	4035077.40927113\\
981.097293	4038280.52043007\\
982.097293	4041481.76856273\\
983.097293	4044681.15366912\\
984.097293	4047878.67574924\\
985.097293	4051074.33480308\\
986.097293	4054268.13083065\\
987.097293	4057460.06383194\\
988.097293	4060650.13380696\\
989.097293	4063838.34075571\\
990.097293	4067024.68467818\\
991.097293	4070209.16557438\\
992.097293	4073391.78344431\\
993.097293	4076572.53828796\\
994.097293	4079751.43010534\\
995.097293	4082928.45889644\\
996.097293	4086103.62466127\\
997.097293	4089276.92739983\\
998.097293	4092448.36711211\\
999.097293	4095617.94379812\\
1000.097293	4098785.65745786\\
1001.097293	4101951.50809132\\
1002.097293	4105115.49569851\\
1003.097293	4108277.62027942\\
1004.097293	4111437.88183406\\
1005.097293	4114596.28036243\\
1006.097293	4117752.81586452\\
1007.097293	4120907.48834034\\
1008.097293	4124060.29778989\\
1009.097293	4127211.24421316\\
1010.097293	4130360.32761016\\
1011.097293	4133507.54798088\\
1012.097293	4136652.90532533\\
1013.097293	4139796.39964351\\
1014.097293	4142938.03093541\\
1015.097293	4146077.79920104\\
1016.097293	4149215.70444039\\
1017.097293	4152351.74665347\\
1018.097293	4155485.92584028\\
1019.097293	4158618.24200082\\
1020.097293	4161748.69513508\\
1021.097293	4164877.28524306\\
1022.097293	4168004.01232477\\
1023.097293	4171128.87638021\\
1024.097293	4174251.87740938\\
1025.097293	4177373.01541227\\
1026.097293	4180492.29038889\\
1027.097293	4183609.70233923\\
1028.097293	4186725.2512633\\
1029.097293	4189838.93716109\\
1030.097293	4192950.76003262\\
1031.097293	4196060.71987787\\
1032.097293	4199168.81669684\\
1033.097293	4202275.05048954\\
1034.097293	4205379.42125597\\
1035.097293	4208481.92899612\\
1036.097293	4211582.57371\\
1037.097293	4214681.35539761\\
1038.097293	4217778.27405894\\
1039.097293	4220873.329694\\
1040.097293	4223966.52230278\\
1041.097293	4227057.85188529\\
1042.097293	4230147.31844153\\
1043.097293	4233234.92197149\\
1044.097293	4236320.66247518\\
1045.097293	4239404.53995259\\
1046.097293	4242486.55440374\\
1047.097293	4245566.7058286\\
1048.097293	4248644.9942272\\
1049.097293	4251721.41959952\\
1050.097293	4254795.98194556\\
1051.097293	4257868.68126534\\
1052.097293	4260939.51755884\\
1053.097293	4264008.49082606\\
1054.097293	4267075.60106701\\
1055.097293	4270140.84828169\\
1056.097293	4273204.23247009\\
1057.097293	4276265.75363222\\
1058.097293	4279325.41176808\\
1059.097293	4282383.20687766\\
1060.097293	4285439.13896097\\
1061.097293	4288493.20801801\\
1062.097293	4291545.41404877\\
1063.097293	4294595.75705325\\
1064.097293	4297644.23703147\\
1065.097293	4300690.85398341\\
1066.097293	4303735.60790907\\
1067.097293	4306778.49880847\\
1068.097293	4309819.52668158\\
1069.097293	4312858.69152843\\
1070.097293	4315895.993349\\
1071.097293	4318931.4321433\\
1072.097293	4321965.00791132\\
1073.097293	4324996.72065307\\
1074.097293	4328026.57036855\\
1075.097293	4331054.55705775\\
1076.097293	4334080.68072067\\
1077.097293	4337104.94135733\\
1078.097293	4340127.33896771\\
1079.097293	4343147.87355182\\
1080.097293	4346166.54510965\\
1081.097293	4349183.35364121\\
1082.097293	4352198.29914649\\
1083.097293	4355211.38162551\\
1084.097293	4358222.60107824\\
1085.097293	4361231.95750471\\
1086.097293	4364239.4509049\\
1087.097293	4367245.08127881\\
1088.097293	4370248.84862646\\
1089.097293	4373250.75294783\\
1090.097293	4376250.79424292\\
1091.097293	4379248.97251174\\
1092.097293	4382245.28775429\\
1093.097293	4385239.73997056\\
1094.097293	4388232.32916056\\
1095.097293	4391223.05532429\\
1096.097293	4394211.91846174\\
1097.097293	4397198.91857292\\
1098.097293	4400184.05565783\\
1099.097293	4403167.32971646\\
1100.097293	4406148.74074882\\
1101.097293	4409128.2887549\\
1102.097293	4412105.97373471\\
1103.097293	4415081.79568824\\
1104.097293	4418055.75461551\\
1105.097293	4421027.85051649\\
1106.097293	4423998.08339121\\
1107.097293	4426966.45323965\\
1108.097293	4429932.96006182\\
1109.097293	4432897.60385771\\
1110.097293	4435860.38462733\\
1111.097293	4438821.30237068\\
1112.097293	4441780.35708775\\
1113.097293	4444737.54877855\\
1114.097293	4447692.87744307\\
1115.097293	4450646.34308132\\
1116.097293	4453597.9456933\\
1117.097293	4456547.685279\\
1118.097293	4459495.56183843\\
1119.097293	4462441.57537159\\
1120.097293	4465385.72587847\\
1121.097293	4468328.01335908\\
1122.097293	4471268.43781341\\
1123.097293	4474206.99924147\\
1124.097293	4477143.69764326\\
1125.097293	4480078.53301877\\
1126.097293	4483011.50536801\\
1127.097293	4485942.61469097\\
1128.097293	4488871.86098767\\
1129.097293	4491799.24425808\\
1130.097293	4494724.76450223\\
1131.097293	4497648.4217201\\
1132.097293	4500570.21591169\\
1133.097293	4503490.14707702\\
1134.097293	4506408.21521607\\
1135.097293	4509324.42032884\\
1136.097293	4512238.76241534\\
1137.097293	4515151.24147557\\
1138.097293	4518061.85750952\\
1139.097293	4520970.6105172\\
1140.097293	4523877.50049861\\
1141.097293	4526782.52745374\\
1142.097293	4529685.6913826\\
1143.097293	4532586.99228518\\
1144.097293	4535486.43016149\\
1145.097293	4538384.00501153\\
1146.097293	4541279.7168353\\
1147.097293	4544173.56563278\\
1148.097293	4547065.551404\\
1149.097293	4549955.67414894\\
1150.097293	4552843.93386761\\
1151.097293	4555730.33056001\\
1152.097293	4558614.86422613\\
1153.097293	4561497.53486597\\
1154.097293	4564378.34247955\\
1155.097293	4567257.28706685\\
1156.097293	4570134.36862787\\
1157.097293	4573009.58716262\\
1158.097293	4575882.9426711\\
1159.097293	4578754.43515331\\
1160.097293	4581624.06460924\\
1161.097293	4584491.83103889\\
1162.097293	4587357.73444228\\
1163.097293	4590221.77481938\\
1164.097293	4593083.95217022\\
1165.097293	4595944.26649478\\
1166.097293	4598802.71779307\\
1167.097293	4601659.30606508\\
1168.097293	4604514.03131082\\
1169.097293	4607366.89353029\\
1170.097293	4610217.89272348\\
1171.097293	4613067.0288904\\
1172.097293	4615914.30203105\\
1173.097293	4618759.71214542\\
1174.097293	4621603.25923351\\
1175.097293	4624444.94329534\\
1176.097293	4627284.76433089\\
1177.097293	4630122.72234016\\
1178.097293	4632958.81732317\\
1179.097293	4635793.04927989\\
1180.097293	4638625.41821035\\
1181.097293	4641455.92411453\\
1182.097293	4644284.56699244\\
1183.097293	4647111.34684407\\
1184.097293	4649936.26366943\\
1185.097293	4652759.31746852\\
1186.097293	4655580.50824133\\
1187.097293	4658399.83598787\\
1188.097293	4661217.30070813\\
1189.097293	4664032.90240212\\
1190.097293	4666846.64106984\\
1191.097293	4669658.51671128\\
1192.097293	4672468.52932645\\
1193.097293	4675276.67891535\\
1194.097293	4678082.96547797\\
1195.097293	4680887.38901432\\
1196.097293	4683689.94952439\\
1197.097293	4686490.64700819\\
1198.097293	4689289.48146572\\
1199.097293	4692086.45289697\\
1200.097293	4694881.56130195\\
1201.097293	4697674.80668066\\
1202.097293	4700466.18903309\\
1203.097293	4703255.70835925\\
1204.097293	4706043.36465913\\
1205.097293	4708829.15793274\\
1206.097293	4711613.08818008\\
1207.097293	4714395.15540114\\
1208.097293	4717175.35959593\\
1209.097293	4719953.70076445\\
1210.097293	4722730.17890669\\
1211.097293	4725504.79402265\\
1212.097293	4728277.54611235\\
1213.097293	4731048.43517577\\
1214.097293	4733817.46121291\\
1215.097293	4736584.62422379\\
1216.097293	4739349.92420838\\
1217.097293	4742113.36116671\\
1218.097293	4744874.93509876\\
1219.097293	4747634.64600454\\
1220.097293	4750392.49388404\\
1221.097293	4753148.47873727\\
1222.097293	4755902.60056423\\
1223.097293	4758654.85936491\\
1224.097293	4761405.25513932\\
1225.097293	4764153.78788745\\
1226.097293	4766900.45760931\\
1227.097293	4769645.2643049\\
1228.097293	4772388.20797421\\
1229.097293	4775129.28861725\\
1230.097293	4777868.50623402\\
1231.097293	4780605.86082451\\
1232.097293	4783341.35238873\\
1233.097293	4786074.98092667\\
1234.097293	4788806.74643834\\
1235.097293	4791536.64892374\\
1236.097293	4794264.68838286\\
1237.097293	4796990.86481571\\
1238.097293	4799715.17822229\\
1239.097293	4802437.62860259\\
1240.097293	4805158.21595662\\
1241.097293	4807876.94028437\\
1242.097293	4810593.80158585\\
1243.097293	4813308.79986106\\
1244.097293	4816021.93510999\\
1245.097293	4818733.20733265\\
1246.097293	4821442.61652904\\
1247.097293	4824150.16269915\\
1248.097293	4826855.84584299\\
1249.097293	4829559.66596055\\
1250.097293	4832261.62305184\\
1251.097293	4834961.71711686\\
1252.097293	4837659.9481556\\
1253.097293	4840356.31616807\\
1254.097293	4843050.82115426\\
1255.097293	4845743.46311418\\
1256.097293	4848434.24204783\\
1257.097293	4851123.1579552\\
1258.097293	4853810.2108363\\
1259.097293	4856495.40069113\\
1260.097293	4859178.72751968\\
1261.097293	4861860.19132196\\
1262.097293	4864539.79209797\\
1263.097293	4867217.5298477\\
1264.097293	4869893.40457115\\
1265.097293	4872567.41626834\\
1266.097293	4875239.56493925\\
1267.097293	4877909.85058388\\
1268.097293	4880578.27320224\\
1269.097293	4883244.83279433\\
1270.097293	4885909.52936015\\
1271.097293	4888572.36289969\\
1272.097293	4891233.33341295\\
1273.097293	4893892.44089995\\
1274.097293	4896549.68536067\\
1275.097293	4899205.06679511\\
1276.097293	4901858.58520328\\
1277.097293	4904510.24058518\\
1278.097293	4907160.03294081\\
1279.097293	4909807.96227016\\
1280.097293	4912454.02857323\\
1281.097293	4915098.23185004\\
1282.097293	4917740.57210056\\
1283.097293	4920381.04932482\\
1284.097293	4923019.6635228\\
1285.097293	4925656.41469451\\
1286.097293	4928291.30283994\\
1287.097293	4930924.3279591\\
1288.097293	4933555.49005199\\
1289.097293	4936184.7891186\\
1290.097293	4938812.22515894\\
1291.097293	4941437.79817301\\
1292.097293	4944061.5081608\\
1293.097293	4946683.35512231\\
1294.097293	4949303.33905756\\
1295.097293	4951921.45996653\\
1296.097293	4954537.71784922\\
1297.097293	4957152.11270565\\
1298.097293	4959764.6445358\\
1299.097293	4962375.31333967\\
1300.097293	4964984.11911727\\
1301.097293	4967591.0618686\\
1302.097293	4970196.14159365\\
1303.097293	4972799.35829243\\
1304.097293	4975400.71196494\\
1305.097293	4978000.20261117\\
1306.097293	4980597.83023113\\
1307.097293	4983193.59482481\\
1308.097293	4985787.49639222\\
1309.097293	4988379.53493336\\
1310.097293	4990969.71044822\\
1311.097293	4993558.02293681\\
1312.097293	4996144.47239913\\
1313.097293	4998729.05883517\\
1314.097293	5001311.78224494\\
1315.097293	5003892.64262843\\
1316.097293	5006471.63998565\\
1317.097293	5009048.7743166\\
1318.097293	5011624.04562127\\
1319.097293	5014197.45389967\\
1320.097293	5016768.9991518\\
1321.097293	5019338.68137765\\
1322.097293	5021906.50057723\\
1323.097293	5024472.45675053\\
1324.097293	5027036.54989756\\
1325.097293	5029598.78001832\\
1326.097293	5032159.1471128\\
1327.097293	5034717.65118101\\
1328.097293	5037274.29222294\\
1329.097293	5039829.0702386\\
1330.097293	5042381.98522799\\
1331.097293	5044933.03719111\\
1332.097293	5047482.22612795\\
1333.097293	5050029.55203851\\
1334.097293	5052575.01492281\\
1335.097293	5055118.61478082\\
1336.097293	5057660.35161257\\
1337.097293	5060200.22541804\\
1338.097293	5062738.23619724\\
1339.097293	5065274.38395016\\
1340.097293	5067808.66867681\\
1341.097293	5070341.09037719\\
1342.097293	5072871.64905129\\
1343.097293	5075400.34469912\\
1344.097293	5077927.17732067\\
1345.097293	5080452.14691595\\
1346.097293	5082975.25348496\\
1347.097293	5085496.49702769\\
1348.097293	5088015.87754415\\
1349.097293	5090533.39503434\\
1350.097293	5093049.04949825\\
1351.097293	5095562.84093589\\
1352.097293	5098074.76934725\\
1353.097293	5100584.83473234\\
1354.097293	5103093.03709116\\
1355.097293	5105599.3764237\\
1356.097293	5108103.85272997\\
1357.097293	5110606.46600997\\
1358.097293	5113107.21626369\\
1359.097293	5115606.10349113\\
1360.097293	5118103.12769231\\
1361.097293	5120598.28886721\\
1362.097293	5123091.58701584\\
1363.097293	5125583.02213819\\
1364.097293	5128072.59423427\\
1365.097293	5130560.30330407\\
1366.097293	5133046.1493476\\
1367.097293	5135530.13236486\\
1368.097293	5138012.25235585\\
1369.097293	5140492.50932056\\
1370.097293	5142970.90325899\\
1371.097293	5145447.43417116\\
1372.097293	5147922.10205704\\
1373.097293	5150394.90691666\\
1374.097293	5152865.84875\\
1375.097293	5155334.92755707\\
1376.097293	5157802.14333786\\
1377.097293	5160267.49609238\\
1378.097293	5162730.98582063\\
1379.097293	5165192.6125226\\
1380.097293	5167652.3761983\\
1381.097293	5170110.27684772\\
1382.097293	5172566.31447087\\
1383.097293	5175020.48906775\\
1384.097293	5177472.80063835\\
1385.097293	5179923.24918268\\
1386.097293	5182371.83470074\\
1387.097293	5184818.55719252\\
1388.097293	5187263.41665803\\
1389.097293	5189706.41309726\\
1390.097293	5192147.54651022\\
1391.097293	5194586.81689691\\
1392.097293	5197024.22425732\\
1393.097293	5199459.76859146\\
1394.097293	5201893.44989933\\
1395.097293	5204325.26818092\\
1396.097293	5206755.22343624\\
1397.097293	5209183.31566528\\
1398.097293	5211609.54486805\\
1399.097293	5214033.91104455\\
1400.097293	5216456.41419477\\
1401.097293	5218877.05431872\\
1402.097293	5221295.83141639\\
1403.097293	5223712.7454878\\
1404.097293	5226127.79653292\\
1405.097293	5228540.98455178\\
1406.097293	5230952.30954436\\
1407.097293	5233361.77151067\\
1408.097293	5235769.3704507\\
1409.097293	5238175.10636446\\
1410.097293	5240578.97925194\\
1411.097293	5242980.98911315\\
1412.097293	5245381.13594809\\
1413.097293	5247779.41975675\\
1414.097293	5250175.84053914\\
1415.097293	5252570.39829526\\
1416.097293	5254963.0930251\\
1417.097293	5257353.92472867\\
1418.097293	5259742.89340596\\
1419.097293	5262129.99905699\\
1420.097293	5264515.24168173\\
1421.097293	5266898.62128021\\
1422.097293	5269280.13785241\\
1423.097293	5271659.79139833\\
1424.097293	5274037.58191798\\
1425.097293	5276413.50941136\\
1426.097293	5278787.57387847\\
1427.097293	5281159.7753193\\
1428.097293	5283530.11373385\\
1429.097293	5285898.58912214\\
1430.097293	5288265.20148415\\
1431.097293	5290629.95081988\\
1432.097293	5292992.83712934\\
1433.097293	5295353.86041253\\
1434.097293	5297713.02066945\\
1435.097293	5300070.31790009\\
1436.097293	5302425.75210445\\
1437.097293	5304779.32328255\\
1438.097293	5307131.03143436\\
1439.097293	5309480.87655991\\
1440.097293	5311828.85865918\\
1441.097293	5314174.97773218\\
1442.097293	5316519.2337789\\
1443.097293	5318861.62679935\\
1444.097293	5321202.15679353\\
1445.097293	5323540.82376143\\
1446.097293	5325877.62770306\\
1447.097293	5328212.56861842\\
1448.097293	5330545.6465075\\
1449.097293	5332876.86137031\\
1450.097293	5335206.21320684\\
1451.097293	5337533.7020171\\
1452.097293	5339859.32780109\\
1453.097293	5342183.0905588\\
1454.097293	5344504.99029024\\
1455.097293	5346825.0269954\\
1456.097293	5349143.20067429\\
1457.097293	5351459.51132691\\
1458.097293	5353773.95895325\\
1459.097293	5356086.54355332\\
1460.097293	5358397.26512712\\
1461.097293	5360706.12367464\\
1462.097293	5363013.11919589\\
1463.097293	5365318.25169086\\
1464.097293	5367621.52115956\\
1465.097293	5369922.92760199\\
1466.097293	5372222.47101814\\
1467.097293	5374520.15140802\\
1468.097293	5376815.96877163\\
1469.097293	5379109.92310896\\
1470.097293	5381402.01442002\\
1471.097293	5383692.2427048\\
1472.097293	5385980.60796331\\
1473.097293	5388267.11019555\\
1474.097293	5390551.74940151\\
1475.097293	5392834.5255812\\
1476.097293	5395115.43873462\\
1477.097293	5397394.48886176\\
1478.097293	5399671.67596263\\
1479.097293	5401947.00003722\\
1480.097293	5404220.46108554\\
1481.097293	5406492.05910759\\
1482.097293	5408761.79410336\\
1483.097293	5411029.66607286\\
1484.097293	5413295.67501608\\
1485.097293	5415559.82093304\\
1486.097293	5417822.10382371\\
1487.097293	5420082.52368812\\
1488.097293	5422341.08052625\\
1489.097293	5424597.7743381\\
1490.097293	5426852.60512369\\
1491.097293	5429105.57288299\\
1492.097293	5431356.67761603\\
1493.097293	5433605.91932279\\
1494.097293	5435853.29800328\\
1495.097293	5438098.81365749\\
1496.097293	5440342.46628543\\
1497.097293	5442584.2558871\\
1498.097293	5444824.18246249\\
1499.097293	5447062.24601161\\
1500.097293	5449298.44653445\\
1501.097293	5451532.78403102\\
1502.097293	5453765.25850132\\
1503.097293	5455995.86994534\\
1504.097293	5458224.61836309\\
1505.097293	5460451.50375457\\
1506.097293	5462676.52611977\\
1507.097293	5464899.6854587\\
1508.097293	5467120.98177135\\
1509.097293	5469340.41505773\\
1510.097293	5471557.98531784\\
1511.097293	5473773.69255167\\
1512.097293	5475987.53675923\\
1513.097293	5478199.51794052\\
1514.097293	5480409.63609553\\
1515.097293	5482617.89122427\\
1516.097293	5484824.28332673\\
1517.097293	5487028.81240292\\
1518.097293	5489231.47845284\\
1519.097293	5491432.28147648\\
1520.097293	5493631.22147385\\
1521.097293	5495828.29844495\\
1522.097293	5498023.51238977\\
1523.097293	5500216.86330831\\
1524.097293	5502408.35120059\\
1525.097293	5504597.97606659\\
1526.097293	5506785.73790631\\
1527.097293	5508971.63671977\\
1528.097293	5511155.67250695\\
1529.097293	5513337.84526785\\
1530.097293	5515518.15500248\\
1531.097293	5517696.60171084\\
1532.097293	5519873.18539292\\
1533.097293	5522047.90604873\\
1534.097293	5524220.76367827\\
1535.097293	5526391.75828153\\
1536.097293	5528560.88985852\\
1537.097293	5530728.15840923\\
1538.097293	5532893.56393368\\
1539.097293	5535057.10643184\\
1540.097293	5537218.78590374\\
1541.097293	5539378.60234936\\
1542.097293	5541536.5557687\\
1543.097293	5543692.64616177\\
1544.097293	5545846.87352857\\
1545.097293	5547999.2378691\\
1546.097293	5550149.73918335\\
1547.097293	5552298.37747132\\
1548.097293	5554445.15273303\\
1549.097293	5556590.06496846\\
1550.097293	5558733.11417761\\
1551.097293	5560874.30036049\\
1552.097293	5563013.6235171\\
1553.097293	5565151.08364743\\
1554.097293	5567286.6807515\\
1555.097293	5569420.41482928\\
1556.097293	5571552.28588079\\
1557.097293	5573682.29390603\\
1558.097293	5575810.438905\\
1559.097293	5577936.72087769\\
1560.097293	5580061.13982411\\
1561.097293	5582183.69574425\\
1562.097293	5584304.38863812\\
1563.097293	5586423.21850572\\
1564.097293	5588540.18534704\\
1565.097293	5590655.28916209\\
1566.097293	5592768.52995086\\
1567.097293	5594879.90771337\\
1568.097293	5596989.42244959\\
1569.097293	5599097.07415955\\
1570.097293	5601202.86284323\\
1571.097293	5603306.78850063\\
1572.097293	5605408.85113177\\
1573.097293	5607509.05073662\\
1574.097293	5609607.38731521\\
1575.097293	5611703.86086752\\
1576.097293	5613798.47139356\\
1577.097293	5615891.21889332\\
1578.097293	5617982.10336681\\
1579.097293	5620071.12481403\\
1580.097293	5622158.28323497\\
1581.097293	5624243.57862964\\
1582.097293	5626327.01099803\\
1583.097293	5628408.58034015\\
1584.097293	5630488.286656\\
1585.097293	5632566.12994557\\
1586.097293	5634642.11020887\\
1587.097293	5636716.2274459\\
1588.097293	5638788.48165665\\
1589.097293	5640858.87284113\\
1590.097293	5642927.40099933\\
1591.097293	5644994.06613126\\
1592.097293	5647058.86823692\\
1593.097293	5649121.8073163\\
1594.097293	5651182.88336941\\
1595.097293	5653242.09639625\\
1596.097293	5655299.44639681\\
1597.097293	5657354.93337109\\
1598.097293	5659408.55731911\\
1599.097293	5661460.31824085\\
1600.097293	5663510.21613632\\
1601.097293	5665558.25100551\\
1602.097293	5667604.42284843\\
1603.097293	5669648.73166507\\
1604.097293	5671691.17745544\\
1605.097293	5673731.76021954\\
1606.097293	5675770.47995736\\
1607.097293	5677807.33666891\\
1608.097293	5679842.33035419\\
1609.097293	5681875.46101319\\
1610.097293	5683906.72864592\\
1611.097293	5685936.13325238\\
1612.097293	5687963.67483256\\
1613.097293	5689989.35338646\\
1614.097293	5692013.1689141\\
1615.097293	5694035.12141546\\
1616.097293	5696055.21089054\\
1617.097293	5698073.43733936\\
1618.097293	5700089.8007619\\
1619.097293	5702104.30115816\\
1620.097293	5704116.93852815\\
1621.097293	5706127.71287187\\
1622.097293	5708136.62418931\\
1623.097293	5710143.67248048\\
1624.097293	5712148.85774538\\
1625.097293	5714152.179984\\
1626.097293	5716153.63919635\\
1627.097293	5718153.23538242\\
1628.097293	5720150.96854222\\
1629.097293	5722146.83867575\\
1630.097293	5724140.845783\\
1631.097293	5726132.98986398\\
1632.097293	5728123.27091868\\
1633.097293	5730111.68894712\\
1634.097293	5732098.24394927\\
1635.097293	5734082.93592516\\
1636.097293	5736065.76487477\\
1637.097293	5738046.7307981\\
1638.097293	5740025.83369517\\
1639.097293	5742003.07356596\\
1640.097293	5743978.45041047\\
1641.097293	5745951.96422871\\
1642.097293	5747923.61502068\\
1643.097293	5749893.40278637\\
1644.097293	5751861.32752579\\
1645.097293	5753827.38923894\\
1646.097293	5755791.58792581\\
1647.097293	5757753.92358641\\
1648.097293	5759714.39622074\\
1649.097293	5761673.00582879\\
1650.097293	5763629.75241056\\
1651.097293	5765584.63596607\\
1652.097293	5767537.6564953\\
1653.097293	5769488.81399825\\
1654.097293	5771438.10847494\\
1655.097293	5773385.53992534\\
1656.097293	5775331.10834948\\
1657.097293	5777274.81374734\\
1658.097293	5779216.65611893\\
1659.097293	5781156.63546424\\
1660.097293	5783094.75178328\\
1661.097293	5785031.00507605\\
1662.097293	5786965.39534254\\
1663.097293	5788897.92258276\\
1664.097293	5790828.5867967\\
1665.097293	5792757.38798437\\
1666.097293	5794684.32614577\\
1667.097293	5796609.40128089\\
1668.097293	5798532.61338974\\
1669.097293	5800453.96247232\\
1670.097293	5802373.44852862\\
1671.097293	5804291.07155864\\
1672.097293	5806206.8315624\\
1673.097293	5808120.72853988\\
1674.097293	5810032.76249109\\
1675.097293	5811942.93341602\\
1676.097293	5813851.24131468\\
1677.097293	5815757.68618706\\
1678.097293	5817662.26803317\\
1679.097293	5819564.98685301\\
1680.097293	5821465.84264658\\
1681.097293	5823364.83541387\\
1682.097293	5825261.96515488\\
1683.097293	5827157.23186962\\
1684.097293	5829050.63555809\\
1685.097293	5830942.17622029\\
1686.097293	5832831.85385621\\
1687.097293	5834719.66846586\\
1688.097293	5836605.62004923\\
1689.097293	5838489.70860633\\
1690.097293	5840371.93413716\\
1691.097293	5842252.29664171\\
1692.097293	5844130.79611999\\
1693.097293	5846007.43257199\\
1694.097293	5847882.20599772\\
1695.097293	5849755.11639718\\
1696.097293	5851626.16377036\\
1697.097293	5853495.34811727\\
1698.097293	5855362.66943791\\
1699.097293	5857228.12773227\\
1700.097293	5859091.72300036\\
1701.097293	5860953.45524217\\
1702.097293	5862813.32445772\\
1703.097293	5864671.33064698\\
1704.097293	5866527.47380998\\
1705.097293	5868381.75394669\\
1706.097293	5870234.17105714\\
1707.097293	5872084.72514131\\
1708.097293	5873933.41619921\\
1709.097293	5875780.24423083\\
1710.097293	5877625.20923618\\
1711.097293	5879468.31121526\\
1712.097293	5881309.55016806\\
1713.097293	5883148.92609459\\
1714.097293	5884986.43899485\\
1715.097293	5886822.08886883\\
1716.097293	5888655.87571654\\
1717.097293	5890487.79953797\\
1718.097293	5892317.86033313\\
1719.097293	5894146.05810202\\
1720.097293	5895972.39284463\\
1721.097293	5897796.86456097\\
1722.097293	5899619.47325103\\
1723.097293	5901440.21891483\\
1724.097293	5903259.10155234\\
1725.097293	5905076.12116359\\
1726.097293	5906891.27774856\\
1727.097293	5908704.57130725\\
1728.097293	5910516.00183968\\
1729.097293	5912325.56934582\\
1730.097293	5914133.2738257\\
1731.097293	5915939.1152793\\
1732.097293	5917743.09370663\\
1733.097293	5919545.20910768\\
1734.097293	5921345.46148246\\
1735.097293	5923143.85083096\\
1736.097293	5924940.3771532\\
1737.097293	5926735.04044916\\
1738.097293	5928527.84071884\\
1739.097293	5930318.77796225\\
1740.097293	5932107.85217939\\
1741.097293	5933895.06337025\\
1742.097293	5935680.41153484\\
1743.097293	5937463.89667316\\
1744.097293	5939245.5187852\\
1745.097293	5941025.27787097\\
1746.097293	5942803.17393046\\
1747.097293	5944579.20696368\\
1748.097293	5946353.37697063\\
1749.097293	5948125.6839513\\
1750.097293	5949896.1279057\\
1751.097293	5951664.70883382\\
1752.097293	5953431.42673568\\
1753.097293	5955196.28161125\\
1754.097293	5956959.27346056\\
1755.097293	5958720.40228359\\
1756.097293	5960479.66808034\\
1757.097293	5962237.07085083\\
1758.097293	5963992.61059504\\
1759.097293	5965746.28731297\\
1760.097293	5967498.10100463\\
1761.097293	5969248.05167002\\
1762.097293	5970996.13930913\\
1763.097293	5972742.36392198\\
1764.097293	5974486.72550854\\
1765.097293	5976229.22406883\\
1766.097293	5977969.85960285\\
1767.097293	5979708.6321106\\
1768.097293	5981445.54159207\\
1769.097293	5983180.58804727\\
1770.097293	5984913.77147619\\
1771.097293	5986645.09187884\\
1772.097293	5988374.54925521\\
1773.097293	5990102.14360532\\
1774.097293	5991827.87492914\\
1775.097293	5993551.7432267\\
1776.097293	5995273.74849798\\
1777.097293	5996993.89074299\\
1778.097293	5998712.16996172\\
1779.097293	6000428.58615418\\
1780.097293	6002143.13932037\\
1781.097293	6003855.82946028\\
1782.097293	6005566.65657392\\
1783.097293	6007275.62066128\\
1784.097293	6008982.72172237\\
1785.097293	6010687.95975719\\
1786.097293	6012391.33476573\\
1787.097293	6014092.846748\\
1788.097293	6015792.495704\\
1789.097293	6017490.28163372\\
1790.097293	6019186.20453716\\
1791.097293	6020880.26441434\\
1792.097293	6022572.46126524\\
1793.097293	6024262.79508987\\
1794.097293	6025951.26588822\\
1795.097293	6027637.8736603\\
1796.097293	6029322.6184061\\
1797.097293	6031005.50012563\\
1798.097293	6032686.51881889\\
1799.097293	6034365.67448588\\
1800.097293	6036042.96712659\\
1801.097293	6037718.39674102\\
1802.097293	6039391.96332918\\
1803.097293	6041063.66689107\\
1804.097293	6042733.50742669\\
1805.097293	6044401.48493603\\
1806.097293	6046067.5994191\\
1807.097293	6047731.85087589\\
1808.097293	6049394.23930641\\
1809.097293	6051054.76471065\\
1810.097293	6052713.42708863\\
1811.097293	6054370.22644033\\
1812.097293	6056025.16276575\\
1813.097293	6057678.2360649\\
1814.097293	6059329.44633778\\
1815.097293	6060978.79358438\\
1816.097293	6062626.27780471\\
1817.097293	6064271.89899877\\
1818.097293	6065915.65716655\\
1819.097293	6067557.55230806\\
1820.097293	6069197.58442329\\
1821.097293	6070835.75351225\\
1822.097293	6072472.05957494\\
1823.097293	6074106.50261135\\
1824.097293	6075739.08262149\\
1825.097293	6077369.79960536\\
1826.097293	6078998.65356295\\
1827.097293	6080625.64449427\\
1828.097293	6082250.77239931\\
1829.097293	6083874.03727808\\
1830.097293	6085495.43913058\\
1831.097293	6087114.9779568\\
1832.097293	6088732.65375675\\
1833.097293	6090348.46653042\\
1834.097293	6091962.41627783\\
1835.097293	6093574.50299895\\
1836.097293	6095184.72669381\\
1837.097293	6096793.08736239\\
1838.097293	6098399.58500469\\
1839.097293	6100004.21962073\\
1840.097293	6101606.99121049\\
1841.097293	6103207.89977397\\
1842.097293	6104806.94531118\\
1843.097293	6106404.12782212\\
1844.097293	6107999.44730678\\
1845.097293	6109592.90376517\\
1846.097293	6111184.49719729\\
1847.097293	6112774.22760313\\
1848.097293	6114362.0949827\\
1849.097293	6115948.09933599\\
1850.097293	6117532.24066302\\
1851.097293	6119114.51896376\\
1852.097293	6120694.93423824\\
1853.097293	6122273.48648644\\
1854.097293	6123850.17570836\\
1855.097293	6125425.00190401\\
1856.097293	6126997.96507339\\
1857.097293	6128569.0652165\\
1858.097293	6130138.30233333\\
1859.097293	6131705.67642388\\
1860.097293	6133271.18748817\\
1861.097293	6134834.83552618\\
1862.097293	6136396.62053791\\
1863.097293	6137956.54252337\\
1864.097293	6139514.60148256\\
1865.097293	6141070.79741548\\
1866.097293	6142625.13032212\\
1867.097293	6144177.60020248\\
1868.097293	6145728.20705658\\
1869.097293	6147276.9508844\\
1870.097293	6148823.83168594\\
1871.097293	6150368.84946121\\
1872.097293	6151912.00421021\\
1873.097293	6153453.29593293\\
1874.097293	6154992.72462939\\
1875.097293	6156530.29029956\\
1876.097293	6158065.99294346\\
1877.097293	6159599.83256109\\
1878.097293	6161131.80915245\\
1879.097293	6162661.92271753\\
1880.097293	6164190.17325634\\
1881.097293	6165716.56076887\\
1882.097293	6167241.08525513\\
1883.097293	6168763.74671512\\
1884.097293	6170284.54514883\\
1885.097293	6171803.48055627\\
1886.097293	6173320.55293743\\
1887.097293	6174835.76229232\\
1888.097293	6176349.10862094\\
1889.097293	6177860.59192328\\
1890.097293	6179370.21219935\\
1891.097293	6180877.96944915\\
1892.097293	6182383.86367267\\
1893.097293	6183887.89486992\\
1894.097293	6185390.06304089\\
1895.097293	6186890.3681856\\
1896.097293	6188388.81030402\\
1897.097293	6189885.38939618\\
1898.097293	6191380.10546205\\
1899.097293	6192872.95850166\\
1900.097293	6194363.94851499\\
1901.097293	6195853.07550205\\
1902.097293	6197340.33946284\\
1903.097293	6198825.74039735\\
1904.097293	6200309.27830558\\
1905.097293	6201790.95318754\\
1906.097293	6203270.76504323\\
1907.097293	6204748.71387265\\
1908.097293	6206224.79967579\\
1909.097293	6207699.02245266\\
1910.097293	6209171.38220325\\
1911.097293	6210641.87892757\\
1912.097293	6212110.51262562\\
1913.097293	6213577.28329739\\
1914.097293	6215042.19094289\\
1915.097293	6216505.23556212\\
1916.097293	6217966.41715507\\
1917.097293	6219425.73572175\\
1918.097293	6220883.19126215\\
1919.097293	6222338.78377628\\
1920.097293	6223792.51326413\\
1921.097293	6225244.37972572\\
1922.097293	6226694.38316103\\
1923.097293	6228142.52357006\\
1924.097293	6229588.80095282\\
1925.097293	6231033.21530931\\
1926.097293	6232475.76663952\\
1927.097293	6233916.45494346\\
1928.097293	6235355.28022113\\
1929.097293	6236792.24247252\\
1930.097293	6238227.34169764\\
1931.097293	6239660.57789648\\
1932.097293	6241091.95106906\\
1933.097293	6242521.46121535\\
1934.097293	6243949.10833538\\
1935.097293	6245374.89242912\\
1936.097293	6246798.8134966\\
1937.097293	6248220.8715378\\
1938.097293	6249641.06655273\\
1939.097293	6251059.39854139\\
1940.097293	6252475.86750377\\
1941.097293	6253890.47343987\\
1942.097293	6255303.21634971\\
1943.097293	6256714.09623327\\
1944.097293	6258123.11309055\\
1945.097293	6259530.26692156\\
1946.097293	6260935.5577263\\
1947.097293	6262338.98550476\\
1948.097293	6263740.55025696\\
1949.097293	6265140.25198287\\
1950.097293	6266538.09068251\\
1951.097293	6267934.06635588\\
1952.097293	6269328.17900298\\
1953.097293	6270720.4286238\\
1954.097293	6272110.81521835\\
1955.097293	6273499.33878662\\
1956.097293	6274885.99932862\\
1957.097293	6276270.79684435\\
1958.097293	6277653.7313338\\
1959.097293	6279034.80279698\\
1960.097293	6280414.01123388\\
1961.097293	6281791.35664452\\
1962.097293	6283166.83902887\\
1963.097293	6284540.45838696\\
1964.097293	6285912.21471877\\
1965.097293	6287282.1080243\\
1966.097293	6288650.13830356\\
1967.097293	6290016.30555655\\
1968.097293	6291380.60978327\\
1969.097293	6292743.05098371\\
1970.097293	6294103.62915787\\
1971.097293	6295462.34430577\\
1972.097293	6296819.19642739\\
1973.097293	6298174.18552273\\
1974.097293	6299527.31159181\\
1975.097293	6300878.5746346\\
1976.097293	6302227.97465113\\
1977.097293	6303575.51164138\\
1978.097293	6304921.18560536\\
1979.097293	6306264.99654306\\
1980.097293	6307606.94445449\\
1981.097293	6308947.02933964\\
1982.097293	6310285.25119853\\
1983.097293	6311621.61003113\\
1984.097293	6312956.10583747\\
1985.097293	6314288.73861753\\
1986.097293	6315619.50837131\\
1987.097293	6316948.41509883\\
1988.097293	6318275.45880007\\
1989.097293	6319600.63947503\\
1990.097293	6320923.95712372\\
1991.097293	6322245.41174614\\
1992.097293	6323565.00334228\\
1993.097293	6324882.73191216\\
1994.097293	6326198.59745575\\
1995.097293	6327512.59997308\\
1996.097293	6328824.73946412\\
1997.097293	6330135.0159289\\
1998.097293	6331443.4293674\\
1999.097293	6332749.97977963\\
2000.097293	6334054.66716558\\
2001.097293	6335357.49152526\\
2002.097293	6336658.45285867\\
2003.097293	6337957.5511658\\
2004.097293	6339254.78644666\\
2005.097293	6340550.15870124\\
2006.097293	6341843.66792955\\
2007.097293	6343135.31413159\\
2008.097293	6344425.09730735\\
2009.097293	6345713.01745684\\
2010.097293	6346999.07458006\\
2011.097293	6348283.268677\\
2012.097293	6349565.59974767\\
2013.097293	6350846.06779207\\
2014.097293	6352124.67281018\\
2015.097293	6353401.41480203\\
2016.097293	6354676.2937676\\
2017.097293	6355949.30970691\\
2018.097293	6357220.46261993\\
2019.097293	6358489.75250668\\
2020.097293	6359757.17936716\\
2021.097293	6361022.74320136\\
2022.097293	6362286.44400929\\
2023.097293	6363548.28179095\\
2024.097293	6364808.25654634\\
2025.097293	6366066.36827544\\
2026.097293	6367322.61697828\\
2027.097293	6368577.00265484\\
2028.097293	6369829.52530513\\
2029.097293	6371080.18492914\\
2030.097293	6372328.98152688\\
2031.097293	6373575.91509835\\
2032.097293	6374820.98564354\\
2033.097293	6376064.19316246\\
2034.097293	6377305.53765511\\
2035.097293	6378545.01912148\\
2036.097293	6379782.63756157\\
2037.097293	6381018.3929754\\
2038.097293	6382252.28536295\\
2039.097293	6383484.31472423\\
2040.097293	6384714.48105923\\
2041.097293	6385942.78436796\\
2042.097293	6387169.22465041\\
2043.097293	6388393.80190659\\
2044.097293	6389616.5161365\\
2045.097293	6390837.36734013\\
2046.097293	6392056.35551749\\
2047.097293	6393273.48066858\\
2048.097293	6394488.74279339\\
2049.097293	6395702.14189193\\
2050.097293	6396913.67796419\\
2051.097293	6398123.35101018\\
2052.097293	6399331.1610299\\
2053.097293	6400537.10802334\\
2054.097293	6401741.19199051\\
2055.097293	6402943.41293141\\
2056.097293	6404143.77084603\\
2057.097293	6405342.26573438\\
2058.097293	6406538.89759645\\
2059.097293	6407733.66643225\\
2060.097293	6408926.57224178\\
2061.097293	6410117.61502503\\
2062.097293	6411306.79478201\\
2063.097293	6412494.11151272\\
2064.097293	6413679.56521715\\
2065.097293	6414863.15589531\\
2066.097293	6416044.88354719\\
2067.097293	6417224.7481728\\
2068.097293	6418402.74977214\\
2069.097293	6419578.8883452\\
2070.097293	6420753.16389199\\
2071.097293	6421925.57641251\\
2072.097293	6423096.12590675\\
2073.097293	6424264.81237471\\
2074.097293	6425431.63581641\\
2075.097293	6426596.59623183\\
2076.097293	6427759.69362098\\
2077.097293	6428920.92798385\\
2078.097293	6430080.29932045\\
2079.097293	6431237.80763077\\
2080.097293	6432393.45291482\\
2081.097293	6433547.2351726\\
2082.097293	6434699.1544041\\
2083.097293	6435849.21060933\\
2084.097293	6436997.40378829\\
2085.097293	6438143.73394097\\
2086.097293	6439288.20106738\\
2087.097293	6440430.80516751\\
2088.097293	6441571.54624137\\
2089.097293	6442710.42428896\\
2090.097293	6443847.43931028\\
2091.097293	6444982.59130532\\
2092.097293	6446115.88027408\\
2093.097293	6447247.30621657\\
2094.097293	6448376.86913279\\
2095.097293	6449504.56902274\\
2096.097293	6450630.40588641\\
2097.097293	6451754.3797238\\
2098.097293	6452876.49053493\\
2099.097293	6453996.73831978\\
2100.097293	6455115.12307835\\
2101.097293	6456231.64481065\\
2102.097293	6457346.30351668\\
2103.097293	6458459.09919643\\
2104.097293	6459570.03184992\\
2105.097293	6460679.10147712\\
2106.097293	6461786.30807805\\
2107.097293	6462891.65165271\\
2108.097293	6463995.1322011\\
2109.097293	6465096.74972321\\
2110.097293	6466196.50421905\\
2111.097293	6467294.39568861\\
2112.097293	6468390.4241319\\
2113.097293	6469484.58954892\\
2114.097293	6470576.89193966\\
2115.097293	6471667.33130413\\
2116.097293	6472755.90764232\\
2117.097293	6473842.62095425\\
2118.097293	6474927.47123989\\
2119.097293	6476010.45849927\\
2120.097293	6477091.58273237\\
2121.097293	6478170.84393919\\
2122.097293	6479248.24211974\\
2123.097293	6480323.77727402\\
2124.097293	6481397.44940203\\
2125.097293	6482469.25850376\\
2126.097293	6483539.20457922\\
2127.097293	6484607.2876284\\
2128.097293	6485673.50765131\\
2129.097293	6486737.86464794\\
2130.097293	6487800.35861831\\
2131.097293	6488860.98956239\\
2132.097293	6489919.75748021\\
2133.097293	6490976.66237175\\
2134.097293	6492031.70423702\\
2135.097293	6493084.88307601\\
2136.097293	6494136.19888873\\
2137.097293	6495185.65167518\\
2138.097293	6496233.24143535\\
2139.097293	6497278.96816925\\
2140.097293	6498322.83187687\\
2141.097293	6499364.83255822\\
2142.097293	6500404.9702133\\
2143.097293	6501443.2448421\\
2144.097293	6502479.65644463\\
2145.097293	6503514.20502088\\
2146.097293	6504546.89057086\\
2147.097293	6505577.71309457\\
2148.097293	6506606.67259201\\
2149.097293	6507633.76906317\\
2150.097293	6508659.00250805\\
2151.097293	6509682.37292667\\
2152.097293	6510703.88031901\\
2153.097293	6511723.52468507\\
2154.097293	6512741.30602486\\
2155.097293	6513757.22433838\\
2156.097293	6514771.27962562\\
2157.097293	6515783.47188659\\
2158.097293	6516793.80112129\\
2159.097293	6517802.26732971\\
2160.097293	6518808.87051186\\
2161.097293	6519813.61066774\\
2162.097293	6520816.48779734\\
2163.097293	6521817.50190066\\
2164.097293	6522816.65297772\\
2165.097293	6523813.9410285\\
2166.097293	6524809.366053\\
2167.097293	6525802.92805123\\
2168.097293	6526794.62702319\\
2169.097293	6527784.46296888\\
2170.097293	6528772.43588829\\
2171.097293	6529758.54578143\\
2172.097293	6530742.79264829\\
2173.097293	6531725.17648888\\
2174.097293	6532705.69730319\\
2175.097293	6533684.35509123\\
2176.097293	6534661.149853\\
2177.097293	6535636.0815885\\
2178.097293	6536609.15029772\\
2179.097293	6537580.35598066\\
2180.097293	6538549.69863734\\
2181.097293	6539517.17826774\\
2182.097293	6540482.79487186\\
2183.097293	6541446.54844971\\
2184.097293	6542408.43900129\\
2185.097293	6543368.46652659\\
2186.097293	6544326.63102563\\
2187.097293	6545282.93249838\\
2188.097293	6546237.37094486\\
2189.097293	6547189.94636507\\
2190.097293	6548140.65875901\\
2191.097293	6549089.50812667\\
2192.097293	6550036.49446806\\
2193.097293	6550981.61778317\\
2194.097293	6551924.87807201\\
2195.097293	6552866.27533458\\
2196.097293	6553805.80957087\\
2197.097293	6554743.48078089\\
2198.097293	6555679.28896463\\
2199.097293	6556613.2341221\\
2200.097293	6557545.3162533\\
2201.097293	6558475.53535822\\
2202.097293	6559403.89143688\\
2203.097293	6560330.38448925\\
2204.097293	6561255.01451535\\
2205.097293	6562177.78151518\\
2206.097293	6563098.68548874\\
2207.097293	6564017.72643602\\
2208.097293	6564934.90435702\\
2209.097293	6565850.21925176\\
2210.097293	6566763.67112022\\
2211.097293	6567675.2599624\\
2212.097293	6568584.98577831\\
2213.097293	6569492.84856795\\
2214.097293	6570398.84833132\\
2215.097293	6571302.98506841\\
2216.097293	6572205.25877922\\
2217.097293	6573105.66946377\\
2218.097293	6574004.21712204\\
2219.097293	6574900.90175403\\
2220.097293	6575795.72335975\\
2221.097293	6576688.6819392\\
2222.097293	6577579.77749238\\
2223.097293	6578469.01001928\\
2224.097293	6579356.3795199\\
2225.097293	6580241.88599426\\
2226.097293	6581125.52944234\\
2227.097293	6582007.30986414\\
2228.097293	6582887.22725967\\
2229.097293	6583765.28162893\\
2230.097293	6584641.47297191\\
2231.097293	6585515.80128862\\
2232.097293	6586388.26657906\\
2233.097293	6587258.86884322\\
2234.097293	6588127.60808111\\
2235.097293	6588994.48429273\\
2236.097293	6589859.49747807\\
2237.097293	6590722.64763713\\
2238.097293	6591583.93476993\\
2239.097293	6592443.35887645\\
2240.097293	6593300.91995669\\
2241.097293	6594156.61801067\\
2242.097293	6595010.45303836\\
2243.097293	6595862.42503979\\
2244.097293	6596712.53401494\\
2245.097293	6597560.77996382\\
2246.097293	6598407.16288642\\
2247.097293	6599251.68278275\\
2248.097293	6600094.33965281\\
2249.097293	6600935.13349659\\
2250.097293	6601774.0643141\\
2251.097293	6602611.13210533\\
2252.097293	6603446.33687029\\
2253.097293	6604279.67860898\\
2254.097293	6605111.15732139\\
2255.097293	6605940.77300753\\
2256.097293	6606768.5256674\\
2257.097293	6607594.41530099\\
2258.097293	6608418.44190831\\
2259.097293	6609240.60548935\\
2260.097293	6610060.90604412\\
2261.097293	6610879.34357262\\
2262.097293	6611695.91807484\\
2263.097293	6612510.62955079\\
2264.097293	6613323.47800046\\
2265.097293	6614134.46342387\\
2266.097293	6614943.58582099\\
2267.097293	6615750.84519185\\
2268.097293	6616556.24153643\\
2269.097293	6617359.77485473\\
2270.097293	6618161.44514676\\
2271.097293	6618961.25241253\\
2272.097293	6619759.19665201\\
2273.097293	6620555.27786522\\
2274.097293	6621349.49605216\\
2275.097293	6622141.85121282\\
2276.097293	6622932.34334721\\
2277.097293	6623720.97245533\\
2278.097293	6624507.73853717\\
2279.097293	6625292.64159274\\
2280.097293	6626075.68162203\\
2281.097293	6626856.85862505\\
2282.097293	6627636.1726018\\
2283.097293	6628413.62355227\\
2284.097293	6629189.21147648\\
2285.097293	6629962.9363744\\
2286.097293	6630734.79824605\\
2287.097293	6631504.79709143\\
2288.097293	6632272.93291054\\
2289.097293	6633039.20570337\\
2290.097293	6633803.61546992\\
2291.097293	6634566.16221021\\
2292.097293	6635326.84592422\\
2293.097293	6636085.66661195\\
2294.097293	6636842.62427341\\
2295.097293	6637597.7189086\\
2296.097293	6638350.95051751\\
2297.097293	6639102.31910016\\
2298.097293	6639851.82465652\\
2299.097293	6640599.46718661\\
2300.097293	6641345.24669044\\
2301.097293	6642089.16316798\\
2302.097293	6642831.21661925\\
2303.097293	6643571.40704425\\
2304.097293	6644309.73444297\\
2305.097293	6645046.19881542\\
2306.097293	6645780.8001616\\
2307.097293	6646513.5384815\\
2308.097293	6647244.41377513\\
2309.097293	6647973.42604249\\
2310.097293	6648700.57528357\\
2311.097293	6649425.86149838\\
2312.097293	6650149.28468691\\
2313.097293	6650870.84484917\\
2314.097293	6651590.54198516\\
2315.097293	6652308.37609487\\
2316.097293	6653024.34717831\\
2317.097293	6653738.45523547\\
2318.097293	6654450.70026636\\
2319.097293	6655161.08227098\\
2320.097293	6655869.60124932\\
2321.097293	6656576.25720139\\
2322.097293	6657281.05012719\\
2323.097293	6657983.98002671\\
2324.097293	6658685.04689996\\
2325.097293	6659384.25074693\\
2326.097293	6660081.59156764\\
2327.097293	6660777.06936206\\
2328.097293	6661470.68413022\\
2329.097293	6662162.43587209\\
2330.097293	6662852.3245877\\
2331.097293	6663540.35027703\\
2332.097293	6664226.51294009\\
2333.097293	6664910.81257687\\
2334.097293	6665593.24918738\\
2335.097293	6666273.82277162\\
2336.097293	6666952.53332958\\
2337.097293	6667629.38086127\\
2338.097293	6668304.36536669\\
2339.097293	6668977.48684583\\
2340.097293	6669648.7452987\\
2341.097293	6670318.14072529\\
2342.097293	6670985.67312561\\
2343.097293	6671651.34249966\\
2344.097293	6672315.14884743\\
2345.097293	6672977.09216893\\
2346.097293	6673637.17246416\\
2347.097293	6674295.38973311\\
2348.097293	6674951.74397579\\
2349.097293	6675606.23519219\\
2350.097293	6676258.86338232\\
2351.097293	6676909.62854617\\
2352.097293	6677558.53068376\\
2353.097293	6678205.56979507\\
2354.097293	6678850.7458801\\
2355.097293	6679494.05893886\\
2356.097293	6680135.50897135\\
2357.097293	6680775.09597756\\
2358.097293	6681412.8199575\\
2359.097293	6682048.68091117\\
2360.097293	6682682.67883856\\
2361.097293	6683314.81373968\\
2362.097293	6683945.08561452\\
2363.097293	6684573.4944631\\
2364.097293	6685200.04028539\\
2365.097293	6685824.72308142\\
2366.097293	6686447.54285116\\
2367.097293	6687068.49959464\\
2368.097293	6687687.59331184\\
2369.097293	6688304.82400277\\
2370.097293	6688920.19166743\\
2371.097293	6689533.69630581\\
2372.097293	6690145.33791791\\
2373.097293	6690755.11650375\\
2374.097293	6691363.03206331\\
2375.097293	6691969.08459659\\
2376.097293	6692573.2741036\\
2377.097293	6693175.60058434\\
2378.097293	6693776.0640388\\
2379.097293	6694374.66446699\\
2380.097293	6694971.40186891\\
2381.097293	6695566.27624455\\
2382.097293	6696159.28759392\\
2383.097293	6696750.43591702\\
2384.097293	6697339.72121384\\
2385.097293	6697927.14348439\\
2386.097293	6698512.70272866\\
2387.097293	6699096.39894666\\
2388.097293	6699678.23213839\\
2389.097293	6700258.20230384\\
2390.097293	6700836.30944302\\
2391.097293	6701412.55355592\\
2392.097293	6701986.93464256\\
2393.097293	6702559.45270291\\
2394.097293	6703130.107737\\
2395.097293	6703698.89974481\\
2396.097293	6704265.82872634\\
2397.097293	6704830.8946816\\
2398.097293	6705394.09761059\\
2399.097293	6705955.43751331\\
2400.097293	6706514.91438975\\
2401.097293	6707072.52823992\\
2402.097293	6707628.27906381\\
2403.097293	6708182.16686143\\
2404.097293	6708734.19163278\\
2405.097293	6709284.35337785\\
2406.097293	6709832.65209664\\
2407.097293	6710379.08778917\\
2408.097293	6710923.66045542\\
2409.097293	6711466.3700954\\
2410.097293	6712007.2167091\\
2411.097293	6712546.20029653\\
2412.097293	6713083.32085769\\
2413.097293	6713618.57839257\\
2414.097293	6714151.97290117\\
2415.097293	6714683.50438351\\
2416.097293	6715213.17283957\\
2417.097293	6715740.97826936\\
2418.097293	6716266.92067287\\
2419.097293	6716791.00005011\\
2420.097293	6717313.21640107\\
2421.097293	6717833.56972576\\
2422.097293	6718352.06002418\\
2423.097293	6718868.68729633\\
2424.097293	6719383.4515422\\
2425.097293	6719896.35276179\\
2426.097293	6720407.39095512\\
2427.097293	6720916.56612216\\
2428.097293	6721423.87826294\\
2429.097293	6721929.32737744\\
2430.097293	6722432.91346567\\
2431.097293	6722934.63652762\\
2432.097293	6723434.4965633\\
2433.097293	6723932.49357271\\
2434.097293	6724428.62755584\\
2435.097293	6724922.8985127\\
2436.097293	6725415.30644328\\
2437.097293	6725905.8513476\\
2438.097293	6726394.53322563\\
2439.097293	6726881.3520774\\
2440.097293	6727366.30790288\\
2441.097293	6727849.4007021\\
2442.097293	6728330.63047504\\
2443.097293	6728809.99722171\\
2444.097293	6729287.50094211\\
2445.097293	6729763.14163623\\
2446.097293	6730236.91930407\\
2447.097293	6730708.83394565\\
2448.097293	6731178.88556095\\
2449.097293	6731647.07414997\\
2450.097293	6732113.39971272\\
2451.097293	6732577.8622492\\
2452.097293	6733040.46175941\\
2453.097293	6733501.19824334\\
2454.097293	6733960.07170099\\
2455.097293	6734417.08213238\\
2456.097293	6734872.22953749\\
2457.097293	6735325.51391632\\
2458.097293	6735776.93526888\\
2459.097293	6736226.49359517\\
2460.097293	6736674.18889518\\
2461.097293	6737120.02116892\\
2462.097293	6737563.99041639\\
2463.097293	6738006.09663758\\
2464.097293	6738446.3398325\\
2465.097293	6738884.72000115\\
2466.097293	6739321.23714352\\
2467.097293	6739755.89125962\\
2468.097293	6740188.68234944\\
2469.097293	6740619.61041299\\
2470.097293	6741048.67545027\\
2471.097293	6741475.87746127\\
2472.097293	6741901.216446\\
2473.097293	6742324.69240445\\
2474.097293	6742746.30533663\\
2475.097293	6743166.05524254\\
2476.097293	6743583.94212218\\
2477.097293	6743999.96597553\\
2478.097293	6744414.12680262\\
2479.097293	6744826.42460343\\
2480.097293	6745236.85937797\\
2481.097293	6745645.43112623\\
2482.097293	6746052.13984823\\
2483.097293	6746456.98554394\\
2484.097293	6746859.96821339\\
2485.097293	6747261.08785656\\
2486.097293	6747660.34447345\\
2487.097293	6748057.73806407\\
2488.097293	6748453.26862842\\
2489.097293	6748846.9361665\\
2490.097293	6749238.7406783\\
2491.097293	6749628.68216382\\
2492.097293	6750016.76062308\\
2493.097293	6750402.97605606\\
2494.097293	6750787.32846276\\
2495.097293	6751169.81784319\\
2496.097293	6751550.44419735\\
2497.097293	6751929.20752523\\
2498.097293	6752306.10782685\\
2499.097293	6752681.14510218\\
2500.097293	6753054.31935124\\
2501.097293	6753425.63057403\\
2502.097293	6753795.07877055\\
2503.097293	6754162.66394079\\
2504.097293	6754528.38608476\\
2505.097293	6754892.24520245\\
2506.097293	6755254.24129387\\
2507.097293	6755614.37435902\\
2508.097293	6755972.64439789\\
2509.097293	6756329.05141049\\
2510.097293	6756683.59539681\\
2511.097293	6757036.27635687\\
2512.097293	6757387.09429064\\
2513.097293	6757736.04919815\\
2514.097293	6758083.14107938\\
2515.097293	6758428.36993433\\
2516.097293	6758771.73576301\\
2517.097293	6759113.23856542\\
2518.097293	6759452.87834156\\
2519.097293	6759790.65509142\\
2520.097293	6760126.568815\\
2521.097293	6760460.61951232\\
2522.097293	6760792.80718336\\
2523.097293	6761123.13182812\\
2524.097293	6761451.59344662\\
2525.097293	6761778.19203883\\
2526.097293	6762102.92760478\\
2527.097293	6762425.80014445\\
2528.097293	6762746.80965785\\
2529.097293	6763065.95614497\\
2530.097293	6763383.23960582\\
2531.097293	6763698.66004039\\
2532.097293	6764012.2174487\\
2533.097293	6764323.91183072\\
2534.097293	6764633.74318648\\
2535.097293	6764941.71151596\\
2536.097293	6765247.81681916\\
2537.097293	6765552.0590961\\
2538.097293	6765854.43834676\\
2539.097293	6766154.95457114\\
2540.097293	6766453.60776925\\
2541.097293	6766750.39794109\\
2542.097293	6767045.32508665\\
2543.097293	6767338.38920594\\
2544.097293	6767629.59029896\\
2545.097293	6767918.9283657\\
2546.097293	6768206.40340617\\
2547.097293	6768492.01542037\\
2548.097293	6768775.76440829\\
2549.097293	6769057.65036994\\
2550.097293	6769337.67330531\\
2551.097293	6769615.83321441\\
2552.097293	6769892.13009724\\
2553.097293	6770166.56395379\\
2554.097293	6770439.13478407\\
2555.097293	6770709.84258807\\
2556.097293	6770978.6873658\\
2557.097293	6771245.66911726\\
2558.097293	6771510.78784244\\
2559.097293	6771774.04354135\\
2560.097293	6772035.43621399\\
2561.097293	6772294.96586035\\
2562.097293	6772552.63248044\\
2563.097293	6772808.43607426\\
2564.097293	6773062.37664179\\
2565.097293	6773314.45418306\\
2566.097293	6773564.66869805\\
2567.097293	6773813.02018677\\
2568.097293	6774059.50864922\\
2569.097293	6774304.13408539\\
2570.097293	6774546.89649529\\
2571.097293	6774787.79587891\\
2572.097293	6775026.83223626\\
2573.097293	6775264.00556734\\
2574.097293	6775499.31587214\\
2575.097293	6775732.76315067\\
2576.097293	6775964.34740293\\
2577.097293	6776194.06862891\\
2578.097293	6776421.92682862\\
2579.097293	6776647.92200205\\
2580.097293	6776872.05414921\\
2581.097293	6777094.3232701\\
2582.097293	6777314.72936471\\
2583.097293	6777533.27243305\\
2584.097293	6777749.95247511\\
2585.097293	6777964.76949091\\
2586.097293	6778177.72348042\\
2587.097293	6778388.81444367\\
2588.097293	6778598.04238064\\
2589.097293	6778805.40729133\\
2590.097293	6779010.90917575\\
2591.097293	6779214.5480339\\
2592.097293	6779416.32386578\\
2593.097293	6779616.23667138\\
2594.097293	6779814.28645071\\
2595.097293	6780010.47320376\\
2596.097293	6780204.79693054\\
2597.097293	6780397.25763105\\
2598.097293	6780587.85530528\\
2599.097293	6780776.58995324\\
2600.097293	6780963.46157492\\
2601.097293	6781148.47017033\\
2602.097293	6781331.61573947\\
2603.097293	6781512.89828233\\
2604.097293	6781692.31779892\\
2605.097293	6781869.87428924\\
2606.097293	6782045.56775328\\
2607.097293	6782219.39819105\\
2608.097293	6782391.36560254\\
2609.097293	6782561.46998776\\
2610.097293	6782729.71134671\\
2611.097293	6782896.08967938\\
2612.097293	6783060.60498578\\
2613.097293	6783223.25726591\\
2614.097293	6783384.04651976\\
2615.097293	6783542.97274734\\
2616.097293	6783700.03594864\\
2617.097293	6783855.23612367\\
2618.097293	6784008.57327243\\
2619.097293	6784160.04739491\\
2620.097293	6784309.65849112\\
2621.097293	6784457.40656105\\
2622.097293	6784603.29160471\\
2623.097293	6784747.3136221\\
2624.097293	6784889.47261322\\
2625.097293	6785029.76857806\\
2626.097293	6785168.20151662\\
2627.097293	6785304.77142891\\
2628.097293	6785439.47831493\\
2629.097293	6785572.32217468\\
2630.097293	6785703.30300815\\
2631.097293	6785832.42081535\\
2632.097293	6785959.67559627\\
2633.097293	6786085.06735092\\
2634.097293	6786208.5960793\\
2635.097293	6786330.2617814\\
2636.097293	6786450.06445723\\
2637.097293	6786568.00410678\\
2638.097293	6786684.08073006\\
2639.097293	6786798.29432707\\
2640.097293	6786910.6448978\\
2641.097293	6787021.13244226\\
2642.097293	6787129.75696045\\
2643.097293	6787236.51845236\\
2644.097293	6787341.416918\\
2645.097293	6787444.45235736\\
2646.097293	6787545.62477045\\
2647.097293	6787644.93415727\\
2648.097293	6787742.38051781\\
2649.097293	6787837.96385208\\
2650.097293	6787931.68416008\\
2651.097293	6788023.5414418\\
2652.097293	6788113.53569725\\
2653.097293	6788201.66692642\\
2654.097293	6788287.93512932\\
2655.097293	6788372.34030595\\
2656.097293	6788454.8824563\\
2657.097293	6788535.56158038\\
2658.097293	6788614.37767819\\
2659.097293	6788691.33074972\\
2660.097293	6788766.42079497\\
2661.097293	6788839.64781396\\
2662.097293	6788911.01180667\\
2663.097293	6788980.51277311\\
2664.097293	6789048.15071327\\
2665.097293	6789113.92562716\\
2666.097293	6789177.83751477\\
2667.097293	6789239.88637611\\
2668.097293	6789300.07221118\\
2669.097293	6789358.39501997\\
2670.097293	6789414.85480249\\
2671.097293	6789469.45155874\\
2672.097293	6789522.18528871\\
2673.097293	6789573.05599241\\
2674.097293	6789622.06366984\\
2675.097293	6789669.20832099\\
2676.097293	6789714.48994586\\
2677.097293	6789757.90854447\\
2678.097293	6789799.4641168\\
2679.097293	6789839.15666285\\
2680.097293	6789876.98618263\\
2681.097293	6789912.95267614\\
2682.097293	6789947.05614338\\
2683.097293	6789979.29658434\\
2684.097293	6790009.67399902\\
2685.097293	6790038.18838744\\
2686.097293	6790064.83974958\\
2687.097293	6790089.62808544\\
2688.097293	6790112.55339503\\
2689.097293	6790133.61567835\\
2690.097293	6790152.81493539\\
2691.097293	6790170.15116617\\
2692.097293	6790185.62437066\\
2693.097293	6790199.23454889\\
2694.097293	6790210.98170083\\
2695.097293	6790220.86582651\\
2696.097293	6790228.88692591\\
2697.097293	6790235.04499904\\
2698.097293	6790239.34004589\\
2699.097293	6790241.77206647\\
};
\addplot [color=mycolor2,solid,line width=2.0pt,forget plot]
  table[row sep=crcr]{%
2699.902707	6790242.37633132\\
3199.902707	6790242.37633132\\
};
\end{axis}
\end{tikzpicture}%
    \caption{Utility of reserves}
    \label{fig:ReserveUtility}
\end{figure}
\end{minipage}\\


Once again, the previous assumptions made still hold, but the hard constraint is replaced in favor of the operating reserve demand curve defined. The optimization model can be described as follows : \\

\begin{center}
\boxput*(0,1){\colorbox{white}{\textbf{ Operating Reserve Demand Curve [ORDC] }}}{
\setlength{\fboxsep}{10pt}
\fbox{\begin{minipage}{0.9\textwidth} \vspace{0.2cm}
\begin{align*}
 \max\limits_{r_g, p_g, r_1,r_2,r_3 \geq 0, p_{exch}} \quad  & r_1\: L + (a\: r_2 +2L)\: \frac{r_2}{2} - \sum_{g \in \G} \int_0^{p_g} MC_g(x) dx - 31.2424 \: \frac{p_{exch}^2}{2} - 0.005\: p_{exch} \\
\end{align*}
$$
\begin{array}{llll}
(\lambda)				& \sum_{g \in \G} p_g + p_{exch} + p^{t}_{r} \geq \D^{t} & & (1) \\
(\mu)					& r_1 + r_2 + r_3 = \sum_{g \in \G} r_g & & (2) \\
							& r_g \leq 15\: R^{t}_g & & (3) \\
							& p_g + r_g \leq P^{t}_g & & (4) \\
							& -ATC_1^{t} \leq p_{exch} \leq ATC_2^{t} & & (5) \\
							& r_1 \leq 16 & & (6) \\
							& r_2 \leq 2700 & & (7) \\
\end{array}
$$
\vspace{0.1cm}
\end{minipage}}}
\end{center}

We want to maximize the welfare due to the availability of reserves against the production cost. The parameter $L$ is the value of lost load, which is evaluated to $5000$ \euro /MW. And the variable $r_1$, $r_2$ and $r_3$ are the reserve quantities made in each part of the utility function (linear, quadratic and constant).