% QUESTION 4

The percent of generator revenue corresponding to the ancillary service is shown on table [\ref{pourcentage}] and figure [\ref{pourcent}]. We can see that quite logically, the cheapest generators (nuclear and biomass) are used for the production and not for the requirement of reserves. And the most expensive generators (oil) are almost exclusively used for making reserve. It makes sense since the reserves are cost-free, as opposed to the power production. Notice that since the ORDC model ask for more reserves than the two others, even the biomass generators have to participate in the procurement of reserves. Moreover, we can see (figure XXXX) that oil generators made a big leap towards profitability. The way the ORDC model guarantee the reliability of the system makes the oil generator far more valuable than expected.
 
\begin{table}[H]
\centering
\begin{tabular}{l | c  c  c}
type & EDR & ImpExp & ORDC \\
\hline
biomass &  $0.26$ & $0.53$ & $5.83$ \\
nuclear & $0$ & $0$ & $0$ \\
gas & $2$ & $3.42$ & $38.98$ \\
oil & $100$ & $100$ & $90.4$ \\
\end{tabular}
\caption{Percent of generator revenue corresponding to the ancillary services [\%]}
\label{pourcentage}
\end{table} 
 
\begin{figure}[H]
    \centering
    \setlength\fheight{4cm}
    \setlength\fwidth{0.8\textwidth}
    % This file was created by matlab2tikz.
% Minimal pgfplots version: 1.3
%
%The latest updates can be retrieved from
%  http://www.mathworks.com/matlabcentral/fileexchange/22022-matlab2tikz
%where you can also make suggestions and rate matlab2tikz.
%
\definecolor{mycolor1}{rgb}{0.84706,0.16078,0.00000}%
\definecolor{mycolor2}{rgb}{0.04314,0.51765,0.78039}%
\definecolor{mycolor3}{rgb}{0.87059,0.49020,0.00000}%
%
\begin{tikzpicture}

\begin{axis}[%
width=\fwidth,
height=\fheight,
at={(\fwidth,\fheight)},
scale only axis,
area legend,
separate axis lines,
every outer x axis line/.append style={black},
every x tick label/.append style={font=\color{black}},
xmin=0,
xmax=5,
xtick={0,1,2,3,4,5},
xticklabels={{},{Bio.},{Nucl.},{Gas},{Oil},{}},
xlabel={Generator type},
every outer y axis line/.append style={black},
every y tick label/.append style={font=\color{black}},
ymin=0,
ymax=110,
title={Percent of generator revenue corresponding to the ancillary services [\%]},
ymajorgrids,
legend style={at={(0.5,0.97)},anchor=north,legend columns=3,legend cell align=left,align=left,draw=white,fill=white}
]
\addplot[ybar,bar width=0.029662\fwidth,bar shift=-0.037077\fwidth,draw=black,fill=mycolor1] plot table[row sep=crcr] {%
1.00000	0.25971\\
2.00000	0.00000\\
3.00000	2.00764\\
4.00000	100.00000\\
};
\addlegendentry{ EDR};

\addplot [color=black,solid,forget plot]
  table[row sep=crcr]{%
0.50000	0.00000\\
4.50000	0.00000\\
};
\addplot[ybar,bar width=0.029662\fwidth,draw=black,fill=mycolor2] plot table[row sep=crcr] {%
1.00000	0.53238\\
2.00000	0.00000\\
3.00000	3.42201\\
4.00000	100.00005\\
};
\addlegendentry{ ImpExp};

\addplot[ybar,bar width=0.029662\fwidth,bar shift=0.037077\fwidth,draw=black,fill=mycolor3] plot table[row sep=crcr] {%
1.00000	5.82569\\
2.00000	0.00000\\
3.00000	38.97705\\
4.00000	90.40499\\
};
\addlegendentry{ ORDC};

\end{axis}
\end{tikzpicture}%
    \caption{Percent of generator revenue corresponding to the ancillary services [\%]}
    \label{pourcent}
\end{figure}
