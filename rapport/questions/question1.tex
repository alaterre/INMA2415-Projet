% QUESTION 1

The profit margin of each generator in the system are shown on figures [\ref{fig:biomass}],[\ref{fig:nuclear}], [\ref{fig:gas}], and [\ref{fig:oil}]. The dotted lines represent the investment cost for each type of generator. Apparently, generators tend to have a profit margin below their investment cost. Although one or two generators per technology play their cards right, shareholders are far from a return on their investments. The question is therefore, whether or not it is wise to continue to invest, based on those results. Hard to say. Take for example the nuclear generators. Even if it's not doing so bad, continue to invest in may reduce profitability. Indeed, nuclear generators are usually used to satisfy the most stable part of the demand. However, if it has already been fully sated, investing in nuclear will not increase profitability as new generators will not be used very often.\\

The second model leads to more profitable generators, even if the improvement isn't incredible. The only reason is that the demand level is higher than in the first model since the amount of energy imported from neighboring markets are not deduce anymore from the demand level.\\

Notice that almost all generators tends to be more profitable in the ORDC model. It comes from the fact that the ORDC model tends to ask for far more reserve than the fixed reserve requirement of the first two models. And since reserves have physical no cost, the generators generate huge profits without producing anything. \\

Finally, figure [\ref{fig:margin}] shows the average profit margin for each kind of generators and for each model. The conclusion is the same for all of them : generators are not money-making.

\begin{minipage}{0.495\textwidth} 
\begin{figure}[H]
    \centering
    \setlength\fheight{4cm}
    \setlength\fwidth{0.8\textwidth}
    % This file was created by matlab2tikz.
% Minimal pgfplots version: 1.3
%
%The latest updates can be retrieved from
%  http://www.mathworks.com/matlabcentral/fileexchange/22022-matlab2tikz
%where you can also make suggestions and rate matlab2tikz.
%
\definecolor{mycolor1}{rgb}{0.84706,0.16078,0.00000}%
\definecolor{mycolor2}{rgb}{0.04314,0.51765,0.78039}%
\definecolor{mycolor3}{rgb}{0.87059,0.49020,0.00000}%
%
\begin{tikzpicture}

\begin{axis}[%
width=\fwidth,
height=\fheight,
at={(\fwidth,0\fheight)},
scale only axis,
area legend,
separate axis lines,
every outer x axis line/.append style={black},
every x tick label/.append style={font=\color{black}},
xmin=0,
xmax=10,
xtick={1, 2, 3, 4, 5, 6, 7, 8, 9},
xlabel={Biomass Generator},
every outer y axis line/.append style={black},
every y tick label/.append style={font=\color{black}},
ymin=0,
ymax=1300,
title={Profit margin [\euro/ MW-day]},
ymajorgrids,
legend style={at={(0.5,0.97)},anchor=north,legend columns=4,legend cell align=left,align=left,draw=white,fill=white}
]
\addplot[ybar,bar width=0.02\fwidth,bar shift=-0.02\fwidth,draw=black,fill=mycolor1] plot table[row sep=crcr] {%
1	890.144524\\
2	469.454365\\
3	402.610557\\
4	331.182026\\
5	161.417783\\
6	113.799985\\
7	98.931328\\
8	1.094311\\
9	0.547255\\
};
\addlegendentry{ EDR};

\addplot [color=black,solid,forget plot]
  table[row sep=crcr]{%
0	0\\
10	0\\
};
\addplot[ybar,bar width=0.02\fwidth,draw=black,fill=mycolor2] plot table[row sep=crcr] {%
1	1000.319513\\
2	579.629353\\
3	509.511515\\
4	441.318705\\
5	260.517653\\
6	199.421224\\
7	178.554357\\
8	15.075046\\
9	9.041929\\
};
\addlegendentry{ ImpExp};

\addplot[ybar,bar width=0.02\fwidth,bar shift=0.02\fwidth,draw=black,fill=mycolor3] plot table[row sep=crcr] {%
1	1066.69968\\
2	646.009491\\
3	573.442362\\
4	507.712942\\
5	332.210293\\
6	269.651539\\
7	248.87988\\
8	61.88185\\
9	36.98737\\
};
\addlegendentry{ ORDC};

\addplot [color=black,dotted,line width=2.0pt]
  table[row sep=crcr]{%
0	669.6\\
10	669.6\\
};

\end{axis}
\end{tikzpicture}%
    \caption{Profit margin of biomass generators for each model}
    \label{fig:biomass}
\end{figure}
\end{minipage}
\begin{minipage}{0.495\textwidth} 
\begin{figure}[H]
    \centering
    \setlength\fheight{4cm}
    \setlength\fwidth{0.8\textwidth}
    % This file was created by matlab2tikz.
% Minimal pgfplots version: 1.3
%
%The latest updates can be retrieved from
%  http://www.mathworks.com/matlabcentral/fileexchange/22022-matlab2tikz
%where you can also make suggestions and rate matlab2tikz.
%
\definecolor{mycolor1}{rgb}{0.84706,0.16078,0.00000}%
\definecolor{mycolor2}{rgb}{0.04314,0.51765,0.78039}%
\definecolor{mycolor3}{rgb}{0.87059,0.49020,0.00000}%
%
\begin{tikzpicture}

\begin{axis}[%
width=\fwidth,
height=\fheight,
at={(\fwidth,\fheight)},
scale only axis,
area legend,
separate axis lines,
every outer x axis line/.append style={black},
every x tick label/.append style={font=\color{black}},
xmin=0,
xmax=8,
xlabel={Nuclear generator},
xtick={1, 2, 3, 4, 5, 6, 7},
every outer y axis line/.append style={black},
every y tick label/.append style={font=\color{black}},
ymin=0,
ymax=1000,
title={Profit margin [\euro / MW-day]},
ymajorgrids,
legend style={at={(0.5,0.97)},anchor=north,legend columns=4,legend cell align=left,align=left,draw=white,fill=white}
]
\addplot[ybar,bar width=0.02\fwidth,bar shift=-0.02\fwidth,draw=black,fill=mycolor1] plot table[row sep=crcr] {%
1	391.841753\\
2	625.488172\\
3	199.950964\\
4	662.371119\\
5	684.183826\\
6	194.318354\\
7	693.458351\\
};
\addlegendentry{ EDR};

\addplot [color=black,solid,forget plot]
  table[row sep=crcr]{%
0	0\\
8	0\\
};
\addplot[ybar,bar width=0.02\fwidth,draw=black,fill=mycolor2] plot table[row sep=crcr] {%
1	393.435515\\
2	635.629893\\
3	181.945509\\
4	681.775726\\
5	696.216683\\
6	176.455306\\
7	708.18522\\
};
\addlegendentry{ ImpExp};

\addplot[ybar,bar width=0.02\fwidth,bar shift=0.02\fwidth,draw=black,fill=mycolor3] plot table[row sep=crcr] {%
1	408.735275\\
2	666.686268\\
3	183.37229\\
4	736.209532\\
5	749.013116\\
6	177.860866\\
7	762.168769\\
};
\addlegendentry{ ORDC};

\addplot [color=black,dotted,line width=2.0pt]
  table[row sep=crcr]{%
0	762.2\\
8	762.2\\
};

\end{axis}
\end{tikzpicture}%
    \caption{Profit margin of nuclear generators for each model}
    \label{fig:nuclear}
\end{figure}
\end{minipage}\\
\begin{minipage}{\textwidth} 
\begin{figure}[H]
    \centering
    \setlength\fheight{4cm}
    \setlength\fwidth{0.9\textwidth}
    % This file was created by matlab2tikz.
% Minimal pgfplots version: 1.3
%
%The latest updates can be retrieved from
%  http://www.mathworks.com/matlabcentral/fileexchange/22022-matlab2tikz
%where you can also make suggestions and rate matlab2tikz.
%
\definecolor{mycolor1}{rgb}{0.84706,0.16078,0.00000}%
\definecolor{mycolor2}{rgb}{0.04314,0.51765,0.78039}%
\definecolor{mycolor3}{rgb}{0.87059,0.49020,0.00000}%
%
\begin{tikzpicture}

\begin{axis}[%
width=\fwidth,
height=\fheight,
at={(\fwidth,\fheight)},
scale only axis,
area legend,
separate axis lines,
every outer x axis line/.append style={black},
every x tick label/.append style={font=\color{black}},
xmin=0,
xmax=44,
xlabel={Gas generator},
every outer y axis line/.append style={black},
every y tick label/.append style={font=\color{black}},
ymin=0,
ymax=520,
title={Profit margin [\euro / MW-day]},
ymajorgrids,
legend style={at={(0.5,0.97)},anchor=north,legend columns=4,legend cell align=left,align=left,draw=white,fill=white}
]
\addplot[ybar,bar width=0.007\fwidth,bar shift=-0.007\fwidth,draw=black,fill=mycolor1] plot table[row sep=crcr] {%
1	446.137971\\
2	119.121063\\
3	121.601622\\
4	118.063942\\
5	94.394361\\
6	54.435866\\
7	63.681298\\
8	70.719834\\
9	68.925917\\
10	33.929164\\
11	20.890385\\
12	20.890385\\
13	14.703568\\
14	16.87723\\
15	16.717246\\
16	16.717246\\
17	7.133388\\
18	9.343958\\
19	5.982131\\
20	6.132066\\
21	4.521989\\
22	5.86518\\
23	5.026359\\
24	4.855727\\
25	3.288458\\
26	5.476274\\
27	3.697205\\
28	3.620683\\
29	3.503884\\
30	2.42636\\
31	2.044421\\
32	3.044049\\
33	5.078886\\
34	4.766866\\
35	4.766866\\
36	4.766866\\
37	4.74255\\
38	1.162388\\
39	1.012686\\
40	1.923808\\
41	4.676752\\
42	0.942268\\
43	4.664245\\
};
\addlegendentry{ EDR};

\addplot [color=black,solid,forget plot]
  table[row sep=crcr]{%
0	0\\
44	0\\
};
\addplot[ybar,bar width=0.007\fwidth,draw=black,fill=mycolor2] plot table[row sep=crcr] {%
1	458.38911\\
2	134.362042\\
3	140.619298\\
4	135.147026\\
5	111.628856\\
6	70.629623\\
7	81.120977\\
8	90.368348\\
9	88.540415\\
10	48.420255\\
11	35.482169\\
12	35.482169\\
13	26.236998\\
14	29.870809\\
15	29.632263\\
16	29.632266\\
17	12.442325\\
18	17.711461\\
19	11.147036\\
20	11.544586\\
21	11.024107\\
22	9.992967\\
23	8.975957\\
24	8.29519\\
25	4.490975\\
26	10.280113\\
27	5.049382\\
28	4.873917\\
29	4.622847\\
30	2.165265\\
31	2.053126\\
32	3.93939\\
33	9.462649\\
34	9.406491\\
35	9.406491\\
36	9.406492\\
37	9.40497\\
38	1.880904\\
39	1.880904\\
40	3.761807\\
41	9.404519\\
42	1.880903\\
43	9.404519\\
};
\addlegendentry{ ImpExp};

\addplot[ybar,bar width=0.007\fwidth,bar shift=0.007\fwidth,draw=black,fill=mycolor3] plot table[row sep=crcr] {%
1	509.031622\\
2	180.263545\\
3	196.525709\\
4	199.771903\\
5	173.423449\\
6	130.915863\\
7	157.82485\\
8	148.145281\\
9	146.553509\\
10	126.084697\\
11	106.28886\\
12	123.884776\\
13	102.467043\\
14	101.699373\\
15	101.500337\\
16	101.500409\\
17	68.059127\\
18	90.974734\\
19	91.417278\\
20	92.323003\\
21	84.675595\\
22	82.150777\\
23	77.675397\\
24	77.065699\\
25	50.51984\\
26	120.344399\\
27	68.36467\\
28	67.527385\\
29	66.255322\\
30	43.68047\\
31	39.731264\\
32	61.232983\\
33	116.482618\\
34	113.584585\\
35	113.584585\\
36	113.584605\\
37	113.39449\\
38	31.550451\\
39	30.131534\\
40	50.733724\\
41	112.7646\\
42	29.230676\\
43	112.537817\\
};
\addlegendentry{ ORDC};

\addplot [color=black,dotted,line width=2.0pt]
  table[row sep=crcr]{%
0	122.4\\
44	122.4\\
};

\end{axis}
\end{tikzpicture}%
        \caption{Profit margin of gas generators for each model}
    \label{fig:gas}
\end{figure}
\end{minipage} \\
\begin{minipage}{0.495\textwidth} 
\begin{figure}[H]
    \centering
    \setlength\fheight{4cm}
    \setlength\fwidth{0.8\textwidth}
    % This file was created by matlab2tikz.
% Minimal pgfplots version: 1.3
%
%The latest updates can be retrieved from
%  http://www.mathworks.com/matlabcentral/fileexchange/22022-matlab2tikz
%where you can also make suggestions and rate matlab2tikz.
%
\definecolor{mycolor1}{rgb}{0.84706,0.16078,0.00000}%
\definecolor{mycolor2}{rgb}{0.04314,0.51765,0.78039}%
\definecolor{mycolor3}{rgb}{0.87059,0.49020,0.00000}%
%
\begin{tikzpicture}

\begin{axis}[%
width=\fwidth,
height=\fheight,
at={(0\fwidth,0\fheight)},
scale only axis,
area legend,
separate axis lines,
every outer x axis line/.append style={black},
every x tick label/.append style={font=\color{black}},
xmin=0,
xmax=4,
xlabel={Oil generator},
every outer y axis line/.append style={black},
every y tick label/.append style={font=\color{black}},
ymin=0,
ymax=60,
title={Profit margin [\euro / MW-day]},
ymajorgrids,
legend style={at={(0.5,0.97)},anchor=north,legend columns=4,legend cell align=left,align=left,draw=white,fill=white}
]
\addplot[ybar,bar width=0.029\fwidth,bar shift=-0.037\fwidth,draw=black,fill=mycolor1] plot table[row sep=crcr] {%
1	0.932849\\
2	0.932849\\
3	0.932849\\
};
\addlegendentry{ EDR};

\addplot [color=black,solid,forget plot]
  table[row sep=crcr]{%
0	0\\
4	0\\
};
\addplot[ybar,bar width=0.029\fwidth,draw=black,fill=mycolor2] plot table[row sep=crcr] {%
1	1.880901\\
2	1.880902\\
3	1.880904\\
};
\addlegendentry{ ImpExp};

\addplot[ybar,bar width=0.029\fwidth,bar shift=0.037\fwidth,draw=black,fill=mycolor3] plot table[row sep=crcr] {%
1	24.773702\\
2	24.820957\\
3	24.225142\\
};
\addlegendentry{ ORDC};

\addplot [color=black,dotted,line width=2.0pt]
  table[row sep=crcr]{%
0	40.8\\
4	40.8\\
};

\end{axis}
\end{tikzpicture}%
        \caption{Profit margin of oil generators for each model}
    \label{fig:oil}
\end{figure}
\end{minipage}
\begin{minipage}{0.495\textwidth} 
\begin{figure}[H]
    \centering
    \setlength\fheight{4cm}
    \setlength\fwidth{0.8\textwidth}
    % This file was created by matlab2tikz.
% Minimal pgfplots version: 1.3
%
%The latest updates can be retrieved from
%  http://www.mathworks.com/matlabcentral/fileexchange/22022-matlab2tikz
%where you can also make suggestions and rate matlab2tikz.
%
\definecolor{mycolor1}{rgb}{0.84706,0.16078,0.00000}%
\definecolor{mycolor2}{rgb}{0.04314,0.51765,0.78039}%
\definecolor{mycolor3}{rgb}{0.87059,0.49020,0.00000}%
%
\begin{tikzpicture}

\begin{axis}[%
width=\fwidth,
height=\fheight,
at={(\fwidth,\fheight)},
scale only axis,
area legend,
separate axis lines,
every outer x axis line/.append style={black},
every x tick label/.append style={font=\color{black}},
xmin=0,
xmax=5,
xtick={0,1,2,3,4,5},
xticklabels={{},{Bio.},{Nucl.},{Gas},{Oil},{}},
xlabel={Generator type},
every outer y axis line/.append style={black},
every y tick label/.append style={font=\color{black}},
ymin=0,
ymax=650,
title={Average profit margin [\euro / MW-day]},
ymajorgrids,
legend style={at={(0.5,0.97)},anchor=north,legend columns=3,legend cell align=left,align=left,draw=white,fill=white}
]
\addplot[ybar,bar width=0.029662\fwidth,bar shift=-0.037077\fwidth,draw=black,fill=mycolor1] plot table[row sep=crcr] {%
1	278.961397333333\\
2	493.087505571429\\
3	32.8435676976744\\
4	0.932849\\
};
\addlegendentry{ EDR};

\addplot [color=black,solid,forget plot]
  table[row sep=crcr]{%
0	0\\
5	0\\
};
\addplot[ybar,bar width=0.029662\fwidth,draw=black,fill=mycolor2] plot table[row sep=crcr] {%
1	286.350823666667\\
2	496.234836\\
3	40.6853934186047\\
4	1.88090233333333\\
};
\addlegendentry{ ImpExp};

\addplot[ybar,bar width=0.029662\fwidth,bar shift=0.037077\fwidth,draw=black,fill=mycolor3] plot table[row sep=crcr] {%
1	337.644301222222\\
2	526.292302285714\\
3	109.894391953488\\
4	24.6066003333333\\
};
\addlegendentry{ ORDC};

\end{axis}
\end{tikzpicture}%
        \caption{Average profit margin of biomass generators for each model}
    \label{fig:margin}
\end{figure}
\end{minipage}
