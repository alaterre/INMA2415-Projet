% QUESTION 5

The reserve prices for the different models are available in figure [\ref{fig:EDR_R}], [\ref{fig:ImpExp_R}] et [\ref{fig:ORDC_R1}], and a zoom for the third model  is available on figure [\ref{fig:ORDC_R2}]. Notice that, as expected, the reserve prices are lower than energy prices. It can be interesting to  compare such results with the figure [\ref{fig:ratio}] which is re-displayed on figure [\ref{fig:ratio2}].

\begin{figure}[H]
    \centering
    \setlength\fheight{4cm}
    \setlength\fwidth{0.8\textwidth}
    % This file was created by matlab2tikz.
% Minimal pgfplots version: 1.3
%
%The latest updates can be retrieved from
%  http://www.mathworks.com/matlabcentral/fileexchange/22022-matlab2tikz
%where you can also make suggestions and rate matlab2tikz.
%
\definecolor{mycolor1}{rgb}{0.84706,0.16078,0.00000}%
%
\begin{tikzpicture}

\begin{axis}[%
width=\fwidth,
height=\fheight,
at={(0\fwidth,0\fheight)},
scale only axis,
separate axis lines,
every outer x axis line/.append style={black},
every x tick label/.append style={font=\color{black}},
xmin=0,
xmax=8760,
xlabel={time [hour]},
xmajorgrids,
every outer y axis line/.append style={black},
every y tick label/.append style={font=\color{black}},
ymin=0.5,
ymax=2.5,
ylabel={ratio level},
ymajorgrids
]
\addplot [color=mycolor1,line width=1.3pt,solid,forget plot]
  table[row sep=crcr]{%
1	1.76277299026698\\
2	1.9362109758773\\
3	2.01383811417918\\
4	2.14556960277483\\
5	2.21691525772622\\
6	2.25450999839374\\
7	2.26315544633488\\
8	2.35434350380979\\
9	2.43391137968296\\
10	2.41028010959221\\
11	2.32649951900646\\
12	2.31734256915795\\
13	2.27152828955165\\
14	2.17701182593871\\
15	2.18950187093352\\
16	2.16738441375517\\
17	2.10329602620842\\
18	1.97969107222188\\
19	2.0982931513597\\
20	2.12403009322197\\
21	2.06913982707518\\
22	2.07783314631548\\
23	1.97292246943499\\
24	1.93649148120353\\
25	2.13109399445936\\
26	2.28501729833992\\
27	2.40536610933119\\
28	2.48094107909005\\
29	2.44346375633962\\
30	2.27445515666959\\
31	2.03308065689467\\
32	1.88155208674845\\
33	1.73216866221003\\
34	1.64000652518427\\
35	1.60057681297788\\
36	1.58151613686274\\
37	1.58058487502011\\
38	1.56606783677838\\
39	1.58544314319611\\
40	1.55974041513492\\
41	1.52603227566969\\
42	1.53144300145539\\
43	1.525416005635\\
44	1.52953071537887\\
45	1.59593745591488\\
46	1.67644591024023\\
47	1.63861553513276\\
48	1.65121653230106\\
49	1.77613915853804\\
50	1.94773556379149\\
51	2.03533445565709\\
52	2.11733591042659\\
53	2.11885435849448\\
54	2.0276289081425\\
55	1.85174931598301\\
56	1.73599673433243\\
57	1.66927759045363\\
58	1.66690536535875\\
59	1.61999094793536\\
60	1.59120753866144\\
61	1.58786097022828\\
62	1.62368626516258\\
63	1.66078900659206\\
64	1.64645781444153\\
65	1.59058698874881\\
66	1.51986726206597\\
67	1.61304799908995\\
68	1.57010443545918\\
69	1.62070690520079\\
70	1.66772940795751\\
71	1.62282496964554\\
72	1.61791168836035\\
73	1.75267714540524\\
74	1.83123061439868\\
75	1.92988005157148\\
76	1.98364437652562\\
77	1.99543556846451\\
78	1.98695024960769\\
79	1.90051604156909\\
80	1.85169720657404\\
81	1.71267194465335\\
82	1.64541711806509\\
83	1.64862209604406\\
84	1.60332965175916\\
85	1.67281694661606\\
86	1.59948840821261\\
87	1.61107300828397\\
88	1.60686449608187\\
89	1.56867037987429\\
90	1.51548309597428\\
91	1.57540234481252\\
92	1.62941053054905\\
93	1.70433420248443\\
94	1.63735890708814\\
95	1.63629059667911\\
96	1.60060428716605\\
97	1.60333361776825\\
98	1.708673988582\\
99	1.80413887459463\\
100	1.84530409485739\\
101	1.85130687020377\\
102	1.83900078997816\\
103	1.84939800153386\\
104	1.83193932063438\\
105	1.77380157351207\\
106	1.7142700447578\\
107	1.69152794751149\\
108	1.66794040072872\\
109	1.65017016272179\\
110	1.6452922441227\\
111	1.68088411609648\\
112	1.64843811576025\\
113	1.59744676877715\\
114	1.47500979093346\\
115	1.50530933054361\\
116	1.55305918071202\\
117	1.62607089550037\\
118	1.77848352311283\\
119	1.81048840080216\\
120	1.77720217708164\\
121	1.92955755097854\\
122	2.0586650212972\\
123	2.1598769632186\\
124	2.24455928924168\\
125	2.21228853250866\\
126	2.03376000182603\\
127	1.75874791088956\\
128	1.54699661737417\\
129	1.56648080193148\\
130	1.55673122189553\\
131	1.45903852996141\\
132	1.45690692494384\\
133	1.49567169832483\\
134	1.48083543336548\\
135	1.4779962361954\\
136	1.47477542990651\\
137	1.44906650377054\\
138	1.3604406435506\\
139	1.41912403058445\\
140	1.50388994692369\\
141	1.45971957118769\\
142	1.54310988359145\\
143	1.53823670087808\\
144	1.5957179230879\\
145	1.71352765438218\\
146	1.83434967945918\\
147	1.9298084065673\\
148	1.97640821697432\\
149	1.93860794318979\\
150	1.8216968497538\\
151	1.60093419425915\\
152	1.44256959766703\\
153	1.36260426079029\\
154	1.41204413651093\\
155	1.43865777808738\\
156	1.43382352713922\\
157	1.47426399251346\\
158	1.45456417218019\\
159	1.43658453492889\\
160	1.36700176783911\\
161	1.36056436315477\\
162	1.28154614249717\\
163	1.27501489230533\\
164	1.35115937186708\\
165	1.40542391889273\\
166	1.42776817561145\\
167	1.41974512425605\\
168	1.36596777402959\\
169	1.49388492672713\\
170	1.61341759269041\\
171	1.76656294926166\\
172	1.80540856979635\\
173	1.74095449895641\\
174	1.62969123554603\\
175	1.44203464831519\\
176	1.35588876972628\\
177	1.26549609812125\\
178	1.31240747128363\\
179	1.35282654998826\\
180	1.37722093617662\\
181	1.32003445397444\\
182	1.20120839897165\\
183	1.17864165503985\\
184	1.14837152696739\\
185	1.1324719261063\\
186	1.09369475619244\\
187	1.08296583518112\\
188	1.13393964490555\\
189	1.1534939896481\\
190	1.21425820839644\\
191	1.2072493286163\\
192	1.24985209442554\\
193	1.47786225976837\\
194	1.60703148308646\\
195	1.68771620741077\\
196	1.71999475243835\\
197	1.71036030092028\\
198	1.63790889272756\\
199	1.47141800969484\\
200	1.33413454908006\\
201	1.28769053648502\\
202	1.34515586211072\\
203	1.34616681130741\\
204	1.33537541599863\\
205	1.2984571870324\\
206	1.29047608089168\\
207	1.30252820413333\\
208	1.29644384757354\\
209	1.30748395845728\\
210	1.29851816762616\\
211	1.28592995826083\\
212	1.33319453925039\\
213	1.40518548656661\\
214	1.40175012865366\\
215	1.37204876271802\\
216	1.37931051578157\\
217	1.44203407418592\\
218	1.5329550020917\\
219	1.58922488996188\\
220	1.60005729224535\\
221	1.56657308444984\\
222	1.50507658910994\\
223	1.344867776067\\
224	1.21421912255514\\
225	1.18686565116026\\
226	1.241115355355\\
227	1.26697313609213\\
228	1.24434454821355\\
229	1.30892654343811\\
230	1.28631726467607\\
231	1.27072022446387\\
232	1.25486438864895\\
233	1.23053138918279\\
234	1.19175647135829\\
235	1.21057509119057\\
236	1.27222433362268\\
237	1.31865765608143\\
238	1.34677117714842\\
239	1.32770463768483\\
240	1.32075888993452\\
241	1.46066681046678\\
242	1.547044957561\\
243	1.58624427321258\\
244	1.64651415851665\\
245	1.67388840923648\\
246	1.65220191938573\\
247	1.59427635094351\\
248	1.59892002270501\\
249	1.45262103531662\\
250	1.40824512293663\\
251	1.39409060694883\\
252	1.39127721090953\\
253	1.3862368597072\\
254	1.38264197700962\\
255	1.41432073847967\\
256	1.40449281104224\\
257	1.38033142510119\\
258	1.30227087688349\\
259	1.32648268347874\\
260	1.36001145131298\\
261	1.43664635278679\\
262	1.41519867049676\\
263	1.36716440482475\\
264	1.44093747579285\\
265	1.42708775687099\\
266	1.45579779437206\\
267	1.5018144498523\\
268	1.54531878932517\\
269	1.5607828473973\\
270	1.57020401355159\\
271	1.55422790806552\\
272	1.53007104958006\\
273	1.47780263037668\\
274	1.44850251576378\\
275	1.44763465213742\\
276	1.48436113014352\\
277	1.48922489452031\\
278	1.52269035611753\\
279	1.58907165256958\\
280	1.55955281541541\\
281	1.52052486096639\\
282	1.39567062084007\\
283	1.3939770892222\\
284	1.42990942972091\\
285	1.50906604656547\\
286	1.58245128202283\\
287	1.60741687258773\\
288	1.5207598905816\\
289	1.67452532482118\\
290	1.77546489765398\\
291	1.85957918589554\\
292	1.91402148458577\\
293	1.89650070517672\\
294	1.74696708840451\\
295	1.50501748465052\\
296	1.3798084036594\\
297	1.38313294903386\\
298	1.3611042249003\\
299	1.31881420134862\\
300	1.3090446410692\\
301	1.33623091364315\\
302	1.33123361366905\\
303	1.34069078970888\\
304	1.32718363395993\\
305	1.30857466833095\\
306	1.28471155305273\\
307	1.25691034598817\\
308	1.33335631294786\\
309	1.36409933389232\\
310	1.42725057367219\\
311	1.4094913890812\\
312	1.43161837728613\\
313	1.574892929793\\
314	1.66336113250588\\
315	1.71557384708476\\
316	1.74978059408488\\
317	1.72650176668612\\
318	1.6179268743252\\
319	1.43004668188753\\
320	1.29662255454809\\
321	1.28774957483874\\
322	1.26190178514174\\
323	1.25859273979119\\
324	1.24772169123354\\
325	1.26472785981096\\
326	1.27342532226228\\
327	1.25336353826995\\
328	1.24188587235901\\
329	1.2256149716125\\
330	1.18724701068376\\
331	1.21491913485517\\
332	1.24042163532429\\
333	1.26041684345534\\
334	1.33549368655742\\
335	1.31910208885942\\
336	1.34178377902336\\
337	1.4414563995484\\
338	1.51649026335007\\
339	1.58874500977674\\
340	1.64401231065145\\
341	1.64333927896388\\
342	1.58487919328203\\
343	1.44765681706216\\
344	1.32349924033057\\
345	1.32040389513651\\
346	1.32683655341044\\
347	1.30646473725458\\
348	1.30183153724242\\
349	1.31938225968908\\
350	1.31653810883307\\
351	1.33728867550053\\
352	1.33449907545004\\
353	1.32684582732477\\
354	1.29624514138962\\
355	1.30342826413165\\
356	1.31598558017386\\
357	1.37149665391699\\
358	1.44411083117731\\
359	1.42273409673338\\
360	1.43288212789452\\
361	1.56156681111608\\
362	1.65805066814055\\
363	1.73383723304027\\
364	1.80027569463498\\
365	1.79076921017304\\
366	1.70237334242708\\
367	1.50627034577754\\
368	1.38280450851465\\
369	1.40363232494979\\
370	1.38039305901844\\
371	1.38561609737659\\
372	1.35886004449793\\
373	1.33061081263968\\
374	1.2762345619689\\
375	1.28868787611307\\
376	1.30370296482384\\
377	1.31588247829049\\
378	1.26232366833322\\
379	1.29904372010335\\
380	1.32493243669322\\
381	1.36408016984428\\
382	1.45092710060791\\
383	1.44301840641233\\
384	1.46803816272251\\
385	1.59014635166389\\
386	1.69124572927249\\
387	1.77323684872064\\
388	1.79692178576831\\
389	1.77942376903925\\
390	1.70842889842567\\
391	1.52400821779961\\
392	1.43431186190209\\
393	1.41364201785654\\
394	1.37109301865032\\
395	1.38852269101093\\
396	1.41491375462163\\
397	1.45016793839661\\
398	1.43598470585223\\
399	1.43376523935839\\
400	1.39392912081024\\
401	1.34787697522538\\
402	1.29405727168018\\
403	1.33263931757675\\
404	1.36772197434555\\
405	1.39394917679652\\
406	1.45785006353957\\
407	1.43687687000337\\
408	1.43756091893541\\
409	1.54513831675944\\
410	1.64113865265369\\
411	1.71862852481592\\
412	1.76524859711787\\
413	1.78858206471907\\
414	1.78215758801291\\
415	1.72590314407838\\
416	1.70769707835731\\
417	1.65374752305444\\
418	1.58853190547876\\
419	1.60792899698593\\
420	1.62537315846665\\
421	1.67945378457126\\
422	1.63589096353438\\
423	1.63413072563917\\
424	1.61398532914327\\
425	1.5598769260042\\
426	1.46610498216442\\
427	1.45798021940989\\
428	1.51970504444844\\
429	1.50500070260286\\
430	1.55883645125382\\
431	1.5642349073119\\
432	1.51760651649515\\
433	1.62134594600579\\
434	1.7299694453504\\
435	1.82609237324793\\
436	1.88109326894015\\
437	1.89297688525953\\
438	1.86954481734432\\
439	1.81858381661576\\
440	1.77531511191274\\
441	1.6962042059528\\
442	1.61739867472062\\
443	1.54522851459845\\
444	1.51219411784021\\
445	1.49357737919357\\
446	1.50762819614386\\
447	1.53846278673096\\
448	1.5369003657501\\
449	1.52938868049174\\
450	1.41576693939692\\
451	1.41954990165677\\
452	1.42689013367824\\
453	1.46615712933264\\
454	1.46560038353042\\
455	1.45953229505914\\
456	1.45174464719133\\
457	1.54359611563728\\
458	1.61802584355109\\
459	1.68459873285971\\
460	1.72303378197357\\
461	1.69861027943118\\
462	1.58686706626664\\
463	1.39976813220679\\
464	1.27520468506693\\
465	1.2847985204491\\
466	1.24220220381518\\
467	1.22855651510009\\
468	1.22016794031514\\
469	1.2450583394897\\
470	1.21590632775143\\
471	1.2219561900253\\
472	1.2158114460006\\
473	1.21235429323892\\
474	1.1916377178075\\
475	1.21335468320486\\
476	1.27196216754927\\
477	1.26932299767581\\
478	1.3755625747909\\
479	1.35404410395878\\
480	1.40172457699525\\
481	1.45344584899125\\
482	1.53708673344626\\
483	1.60293159732125\\
484	1.62303194170461\\
485	1.60835691862427\\
486	1.53757066792223\\
487	1.37327195455553\\
488	1.25152693074666\\
489	1.27313193920395\\
490	1.31139085201476\\
491	1.31294834155599\\
492	1.2786600410626\\
493	1.29458152892861\\
494	1.26897251758148\\
495	1.26593196571905\\
496	1.25878626657506\\
497	1.24713192049216\\
498	1.19395580505083\\
499	1.23576849880719\\
500	1.24268104965154\\
501	1.28854048682447\\
502	1.37442629548452\\
503	1.35954938540214\\
504	1.40446036298426\\
505	1.50441005927358\\
506	1.5996712178095\\
507	1.65111979905229\\
508	1.70210305802094\\
509	1.69733858565748\\
510	1.6188021154037\\
511	1.44519807008918\\
512	1.3313803862551\\
513	1.39051310205313\\
514	1.36482779727784\\
515	1.40483236387018\\
516	1.40081751171981\\
517	1.41690477471042\\
518	1.36939238212201\\
519	1.34687223843381\\
520	1.31890100056159\\
521	1.29432370040765\\
522	1.23448013905997\\
523	1.25388636083406\\
524	1.26339458873812\\
525	1.30136029501626\\
526	1.36069355959173\\
527	1.334616455149\\
528	1.3479062319039\\
529	1.45481212646417\\
530	1.52166570116622\\
531	1.56446933306228\\
532	1.61141915937192\\
533	1.60603872291545\\
534	1.53835877765572\\
535	1.38940337070939\\
536	1.27584549408996\\
537	1.29285316959904\\
538	1.28897411425697\\
539	1.29391135078365\\
540	1.30131126627011\\
541	1.31787921464603\\
542	1.29506739435805\\
543	1.30635435807575\\
544	1.30187104975722\\
545	1.3027247225854\\
546	1.25822590655936\\
547	1.29906035509768\\
548	1.33939292922362\\
549	1.38578264052682\\
550	1.45717525502922\\
551	1.43312031928977\\
552	1.44900088710226\\
553	1.55937696839204\\
554	1.63501686817876\\
555	1.69825399513108\\
556	1.71673484929485\\
557	1.68000184351256\\
558	1.59031421552231\\
559	1.39909073282687\\
560	1.27103151627006\\
561	1.29705670329196\\
562	1.31298671718102\\
563	1.33931474977036\\
564	1.34971069486066\\
565	1.36991431763676\\
566	1.37300173925463\\
567	1.3676158718002\\
568	1.3590842523661\\
569	1.34043924350929\\
570	1.28751488318873\\
571	1.31159591883922\\
572	1.34074268343609\\
573	1.38428162882628\\
574	1.46728353059486\\
575	1.42766947528636\\
576	1.43077365205217\\
577	1.46965592667972\\
578	1.55409949860337\\
579	1.63112223478094\\
580	1.68753651701994\\
581	1.72385709531491\\
582	1.7486000407942\\
583	1.71485279364862\\
584	1.65341377580699\\
585	1.55181332510421\\
586	1.50246746378642\\
587	1.47591604266792\\
588	1.47817806382629\\
589	1.47648757596111\\
590	1.49668606436226\\
591	1.53429645667311\\
592	1.54657927475067\\
593	1.54819179671975\\
594	1.50107870369551\\
595	1.50691695757208\\
596	1.55004630281238\\
597	1.63408831096859\\
598	1.70978822012148\\
599	1.6944591711967\\
600	1.66074523024259\\
601	1.84524928648012\\
602	1.97427958129578\\
603	1.9660597947546\\
604	1.99934685014634\\
605	1.9965626800371\\
606	1.95879041668584\\
607	1.88856356217999\\
608	1.84397626118771\\
609	1.74380632244057\\
610	1.67052369384502\\
611	1.62711929326756\\
612	1.62432160287537\\
613	1.68293577762177\\
614	1.64192684402621\\
615	1.69827504430018\\
616	1.71281082957701\\
617	1.71714891336484\\
618	1.64253052822686\\
619	1.70810125833876\\
620	1.71641323052057\\
621	1.610939613585\\
622	1.68466596648788\\
623	1.66924164434493\\
624	1.68926615694046\\
625	1.79173315575833\\
626	1.8895456475136\\
627	1.95497158077937\\
628	1.9632395147859\\
629	1.93272569793698\\
630	1.8068722183789\\
631	1.56907480502023\\
632	1.42438792911335\\
633	1.41629012897633\\
634	1.42021381795637\\
635	1.41449757964959\\
636	1.42114761720527\\
637	1.41624134453463\\
638	1.4105352894036\\
639	1.40796312305481\\
640	1.38083416960464\\
641	1.35758903685862\\
642	1.37651597731509\\
643	1.34087984697575\\
644	1.3943740736046\\
645	1.40900782611426\\
646	1.48405206524128\\
647	1.46675979578271\\
648	1.50042115530657\\
649	1.57704352113774\\
650	1.68550491404402\\
651	1.75785485568288\\
652	1.80529597597514\\
653	1.80472682539044\\
654	1.72292484599695\\
655	1.52731434800408\\
656	1.39117531321251\\
657	1.43240077204004\\
658	1.43476001300265\\
659	1.4282522784179\\
660	1.41731447177099\\
661	1.43466689998475\\
662	1.39747231627165\\
663	1.38870433466752\\
664	1.38557623323142\\
665	1.37930929628494\\
666	1.36208197108929\\
667	1.35616450006906\\
668	1.41736774690988\\
669	1.40405316011646\\
670	1.47956539000948\\
671	1.41024674014638\\
672	1.42344418122247\\
673	1.52325803237152\\
674	1.59533141440017\\
675	1.6505564709669\\
676	1.69652727946484\\
677	1.68846195401973\\
678	1.62230223076158\\
679	1.48692803428247\\
680	1.36224369716512\\
681	1.3989263607249\\
682	1.41634710975467\\
683	1.45541238110692\\
684	1.46336381528691\\
685	1.50759412554392\\
686	1.46231899419032\\
687	1.46864593217488\\
688	1.41624323824673\\
689	1.38063134541111\\
690	1.35477776778578\\
691	1.3124969732399\\
692	1.35691058341962\\
693	1.39050102140604\\
694	1.45048734994841\\
695	1.41514120107166\\
696	1.41968223960922\\
697	1.50658799381731\\
698	1.56716720961811\\
699	1.61672363131233\\
700	1.64387382392022\\
701	1.64758434846582\\
702	1.57114311960646\\
703	1.39538546285337\\
704	1.27453165043858\\
705	1.32637338316898\\
706	1.28784333473782\\
707	1.32630073733953\\
708	1.31373001990013\\
709	1.35353770978764\\
710	1.3389170180987\\
711	1.33448210637232\\
712	1.31638979835614\\
713	1.27744058876725\\
714	1.23086904047905\\
715	1.24328449590893\\
716	1.28443134705923\\
717	1.28831422952243\\
718	1.34981910891928\\
719	1.31751590455564\\
720	1.3250984687833\\
721	1.41029381786496\\
722	1.47271568453849\\
723	1.52511327109967\\
724	1.55530556132148\\
725	1.55298156179936\\
726	1.49084713950088\\
727	1.34552300752499\\
728	1.24355906130386\\
729	1.32191630991776\\
730	1.36580484448721\\
731	1.38818902663135\\
732	1.42453607201295\\
733	1.50568883311062\\
734	1.48015718000949\\
735	1.50877943097414\\
736	1.51444399329173\\
737	1.46671678543549\\
738	1.41540005388023\\
739	1.39675921516997\\
740	1.47420084196088\\
741	1.51309090010367\\
742	1.60314345034369\\
743	1.55891283964275\\
744	1.5663111153011\\
745	1.69312795693061\\
746	1.81661551150348\\
747	1.90260740418457\\
748	2.01874083549618\\
749	2.03977513107709\\
750	2.06088406889545\\
751	1.98006914670823\\
752	1.88272667411938\\
753	1.74453497973873\\
754	1.62420122970536\\
755	1.57111266755559\\
756	1.55641415600064\\
757	1.56620462673971\\
758	1.61877454000787\\
759	1.67990588330399\\
760	1.68256976758404\\
761	1.64874445800944\\
762	1.59728238502265\\
763	1.57546667661705\\
764	1.62058435974641\\
765	1.61118186076976\\
766	1.69074968040058\\
767	1.66389200712825\\
768	1.63749096504245\\
769	1.74521546561771\\
770	1.82523954905672\\
771	1.88369980800598\\
772	1.94603156291641\\
773	1.9931706110492\\
774	1.97706548395676\\
775	1.9657631653116\\
776	1.93629872464328\\
777	1.92762780438199\\
778	1.85127747329721\\
779	1.80614912803815\\
780	1.84183323828688\\
781	1.82971978275727\\
782	1.80673575813586\\
783	1.83209099976229\\
784	1.80146496904007\\
785	1.74112018505493\\
786	1.63620329096007\\
787	1.56239758790357\\
788	1.56133371021367\\
789	1.56526549258313\\
790	1.60291230934364\\
791	1.58096706003214\\
792	1.57160079126442\\
793	1.65024041418316\\
794	1.69549572236993\\
795	1.76453354348502\\
796	1.77771965203256\\
797	1.76434613352481\\
798	1.66792294655795\\
799	1.4726994966479\\
800	1.35123674826025\\
801	1.3879343116216\\
802	1.42620771940288\\
803	1.49182620814653\\
804	1.52114691085046\\
805	1.53887394517554\\
806	1.5011412453587\\
807	1.47808953502875\\
808	1.43111661761871\\
809	1.38545348625869\\
810	1.31397426159115\\
811	1.27892673805515\\
812	1.32452516571302\\
813	1.36838401324018\\
814	1.44268179624416\\
815	1.4221425234759\\
816	1.44205624768149\\
817	1.54633180058829\\
818	1.634870377183\\
819	1.70714180752753\\
820	1.76368526898442\\
821	1.76324107776226\\
822	1.68098099772347\\
823	1.48712609375201\\
824	1.36581602185607\\
825	1.41633156847929\\
826	1.41757913236043\\
827	1.42518289338379\\
828	1.41870378206515\\
829	1.48406492380573\\
830	1.4818217618441\\
831	1.49099733914413\\
832	1.45618971466303\\
833	1.42669944015305\\
834	1.37498702569358\\
835	1.33216594053293\\
836	1.39644531816491\\
837	1.41938768124388\\
838	1.52003720578367\\
839	1.49704043314214\\
840	1.521003682554\\
841	1.64521534923316\\
842	1.74865687424519\\
843	1.844020267108\\
844	1.89804313849664\\
845	1.88111088606937\\
846	1.80505130214543\\
847	1.58980536237336\\
848	1.48629598035734\\
849	1.42788900608666\\
850	1.44257561170201\\
851	1.42186849609995\\
852	1.42859525501049\\
853	1.4633932732358\\
854	1.4658599392089\\
855	1.50307760055265\\
856	1.47631532693688\\
857	1.44151765967084\\
858	1.36654744516495\\
859	1.37347738727311\\
860	1.42753918173368\\
861	1.48569428473421\\
862	1.54035622844191\\
863	1.52261597863579\\
864	1.53469167066894\\
865	1.65119713434204\\
866	1.75721840912235\\
867	1.82476338763658\\
868	1.86716994115569\\
869	1.84120614219687\\
870	1.76403278965096\\
871	1.57350134328106\\
872	1.49106334555056\\
873	1.49366230267771\\
874	1.47881111906073\\
875	1.47012370188843\\
876	1.45904716839532\\
877	1.45729861984901\\
878	1.42416624889929\\
879	1.42113976209236\\
880	1.42193104176669\\
881	1.40334629441289\\
882	1.35390958758701\\
883	1.37994750221938\\
884	1.41930795610026\\
885	1.50889179359612\\
886	1.55461034613988\\
887	1.5214481089317\\
888	1.54144564960204\\
889	1.66247581609949\\
890	1.74837899723901\\
891	1.8313750521402\\
892	1.87991195674547\\
893	1.87487045198552\\
894	1.80906090903083\\
895	1.61538029761186\\
896	1.50656396502617\\
897	1.50562551414465\\
898	1.47992271558814\\
899	1.42984968458926\\
900	1.36647573655411\\
901	1.399265528488\\
902	1.39152636079161\\
903	1.40714185134752\\
904	1.43237364118373\\
905	1.4796397109341\\
906	1.45499891954891\\
907	1.42255533792565\\
908	1.4637792016313\\
909	1.47571063943682\\
910	1.54560279274649\\
911	1.49637063699369\\
912	1.50644402335816\\
913	1.64095134846136\\
914	1.76655506853929\\
915	1.85169610411146\\
916	1.9550732967038\\
917	1.98811424541056\\
918	2.0295579324086\\
919	1.99456912037404\\
920	1.90831344503783\\
921	1.88240140879383\\
922	1.72506653982239\\
923	1.64743988972254\\
924	1.62896914972356\\
925	1.63508577264848\\
926	1.62693591095608\\
927	1.68858655290345\\
928	1.70165051868589\\
929	1.71211723253997\\
930	1.69522605053988\\
931	1.64213539082683\\
932	1.69361384529426\\
933	1.81790138662435\\
934	1.80487043961167\\
935	1.78939048649405\\
936	1.70467365940436\\
937	1.8437246357153\\
938	1.96291987526031\\
939	2.06159363965088\\
940	2.11625509932\\
941	2.14305138463847\\
942	2.14450642632571\\
943	2.07689007586728\\
944	2.03632435745472\\
945	1.98279245571543\\
946	1.88143228691825\\
947	1.85378804321048\\
948	1.82342656930013\\
949	1.81059847081694\\
950	1.86246835786853\\
951	1.93080628877364\\
952	1.90292890755801\\
953	1.85667907649698\\
954	1.74221015370484\\
955	1.6089250089362\\
956	1.6560649401284\\
957	1.64607088083511\\
958	1.71418491384621\\
959	1.68344246835954\\
960	1.60882677268093\\
961	1.67132819091707\\
962	1.72555516899171\\
963	1.77395723700183\\
964	1.79974412687181\\
965	1.75722522627825\\
966	1.66263727634158\\
967	1.46491959477324\\
968	1.31671221874543\\
969	1.31000900144211\\
970	1.31105534121874\\
971	1.34150114675608\\
972	1.32049163449593\\
973	1.35208248471259\\
974	1.30694859248855\\
975	1.3056796611363\\
976	1.28985572848652\\
977	1.28163973646872\\
978	1.23708156357472\\
979	1.23263819568989\\
980	1.27655062245345\\
981	1.3312759378392\\
982	1.38596304679113\\
983	1.37763506164926\\
984	1.39538360242385\\
985	1.50894801764543\\
986	1.61664389923648\\
987	1.67362610521443\\
988	1.72611573487201\\
989	1.72263354742266\\
990	1.63730140148844\\
991	1.49044782470049\\
992	1.42038857785016\\
993	1.44938679205965\\
994	1.45322839661172\\
995	1.49337573802593\\
996	1.47235105704674\\
997	1.51741748295604\\
998	1.49343186233545\\
999	1.47857161361539\\
1000	1.45848335473745\\
1001	1.42836115621628\\
1002	1.36301259836134\\
1003	1.34882501794498\\
1004	1.37386120893641\\
1005	1.4153343116423\\
1006	1.45354418012722\\
1007	1.43332551826903\\
1008	1.4427330067618\\
1009	1.57424484634047\\
1010	1.67224650123079\\
1011	1.74454373587716\\
1012	1.7673704406926\\
1013	1.75048302581888\\
1014	1.67351969563248\\
1015	1.49510195567119\\
1016	1.33013726410298\\
1017	1.39709039000402\\
1018	1.45587032411577\\
1019	1.55232287633046\\
1020	1.5776603903491\\
1021	1.58374480790873\\
1022	1.54456317195354\\
1023	1.53128507799663\\
1024	1.50049415357034\\
1025	1.46935885884903\\
1026	1.39705523322851\\
1027	1.3860071014324\\
1028	1.44062515351358\\
1029	1.48361467760865\\
1030	1.53571763920509\\
1031	1.52097841994156\\
1032	1.55695828662321\\
1033	1.68568835994277\\
1034	1.78707439912066\\
1035	1.84713248447074\\
1036	1.8790455424182\\
1037	1.85105993934949\\
1038	1.73016525699352\\
1039	1.54066388131943\\
1040	1.46423862487333\\
1041	1.44409807798525\\
1042	1.37072846645109\\
1043	1.35937500557471\\
1044	1.30765262988547\\
1045	1.31455165436822\\
1046	1.30069896496736\\
1047	1.3132226273768\\
1048	1.31433358389226\\
1049	1.30290503098499\\
1050	1.28851264875118\\
1051	1.29992203540171\\
1052	1.31439882460924\\
1053	1.38449667324796\\
1054	1.43353217012013\\
1055	1.41875802244783\\
1056	1.44082788992884\\
1057	1.51876833470765\\
1058	1.59651308186906\\
1059	1.66506821178456\\
1060	1.67614379268271\\
1061	1.65868065520219\\
1062	1.57788380915965\\
1063	1.3999346145719\\
1064	1.36923496882436\\
1065	1.4094496309803\\
1066	1.42822400878646\\
1067	1.45115474338077\\
1068	1.44200252952516\\
1069	1.43340494507921\\
1070	1.40251185098932\\
1071	1.39892990072047\\
1072	1.40960483957274\\
1073	1.41448926120727\\
1074	1.37757632871137\\
1075	1.43444644083881\\
1076	1.44988064418772\\
1077	1.51389512199336\\
1078	1.61649070187475\\
1079	1.57783043164336\\
1080	1.57948373189773\\
1081	1.73418352616715\\
1082	1.84317561747041\\
1083	1.94674236292082\\
1084	2.00828716588167\\
1085	2.01860126077626\\
1086	1.99586568962558\\
1087	1.91253637441223\\
1088	1.84217069234444\\
1089	1.81418068537273\\
1090	1.81752531641432\\
1091	1.71914294164772\\
1092	1.74266396939076\\
1093	1.80468110878044\\
1094	1.79802574778965\\
1095	1.83398481761457\\
1096	1.80525511400656\\
1097	1.77457408845398\\
1098	1.7097823020377\\
1099	1.66346135515253\\
1100	1.70830133148827\\
1101	1.77548399670531\\
1102	1.82616298123159\\
1103	1.74282534486713\\
1104	1.69227453217884\\
1105	1.78154237439685\\
1106	1.87606952171848\\
1107	1.94776430679222\\
1108	2.00055212707833\\
1109	2.0317298550167\\
1110	2.02677108784173\\
1111	1.9612154012034\\
1112	1.93058484874829\\
1113	1.87299832091689\\
1114	1.84890387234109\\
1115	1.84378229039902\\
1116	1.82978613907045\\
1117	1.85617918758327\\
1118	1.89172542611858\\
1119	1.94576232372104\\
1120	1.891292049772\\
1121	1.83464558254731\\
1122	1.7361635451403\\
1123	1.59472095999948\\
1124	1.53626895064466\\
1125	1.53266254771631\\
1126	1.56936328762474\\
1127	1.53237270227949\\
1128	1.52780875711811\\
1129	1.62914697623003\\
1130	1.70976532785867\\
1131	1.77444675760729\\
1132	1.80208436312904\\
1133	1.80256284269429\\
1134	1.69381241785247\\
1135	1.48624529764477\\
1136	1.35558423369473\\
1137	1.37382352078158\\
1138	1.41603960956189\\
1139	1.43085764994221\\
1140	1.4157662484423\\
1141	1.46799235969682\\
1142	1.47586247120746\\
1143	1.47914936054792\\
1144	1.46064706104408\\
1145	1.49755421461314\\
1146	1.43841391533394\\
1147	1.33490754047782\\
1148	1.34741107276169\\
1149	1.36916578503131\\
1150	1.43938449404708\\
1151	1.44415666785907\\
1152	1.46413645182811\\
1153	1.56676835124536\\
1154	1.68134294892787\\
1155	1.72749169674902\\
1156	1.77113549880987\\
1157	1.74370605836107\\
1158	1.6622781808038\\
1159	1.46883382597432\\
1160	1.40505027198735\\
1161	1.42505117805576\\
1162	1.4031398690306\\
1163	1.37754224128762\\
1164	1.34516865499778\\
1165	1.35212186955506\\
1166	1.33989112400067\\
1167	1.34192471492456\\
1168	1.3381780203424\\
1169	1.33352972431269\\
1170	1.29173057333978\\
1171	1.28172427725849\\
1172	1.32270899815576\\
1173	1.35658770252284\\
1174	1.44256060065255\\
1175	1.41899999129688\\
1176	1.44455728308238\\
1177	1.54716654607358\\
1178	1.63895035997102\\
1179	1.69528826144815\\
1180	1.71753325155394\\
1181	1.70604927609225\\
1182	1.61311983379675\\
1183	1.43011351138937\\
1184	1.34096832613863\\
1185	1.36494041505172\\
1186	1.35692566990408\\
1187	1.32413428124812\\
1188	1.29832548065922\\
1189	1.34451838854499\\
1190	1.33642490005658\\
1191	1.34528306547003\\
1192	1.33719362356672\\
1193	1.33143253026307\\
1194	1.30717042483666\\
1195	1.28699753265233\\
1196	1.30071540197641\\
1197	1.31977526311931\\
1198	1.38211224027533\\
1199	1.37213276154943\\
1200	1.40544091565439\\
1201	1.54600757081468\\
1202	1.65571047132507\\
1203	1.72908688272841\\
1204	1.81375617109401\\
1205	1.82549156106991\\
1206	1.72296749314694\\
1207	1.54964971186277\\
1208	1.50512413012915\\
1209	1.52318854159098\\
1210	1.51934242244397\\
1211	1.4904264290003\\
1212	1.42015123159394\\
1213	1.43765546374575\\
1214	1.42709584723878\\
1215	1.43125046260825\\
1216	1.4294696397711\\
1217	1.41446075282807\\
1218	1.38750100881769\\
1219	1.35966737410752\\
1220	1.39556448354121\\
1221	1.43928024060472\\
1222	1.49117563825254\\
1223	1.46093066891589\\
1224	1.47431628851414\\
1225	1.5702892436199\\
1226	1.65342391651488\\
1227	1.73958356940577\\
1228	1.77581647152714\\
1229	1.75252308001367\\
1230	1.65004500457537\\
1231	1.46956108396099\\
1232	1.41541568032266\\
1233	1.4404840930802\\
1234	1.44425556851776\\
1235	1.43109241740852\\
1236	1.43335937218678\\
1237	1.50783226912512\\
1238	1.51647939277927\\
1239	1.53089432825976\\
1240	1.52569688970089\\
1241	1.50764107425346\\
1242	1.46310086532589\\
1243	1.37171466336432\\
1244	1.41565065859734\\
1245	1.48133044168859\\
1246	1.56079048037675\\
1247	1.51530596275611\\
1248	1.52590567359812\\
1249	1.65830738857164\\
1250	1.77284659278082\\
1251	1.84627269789947\\
1252	1.92916958691043\\
1253	1.9514692096126\\
1254	1.91936108516281\\
1255	1.85311838917957\\
1256	1.78960761691917\\
1257	1.6984197898694\\
1258	1.60234131220283\\
1259	1.56517960449799\\
1260	1.5601282652347\\
1261	1.59418739433646\\
1262	1.66455205067402\\
1263	1.69594281764828\\
1264	1.68524320143251\\
1265	1.6637412476646\\
1266	1.59013997143677\\
1267	1.53330745590325\\
1268	1.5305837609147\\
1269	1.62017425838932\\
1270	1.69456206914855\\
1271	1.67524295653315\\
1272	1.61997175275389\\
1273	1.7128122291888\\
1274	1.83892109633137\\
1275	1.92821007330604\\
1276	1.99709661667188\\
1277	2.02470207018891\\
1278	2.0167708069749\\
1279	1.99346655253803\\
1280	2.0020826797628\\
1281	2.13494179147894\\
1282	2.08165802325483\\
1283	2.02736485298379\\
1284	2.05772736749435\\
1285	2.11739969432957\\
1286	2.08410972896309\\
1287	2.12454525944562\\
1288	2.06328396489125\\
1289	1.96729191393079\\
1290	1.83334565737404\\
1291	1.70053965614033\\
1292	1.65554380700852\\
1293	1.6306205619437\\
1294	1.6848442640391\\
1295	1.6936906605813\\
1296	1.68703692432894\\
1297	1.81779042153918\\
1298	1.91734803120658\\
1299	1.99753635343761\\
1300	2.04069685995945\\
1301	2.0041323861599\\
1302	1.86057238095737\\
1303	1.62660400581672\\
1304	1.53441390584699\\
1305	1.55285993887301\\
1306	1.5771227598962\\
1307	1.60086900917826\\
1308	1.65509076792046\\
1309	1.82705438066427\\
1310	1.69636652960844\\
1311	1.68122719697985\\
1312	1.61561404023547\\
1313	1.56220533760339\\
1314	1.48000616457412\\
1315	1.43375860464433\\
1316	1.42809870762589\\
1317	1.46380640575576\\
1318	1.54736593938664\\
1319	1.53630099156625\\
1320	1.5772069827285\\
1321	1.72666829983797\\
1322	1.83308546624763\\
1323	1.92860021277753\\
1324	1.96987612553902\\
1325	1.94407367345058\\
1326	1.82148681139616\\
1327	1.58634071762336\\
1328	1.4516716026449\\
1329	1.48433081978454\\
1330	1.52435060148051\\
1331	1.56482778914137\\
1332	1.55894405155434\\
1333	1.55291902719336\\
1334	1.52759096941009\\
1335	1.52278695974096\\
1336	1.49419614550913\\
1337	1.47481670109982\\
1338	1.42302706677315\\
1339	1.3852507082783\\
1340	1.39280115762667\\
1341	1.47150274418752\\
1342	1.49513474110871\\
1343	1.50841716221653\\
1344	1.525557710434\\
1345	1.59360655822387\\
1346	1.68621000822805\\
1347	1.7324332833404\\
1348	1.75403959686966\\
1349	1.72959305903211\\
1350	1.64206705515076\\
1351	1.46058117212\\
1352	1.40086565834898\\
1353	1.40356664142018\\
1354	1.37432197351209\\
1355	1.39485641607943\\
1356	1.4127757958776\\
1357	1.45468730151603\\
1358	1.45838102350342\\
1359	1.46411677847241\\
1360	1.44145826443804\\
1361	1.46346584666619\\
1362	1.43813661068\\
1363	1.34657057071765\\
1364	1.33715903934482\\
1365	1.40124385755666\\
1366	1.44231491980583\\
1367	1.42938421110351\\
1368	1.45351774799229\\
1369	1.58138740191801\\
1370	1.67672615186208\\
1371	1.75003893501209\\
1372	1.7999463368322\\
1373	1.78177422217253\\
1374	1.7168898775182\\
1375	1.53226876994737\\
1376	1.46047049375954\\
1377	1.55016230516151\\
1378	1.53381096483716\\
1379	1.51570071552217\\
1380	1.49514090130997\\
1381	1.46458343098298\\
1382	1.43410343242429\\
1383	1.4254082577596\\
1384	1.42084665961113\\
1385	1.41223462718777\\
1386	1.38083500718421\\
1387	1.3524611704329\\
1388	1.34219379276061\\
1389	1.37150598513743\\
1390	1.39924721265692\\
1391	1.38068118253955\\
1392	1.40196038423815\\
1393	1.53443022052717\\
1394	1.62900197407407\\
1395	1.68789366093929\\
1396	1.72018749582986\\
1397	1.69598503345778\\
1398	1.58750606444366\\
1399	1.40888645852836\\
1400	1.38150107632478\\
1401	1.40453323949878\\
1402	1.37938233679319\\
1403	1.34109685051927\\
1404	1.31970945985732\\
1405	1.30821381759089\\
1406	1.30131621996911\\
1407	1.31146954924539\\
1408	1.31744114181831\\
1409	1.33283286992557\\
1410	1.32136484552733\\
1411	1.32673567568304\\
1412	1.33547790600582\\
1413	1.35427568205696\\
1414	1.41551005276263\\
1415	1.36807806449716\\
1416	1.37530124004129\\
1417	1.50168182683362\\
1418	1.59919907420464\\
1419	1.67558168973102\\
1420	1.71702768699532\\
1421	1.72778884535952\\
1422	1.72628401587013\\
1423	1.66516483714779\\
1424	1.62587546375015\\
1425	1.61584059243753\\
1426	1.58094368368277\\
1427	1.54991786658728\\
1428	1.53441076548761\\
1429	1.48038882631661\\
1430	1.5178678587332\\
1431	1.55305750163302\\
1432	1.54358904734156\\
1433	1.55809171406616\\
1434	1.54138039413382\\
1435	1.46920816632231\\
1436	1.46867907159527\\
1437	1.54470153094711\\
1438	1.56091513706473\\
1439	1.53123459730367\\
1440	1.5202089901251\\
1441	1.59507679777851\\
1442	1.69391979925913\\
1443	1.77004241593774\\
1444	1.82562484134006\\
1445	1.85078851097035\\
1446	1.85846245218051\\
1447	1.83123107282825\\
1448	1.85596066594132\\
1449	1.94614240362187\\
1450	1.95439267279642\\
1451	2.0162418271528\\
1452	2.0495945739848\\
1453	2.11247447490105\\
1454	2.20363672188879\\
1455	2.24523750606185\\
1456	2.15487445730973\\
1457	2.02541537529442\\
1458	1.87220184323161\\
1459	1.8099706619807\\
1460	1.75205688404772\\
1461	1.68059939721125\\
1462	1.75945674734015\\
1463	1.74431498905925\\
1464	1.74905871448169\\
1465	1.86794069700398\\
1466	1.99117587025208\\
1467	2.07166175494405\\
1468	2.10029676957498\\
1469	2.0574036182004\\
1470	1.87453530116228\\
1471	1.63672415725931\\
1472	1.58548761528739\\
1473	1.54535297946052\\
1474	1.54907060445952\\
1475	1.55219104674505\\
1476	1.55324063961403\\
1477	1.60710110950456\\
1478	1.58358759005959\\
1479	1.55924947305533\\
1480	1.54052679571469\\
1481	1.516130193406\\
1482	1.44536937598167\\
1483	1.43815502854824\\
1484	1.42050435975442\\
1485	1.46217656232679\\
1486	1.50563157807399\\
1487	1.45940346329604\\
1488	1.45625588716567\\
1489	1.51921781955863\\
1490	1.59530272691583\\
1491	1.65566292720738\\
1492	1.66716210392124\\
1493	1.63366374585093\\
1494	1.54931815976876\\
1495	1.41246795665539\\
1496	1.3608661082163\\
1497	1.37959752525329\\
1498	1.36994551408288\\
1499	1.37178802752807\\
1500	1.37133145141073\\
1501	1.3932283232757\\
1502	1.39840905176265\\
1503	1.38499570788644\\
1504	1.36463223234458\\
1505	1.35778212231777\\
1506	1.30438025078657\\
1507	1.30175930997948\\
1508	1.28240855959225\\
1509	1.33135790458699\\
1510	1.35951327912801\\
1511	1.33088752828115\\
1512	1.342391139714\\
1513	1.43289680289457\\
1514	1.50498245090555\\
1515	1.56782241172476\\
1516	1.60466047494602\\
1517	1.58679205182118\\
1518	1.51876461029249\\
1519	1.37924235170322\\
1520	1.33039402821076\\
1521	1.3467614928955\\
1522	1.3780063479666\\
1523	1.38433063631076\\
1524	1.40269719637695\\
1525	1.44315335939091\\
1526	1.45084889138621\\
1527	1.45034427293121\\
1528	1.41940821161614\\
1529	1.40702302933709\\
1530	1.35243138606194\\
1531	1.3750792919581\\
1532	1.31808484248431\\
1533	1.3071664277574\\
1534	1.35908917568277\\
1535	1.33758717364877\\
1536	1.36848470516466\\
1537	1.44748587889809\\
1538	1.53734398499614\\
1539	1.64070652290812\\
1540	1.66447559468948\\
1541	1.65754605684342\\
1542	1.56999054904156\\
1543	1.43844868792774\\
1544	1.46128237330689\\
1545	1.47183338338961\\
1546	1.52462883813733\\
1547	1.48498313811263\\
1548	1.52963159133752\\
1549	1.58118044932827\\
1550	1.60220806681203\\
1551	1.61560514922772\\
1552	1.56428594459024\\
1553	1.52454022558502\\
1554	1.4306329873745\\
1555	1.41750683392254\\
1556	1.37367168967247\\
1557	1.36594801328238\\
1558	1.45631697971245\\
1559	1.44573644002731\\
1560	1.47489564176708\\
1561	1.62429619946677\\
1562	1.75563355881954\\
1563	1.83242635719538\\
1564	1.85746755308446\\
1565	1.81079859515086\\
1566	1.72942121047204\\
1567	1.5514024310081\\
1568	1.57275683891228\\
1569	1.57451989411418\\
1570	1.57044797064252\\
1571	1.54331330397133\\
1572	1.56869765284999\\
1573	1.61762568556321\\
1574	1.61594345964667\\
1575	1.58556763611253\\
1576	1.55275106936093\\
1577	1.52228056295366\\
1578	1.45817308710399\\
1579	1.45906048934699\\
1580	1.42562645991018\\
1581	1.40727641957975\\
1582	1.48243487965044\\
1583	1.45595926221496\\
1584	1.45482759021898\\
1585	1.55796116279909\\
1586	1.67982594510891\\
1587	1.79965601897131\\
1588	1.85857412869265\\
1589	1.87814112660555\\
1590	1.85735614877135\\
1591	1.8137446257556\\
1592	1.81673758816237\\
1593	1.76946495946296\\
1594	1.77145999668182\\
1595	1.84168713422355\\
1596	1.90583768608713\\
1597	1.92096984840545\\
1598	1.98665755001086\\
1599	2.00126782872708\\
1600	1.93035282183811\\
1601	1.83997915583286\\
1602	1.71090617574316\\
1603	1.59501272086269\\
1604	1.63727209576367\\
1605	1.73746729081917\\
1606	1.80149809904322\\
1607	1.73936625913349\\
1608	1.75815772123314\\
1609	1.889125721908\\
1610	1.99648928035087\\
1611	2.09178559434901\\
1612	2.10224433036551\\
1613	2.12153766438367\\
1614	2.14307574426379\\
1615	2.03818677033229\\
1616	2.06879852842859\\
1617	2.07258354101384\\
1618	2.05514477123435\\
1619	2.10928573639396\\
1620	2.11430343264358\\
1621	2.12668447545066\\
1622	2.2223374708055\\
1623	2.27960964757037\\
1624	2.17608798243745\\
1625	2.09466618423347\\
1626	1.92733995660852\\
1627	1.76677144041151\\
1628	1.66081583639286\\
1629	1.68424303642453\\
1630	1.72863444219807\\
1631	1.72047354995512\\
1632	1.69798580893511\\
1633	1.7573074431079\\
1634	1.83454350168685\\
1635	1.8759460895251\\
1636	1.89081150032069\\
1637	1.84736403073938\\
1638	1.69439280796366\\
1639	1.46090413248887\\
1640	1.44259810938236\\
1641	1.46252539770256\\
1642	1.49768144745225\\
1643	1.49464022591201\\
1644	1.50765745379114\\
1645	1.57525491826685\\
1646	1.59259902917661\\
1647	1.57555918127084\\
1648	1.54754161563465\\
1649	1.50573011635047\\
1650	1.41748685900654\\
1651	1.35382705508541\\
1652	1.38603411627146\\
1653	1.42827621005414\\
1654	1.47936141172255\\
1655	1.47484564850755\\
1656	1.50490501093331\\
1657	1.67978292973052\\
1658	1.7981577212744\\
1659	1.84470118513926\\
1660	1.84672085351022\\
1661	1.82153524846446\\
1662	1.70754224882545\\
1663	1.51158281735357\\
1664	1.50251572595272\\
1665	1.4722029846499\\
1666	1.4606812537432\\
1667	1.38784862406108\\
1668	1.41389564012361\\
1669	1.49860619408095\\
1670	1.5636632955075\\
1671	1.61789955546484\\
1672	1.58629574662685\\
1673	1.54833784102172\\
1674	1.46721619480739\\
1675	1.4008898023453\\
1676	1.4107517345248\\
1677	1.41050519916173\\
1678	1.49491242404481\\
1679	1.49032547220823\\
1680	1.49709869112513\\
1681	1.62459376572139\\
1682	1.71348889133602\\
1683	1.74860267977442\\
1684	1.76739178464789\\
1685	1.72817095292387\\
1686	1.61784194857867\\
1687	1.43723725282999\\
1688	1.44762286715377\\
1689	1.4742994115643\\
1690	1.4193922585723\\
1691	1.49056537934615\\
1692	1.53343044564681\\
1693	1.57964579307061\\
1694	1.58458509917406\\
1695	1.59603181550982\\
1696	1.55532110478149\\
1697	1.50231737394992\\
1698	1.41871205569991\\
1699	1.39938652586993\\
1700	1.34189301702986\\
1701	1.38007774878638\\
1702	1.40720814113439\\
1703	1.39591370423512\\
1704	1.45267068634472\\
1705	1.55608494773627\\
1706	1.63550364731917\\
1707	1.70027730810307\\
1708	1.71612924848952\\
1709	1.67414420269927\\
1710	1.58879508325501\\
1711	1.39725264495193\\
1712	1.34934101068217\\
1713	1.36806039323772\\
1714	1.41053448082051\\
1715	1.44337589280634\\
1716	1.48571185434296\\
1717	1.53365149349081\\
1718	1.54148051801797\\
1719	1.54027966659052\\
1720	1.53708589607791\\
1721	1.49447594209228\\
1722	1.43938879701447\\
1723	1.42703874132109\\
1724	1.35914291202739\\
1725	1.38285705858426\\
1726	1.42253724482949\\
1727	1.4131204135611\\
1728	1.4560139904295\\
1729	1.55890674253538\\
1730	1.63968905791443\\
1731	1.69579117670868\\
1732	1.71194400437713\\
1733	1.69625089320463\\
1734	1.60662952037256\\
1735	1.43447022362948\\
1736	1.4289854354799\\
1737	1.4488415761731\\
1738	1.49761255845512\\
1739	1.45615943305603\\
1740	1.48639204715042\\
1741	1.54486715584096\\
1742	1.58661335982603\\
1743	1.61318336238898\\
1744	1.58403227864481\\
1745	1.52612286033176\\
1746	1.45793657418767\\
1747	1.41480591744411\\
1748	1.39575256811726\\
1749	1.43059351379548\\
1750	1.41988541097448\\
1751	1.4161615610348\\
1752	1.45388974709468\\
1753	1.63695102681476\\
1754	1.73040313815786\\
1755	1.80297173784277\\
1756	1.85460222516377\\
1757	1.86971670899039\\
1758	1.87577281153016\\
1759	1.79371222171891\\
1760	1.75440137238114\\
1761	1.65395889676411\\
1762	1.56832947627\\
1763	1.56956781480838\\
1764	1.56037696387873\\
1765	1.57407886400857\\
1766	1.62979009579208\\
1767	1.69591165327552\\
1768	1.70376595381493\\
1769	1.67967940519809\\
1770	1.60789575288788\\
1771	1.54692549774427\\
1772	1.50221644691561\\
1773	1.54286511280289\\
1774	1.6119384417947\\
1775	1.62466493238114\\
1776	1.68238935792879\\
1777	1.7716261324079\\
1778	1.89617713467987\\
1779	2.00149403133383\\
1780	1.99146210053675\\
1781	2.00035429406344\\
1782	2.0151633650197\\
1783	1.94455241046621\\
1784	1.9756084817401\\
1785	1.89417172990412\\
1786	1.81842054568902\\
1787	1.77107591621315\\
1788	1.77099075145364\\
1789	1.79406632340052\\
1790	1.88447332974286\\
1791	1.98023216388053\\
1792	1.9749635871867\\
1793	1.93109558246915\\
1794	1.77785467034383\\
1795	1.65937214111547\\
1796	1.51796872338027\\
1797	1.51907720118241\\
1798	1.6061512061118\\
1799	1.61089500340504\\
1800	1.61359189767765\\
1801	1.74477292406613\\
1802	1.81596505981108\\
1803	1.87428337693944\\
1804	1.88650880602629\\
1805	1.82925161869589\\
1806	1.69887468305512\\
1807	1.46081176752934\\
1808	1.40905385442238\\
1809	1.41698212611693\\
1810	1.38726289260196\\
1811	1.32953623371697\\
1812	1.31625068922819\\
1813	1.37406028610578\\
1814	1.38674306413162\\
1815	1.38915323755735\\
1816	1.37112887059202\\
1817	1.34990954387465\\
1818	1.28264094207133\\
1819	1.2948907869418\\
1820	1.26619633130976\\
1821	1.31417235963518\\
1822	1.35149569338848\\
1823	1.36351366194984\\
1824	1.40257153346212\\
1825	1.513730380931\\
1826	1.61184346581024\\
1827	1.69042144381051\\
1828	1.70696220373394\\
1829	1.68995510552026\\
1830	1.60962686805314\\
1831	1.41206296728\\
1832	1.33076296333963\\
1833	1.35415638685079\\
1834	1.33511776681012\\
1835	1.36880960651899\\
1836	1.33416931831329\\
1837	1.35087188155977\\
1838	1.35529756790403\\
1839	1.38873404053547\\
1840	1.40624836752587\\
1841	1.40640025527753\\
1842	1.37249567976758\\
1843	1.36457179369257\\
1844	1.38659718123026\\
1845	1.43022767252236\\
1846	1.48702558171287\\
1847	1.47069785490974\\
1848	1.51691200035896\\
1849	1.65776043623007\\
1850	1.77343192701197\\
1851	1.82962263700183\\
1852	1.81513112737509\\
1853	1.76217233279003\\
1854	1.63928394254754\\
1855	1.44541225264629\\
1856	1.34912929828007\\
1857	1.45550972294685\\
1858	1.46969093720202\\
1859	1.38295157817602\\
1860	1.39391058278457\\
1861	1.43253194617518\\
1862	1.46235390781029\\
1863	1.48778842532059\\
1864	1.47081604193697\\
1865	1.44189731502055\\
1866	1.35027618622874\\
1867	1.37984530549828\\
1868	1.31642659283041\\
1869	1.33257362343575\\
1870	1.33077043969421\\
1871	1.40860751743824\\
1872	1.45810393898916\\
1873	1.55666609183253\\
1874	1.67740201521435\\
1875	1.76195674399085\\
1876	1.76303645233266\\
1877	1.74590758165143\\
1878	1.71498270705072\\
1879	1.5642467433521\\
1880	1.48086815545516\\
1881	1.63186999945816\\
1882	1.64552008873262\\
1883	1.66812570587663\\
1884	1.65545381804819\\
1885	1.73524766941985\\
1886	1.85360323686218\\
1887	1.87011244689236\\
1888	1.86027920708547\\
1889	1.70774128933448\\
1890	1.60462052134116\\
1891	1.59016896340545\\
1892	1.52660433873764\\
1893	1.47526706351551\\
1894	1.52803785843894\\
1895	1.54016786165953\\
1896	1.58028571960507\\
1897	1.73722323833139\\
1898	1.85400486933866\\
1899	1.914012621466\\
1900	1.89844945688985\\
1901	1.83326394187926\\
1902	1.70799935722249\\
1903	1.49678027829448\\
1904	1.3866625615625\\
1905	1.37372337893363\\
1906	1.35298081420226\\
1907	1.33920660534979\\
1908	1.29387807222542\\
1909	1.32512290018497\\
1910	1.36154073707651\\
1911	1.41851612703381\\
1912	1.44218756656511\\
1913	1.44384577123015\\
1914	1.42282456896212\\
1915	1.47825970224873\\
1916	1.42451614475714\\
1917	1.455640042439\\
1918	1.45528664248027\\
1919	1.44711405815108\\
1920	1.47201243838848\\
1921	1.62427523638119\\
1922	1.77139730908032\\
1923	1.87642804588696\\
1924	1.93309496789292\\
1925	1.96368739176182\\
1926	1.97087770099373\\
1927	1.94297786447001\\
1928	1.92995087174754\\
1929	1.87838890952762\\
1930	1.7792711013168\\
1931	1.70495593015082\\
1932	1.69857004074966\\
1933	1.68734060669256\\
1934	1.73386443571451\\
1935	1.77158333940517\\
1936	1.78734876680459\\
1937	1.74923622113497\\
1938	1.70176367764267\\
1939	1.73137680924973\\
1940	1.61287202835278\\
1941	1.67513699926892\\
1942	1.62295451078069\\
1943	1.62451691787511\\
1944	1.61397209209273\\
1945	1.68504432476538\\
1946	1.79372126494185\\
1947	1.87793708267298\\
1948	1.92803420941212\\
1949	1.93971387575075\\
1950	1.88929353978743\\
1951	1.86741046636232\\
1952	1.88782585664846\\
1953	1.86843253941805\\
1954	1.78787893189809\\
1955	1.76739587526286\\
1956	1.72022086450511\\
1957	1.68343823817002\\
1958	1.77387747410252\\
1959	1.84814932559063\\
1960	1.81449959543903\\
1961	1.73988279226465\\
1962	1.63993718597877\\
1963	1.54325771283558\\
1964	1.54941545471038\\
1965	1.54685819726808\\
1966	1.59180220585576\\
1967	1.61389201025443\\
1968	1.53186214711856\\
1969	1.6562602399322\\
1970	1.73973087426389\\
1971	1.77469881772308\\
1972	1.80224551631018\\
1973	1.74198639501277\\
1974	1.61683462525867\\
1975	1.42518622353491\\
1976	1.33854902225415\\
1977	1.41771661631\\
1978	1.43249233844722\\
1979	1.43883217883858\\
1980	1.4197472322135\\
1981	1.44705159016528\\
1982	1.44803986314577\\
1983	1.45744461705446\\
1984	1.41842554060967\\
1985	1.39739711782646\\
1986	1.32792677382235\\
1987	1.3465854806629\\
1988	1.28873378215774\\
1989	1.31291499090452\\
1990	1.36914038505421\\
1991	1.33957901983481\\
1992	1.38285870321052\\
1993	1.50137579804359\\
1994	1.58880382475931\\
1995	1.64586414667484\\
1996	1.66845089465294\\
1997	1.64475240938137\\
1998	1.54337933980264\\
1999	1.39240325192606\\
2000	1.41530942843544\\
2001	1.36886325024338\\
2002	1.3991903815184\\
2003	1.4461309881888\\
2004	1.46731355563547\\
2005	1.49913292312822\\
2006	1.49346677469573\\
2007	1.4818674447843\\
2008	1.44227426169692\\
2009	1.41599871637536\\
2010	1.37803380494369\\
2011	1.33580948086746\\
2012	1.32973233531299\\
2013	1.34299186609224\\
2014	1.40885021948812\\
2015	1.43363026919636\\
2016	1.44849030411333\\
2017	1.53298500377149\\
2018	1.6213728883071\\
2019	1.6829265434353\\
2020	1.71262939401528\\
2021	1.6750984709377\\
2022	1.55257374226178\\
2023	1.38995897042979\\
2024	1.37186886176007\\
2025	1.39993810395868\\
2026	1.40729234965379\\
2027	1.39500260148793\\
2028	1.39210137607535\\
2029	1.41152915559117\\
2030	1.29528327554444\\
2031	1.28415362028209\\
2032	1.25524114061316\\
2033	1.23094160661644\\
2034	1.18621696033999\\
2035	1.18240410194899\\
2036	1.08190854109592\\
2037	1.0959119559347\\
2038	1.1110907052666\\
2039	1.10803684505969\\
2040	1.11870695774836\\
2041	1.20388727638715\\
2042	1.28345400580662\\
2043	1.33579816205302\\
2044	1.35753965449947\\
2045	1.34846482013577\\
2046	1.27394573378156\\
2047	1.14935677581808\\
2048	1.11481004433171\\
2049	1.11957739316284\\
2050	1.12774296246977\\
2051	1.12200784757781\\
2052	1.14358709638932\\
2053	1.20029010197787\\
2054	1.192008986553\\
2055	1.18905711322673\\
2056	1.15910108325716\\
2057	1.13247250461958\\
2058	1.06861594711045\\
2059	1.02435859472313\\
2060	1.04605749810883\\
2061	1.06953696383025\\
2062	1.11499742408368\\
2063	1.10356003400873\\
2064	1.12421519946254\\
2065	1.20550297220782\\
2066	1.27156801943399\\
2067	1.32379843934259\\
2068	1.34476083523002\\
2069	1.33067384790344\\
2070	1.25564244415345\\
2071	1.13540613306394\\
2072	1.12897895652548\\
2073	1.14262546619346\\
2074	1.14315007029485\\
2075	1.18076292552699\\
2076	1.16168889149222\\
2077	1.17463786893323\\
2078	1.16826641675761\\
2079	1.19778965302969\\
2080	1.19973839331625\\
2081	1.19193636341114\\
2082	1.15125961691329\\
2083	1.11459134180412\\
2084	1.14897550548535\\
2085	1.13939546387112\\
2086	1.16865299055247\\
2087	1.16855593726436\\
2088	1.20441153324046\\
2089	1.31019504851979\\
2090	1.43017588308937\\
2091	1.49651417678919\\
2092	1.52075077256715\\
2093	1.515327431826\\
2094	1.49482567213735\\
2095	1.46602043876633\\
2096	1.43641650361988\\
2097	1.40867863395091\\
2098	1.40984645130894\\
2099	1.45200760216002\\
2100	1.49058486185025\\
2101	1.52617497497367\\
2102	1.57058733074416\\
2103	1.60690511993328\\
2104	1.5511024605411\\
2105	1.49981354137923\\
2106	1.40400515963621\\
2107	1.33345929458362\\
2108	1.35794117531352\\
2109	1.28065221816755\\
2110	1.35855488120386\\
2111	1.36621796861262\\
2112	1.37540271279591\\
2113	1.48131222245934\\
2114	1.53520784111136\\
2115	1.58602173467951\\
2116	1.61765212465098\\
2117	1.5973120276463\\
2118	1.56232629722312\\
2119	1.55303591365258\\
2120	1.52180530071842\\
2121	1.49047009408056\\
2122	1.50646127778187\\
2123	1.53953291035341\\
2124	1.57175875396113\\
2125	1.66050574526957\\
2126	1.71470279432474\\
2127	1.70445121349626\\
2128	1.66397360886721\\
2129	1.5476028960032\\
2130	1.49356049276508\\
2131	1.38743658983168\\
2132	1.31340688851364\\
2133	1.36124882512449\\
2134	1.32843669503136\\
2135	1.40977084844161\\
2136	1.45729178838005\\
2137	1.49614580710222\\
2138	1.53040319507457\\
2139	1.53251453491788\\
2140	1.49887061218682\\
2141	1.3946010972877\\
2142	1.21411778591768\\
2143	1.116401845099\\
2144	1.15250902237845\\
2145	1.17246909136943\\
2146	1.17265142155019\\
2147	1.21818821656209\\
2148	1.29386290766152\\
2149	1.3053255586896\\
2150	1.30080585202713\\
2151	1.2917248429784\\
2152	1.26136541747359\\
2153	1.20162832028489\\
2154	1.15252295848605\\
2155	1.18468043273594\\
2156	1.0998735026785\\
2157	1.14642716719058\\
2158	1.16865406271068\\
2159	1.23024645139652\\
2160	1.25362231827604\\
2161	1.33211962576126\\
2162	1.34698380147525\\
2163	1.36215022747716\\
2164	1.33600244187693\\
2165	1.29916447238862\\
2166	1.15194948808992\\
2167	1.11001375524118\\
2168	1.14488673052728\\
2169	1.15313012647366\\
2170	1.16079152357812\\
2171	1.16449684309263\\
2172	1.21238486835621\\
2173	1.21192453892436\\
2174	1.21235838656915\\
2175	1.20110650988779\\
2176	1.17383139258003\\
2177	1.1299454653307\\
2178	1.10273476349998\\
2179	1.13038853649776\\
2180	1.1099338367366\\
2181	1.10920691776337\\
2182	1.11015718993717\\
2183	1.15144718553193\\
2184	1.25921382441375\\
2185	1.34552152785436\\
2186	1.41173597875105\\
2187	1.42964089933858\\
2188	1.42990686604993\\
2189	1.35415286007748\\
2190	1.18344299113468\\
2191	1.1310611122114\\
2192	1.14664927399403\\
2193	1.15261377398376\\
2194	1.15189778975754\\
2195	1.16844673333761\\
2196	1.1856195638371\\
2197	1.1941319571991\\
2198	1.20821120391057\\
2199	1.2215083603636\\
2200	1.20930524558915\\
2201	1.17409309155889\\
2202	1.14103215067002\\
2203	1.13545361352693\\
2204	1.13161443864014\\
2205	1.12579861085768\\
2206	1.14445226780974\\
2207	1.20138378661713\\
2208	1.30723863397228\\
2209	1.39188712462851\\
2210	1.45251254156875\\
2211	1.45986604251502\\
2212	1.47227118092665\\
2213	1.39934378319654\\
2214	1.23844028773193\\
2215	1.15869458417374\\
2216	1.16949786960425\\
2217	1.17979775915639\\
2218	1.19823130494182\\
2219	1.18753840368005\\
2220	1.2250886064538\\
2221	1.22382026913614\\
2222	1.22585497227867\\
2223	1.23222683499767\\
2224	1.20314067917292\\
2225	1.14961444781581\\
2226	1.10736311791139\\
2227	1.08797243137346\\
2228	1.10002363382955\\
2229	1.11953994464978\\
2230	1.14611233540154\\
2231	1.17381456324535\\
2232	1.2977478906328\\
2233	1.36016548490334\\
2234	1.42509786365864\\
2235	1.42572573009011\\
2236	1.40517902888599\\
2237	1.32597586857305\\
2238	1.18271299065312\\
2239	1.10802290768129\\
2240	1.08731028979321\\
2241	1.08401313491717\\
2242	1.08067338329244\\
2243	1.0680488126064\\
2244	1.1131930311788\\
2245	1.1393891337524\\
2246	1.15816461818544\\
2247	1.18205908237356\\
2248	1.17642841463357\\
2249	1.140908849017\\
2250	1.12550155873831\\
2251	1.15847280999027\\
2252	1.14592203534015\\
2253	1.1755330712329\\
2254	1.1464245299415\\
2255	1.16997837449482\\
2256	1.29697677213853\\
2257	1.37378687541612\\
2258	1.43806431148202\\
2259	1.46098595064787\\
2260	1.46717699397583\\
2261	1.43343890518131\\
2262	1.37333557445879\\
2263	1.34218151070145\\
2264	1.27716739116472\\
2265	1.27732974659422\\
2266	1.27764520201606\\
2267	1.27095159630006\\
2268	1.30034698258039\\
2269	1.36041604747814\\
2270	1.40725192248557\\
2271	1.41701527624907\\
2272	1.41124911975999\\
2273	1.37816287363181\\
2274	1.30485957530233\\
2275	1.32077692006871\\
2276	1.29343242075199\\
2277	1.25842041112445\\
2278	1.28668748016478\\
2279	1.30099890388597\\
2280	1.39556208281814\\
2281	1.5087290453171\\
2282	1.59114428559744\\
2283	1.6301729105654\\
2284	1.66097522562648\\
2285	1.64443779745848\\
2286	1.59584227386098\\
2287	1.61554922888718\\
2288	1.56978848943736\\
2289	1.54249362260821\\
2290	1.52615539719417\\
2291	1.48310846470841\\
2292	1.42417189743338\\
2293	1.47952096023\\
2294	1.50532579406946\\
2295	1.52022621244826\\
2296	1.51645721091484\\
2297	1.45170418964588\\
2298	1.37416633057681\\
2299	1.35027105963606\\
2300	1.33165496668819\\
2301	1.3283142329384\\
2302	1.3444170015292\\
2303	1.37678338153215\\
2304	1.46473610788984\\
2305	1.54421205665638\\
2306	1.59932854886974\\
2307	1.60793414190795\\
2308	1.57665388746556\\
2309	1.47174786592852\\
2310	1.29400885220314\\
2311	1.25523808863086\\
2312	1.18286109221035\\
2313	1.1681242605485\\
2314	1.15480082543752\\
2315	1.16004771549767\\
2316	1.211633796579\\
2317	1.20768841413582\\
2318	1.21984005454083\\
2319	1.21419095067177\\
2320	1.21761609855799\\
2321	1.19308111030553\\
2322	1.21677834888146\\
2323	1.22214340779357\\
2324	1.20742662800161\\
2325	1.18180087565348\\
2326	1.19656697335478\\
2327	1.26069390956358\\
2328	1.39497790571682\\
2329	1.50023165710484\\
2330	1.56544020403265\\
2331	1.56901095728355\\
2332	1.53259375725895\\
2333	1.42787275750208\\
2334	1.27413302687551\\
2335	1.23941383329599\\
2336	1.19917492384489\\
2337	1.21103994948329\\
2338	1.2255178326586\\
2339	1.20933794822475\\
2340	1.26732308358186\\
2341	1.29611964690236\\
2342	1.30383912423664\\
2343	1.28165410849572\\
2344	1.2533760913301\\
2345	1.2015410388857\\
2346	1.23596112182939\\
2347	1.20691228716618\\
2348	1.11549810896798\\
2349	1.11125423281417\\
2350	1.10636252310033\\
2351	1.14376279668373\\
2352	1.22712499281646\\
2353	1.29491079146687\\
2354	1.33190399029072\\
2355	1.34773059361839\\
2356	1.32938663110779\\
2357	1.24264483112728\\
2358	1.13293049534791\\
2359	1.11267699175638\\
2360	1.10057644635096\\
2361	1.09978611589633\\
2362	1.12602394011848\\
2363	1.13414669067561\\
2364	1.19785898931502\\
2365	1.20835748683161\\
2366	1.24089396469019\\
2367	1.23911736496541\\
2368	1.21392546699658\\
2369	1.1713382388073\\
2370	1.1328321329232\\
2371	1.13512934890657\\
2372	1.14831144470472\\
2373	1.11885250293989\\
2374	1.12834542423324\\
2375	1.13471019761957\\
2376	1.2139496538016\\
2377	1.27680322990487\\
2378	1.31740345964107\\
2379	1.33742271042549\\
2380	1.3220500545829\\
2381	1.25009340761784\\
2382	1.12937306303011\\
2383	1.08872555154845\\
2384	1.0859101409649\\
2385	1.09983895749211\\
2386	1.11292667325356\\
2387	1.10472141941404\\
2388	1.16518225274952\\
2389	1.16092851526386\\
2390	1.16820609957146\\
2391	1.16529179925058\\
2392	1.15591962091945\\
2393	1.11789864764514\\
2394	1.09138081307153\\
2395	1.12954848815279\\
2396	1.11476865433246\\
2397	1.10266350896588\\
2398	1.11299973377227\\
2399	1.13232597274871\\
2400	1.23596810017628\\
2401	1.30312620252703\\
2402	1.35557894683369\\
2403	1.35933114394581\\
2404	1.35076964330021\\
2405	1.2754276065423\\
2406	1.17084074956832\\
2407	1.12778048488893\\
2408	1.10816412634679\\
2409	1.11599554095574\\
2410	1.1227061693306\\
2411	1.13464506371662\\
2412	1.16594656474901\\
2413	1.19708970858241\\
2414	1.21722249427645\\
2415	1.20561389513338\\
2416	1.20463798048718\\
2417	1.18566024069001\\
2418	1.16225154205194\\
2419	1.17610162869819\\
2420	1.1683460842755\\
2421	1.15260895809329\\
2422	1.15072053200747\\
2423	1.19311220132861\\
2424	1.27377917554457\\
2425	1.32528548355039\\
2426	1.38071594944198\\
2427	1.39911935794532\\
2428	1.39425816247011\\
2429	1.38433294778029\\
2430	1.3538383965631\\
2431	1.33107747473093\\
2432	1.2774136360115\\
2433	1.24581401493381\\
2434	1.262437634681\\
2435	1.31408767459794\\
2436	1.37146152088835\\
2437	1.44419537863653\\
2438	1.47896079596764\\
2439	1.46626633746875\\
2440	1.42568565183467\\
2441	1.36225250767165\\
2442	1.30713130162413\\
2443	1.28644698761417\\
2444	1.32456780771198\\
2445	1.28844063274588\\
2446	1.29210304011601\\
2447	1.3168093075537\\
2448	1.54569738496007\\
2449	1.62736138347947\\
2450	1.7045923436738\\
2451	1.72674686779652\\
2452	1.7340870591381\\
2453	1.70518596964295\\
2454	1.65752649389264\\
2455	1.64511110629238\\
2456	1.57916352005726\\
2457	1.58175052099711\\
2458	1.60050300250743\\
2459	1.5820747930671\\
2460	1.61063580045866\\
2461	1.67252834397789\\
2462	1.77702014454312\\
2463	1.79703931306778\\
2464	1.76590888932577\\
2465	1.66667138274024\\
2466	1.5449015580938\\
2467	1.46074089378204\\
2468	1.43803710081372\\
2469	1.38838094894339\\
2470	1.40372923548479\\
2471	1.45584617975335\\
2472	1.61719600001167\\
2473	1.70775233333337\\
2474	1.75254506910591\\
2475	1.78607547524171\\
2476	1.76725290535933\\
2477	1.64183909110945\\
2478	1.45301256859489\\
2479	1.37636833656826\\
2480	1.33346978661795\\
2481	1.34392089353061\\
2482	1.3434472233046\\
2483	1.34745134594234\\
2484	1.37970314352013\\
2485	1.38316479514204\\
2486	1.41236118578886\\
2487	1.41388764191191\\
2488	1.41335883718781\\
2489	1.35896353754277\\
2490	1.31844429544859\\
2491	1.29305720559928\\
2492	1.30037472447523\\
2493	1.26697672779742\\
2494	1.26727038353837\\
2495	1.28824194297648\\
2496	1.38623373761712\\
2497	1.43934534523005\\
2498	1.48340133081076\\
2499	1.50217478062299\\
2500	1.46297310781972\\
2501	1.39116748016938\\
2502	1.26329772666588\\
2503	1.21895637108433\\
2504	1.1924497042498\\
2505	1.20370512168887\\
2506	1.23334766699142\\
2507	1.24999000590635\\
2508	1.28734932154567\\
2509	1.30151438558564\\
2510	1.31839038732055\\
2511	1.32686232028049\\
2512	1.332024832477\\
2513	1.30247769249356\\
2514	1.25415712774159\\
2515	1.23369692757872\\
2516	1.23312244960995\\
2517	1.22336680697958\\
2518	1.20166523346958\\
2519	1.22887194156125\\
2520	1.31450536298898\\
2521	1.38009457525389\\
2522	1.4359783829818\\
2523	1.44451415574379\\
2524	1.41915123222873\\
2525	1.35278771784686\\
2526	1.23761028834652\\
2527	1.21705014516083\\
2528	1.2466677047671\\
2529	1.26551694863836\\
2530	1.27246142292551\\
2531	1.31104831961575\\
2532	1.33884714159865\\
2533	1.39074561949833\\
2534	1.38985965783256\\
2535	1.37864150280566\\
2536	1.35118644883546\\
2537	1.30043760379654\\
2538	1.23791873695638\\
2539	1.22335178988923\\
2540	1.2150831203055\\
2541	1.22740174990761\\
2542	1.2041427860058\\
2543	1.22626414502963\\
2544	1.3257494308657\\
2545	1.41532957472805\\
2546	1.46862943375303\\
2547	1.49311595766811\\
2548	1.50071439546335\\
2549	1.44818591056557\\
2550	1.32255930881883\\
2551	1.31509650543724\\
2552	1.28811764301072\\
2553	1.29578260450617\\
2554	1.31393769432316\\
2555	1.30654375495691\\
2556	1.37693620950614\\
2557	1.40562418581113\\
2558	1.43701424903589\\
2559	1.45008136909763\\
2560	1.41032666258952\\
2561	1.35243965499214\\
2562	1.31210344809688\\
2563	1.27924474953415\\
2564	1.26669281962999\\
2565	1.26364900979838\\
2566	1.25422801384745\\
2567	1.26574055607666\\
2568	1.37190347980952\\
2569	1.46488785693239\\
2570	1.5218000207516\\
2571	1.55038106770243\\
2572	1.54502494835417\\
2573	1.48823078155585\\
2574	1.35395768790798\\
2575	1.3308065528321\\
2576	1.30948607967116\\
2577	1.32211777319854\\
2578	1.33263929982467\\
2579	1.34597165877552\\
2580	1.36396720694012\\
2581	1.37916767424388\\
2582	1.40989921998949\\
2583	1.44075327336176\\
2584	1.42214830077937\\
2585	1.40468679552339\\
2586	1.36040222312283\\
2587	1.3549085945791\\
2588	1.38480887868092\\
2589	1.33352683146\\
2590	1.29724101831336\\
2591	1.30261583551133\\
2592	1.40713128225155\\
2593	1.50871457425981\\
2594	1.57341550225123\\
2595	1.5973228399871\\
2596	1.57370639269312\\
2597	1.56121004509647\\
2598	1.55418459100415\\
2599	1.52751603309624\\
2600	1.46403494318029\\
2601	1.44210635919353\\
2602	1.44630151224562\\
2603	1.42585724321132\\
2604	1.47737977346363\\
2605	1.57612154921073\\
2606	1.66100078748186\\
2607	1.71211465293857\\
2608	1.67929253862999\\
2609	1.59927165455158\\
2610	1.52396571082729\\
2611	1.48713710546762\\
2612	1.52094339575594\\
2613	1.52911614714488\\
2614	1.54768260481383\\
2615	1.5416127649904\\
2616	1.61856833304569\\
2617	1.74106241600103\\
2618	1.83185623282691\\
2619	1.86828133254571\\
2620	1.89703668379607\\
2621	1.86807388539285\\
2622	1.81196320830395\\
2623	1.80843274479394\\
2624	1.71069176403488\\
2625	1.6479618801787\\
2626	1.64995776787393\\
2627	1.66726745470556\\
2628	1.66794981749934\\
2629	1.70408763318624\\
2630	1.75590622185129\\
2631	1.79656337300872\\
2632	1.79147744249027\\
2633	1.67742110609325\\
2634	1.57899791437189\\
2635	1.49631156078159\\
2636	1.50874887081565\\
2637	1.47504863725732\\
2638	1.46047923775862\\
2639	1.48338298520586\\
2640	1.52516311773139\\
2641	1.60570141587666\\
2642	1.67184350094325\\
2643	1.72181440921284\\
2644	1.72082342892872\\
2645	1.68697727278875\\
2646	1.65485873693513\\
2647	1.67827766747208\\
2648	1.65453975028823\\
2649	1.64363758853879\\
2650	1.63577564464782\\
2651	1.69424943735699\\
2652	1.70098717301376\\
2653	1.78295520099112\\
2654	1.76240185606654\\
2655	1.75210621846416\\
2656	1.68799302566996\\
2657	1.55672892586897\\
2658	1.4538390391255\\
2659	1.42124163222008\\
2660	1.44536597511798\\
2661	1.45961274024156\\
2662	1.41020908243657\\
2663	1.4048154155646\\
2664	1.4804967012563\\
2665	1.57313067912719\\
2666	1.6274398129024\\
2667	1.63627378609345\\
2668	1.60910554437361\\
2669	1.50282722502828\\
2670	1.32299779316529\\
2671	1.22669761984199\\
2672	1.22954978065047\\
2673	1.24780555541982\\
2674	1.25465639791746\\
2675	1.28042838891942\\
2676	1.31514594187815\\
2677	1.31761052018671\\
2678	1.31432434008686\\
2679	1.31722110046228\\
2680	1.29392284056994\\
2681	1.24717700919381\\
2682	1.21800507493495\\
2683	1.20879285070284\\
2684	1.23578407752829\\
2685	1.22996456377151\\
2686	1.21329049008808\\
2687	1.25293507399839\\
2688	1.34865456141543\\
2689	1.43130831560934\\
2690	1.49884483603819\\
2691	1.5024228229863\\
2692	1.4874919077175\\
2693	1.39517921630652\\
2694	1.26401747435107\\
2695	1.2353476057694\\
2696	1.24713172814904\\
2697	1.25815683882602\\
2698	1.26741469348328\\
2699	1.26154593379812\\
2700	1.29229590674641\\
2701	1.28764042680623\\
2702	1.2721902130548\\
2703	1.26205589542907\\
2704	1.24752587886291\\
2705	1.23890993771766\\
2706	1.2031434691523\\
2707	1.20218488132074\\
2708	1.23313693442622\\
2709	1.18176480239868\\
2710	1.19314769892681\\
2711	1.25258946910222\\
2712	1.36125299596497\\
2713	1.43402336888919\\
2714	1.48380376034148\\
2715	1.51524114552537\\
2716	1.49818136240814\\
2717	1.42163242990463\\
2718	1.27702806006731\\
2719	1.20178935662885\\
2720	1.178797053884\\
2721	1.1652975620232\\
2722	1.18551497147898\\
2723	1.16695964491392\\
2724	1.19578442930858\\
2725	1.20275222536312\\
2726	1.22174214673781\\
2727	1.21266160170127\\
2728	1.21030129589833\\
2729	1.19911763530019\\
2730	1.19022154734175\\
2731	1.20828219369449\\
2732	1.23070343073045\\
2733	1.24742136063187\\
2734	1.2824652174485\\
2735	1.33433815408127\\
2736	1.46301780501111\\
2737	1.55630565801715\\
2738	1.6057028590418\\
2739	1.62933942023015\\
2740	1.6141781325833\\
2741	1.53775276113273\\
2742	1.39323826000312\\
2743	1.30925036937679\\
2744	1.33483412649255\\
2745	1.31224352036548\\
2746	1.36642636398326\\
2747	1.39444728635259\\
2748	1.4780944632717\\
2749	1.4666870932021\\
2750	1.42378118244147\\
2751	1.42043143865979\\
2752	1.38587628785267\\
2753	1.34384151492922\\
2754	1.29459083551554\\
2755	1.29432903753025\\
2756	1.30822385187185\\
2757	1.3016902080609\\
2758	1.32136015835214\\
2759	1.35848239855843\\
2760	1.48119278531623\\
2761	1.58008463165729\\
2762	1.66767657818132\\
2763	1.67967815102297\\
2764	1.67730297004474\\
2765	1.67521556707195\\
2766	1.67268258171578\\
2767	1.63718700122009\\
2768	1.57800394495903\\
2769	1.56130106475819\\
2770	1.58562013348031\\
2771	1.61149536855122\\
2772	1.65821998136198\\
2773	1.68843575910541\\
2774	1.70417603672251\\
2775	1.67226143571594\\
2776	1.60621061603248\\
2777	1.52167646694611\\
2778	1.49131732340811\\
2779	1.43301030487338\\
2780	1.44817201370191\\
2781	1.44237868590036\\
2782	1.47691444354499\\
2783	1.49009983499636\\
2784	1.60427812771019\\
2785	1.70244621603674\\
2786	1.78007231815471\\
2787	1.80544104673792\\
2788	1.81311487548213\\
2789	1.80425101087162\\
2790	1.80808164568491\\
2791	1.79133119362592\\
2792	1.73739985306386\\
2793	1.68931564172938\\
2794	1.68320999129273\\
2795	1.66633265017385\\
2796	1.65071332945278\\
2797	1.72126732447105\\
2798	1.73434652428938\\
2799	1.73492416114588\\
2800	1.71420446289097\\
2801	1.6274714455784\\
2802	1.54627627379111\\
2803	1.4931559960267\\
2804	1.49774013818914\\
2805	1.44695407204197\\
2806	1.45828420864959\\
2807	1.47895861899845\\
2808	1.57787051033176\\
2809	1.64429877050676\\
2810	1.69582001933958\\
2811	1.71235534657726\\
2812	1.6707279994851\\
2813	1.54013008368874\\
2814	1.37200233921205\\
2815	1.29868981497081\\
2816	1.28382853347628\\
2817	1.31894654825127\\
2818	1.32458761099919\\
2819	1.26584671240289\\
2820	1.26554290570379\\
2821	1.27102103940332\\
2822	1.27993106051647\\
2823	1.26932686719795\\
2824	1.2543207748956\\
2825	1.24447648554614\\
2826	1.22474699411563\\
2827	1.21185486311818\\
2828	1.23045663192858\\
2829	1.25117558902969\\
2830	1.23885718857785\\
2831	1.29554839643957\\
2832	1.39332762852082\\
2833	1.46314333848957\\
2834	1.51566237110565\\
2835	1.5251588171127\\
2836	1.51705806008034\\
2837	1.44480533272068\\
2838	1.33719038076799\\
2839	1.26532296518722\\
2840	1.21677202227245\\
2841	1.19666266911192\\
2842	1.20940272807806\\
2843	1.18362737453917\\
2844	1.20438516914433\\
2845	1.18370504336814\\
2846	1.18110698296097\\
2847	1.19523799640249\\
2848	1.20085974816424\\
2849	1.21549219083678\\
2850	1.21124063265605\\
2851	1.24462962584249\\
2852	1.26546566979005\\
2853	1.25971113912951\\
2854	1.24892347626245\\
2855	1.32417431005052\\
2856	1.40261835661346\\
2857	1.47859156172959\\
2858	1.52578652882179\\
2859	1.53707213685171\\
2860	1.52280349600958\\
2861	1.44534021338878\\
2862	1.32389088326081\\
2863	1.2465290901564\\
2864	1.23412603175284\\
2865	1.21576758756779\\
2866	1.21904881118622\\
2867	1.2360235560359\\
2868	1.27114026938582\\
2869	1.27871632207873\\
2870	1.30627613491297\\
2871	1.32512585350739\\
2872	1.32816831982406\\
2873	1.30160400881182\\
2874	1.28942760009927\\
2875	1.29449212758005\\
2876	1.30693192822099\\
2877	1.27406527698735\\
2878	1.25841353095422\\
2879	1.30963771720873\\
2880	1.43010610189892\\
2881	1.52477329599021\\
2882	1.58874374482988\\
2883	1.60661241263019\\
2884	1.629757570769\\
2885	1.62853745639297\\
2886	1.65310480793013\\
2887	1.68766117742958\\
2888	1.6631226097365\\
2889	1.64186001712883\\
2890	1.6252510900762\\
2891	1.59061216967792\\
2892	1.55140129452212\\
2893	1.61738490604192\\
2894	1.61419472971525\\
2895	1.61495585383827\\
2896	1.57397704418468\\
2897	1.51098625450491\\
2898	1.43741773661076\\
2899	1.41809226156888\\
2900	1.428802803546\\
2901	1.40151283847672\\
2902	1.35279757587901\\
2903	1.39134562035848\\
2904	1.55456881074965\\
2905	1.63400570030979\\
2906	1.70249452680334\\
2907	1.73018143975509\\
2908	1.73168025654978\\
2909	1.6771247358467\\
2910	1.58843407935424\\
2911	1.57713899770444\\
2912	1.54596078333973\\
2913	1.52267116201845\\
2914	1.55617464973651\\
2915	1.59050857665978\\
2916	1.61239952680814\\
2917	1.70964493965081\\
2918	1.74440909301212\\
2919	1.76507481200469\\
2920	1.71473539823691\\
2921	1.62706576329385\\
2922	1.55550033549874\\
2923	1.52915932865571\\
2924	1.51395064158707\\
2925	1.50451546374351\\
2926	1.46820489367934\\
2927	1.49459043555808\\
2928	1.61781575794469\\
2929	1.72366304261695\\
2930	1.78333966311024\\
2931	1.81346255026161\\
2932	1.7987125301121\\
2933	1.76529966573959\\
2934	1.74601003137112\\
2935	1.68253730217259\\
2936	1.61694097676544\\
2937	1.5957957896605\\
2938	1.61851495369323\\
2939	1.66785373892567\\
2940	1.69630664634076\\
2941	1.75829569578537\\
2942	1.77759551935295\\
2943	1.77757332137522\\
2944	1.72582342555817\\
2945	1.62656786340615\\
2946	1.54097084354168\\
2947	1.52427231962874\\
2948	1.57560345589223\\
2949	1.55240481514119\\
2950	1.53471817787526\\
2951	1.5509424582395\\
2952	1.52351696296232\\
2953	1.61379966329205\\
2954	1.66273556517011\\
2955	1.67793909044954\\
2956	1.67848842832389\\
2957	1.66136948920028\\
2958	1.67549071026775\\
2959	1.65052711144532\\
2960	1.64931262430877\\
2961	1.66064439311481\\
2962	1.68715179093269\\
2963	1.75487471771076\\
2964	1.79954582857799\\
2965	1.86665912446352\\
2966	1.95414966060664\\
2967	1.94143281038642\\
2968	1.82860362202457\\
2969	1.6875560073716\\
2970	1.5384119903671\\
2971	1.46691585024388\\
2972	1.45954491751169\\
2973	1.46860504090133\\
2974	1.44197355214318\\
2975	1.48112909777243\\
2976	1.58900991947209\\
2977	1.6492061331502\\
2978	1.71388445829987\\
2979	1.72470307103565\\
2980	1.68447170322625\\
2981	1.55606142056463\\
2982	1.41841206424193\\
2983	1.29695004162293\\
2984	1.338195700182\\
2985	1.3297561639832\\
2986	1.37611483226351\\
2987	1.43497606678726\\
2988	1.49534379869371\\
2989	1.481761823544\\
2990	1.47382190330478\\
2991	1.47541053270482\\
2992	1.43336632820087\\
2993	1.36241554998217\\
2994	1.31087485207229\\
2995	1.29113743083812\\
2996	1.31682387785546\\
2997	1.32750182583552\\
2998	1.33048173685631\\
2999	1.40861353083552\\
3000	1.56076167455419\\
3001	1.67599651653322\\
3002	1.7294154584597\\
3003	1.74009422597288\\
3004	1.71774984621465\\
3005	1.61692322990636\\
3006	1.47850401254996\\
3007	1.35059384100487\\
3008	1.30459516062067\\
3009	1.26770404482015\\
3010	1.2762124884976\\
3011	1.28134582306326\\
3012	1.34426333785239\\
3013	1.36172003810569\\
3014	1.3887744028308\\
3015	1.38545924027411\\
3016	1.37086953195588\\
3017	1.32714928937876\\
3018	1.34245623679965\\
3019	1.29938150042557\\
3020	1.32426603645872\\
3021	1.35583972344898\\
3022	1.33596022024731\\
3023	1.3859898084714\\
3024	1.50200658174123\\
3025	1.58608762425269\\
3026	1.65562854565878\\
3027	1.66423576880101\\
3028	1.64481730942736\\
3029	1.54808780127294\\
3030	1.42007644774353\\
3031	1.30878371148241\\
3032	1.28166751973344\\
3033	1.29234008311136\\
3034	1.31002587401842\\
3035	1.30427608034439\\
3036	1.31755521035858\\
3037	1.32522874408011\\
3038	1.35854755281493\\
3039	1.38178429816443\\
3040	1.39097311358233\\
3041	1.36434427908645\\
3042	1.33124061005872\\
3043	1.31572314804926\\
3044	1.3297374337493\\
3045	1.32947915426585\\
3046	1.29635457056906\\
3047	1.35059207001644\\
3048	1.50032302888691\\
3049	1.60498624213676\\
3050	1.68662725343702\\
3051	1.70696386801105\\
3052	1.66736813734637\\
3053	1.61344195176821\\
3054	1.40603709271621\\
3055	1.34528064733439\\
3056	1.3221421471203\\
3057	1.32139106253662\\
3058	1.27997927227389\\
3059	1.25407539053781\\
3060	1.28372343744995\\
3061	1.26598023397519\\
3062	1.26862492807164\\
3063	1.26270468159325\\
3064	1.26218266888468\\
3065	1.25059564642472\\
3066	1.25021022269834\\
3067	1.3022864974074\\
3068	1.35508599186767\\
3069	1.3782350388799\\
3070	1.3820788579868\\
3071	1.43163050132078\\
3072	1.58988690472834\\
3073	1.67028533827484\\
3074	1.74015861573471\\
3075	1.73151310362504\\
3076	1.71638138196528\\
3077	1.63018572320245\\
3078	1.4530828198303\\
3079	1.31993906522812\\
3080	1.31742931785501\\
3081	1.32135672946574\\
3082	1.353382866227\\
3083	1.34634730731225\\
3084	1.42737527321303\\
3085	1.48612329758788\\
3086	1.51946319262438\\
3087	1.54306102752882\\
3088	1.53981889928724\\
3089	1.50032024807032\\
3090	1.47062430579373\\
3091	1.46360793638203\\
3092	1.51393992545702\\
3093	1.4852142636451\\
3094	1.39154309617949\\
3095	1.41843282478392\\
3096	1.49342054911632\\
3097	1.60355968662351\\
3098	1.67384770423102\\
3099	1.71152111854656\\
3100	1.73423616609154\\
3101	1.72163099094091\\
3102	1.71159207144974\\
3103	1.64152514705456\\
3104	1.51765348824139\\
3105	1.4716710304345\\
3106	1.44252826235223\\
3107	1.47004942929362\\
3108	1.5083871206395\\
3109	1.54624016048612\\
3110	1.5787221569081\\
3111	1.59213760118659\\
3112	1.6501451685875\\
3113	1.63955618459683\\
3114	1.61104604917153\\
3115	1.61745313446041\\
3116	1.67819384598397\\
3117	1.67334323146827\\
3118	1.63722299352282\\
3119	1.66283991279328\\
3120	1.77691201942842\\
3121	1.93607386954957\\
3122	2.06972624774105\\
3123	2.1043370835617\\
3124	2.12993053724072\\
3125	2.13577251994424\\
3126	2.19042992453352\\
3127	2.15147441283882\\
3128	2.05320727765085\\
3129	2.01139375427059\\
3130	2.01867439192536\\
3131	1.98362000411839\\
3132	2.02836476388767\\
3133	2.08034576463229\\
3134	2.17165314406387\\
3135	2.21223088022835\\
3136	2.14689964238336\\
3137	1.9835123396369\\
3138	1.78882966133314\\
3139	1.75795179311635\\
3140	1.74273403878859\\
3141	1.7152032102684\\
3142	1.66637260408444\\
3143	1.70496535135049\\
3144	1.83336886883493\\
3145	1.88258737399275\\
3146	1.95646249763428\\
3147	1.96413238889372\\
3148	1.88073177761805\\
3149	1.75069824773805\\
3150	1.54580987554458\\
3151	1.46514814535492\\
3152	1.40413632318222\\
3153	1.40051430053381\\
3154	1.37558542282945\\
3155	1.37418369419289\\
3156	1.41724063215675\\
3157	1.37128415405376\\
3158	1.38246983586313\\
3159	1.37225886363751\\
3160	1.3329702583026\\
3161	1.30363658520668\\
3162	1.29416241439231\\
3163	1.27440547797985\\
3164	1.24445558903586\\
3165	1.25528550322493\\
3166	1.23799437553153\\
3167	1.27905789398656\\
3168	1.39224546044384\\
3169	1.50986882628868\\
3170	1.55887661334803\\
3171	1.57809572222256\\
3172	1.5586766377644\\
3173	1.49435135921643\\
3174	1.37123558807571\\
3175	1.34305585762286\\
3176	1.24255569762498\\
3177	1.23537644425507\\
3178	1.29837813822632\\
3179	1.28337902347451\\
3180	1.24083145192207\\
3181	1.29444198365764\\
3182	1.32153674810697\\
3183	1.29774817817401\\
3184	1.30196184405084\\
3185	1.32814280381696\\
3186	1.26683662589449\\
3187	1.27569286217303\\
3188	1.29070143650631\\
3189	1.29198690585418\\
3190	1.24834314431301\\
3191	1.28950288584908\\
3192	1.38272853047242\\
3193	1.46599908062339\\
3194	1.52034396325737\\
3195	1.52995941965078\\
3196	1.50347131047847\\
3197	1.42955984741482\\
3198	1.32049766529779\\
3199	1.25054157981764\\
3200	1.26082302086434\\
3201	1.29941380018177\\
3202	1.28592943344824\\
3203	1.29045615936632\\
3204	1.30169471950475\\
3205	1.33496054951618\\
3206	1.3544019218714\\
3207	1.3639260846577\\
3208	1.37584623837483\\
3209	1.35214859245916\\
3210	1.33242692074987\\
3211	1.32746522494677\\
3212	1.33731681799662\\
3213	1.3263898735602\\
3214	1.26936111348618\\
3215	1.30445716949973\\
3216	1.44246011537526\\
3217	1.49851716302101\\
3218	1.55083629323821\\
3219	1.55366668843999\\
3220	1.53244213027414\\
3221	1.45920226390856\\
3222	1.35022700545099\\
3223	1.28515074242048\\
3224	1.2708480252462\\
3225	1.2329522749346\\
3226	1.27798153475977\\
3227	1.30246231641602\\
3228	1.38889942110729\\
3229	1.43364028199064\\
3230	1.44495080544145\\
3231	1.44600731876367\\
3232	1.40910499713108\\
3233	1.34890346315035\\
3234	1.30676732477788\\
3235	1.29386389211209\\
3236	1.30856256469901\\
3237	1.31006625289483\\
3238	1.26800212298789\\
3239	1.30865292632285\\
3240	1.42724570366435\\
3241	1.50679937714101\\
3242	1.55907504962773\\
3243	1.56906439538011\\
3244	1.54191781281377\\
3245	1.47319600560829\\
3246	1.37062417179471\\
3247	1.27718499194161\\
3248	1.28212003873568\\
3249	1.30493013743004\\
3250	1.32547037469605\\
3251	1.36550828585216\\
3252	1.43192875736553\\
3253	1.4619326759724\\
3254	1.47386320317288\\
3255	1.50279779829202\\
3256	1.46519943905474\\
3257	1.41323725301835\\
3258	1.36612455317384\\
3259	1.34119200998597\\
3260	1.36033342631902\\
3261	1.37056829218819\\
3262	1.35748584674218\\
3263	1.35648293354937\\
3264	1.41080157154587\\
3265	1.50125597383828\\
3266	1.56078635123696\\
3267	1.58762764953674\\
3268	1.61783774733454\\
3269	1.61701021316971\\
3270	1.60827152048674\\
3271	1.54497607805231\\
3272	1.48652498910574\\
3273	1.46826172757546\\
3274	1.53616065520112\\
3275	1.58586872548615\\
3276	1.60254407817294\\
3277	1.68784516402285\\
3278	1.74584272920093\\
3279	1.73444523898619\\
3280	1.65245852744748\\
3281	1.57340246974063\\
3282	1.48832378213865\\
3283	1.43984835709963\\
3284	1.43645397406676\\
3285	1.42232314629383\\
3286	1.44393386121314\\
3287	1.42786570356729\\
3288	1.49746540357466\\
3289	1.63496492738467\\
3290	1.69324669635898\\
3291	1.72066030650189\\
3292	1.74713149984665\\
3293	1.76529860950729\\
3294	1.77427419508714\\
3295	1.73549026407388\\
3296	1.71547203534736\\
3297	1.72431697247761\\
3298	1.73853371122601\\
3299	1.77535380920386\\
3300	1.820779951402\\
3301	1.89047697451506\\
3302	1.92386840185119\\
3303	1.93568387580577\\
3304	1.85495539768721\\
3305	1.7691041369076\\
3306	1.63628723145542\\
3307	1.59375246503559\\
3308	1.57118432292419\\
3309	1.54559225738105\\
3310	1.51385771298304\\
3311	1.54286481145688\\
3312	1.66803299437027\\
3313	1.74501142046612\\
3314	1.80178353154346\\
3315	1.82022141355844\\
3316	1.762166920183\\
3317	1.66618826096858\\
3318	1.52012671003818\\
3319	1.41327606614251\\
3320	1.36495471377756\\
3321	1.3640878188046\\
3322	1.33921797211575\\
3323	1.34136625115527\\
3324	1.40612392870974\\
3325	1.40946783764641\\
3326	1.40431330862732\\
3327	1.39879108613499\\
3328	1.39042418118953\\
3329	1.3476821251704\\
3330	1.32245442822313\\
3331	1.32356323368919\\
3332	1.36678957723635\\
3333	1.39687840039249\\
3334	1.36427347009581\\
3335	1.42005436657419\\
3336	1.51526753618535\\
3337	1.60788820461274\\
3338	1.69213000193478\\
3339	1.71212827575141\\
3340	1.67691086800042\\
3341	1.62648124545162\\
3342	1.47843370047879\\
3343	1.37848362399023\\
3344	1.34469831308842\\
3345	1.33398414041637\\
3346	1.32849985764695\\
3347	1.34741368236285\\
3348	1.39773196240349\\
3349	1.39399599906194\\
3350	1.38551068242474\\
3351	1.35717826909054\\
3352	1.33437154095154\\
3353	1.29041304807137\\
3354	1.26542765416139\\
3355	1.25039313661155\\
3356	1.28509087171595\\
3357	1.26343660807086\\
3358	1.28864420916392\\
3359	1.32428747942227\\
3360	1.39525390942568\\
3361	1.46630120959867\\
3362	1.5172129512798\\
3363	1.54764232693269\\
3364	1.54034598996163\\
3365	1.49173601371162\\
3366	1.38316075251922\\
3367	1.34367526566712\\
3368	1.28910195809662\\
3369	1.25789369144479\\
3370	1.2888901291317\\
3371	1.23852090080989\\
3372	1.25332475765947\\
3373	1.27201195479046\\
3374	1.26673365668897\\
3375	1.25153391965774\\
3376	1.25934842024774\\
3377	1.26268975711321\\
3378	1.26722588911801\\
3379	1.28095761601508\\
3380	1.31767504707716\\
3381	1.31407404516357\\
3382	1.29539281255859\\
3383	1.35987790300152\\
3384	1.61594707365686\\
3385	1.79132064893098\\
3386	1.88158807311251\\
3387	1.88498566944586\\
3388	1.83470898003691\\
3389	1.75172180260946\\
3390	1.61043732266182\\
3391	1.46174167686307\\
3392	1.41419331358836\\
3393	1.40621282217625\\
3394	1.41142860182528\\
3395	1.44241639462054\\
3396	1.47023232122599\\
3397	1.45786604259448\\
3398	1.44533032896692\\
3399	1.43725284207534\\
3400	1.39740530670913\\
3401	1.37551035472875\\
3402	1.31920626336905\\
3403	1.36069091555962\\
3404	1.38980705388244\\
3405	1.36027274468698\\
3406	1.32902099523441\\
3407	1.39468891860774\\
3408	1.53317494916422\\
3409	1.62609327187703\\
3410	1.69954087617078\\
3411	1.70771432402119\\
3412	1.67132046542995\\
3413	1.61681720885396\\
3414	1.48012681342022\\
3415	1.3568684019681\\
3416	1.36188409987195\\
3417	1.3117467075406\\
3418	1.33385717357527\\
3419	1.33656111021969\\
3420	1.3675393183701\\
3421	1.36149987419627\\
3422	1.38584414019356\\
3423	1.3966567294372\\
3424	1.4030710212163\\
3425	1.35784220568983\\
3426	1.36756383204136\\
3427	1.37976397616426\\
3428	1.41919030573347\\
3429	1.46161474451579\\
3430	1.41432579392631\\
3431	1.43317467021934\\
3432	1.47281290949161\\
3433	1.56262111655571\\
3434	1.65824144324321\\
3435	1.71897535347258\\
3436	1.73110599194389\\
3437	1.76780695619659\\
3438	1.77095005917909\\
3439	1.71205712485509\\
3440	1.61649351778959\\
3441	1.61658726191139\\
3442	1.62908229567601\\
3443	1.66303976607236\\
3444	1.67561359593678\\
3445	1.77665004896764\\
3446	1.87334909791026\\
3447	1.8752947251489\\
3448	1.77542451937641\\
3449	1.66532414885776\\
3450	1.63191509002869\\
3451	1.58629056808435\\
3452	1.58261894229468\\
3453	1.63859011570852\\
3454	1.59260664817838\\
3455	1.66645475579075\\
3456	1.66612758869444\\
3457	1.73972144184186\\
3458	1.81971841162414\\
3459	1.83114987830095\\
3460	1.82510456271422\\
3461	1.84343365726735\\
3462	1.83157298540778\\
3463	1.76263196798128\\
3464	1.71961242101704\\
3465	1.70314618114934\\
3466	1.74095215346314\\
3467	1.78390212485685\\
3468	1.83014626247074\\
3469	1.89367040396159\\
3470	1.93140397903628\\
3471	1.89052395293083\\
3472	1.78628779462794\\
3473	1.66958798776458\\
3474	1.58703982480947\\
3475	1.54581442190018\\
3476	1.5769731172769\\
3477	1.56518434112045\\
3478	1.53831612931946\\
3479	1.57039814742227\\
3480	1.6521681058744\\
3481	1.74690180514905\\
3482	1.80655700125968\\
3483	1.81393693080482\\
3484	1.76954660165159\\
3485	1.69021367543069\\
3486	1.5310755320018\\
3487	1.34391415225513\\
3488	1.27868654411925\\
3489	1.24700879416168\\
3490	1.23776786633892\\
3491	1.23157299031289\\
3492	1.26797862801856\\
3493	1.2972487467166\\
3494	1.3048590693374\\
3495	1.30339370910691\\
3496	1.29783702054218\\
3497	1.28531041336101\\
3498	1.29053900488729\\
3499	1.33223357432914\\
3500	1.37441568037042\\
3501	1.40784757122763\\
3502	1.38343535839022\\
3503	1.42552532145859\\
3504	1.53789818600151\\
3505	1.58998504082942\\
3506	1.62810408197475\\
3507	1.6384141514744\\
3508	1.61469155202715\\
3509	1.56937500998751\\
3510	1.44870688543115\\
3511	1.34463938739476\\
3512	1.28059928334195\\
3513	1.24417346306498\\
3514	1.2214141630151\\
3515	1.23325792239952\\
3516	1.30998976853912\\
3517	1.28920939929713\\
3518	1.29006933877187\\
3519	1.25517423987638\\
3520	1.25404998788724\\
3521	1.25404682091675\\
3522	1.25346041950707\\
3523	1.29198659276015\\
3524	1.32273717481753\\
3525	1.3563345266784\\
3526	1.37617768795751\\
3527	1.40351327182402\\
3528	1.47528592363244\\
3529	1.54977655640936\\
3530	1.60034656336183\\
3531	1.60954591059092\\
3532	1.60616978223731\\
3533	1.55074400062311\\
3534	1.42347663104524\\
3535	1.30492611218036\\
3536	1.27542846589857\\
3537	1.24733913451658\\
3538	1.22524108598256\\
3539	1.22925654942286\\
3540	1.2224559181606\\
3541	1.23483608598032\\
3542	1.25898616100278\\
3543	1.27368045401772\\
3544	1.28592727030291\\
3545	1.26844210878786\\
3546	1.26869286857035\\
3547	1.29235150789516\\
3548	1.33120280021136\\
3549	1.34974275682199\\
3550	1.2971371181664\\
3551	1.33792135016413\\
3552	1.44924036273717\\
3553	1.5322292704943\\
3554	1.59541124545483\\
3555	1.63692208253824\\
3556	1.67068094196564\\
3557	1.70789486634917\\
3558	1.73523597918368\\
3559	1.69363641761737\\
3560	1.66118589227309\\
3561	1.61452761340408\\
3562	1.5768185379352\\
3563	1.56290645534421\\
3564	1.56942468634392\\
3565	1.63121988180104\\
3566	1.69733271691534\\
3567	1.71987436393198\\
3568	1.72557267772143\\
3569	1.69485947596681\\
3570	1.61069916956613\\
3571	1.54356529479618\\
3572	1.54515463718691\\
3573	1.51367739373205\\
3574	1.48851466791849\\
3575	1.55164483091248\\
3576	1.65441397155151\\
3577	1.73799538310475\\
3578	1.79117709459355\\
3579	1.81466938479262\\
3580	1.80218643834785\\
3581	1.76458771335886\\
3582	1.67702915228013\\
3583	1.59336596965713\\
3584	1.57537732329565\\
3585	1.57210313093325\\
3586	1.58404664371216\\
3587	1.55727568008797\\
3588	1.64919111447361\\
3589	1.69701652389237\\
3590	1.69149932758696\\
3591	1.72908952688909\\
3592	1.69174026650414\\
3593	1.68209387234293\\
3594	1.56634409964128\\
3595	1.51817310532781\\
3596	1.53716074891189\\
3597	1.55808618572393\\
3598	1.52417399160467\\
3599	1.53782941496754\\
3600	1.58858663847764\\
3601	1.67199244323598\\
3602	1.73394240018804\\
3603	1.7479021388503\\
3604	1.73969212078291\\
3605	1.75539473342113\\
3606	1.73430482651318\\
3607	1.66235908240646\\
3608	1.6234061318884\\
3609	1.57803804549519\\
3610	1.65983879327477\\
3611	1.66175692227782\\
3612	1.66393352235026\\
3613	1.69016979981357\\
3614	1.72678397910067\\
3615	1.72777192426236\\
3616	1.67422774028148\\
3617	1.59633273805267\\
3618	1.55661358997438\\
3619	1.52934884427107\\
3620	1.53210255384682\\
3621	1.54015754546434\\
3622	1.51095469502889\\
3623	1.50711765102413\\
3624	1.61253361673398\\
3625	1.69744073510795\\
3626	1.75096968820139\\
3627	1.73850099942952\\
3628	1.74609267288591\\
3629	1.77611295202285\\
3630	1.79176103724345\\
3631	1.74673999066845\\
3632	1.69757602657492\\
3633	1.65146137170387\\
3634	1.63942105286224\\
3635	1.64507404734072\\
3636	1.68035714813859\\
3637	1.73625795890791\\
3638	1.78730805691229\\
3639	1.80976327351318\\
3640	1.77145973174921\\
3641	1.66016125248812\\
3642	1.53404974995119\\
3643	1.50320855497487\\
3644	1.48680998465372\\
3645	1.46942726060924\\
3646	1.48781914445541\\
3647	1.45103288749957\\
3648	1.56929971760489\\
3649	1.64793459294226\\
3650	1.6946471472596\\
3651	1.70286727209379\\
3652	1.66667934061087\\
3653	1.59714174342091\\
3654	1.44073901465412\\
3655	1.32601138932868\\
3656	1.30301323734043\\
3657	1.31067097561998\\
3658	1.3336125897619\\
3659	1.34137039082495\\
3660	1.38190246380524\\
3661	1.37328253615264\\
3662	1.36781987980151\\
3663	1.36881231923175\\
3664	1.3483000075915\\
3665	1.30116201115787\\
3666	1.32899703918848\\
3667	1.31806256180328\\
3668	1.34906816405225\\
3669	1.34621159344954\\
3670	1.2898806268131\\
3671	1.35759877201769\\
3672	1.46399048280467\\
3673	1.52135835656344\\
3674	1.57021480511099\\
3675	1.58797552576248\\
3676	1.59526621297103\\
3677	1.56297554627002\\
3678	1.43779593390847\\
3679	1.30479307475122\\
3680	1.27612548527471\\
3681	1.26843508623987\\
3682	1.29362971128189\\
3683	1.29868774040491\\
3684	1.36976050892584\\
3685	1.40934185923732\\
3686	1.38940747508346\\
3687	1.35794283354143\\
3688	1.3221085620893\\
3689	1.32261712363637\\
3690	1.27710695910869\\
3691	1.27773223158342\\
3692	1.31265085461586\\
3693	1.34551436082535\\
3694	1.29861435260488\\
3695	1.36251358002126\\
3696	1.4577450528693\\
3697	1.48727946773778\\
3698	1.5388317344878\\
3699	1.55452120281891\\
3700	1.52154061254977\\
3701	1.49885676147523\\
3702	1.37376504855579\\
3703	1.2702724617331\\
3704	1.19023874194263\\
3705	1.18413930661907\\
3706	1.1630157864384\\
3707	1.17537368861946\\
3708	1.16339919020576\\
3709	1.16581014547604\\
3710	1.21562816837501\\
3711	1.22277445958487\\
3712	1.2291607139089\\
3713	1.2214768169612\\
3714	1.19868898979943\\
3715	1.23700370048132\\
3716	1.29578889936315\\
3717	1.34048347163383\\
3718	1.37802974481406\\
3719	1.45822808239194\\
3720	1.62542618084639\\
3721	1.68730299142826\\
3722	1.74743473806237\\
3723	1.74205322279998\\
3724	1.72774932969227\\
3725	1.69494664100439\\
3726	1.56468557746798\\
3727	1.42094100815621\\
3728	1.37101936290912\\
3729	1.38033453822624\\
3730	1.42279287325755\\
3731	1.44723406471445\\
3732	1.49583214348397\\
3733	1.51296153745204\\
3734	1.54105483778736\\
3735	1.56471908591135\\
3736	1.51332333354539\\
3737	1.44311707519592\\
3738	1.39298210201162\\
3739	1.37038238834379\\
3740	1.39239817397019\\
3741	1.40831787750714\\
3742	1.34286800616276\\
3743	1.3790829393274\\
3744	1.48194479275096\\
3745	1.53725267940936\\
3746	1.58306753027281\\
3747	1.58803974076524\\
3748	1.57221098167797\\
3749	1.52507385934158\\
3750	1.40956711889265\\
3751	1.28344811093828\\
3752	1.26751691644711\\
3753	1.30322577240288\\
3754	1.33817366140415\\
3755	1.36196216099695\\
3756	1.37990997560199\\
3757	1.40149768005111\\
3758	1.44106140393456\\
3759	1.44733111485408\\
3760	1.42319366424452\\
3761	1.37965334897607\\
3762	1.36175150018695\\
3763	1.3550096747808\\
3764	1.39599200495646\\
3765	1.44448988154368\\
3766	1.38321296646044\\
3767	1.45470505099715\\
3768	1.54610360837696\\
3769	1.63285676431404\\
3770	1.70283788060351\\
3771	1.75752370073179\\
3772	1.77242811655421\\
3773	1.845139379414\\
3774	1.82172366476321\\
3775	1.72109187834806\\
3776	1.63617585851796\\
3777	1.57305554230484\\
3778	1.54900100824105\\
3779	1.55181755128365\\
3780	1.58192125794566\\
3781	1.6238039297677\\
3782	1.67430992750607\\
3783	1.66113008807146\\
3784	1.60872641528051\\
3785	1.56266072923787\\
3786	1.51852139790263\\
3787	1.4842384558708\\
3788	1.4962097154783\\
3789	1.48272336021206\\
3790	1.44702916047302\\
3791	1.47717380881303\\
3792	1.55382520480484\\
3793	1.64134450067479\\
3794	1.70012093148227\\
3795	1.72123594661883\\
3796	1.72419116668653\\
3797	1.77709232047157\\
3798	1.79434058597665\\
3799	1.72700128606725\\
3800	1.6533575496581\\
3801	1.60440696604998\\
3802	1.5920694477183\\
3803	1.60750928296351\\
3804	1.65012075701476\\
3805	1.76370012586292\\
3806	1.82167989339568\\
3807	1.84721942346852\\
3808	1.79380198615085\\
3809	1.70970455462316\\
3810	1.62113977619217\\
3811	1.59440463601407\\
3812	1.57441629961808\\
3813	1.55062003804769\\
3814	1.5692028266816\\
3815	1.56578860682774\\
3816	1.64695455299042\\
3817	1.73985531875917\\
3818	1.82364266918675\\
3819	1.83129097486363\\
3820	1.85859748173624\\
3821	1.89877486272914\\
3822	1.88054116138669\\
3823	1.80386395160727\\
3824	1.74011410639167\\
3825	1.68755906541981\\
3826	1.69116369818957\\
3827	1.67020393189327\\
3828	1.68406761660961\\
3829	1.6885789829241\\
3830	1.73537129877557\\
3831	1.73729078131874\\
3832	1.68708028136734\\
3833	1.62949588014114\\
3834	1.5572716274429\\
3835	1.51407816913073\\
3836	1.48910655890562\\
3837	1.4980440130649\\
3838	1.47082422529746\\
3839	1.49433991807984\\
3840	1.54052871657566\\
3841	1.60428631668962\\
3842	1.63308234734281\\
3843	1.6969558133662\\
3844	1.69496751641167\\
3845	1.62260350459049\\
3846	1.45243584365977\\
3847	1.3228292356722\\
3848	1.24263593780518\\
3849	1.22138283008873\\
3850	1.23827362048467\\
3851	1.22652485901527\\
3852	1.25400584046108\\
3853	1.26094667228788\\
3854	1.25876985005813\\
3855	1.27917477701699\\
3856	1.28073583119562\\
3857	1.25744547095255\\
3858	1.22171156985212\\
3859	1.22891116308928\\
3860	1.26689679245714\\
3861	1.29490158118341\\
3862	1.25986617616603\\
3863	1.30108763867455\\
3864	1.40407772180231\\
3865	1.43696861080253\\
3866	1.48318372415007\\
3867	1.49023083137904\\
3868	1.46064690797695\\
3869	1.44201113404697\\
3870	1.31289587789456\\
3871	1.23586127612009\\
3872	1.19120594942631\\
3873	1.1972638113535\\
3874	1.2398721899103\\
3875	1.25057422285247\\
3876	1.26354581703321\\
3877	1.27044968208399\\
3878	1.29451832747296\\
3879	1.29598308630525\\
3880	1.26422731235269\\
3881	1.2287300933331\\
3882	1.24917263820293\\
3883	1.21545397069644\\
3884	1.21868048553675\\
3885	1.25825698980613\\
3886	1.22510620536524\\
3887	1.26122920319733\\
3888	1.36195714581954\\
3889	1.4227788760954\\
3890	1.47472041636434\\
3891	1.48990667545931\\
3892	1.476084215177\\
3893	1.44091426659924\\
3894	1.33466955241681\\
3895	1.21733333402759\\
3896	1.19258667451659\\
3897	1.23896370481109\\
3898	1.24160896712122\\
3899	1.26536832115598\\
3900	1.31756116606996\\
3901	1.30802354090699\\
3902	1.32336025552507\\
3903	1.31606495486159\\
3904	1.27615938561948\\
3905	1.25592999667836\\
3906	1.21196267173485\\
3907	1.21370914781147\\
3908	1.22968850754999\\
3909	1.24736051735245\\
3910	1.21217929774862\\
3911	1.26447318691481\\
3912	1.3117676751058\\
3913	1.37717033794416\\
3914	1.42006009476563\\
3915	1.43749764543925\\
3916	1.48179465658843\\
3917	1.44029768075546\\
3918	1.32510423346029\\
3919	1.20718088208272\\
3920	1.1938366114696\\
3921	1.21247297405552\\
3922	1.22776046245526\\
3923	1.23688781852953\\
3924	1.26364601638577\\
3925	1.27125675304712\\
3926	1.2970995147342\\
3927	1.32115271201425\\
3928	1.29236568355441\\
3929	1.2808615329221\\
3930	1.26359392223323\\
3931	1.2599150213718\\
3932	1.27375197498322\\
3933	1.29237982149662\\
3934	1.25319915935664\\
3935	1.29640593669678\\
3936	1.35762389833102\\
3937	1.43199603588574\\
3938	1.4363765672514\\
3939	1.46439119466006\\
3940	1.49501738750928\\
3941	1.52625790685329\\
3942	1.5442944915498\\
3943	1.47455622136698\\
3944	1.4241270723373\\
3945	1.39859170343386\\
3946	1.42986701893472\\
3947	1.43529114793996\\
3948	1.48743775745037\\
3949	1.53862980548707\\
3950	1.61879637728744\\
3951	1.65263377390224\\
3952	1.62132448938724\\
3953	1.5367700767582\\
3954	1.50135255902522\\
3955	1.43043805642336\\
3956	1.44554491491109\\
3957	1.45880427597704\\
3958	1.38959317308818\\
3959	1.42658075524807\\
3960	1.4828942256248\\
3961	1.55767679106148\\
3962	1.61945715869091\\
3963	1.65400475023648\\
3964	1.65529879628369\\
3965	1.70097957904286\\
3966	1.68577766784164\\
3967	1.63784243485142\\
3968	1.60233028135369\\
3969	1.60036305899578\\
3970	1.59258204495623\\
3971	1.59913302706937\\
3972	1.66177063349327\\
3973	1.71252910337078\\
3974	1.79692445855988\\
3975	1.82149129977999\\
3976	1.80320919268491\\
3977	1.69772670913118\\
3978	1.57152389225968\\
3979	1.49236695004415\\
3980	1.4623126337071\\
3981	1.47111193680151\\
3982	1.4299062830571\\
3983	1.46931162556171\\
3984	1.52019967917529\\
3985	1.63273460985966\\
3986	1.67102491117197\\
3987	1.66487172877752\\
3988	1.62559085268999\\
3989	1.55969390873533\\
3990	1.42438571926197\\
3991	1.29459521637564\\
3992	1.26179069912829\\
3993	1.28198386945227\\
3994	1.28976391795728\\
3995	1.28250838589964\\
3996	1.28666614901737\\
3997	1.30609644898647\\
3998	1.32957544298294\\
3999	1.33215841507071\\
4000	1.32964287270559\\
4001	1.31618941572949\\
};
\addplot [color=mycolor1,line width=1.3pt,solid,forget plot]
  table[row sep=crcr]{%
4001	1.31618941572949\\
4002	1.29559453608764\\
4003	1.31608266758528\\
4004	1.33720879124765\\
4005	1.35632342689396\\
4006	1.32519426188851\\
4007	1.37028832738409\\
4008	1.4977389816997\\
4009	1.52743305227643\\
4010	1.55597518957749\\
4011	1.53741919955852\\
4012	1.52273775635947\\
4013	1.46338672809951\\
4014	1.38049536430936\\
4015	1.29299617862611\\
4016	1.22901716047218\\
4017	1.19161613294501\\
4018	1.26613794170925\\
4019	1.22752355109554\\
4020	1.32703310646884\\
4021	1.39272714364408\\
4022	1.38956252880666\\
4023	1.38002175167595\\
4024	1.37688430919111\\
4025	1.36598053189071\\
4026	1.39530392793119\\
4027	1.38888498612239\\
4028	1.37708044681498\\
4029	1.37742622847031\\
4030	1.32503918265647\\
4031	1.3589989478727\\
4032	1.46296190842913\\
4033	1.5489825040329\\
4034	1.57890942962955\\
4035	1.58848172196584\\
4036	1.56434640038085\\
4037	1.52575360967673\\
4038	1.39913292452421\\
4039	1.29331335102123\\
4040	1.28752944825253\\
4041	1.2788540262545\\
4042	1.30497155265534\\
4043	1.30607507532779\\
4044	1.30419532142317\\
4045	1.34798516447704\\
4046	1.36962170502019\\
4047	1.37979812705748\\
4048	1.36015576330416\\
4049	1.32533379648762\\
4050	1.30916242532408\\
4051	1.31116368606774\\
4052	1.32910174823948\\
4053	1.3550526159864\\
4054	1.3094079319432\\
4055	1.33401645056993\\
4056	1.41082974207069\\
4057	1.48754059550494\\
4058	1.5309922071067\\
4059	1.54488315117306\\
4060	1.51825912317412\\
4061	1.47730403143664\\
4062	1.36202861622388\\
4063	1.28094636819851\\
4064	1.23951616471639\\
4065	1.24932626328895\\
4066	1.25655299482283\\
4067	1.28800106024549\\
4068	1.33371615525912\\
4069	1.35012796744352\\
4070	1.32569268443033\\
4071	1.3389173000853\\
4072	1.30711015174073\\
4073	1.26613376257791\\
4074	1.24000279063002\\
4075	1.27633836640107\\
4076	1.32565494149845\\
4077	1.35646304604422\\
4078	1.31605339367937\\
4079	1.36412268946916\\
4080	1.43432212815354\\
4081	1.4941993713795\\
4082	1.54922884134155\\
4083	1.5543428566154\\
4084	1.5544027752952\\
4085	1.52800711819146\\
4086	1.41858508721773\\
4087	1.34264594284767\\
4088	1.30000514987923\\
4089	1.2943399879378\\
4090	1.28493495620388\\
4091	1.29432312435724\\
4092	1.33368370772\\
4093	1.35827401510497\\
4094	1.33482638389293\\
4095	1.34932563517064\\
4096	1.32686895532556\\
4097	1.30851215315966\\
4098	1.30846913604266\\
4099	1.30062398604511\\
4100	1.306734996058\\
4101	1.32419673563923\\
4102	1.28197832097944\\
4103	1.30794238415821\\
4104	1.39463851459985\\
4105	1.46871050131794\\
4106	1.52509604660416\\
4107	1.55920180150872\\
4108	1.59059373299992\\
4109	1.62114511776913\\
4110	1.60898600693006\\
4111	1.53307665113458\\
4112	1.3965255885567\\
4113	1.39976472205997\\
4114	1.38352885513267\\
4115	1.41543659306338\\
4116	1.45901980087057\\
4117	1.48301428051985\\
4118	1.51606729874238\\
4119	1.53200780221918\\
4120	1.50760054625678\\
4121	1.48277494561907\\
4122	1.43654859047325\\
4123	1.40512665027457\\
4124	1.40178010498662\\
4125	1.40579542166428\\
4126	1.38121023434916\\
4127	1.38277274932911\\
4128	1.48593139271085\\
4129	1.56677604790267\\
4130	1.61588910652981\\
4131	1.65770656662642\\
4132	1.64818003488658\\
4133	1.69122635443168\\
4134	1.70687931213232\\
4135	1.64176620241366\\
4136	1.6096159211088\\
4137	1.57633759815222\\
4138	1.57807174480689\\
4139	1.58478422790385\\
4140	1.62589802095437\\
4141	1.6836661485127\\
4142	1.7284461776359\\
4143	1.72569223282136\\
4144	1.66631834281753\\
4145	1.58010044680156\\
4146	1.51981623098922\\
4147	1.48057370980043\\
4148	1.43551043421745\\
4149	1.41322940987785\\
4150	1.36455491342459\\
4151	1.39947015660366\\
4152	1.50055314312815\\
4153	1.56776093409121\\
4154	1.62701992615919\\
4155	1.63960017321494\\
4156	1.6272269599899\\
4157	1.56153425268857\\
4158	1.39918335456132\\
4159	1.27818261784583\\
4160	1.25744054867085\\
4161	1.24922171512694\\
4162	1.28245111978724\\
4163	1.29013877996645\\
4164	1.31061269710729\\
4165	1.31554713075102\\
4166	1.3357969529238\\
4167	1.32733589432155\\
4168	1.30187054689532\\
4169	1.27456663012423\\
4170	1.23976968286901\\
4171	1.2517705133191\\
4172	1.27925702398332\\
4173	1.33134761763261\\
4174	1.27308223218856\\
4175	1.27153217639943\\
4176	1.37033879151016\\
4177	1.44889311637665\\
4178	1.50113441758113\\
4179	1.51444214493756\\
4180	1.50294851761252\\
4181	1.47894263827432\\
4182	1.38115129611963\\
4183	1.29453127780985\\
4184	1.27677338202901\\
4185	1.27940665477827\\
4186	1.2504679660399\\
4187	1.21070151197521\\
4188	1.24669560883242\\
4189	1.22152868883905\\
4190	1.24662258234907\\
4191	1.27121023677755\\
4192	1.27800735373782\\
4193	1.25807399495768\\
4194	1.32342616543423\\
4195	1.30588897356459\\
4196	1.27173147058311\\
4197	1.30617960339352\\
4198	1.24784257551314\\
4199	1.29168720544665\\
4200	1.36930254930289\\
4201	1.43860399795486\\
4202	1.54115706589538\\
4203	1.55251368551415\\
4204	1.5318000913884\\
4205	1.51940348360105\\
4206	1.40357034748643\\
4207	1.30020666305435\\
4208	1.30449963782686\\
4209	1.30616872966461\\
4210	1.3205758628378\\
4211	1.33279259540913\\
4212	1.36111402758336\\
4213	1.32482021991956\\
4214	1.337075747628\\
4215	1.33923316122851\\
4216	1.32707581540292\\
4217	1.30093783383335\\
4218	1.29503857335015\\
4219	1.29023073893879\\
4220	1.30142816785006\\
4221	1.32317031619275\\
4222	1.27423209953808\\
4223	1.3050185353304\\
4224	1.39381099869254\\
4225	1.47259620424479\\
4226	1.5099449957334\\
4227	1.52247529444703\\
4228	1.49103816559809\\
4229	1.45999060519064\\
4230	1.35686387895285\\
4231	1.31204839104068\\
4232	1.29633566490037\\
4233	1.2672602861788\\
4234	1.25653432272031\\
4235	1.28336056932423\\
4236	1.29869825623951\\
4237	1.31149363486506\\
4238	1.31865120458664\\
4239	1.31816688386565\\
4240	1.29365060998065\\
4241	1.26707548168283\\
4242	1.26294105358684\\
4243	1.31685809554586\\
4244	1.32349105841922\\
4245	1.32945601747519\\
4246	1.275263031251\\
4247	1.28937517252696\\
4248	1.37516975771258\\
4249	1.4507748715812\\
4250	1.50428529116204\\
4251	1.53615720441293\\
4252	1.53014174788755\\
4253	1.49786195506492\\
4254	1.3897027345737\\
4255	1.31703111984698\\
4256	1.30014302122484\\
4257	1.25306773624843\\
4258	1.21205868549745\\
4259	1.2507790566151\\
4260	1.24386710789626\\
4261	1.22504138173063\\
4262	1.27566476678454\\
4263	1.34992393602151\\
4264	1.33608599772793\\
4265	1.32150293864922\\
4266	1.34047699202803\\
4267	1.35594931500238\\
4268	1.36632389642256\\
4269	1.36836322203444\\
4270	1.35212261250182\\
4271	1.38069026526006\\
4272	1.51156179774314\\
4273	1.59642943063689\\
4274	1.6670570419327\\
4275	1.70023517190724\\
4276	1.71780830959404\\
4277	1.75055956912029\\
4278	1.70717881839584\\
4279	1.6357603237253\\
4280	1.55633829594116\\
4281	1.50071451922941\\
4282	1.46417798799434\\
4283	1.43906343395589\\
4284	1.4458943989111\\
4285	1.46657920835261\\
4286	1.48498592031279\\
4287	1.46871914608118\\
4288	1.45223764355814\\
4289	1.42466349117417\\
4290	1.37780666905603\\
4291	1.43576404493913\\
4292	1.47440429002814\\
4293	1.45412490436344\\
4294	1.41485761799737\\
4295	1.45086280797502\\
4296	1.54660668211826\\
4297	1.62009706720831\\
4298	1.67512918834857\\
4299	1.69817175230901\\
4300	1.71907775282803\\
4301	1.70966078713112\\
4302	1.70552380828391\\
4303	1.67176822278353\\
4304	1.69602012441316\\
4305	1.6557943748279\\
4306	1.63650795491492\\
4307	1.59000037935756\\
4308	1.53565679833737\\
4309	1.58370596013948\\
4310	1.62444714963466\\
4311	1.66236512823489\\
4312	1.6758571222268\\
4313	1.61938441273584\\
4314	1.55152733731941\\
4315	1.50427356818753\\
4316	1.54600304732303\\
4317	1.48005370451145\\
4318	1.52546271982928\\
4319	1.54500236718032\\
4320	1.587055480313\\
4321	1.63986582772198\\
4322	1.70339907407017\\
4323	1.72830732682964\\
4324	1.69774362438139\\
4325	1.63559591193005\\
4326	1.49868134539045\\
4327	1.38747759799134\\
4328	1.36542381918368\\
4329	1.34076686573725\\
4330	1.3475781680163\\
4331	1.32212203584888\\
4332	1.33947405285563\\
4333	1.36646827606846\\
4334	1.39118570783722\\
4335	1.39569368131295\\
4336	1.37213237382825\\
4337	1.34862122664019\\
4338	1.29610999902684\\
4339	1.31094281482464\\
4340	1.30068895924441\\
4341	1.33194401231678\\
4342	1.29866674637721\\
4343	1.33277409611082\\
4344	1.43308884336158\\
4345	1.48896079374823\\
4346	1.52982045067167\\
4347	1.54969139079214\\
4348	1.531677182099\\
4349	1.51350008767921\\
4350	1.40842986192962\\
4351	1.29780928154691\\
4352	1.2828514689639\\
4353	1.30218647487576\\
4354	1.34082400338246\\
4355	1.36965423479867\\
4356	1.40076715329339\\
4357	1.40450760353909\\
4358	1.43486190642438\\
4359	1.43109248568233\\
4360	1.4176997666995\\
4361	1.37036234046939\\
4362	1.33721781023052\\
4363	1.367751105602\\
4364	1.35365030427683\\
4365	1.45178023021802\\
4366	1.41961971994557\\
4367	1.45305889390258\\
4368	1.49734977561513\\
4369	1.56429917586771\\
4370	1.63773994047444\\
4371	1.66566453615465\\
4372	1.6631546211206\\
4373	1.61894937399649\\
4374	1.52040563907253\\
4375	1.41169724829369\\
4376	1.38452894205561\\
4377	1.38246120632814\\
4378	1.39874972921873\\
4379	1.41556952017938\\
4380	1.47433117163176\\
4381	1.47888991377449\\
4382	1.49539917989589\\
4383	1.48617607997818\\
4384	1.46343701365672\\
4385	1.41836549821394\\
4386	1.3814034767776\\
4387	1.38023389184722\\
4388	1.34897251739531\\
4389	1.3700680038937\\
4390	1.34944104354227\\
4391	1.41146659636917\\
4392	1.49375223043106\\
4393	1.56339877036334\\
4394	1.62130907283976\\
4395	1.65428426986773\\
4396	1.64991234127051\\
4397	1.62685934860187\\
4398	1.52303540110398\\
4399	1.44665786131594\\
4400	1.3806424174047\\
4401	1.38622069551985\\
4402	1.41906802928409\\
4403	1.43917135714619\\
4404	1.489123813568\\
4405	1.49428537407353\\
4406	1.52841632212816\\
4407	1.52110293208369\\
4408	1.50318279847849\\
4409	1.4525013863931\\
4410	1.42695151458626\\
4411	1.43628270996561\\
4412	1.44921013147167\\
4413	1.45153943630807\\
4414	1.43264545473599\\
4415	1.46572374638287\\
4416	1.56117729606036\\
4417	1.63494625125762\\
4418	1.66730014758874\\
4419	1.67817866278197\\
4420	1.65854488156546\\
4421	1.62582398741912\\
4422	1.51918364787478\\
4423	1.40919358901159\\
4424	1.37456566965475\\
4425	1.36938750307759\\
4426	1.36851598028361\\
4427	1.35721676062842\\
4428	1.37102426727749\\
4429	1.37814190348712\\
4430	1.38182402966115\\
4431	1.39357266810357\\
4432	1.41966709337951\\
4433	1.39570683499555\\
4434	1.37664814182431\\
4435	1.39789643702387\\
4436	1.41702305726996\\
4437	1.4529054790648\\
4438	1.43357288746509\\
4439	1.50340537907493\\
4440	1.58627882237049\\
4441	1.67139264916676\\
4442	1.74721595307551\\
4443	1.79126752408154\\
4444	1.81345383280937\\
4445	1.86712780340955\\
4446	1.89911390117386\\
4447	1.89159565700153\\
4448	1.81344855658169\\
4449	1.72682297899704\\
4450	1.72481604771903\\
4451	1.71935614221376\\
4452	1.70919850584017\\
4453	1.71116454181024\\
4454	1.75262711598611\\
4455	1.74152683904705\\
4456	1.75140940000694\\
4457	1.7664738419082\\
4458	1.80644016836022\\
4459	1.82056468696452\\
4460	1.73554379043239\\
4461	1.70088990979367\\
4462	1.6391046229217\\
4463	1.68387426488656\\
4464	1.74015729387358\\
4465	1.82618560642051\\
4466	1.89788463572942\\
4467	1.96911901879963\\
4468	2.06242203825983\\
4469	2.10460953356121\\
4470	2.07989674249205\\
4471	2.01015472071837\\
4472	1.91898202071455\\
4473	1.85366613666493\\
4474	1.79303880888077\\
4475	1.8115170474606\\
4476	1.77723305597322\\
4477	1.77420648400289\\
4478	1.83198703343779\\
4479	1.83157906652971\\
4480	1.77111175630129\\
4481	1.71601188032177\\
4482	1.63568537282106\\
4483	1.60292933405205\\
4484	1.63823393067935\\
4485	1.63616405576641\\
4486	1.64214708472491\\
4487	1.67968903212644\\
4488	1.85086342053808\\
4489	1.89213865348715\\
4490	1.93079290992442\\
4491	1.94097741474605\\
4492	1.90153297139307\\
4493	1.80962426593558\\
4494	1.66034015945357\\
4495	1.48322529934938\\
4496	1.45676969024054\\
4497	1.44872628459886\\
4498	1.45742112137186\\
4499	1.44927901326722\\
4500	1.48253062990321\\
4501	1.46266001076143\\
4502	1.44232624143255\\
4503	1.43095134873069\\
4504	1.39703991266497\\
4505	1.41541745428682\\
4506	1.38118832530763\\
4507	1.3898779969399\\
4508	1.3626614495872\\
4509	1.3711925161759\\
4510	1.346548670668\\
4511	1.40229422129295\\
4512	1.51908662170931\\
4513	1.56949102626469\\
4514	1.62673430339605\\
4515	1.65522774760241\\
4516	1.65453580538204\\
4517	1.60999520846384\\
4518	1.50902906260639\\
4519	1.39323995463441\\
4520	1.30075678956672\\
4521	1.25914719882269\\
4522	1.24204335334177\\
4523	1.22705456276675\\
4524	1.25434556855222\\
4525	1.25624565363706\\
4526	1.25845731507884\\
4527	1.27481015404865\\
4528	1.28916955303821\\
4529	1.27474672671309\\
4530	1.28876295501267\\
4531	1.31620092247046\\
4532	1.33875450324422\\
4533	1.36370401235384\\
4534	1.33986278526885\\
4535	1.38578034642895\\
4536	1.51576137808137\\
4537	1.59773572525181\\
4538	1.65084962291435\\
4539	1.69944063584339\\
4540	1.73141648004091\\
4541	1.70811144672443\\
4542	1.5989073885938\\
4543	1.47890520288579\\
4544	1.40163689680588\\
4545	1.38218241204369\\
4546	1.34398461819648\\
4547	1.30551437190779\\
4548	1.3189031496298\\
4549	1.32542099030167\\
4550	1.33210588862896\\
4551	1.34559537679521\\
4552	1.35219570478913\\
4553	1.35782837680865\\
4554	1.37188077595834\\
4555	1.43414803589739\\
4556	1.48657104940747\\
4557	1.49166120453397\\
4558	1.47427716095113\\
4559	1.53326563493965\\
4560	1.66026418908893\\
4561	1.74973464018032\\
4562	1.81410121529767\\
4563	1.83728822992848\\
4564	1.80951872470068\\
4565	1.74328870369438\\
4566	1.60808549317835\\
4567	1.49231831445394\\
4568	1.39357565458146\\
4569	1.3733343406562\\
4570	1.39607238352986\\
4571	1.39414024193057\\
4572	1.35756491810609\\
4573	1.37540724405425\\
4574	1.39858145675128\\
4575	1.41149710352945\\
4576	1.42380046459854\\
4577	1.39748082007369\\
4578	1.39476678655035\\
4579	1.40678593059262\\
4580	1.46203329074518\\
4581	1.43009229546106\\
4582	1.40716270325324\\
4583	1.45982886349861\\
4584	1.57076282773402\\
4585	1.64901028246195\\
4586	1.67704434967407\\
4587	1.68945641370108\\
4588	1.69143367106243\\
4589	1.64202550255197\\
4590	1.53348693424498\\
4591	1.43442636352814\\
4592	1.43777076953143\\
4593	1.45609350827763\\
4594	1.43578341500415\\
4595	1.34356191608332\\
4596	1.3517702214149\\
4597	1.38796859423923\\
4598	1.41721169923852\\
4599	1.43955606133263\\
4600	1.44328519536852\\
4601	1.41248836347422\\
4602	1.41099903478099\\
4603	1.44706819654924\\
4604	1.42452859406177\\
4605	1.43363840147953\\
4606	1.41099990155249\\
4607	1.4164633150049\\
4608	1.48959269632043\\
4609	1.56482829603809\\
4610	1.63321459409466\\
4611	1.65867567202244\\
4612	1.68885727694936\\
4613	1.71107093472958\\
4614	1.71447633192934\\
4615	1.64426748601285\\
4616	1.55598306130484\\
4617	1.49471001728977\\
4618	1.46140184814382\\
4619	1.46756423983987\\
4620	1.4825638511596\\
4621	1.53004746963212\\
4622	1.57171399952105\\
4623	1.61232305970308\\
4624	1.5937741016283\\
4625	1.54667022151225\\
4626	1.51328147473042\\
4627	1.4789564989673\\
4628	1.49766778947659\\
4629	1.49874646914091\\
4630	1.46692690083988\\
4631	1.51917442172625\\
4632	1.59643110658413\\
4633	1.68205665066274\\
4634	1.76562988138068\\
4635	1.82139779965719\\
4636	1.85680281667358\\
4637	1.88634762553177\\
4638	1.90810067247642\\
4639	1.85109102391815\\
4640	1.78710531015798\\
4641	1.70537669035585\\
4642	1.64310502112223\\
4643	1.60752547456017\\
4644	1.60950587106563\\
4645	1.69854894657558\\
4646	1.75441749964696\\
4647	1.7927196285187\\
4648	1.78622703592402\\
4649	1.72021499991657\\
4650	1.63249995086164\\
4651	1.57525568025951\\
4652	1.55943295739602\\
4653	1.55590215686009\\
4654	1.5337712576679\\
4655	1.54670991201498\\
4656	1.62824688657946\\
4657	1.80015002456493\\
4658	1.86067167656401\\
4659	1.86658151007515\\
4660	1.83526076366312\\
4661	1.758554196914\\
4662	1.63388134749322\\
4663	1.49312248466233\\
4664	1.4159921309297\\
4665	1.43717093063491\\
4666	1.40105205718623\\
4667	1.38121887229992\\
4668	1.379420586745\\
4669	1.41736641358679\\
4670	1.44220858110202\\
4671	1.47427231216578\\
4672	1.4463027784979\\
4673	1.41411635440644\\
4674	1.39281215045093\\
4675	1.42045892611969\\
4676	1.43197511380951\\
4677	1.45198940349357\\
4678	1.41051716632336\\
4679	1.44570208590064\\
4680	1.56441372331271\\
4681	1.64520619050764\\
4682	1.71998946591133\\
4683	1.73928363703857\\
4684	1.73862636926446\\
4685	1.71448323193354\\
4686	1.60099615075696\\
4687	1.52544375290271\\
4688	1.47758088760188\\
4689	1.45281651911276\\
4690	1.39621914128058\\
4691	1.36065591674509\\
4692	1.35367553326594\\
4693	1.36683856624594\\
4694	1.37709943143013\\
4695	1.3701975618758\\
4696	1.37400036009692\\
4697	1.34351682421745\\
4698	1.33653357345793\\
4699	1.38730797561503\\
4700	1.39614313272518\\
4701	1.38567934719607\\
4702	1.3537124141453\\
4703	1.32735883451147\\
4704	1.41632498789465\\
4705	1.47579846449675\\
4706	1.53419652670055\\
4707	1.55052427503633\\
4708	1.54491365307537\\
4709	1.49543251762708\\
4710	1.41997404107156\\
4711	1.31657633738457\\
4712	1.26389757590045\\
4713	1.25966400963326\\
4714	1.27244325770386\\
4715	1.29949523023552\\
4716	1.35503269895928\\
4717	1.37840477525719\\
4718	1.40393192892372\\
4719	1.37865827204192\\
4720	1.38819032009804\\
4721	1.36177115447024\\
4722	1.31702585062606\\
4723	1.33288011375792\\
4724	1.38944016745084\\
4725	1.38917997143364\\
4726	1.33370630894813\\
4727	1.37348820032172\\
4728	1.45637121478727\\
4729	1.54325492971439\\
4730	1.5860351347298\\
4731	1.60344269108749\\
4732	1.60241388249113\\
4733	1.55757492769731\\
4734	1.47908198576506\\
4735	1.39840907533089\\
4736	1.33405975485323\\
4737	1.35872118709408\\
4738	1.37368454933541\\
4739	1.38699198157114\\
4740	1.39437850851295\\
4741	1.42694219815011\\
4742	1.4358988985985\\
4743	1.41664217601781\\
4744	1.40526567403826\\
4745	1.36460645055754\\
4746	1.34707465386839\\
4747	1.3662306970077\\
4748	1.41009324785018\\
4749	1.40746891907867\\
4750	1.37287335755327\\
4751	1.44316172079412\\
4752	1.5367839624099\\
4753	1.6454164712439\\
4754	1.70175677134148\\
4755	1.71687733837094\\
4756	1.71489152994698\\
4757	1.66131631364338\\
4758	1.57673992238283\\
4759	1.48638285845249\\
4760	1.41529170963782\\
4761	1.42928132490828\\
4762	1.436169521318\\
4763	1.4079416405844\\
4764	1.49510271864436\\
4765	1.56972322470649\\
4766	1.57365237071526\\
4767	1.54420650195928\\
4768	1.51722272960743\\
4769	1.46358228578607\\
4770	1.44037034043947\\
4771	1.41449613677074\\
4772	1.4220252268564\\
4773	1.43262990789637\\
4774	1.39613239688236\\
4775	1.46160725966154\\
4776	1.55076214797843\\
4777	1.71113143705093\\
4778	1.80692979979074\\
4779	1.88286163290274\\
4780	1.89838451304531\\
4781	1.95803987751944\\
4782	1.98463940441448\\
4783	1.9316290137851\\
4784	1.88483927324364\\
4785	1.8828981461269\\
4786	1.84872705952539\\
4787	1.96785469738677\\
4788	2.00959217780517\\
4789	1.99696822998982\\
4790	2.01802723008759\\
4791	1.97163969996087\\
4792	1.86814419746603\\
4793	1.76161080696037\\
4794	1.64209334311521\\
4795	1.5830768789506\\
4796	1.56383980879565\\
4797	1.60328359816925\\
4798	1.61911437800292\\
4799	1.61268820009824\\
4800	1.6835510361411\\
4801	1.7612807848468\\
4802	1.82229998084255\\
4803	1.8409248704038\\
4804	1.86778733933902\\
4805	1.87055365313631\\
4806	1.90265920113458\\
4807	1.86220440035996\\
4808	1.84146216677447\\
4809	1.76736351882109\\
4810	1.73358058090014\\
4811	1.74994800436423\\
4812	1.83399759788768\\
4813	1.83020386314253\\
4814	1.86015127329546\\
4815	1.90747294279015\\
4816	1.89961049238829\\
4817	1.88669371902829\\
4818	1.77882847063897\\
4819	1.70620640241822\\
4820	1.67787494940387\\
4821	1.66948884160969\\
4822	1.64974644937316\\
4823	1.68528977251559\\
4824	1.76547050375868\\
4825	1.85371676425487\\
4826	1.91164660786981\\
4827	1.96169788708705\\
4828	1.98443216780835\\
4829	2.01042628229444\\
4830	2.0291803130122\\
4831	1.96268752508897\\
4832	1.95555077393018\\
4833	1.87744129479487\\
4834	1.82268453042181\\
4835	1.85740629918543\\
4836	1.86417469450505\\
4837	1.81862010501414\\
4838	1.86030961873876\\
4839	1.89485246827313\\
4840	1.90107418324606\\
4841	1.86574660751424\\
4842	1.80909577564819\\
4843	1.79614900925671\\
4844	1.78708631690126\\
4845	1.77774472343373\\
4846	1.74443404188909\\
4847	1.79909011978581\\
4848	1.905945297287\\
4849	2.00254192002816\\
4850	2.07444399903042\\
4851	2.11371501523038\\
4852	2.10876263914645\\
4853	2.01190351177934\\
4854	1.86652443172401\\
4855	1.70962054720054\\
4856	1.59960197208602\\
4857	1.5533109549666\\
4858	1.55666290725773\\
4859	1.56466343196595\\
4860	1.62347457918826\\
4861	1.66336974400224\\
4862	1.70007192927527\\
4863	1.72744409033513\\
4864	1.73463252445493\\
4865	1.63876537071008\\
4866	1.61406687849377\\
4867	1.59339160555086\\
4868	1.57454075451458\\
4869	1.5977752908448\\
4870	1.57013228994685\\
4871	1.59494133446264\\
4872	1.72744498833086\\
4873	1.81880330211071\\
4874	1.88289572349143\\
4875	1.88381917203693\\
4876	1.88413072412988\\
4877	1.82725561358592\\
4878	1.76391936942108\\
4879	1.62500287078914\\
4880	1.56160487694797\\
4881	1.56853141219304\\
4882	1.59004603994113\\
4883	1.61699074762484\\
4884	1.69618214621117\\
4885	1.70980129877867\\
4886	1.72364875130709\\
4887	1.70196836667236\\
4888	1.67992038484474\\
4889	1.64887251180275\\
4890	1.62341148980498\\
4891	1.61331082972964\\
4892	1.60901041745858\\
4893	1.6167888230751\\
4894	1.55846212306039\\
4895	1.60484565626097\\
4896	1.71848761219632\\
4897	1.81964169044996\\
4898	1.86259764681552\\
4899	1.87348395449316\\
4900	1.8486400472033\\
4901	1.80918121993009\\
4902	1.7159125307324\\
4903	1.58933785249964\\
4904	1.57539940516794\\
4905	1.6008950267197\\
4906	1.64448830657587\\
4907	1.66553026417196\\
4908	1.66102770693597\\
4909	1.69252862786778\\
4910	1.71373967047149\\
4911	1.71218786438991\\
4912	1.68208041113866\\
4913	1.63055307356014\\
4914	1.60345581232375\\
4915	1.60251020065602\\
4916	1.58834678928129\\
4917	1.57791460829857\\
4918	1.53825648636354\\
4919	1.57322272022105\\
4920	1.67313177336046\\
4921	1.7476778886419\\
4922	1.80357778950498\\
4923	1.8106174509066\\
4924	1.79506213672045\\
4925	1.75795560184864\\
4926	1.68637131227605\\
4927	1.62447745270393\\
4928	1.53127677518992\\
4929	1.53925109264216\\
4930	1.56063915451773\\
4931	1.57454007572243\\
4932	1.59184305316604\\
4933	1.59948669322999\\
4934	1.59971185235114\\
4935	1.61113527883714\\
4936	1.62112318995445\\
4937	1.61703895491777\\
4938	1.59367069216149\\
4939	1.58035644530255\\
4940	1.59262165839242\\
4941	1.58312168194018\\
4942	1.53903304680397\\
4943	1.60274772170575\\
4944	1.64911441524974\\
4945	1.7294289797867\\
4946	1.79796213842512\\
4947	1.81985052470703\\
4948	1.83681941723779\\
4949	1.84434999235046\\
4950	1.83840910137275\\
4951	1.77451277725081\\
4952	1.71896618126226\\
4953	1.66561937010653\\
4954	1.65600660459341\\
4955	1.65837546966231\\
4956	1.68535008702192\\
4957	1.72959621294691\\
4958	1.78384264688621\\
4959	1.7898996791458\\
4960	1.75857664710909\\
4961	1.69549324553552\\
4962	1.67630774439399\\
4963	1.63080625522925\\
4964	1.61860175928722\\
4965	1.61812386986629\\
4966	1.59020053971504\\
4967	1.62273209933302\\
4968	1.73172778380117\\
4969	1.79250251454019\\
4970	1.84765893362585\\
4971	1.88667572673274\\
4972	1.90966192632447\\
4973	1.89946962366138\\
4974	1.93171208021666\\
4975	1.88451749837483\\
4976	1.84799288624546\\
4977	1.82676415740296\\
4978	1.88176883690491\\
4979	1.9114559227512\\
4980	1.96439988250826\\
4981	2.0134450101113\\
4982	2.03685617223114\\
4983	2.02117428741974\\
4984	1.96159526698572\\
4985	1.85282989629305\\
4986	1.75235608781471\\
4987	1.66547847190056\\
4988	1.62968739978369\\
4989	1.60979722249612\\
4990	1.61731847002928\\
4991	1.65104421276981\\
4992	1.7166140394037\\
4993	1.77290814718587\\
4994	1.83086023545586\\
4995	1.85256267079421\\
4996	1.83476126083532\\
4997	1.7708657720389\\
4998	1.68846573948948\\
4999	1.58268012990512\\
5000	1.51692483792136\\
5001	1.4867689134258\\
5002	1.48082339913355\\
5003	1.48491734050944\\
5004	1.48656367523982\\
5005	1.5165270779851\\
5006	1.53479128015968\\
5007	1.55405516917485\\
5008	1.54591603977543\\
5009	1.51626416234671\\
5010	1.51873866376434\\
5011	1.56820193470871\\
5012	1.60033548418052\\
5013	1.61153750555251\\
5014	1.55051441309273\\
5015	1.63284546290318\\
5016	1.73161353122053\\
5017	1.8115714429989\\
5018	1.88663941310593\\
5019	1.90138271338261\\
5020	1.89523654531382\\
5021	1.83886058890445\\
5022	1.75810156451926\\
5023	1.65386977402118\\
5024	1.58363824714227\\
5025	1.61054422338444\\
5026	1.61820325095725\\
5027	1.61809156963678\\
5028	1.59822428142478\\
5029	1.62602384566583\\
5030	1.65797821105665\\
5031	1.67409499720823\\
5032	1.65487478905342\\
5033	1.60424140757313\\
5034	1.58385799538476\\
5035	1.59566172634704\\
5036	1.58107770666115\\
5037	1.58315851178155\\
5038	1.50773612723059\\
5039	1.57338895937081\\
5040	1.66757852717405\\
5041	1.73605181008837\\
5042	1.78926302476676\\
5043	1.81385147630244\\
5044	1.81714886431123\\
5045	1.77389946741239\\
5046	1.68559943745563\\
5047	1.57775611467238\\
5048	1.53408596122801\\
5049	1.55419071098099\\
5050	1.57374051980781\\
5051	1.59108689728736\\
5052	1.63258477886812\\
5053	1.65343020470828\\
5054	1.63720708520389\\
5055	1.62731791701827\\
5056	1.59598030319354\\
5057	1.54797866157849\\
5058	1.52501544343363\\
5059	1.52810935058777\\
5060	1.54040161133377\\
5061	1.55433938742125\\
5062	1.48162796157517\\
5063	1.57710233351075\\
5064	1.65633597826658\\
5065	1.74967760847099\\
5066	1.81280157818883\\
5067	1.84479626036456\\
5068	1.83051299987061\\
5069	1.78151888702967\\
5070	1.70081070096294\\
5071	1.58915840467819\\
5072	1.55371143179058\\
5073	1.55720093785001\\
5074	1.54160679574318\\
5075	1.5798640425226\\
5076	1.57797769900633\\
5077	1.60155006491717\\
5078	1.60781128641772\\
5079	1.60918592605525\\
5080	1.60020241066137\\
5081	1.60950851383479\\
5082	1.58003059675887\\
5083	1.58244441940473\\
5084	1.54937033730299\\
5085	1.54080672416639\\
5086	1.47777251373869\\
5087	1.56573471745639\\
5088	1.67376730318411\\
5089	1.74734826794991\\
5090	1.80316304759763\\
5091	1.81075107951983\\
5092	1.81920290373721\\
5093	1.76704775578893\\
5094	1.6872228453427\\
5095	1.57078783391539\\
5096	1.52792117185262\\
5097	1.55356740505943\\
5098	1.58575156778369\\
5099	1.63122357907021\\
5100	1.61838305709686\\
5101	1.63282691505675\\
5102	1.64717377608998\\
5103	1.63528336405328\\
5104	1.62734605726679\\
5105	1.58223498990548\\
5106	1.54462242122231\\
5107	1.52383532445451\\
5108	1.50714520842046\\
5109	1.51708310458728\\
5110	1.46890113334712\\
5111	1.53192625856298\\
5112	1.63014375188912\\
5113	1.70782700355481\\
5114	1.78572825605775\\
5115	1.84820045665323\\
5116	1.87760196749425\\
5117	1.88900060069232\\
5118	1.90579329997623\\
5119	1.8414552400793\\
5120	1.7602877270298\\
5121	1.71969329404051\\
5122	1.69483311578048\\
5123	1.68425881146882\\
5124	1.73071569784826\\
5125	1.74355719873472\\
5126	1.77856662515523\\
5127	1.74978479329568\\
5128	1.69096680071591\\
5129	1.62202069139607\\
5130	1.59635912547937\\
5131	1.59553773707085\\
5132	1.57030763268662\\
5133	1.56497115459285\\
5134	1.55809580496493\\
5135	1.61074479066161\\
5136	1.73734458827019\\
5137	1.82502577365379\\
5138	1.90396230228593\\
5139	1.93556170413346\\
5140	1.93900888593752\\
5141	1.93485378900737\\
5142	1.93870669227527\\
5143	1.89789308580299\\
5144	1.85210376287785\\
5145	1.88741557551948\\
5146	1.89308855224634\\
5147	1.90130613593707\\
5148	1.94403826309424\\
5149	2.02408397691402\\
5150	2.08274590293557\\
5151	2.08997143586383\\
5152	2.02954623931267\\
5153	1.92223915227928\\
5154	1.86717796399558\\
5155	1.76341568771223\\
5156	1.73276232162147\\
5157	1.74623382952743\\
5158	1.71870825328518\\
5159	1.77540407154056\\
5160	1.77514182658803\\
5161	1.84160606562223\\
5162	1.90191231805711\\
5163	1.91458906251578\\
5164	1.89212618992821\\
5165	1.80851442392841\\
5166	1.63040586827657\\
5167	1.53170629277662\\
5168	1.44267265534897\\
5169	1.41583933595077\\
5170	1.41751349899803\\
5171	1.45773185081765\\
5172	1.47580528765776\\
5173	1.52193617103804\\
5174	1.55810706303978\\
5175	1.56043556540382\\
5176	1.55725465195629\\
5177	1.5072182374253\\
5178	1.52077556169586\\
5179	1.50515625634747\\
5180	1.47830295546505\\
5181	1.47459773941874\\
5182	1.42633000886513\\
5183	1.50894137781055\\
5184	1.62274076310759\\
5185	1.69801185877577\\
5186	1.74880952266386\\
5187	1.77398741817662\\
5188	1.77601374373296\\
5189	1.71592068516447\\
5190	1.61416614154227\\
5191	1.50302543294371\\
5192	1.47718305484837\\
5193	1.4982703868072\\
5194	1.51744582661318\\
5195	1.55902534663892\\
5196	1.45401256465141\\
5197	1.46088106454156\\
5198	1.48345262545796\\
5199	1.47868671455192\\
5200	1.4452289492623\\
5201	1.39130027189605\\
5202	1.35498253973797\\
5203	1.33563073129264\\
5204	1.33843077741259\\
5205	1.32990050309456\\
5206	1.29594162689493\\
5207	1.37998028538411\\
5208	1.50931882402675\\
5209	1.58680840816009\\
5210	1.65598476044806\\
5211	1.67629046453695\\
5212	1.70248968516289\\
5213	1.64742319156116\\
5214	1.57341055106209\\
5215	1.47813585156727\\
5216	1.41288527457515\\
5217	1.40911653878314\\
5218	1.41218156137593\\
5219	1.41100737218043\\
5220	1.44147498107671\\
5221	1.45788734030126\\
5222	1.47189193635809\\
5223	1.44062433100851\\
5224	1.40938804083369\\
5225	1.38094763711316\\
5226	1.37807359820861\\
5227	1.4088191845344\\
5228	1.42594450663982\\
5229	1.42303374286449\\
5230	1.39577929791604\\
5231	1.47416928194715\\
5232	1.54607456866111\\
5233	1.61853287878565\\
5234	1.6597083530823\\
5235	1.66398410347946\\
5236	1.64110545788209\\
5237	1.57235570498506\\
5238	1.48756502800201\\
5239	1.38894035402197\\
5240	1.34817971626027\\
5241	1.27784859919047\\
5242	1.29449558505161\\
5243	1.31513916046334\\
5244	1.31519956871601\\
5245	1.32989959811272\\
5246	1.34304371648652\\
5247	1.35606821483396\\
5248	1.36840808778791\\
5249	1.36026057765293\\
5250	1.34531113652791\\
5251	1.37067186544196\\
5252	1.33394425717413\\
5253	1.32412324279651\\
5254	1.31609459044885\\
5255	1.38152678354068\\
5256	1.45710224135249\\
5257	1.5372800707245\\
5258	1.58846422888717\\
5259	1.61401269979424\\
5260	1.61227001863672\\
5261	1.55394062218637\\
5262	1.47137860616508\\
5263	1.35616052354667\\
5264	1.30441055706259\\
5265	1.28624112563431\\
5266	1.30870565903884\\
5267	1.30912789195062\\
5268	1.32655439890589\\
5269	1.32156378749173\\
5270	1.33743493343918\\
5271	1.31385326192454\\
5272	1.30910422411239\\
5273	1.3310897619535\\
5274	1.33886382177027\\
5275	1.34099752573996\\
5276	1.37947404027159\\
5277	1.41050511575643\\
5278	1.41737739296711\\
5279	1.48415213543186\\
5280	1.60159451357936\\
5281	1.69594425879808\\
5282	1.76807540132066\\
5283	1.83403835894509\\
5284	1.85823354886763\\
5285	1.89578095518565\\
5286	1.87909598015072\\
5287	1.80924484423009\\
5288	1.71893845482719\\
5289	1.67369711565072\\
5290	1.66643139323896\\
5291	1.67139625902726\\
5292	1.75214174896658\\
5293	1.8366759049439\\
5294	1.83611447246856\\
5295	1.83304899897076\\
5296	1.78551166138504\\
5297	1.7021761780697\\
5298	1.62347881545896\\
5299	1.55206499068132\\
5300	1.58311649278145\\
5301	1.54909513929099\\
5302	1.51462197643426\\
5303	1.51848816054046\\
5304	1.61108557304273\\
5305	1.67485941268947\\
5306	1.74688930847379\\
5307	1.80713574042204\\
5308	1.81725705781646\\
5309	1.81353199867003\\
5310	1.86318941114022\\
5311	1.8632856567444\\
5312	1.81294109976004\\
5313	1.75042213253978\\
5314	1.71030391194284\\
5315	1.66407519997308\\
5316	1.6832142130594\\
5317	1.818208098987\\
5318	1.88189849297426\\
5319	1.94131938085684\\
5320	1.91007458231898\\
5321	1.85254388037823\\
5322	1.79366052196478\\
5323	1.77941849353858\\
5324	1.78026288096084\\
5325	1.7292115625666\\
5326	1.69676847016518\\
5327	1.76486022132411\\
5328	1.95634149695228\\
5329	2.03692559561564\\
5330	2.11693667105648\\
5331	2.11710379948344\\
5332	2.06115219033984\\
5333	1.92594419867381\\
5334	1.78099130667032\\
5335	1.62646104925123\\
5336	1.53013562995122\\
5337	1.56810068217786\\
5338	1.60095733737027\\
5339	1.60538389946524\\
5340	1.61893464266262\\
5341	1.62766999330259\\
5342	1.6358129796159\\
5343	1.63065145225729\\
5344	1.641455943792\\
5345	1.62123280486859\\
5346	1.57084192021961\\
5347	1.53956801892375\\
5348	1.53658878147442\\
5349	1.55277474148025\\
5350	1.50659371302244\\
5351	1.58471345906682\\
5352	1.70167542495962\\
5353	1.79191357455511\\
5354	1.83644959050952\\
5355	1.85421281616043\\
5356	1.83217155420689\\
5357	1.74451211216487\\
5358	1.62358719994193\\
5359	1.51868096832261\\
5360	1.48814251124298\\
5361	1.50002527210344\\
5362	1.51471096795269\\
5363	1.4950917900688\\
5364	1.54378742563379\\
5365	1.5435824246366\\
5366	1.57118218572204\\
5367	1.55295935427411\\
5368	1.51834499558395\\
5369	1.47272744480672\\
5370	1.42406740199846\\
5371	1.448154765647\\
5372	1.44602714915264\\
5373	1.43174062849104\\
5374	1.4034338730743\\
5375	1.49903291678204\\
5376	1.59534023401427\\
5377	1.6644642841554\\
5378	1.72031217376165\\
5379	1.73916179486891\\
5380	1.71252589842983\\
5381	1.63332238379239\\
5382	1.52610772464493\\
5383	1.43963827649943\\
5384	1.35418647296709\\
5385	1.35053504745947\\
5386	1.3425303174889\\
5387	1.33048171111976\\
5388	1.36659515084737\\
5389	1.38706680115272\\
5390	1.40272747641675\\
5391	1.41256725429627\\
5392	1.4154640095923\\
5393	1.40744563683002\\
5394	1.42411605783239\\
5395	1.38339276404149\\
5396	1.39176403141773\\
5397	1.38364230892908\\
5398	1.3637329599118\\
5399	1.4525950794569\\
5400	1.53465429083845\\
5401	1.60691282498894\\
5402	1.63385598665713\\
5403	1.65241461495679\\
5404	1.63599171935179\\
5405	1.56187399504891\\
5406	1.45679454600768\\
5407	1.34574494252237\\
5408	1.28710181256362\\
5409	1.26521994767664\\
5410	1.22338231673277\\
5411	1.22249590992553\\
5412	1.25735839657224\\
5413	1.27444494492442\\
5414	1.29331828632878\\
5415	1.29473945187724\\
5416	1.2901334071982\\
5417	1.28962633886417\\
5418	1.26046160086775\\
5419	1.28576299015438\\
5420	1.30156549689527\\
5421	1.29427749513685\\
5422	1.30887320185752\\
5423	1.35016341010438\\
5424	1.46701707210354\\
5425	1.54338391829028\\
5426	1.59222482495475\\
5427	1.61039715570794\\
5428	1.63406339534216\\
5429	1.63303567945935\\
5430	1.65047474883555\\
5431	1.63233513503368\\
5432	1.60151940501311\\
5433	1.5723367716938\\
5434	1.55246457136426\\
5435	1.5712670201461\\
5436	1.60357181727302\\
5437	1.6563657720104\\
5438	1.71807550126526\\
5439	1.74880129061556\\
5440	1.76208201506812\\
5441	1.70082676582338\\
5442	1.63205192582561\\
5443	1.57817281266107\\
5444	1.53421070422567\\
5445	1.48389533148456\\
5446	1.44063424859629\\
5447	1.50749066291435\\
5448	1.6060610934891\\
5449	1.68114943678217\\
5450	1.7598631561066\\
5451	1.78507259088368\\
5452	1.77504040396323\\
5453	1.73319460741223\\
5454	1.70486587290205\\
5455	1.63330124887663\\
5456	1.55137835780258\\
5457	1.52906235274967\\
5458	1.52086478730684\\
5459	1.53587181314355\\
5460	1.56823956965305\\
5461	1.62380505048316\\
5462	1.67558406565666\\
5463	1.68965017226418\\
5464	1.66229588603983\\
5465	1.60881942893149\\
5466	1.53112434107205\\
5467	1.51043805895321\\
5468	1.5698492112396\\
5469	1.56864764387915\\
5470	1.54261893939668\\
5471	1.59834994022549\\
5472	1.73936954069078\\
5473	1.84381862162982\\
5474	1.9230401475829\\
5475	1.98490142772909\\
5476	2.0207098743116\\
5477	1.97574382470606\\
5478	2.00037331938698\\
5479	1.97304086330386\\
5480	1.92610628910919\\
5481	1.84615843393231\\
5482	1.83181542035584\\
5483	1.8263522298329\\
5484	1.82657875690754\\
5485	1.8873455631738\\
5486	1.92261359116176\\
5487	1.92863984331248\\
5488	1.89612435575911\\
5489	1.80872329126979\\
5490	1.72831106567526\\
5491	1.64314957112438\\
5492	1.6392542606851\\
5493	1.63015755757443\\
5494	1.64203660446223\\
5495	1.66339736132379\\
5496	1.8089730536639\\
5497	1.88690833254568\\
5498	1.929683763976\\
5499	1.9475791810379\\
5500	1.9270507122255\\
5501	1.79389649811207\\
5502	1.59881153854887\\
5503	1.47420432205644\\
5504	1.41173133137598\\
5505	1.4474236323701\\
5506	1.44228319495421\\
5507	1.43328493734761\\
5508	1.44439265379685\\
5509	1.44127383534997\\
5510	1.43935390793754\\
5511	1.40993935582571\\
5512	1.38713995579126\\
5513	1.32821012306132\\
5514	1.29955847955\\
5515	1.26806385462351\\
5516	1.26088106274925\\
5517	1.26858422294256\\
5518	1.26385353440064\\
5519	1.2980479566834\\
5520	1.39787354837957\\
5521	1.47625874299585\\
5522	1.52820040443716\\
5523	1.55524581819548\\
5524	1.55334182054214\\
5525	1.50031619220396\\
5526	1.38522649285232\\
5527	1.32629965749529\\
5528	1.2617783623159\\
5529	1.25078305474792\\
5530	1.2657068082433\\
5531	1.28207578567975\\
5532	1.31392872827748\\
5533	1.37707578332813\\
5534	1.35208132529964\\
5535	1.36415605853591\\
5536	1.35123203830619\\
5537	1.33983933643455\\
5538	1.26604563445237\\
5539	1.25547986011411\\
5540	1.23265588858431\\
5541	1.24827742220998\\
5542	1.24568815367616\\
5543	1.30241909461925\\
5544	1.36699408418879\\
5545	1.43292201656171\\
5546	1.48468065253771\\
5547	1.49748764087071\\
5548	1.493385466564\\
5549	1.41570133148861\\
5550	1.30405988640266\\
5551	1.22353343181764\\
5552	1.21859330773147\\
5553	1.21397605226448\\
5554	1.23706142686088\\
5555	1.24160134701325\\
5556	1.28637985789731\\
5557	1.28883063038037\\
5558	1.31048969653655\\
5559	1.28411528000547\\
5560	1.25989638148022\\
5561	1.21097944321349\\
5562	1.21921393275501\\
5563	1.21531651580521\\
5564	1.21357382965277\\
5565	1.18016151959555\\
5566	1.18756528926452\\
5567	1.25452761629584\\
5568	1.36015967609146\\
5569	1.42706333086089\\
5570	1.45831613942052\\
5571	1.48401446226914\\
5572	1.47145838993584\\
5573	1.42022403900495\\
5574	1.30662743064457\\
5575	1.22454813323655\\
5576	1.20045939304318\\
5577	1.25763646414477\\
5578	1.28241092387083\\
5579	1.28502988156895\\
5580	1.26523675758345\\
5581	1.29438533410845\\
5582	1.26653911991705\\
5583	1.27604981480897\\
5584	1.2751015275153\\
5585	1.28070628434447\\
5586	1.2282199219755\\
5587	1.25995562996603\\
5588	1.26609833675301\\
5589	1.29395654703754\\
5590	1.29534675930638\\
5591	1.33784159044315\\
5592	1.47212816508155\\
5593	1.54226359899258\\
5594	1.58749114269617\\
5595	1.62995292674642\\
5596	1.62095490684671\\
5597	1.53351055331166\\
5598	1.41372736192254\\
5599	1.35533900212133\\
5600	1.26603055245063\\
5601	1.30755719920688\\
5602	1.30603672996378\\
5603	1.29118846559368\\
5604	1.3167957260591\\
5605	1.28543304761577\\
5606	1.30418705045692\\
5607	1.3026572527409\\
5608	1.307465315595\\
5609	1.29866740907259\\
5610	1.29695946526879\\
5611	1.27567837986484\\
5612	1.24435184214656\\
5613	1.22886152215657\\
5614	1.22624536992362\\
5615	1.2840601358095\\
5616	1.39361771579369\\
5617	1.45982176849056\\
5618	1.39801177216908\\
5619	1.43449938925321\\
5620	1.46108772208603\\
5621	1.46265335738634\\
5622	1.44448252385334\\
5623	1.41296669666264\\
5624	1.38073775340474\\
5625	1.42095400934462\\
5626	1.42193549484815\\
5627	1.4514484720637\\
5628	1.47474605235213\\
5629	1.48453048761654\\
5630	1.55044464676665\\
5631	1.55640259780404\\
5632	1.50931312794429\\
5633	1.43630996667747\\
5634	1.37101325279547\\
5635	1.32010863107557\\
5636	1.27821193122439\\
5637	1.28696306696023\\
5638	1.31037389486524\\
5639	1.32563339576269\\
5640	1.40841548778639\\
5641	1.60504500409052\\
5642	1.71125782775489\\
5643	1.76004685608421\\
5644	1.76979131235766\\
5645	1.76470723138057\\
5646	1.72020998641363\\
5647	1.63358727834434\\
5648	1.61686176534449\\
5649	1.61289088773149\\
5650	1.62723555575142\\
5651	1.60721160973948\\
5652	1.62167374480168\\
5653	1.73456167743034\\
5654	1.81593543900601\\
5655	1.75317913874717\\
5656	1.72985374340393\\
5657	1.62205625260664\\
5658	1.49384782840842\\
5659	1.41770001662979\\
5660	1.37052119193058\\
5661	1.31681906463247\\
5662	1.34208518594155\\
5663	1.38182363289655\\
5664	1.48922218376806\\
5665	1.55407636243386\\
5666	1.59997772850005\\
5667	1.61082639187422\\
5668	1.59195972053319\\
5669	1.49558273931934\\
5670	1.35299787618642\\
5671	1.28770368716674\\
5672	1.2250055854912\\
5673	1.19620369386882\\
5674	1.18200286254244\\
5675	1.17113251289197\\
5676	1.15499172560057\\
5677	1.14842180188832\\
5678	1.14949266808446\\
5679	1.1372142454434\\
5680	1.12853806322252\\
5681	1.14061574324205\\
5682	1.14748550378589\\
5683	1.1823414532674\\
5684	1.20412657272146\\
5685	1.25465900420794\\
5686	1.26483047570847\\
5687	1.32516838618609\\
5688	1.41409906265847\\
5689	1.46775210072691\\
5690	1.51320459437839\\
5691	1.51054636219352\\
5692	1.48981826823772\\
5693	1.42817155634371\\
5694	1.29295243275999\\
5695	1.20641740825486\\
5696	1.16106747005629\\
5697	1.11434017174618\\
5698	1.10194914536474\\
5699	1.09736532930531\\
5700	1.12059801924251\\
5701	1.13034856795127\\
5702	1.14534234052611\\
5703	1.12469465654186\\
5704	1.1277398899236\\
5705	1.12483976674299\\
5706	1.15059214103803\\
5707	1.15933093324485\\
5708	1.18415526274676\\
5709	1.20216708595653\\
5710	1.22266215333283\\
5711	1.27557832292514\\
5712	1.39142884025304\\
5713	1.46998819146466\\
5714	1.51361466233164\\
5715	1.51879850114344\\
5716	1.49969661563402\\
5717	1.42502552974751\\
5718	1.29773198358998\\
5719	1.23467294420552\\
5720	1.17026404208852\\
5721	1.17698745942853\\
5722	1.20478908754291\\
5723	1.22449524960657\\
5724	1.24432587597307\\
5725	1.23982624892832\\
5726	1.24722414516958\\
5727	1.23753503882601\\
5728	1.22971316439363\\
5729	1.21919639367707\\
5730	1.17942736661767\\
5731	1.18415767407641\\
5732	1.20431160166579\\
5733	1.19604590154725\\
5734	1.19733361527127\\
5735	1.27759178524405\\
5736	1.39126004907849\\
5737	1.45686212090595\\
5738	1.51874624980784\\
5739	1.55252742222899\\
5740	1.54204021844437\\
5741	1.47441862799314\\
5742	1.3401970419681\\
5743	1.24826486651964\\
5744	1.19779157417811\\
5745	1.18454691491549\\
5746	1.17023765218167\\
5747	1.17110727012785\\
5748	1.19873527910845\\
5749	1.20296739703471\\
5750	1.21116334161797\\
5751	1.19375052617038\\
5752	1.20802076600923\\
5753	1.22297319114427\\
5754	1.20191039211255\\
5755	1.22139480617933\\
5756	1.23180192903231\\
5757	1.24622534193359\\
5758	1.27486434801798\\
5759	1.3376843091237\\
5760	1.44842229558077\\
5761	1.52453791509157\\
5762	1.5809775329307\\
5763	1.58897283172159\\
5764	1.57419868572254\\
5765	1.50264083035475\\
5766	1.36313985408534\\
5767	1.28041084101333\\
5768	1.23543293154148\\
5769	1.25706239198116\\
5770	1.28892914072334\\
5771	1.30699567796741\\
5772	1.33331000907527\\
5773	1.37344121539491\\
5774	1.38556095517996\\
5775	1.42977511474274\\
5776	1.42985517086575\\
5777	1.39787320229039\\
5778	1.35498583078723\\
5779	1.35588477449822\\
5780	1.35041598156146\\
5781	1.37608262429397\\
5782	1.37732730046673\\
5783	1.43087523554639\\
5784	1.57899557756904\\
5785	1.6517181921652\\
5786	1.59092565801811\\
5787	1.62774765936544\\
5788	1.64703012260402\\
5789	1.65405676786047\\
5790	1.60130870579257\\
5791	1.54582361190645\\
5792	1.46423660166304\\
5793	1.3781005850287\\
5794	1.32277298260219\\
5795	1.2926565201257\\
5796	1.30551433909178\\
5797	1.32308602621241\\
5798	1.33375316059592\\
5799	1.33424151896849\\
5800	1.32568880327332\\
5801	1.2820414833445\\
5802	1.25393490297106\\
5803	1.23974602084725\\
5804	1.26198972976862\\
5805	1.27866069066774\\
5806	1.31232566045382\\
5807	1.35040058887513\\
5808	1.42971626700813\\
5809	1.50579917981056\\
5810	1.55894276301035\\
5811	1.58576888105651\\
5812	1.5872827042239\\
5813	1.58438574239399\\
5814	1.56343109047493\\
5815	1.54560684339355\\
5816	1.46324900974792\\
5817	1.39234482821837\\
5818	1.35485851920503\\
5819	1.3287628335157\\
5820	1.35499376035937\\
5821	1.41532770671408\\
5822	1.50210036562348\\
5823	1.55670528481722\\
5824	1.54541576085679\\
5825	1.45307367049636\\
5826	1.36208460898171\\
5827	1.26443908826209\\
5828	1.267337748578\\
5829	1.26588035095846\\
5830	1.25296212696964\\
5831	1.29627469613704\\
5832	1.37954066203569\\
5833	1.43523422848389\\
5834	1.47195077473596\\
5835	1.47391719490485\\
5836	1.45448786215656\\
5837	1.36247730197042\\
5838	1.19656569126235\\
5839	1.11631788056405\\
5840	1.11136244123613\\
5841	1.13140099695223\\
5842	1.14470740689901\\
5843	1.15660102697161\\
5844	1.16539911701212\\
5845	1.1651216024041\\
5846	1.15654056786033\\
5847	1.14821710317504\\
5848	1.13468228610589\\
5849	1.11787106510119\\
5850	1.10391495610845\\
5851	1.08120069244967\\
5852	1.09948168633151\\
5853	1.11260845752486\\
5854	1.12036684355616\\
5855	1.18746772939355\\
5856	1.25693787343719\\
5857	1.31656775047032\\
5858	1.35822007813894\\
5859	1.37194552509346\\
5860	1.36056761068959\\
5861	1.29889106701967\\
5862	1.18203120086223\\
5863	1.08414306941227\\
5864	1.07700927025048\\
5865	1.09155187176332\\
5866	1.12648669596175\\
5867	1.13583658068826\\
5868	1.15758430339695\\
5869	1.16354850685126\\
5870	1.1596230287892\\
5871	1.16442807559557\\
5872	1.14733291422169\\
5873	1.12650849873158\\
5874	1.10960632103904\\
5875	1.11057383766464\\
5876	1.11044628641042\\
5877	1.10811693366603\\
5878	1.12187328085963\\
5879	1.18177823579072\\
5880	1.26391221428105\\
5881	1.31630228879579\\
5882	1.34893799697681\\
5883	1.3681620642827\\
5884	1.3458414812721\\
5885	1.28837921619505\\
5886	1.17473335316091\\
5887	1.08026358750502\\
5888	1.07559771877615\\
5889	1.09313862128104\\
5890	1.1191026130816\\
5891	1.14973327520784\\
5892	1.19156268073116\\
5893	1.21425800291035\\
5894	1.22991687507756\\
5895	1.22589865464954\\
5896	1.18745309985811\\
5897	1.14614640788419\\
5898	1.1348680929746\\
5899	1.13773304291437\\
5900	1.10131118578349\\
5901	1.10056741762111\\
5902	1.1384744124771\\
5903	1.24253504700937\\
5904	1.3058739313817\\
5905	1.35917322252037\\
5906	1.37024330973227\\
5907	1.37981721843537\\
5908	1.37247255899177\\
5909	1.30617275144677\\
5910	1.17930684913725\\
5911	1.07651353758528\\
5912	1.07456476284438\\
5913	1.08293092780262\\
5914	1.10162108974457\\
5915	1.11468061953916\\
5916	1.15836678216867\\
5917	1.16879001881938\\
5918	1.17871087902505\\
5919	1.16630708813726\\
5920	1.12636404805949\\
5921	1.10424215884954\\
5922	1.05847068200151\\
5923	1.05862476807818\\
5924	1.0435211169413\\
5925	1.06583050263021\\
5926	1.08775551475772\\
5927	1.16458799768259\\
5928	1.23547070162894\\
5929	1.28061690546641\\
5930	1.31796478025354\\
5931	1.32900004074685\\
5932	1.31582752334871\\
5933	1.27543839109679\\
5934	1.16273857711909\\
5935	1.06218355581919\\
5936	1.03022387452698\\
5937	1.02253738616966\\
5938	1.03970365329546\\
5939	1.05601974619787\\
5940	1.10180546878053\\
5941	1.1132680619926\\
5942	1.12398585528369\\
5943	1.13641874303321\\
5944	1.12207561363152\\
5945	1.10102923270691\\
5946	1.09174807632419\\
5947	1.11146812437987\\
5948	1.09617047926679\\
5949	1.10798960600717\\
5950	1.14577312650666\\
5951	1.15282352972269\\
5952	1.14937252999667\\
5953	1.22998859307845\\
5954	1.28636649569764\\
5955	1.30705328552898\\
5956	1.33260309517695\\
5957	1.32010918232318\\
5958	1.27044449245754\\
5959	1.24109621964143\\
5960	1.18015506863458\\
5961	1.14369354232299\\
5962	1.08634688933081\\
5963	1.06575715456617\\
5964	1.07898969503636\\
5965	1.11065768839045\\
5966	1.12391784592702\\
5967	1.137606959095\\
5968	1.12950818346217\\
5969	1.08163977827933\\
5970	1.09158872546733\\
5971	1.09060222864698\\
5972	1.06662591528974\\
5973	1.08727439935209\\
5974	1.09054434965721\\
5975	1.11805329530521\\
5976	1.15962480165253\\
5977	1.2006740370044\\
5978	1.21320624076844\\
5979	1.22438771886312\\
5980	1.2338737459194\\
5981	1.22629635890309\\
5982	1.19076560866288\\
5983	1.19855154216158\\
5984	1.16919392967301\\
5985	1.15702277685768\\
5986	1.14089629979018\\
5987	1.13640510453971\\
5988	1.16066070538113\\
5989	1.20587213577556\\
5990	1.23338370143521\\
5991	1.22357037280189\\
5992	1.26978755757116\\
5993	1.28673879213862\\
5994	1.21388307776201\\
5995	1.17066682182889\\
5996	1.1394785501454\\
5997	1.13889333366554\\
5998	1.15745863390384\\
5999	1.22926841923568\\
6000	1.32479479079691\\
6001	1.36478858247178\\
6002	1.37903759264649\\
6003	1.38747876493051\\
6004	1.34714189412515\\
6005	1.2717794789387\\
6006	1.12166962045505\\
6007	1.02835782266586\\
6008	1.01626684684203\\
6009	1.02271592357479\\
6010	1.02712649148635\\
6011	1.02616922826202\\
6012	1.06900017267024\\
6013	1.10174748485759\\
6014	1.11716098100827\\
6015	1.12415304688421\\
6016	1.09141509721763\\
6017	1.05000368320096\\
6018	1.0374995527732\\
6019	1.04046010329966\\
6020	1.06420219878623\\
6021	1.07885148383994\\
6022	1.1043960428336\\
6023	1.14735528107585\\
6024	1.22691682975854\\
6025	1.27956120900612\\
6026	1.31020875241832\\
6027	1.31661986529784\\
6028	1.31348174538791\\
6029	1.25032748965298\\
6030	1.12231456277701\\
6031	1.0573693651132\\
6032	1.05777487037285\\
6033	1.10030652177927\\
6034	1.07967636410607\\
6035	1.06651108704169\\
6036	1.11012940970004\\
6037	1.12126844307729\\
6038	1.11352020116095\\
6039	1.10954756534984\\
6040	1.08768542674534\\
6041	1.0524387552992\\
6042	1.03632052009409\\
6043	1.07342849763167\\
6044	1.06288498650172\\
6045	1.08745289994388\\
6046	1.09640671600792\\
6047	1.17134316839144\\
6048	1.24897424955856\\
6049	1.30015402267315\\
6050	1.33311624613131\\
6051	1.34558700865597\\
6052	1.336144079424\\
6053	1.27053655685253\\
6054	1.14257266731081\\
6055	1.04891996453425\\
6056	1.03364959432722\\
6057	1.06904519574322\\
6058	1.09213732937494\\
6059	1.10532546019099\\
6060	1.12869217742457\\
6061	1.14741905463951\\
6062	1.15889831361385\\
6063	1.14122796413349\\
6064	1.10993835110732\\
6065	1.11265603690942\\
6066	1.06254685589474\\
6067	1.08957638347858\\
6068	1.07864332670375\\
6069	1.06886800722926\\
6070	1.12384541352968\\
6071	1.17660654743931\\
6072	1.26057443816591\\
6073	1.31081636878636\\
6074	1.34745123278675\\
6075	1.35337381043442\\
6076	1.33039775506741\\
6077	1.26565224377484\\
6078	1.12236037875668\\
6079	1.06157945531746\\
6080	1.05790408010644\\
6081	1.04370457175235\\
6082	1.05654786350356\\
6083	1.04802636466725\\
6084	1.08616259191234\\
6085	1.12649112795149\\
6086	1.10957472410668\\
6087	1.12442563525481\\
6088	1.11733153015403\\
6089	1.07727398244711\\
6090	1.06963522818834\\
6091	1.09058190568092\\
6092	1.08735627929065\\
6093	1.08935939551936\\
6094	1.12812102392293\\
6095	1.18040528723248\\
6096	1.27665801441634\\
6097	1.33211691920725\\
6098	1.359730816202\\
6099	1.36545276805536\\
6100	1.35060214287512\\
6101	1.28682629096096\\
6102	1.14428221440036\\
6103	1.08637831144836\\
6104	1.06030982881213\\
6105	1.08772209753902\\
6106	1.11890876596549\\
6107	1.17206645635426\\
6108	1.20269346410765\\
6109	1.21802988990734\\
6110	1.22325818316481\\
6111	1.23861236771138\\
6112	1.20624846318948\\
6113	1.15878884757743\\
6114	1.12771533594868\\
6115	1.11505857643594\\
6116	1.08355480754121\\
6117	1.12216105028957\\
6118	1.13708045492727\\
6119	1.19129918860632\\
6120	1.28973000782803\\
6121	1.3608403105672\\
6122	1.39348387301045\\
6123	1.42825207811461\\
6124	1.45036297876256\\
6125	1.431637893475\\
6126	1.3730390242025\\
6127	1.3234361627931\\
6128	1.24492193264778\\
6129	1.22523807973216\\
6130	1.24271831778083\\
6131	1.26923015280247\\
6132	1.31857164737088\\
6133	1.36064065534514\\
6134	1.43042639634228\\
6135	1.44842752066435\\
6136	1.40980035645572\\
6137	1.33660313983074\\
6138	1.28756867835219\\
6139	1.26706012551224\\
6140	1.2630487370346\\
6141	1.28556742210903\\
6142	1.28667642124576\\
6143	1.35105852601948\\
6144	1.39920819426311\\
6145	1.43568656069422\\
6146	1.48350071541414\\
6147	1.5259872280348\\
6148	1.50820972395439\\
6149	1.49538219625831\\
6150	1.44014480100143\\
6151	1.40728367893047\\
6152	1.33824340026268\\
6153	1.26811371140451\\
6154	1.24427660659886\\
6155	1.23011889955567\\
6156	1.25714074859751\\
6157	1.32528508441351\\
6158	1.36704146857922\\
6159	1.40224131353645\\
6160	1.39237867810958\\
6161	1.33985175859981\\
6162	1.31494513930004\\
6163	1.26293439400911\\
6164	1.22185704867832\\
6165	1.24917837666145\\
6166	1.25336444089468\\
6167	1.31105004801059\\
6168	1.39272021035613\\
6169	1.44233253878802\\
6170	1.46347373189142\\
6171	1.47283428509874\\
6172	1.44929268097609\\
6173	1.35067698585689\\
6174	1.18066787221276\\
6175	1.08178084191956\\
6176	1.06098054869838\\
6177	1.05636294274117\\
6178	1.0589506559257\\
6179	1.07013294906244\\
6180	1.11663067339669\\
6181	1.12752656381411\\
6182	1.12450394608213\\
6183	1.11853091226587\\
6184	1.10041441623096\\
6185	1.07661140038863\\
6186	1.04879083077439\\
6187	1.0403406776939\\
6188	1.04296457362708\\
6189	1.07610584627451\\
6190	1.11059202419838\\
6191	1.17783596480805\\
6192	1.27601713496547\\
6193	1.31607136031382\\
6194	1.35341744603008\\
6195	1.35138055424374\\
6196	1.33716289316041\\
6197	1.26584526671437\\
6198	1.11999379658759\\
6199	1.04115272999355\\
6200	1.03111949267814\\
6201	1.03526393389279\\
6202	1.07010684022263\\
6203	1.07702100184578\\
6204	1.12602337834447\\
6205	1.12680613811287\\
6206	1.13943831036479\\
6207	1.13085972592705\\
6208	1.08869784246234\\
6209	1.05478517286334\\
6210	1.03445610089754\\
6211	1.04156224706206\\
6212	1.0267209985147\\
6213	1.0569340706636\\
6214	1.09015673144429\\
6215	1.15369247150126\\
6216	1.26406558668682\\
6217	1.30830238327794\\
6218	1.3574709687194\\
6219	1.3545359207062\\
6220	1.34529857530903\\
6221	1.28452809977606\\
6222	1.13505862095601\\
6223	1.06018121914627\\
6224	1.06079363438171\\
6225	1.08662937534408\\
6226	1.09069267867392\\
6227	1.09380175025575\\
6228	1.14698781157204\\
6229	1.16457028762464\\
6230	1.16752522613824\\
6231	1.15140659166339\\
6232	1.14360928618166\\
6233	1.11090849013535\\
6234	1.06752203085328\\
6235	1.0660478212801\\
6236	1.05177412379669\\
6237	1.08836229888804\\
6238	1.13117857282727\\
6239	1.19743101334089\\
6240	1.27233062058796\\
6241	1.3405762519598\\
6242	1.37287331299431\\
6243	1.37333943463659\\
6244	1.35523204941726\\
6245	1.28679679179803\\
6246	1.15210127989248\\
6247	1.05610422175811\\
6248	1.02328918272038\\
6249	1.0056835200973\\
6250	1.00059279973531\\
6251	0.97507528725633\\
6252	1.01539917642716\\
6253	1.02382828167901\\
6254	1.0204997673555\\
6255	1.02874798914277\\
6256	1.02842637889958\\
6257	1.003221943468\\
6258	1.00906794426239\\
6259	1.02481714065144\\
6260	1.01466166234438\\
6261	1.06512463445152\\
6262	1.10900063712369\\
6263	1.16571909896689\\
6264	1.23516422725939\\
6265	1.28348310828458\\
6266	1.33421539911658\\
6267	1.34086978635431\\
6268	1.32742756609383\\
6269	1.26240126683649\\
6270	1.12474082129364\\
6271	1.03164483894063\\
6272	1.01982461541432\\
6273	1.00652577682091\\
6274	1.01720748497982\\
6275	1.02243355366693\\
6276	1.04608035205603\\
6277	1.05603602531326\\
6278	1.04791627831746\\
6279	1.05401417447043\\
6280	1.04920638100652\\
6281	1.03987776459649\\
6282	1.04248504705885\\
6283	1.0687470613805\\
6284	1.04620568071051\\
6285	1.08482727601794\\
6286	1.09811387572476\\
6287	1.14341263381382\\
6288	1.2350023808362\\
6289	1.2981220981621\\
6290	1.35632714322208\\
6291	1.38652180526813\\
6292	1.40009152398466\\
6293	1.39508588806609\\
6294	1.35641050551636\\
6295	1.28204189592731\\
6296	1.23315566424165\\
6297	1.17475253730895\\
6298	1.19433609390827\\
6299	1.20553811036044\\
6300	1.22153658277807\\
6301	1.22526043163222\\
6302	1.2483741798257\\
6303	1.25022299978474\\
6304	1.24119786007172\\
6305	1.21581292871173\\
6306	1.18469865011385\\
6307	1.18963553267854\\
6308	1.15385926292798\\
6309	1.20946490421989\\
6310	1.22044079720181\\
6311	1.28311960131124\\
6312	1.35242260801567\\
6313	1.4152817383348\\
6314	1.47582541743518\\
6315	1.54356540675367\\
6316	1.58938054373749\\
6317	1.60679665978099\\
6318	1.58196283678221\\
6319	1.51826973181335\\
6320	1.45009900539192\\
6321	1.3595650200756\\
6322	1.32644423703973\\
6323	1.33172374843957\\
6324	1.4026596764631\\
6325	1.51880094354691\\
6326	1.5924808246851\\
6327	1.63002213589022\\
6328	1.5962092652692\\
6329	1.48964086800758\\
6330	1.35386540900063\\
6331	1.29824960134502\\
6332	1.2697692956939\\
6333	1.28024253432463\\
6334	1.28461633319719\\
6335	1.36597822794308\\
6336	1.40965434994522\\
6337	1.46216916702351\\
6338	1.48762450170886\\
6339	1.51114242431265\\
6340	1.5019743392863\\
6341	1.40224780567657\\
6342	1.20639935923502\\
6343	1.10677913582642\\
6344	1.07213273481938\\
6345	1.05625510620986\\
6346	1.06951219926787\\
6347	1.07585959519414\\
6348	1.12317265120748\\
6349	1.1296675248566\\
6350	1.13172392171373\\
6351	1.13086035869675\\
6352	1.11263081019853\\
6353	1.10071524707479\\
6354	1.0641804341889\\
6355	1.05974108240756\\
6356	1.0547183621253\\
6357	1.07390513505327\\
6358	1.1073096936615\\
6359	1.18368237389488\\
6360	1.24850245359162\\
6361	1.30888703433087\\
6362	1.34637111717395\\
6363	1.36293965426417\\
6364	1.33876862211356\\
6365	1.2691763502699\\
6366	1.11850237493455\\
6367	1.02360173869168\\
6368	1.02511567763761\\
6369	1.04849241440744\\
6370	1.05786192018167\\
6371	1.07619231696406\\
6372	1.11613846045216\\
6373	1.13095913344736\\
6374	1.13104783995809\\
6375	1.11642625522885\\
6376	1.0861961940058\\
6377	1.05612535736652\\
6378	1.0435754824804\\
6379	1.02429948747093\\
6380	1.01284185188271\\
6381	1.06091694920792\\
6382	1.09140962127799\\
6383	1.15993819666373\\
6384	1.25520216686525\\
6385	1.32424633704283\\
6386	1.38276173377805\\
6387	1.40383550218864\\
6388	1.396857166497\\
6389	1.33549590379938\\
6390	1.18054798560843\\
6391	1.0841891163432\\
6392	1.07062072550032\\
6393	1.0669681010496\\
6394	1.07894532467343\\
6395	1.05547985580713\\
6396	1.07076327667464\\
6397	1.07164606045432\\
6398	1.08509892424757\\
6399	1.08816544315345\\
6400	1.09056934656639\\
6401	1.06342602307418\\
6402	1.08169116356397\\
6403	1.08876736935036\\
6404	1.07665504715491\\
6405	1.12382761091477\\
6406	1.15058808546595\\
6407	1.22224361334871\\
6408	1.30763003257696\\
6409	1.36521056448212\\
6410	1.39345205910147\\
6411	1.39926903500698\\
6412	1.38669492789855\\
6413	1.30606873480122\\
6414	1.15066832965282\\
6415	1.04890706033846\\
6416	1.03919585150504\\
6417	1.04677761128502\\
6418	1.06645927095258\\
6419	1.08165657769298\\
6420	1.13945879419626\\
6421	1.15679044298635\\
6422	1.17216953095948\\
6423	1.17372075740324\\
6424	1.16192648818333\\
6425	1.13650926808378\\
6426	1.09767289110602\\
6427	1.0888806500235\\
6428	1.07068577027992\\
6429	1.12065392151705\\
6430	1.18027551976377\\
6431	1.23238399747165\\
6432	1.2866599207292\\
6433	1.35111434009088\\
6434	1.39727528562092\\
6435	1.4152538640466\\
6436	1.39784717918361\\
6437	1.32896363715623\\
6438	1.16729278572394\\
6439	1.0557261221014\\
6440	1.05652997107488\\
6441	1.03958069307119\\
6442	1.06183240848527\\
6443	1.05986732720205\\
6444	1.05915659132188\\
6445	1.06552326424243\\
6446	1.05564741488684\\
6447	1.04975711557905\\
6448	1.02390639130751\\
6449	1.00743468819342\\
6450	1.00520263341083\\
6451	1.00147274727354\\
6452	1.02189511073797\\
6453	1.06522029500804\\
6454	1.06954918322061\\
6455	1.11697818325123\\
6456	1.19068440665822\\
6457	1.19987207687246\\
6458	1.2516398162406\\
6459	1.27969996655668\\
6460	1.28952704442446\\
6461	1.28450521566483\\
6462	1.23614002651555\\
6463	1.17410814087044\\
6464	1.12168097536531\\
6465	1.08657262610254\\
6466	1.06985940802181\\
6467	1.0811698917824\\
6468	1.1347521662066\\
6469	1.16590469344863\\
6470	1.19535333595407\\
6471	1.2084325391962\\
6472	1.18359640864364\\
6473	1.12548897465758\\
6474	1.09947214097822\\
6475	1.08176567244242\\
6476	1.06680200457655\\
6477	1.09058277901997\\
6478	1.12580729712709\\
6479	1.17756294098773\\
6480	1.2608748935644\\
6481	1.37621991794089\\
6482	1.4332060190306\\
6483	1.46631424720739\\
6484	1.47553989091465\\
6485	1.46236134174885\\
6486	1.40439536029458\\
6487	1.36500440021416\\
6488	1.33844322115599\\
6489	1.30502103963503\\
6490	1.30812624201994\\
6491	1.32939135662171\\
6492	1.37115015196389\\
6493	1.45355714095344\\
6494	1.4886351809948\\
6495	1.48111452370008\\
6496	1.43969476954237\\
6497	1.31199470069318\\
6498	1.18413965442275\\
6499	1.07784686310774\\
6500	1.05907048229507\\
6501	1.14837224729582\\
6502	1.19482294068329\\
6503	1.26669129173711\\
6504	1.3557140970899\\
6505	1.40646378046396\\
6506	1.45277781216671\\
6507	1.45459835894429\\
6508	1.4307583616594\\
6509	1.32722070155798\\
6510	1.16354212517854\\
6511	1.05787684044037\\
6512	1.04619865312199\\
6513	1.06153238358974\\
6514	1.07569129643722\\
6515	1.07675027897413\\
6516	1.1123057228909\\
6517	1.10885745184698\\
6518	1.10091085672828\\
6519	1.03534829360777\\
6520	1.0146662921039\\
6521	0.968181360991986\\
6522	0.953340839431368\\
6523	0.966718322843213\\
6524	0.989141898021704\\
6525	1.02091593940525\\
6526	1.06306912545735\\
6527	1.12761281205554\\
6528	1.21857610042671\\
6529	1.29287428809378\\
6530	1.34872478889377\\
6531	1.3411028215546\\
6532	1.34122349366009\\
6533	1.27697486402254\\
6534	1.06171837896335\\
6535	0.972001227496914\\
6536	0.957372389217668\\
6537	0.951803328308159\\
6538	0.974665831623613\\
6539	0.994080137280165\\
6540	1.01028131838187\\
6541	1.01280003179309\\
6542	1.05316783730325\\
6543	1.0464016290494\\
6544	1.03401488664035\\
6545	0.9919435322977\\
6546	0.974019082196425\\
6547	0.962525427621652\\
6548	0.966396311433782\\
6549	0.997435296249056\\
6550	1.03752215402897\\
6551	1.10403313810877\\
6552	1.20991494825769\\
6553	1.26310349113423\\
6554	1.32077104046606\\
6555	1.34201045454009\\
6556	1.32154917291404\\
6557	1.25711820616351\\
6558	1.10681082470751\\
6559	1.00303856246414\\
6560	1.02707005678177\\
6561	1.03733824741815\\
6562	1.06804005753761\\
6563	1.05819545602494\\
6564	1.1028996244047\\
6565	1.1006121154901\\
6566	1.10327200925608\\
6567	1.09796082841257\\
6568	1.07828091286787\\
6569	1.0262524457742\\
6570	0.996889087741182\\
6571	1.00709860226146\\
6572	1.01949688616563\\
6573	1.01273060640143\\
6574	1.03566121558723\\
6575	1.10524254464352\\
6576	1.18205784854624\\
6577	1.23675427129827\\
6578	1.26390832129631\\
6579	1.2900256994354\\
6580	1.27516742144448\\
6581	1.20974111506368\\
6582	1.06453032497273\\
6583	0.9717649230554\\
6584	0.966932372127469\\
6585	0.965721249812264\\
6586	0.975361095887292\\
6587	0.974085724355514\\
6588	0.991157129573679\\
6589	0.998546382756248\\
6590	0.990945045985716\\
6591	0.982027036702531\\
6592	0.976214682038873\\
6593	0.965597683521814\\
6594	0.969216403666555\\
6595	0.972754029985539\\
6596	0.977966626348916\\
6597	0.995353790690302\\
6598	1.0289972906188\\
6599	1.10074297291895\\
6600	1.17611771804873\\
6601	1.22196575920732\\
6602	1.25571212957983\\
6603	1.27784807928109\\
6604	1.26292805954421\\
6605	1.203996115373\\
6606	1.06818789429423\\
6607	0.962063296549035\\
6608	0.977326963047017\\
6609	0.995620510095195\\
6610	1.02935597522665\\
6611	1.02061400902273\\
6612	1.07952338207332\\
6613	1.09257008437647\\
6614	1.09422362423743\\
6615	1.09479707971062\\
6616	1.06500122875222\\
6617	1.02936005442554\\
6618	1.01570387942216\\
6619	1.0156528248611\\
6620	0.999163839785109\\
6621	1.0299421730448\\
6622	1.06214803024598\\
6623	1.11491709384988\\
6624	1.22073734551062\\
6625	1.29499796668145\\
6626	1.36315224818882\\
6627	1.40455797039164\\
6628	1.4610325034006\\
6629	1.47842148317372\\
6630	1.4211420219806\\
6631	1.34687935451874\\
6632	1.32526732443771\\
6633	1.33523733085211\\
6634	1.35229751311211\\
6635	1.36378854590468\\
6636	1.40626169933279\\
6637	1.42566813840641\\
6638	1.43771422185611\\
6639	1.3865811321534\\
6640	1.33176549422743\\
6641	1.24584239115482\\
6642	1.21165192524454\\
6643	1.22295827150337\\
6644	1.20886470558974\\
6645	1.22859046759303\\
6646	1.27631999517119\\
6647	1.31994316516016\\
6648	1.4064365074846\\
6649	1.47637547049649\\
6650	1.51738956116994\\
6651	1.52421971262825\\
6652	1.52026886626543\\
6653	1.48474594519533\\
6654	1.43000408768385\\
6655	1.35703554121798\\
6656	1.31558004551559\\
6657	1.26120129361356\\
6658	1.22232426844006\\
6659	1.19725150667439\\
6660	1.19813698254526\\
6661	1.2285323397458\\
6662	1.26835856061023\\
6663	1.26550275754198\\
6664	1.23366153116343\\
6665	1.17765489586021\\
6666	1.10384797983668\\
6667	1.10720834450425\\
6668	1.11754591859181\\
6669	1.10522173023921\\
6670	1.14042736484063\\
6671	1.19736634260638\\
6672	1.28298096604447\\
6673	1.35006931068142\\
6674	1.39436881100193\\
6675	1.39545136445762\\
6676	1.37656072209548\\
6677	1.29861035441242\\
6678	1.11534421401455\\
6679	0.989740393677629\\
6680	0.996439112395265\\
6681	0.991134463374204\\
6682	1.01813670806303\\
6683	1.03172031083869\\
6684	1.05820127099352\\
6685	1.03415387762129\\
6686	1.03661115447253\\
6687	1.07045434793573\\
6688	1.05746018526789\\
6689	1.03421673059398\\
6690	1.03469560491722\\
6691	1.07826780902852\\
6692	1.11409373433527\\
6693	1.16327259088569\\
6694	1.20322366181909\\
6695	1.28097851957772\\
6696	1.38393179420443\\
6697	1.48593377113569\\
6698	1.53834970928921\\
6699	1.56558916206962\\
6700	1.53645185387728\\
6701	1.42889130852181\\
6702	1.23309110220794\\
6703	1.08352016234384\\
6704	1.1127055714725\\
6705	1.13392548025908\\
6706	1.16389088303529\\
6707	1.15661225978082\\
6708	1.24433275789985\\
6709	1.21557665195322\\
6710	1.24144727270521\\
6711	1.2475966186886\\
6712	1.24011880046443\\
6713	1.17393181241229\\
6714	1.10457197869373\\
6715	1.10847768914979\\
6716	1.10983348046282\\
6717	1.1393993600164\\
6718	1.18133739779865\\
6719	1.235507223003\\
6720	1.33126258717378\\
6721	1.39650505108989\\
6722	1.43994378449407\\
6723	1.44056023770341\\
6724	1.41372943364306\\
6725	1.33688301877372\\
6726	1.18270347729878\\
6727	1.08817262712472\\
6728	1.07567862042529\\
6729	1.04085748795842\\
6730	1.04607098835442\\
6731	1.03526269560239\\
6732	1.0294420172673\\
6733	1.05013222146517\\
6734	1.08024223923421\\
6735	1.1072310867184\\
6736	1.12905648175428\\
6737	1.10733714565771\\
6738	1.08782017910056\\
6739	1.10504124164922\\
6740	1.11157900284245\\
6741	1.15502953263868\\
6742	1.17198091377341\\
6743	1.24323410758501\\
6744	1.33661360689791\\
6745	1.40487173403863\\
6746	1.44115160086132\\
6747	1.46252645664046\\
6748	1.45062743644539\\
6749	1.36568564425024\\
6750	1.19449021941435\\
6751	1.08871494267826\\
6752	1.05662092797719\\
6753	1.09417255438799\\
6754	1.09296409702309\\
6755	1.11420920700696\\
6756	1.20140443986314\\
6757	1.19332758617604\\
6758	1.16522375894883\\
6759	1.14680374608857\\
6760	1.13650829841715\\
6761	1.09499575094029\\
6762	1.10776188516708\\
6763	1.08681959981613\\
6764	1.11242844480959\\
6765	1.1384449799993\\
6766	1.18297722484497\\
6767	1.23204060617058\\
6768	1.3409530198561\\
6769	1.41685419230562\\
6770	1.47929966747219\\
6771	1.4879489898328\\
6772	1.47003481546988\\
6773	1.36383820309061\\
6774	1.17628742414817\\
6775	1.04615382733413\\
6776	1.06810896178931\\
6777	1.0766482968258\\
6778	1.09301365261411\\
6779	1.06360550716379\\
6780	1.12454575229459\\
6781	1.13174936458881\\
6782	1.13038205275117\\
6783	1.14158321667999\\
6784	1.12463489047063\\
6785	1.08686408817342\\
6786	1.04904596189162\\
6787	1.03149061014403\\
6788	1.02147845839543\\
6789	1.06319031328911\\
6790	1.07594177844274\\
6791	1.1078125314193\\
6792	1.2128541373209\\
6793	1.30201435157995\\
6794	1.35358548160911\\
6795	1.38417977696096\\
6796	1.39255985051169\\
6797	1.36590750810118\\
6798	1.30223819084585\\
6799	1.22073479911604\\
6800	1.1716833749041\\
6801	1.14864986558153\\
6802	1.14152920743883\\
6803	1.14952527380694\\
6804	1.17455271985975\\
6805	1.18674163919165\\
6806	1.22681536083519\\
6807	1.23293206026219\\
6808	1.21746829889102\\
6809	1.185672205983\\
6810	1.1685146271354\\
6811	1.12583216730776\\
6812	1.15132432682927\\
6813	1.18434723790799\\
6814	1.23059473177806\\
6815	1.27475610249237\\
6816	1.32448151483397\\
6817	1.39200846167806\\
6818	1.42539448649117\\
6819	1.43392287391302\\
6820	1.43447275226822\\
6821	1.40449324407957\\
6822	1.36485168507107\\
6823	1.30323442787665\\
6824	1.26803978159475\\
6825	1.24232058772887\\
6826	1.23212433388416\\
6827	1.2546734352652\\
6828	1.27907713128063\\
6829	1.31906698045827\\
6830	1.35102753246507\\
6831	1.35337925982347\\
6832	1.33623788921047\\
6833	1.29221617639035\\
6834	1.25922631765158\\
6835	1.23010685601932\\
6836	1.28405405537422\\
6837	1.31118539330367\\
6838	1.30940430563317\\
6839	1.35040461803892\\
6840	1.41162934222784\\
6841	1.46826413480329\\
6842	1.50447551030341\\
6843	1.48603789873669\\
6844	1.4462016474513\\
6845	1.34531453481803\\
6846	1.18971406850158\\
6847	1.10628752075115\\
6848	1.12107124400362\\
6849	1.15341841040044\\
6850	1.15145286814253\\
6851	1.13813557787785\\
6852	1.16063061824991\\
6853	1.15468712140335\\
6854	1.1585751195495\\
6855	1.15579847897405\\
6856	1.14400480235947\\
6857	1.10576690878591\\
6858	1.07943299662117\\
6859	1.07002945794879\\
6860	1.10554883968955\\
6861	1.13577321315801\\
6862	1.14659581354189\\
6863	1.21337867464061\\
6864	1.31687066907833\\
6865	1.41213216127112\\
6866	1.46243282553604\\
6867	1.46033559666298\\
6868	1.45196885758168\\
6869	1.35656220369682\\
6870	1.19329837142654\\
6871	1.05621824963868\\
6872	1.08262425756697\\
6873	1.10922887085\\
6874	1.14650891007591\\
6875	1.14385133863084\\
6876	1.19310148384587\\
6877	1.16687713577619\\
6878	1.15848928243148\\
6879	1.13743961958386\\
6880	1.11129335403783\\
6881	1.05810327364324\\
6882	1.02876846069083\\
6883	1.00776107351858\\
6884	1.02885280347141\\
6885	1.06592686594645\\
6886	1.08137788075266\\
6887	1.14412944144124\\
6888	1.21736679236901\\
6889	1.22725581080623\\
6890	1.26210115514195\\
6891	1.26827516103396\\
6892	1.23879804930237\\
6893	1.17743677800395\\
6894	1.03784036751909\\
6895	0.94088416929177\\
6896	0.999429611507393\\
6897	1.01431655137818\\
6898	1.01266210450956\\
6899	1.01399080766388\\
6900	1.04199240055136\\
6901	1.04450627652745\\
6902	1.04842995397843\\
6903	1.04029991087836\\
6904	1.02665413714874\\
6905	0.994330478815297\\
6906	0.97010776729696\\
6907	0.950309371831047\\
6908	0.984427497075079\\
6909	1.0311092857701\\
6910	1.05456159736668\\
6911	1.11043089315896\\
6912	1.21560880139525\\
6913	1.28301193945745\\
6914	1.34264503710169\\
6915	1.35703329946491\\
6916	1.35717988984943\\
6917	1.30091964139476\\
6918	1.15161521146225\\
6919	1.02706240569151\\
6920	1.03736260182257\\
6921	1.03203931935318\\
6922	1.03404140340997\\
6923	1.04645906384557\\
6924	1.06512219752681\\
6925	1.0754379248127\\
6926	1.09042300223007\\
6927	1.08479861652835\\
6928	1.08217902966293\\
6929	1.02969001269196\\
6930	0.992431622124358\\
6931	1.00655407317896\\
6932	1.02346021408403\\
6933	1.04847029064283\\
6934	1.06516121859179\\
6935	1.10553919341309\\
6936	1.22267018536434\\
6937	1.29648243326967\\
6938	1.33725948585811\\
6939	1.34538768537787\\
6940	1.3197033411302\\
6941	1.25443787983769\\
6942	1.11955460878253\\
6943	1.00876635299961\\
6944	1.00088609972708\\
6945	1.01168624018989\\
6946	1.02718902799072\\
6947	1.05932214040482\\
6948	1.0982458112139\\
6949	1.10648409953122\\
6950	1.1202625618742\\
6951	1.11429458682805\\
6952	1.08728674435378\\
6953	1.06137644167798\\
6954	1.02179831326305\\
6955	1.01328224905639\\
6956	1.04088374727933\\
6957	1.10704093437705\\
6958	1.1237992440107\\
6959	1.17869015578986\\
6960	1.29541910390175\\
6961	1.37890615271704\\
6962	1.47367469224211\\
6963	1.51029997186562\\
6964	1.54548378909121\\
6965	1.55479422578487\\
6966	1.46492129408149\\
6967	1.35257623637463\\
6968	1.29809689644421\\
6969	1.29854143644691\\
6970	1.26002325934467\\
6971	1.25582135561944\\
6972	1.32297772260869\\
6973	1.36154790661459\\
6974	1.39003332474836\\
6975	1.37369642692658\\
6976	1.34498267289797\\
6977	1.24605494148905\\
6978	1.14844846882675\\
6979	1.15473713351984\\
6980	1.22119711096354\\
6981	1.27081307429859\\
6982	1.26633022186902\\
6983	1.31878229212258\\
6984	1.42835542741908\\
6985	1.5076104141312\\
6986	1.59748136727985\\
6987	1.64524638938996\\
6988	1.64042581999128\\
6989	1.62970565627301\\
6990	1.57130553741902\\
6991	1.49025491031799\\
6992	1.43839899187725\\
6993	1.42569185822938\\
6994	1.41692303544156\\
6995	1.42182753380738\\
6996	1.46190510273933\\
6997	1.56228161502629\\
6998	1.5813937892952\\
6999	1.57322422744615\\
7000	1.56550569113687\\
7001	1.44014498890027\\
7002	1.33914413522088\\
7003	1.25345411008865\\
7004	1.27199479311024\\
7005	1.30789686073201\\
7006	1.34851944709077\\
7007	1.39725787265236\\
7008	1.54354399129661\\
7009	1.61824252799252\\
7010	1.63565672887708\\
7011	1.64914052194\\
7012	1.60893962006132\\
7013	1.49274729780713\\
7014	1.26134037701063\\
7015	1.09230571420108\\
7016	1.07192570619372\\
7017	1.07528368022789\\
7018	1.08720826557139\\
7019	1.09236344249377\\
7020	1.13056517463954\\
7021	1.12479946470986\\
7022	1.12880503954555\\
7023	1.12880366691605\\
7024	1.11965364349117\\
7025	1.07950540228618\\
7026	1.10173163023126\\
7027	1.115917746875\\
7028	1.1341791044516\\
7029	1.22229376146049\\
7030	1.2328862264288\\
7031	1.28844126319324\\
7032	1.36478840117516\\
7033	1.46198557362463\\
7034	1.52117866264049\\
7035	1.55706191288974\\
7036	1.53271367090381\\
7037	1.45028916796151\\
7038	1.26586965080933\\
7039	1.14190837677352\\
7040	1.12465335123485\\
7041	1.1273659102006\\
7042	1.13018451629937\\
7043	1.0964068325515\\
7044	1.14104832447419\\
7045	1.14951308681747\\
7046	1.16722406680986\\
7047	1.19077010251335\\
7048	1.19392003566595\\
7049	1.14751343958625\\
7050	1.150830716494\\
7051	1.17085666462176\\
7052	1.25232667478089\\
7053	1.30870079073767\\
7054	1.33206553527051\\
7055	1.39684265958892\\
7056	1.47455881362571\\
7057	1.56783254945078\\
7058	1.62166012291419\\
7059	1.64160936151336\\
7060	1.60486193122236\\
7061	1.50908860104034\\
7062	1.28972387665833\\
7063	1.13890642271139\\
7064	1.14733250016793\\
7065	1.16984774504266\\
7066	1.19617812624426\\
7067	1.19737843566587\\
7068	1.21605177298939\\
7069	1.23804363635746\\
7070	1.23623052346014\\
7071	1.20999717857827\\
7072	1.19019014609573\\
7073	1.15622639618842\\
7074	1.06764745855317\\
7075	1.08885904081132\\
7076	1.08437365204746\\
7077	1.14770529467111\\
7078	1.18852065960063\\
7079	1.26938820956142\\
7080	1.36271647648119\\
7081	1.4283164943781\\
7082	1.48496129157021\\
7083	1.50252933454902\\
7084	1.48204719680036\\
7085	1.4021679961566\\
7086	1.23529427734295\\
7087	1.11219602509224\\
7088	1.12146100434064\\
7089	1.13667434906374\\
7090	1.14891715763596\\
7091	1.14377467177243\\
7092	1.17578284086238\\
7093	1.16391003475416\\
7094	1.17728573696249\\
7095	1.15453612543353\\
7096	1.15192540198619\\
7097	1.13451644717194\\
7098	1.14466888456913\\
7099	1.14036902141265\\
7100	1.19866590556623\\
7101	1.23948796957708\\
7102	1.26810551566091\\
7103	1.3143893651321\\
7104	1.40489980758166\\
7105	1.50563200105343\\
7106	1.57004892804628\\
7107	1.59682086346622\\
7108	1.58125895181316\\
7109	1.49280893594251\\
7110	1.31183327000214\\
7111	1.21192972233685\\
7112	1.18176335146459\\
7113	1.14679925067967\\
7114	1.15029777239556\\
7115	1.14196647196734\\
7116	1.13020773802295\\
7117	1.15835365515408\\
7118	1.18832270711159\\
7119	1.18496767818014\\
7120	1.18099564984477\\
7121	1.16829484397023\\
7122	1.11744601776078\\
7123	1.18219759526424\\
7124	1.20543229998669\\
7125	1.26252792223447\\
7126	1.25278692000894\\
7127	1.22324669353896\\
7128	1.30858807521286\\
7129	1.38090878147762\\
7130	1.42875828703677\\
7131	1.45684924323796\\
7132	1.45504415411736\\
7133	1.42322829819961\\
7134	1.35424716259617\\
7135	1.2856539525844\\
7136	1.20102787980958\\
7137	1.18226501364758\\
7138	1.18247701948999\\
7139	1.16849944340784\\
7140	1.16179786905739\\
7141	1.25165954439495\\
7142	1.29360645674658\\
7143	1.30618514385679\\
7144	1.27968649141031\\
7145	1.21661791373529\\
7146	1.16425065104958\\
7147	1.20505020695239\\
7148	1.22680771272074\\
7149	1.27888650552904\\
7150	1.2714643627725\\
7151	1.28653772982765\\
7152	1.40892117028263\\
7153	1.50990560355304\\
7154	1.57490158711406\\
7155	1.59641867872458\\
7156	1.6125041902525\\
7157	1.60612061593765\\
7158	1.61612958414921\\
7159	1.56410612815446\\
7160	1.51095107722316\\
7161	1.45705897862537\\
7162	1.45991103933224\\
7163	1.4904206829563\\
7164	1.49230245856183\\
7165	1.50237205826144\\
7166	1.54456520444981\\
7167	1.54910980535535\\
7168	1.4859662347439\\
7169	1.45815439900055\\
7170	1.36571789763384\\
7171	1.27163710271483\\
7172	1.30464337296201\\
7173	1.38243247272\\
7174	1.40998766867369\\
7175	1.43188219852521\\
7176	1.42386199637341\\
7177	1.54595280698592\\
7178	1.63495486515114\\
7179	1.68692601952083\\
7180	1.68248507056633\\
7181	1.63720349220878\\
7182	1.52252436056007\\
7183	1.33503335701666\\
7184	1.23144808692181\\
7185	1.22203074458404\\
7186	1.23201171515131\\
7187	1.25698904540119\\
7188	1.31130751287233\\
7189	1.30245654909858\\
7190	1.30017853485907\\
7191	1.29745844869234\\
7192	1.26116644693038\\
7193	1.20082142580462\\
7194	1.16389617602985\\
7195	1.10290795857398\\
7196	1.15113167337087\\
7197	1.17151663783608\\
7198	1.23753568468017\\
7199	1.27706643936252\\
7200	1.31837113758805\\
7201	1.41722099504143\\
7202	1.50709774477095\\
7203	1.55848654413372\\
7204	1.58324766603661\\
7205	1.55133698583864\\
7206	1.46064927158724\\
7207	1.2887071765976\\
7208	1.21026211169725\\
7209	1.21958233481824\\
7210	1.24627006555265\\
7211	1.25974197753527\\
7212	1.27202018754474\\
7213	1.29839540385934\\
7214	1.29215277205447\\
7215	1.28710362158937\\
7216	1.24281032182655\\
7217	1.17941221336937\\
7218	1.12548633946012\\
7219	1.10507767827764\\
7220	1.13522651810542\\
7221	1.18055767906335\\
7222	1.24580573421381\\
7223	1.25435693710732\\
7224	1.29055265800647\\
7225	1.40662603645394\\
7226	1.48733019071202\\
7227	1.5425725669344\\
7228	1.55084653359096\\
7229	1.52407074856935\\
7230	1.43015089387623\\
7231	1.26875486293798\\
7232	1.17795082217821\\
7233	1.12424499809694\\
7234	1.1207202928696\\
7235	1.09654300070716\\
7236	1.07386928638935\\
7237	1.06940698643026\\
7238	1.06805835714938\\
7239	1.07173507328946\\
7240	1.07136410129226\\
7241	1.06135289816503\\
7242	1.05133249864721\\
7243	1.04443994433303\\
7244	1.08935355767761\\
7245	1.11666876809462\\
7246	1.15746165539138\\
7247	1.17755755131115\\
7248	1.1995628284929\\
7249	1.28864368082486\\
7250	1.36993102059566\\
7251	1.43614808287578\\
7252	1.44573869883087\\
7253	1.42944238653414\\
7254	1.35290935253116\\
7255	1.21328207196265\\
7256	1.16534336193683\\
7257	1.13370399003733\\
7258	1.11364423317598\\
7259	1.1304697297083\\
7260	1.12948931878582\\
7261	1.16417766818839\\
7262	1.16110147722216\\
7263	1.16776763450796\\
7264	1.16246816617096\\
7265	1.15380812136816\\
7266	1.10720418758389\\
7267	1.05213079173931\\
7268	1.10335237509004\\
7269	1.18851917385072\\
7270	1.2370957846523\\
7271	1.25997778472207\\
7272	1.2818493321489\\
7273	1.39597654164446\\
7274	1.4975914283859\\
7275	1.56966990739023\\
7276	1.59187850442756\\
7277	1.5581663529297\\
7278	1.46578311042\\
7279	1.29275710227124\\
7280	1.20772599445982\\
7281	1.18890555476759\\
7282	1.19446968330768\\
7283	1.23296996912909\\
7284	1.24576891155082\\
7285	1.26532526462999\\
7286	1.28520541443232\\
7287	1.29325916148401\\
7288	1.28277812068838\\
7289	1.25796692579806\\
7290	1.20268297005137\\
7291	1.17982612296326\\
7292	1.21132467828415\\
7293	1.28231549112805\\
7294	1.37169613686359\\
7295	1.37186220431969\\
7296	1.3494025094415\\
7297	1.48585874367668\\
7298	1.58036042801981\\
7299	1.65850831842235\\
7300	1.69690336294527\\
7301	1.71633406805881\\
7302	1.70064447503923\\
7303	1.63480002445063\\
7304	1.56785629722909\\
7305	1.55039692015983\\
7306	1.57515333913234\\
7307	1.59922363444031\\
7308	1.60360642238278\\
7309	1.58057979105424\\
7310	1.59191055582433\\
7311	1.60337748712402\\
7312	1.5806833249041\\
7313	1.48217664062599\\
7314	1.36572668449007\\
7315	1.34759493146839\\
7316	1.37270528856187\\
7317	1.42038373701682\\
7318	1.41700233550744\\
7319	1.38677709682858\\
7320	1.39817165255664\\
7321	1.54923863078182\\
7322	1.66786631602759\\
7323	1.74508426582008\\
7324	1.82106305470517\\
7325	1.84013817665236\\
7326	1.82551715756503\\
7327	1.74649698445822\\
7328	1.72171660619636\\
7329	1.71268443786388\\
7330	1.72776746188214\\
7331	1.7410367907147\\
7332	1.74340015031653\\
7333	1.69690267172663\\
7334	1.71152487451682\\
7335	1.6837941885451\\
7336	1.64387234030824\\
7337	1.59445038391844\\
7338	1.49588791868314\\
7339	1.43373698836182\\
7340	1.45092536545902\\
7341	1.51083125751249\\
7342	1.49278159732246\\
7343	1.50352051146666\\
7344	1.52973820363999\\
7345	1.68041366329489\\
7346	1.79173805384269\\
7347	1.85829655215958\\
7348	1.88815315236732\\
7349	1.84011079844167\\
7350	1.69604410525833\\
7351	1.45658687523597\\
7352	1.29768710490835\\
7353	1.29699347183618\\
7354	1.30975322814131\\
7355	1.27680187876811\\
7356	1.28657282556883\\
7357	1.29142590779471\\
7358	1.27195484691779\\
7359	1.26241574599898\\
7360	1.25762244653707\\
7361	1.21707647244036\\
7362	1.16504246941153\\
7363	1.16603041177295\\
7364	1.17767291644263\\
7365	1.23441727583983\\
7366	1.28080416362077\\
7367	1.27536638774925\\
7368	1.28995510917611\\
7369	1.37561398532459\\
7370	1.43894380576157\\
7371	1.48269908783761\\
7372	1.47295279557277\\
7373	1.44749046634773\\
7374	1.36307481398959\\
7375	1.20964535787835\\
7376	1.13096130502688\\
7377	1.14049206803113\\
7378	1.10567684146393\\
7379	1.07259311668044\\
7380	1.0660872691596\\
7381	1.07078107942011\\
7382	1.06679335249707\\
7383	1.06452123697932\\
7384	1.04786165135552\\
7385	1.06658797357636\\
7386	1.00341314625985\\
7387	1.0026421812786\\
7388	1.04956516541799\\
7389	1.05686508478737\\
7390	1.11745431076727\\
7391	1.13510522118355\\
7392	1.1563814181477\\
7393	1.23686588083633\\
7394	1.30125513204759\\
7395	1.34461067502714\\
7396	1.35125798766251\\
7397	1.33675320597229\\
7398	1.27375141864752\\
7399	1.13185698462149\\
7400	1.03063665229302\\
7401	1.03219434529899\\
7402	1.0446457214945\\
7403	1.0363380780069\\
7404	1.05276097770104\\
7405	1.05262931785551\\
7406	1.05266150985609\\
7407	1.05650792706577\\
7408	1.03978210320453\\
7409	1.02624769567328\\
7410	1.01482465683812\\
7411	1.0067272512266\\
7412	1.06277730703351\\
7413	1.05322154382293\\
7414	1.10756838881264\\
7415	1.12326998897477\\
7416	1.18764519825793\\
7417	1.25499416019032\\
7418	1.32845004042068\\
7419	1.37219754384602\\
7420	1.38939284642073\\
7421	1.38016812432982\\
7422	1.31600878386599\\
7423	1.18298398585468\\
7424	1.11290848555478\\
7425	1.1290864348999\\
7426	1.11459022114474\\
7427	1.10998349554398\\
7428	1.13438497806373\\
7429	1.13860936048646\\
7430	1.14293849745297\\
7431	1.15361182401636\\
7432	1.15360880670594\\
7433	1.14485284177677\\
7434	1.14055670982877\\
7435	1.11678370723856\\
7436	1.18719995861357\\
7437	1.20079706339676\\
7438	1.2819052948161\\
7439	1.27996025372244\\
7440	1.32243662074768\\
7441	1.45468773235101\\
7442	1.55840560736773\\
7443	1.63965021501366\\
7444	1.68396425975453\\
7445	1.67013483573619\\
7446	1.57162012980327\\
7447	1.37031627882753\\
7448	1.25755576054567\\
7449	1.23340041545807\\
7450	1.21467096137474\\
7451	1.21004209239358\\
7452	1.18576675337776\\
7453	1.19617104516924\\
7454	1.19959426413406\\
7455	1.21320186404023\\
7456	1.22647058757037\\
7457	1.255323785144\\
7458	1.22175035755911\\
7459	1.1921440127787\\
7460	1.21401175338454\\
7461	1.28479088212639\\
7462	1.34618798124034\\
7463	1.32080637866028\\
7464	1.30379193819666\\
7465	1.41760394696993\\
7466	1.52013214428761\\
7467	1.59008099081273\\
7468	1.66109546208216\\
7469	1.66780400699777\\
7470	1.60568720163971\\
7471	1.5370070294487\\
7472	1.4542050002841\\
7473	1.40739158847877\\
7474	1.39867274552525\\
7475	1.46541317274872\\
7476	1.46906789918025\\
7477	1.45942177014339\\
7478	1.47245748393445\\
7479	1.46161621600085\\
7480	1.44145495324817\\
7481	1.38802538434543\\
7482	1.32371165523007\\
7483	1.2755838605189\\
7484	1.34220098321629\\
7485	1.37301347804693\\
7486	1.39585272475636\\
7487	1.39056425626508\\
7488	1.40838271088115\\
7489	1.5005084824343\\
7490	1.60329105284511\\
7491	1.66678190067631\\
7492	1.69084373164267\\
7493	1.70365540100961\\
7494	1.67086103029063\\
7495	1.63075102506058\\
7496	1.59818651000119\\
7497	1.5528058345542\\
7498	1.49900901367461\\
7499	1.4637446916856\\
7500	1.40418187820297\\
7501	1.37353661600883\\
7502	1.36442309511314\\
7503	1.35969619548758\\
7504	1.34232061173291\\
7505	1.31893192543179\\
7506	1.29189719677127\\
7507	1.25442204560014\\
7508	1.2506011025794\\
7509	1.29101540488475\\
7510	1.3005921306706\\
7511	1.30918506391781\\
7512	1.33059310100441\\
7513	1.38621958503115\\
7514	1.47453053090137\\
7515	1.54105706170118\\
7516	1.56873011121501\\
7517	1.56082179129312\\
7518	1.50051113853253\\
7519	1.37882258188227\\
7520	1.32877943561405\\
7521	1.30540188487521\\
7522	1.3507825008099\\
7523	1.34659618466961\\
7524	1.36147129350223\\
7525	1.41027021147284\\
7526	1.40785688806561\\
7527	1.3941390292872\\
7528	1.35124199652387\\
7529	1.32359642464898\\
7530	1.22575821203442\\
7531	1.21351111275873\\
7532	1.26750000070935\\
7533	1.28959730824382\\
7534	1.34894342306033\\
7535	1.28823683450481\\
7536	1.32565186872595\\
7537	1.43065062478183\\
7538	1.53100563361599\\
7539	1.59071439069412\\
7540	1.63120480990362\\
7541	1.63700291426659\\
7542	1.61677165471014\\
7543	1.5530037443191\\
7544	1.59344302939817\\
7545	1.59136974072039\\
7546	1.5514322076895\\
7547	1.49002136546822\\
7548	1.45831697509721\\
7549	1.43243421318708\\
7550	1.46731502014876\\
7551	1.50088123923361\\
7552	1.47538501373738\\
7553	1.42660523137031\\
7554	1.38009578583954\\
7555	1.29449690608938\\
7556	1.30535486532593\\
7557	1.30490170283861\\
7558	1.36223472137256\\
7559	1.33164068469841\\
7560	1.35226617490796\\
7561	1.48171378342658\\
7562	1.59604116073204\\
7563	1.63775385094809\\
7564	1.63928258672474\\
7565	1.5913155071352\\
7566	1.47741897259339\\
7567	1.26117491957637\\
7568	1.12605243635897\\
7569	1.14512019356463\\
7570	1.13321601039193\\
7571	1.12124950548218\\
7572	1.11945455340099\\
7573	1.12142491767927\\
7574	1.10528985966913\\
7575	1.10517053831621\\
7576	1.10064008727927\\
7577	1.11311709758911\\
7578	1.03517626510538\\
7579	1.02437154386706\\
7580	1.06721021608131\\
7581	1.08908533692766\\
7582	1.14829209920267\\
7583	1.14173213993756\\
7584	1.18128092379569\\
7585	1.24833652856367\\
7586	1.3227713441275\\
7587	1.38070400662217\\
7588	1.39429687274772\\
7589	1.36818779085545\\
7590	1.30234107521328\\
7591	1.15737135318966\\
7592	1.09444762715177\\
7593	1.10557806307501\\
7594	1.1343311299089\\
7595	1.20137045606924\\
7596	1.19690719082193\\
7597	1.20444956406719\\
7598	1.19142229168113\\
7599	1.16203267650688\\
7600	1.16725996853544\\
7601	1.1310107567699\\
7602	1.04243267184555\\
7603	1.02938940467584\\
7604	1.08643636064711\\
7605	1.11768444806023\\
7606	1.18523569812686\\
7607	1.18595917931612\\
7608	1.21370284379914\\
7609	1.33374749886165\\
7610	1.4315139677899\\
7611	1.49902694979398\\
7612	1.51958198785278\\
7613	1.51424205351877\\
7614	1.41704125215453\\
7615	1.23828056051572\\
7616	1.12544788750909\\
7617	1.11353787798353\\
7618	1.09916448001672\\
7619	1.11577132451857\\
7620	1.08918157928777\\
7621	1.1117759264456\\
7622	1.08666427789288\\
7623	1.07670363911564\\
7624	1.07087010772358\\
7625	1.05298155961967\\
7626	1.05420635291141\\
7627	1.05284713728117\\
7628	1.0884174890969\\
7629	1.08452233896528\\
7630	1.14784226385094\\
7631	1.13036801004804\\
7632	1.18626912477169\\
7633	1.29813167214559\\
7634	1.36389847424077\\
7635	1.43367441109569\\
7636	1.47102102320681\\
7637	1.47871381953933\\
7638	1.45147674895472\\
7639	1.39666012860607\\
7640	1.32379561509785\\
7641	1.25314364158755\\
7642	1.20890191324126\\
7643	1.19307557732079\\
7644	1.17653519144751\\
7645	1.16964383052246\\
7646	1.18915077896739\\
7647	1.20979676555704\\
7648	1.22307719150447\\
7649	1.23665244971736\\
7650	1.21800858994917\\
7651	1.20045188190836\\
7652	1.17263602801937\\
7653	1.23160112334579\\
7654	1.29757750556523\\
7655	1.28071144717034\\
7656	1.28596391642507\\
7657	1.3674399057841\\
7658	1.4471660537761\\
7659	1.5129291598983\\
7660	1.54083611171858\\
7661	1.5419362368935\\
7662	1.54640362331748\\
7663	1.5040239048673\\
7664	1.45635952438758\\
7665	1.40199160597668\\
7666	1.3146484173776\\
7667	1.28530317884864\\
7668	1.25960639049589\\
7669	1.23550416383464\\
7670	1.24710479107221\\
7671	1.27204293312177\\
7672	1.26209399620845\\
7673	1.25906630489445\\
7674	1.20965949252367\\
7675	1.19069378772299\\
7676	1.18303745784681\\
7677	1.21495795283603\\
7678	1.24328286706077\\
7679	1.25387719283333\\
7680	1.27960862802315\\
7681	1.37137252953886\\
7682	1.441763021884\\
7683	1.48267704716921\\
7684	1.51211480495453\\
7685	1.49900015511564\\
7686	1.39277094084816\\
7687	1.22243709246244\\
7688	1.11029568243563\\
7689	1.0954600556972\\
7690	1.07789236049355\\
7691	1.08787309385529\\
7692	1.08634566246869\\
7693	1.110442494182\\
7694	1.09696358061795\\
7695	1.06516828066073\\
7696	1.04907254227579\\
7697	1.05954974368782\\
7698	1.00206114046926\\
7699	1.00360406684952\\
7700	1.06509626403912\\
7701	1.09425079790958\\
7702	1.15545085390075\\
7703	1.15015803959366\\
7704	1.1878444937259\\
7705	1.26375385253127\\
7706	1.33729077297691\\
7707	1.38984977145949\\
7708	1.40971698041716\\
7709	1.38919868445432\\
7710	1.30557127640296\\
7711	1.15442902866486\\
7712	1.0474645799112\\
7713	1.05615235942739\\
7714	1.06961146311399\\
7715	1.07725685385346\\
7716	1.07791132606382\\
7717	1.07764716390248\\
7718	1.0629695195443\\
7719	1.06385667051485\\
7720	1.03914281220779\\
7721	1.05869252689062\\
7722	1.0134096381461\\
7723	1.01743073301375\\
7724	1.03941560268363\\
7725	1.0632431683985\\
7726	1.12100278119352\\
7727	1.11989935053898\\
7728	1.1558531840246\\
7729	1.2471011577619\\
7730	1.31548813231623\\
7731	1.37006401423018\\
7732	1.39081287217043\\
7733	1.36868836089931\\
7734	1.30324978876611\\
7735	1.13707436974597\\
7736	1.02108955588854\\
7737	1.0387189220905\\
7738	1.02916605783834\\
7739	1.03256298470261\\
7740	1.03304347665922\\
7741	1.03870192258634\\
7742	1.02186813126664\\
7743	1.03325064125023\\
7744	1.0255193184383\\
7745	1.04857566132759\\
7746	1.02235649025934\\
7747	1.03060424069912\\
7748	1.06941358661043\\
7749	1.05969587448474\\
7750	1.12289442700254\\
7751	1.14871445429994\\
7752	1.18524655583246\\
7753	1.2343487650033\\
7754	1.3112801719613\\
7755	1.34432140809764\\
7756	1.36450659319946\\
7757	1.34465926349191\\
7758	1.27828702119909\\
7759	1.14594857663227\\
7760	1.04456556069701\\
7761	1.06313520487951\\
7762	1.07815414373704\\
7763	1.106534619516\\
7764	1.1212196616789\\
7765	1.12176918987848\\
7766	1.10884375290027\\
7767	1.10272654721681\\
7768	1.09024526735803\\
7769	1.0733347749563\\
7770	1.05604240177076\\
7771	1.05186400691022\\
7772	1.08682569733765\\
7773	1.09341669493135\\
7774	1.16044805396111\\
7775	1.17388620419271\\
7776	1.18143798982648\\
7777	1.25425629476532\\
7778	1.32501784943678\\
7779	1.36248271953444\\
7780	1.3811693840802\\
7781	1.35806542901804\\
7782	1.29893922465766\\
7783	1.15662789386553\\
7784	1.08420580676942\\
7785	1.05800822096083\\
7786	1.05854394207158\\
7787	1.05780571107523\\
7788	1.07176408035707\\
7789	1.08948508090298\\
7790	1.06699622155154\\
7791	1.08423553796223\\
7792	1.08524005301516\\
7793	1.06686457722924\\
7794	1.05552911669761\\
7795	1.05668341155985\\
7796	1.09002042413566\\
7797	1.13894232232077\\
7798	1.19861386564586\\
7799	1.18471523917354\\
7800	1.19768544044825\\
7801	1.29030398014069\\
7802	1.39021410894584\\
7803	1.4657974669166\\
7804	1.51038017421218\\
7805	1.53284982794108\\
7806	1.51635710934188\\
7807	1.47141365231208\\
7808	1.35599667361449\\
7809	1.30395699074751\\
7810	1.27258388216706\\
7811	1.33993902431494\\
7812	1.33597696875151\\
7813	1.32971636577593\\
7814	1.33343083217445\\
7815	1.34060820620004\\
7816	1.32041755754789\\
7817	1.26450818623\\
7818	1.24206757483882\\
7819	1.21095424393698\\
7820	1.20310062248499\\
7821	1.25437087307427\\
7822	1.30711440476669\\
7823	1.29670118293397\\
7824	1.29680050800515\\
7825	1.40956601501911\\
7826	1.49681929467887\\
7827	1.56608517900745\\
7828	1.60512195382431\\
7829	1.60407656373789\\
7830	1.58346029187581\\
7831	1.53672164738418\\
7832	1.50539922504569\\
7833	1.47359159162758\\
7834	1.44420652075804\\
7835	1.44063522149435\\
7836	1.40983983765842\\
7837	1.40745871013928\\
7838	1.43038374460364\\
7839	1.44040801401633\\
7840	1.42624637334458\\
7841	1.36531196889223\\
7842	1.34679147133657\\
7843	1.32630840792285\\
7844	1.28846365225749\\
7845	1.31864377157413\\
7846	1.39662952485392\\
7847	1.39480758791067\\
7848	1.40574249250964\\
7849	1.50504938184915\\
7850	1.56060218418323\\
7851	1.61003418793898\\
7852	1.61476856388513\\
7853	1.58462426942826\\
7854	1.49402332380897\\
7855	1.3099495143442\\
7856	1.19011064056304\\
7857	1.14632233792337\\
7858	1.14839240698732\\
7859	1.16861664819457\\
7860	1.15784407785445\\
7861	1.18914574474172\\
7862	1.21357025934924\\
7863	1.17750319582641\\
7864	1.17058524222801\\
7865	1.14346267740398\\
7866	1.09829877870926\\
7867	1.09303676128223\\
7868	1.11461960309142\\
7869	1.16118714131962\\
7870	1.23033644651135\\
7871	1.21367116680877\\
7872	1.24320936674117\\
7873	1.32750043110479\\
7874	1.39457232656152\\
7875	1.44582091450807\\
7876	1.46930176246939\\
7877	1.45574745931507\\
7878	1.38232155258887\\
7879	1.21895884746888\\
7880	1.10739831192362\\
7881	1.09755337263761\\
7882	1.11719706589047\\
7883	1.13464965755995\\
7884	1.14166392335622\\
7885	1.16912571678647\\
7886	1.15657824413073\\
7887	1.15013098687249\\
7888	1.1517920726982\\
7889	1.12993864826319\\
7890	1.08045829975623\\
7891	1.09658775263752\\
7892	1.10840922387191\\
7893	1.14902777838844\\
7894	1.21403746877215\\
7895	1.20385202063875\\
7896	1.22847860668493\\
7897	1.32028226929814\\
7898	1.39147293037083\\
7899	1.44499642740551\\
7900	1.46004739095762\\
7901	1.4354175257291\\
7902	1.3624413475812\\
7903	1.20642129075911\\
7904	1.1020507549484\\
7905	1.06786273830811\\
7906	1.05717644901794\\
7907	1.03809641179063\\
7908	1.02847482904692\\
7909	1.02757667233323\\
7910	1.02130045487999\\
7911	1.02015126240255\\
7912	1.01570533231853\\
7913	1.0073432300734\\
7914	1.01078808685437\\
7915	1.00517968245639\\
7916	1.0427019530266\\
7917	1.10526384259548\\
7918	1.14053482944537\\
7919	1.14317894287816\\
7920	1.16598108400041\\
7921	1.24691121497714\\
7922	1.32303470537658\\
7923	1.37929965594756\\
7924	1.36398755702996\\
7925	1.35195674500407\\
7926	1.29321904322495\\
7927	1.14290138098221\\
7928	1.04500005803602\\
7929	1.01444295546442\\
7930	1.04502060509009\\
7931	1.05440046184668\\
7932	1.02568138399726\\
7933	1.05079178914671\\
7934	1.04291396497014\\
7935	1.04977889115115\\
7936	1.04688701067462\\
7937	1.03604551578657\\
7938	1.03363182528548\\
7939	1.03964580491907\\
7940	1.04083065344723\\
7941	1.08011665538616\\
7942	1.1499983308061\\
7943	1.14576645051718\\
7944	1.17590264989348\\
7945	1.29982922962705\\
7946	1.3820575296193\\
7947	1.44528941049646\\
7948	1.48122922953787\\
7949	1.47380647500263\\
7950	1.38932478127899\\
7951	1.22825976182295\\
7952	1.16774644573713\\
7953	1.15688066730892\\
7954	1.16278554204201\\
7955	1.15955009856453\\
7956	1.15108188155975\\
7957	1.16794213076964\\
7958	1.1823108314222\\
7959	1.1891521991248\\
7960	1.18914964600505\\
7961	1.15113734856266\\
7962	1.16542283487871\\
7963	1.17852756061514\\
7964	1.18236279705618\\
7965	1.21962503818641\\
7966	1.27804156714253\\
7967	1.24915687458076\\
7968	1.25201229524308\\
7969	1.32090740410289\\
7970	1.40121201120829\\
7971	1.47561980941002\\
7972	1.52032852497847\\
7973	1.52515710571453\\
7974	1.51073501095657\\
7975	1.43689222466292\\
7976	1.35645315347259\\
7977	1.30908310632896\\
7978	1.27219072563634\\
7979	1.28891018923673\\
7980	1.30752247386622\\
7981	1.31043773421744\\
7982	1.32303043093813\\
7983	1.31518829183638\\
7984	1.2731492095661\\
7985	1.25411524724276\\
7986	1.2078148175986\\
7987	1.20643498070998\\
7988	1.26074342060625\\
7989	1.28746698168377\\
7990	1.2980901643087\\
7991	1.26454780341631\\
7992	1.24284822751565\\
7993	1.2853144847359\\
7994	1.35203760491879\\
7995	1.40877184751804\\
7996	1.44454498621928\\
7997	1.46351451227182\\
7998	1.45636412299606\\
7999	1.41636292089711\\
8000	1.38569379734371\\
8001	1.32714338990184\\
};
\addplot [color=mycolor1,line width=1.3pt,solid,forget plot]
  table[row sep=crcr]{%
8001	1.32714338990184\\
8002	1.2674075212338\\
8003	1.14362774899525\\
8004	1.13319750850523\\
8005	1.12447080811752\\
8006	1.15237038635345\\
8007	1.16015641768922\\
8008	1.1388931776476\\
8009	1.09150916218999\\
8010	1.0860738346635\\
8011	1.08138613468958\\
8012	1.06862009194893\\
8013	1.0804327780723\\
8014	1.14203677837624\\
8015	1.14193768572\\
8016	1.1680851209585\\
8017	1.24605609852322\\
8018	1.2976089673785\\
8019	1.36880748681882\\
8020	1.40818123806881\\
8021	1.39562889135193\\
8022	1.32937673970234\\
8023	1.16603230454341\\
8024	1.05427790767217\\
8025	1.00864756110756\\
8026	0.978957808151817\\
8027	0.968801154952644\\
8028	0.948475176961589\\
8029	0.948871398038586\\
8030	0.957600391326347\\
8031	0.944034951921649\\
8032	0.944337176684138\\
8033	0.945376561641493\\
8034	0.936550968501996\\
8035	0.947457915088902\\
8036	0.983822115177679\\
8037	0.982765590698779\\
8038	1.04284271141051\\
8039	1.04815171113407\\
8040	1.06503009210528\\
8041	1.10235603353149\\
8042	1.15213588622339\\
8043	1.1879960280464\\
8044	1.22055266170344\\
8045	1.33679125272692\\
8046	1.28660543185161\\
8047	1.13609739429769\\
8048	1.04523929503981\\
8049	1.03381907691649\\
8050	1.01870914209306\\
8051	1.01674731509505\\
8052	1.0240966384098\\
8053	1.01920201584531\\
8054	1.05626220069971\\
8055	1.04158817276111\\
8056	1.03006444666983\\
8057	1.0211111188244\\
8058	1.01424793275292\\
8059	1.02290240701807\\
8060	1.04327057337923\\
8061	1.08226376946989\\
8062	1.15813989231641\\
8063	1.159215648338\\
8064	1.15563858637609\\
8065	1.20893363087363\\
8066	1.26700888111112\\
8067	1.32741012006269\\
8068	1.35028200343963\\
8069	1.34938406688676\\
8070	1.28913265899389\\
8071	1.15878622002352\\
8072	1.05468381697732\\
8073	1.03795838810899\\
8074	1.03728350250654\\
8075	1.04520038241315\\
8076	1.04058198601095\\
8077	1.03928826158422\\
8078	1.02430520104429\\
8079	1.01997280945339\\
8080	0.994768047331192\\
8081	0.988564731027834\\
8082	0.990786051662645\\
8083	1.0017059501101\\
8084	1.01892099257461\\
8085	1.03216424814607\\
8086	1.0742264008316\\
8087	1.07728897467048\\
8088	1.09973077996003\\
8089	1.12288956837821\\
8090	1.17015407913711\\
8091	1.20996676400709\\
8092	1.24468255847432\\
8093	1.24007561745903\\
8094	1.19871446217658\\
8095	1.07766813704215\\
8096	0.997548267556559\\
8097	0.975779094581519\\
8098	0.975910833539629\\
8099	0.97715941708481\\
8100	0.959052621125101\\
8101	0.959024284284988\\
8102	0.958549032833736\\
8103	0.966191306550613\\
8104	0.962570188330324\\
8105	0.971775966357862\\
8106	0.943372835236049\\
8107	0.948769958911037\\
8108	0.961989949917675\\
8109	0.983741306653651\\
8110	1.04530761677083\\
8111	1.0400872316946\\
8112	1.04121526815226\\
8113	1.13955654336282\\
8114	1.16091790884612\\
8115	1.21321208664552\\
8116	1.24549972200872\\
8117	1.24962218733111\\
8118	1.20064984198152\\
8119	1.07799264454433\\
8120	1.00771601136051\\
8121	1.01790651164035\\
8122	0.998234475490073\\
8123	0.995316411288128\\
8124	0.995942184187268\\
8125	1.0067564788663\\
8126	1.01272572851109\\
8127	1.02714353228813\\
8128	1.030239263904\\
8129	1.01672414834098\\
8130	1.03114569653947\\
8131	1.05482100366943\\
8132	1.0627005401402\\
8133	1.12401527916026\\
8134	1.12315344344446\\
8135	1.11127057848047\\
8136	1.11536411846945\\
8137	1.17074092777347\\
8138	1.2381944951011\\
8139	1.31172244699663\\
8140	1.36015639969534\\
8141	1.39210556517492\\
8142	1.38400518319083\\
8143	1.31899715201476\\
8144	1.25744622202481\\
8145	1.19911255721701\\
8146	1.14612280374359\\
8147	1.13665084539821\\
8148	1.14152381787197\\
8149	1.14854800760858\\
8150	1.15060856031094\\
8151	1.16029829013551\\
8152	1.13074894134994\\
8153	1.09733511545445\\
8154	1.06285499208605\\
8155	1.05832928549834\\
8156	1.0585657608774\\
8157	1.08875361714719\\
8158	1.13217957623135\\
8159	1.12745231394739\\
8160	1.09258788344239\\
8161	1.14767427824545\\
8162	1.22348169054036\\
8163	1.28882118088677\\
8164	1.34814187825097\\
8165	1.39012747324226\\
8166	1.402245041811\\
8167	1.37583035980577\\
8168	1.37047522725175\\
8169	1.324938721044\\
8170	1.27518138566532\\
8171	1.22230717944232\\
8172	1.18893603171069\\
8173	1.17215879003556\\
8174	1.23004960575322\\
8175	1.25283774461116\\
8176	1.25834268306939\\
8177	1.2244318467072\\
8178	1.1942764558703\\
8179	1.19395429171517\\
8180	1.18401143359973\\
8181	1.17894483335176\\
8182	1.19597314372777\\
8183	1.1811631465486\\
8184	1.23686267241558\\
8185	1.34079306583217\\
8186	1.38818655366673\\
8187	1.45827111097811\\
8188	1.50924258077561\\
8189	1.50074630337809\\
8190	1.41490704168261\\
8191	1.23576563543977\\
8192	1.12832285638168\\
8193	1.06716449781852\\
8194	1.07212568982805\\
8195	1.08800158883547\\
8196	1.06363692176055\\
8197	1.04652860330476\\
8198	1.03977130059883\\
8199	1.02181097305879\\
8200	1.01649517240806\\
8201	1.00437072473902\\
8202	1.00651946246229\\
8203	0.988621238831433\\
8204	1.00898284020649\\
8205	1.0609416932194\\
8206	1.13114759457025\\
8207	1.11688143383646\\
8208	1.10710772603147\\
8209	1.19335839216978\\
8210	1.27490091079931\\
8211	1.32069316372277\\
8212	1.35812153779607\\
8213	1.33822121644125\\
8214	1.25465353509658\\
8215	1.10787211993941\\
8216	1.02853696673467\\
8217	0.992984851679174\\
8218	1.02547171963309\\
8219	1.01765998403277\\
8220	1.04302997615157\\
8221	1.03627457211846\\
8222	1.0273910585751\\
8223	1.03501838777361\\
8224	1.03107472764692\\
8225	1.02608232563071\\
8226	1.01913157797651\\
8227	1.04076470993463\\
8228	1.05591225568127\\
8229	1.13147724864141\\
8230	1.20050583128273\\
8231	1.21092004779116\\
8232	1.22727827958888\\
8233	1.3240600780942\\
8234	1.39244069774258\\
8235	1.46893212946179\\
8236	1.51602808039719\\
8237	1.52175378833141\\
8238	1.43381744711612\\
8239	1.25634251202815\\
8240	1.14411278300328\\
8241	1.07893922014659\\
8242	1.08249595157842\\
8243	1.10415301431682\\
8244	1.1113530615556\\
8245	1.14208344108835\\
8246	1.08988252774405\\
8247	1.08617933022818\\
8248	1.04891010714129\\
8249	1.01556956518918\\
8250	0.999107687409188\\
8251	1.01842916826334\\
8252	1.06919734801792\\
8253	1.08387837940913\\
8254	1.15665081923592\\
8255	1.15330301936905\\
8256	1.17803541395921\\
8257	1.26429577434067\\
8258	1.34052850892139\\
8259	1.41256064764692\\
8260	1.44716990994975\\
8261	1.43721383384624\\
8262	1.37121170446572\\
8263	1.23522503551443\\
8264	1.1133259236224\\
8265	1.08142880353185\\
8266	1.08747662709196\\
8267	1.07803823342975\\
8268	1.06843666002356\\
8269	1.07649942296848\\
8270	1.08007528732661\\
8271	1.0792310016126\\
8272	1.07066918070485\\
8273	1.06065849314233\\
8274	1.05005123990128\\
8275	1.06528723050508\\
8276	1.09501865839647\\
8277	1.11625793537462\\
8278	1.18815815615281\\
8279	1.19418736696983\\
8280	1.19066111681011\\
8281	1.28812568652554\\
8282	1.3974591587015\\
8283	1.45265425694484\\
8284	1.49036324833349\\
8285	1.48575315317391\\
8286	1.38683067106767\\
8287	1.22948229372524\\
8288	1.16061057236918\\
8289	1.15459011140171\\
8290	1.14218205517834\\
8291	1.11944930565554\\
8292	1.09814471223264\\
8293	1.10554553795501\\
8294	1.09699381947782\\
8295	1.08958299008165\\
8296	1.08903542471775\\
8297	1.05818312542576\\
8298	1.04918147400521\\
8299	1.030114273205\\
8300	1.07187759113221\\
8301	1.12385533098121\\
8302	1.16768367581928\\
8303	1.13128033532842\\
8304	1.13331632012856\\
8305	1.16607084424252\\
8306	1.23996021528096\\
8307	1.29753690410023\\
8308	1.33288953889236\\
8309	1.34051851350107\\
8310	1.32543300519182\\
8311	1.27855664308297\\
8312	1.2358580158805\\
8313	1.15269755251246\\
8314	1.12869765102761\\
8315	1.10875086506352\\
8316	1.11254384505299\\
8317	1.10585300144546\\
8318	1.11250514115532\\
8319	1.12213870815384\\
8320	1.10136397921569\\
8321	1.08226518655766\\
8322	1.07657003148479\\
8323	1.05820614973157\\
8324	1.09812647082653\\
8325	1.11773055954434\\
8326	1.16464990679266\\
8327	1.14966554450328\\
8328	1.14209261532454\\
8329	1.18590287644316\\
8330	1.24126338634169\\
8331	1.30121598283468\\
8332	1.34914177341647\\
8333	1.37740338956257\\
8334	1.38204674766572\\
8335	1.37089946391536\\
8336	1.36054424513417\\
8337	1.30026539428208\\
8338	1.25745577688457\\
8339	1.23938454639579\\
8340	1.22055288027389\\
8341	1.22601534039678\\
8342	1.2849899515783\\
8343	1.33184408450181\\
8344	1.29442679312928\\
8345	1.28112211739862\\
8346	1.21566747353943\\
8347	1.22138876627376\\
8348	1.26603493800594\\
8349	1.27025876091986\\
8350	1.3148742411285\\
8351	1.31747493451106\\
8352	1.33278817947977\\
8353	1.42557127688401\\
8354	1.51020999642886\\
8355	1.60622540199943\\
8356	1.68378323256844\\
8357	1.71338842621877\\
8358	1.67589858178971\\
8359	1.54792308216008\\
8360	1.45266439934437\\
8361	1.308101939105\\
8362	1.25410605809905\\
8363	1.19043037019656\\
8364	1.15923746168449\\
8365	1.17205581921283\\
8366	1.17146147575814\\
8367	1.18108884682878\\
8368	1.16070300319883\\
8369	1.13339031460694\\
8370	1.09649213410852\\
8371	1.10769519506942\\
8372	1.15531958448714\\
8373	1.24305770510821\\
8374	1.28411074585031\\
8375	1.26335544616121\\
8376	1.23779270026121\\
8377	1.32010525514882\\
8378	1.3996910418652\\
8379	1.46404775056346\\
8380	1.5084051681402\\
8381	1.50479530229249\\
8382	1.39752987611604\\
8383	1.18472360957616\\
8384	1.07249215929381\\
8385	1.03712495993264\\
8386	1.02894388101007\\
8387	1.04458459161933\\
8388	1.04251022364219\\
8389	1.05815952982376\\
8390	1.05453340453167\\
8391	1.04754112505741\\
8392	1.03317961007428\\
8393	1.02257155804158\\
8394	0.986993648029434\\
8395	0.984962378485203\\
8396	1.02362841987835\\
8397	1.03253805226395\\
8398	1.09600406978466\\
8399	1.0956979111307\\
8400	1.12439629125105\\
8401	1.23018345709125\\
8402	1.31264756102896\\
8403	1.40208533058254\\
8404	1.46212724082334\\
8405	1.49165601845845\\
8406	1.43656257889744\\
8407	1.27845727830626\\
8408	1.15322625014044\\
8409	1.0922551306894\\
8410	1.05534376284256\\
8411	1.04096593692628\\
8412	1.0232686110976\\
8413	1.02892795159319\\
8414	1.03048711374206\\
8415	1.02580990730217\\
8416	1.01886154516542\\
8417	0.992714311123493\\
8418	1.00162689959648\\
8419	1.00449952899897\\
8420	1.00270074324297\\
8421	1.06673401605805\\
8422	1.13652494192413\\
8423	1.177302564008\\
8424	1.17141882258827\\
8425	1.2511291782941\\
8426	1.34926432302792\\
8427	1.39853156336043\\
8428	1.43986736229519\\
8429	1.40144495491194\\
8430	1.3347464326993\\
8431	1.18287644649255\\
8432	1.11894665419411\\
8433	1.06826228576687\\
8434	1.16830444486746\\
8435	1.10876621003511\\
8436	1.09314059582075\\
8437	1.14521394076938\\
8438	1.11562659754462\\
8439	1.12927978444503\\
8440	1.13542358446419\\
8441	1.13299949651878\\
8442	1.137949720879\\
8443	1.10969009750241\\
8444	1.15358102137531\\
8445	1.2224167726239\\
8446	1.28541871278212\\
8447	1.27720183637854\\
8448	1.29077603760857\\
8449	1.39967049287388\\
8450	1.50875732661334\\
8451	1.60602079502665\\
8452	1.64724221348206\\
8453	1.64845068132474\\
8454	1.54502851175127\\
8455	1.35538279760793\\
8456	1.22728904570039\\
8457	1.1472869656789\\
8458	1.13966371185971\\
8459	1.1307546013623\\
8460	1.09404945033474\\
8461	1.07699413201875\\
8462	1.1154962028211\\
8463	1.14154793026088\\
8464	1.15660642498816\\
8465	1.15679095273787\\
8466	1.13392032562916\\
8467	1.1346093279584\\
8468	1.18951449088106\\
8469	1.24671693819689\\
8470	1.31108049510035\\
8471	1.27574426105955\\
8472	1.28452005859231\\
8473	1.45371625270884\\
8474	1.55444945752713\\
8475	1.64057248550187\\
8476	1.69454801532458\\
8477	1.72785420148592\\
8478	1.72842357489229\\
8479	1.66605015908381\\
8480	1.62697308854512\\
8481	1.50538761492827\\
8482	1.51401945432959\\
8483	1.49103287206764\\
8484	1.49173369002875\\
8485	1.49748666246559\\
8486	1.48120856615619\\
8487	1.4910369563797\\
8488	1.48429980518813\\
8489	1.44260824512219\\
8490	1.4244266050626\\
8491	1.37846846415113\\
8492	1.36404095639259\\
8493	1.40844480370983\\
8494	1.43482703637698\\
8495	1.40629640592344\\
8496	1.37247022689176\\
8497	1.45323711664685\\
8498	1.53183890857515\\
8499	1.60440030521016\\
8500	1.65373226866551\\
8501	1.66408306627363\\
8502	1.67121211674714\\
8503	1.6440584366952\\
8504	1.63037160358514\\
8505	1.56272482737773\\
8506	1.50352009382737\\
8507	1.46497699804384\\
8508	1.43831164789581\\
8509	1.42264660417006\\
8510	1.41628290800412\\
8511	1.46353933686541\\
8512	1.48018948358867\\
8513	1.46902353603382\\
8514	1.4937460865193\\
8515	1.48754359185873\\
8516	1.48282342416843\\
8517	1.47783415601633\\
8518	1.52278540880518\\
8519	1.49688531073387\\
8520	1.50751631022361\\
8521	1.64911409088117\\
8522	1.7380399270882\\
8523	1.84278219277516\\
8524	1.91102603910148\\
8525	1.88982056334449\\
8526	1.78308413693715\\
8527	1.58976825208901\\
8528	1.46459399453981\\
8529	1.33729922148427\\
8530	1.31338353420483\\
8531	1.29344127781202\\
8532	1.27974128008175\\
8533	1.33744107224697\\
8534	1.34986664334976\\
8535	1.38561570601464\\
8536	1.40619150985377\\
8537	1.40140447905447\\
8538	1.40487682906923\\
8539	1.37813571807168\\
8540	1.45959588515223\\
8541	1.5062553992216\\
8542	1.52724195454186\\
8543	1.49829696700234\\
8544	1.50436584696816\\
8545	1.64174260562922\\
8546	1.77467793906556\\
8547	1.87990479018297\\
8548	1.92268195856248\\
8549	1.90743694064729\\
8550	1.81309794258038\\
8551	1.57370382939131\\
8552	1.46080151982663\\
8553	1.35210966754684\\
8554	1.3371826231411\\
8555	1.33529123118354\\
8556	1.33106044673427\\
8557	1.30626440778397\\
8558	1.32526240118229\\
8559	1.34490157267694\\
8560	1.33478393415608\\
8561	1.34499941511857\\
8562	1.31686967045448\\
8563	1.34250411074167\\
8564	1.34247915014425\\
8565	1.43732234032891\\
8566	1.50367726988001\\
8567	1.46049385895492\\
8568	1.48084589004699\\
8569	1.62525815585464\\
8570	1.78623268844535\\
8571	1.90465573570297\\
8572	1.97750488782112\\
8573	1.98915052955196\\
8574	1.91493708276279\\
8575	1.76732578903348\\
8576	1.67432087898782\\
8577	1.54154792305925\\
8578	1.49003088867765\\
8579	1.44273791779057\\
8580	1.42963955181686\\
8581	1.426840692727\\
8582	1.36649313281802\\
8583	1.37435226912007\\
8584	1.37034179505831\\
8585	1.38464622224157\\
8586	1.4187142441262\\
8587	1.41363638995712\\
8588	1.44774574600538\\
8589	1.51733232224308\\
8590	1.54289111628408\\
8591	1.47906401168228\\
8592	1.3373111019866\\
8593	1.4181156898758\\
8594	1.51661824133831\\
8595	1.6331008785419\\
8596	1.72602632666105\\
8597	1.76685662112437\\
8598	1.76687434399964\\
8599	1.73689244027631\\
8600	1.76018739273318\\
8601	1.69901416154988\\
8602	1.65404534540226\\
8603	1.59499036599155\\
8604	1.55495011923529\\
8605	1.52346443941059\\
8606	1.52208670164402\\
8607	1.52343872800384\\
8608	1.51176478191814\\
8609	1.49477712082175\\
8610	1.47038786551795\\
8611	1.44654468840921\\
8612	1.40319965167553\\
8613	1.39319031460402\\
8614	1.36545934586638\\
8615	1.28848127907015\\
8616	1.27387531757695\\
8617	1.37441531619016\\
8618	1.44671595986635\\
8619	1.51160569489657\\
8620	1.54184993262004\\
8621	1.52687481016744\\
8622	1.46762806637376\\
8623	1.37037682837617\\
8624	1.32753517981743\\
8625	1.2539304156495\\
8626	1.20860842275527\\
8627	1.18510279314986\\
8628	1.18860509464683\\
8629	1.19615268691176\\
8630	1.19993024969369\\
8631	1.18405164532763\\
8632	1.18209855574803\\
8633	1.17345005485258\\
8634	1.19418586716758\\
8635	1.20054728178339\\
8636	1.25373724715499\\
8637	1.29167717473871\\
8638	1.37683124961439\\
8639	1.33296865343761\\
8640	1.33756531417263\\
8641	1.42181667740224\\
8642	1.52798157421027\\
8643	1.63215013860872\\
8644	1.70711197062328\\
8645	1.71713852234749\\
8646	1.69099616008878\\
8647	1.62720733793361\\
8648	1.56788242793241\\
8649	1.46159645706353\\
8650	1.3887735711199\\
8651	1.32469768234208\\
8652	1.2813917876019\\
8653	1.27992309397903\\
8654	1.27157889541129\\
8655	1.28813660915137\\
8656	1.29025021914606\\
8657	1.29645075301038\\
8658	1.28356367416061\\
8659	1.28464153738118\\
8660	1.28931907440499\\
8661	1.31981032823594\\
8662	1.3663827499791\\
8663	1.34595913322719\\
8664	1.32182417817365\\
8665	1.41534059051828\\
8666	1.50373159945767\\
8667	1.57557945415222\\
8668	1.64875183624293\\
8669	1.68894818416681\\
8670	1.68789467932532\\
8671	1.63702247323077\\
8672	1.62543211153994\\
8673	1.53699596934418\\
8674	1.45226720361628\\
8675	1.42473165753451\\
8676	1.40272487441901\\
8677	1.38947610209476\\
8678	1.39421554951891\\
8679	1.37161245835094\\
8680	1.30944101650075\\
8681	1.247635651852\\
8682	1.18096573031712\\
8683	1.16639149896853\\
8684	1.17607403372336\\
8685	1.16380512695592\\
8686	1.18021427257023\\
8687	1.18503415715792\\
8688	1.14404970588347\\
8689	1.21431908874673\\
8690	1.29065111632322\\
8691	1.37458449659015\\
8692	1.43538413224968\\
8693	1.46415401467346\\
8694	1.4461163806062\\
8695	1.35749888266681\\
8696	1.30812927646506\\
8697	1.21321622168228\\
8698	1.21657193783287\\
8699	1.21091471709704\\
8700	1.20401693677836\\
8701	1.21976546691653\\
8702	1.21608718284478\\
8703	1.22338064446681\\
8704	1.20445092203241\\
8705	1.18508735322561\\
8706	1.12926075635526\\
8707	1.10922701088948\\
8708	1.12026200956422\\
8709	1.14839566014159\\
8710	1.1845788772179\\
8711	1.14680187464326\\
8712	1.15168090420463\\
8713	1.23439092099838\\
8714	1.29764072001228\\
8715	1.37056622249755\\
8716	1.41740022872829\\
8717	1.42291367809524\\
8718	1.38408785617229\\
8719	1.29978002028562\\
8720	1.2508312619189\\
8721	1.18371747958621\\
8722	1.15673300303895\\
8723	1.15366871181385\\
8724	1.15643342164923\\
8725	1.14618583267885\\
8726	1.14457399673234\\
8727	1.15368258138542\\
8728	1.14743425371603\\
8729	1.12586706954808\\
8730	1.09176196233481\\
8731	1.10101955832463\\
8732	1.13645231798567\\
8733	1.15185910820279\\
8734	1.20781463026626\\
8735	1.16425463265425\\
8736	1.16208789058946\\
8737	1.2516302016378\\
8738	1.32688103504547\\
8739	1.39402901621521\\
8740	1.42823601179647\\
8741	1.44294004489756\\
8742	1.41427969104101\\
8743	1.33319658219332\\
8744	1.28331933389827\\
8745	1.20406101031944\\
8746	1.17747007448541\\
8747	1.20503401423872\\
8748	1.20825319533468\\
8749	1.22950298882453\\
8750	1.20746979830123\\
8751	1.22318878187661\\
8752	1.21691420791927\\
8753	1.19401951496614\\
8754	1.19550291861363\\
8755	1.21334587223175\\
8756	1.28311931989187\\
8757	1.2904425736041\\
8758	1.33421323961541\\
8759	1.28308881880662\\
8760	1.24069928656258\\
};
\end{axis}
\end{tikzpicture}%
    \caption{Ratio between the maximal Belgian production and the real demand level (the demand level minus the power coming from renewable energies)}
    \label{fig:ratio2}
\end{figure}

\paragraph{EDR - ImpExp} if we compare the two figures [\ref{fig:ratio2}] and [\ref{fig:EDR_R}], we observe that when the belgian ratio between production and demand level approaches the unit, the reserve price increases. The system is aware of the risk for the system reliability and is willing to pay more in exchange for reserves. For the ImpExp model, the behavior is similar, except that the demand level isn't reduced by the importation. Therefore, the pressure on the system is even stronger. However, it can rely on an external generator, if it's ready to pay the price.

\paragraph{ORDC} The reserve price is generally higher in the third model than in the two previous models. The third model attaches great value to reserves, even when the amount has exceeded the required level of the previous models. Meaning that a lot of reserves will be made in order to guarantee system reliability. Furthermore, the two peaks highlights the fact that the system must be very strained for the system operator to realize that : it's enjoyable to have reserves. But it's even better if the energy price isn't too high. Let's remind that operating reserve demand curves have been proposed as an approach for achieving high energy prices in conditions of scarcity through prices spikes in that are more frequent but less elevated. However, it clearly doesn't work in our case. Therefore, it may be interesting to review the parameters of the operating reserve demand curves. 

\begin{table}[H]
\centering
\begin{tabular}{l | c  c  c  c}
model & mean & variance & min & max \\
\hline
EDR & $0.26$ &  $1.91$ & $0$ &  $34.33$ \\
ImpExp & $0.52$ &  $1.14$ & $0$ &  $13.09$ \\
ORDC & $ 6.14$ & $154.2$ & $0$ & $718.04$ \\
\end{tabular}
\caption{Statistics on the predicted reserve prices}
\end{table}

\begin{figure}[H]
    \centering
    \setlength\fheight{0.3\textwidth}
    \setlength\fwidth{0.85\textwidth}
    % This file was created by matlab2tikz.
% Minimal pgfplots version: 1.3
%
%The latest updates can be retrieved from
%  http://www.mathworks.com/matlabcentral/fileexchange/22022-matlab2tikz
%where you can also make suggestions and rate matlab2tikz.
%
\definecolor{mycolor1}{rgb}{0.84706,0.16078,0.00000}%
%
\begin{tikzpicture}

\begin{axis}[%
width=\fwidth,
height=\fheight,
at={(\fwidth,\fheight)},
scale only axis,
separate axis lines,
every outer x axis line/.append style={black},
every x tick label/.append style={font=\color{black}},
xmin=0,
xmax=8760,
xlabel={time [hour]},
xtick={0,1000,2000,3000,4000,5000,6000,7000,8000},
xmajorgrids,
every outer y axis line/.append style={black},
every y tick label/.append style={font=\color{black}},
ymin=0,
ymax=35,
ymajorgrids,
title style={font=\bfseries},
title={EDR - Reserve price [\euro/MWh]}
]
\addplot [color=mycolor1,solid,line width=1.0pt,forget plot]
  table[row sep=crcr]{%
1	0\\
2	0\\
3	0\\
4	0\\
5	0\\
6	0\\
7	0\\
8	0\\
9	0\\
10	0\\
11	0\\
12	0\\
13	0\\
14	0\\
15	0\\
16	0\\
17	0\\
18	0\\
19	0\\
20	0\\
21	0\\
22	0\\
23	0\\
24	0\\
25	0\\
26	0\\
27	0\\
28	0\\
29	0\\
30	0\\
31	0\\
32	0\\
33	0\\
34	0\\
35	0\\
36	0\\
37	0\\
38	0\\
39	0\\
40	0\\
41	0\\
42	0\\
43	0\\
44	0\\
45	0\\
46	0\\
47	0\\
48	0\\
49	0\\
50	0\\
51	0\\
52	0\\
53	0\\
54	0\\
55	0\\
56	0\\
57	0\\
58	0\\
59	0\\
60	0\\
61	0\\
62	0\\
63	0\\
64	0\\
65	0\\
66	0\\
67	0\\
68	0\\
69	0\\
70	0\\
71	0\\
72	0\\
73	0\\
74	0\\
75	0\\
76	0\\
77	0\\
78	0\\
79	0\\
80	0\\
81	0\\
82	0\\
83	0\\
84	0\\
85	0\\
86	0\\
87	0\\
88	0\\
89	0\\
90	0\\
91	0\\
92	0\\
93	0\\
94	0\\
95	0\\
96	0\\
97	0\\
98	0\\
99	0\\
100	0\\
101	0\\
102	0\\
103	0\\
104	0\\
105	0\\
106	0\\
107	0\\
108	0\\
109	0\\
110	0\\
111	0\\
112	0\\
113	0\\
114	0\\
115	0\\
116	0\\
117	0\\
118	0\\
119	0\\
120	0\\
121	0\\
122	0\\
123	0\\
124	0\\
125	0\\
126	0\\
127	0\\
128	0\\
129	0\\
130	0\\
131	0\\
132	0\\
133	0\\
134	0\\
135	0\\
136	0\\
137	0\\
138	0\\
139	0\\
140	0\\
141	0\\
142	0\\
143	0\\
144	0\\
145	0\\
146	0\\
147	0\\
148	0\\
149	0\\
150	0\\
151	0\\
152	0\\
153	0\\
154	0\\
155	0\\
156	0\\
157	0\\
158	0\\
159	0\\
160	0\\
161	0\\
162	0\\
163	0\\
164	0\\
165	0\\
166	0\\
167	0\\
168	0\\
169	0\\
170	0\\
171	0\\
172	0\\
173	0\\
174	0\\
175	0\\
176	0\\
177	0\\
178	0\\
179	0\\
180	0\\
181	0\\
182	1.401674\\
183	1.69738\\
184	2.982675\\
185	2.982675\\
186	3.625322\\
187	4.879875\\
188	4.879875\\
189	2.982675\\
190	0\\
191	0\\
192	0\\
193	0\\
194	0\\
195	0\\
196	0\\
197	0\\
198	0\\
199	0\\
200	0\\
201	0\\
202	0\\
203	0\\
204	0\\
205	0\\
206	0\\
207	0\\
208	0\\
209	0\\
210	0\\
211	0\\
212	0\\
213	0\\
214	0\\
215	0\\
216	0\\
217	0\\
218	0\\
219	0\\
220	0\\
221	0\\
222	0\\
223	0\\
224	0\\
225	0\\
226	0\\
227	0\\
228	0\\
229	0\\
230	0\\
231	0\\
232	0\\
233	0\\
234	0\\
235	0\\
236	0\\
237	0\\
238	0\\
239	0\\
240	0\\
241	0\\
242	0\\
243	0\\
244	0\\
245	0\\
246	0\\
247	0\\
248	0\\
249	0\\
250	0\\
251	0\\
252	0\\
253	0\\
254	0\\
255	0\\
256	0\\
257	0\\
258	0\\
259	0\\
260	0\\
261	0\\
262	0\\
263	0\\
264	0\\
265	0\\
266	0\\
267	0\\
268	0\\
269	0\\
270	0\\
271	0\\
272	0\\
273	0\\
274	0\\
275	0\\
276	0\\
277	0\\
278	0\\
279	0\\
280	0\\
281	0\\
282	0\\
283	0\\
284	0\\
285	0\\
286	0\\
287	0\\
288	0\\
289	0\\
290	0\\
291	0\\
292	0\\
293	0\\
294	0\\
295	0\\
296	0\\
297	0\\
298	0\\
299	0\\
300	0\\
301	0\\
302	0\\
303	0\\
304	0\\
305	0\\
306	0\\
307	0\\
308	0\\
309	0\\
310	0\\
311	0\\
312	0\\
313	0\\
314	0\\
315	0\\
316	0\\
317	0\\
318	0\\
319	0\\
320	0\\
321	0\\
322	0\\
323	0\\
324	0\\
325	0\\
326	0\\
327	0\\
328	0\\
329	0\\
330	0\\
331	0\\
332	0\\
333	0\\
334	0\\
335	0\\
336	0\\
337	0\\
338	0\\
339	0\\
340	0\\
341	0\\
342	0\\
343	0\\
344	0\\
345	0\\
346	0\\
347	0\\
348	0\\
349	0\\
350	0\\
351	0\\
352	0\\
353	0\\
354	0\\
355	0\\
356	0\\
357	0\\
358	0\\
359	0\\
360	0\\
361	0\\
362	0\\
363	0\\
364	0\\
365	0\\
366	0\\
367	0\\
368	0\\
369	0\\
370	0\\
371	0\\
372	0\\
373	0\\
374	0\\
375	0\\
376	0\\
377	0\\
378	0\\
379	0\\
380	0\\
381	0\\
382	0\\
383	0\\
384	0\\
385	0\\
386	0\\
387	0\\
388	0\\
389	0\\
390	0\\
391	0\\
392	0\\
393	0\\
394	0\\
395	0\\
396	0\\
397	0\\
398	0\\
399	0\\
400	0\\
401	0\\
402	0\\
403	0\\
404	0\\
405	0\\
406	0\\
407	0\\
408	0\\
409	0\\
410	0\\
411	0\\
412	0\\
413	0\\
414	0\\
415	0\\
416	0\\
417	0\\
418	0\\
419	0\\
420	0\\
421	0\\
422	0\\
423	0\\
424	0\\
425	0\\
426	0\\
427	0\\
428	0\\
429	0\\
430	0\\
431	0\\
432	0\\
433	0\\
434	0\\
435	0\\
436	0\\
437	0\\
438	0\\
439	0\\
440	0\\
441	0\\
442	0\\
443	0\\
444	0\\
445	0\\
446	0\\
447	0\\
448	0\\
449	0\\
450	0\\
451	0\\
452	0\\
453	0\\
454	0\\
455	0\\
456	0\\
457	0\\
458	0\\
459	0\\
460	0\\
461	0\\
462	0\\
463	0\\
464	0\\
465	0\\
466	0\\
467	0\\
468	0\\
469	0\\
470	0\\
471	0\\
472	0\\
473	0\\
474	0\\
475	0\\
476	0\\
477	0\\
478	0\\
479	0\\
480	0\\
481	0\\
482	0\\
483	0\\
484	0\\
485	0\\
486	0\\
487	0\\
488	0\\
489	0\\
490	0\\
491	0\\
492	0\\
493	0\\
494	0\\
495	0\\
496	0\\
497	0\\
498	0\\
499	0\\
500	0\\
501	0\\
502	0\\
503	0\\
504	0\\
505	0\\
506	0\\
507	0\\
508	0\\
509	0\\
510	0\\
511	0\\
512	0\\
513	0\\
514	0\\
515	0\\
516	0\\
517	0\\
518	0\\
519	0\\
520	0\\
521	0\\
522	0\\
523	2.331976\\
524	0.532124\\
525	0\\
526	0\\
527	0\\
528	0\\
529	0\\
530	0\\
531	0\\
532	0\\
533	0\\
534	0\\
535	0\\
536	0\\
537	0\\
538	0\\
539	0\\
540	0\\
541	0\\
542	0\\
543	0\\
544	0\\
545	0\\
546	0\\
547	0\\
548	0\\
549	0\\
550	0\\
551	0\\
552	0\\
553	0\\
554	0\\
555	0\\
556	0\\
557	0\\
558	0\\
559	0\\
560	0\\
561	0\\
562	0\\
563	0\\
564	0\\
565	0\\
566	0\\
567	0\\
568	0\\
569	0\\
570	0\\
571	0\\
572	0\\
573	0\\
574	0\\
575	0\\
576	0\\
577	0\\
578	0\\
579	0\\
580	0\\
581	0\\
582	0\\
583	0\\
584	0\\
585	0\\
586	0\\
587	0\\
588	0\\
589	0\\
590	0\\
591	0\\
592	0\\
593	0\\
594	0\\
595	0\\
596	0\\
597	0\\
598	0\\
599	0\\
600	0\\
601	0\\
602	0\\
603	0\\
604	0\\
605	0\\
606	0\\
607	0\\
608	0\\
609	0\\
610	0\\
611	0\\
612	0\\
613	0\\
614	0\\
615	0\\
616	0\\
617	0\\
618	0\\
619	0\\
620	0\\
621	0\\
622	0\\
623	0\\
624	0\\
625	0\\
626	0\\
627	0\\
628	0\\
629	0\\
630	0\\
631	0\\
632	0\\
633	0\\
634	0\\
635	0\\
636	0\\
637	0\\
638	0\\
639	0\\
640	0\\
641	0\\
642	0\\
643	0\\
644	0\\
645	0\\
646	0\\
647	0\\
648	0\\
649	0\\
650	0\\
651	0\\
652	0\\
653	0\\
654	0\\
655	0\\
656	0\\
657	0\\
658	0\\
659	0\\
660	0\\
661	0\\
662	0\\
663	0\\
664	0\\
665	0\\
666	0\\
667	0\\
668	0\\
669	0\\
670	0\\
671	0\\
672	0\\
673	0\\
674	0\\
675	0\\
676	0\\
677	0\\
678	0\\
679	0\\
680	0\\
681	0\\
682	0\\
683	0\\
684	0\\
685	0\\
686	0\\
687	0\\
688	0\\
689	0\\
690	0\\
691	0\\
692	0\\
693	0\\
694	0\\
695	0\\
696	0\\
697	0\\
698	0\\
699	0\\
700	0\\
701	0\\
702	0\\
703	0\\
704	0\\
705	0\\
706	0\\
707	0\\
708	0\\
709	0\\
710	0\\
711	0\\
712	0\\
713	0\\
714	0\\
715	0\\
716	0\\
717	0\\
718	0\\
719	0\\
720	0\\
721	0\\
722	0\\
723	0\\
724	0\\
725	0\\
726	0\\
727	0\\
728	0\\
729	0\\
730	0\\
731	0\\
732	0\\
733	0\\
734	0\\
735	0\\
736	0\\
737	0\\
738	0\\
739	0\\
740	0\\
741	0\\
742	0\\
743	0\\
744	0\\
745	0\\
746	0\\
747	0\\
748	0\\
749	0\\
750	0\\
751	0\\
752	0\\
753	0\\
754	0\\
755	0\\
756	0\\
757	0\\
758	0\\
759	0\\
760	0\\
761	0\\
762	0\\
763	0\\
764	0\\
765	0\\
766	0\\
767	0\\
768	0\\
769	0\\
770	0\\
771	0\\
772	0\\
773	0\\
774	0\\
775	0\\
776	0\\
777	0\\
778	0\\
779	0\\
780	0\\
781	0\\
782	0\\
783	0\\
784	0\\
785	0\\
786	0\\
787	0\\
788	0\\
789	0\\
790	0\\
791	0\\
792	0\\
793	0\\
794	0\\
795	0\\
796	0\\
797	0\\
798	0\\
799	0\\
800	0\\
801	0\\
802	0\\
803	0\\
804	0\\
805	0\\
806	0\\
807	0\\
808	0\\
809	0\\
810	0\\
811	0\\
812	0\\
813	0\\
814	0\\
815	0\\
816	0\\
817	0\\
818	0\\
819	0\\
820	0\\
821	0\\
822	0\\
823	0\\
824	0\\
825	0\\
826	0\\
827	0\\
828	0\\
829	0\\
830	0\\
831	0\\
832	0\\
833	0\\
834	0\\
835	0\\
836	0\\
837	0\\
838	0\\
839	0\\
840	0\\
841	0\\
842	0\\
843	0\\
844	0\\
845	0\\
846	0\\
847	0\\
848	0\\
849	0\\
850	0\\
851	0\\
852	0\\
853	0\\
854	0\\
855	0\\
856	0\\
857	0\\
858	0\\
859	0\\
860	0\\
861	0\\
862	0\\
863	0\\
864	0\\
865	0\\
866	0\\
867	0\\
868	0\\
869	0\\
870	0\\
871	0\\
872	0\\
873	0\\
874	0\\
875	0\\
876	0\\
877	0\\
878	0\\
879	0\\
880	0\\
881	0\\
882	0\\
883	0\\
884	0\\
885	0\\
886	0\\
887	0\\
888	0\\
889	0\\
890	0\\
891	0\\
892	0\\
893	0\\
894	0\\
895	0\\
896	0\\
897	0\\
898	0\\
899	0\\
900	0\\
901	0\\
902	0\\
903	0\\
904	0\\
905	0\\
906	0\\
907	0\\
908	0\\
909	0\\
910	0\\
911	0\\
912	0\\
913	0\\
914	0\\
915	0\\
916	0\\
917	0\\
918	0\\
919	0\\
920	0\\
921	0\\
922	0\\
923	0\\
924	0\\
925	0\\
926	0\\
927	0\\
928	0\\
929	0\\
930	0\\
931	0\\
932	0\\
933	0\\
934	0\\
935	0\\
936	0\\
937	0\\
938	0\\
939	0\\
940	0\\
941	0\\
942	0\\
943	0\\
944	0\\
945	0\\
946	0\\
947	0\\
948	0\\
949	0\\
950	0\\
951	0\\
952	0\\
953	0\\
954	0\\
955	0\\
956	0\\
957	0\\
958	0\\
959	0\\
960	0\\
961	0\\
962	0\\
963	0\\
964	0\\
965	0\\
966	0\\
967	0\\
968	0\\
969	0\\
970	0\\
971	0\\
972	0\\
973	0\\
974	0\\
975	0\\
976	0\\
977	0\\
978	0\\
979	0\\
980	0\\
981	0\\
982	0\\
983	0\\
984	0\\
985	0\\
986	0\\
987	0\\
988	0\\
989	0\\
990	0\\
991	0\\
992	0\\
993	0\\
994	0\\
995	0\\
996	0\\
997	0\\
998	0\\
999	0\\
1000	0\\
1001	0\\
1002	0\\
1003	0\\
1004	0\\
1005	0\\
1006	0\\
1007	0\\
1008	0\\
1009	0\\
1010	0\\
1011	0\\
1012	0\\
1013	0\\
1014	0\\
1015	0\\
1016	0\\
1017	0\\
1018	0\\
1019	0\\
1020	0\\
1021	0\\
1022	0\\
1023	0\\
1024	0\\
1025	0\\
1026	0\\
1027	0\\
1028	0\\
1029	0\\
1030	0\\
1031	0\\
1032	0\\
1033	0\\
1034	0\\
1035	0\\
1036	0\\
1037	0\\
1038	0\\
1039	0\\
1040	0\\
1041	0\\
1042	0\\
1043	0\\
1044	0\\
1045	0\\
1046	0\\
1047	0\\
1048	0\\
1049	0\\
1050	0\\
1051	0\\
1052	0\\
1053	0\\
1054	0\\
1055	0\\
1056	0\\
1057	0\\
1058	0\\
1059	0\\
1060	0\\
1061	0\\
1062	0\\
1063	0\\
1064	0\\
1065	0\\
1066	0\\
1067	0\\
1068	0\\
1069	0\\
1070	0\\
1071	0\\
1072	0\\
1073	0\\
1074	0\\
1075	0\\
1076	0\\
1077	0\\
1078	0\\
1079	0\\
1080	0\\
1081	0\\
1082	0\\
1083	0\\
1084	0\\
1085	0\\
1086	0\\
1087	0\\
1088	0\\
1089	0\\
1090	0\\
1091	0\\
1092	0\\
1093	0\\
1094	0\\
1095	0\\
1096	0\\
1097	0\\
1098	0\\
1099	0\\
1100	0\\
1101	0\\
1102	0\\
1103	0\\
1104	0\\
1105	0\\
1106	0\\
1107	0\\
1108	0\\
1109	0\\
1110	0\\
1111	0\\
1112	0\\
1113	0\\
1114	0\\
1115	0\\
1116	0\\
1117	0\\
1118	0\\
1119	0\\
1120	0\\
1121	0\\
1122	0\\
1123	0\\
1124	0\\
1125	0\\
1126	0\\
1127	0\\
1128	0\\
1129	0\\
1130	0\\
1131	0\\
1132	0\\
1133	0\\
1134	0\\
1135	0\\
1136	0\\
1137	0\\
1138	0\\
1139	0\\
1140	0\\
1141	0\\
1142	0\\
1143	0\\
1144	0\\
1145	0\\
1146	0\\
1147	0\\
1148	0\\
1149	0\\
1150	0\\
1151	0\\
1152	0\\
1153	0\\
1154	0\\
1155	0\\
1156	0\\
1157	0\\
1158	0\\
1159	0\\
1160	0\\
1161	0\\
1162	0\\
1163	0\\
1164	0\\
1165	0\\
1166	0\\
1167	0\\
1168	0\\
1169	0\\
1170	0\\
1171	0\\
1172	0\\
1173	0\\
1174	0\\
1175	0\\
1176	0\\
1177	0\\
1178	0\\
1179	0\\
1180	0\\
1181	0\\
1182	0\\
1183	0\\
1184	0\\
1185	0\\
1186	0\\
1187	0\\
1188	0\\
1189	0\\
1190	0\\
1191	0\\
1192	0\\
1193	0\\
1194	0\\
1195	0\\
1196	0\\
1197	0\\
1198	0\\
1199	0\\
1200	0\\
1201	0\\
1202	0\\
1203	0\\
1204	0\\
1205	0\\
1206	0\\
1207	0\\
1208	0\\
1209	0\\
1210	0\\
1211	0\\
1212	0\\
1213	0\\
1214	0\\
1215	0\\
1216	0\\
1217	0\\
1218	0\\
1219	0\\
1220	0\\
1221	0\\
1222	0\\
1223	0\\
1224	0\\
1225	0\\
1226	0\\
1227	0\\
1228	0\\
1229	0\\
1230	0\\
1231	0\\
1232	0\\
1233	0\\
1234	0\\
1235	0\\
1236	0\\
1237	0\\
1238	0\\
1239	0\\
1240	0\\
1241	0\\
1242	0\\
1243	0\\
1244	0\\
1245	0\\
1246	0\\
1247	0\\
1248	0\\
1249	0\\
1250	0\\
1251	0\\
1252	0\\
1253	0\\
1254	0\\
1255	0\\
1256	0\\
1257	0\\
1258	0\\
1259	0\\
1260	0\\
1261	0\\
1262	0\\
1263	0\\
1264	0\\
1265	0\\
1266	0\\
1267	0\\
1268	0\\
1269	0\\
1270	0\\
1271	0\\
1272	0\\
1273	0\\
1274	0\\
1275	0\\
1276	0\\
1277	0\\
1278	0\\
1279	0\\
1280	0\\
1281	0\\
1282	0\\
1283	0\\
1284	0\\
1285	0\\
1286	0\\
1287	0\\
1288	0\\
1289	0\\
1290	0\\
1291	0\\
1292	0\\
1293	0\\
1294	0\\
1295	0\\
1296	0\\
1297	0\\
1298	0\\
1299	0\\
1300	0\\
1301	0\\
1302	0\\
1303	0\\
1304	0\\
1305	0\\
1306	0\\
1307	0\\
1308	0\\
1309	0\\
1310	0\\
1311	0\\
1312	0\\
1313	0\\
1314	0\\
1315	0\\
1316	0\\
1317	0\\
1318	0\\
1319	0\\
1320	0\\
1321	0\\
1322	0\\
1323	0\\
1324	0\\
1325	0\\
1326	0\\
1327	0\\
1328	0\\
1329	0\\
1330	0\\
1331	0\\
1332	0\\
1333	0\\
1334	0\\
1335	0\\
1336	0\\
1337	0\\
1338	0\\
1339	0\\
1340	0\\
1341	0\\
1342	0\\
1343	0\\
1344	0\\
1345	0\\
1346	0\\
1347	0\\
1348	0\\
1349	0\\
1350	0\\
1351	0\\
1352	0\\
1353	0\\
1354	0\\
1355	0\\
1356	0\\
1357	0\\
1358	0\\
1359	0\\
1360	0\\
1361	0\\
1362	0\\
1363	0\\
1364	0\\
1365	0\\
1366	0\\
1367	0\\
1368	0\\
1369	0\\
1370	0\\
1371	0\\
1372	0\\
1373	0\\
1374	0\\
1375	0\\
1376	0\\
1377	0\\
1378	0\\
1379	0\\
1380	0\\
1381	0\\
1382	0\\
1383	0\\
1384	0\\
1385	0\\
1386	0\\
1387	0\\
1388	0\\
1389	0\\
1390	0\\
1391	0\\
1392	0\\
1393	0\\
1394	0\\
1395	0\\
1396	0\\
1397	0\\
1398	0\\
1399	0\\
1400	0\\
1401	0\\
1402	0\\
1403	0\\
1404	0\\
1405	0\\
1406	0\\
1407	0\\
1408	0\\
1409	0\\
1410	0\\
1411	0\\
1412	0\\
1413	0\\
1414	0\\
1415	0\\
1416	0\\
1417	0\\
1418	0\\
1419	0\\
1420	0\\
1421	0\\
1422	0\\
1423	0\\
1424	0\\
1425	0\\
1426	0\\
1427	0\\
1428	0\\
1429	0\\
1430	0\\
1431	0\\
1432	0\\
1433	0\\
1434	0\\
1435	0\\
1436	0\\
1437	0\\
1438	0\\
1439	0\\
1440	0\\
1441	0\\
1442	0\\
1443	0\\
1444	0\\
1445	0\\
1446	0\\
1447	0\\
1448	0\\
1449	0\\
1450	0\\
1451	0\\
1452	0\\
1453	0\\
1454	0\\
1455	0\\
1456	0\\
1457	0\\
1458	0\\
1459	0\\
1460	0\\
1461	0\\
1462	0\\
1463	0\\
1464	0\\
1465	0\\
1466	0\\
1467	0\\
1468	0\\
1469	0\\
1470	0\\
1471	0\\
1472	0\\
1473	0\\
1474	0\\
1475	0\\
1476	0\\
1477	0\\
1478	0\\
1479	0\\
1480	0\\
1481	0\\
1482	0\\
1483	0\\
1484	0\\
1485	0\\
1486	0\\
1487	0\\
1488	0\\
1489	0\\
1490	0\\
1491	0\\
1492	0\\
1493	0\\
1494	0\\
1495	0\\
1496	0\\
1497	0\\
1498	0\\
1499	0\\
1500	0\\
1501	0\\
1502	0\\
1503	0\\
1504	0\\
1505	0\\
1506	0\\
1507	0\\
1508	0\\
1509	0\\
1510	0\\
1511	0\\
1512	0\\
1513	0\\
1514	0\\
1515	0\\
1516	0\\
1517	0\\
1518	0\\
1519	0\\
1520	0\\
1521	0\\
1522	0\\
1523	0\\
1524	0\\
1525	0\\
1526	0\\
1527	0\\
1528	0\\
1529	0\\
1530	0\\
1531	0\\
1532	0\\
1533	0\\
1534	0\\
1535	0\\
1536	0\\
1537	0\\
1538	0\\
1539	0\\
1540	0\\
1541	0\\
1542	0\\
1543	0\\
1544	0\\
1545	0\\
1546	0\\
1547	0\\
1548	0\\
1549	0\\
1550	0\\
1551	0\\
1552	0\\
1553	0\\
1554	0\\
1555	0\\
1556	0\\
1557	0\\
1558	0\\
1559	0\\
1560	0\\
1561	0\\
1562	0\\
1563	0\\
1564	0\\
1565	0\\
1566	0\\
1567	0\\
1568	0\\
1569	0\\
1570	0\\
1571	0\\
1572	0\\
1573	0\\
1574	0\\
1575	0\\
1576	0\\
1577	0\\
1578	0\\
1579	0\\
1580	0\\
1581	0\\
1582	0\\
1583	0\\
1584	0\\
1585	0\\
1586	0\\
1587	0\\
1588	0\\
1589	0\\
1590	0\\
1591	0\\
1592	0\\
1593	0\\
1594	0\\
1595	0\\
1596	0\\
1597	0\\
1598	0\\
1599	0\\
1600	0\\
1601	0\\
1602	0\\
1603	0\\
1604	0\\
1605	0\\
1606	0\\
1607	0\\
1608	0\\
1609	0\\
1610	0\\
1611	0\\
1612	0\\
1613	0\\
1614	0\\
1615	0\\
1616	0\\
1617	0\\
1618	0\\
1619	0\\
1620	0\\
1621	0\\
1622	0\\
1623	0\\
1624	0\\
1625	0\\
1626	0\\
1627	0\\
1628	0\\
1629	0\\
1630	0\\
1631	0\\
1632	0\\
1633	0\\
1634	0\\
1635	0\\
1636	0\\
1637	0\\
1638	0\\
1639	0\\
1640	0\\
1641	0\\
1642	0\\
1643	0\\
1644	0\\
1645	0\\
1646	0\\
1647	0\\
1648	0\\
1649	0\\
1650	0\\
1651	0\\
1652	0\\
1653	0\\
1654	0\\
1655	0\\
1656	0\\
1657	0\\
1658	0\\
1659	0\\
1660	0\\
1661	0\\
1662	0\\
1663	0\\
1664	0\\
1665	0\\
1666	0\\
1667	0\\
1668	0\\
1669	0\\
1670	0\\
1671	0\\
1672	0\\
1673	0\\
1674	0\\
1675	0\\
1676	0\\
1677	0\\
1678	0\\
1679	0\\
1680	0\\
1681	0\\
1682	0\\
1683	0\\
1684	0\\
1685	0\\
1686	0\\
1687	0\\
1688	0\\
1689	0\\
1690	0\\
1691	0\\
1692	0\\
1693	0\\
1694	0\\
1695	0\\
1696	0\\
1697	0\\
1698	0\\
1699	0\\
1700	0\\
1701	0\\
1702	0\\
1703	0\\
1704	0\\
1705	0\\
1706	0\\
1707	0\\
1708	0\\
1709	0\\
1710	0\\
1711	0\\
1712	0\\
1713	0\\
1714	0\\
1715	0\\
1716	0\\
1717	0\\
1718	0\\
1719	0\\
1720	0\\
1721	0\\
1722	0\\
1723	0\\
1724	0\\
1725	0\\
1726	0\\
1727	0\\
1728	0\\
1729	0\\
1730	0\\
1731	0\\
1732	0\\
1733	0\\
1734	0\\
1735	0\\
1736	0\\
1737	0\\
1738	0\\
1739	0\\
1740	0\\
1741	0\\
1742	0\\
1743	0\\
1744	0\\
1745	0\\
1746	0\\
1747	0\\
1748	0\\
1749	0\\
1750	0\\
1751	0\\
1752	0\\
1753	0\\
1754	0\\
1755	0\\
1756	0\\
1757	0\\
1758	0\\
1759	0\\
1760	0\\
1761	0\\
1762	0\\
1763	0\\
1764	0\\
1765	0\\
1766	0\\
1767	0\\
1768	0\\
1769	0\\
1770	0\\
1771	0\\
1772	0\\
1773	0\\
1774	0\\
1775	0\\
1776	0\\
1777	0\\
1778	0\\
1779	0\\
1780	0\\
1781	0\\
1782	0\\
1783	0\\
1784	0\\
1785	0\\
1786	0\\
1787	0\\
1788	0\\
1789	0\\
1790	0\\
1791	0\\
1792	0\\
1793	0\\
1794	0\\
1795	0\\
1796	0\\
1797	0\\
1798	0\\
1799	0\\
1800	0\\
1801	0\\
1802	0\\
1803	0\\
1804	0\\
1805	0\\
1806	0\\
1807	0\\
1808	0\\
1809	0\\
1810	0\\
1811	0\\
1812	0\\
1813	0\\
1814	0\\
1815	0\\
1816	0\\
1817	0\\
1818	0\\
1819	0\\
1820	0\\
1821	0\\
1822	0\\
1823	0\\
1824	0\\
1825	0\\
1826	0\\
1827	0\\
1828	0\\
1829	0\\
1830	0\\
1831	0\\
1832	0\\
1833	0\\
1834	0\\
1835	0\\
1836	0\\
1837	0\\
1838	0\\
1839	0\\
1840	0\\
1841	0\\
1842	0\\
1843	0\\
1844	0\\
1845	0\\
1846	0\\
1847	0\\
1848	0\\
1849	0\\
1850	0\\
1851	0\\
1852	0\\
1853	0\\
1854	0\\
1855	0\\
1856	0\\
1857	0\\
1858	0\\
1859	0\\
1860	0\\
1861	0\\
1862	0\\
1863	0\\
1864	0\\
1865	0\\
1866	0\\
1867	0\\
1868	0\\
1869	0\\
1870	0\\
1871	0\\
1872	0\\
1873	0\\
1874	0\\
1875	0\\
1876	0\\
1877	0\\
1878	0\\
1879	0\\
1880	0\\
1881	0\\
1882	0\\
1883	0\\
1884	0\\
1885	0\\
1886	0\\
1887	0\\
1888	0\\
1889	0\\
1890	0\\
1891	0\\
1892	0\\
1893	0\\
1894	0\\
1895	0\\
1896	0\\
1897	0\\
1898	0\\
1899	0\\
1900	0\\
1901	0\\
1902	0\\
1903	0\\
1904	0\\
1905	0\\
1906	0\\
1907	0\\
1908	0\\
1909	0\\
1910	0\\
1911	0\\
1912	0\\
1913	0\\
1914	0\\
1915	0\\
1916	0\\
1917	0\\
1918	0\\
1919	0\\
1920	0\\
1921	0\\
1922	0\\
1923	0\\
1924	0\\
1925	0\\
1926	0\\
1927	0\\
1928	0\\
1929	0\\
1930	0\\
1931	0\\
1932	0\\
1933	0\\
1934	0\\
1935	0\\
1936	0\\
1937	0\\
1938	0\\
1939	0\\
1940	0\\
1941	0\\
1942	0\\
1943	0\\
1944	0\\
1945	0\\
1946	0\\
1947	0\\
1948	0\\
1949	0\\
1950	0\\
1951	0\\
1952	0\\
1953	0\\
1954	0\\
1955	0\\
1956	0\\
1957	0\\
1958	0\\
1959	0\\
1960	0\\
1961	0\\
1962	0\\
1963	0\\
1964	0\\
1965	0\\
1966	0\\
1967	0\\
1968	0\\
1969	0\\
1970	0\\
1971	0\\
1972	0\\
1973	0\\
1974	0\\
1975	0\\
1976	0\\
1977	0\\
1978	0\\
1979	0\\
1980	0\\
1981	0\\
1982	0\\
1983	0\\
1984	0\\
1985	0\\
1986	0\\
1987	0\\
1988	0\\
1989	0\\
1990	0\\
1991	0\\
1992	0\\
1993	0\\
1994	0\\
1995	0\\
1996	0\\
1997	0\\
1998	0\\
1999	0\\
2000	0\\
2001	0\\
2002	0\\
2003	0\\
2004	0\\
2005	0\\
2006	0\\
2007	0\\
2008	0\\
2009	0\\
2010	0\\
2011	0\\
2012	0\\
2013	0\\
2014	0\\
2015	0\\
2016	0\\
2017	0\\
2018	0\\
2019	0\\
2020	0\\
2021	0\\
2022	0\\
2023	0\\
2024	0\\
2025	0\\
2026	0\\
2027	0\\
2028	0\\
2029	0\\
2030	0\\
2031	0\\
2032	0\\
2033	0\\
2034	0\\
2035	0.460837\\
2036	13.947099\\
2037	10.727301\\
2038	4.226137\\
2039	3.139652\\
2040	4.21853\\
2041	0\\
2042	0\\
2043	0\\
2044	0\\
2045	0\\
2046	0\\
2047	0\\
2048	0\\
2049	0\\
2050	0\\
2051	0\\
2052	0\\
2053	0\\
2054	0\\
2055	0\\
2056	0\\
2057	0\\
2058	0\\
2059	1.469988\\
2060	2.583097\\
2061	2.01957\\
2062	0\\
2063	0\\
2064	0\\
2065	0\\
2066	0\\
2067	0\\
2068	0\\
2069	0\\
2070	0\\
2071	0\\
2072	0\\
2073	0\\
2074	0\\
2075	0\\
2076	0\\
2077	0\\
2078	0\\
2079	0\\
2080	0\\
2081	0\\
2082	0\\
2083	0\\
2084	0\\
2085	0\\
2086	0\\
2087	0\\
2088	0\\
2089	0\\
2090	0\\
2091	0\\
2092	0\\
2093	0\\
2094	0\\
2095	0\\
2096	0\\
2097	0\\
2098	0\\
2099	0\\
2100	0\\
2101	0\\
2102	0\\
2103	0\\
2104	0\\
2105	0\\
2106	0\\
2107	0\\
2108	0\\
2109	0\\
2110	0\\
2111	0\\
2112	0\\
2113	0\\
2114	0\\
2115	0\\
2116	0\\
2117	0\\
2118	0\\
2119	0\\
2120	0\\
2121	0\\
2122	0\\
2123	0\\
2124	0\\
2125	0\\
2126	0\\
2127	0\\
2128	0\\
2129	0\\
2130	0\\
2131	0\\
2132	0\\
2133	0\\
2134	0\\
2135	0\\
2136	0\\
2137	0\\
2138	0\\
2139	0\\
2140	0\\
2141	0\\
2142	0\\
2143	0\\
2144	0\\
2145	0\\
2146	0\\
2147	0\\
2148	0\\
2149	0\\
2150	0\\
2151	0\\
2152	0\\
2153	0\\
2154	0\\
2155	1.213897\\
2156	0\\
2157	0\\
2158	0\\
2159	0\\
2160	0\\
2161	0\\
2162	0\\
2163	0\\
2164	0\\
2165	0\\
2166	0\\
2167	0\\
2168	0\\
2169	0\\
2170	0\\
2171	0\\
2172	0\\
2173	0\\
2174	0\\
2175	0\\
2176	0\\
2177	0\\
2178	0\\
2179	0\\
2180	0\\
2181	0\\
2182	0\\
2183	0\\
2184	0\\
2185	0\\
2186	0\\
2187	0\\
2188	0\\
2189	0\\
2190	0\\
2191	0\\
2192	0\\
2193	0\\
2194	0\\
2195	0\\
2196	0\\
2197	0\\
2198	0\\
2199	0\\
2200	0\\
2201	0\\
2202	0\\
2203	0\\
2204	1.087826\\
2205	1.317321\\
2206	0\\
2207	0\\
2208	0\\
2209	0\\
2210	0\\
2211	0\\
2212	0\\
2213	0\\
2214	0\\
2215	0\\
2216	0\\
2217	0\\
2218	0\\
2219	0\\
2220	0\\
2221	0\\
2222	0\\
2223	0\\
2224	0\\
2225	0\\
2226	0\\
2227	0\\
2228	0\\
2229	0\\
2230	0\\
2231	0\\
2232	0\\
2233	0\\
2234	0\\
2235	0\\
2236	0\\
2237	0\\
2238	0\\
2239	0\\
2240	0\\
2241	0\\
2242	0\\
2243	0\\
2244	0\\
2245	0\\
2246	0\\
2247	0\\
2248	0\\
2249	0\\
2250	0\\
2251	0\\
2252	0\\
2253	0\\
2254	0\\
2255	0\\
2256	0\\
2257	0\\
2258	0\\
2259	0\\
2260	0\\
2261	0\\
2262	0\\
2263	0\\
2264	0\\
2265	0\\
2266	0\\
2267	0\\
2268	0\\
2269	0\\
2270	0\\
2271	0\\
2272	0\\
2273	0\\
2274	0\\
2275	0\\
2276	0\\
2277	0\\
2278	0\\
2279	0\\
2280	0\\
2281	0\\
2282	0\\
2283	0\\
2284	0\\
2285	0\\
2286	0\\
2287	0\\
2288	0\\
2289	0\\
2290	0\\
2291	0\\
2292	0\\
2293	0\\
2294	0\\
2295	0\\
2296	0\\
2297	0\\
2298	0\\
2299	0\\
2300	0\\
2301	0\\
2302	0\\
2303	0\\
2304	0\\
2305	0\\
2306	0\\
2307	0\\
2308	0\\
2309	0\\
2310	0\\
2311	0\\
2312	0\\
2313	0\\
2314	0\\
2315	0\\
2316	0\\
2317	0\\
2318	0\\
2319	0\\
2320	0\\
2321	0\\
2322	0\\
2323	0\\
2324	0\\
2325	0\\
2326	0\\
2327	0\\
2328	0\\
2329	0\\
2330	0\\
2331	0\\
2332	0\\
2333	0\\
2334	0\\
2335	0\\
2336	0\\
2337	0\\
2338	0\\
2339	0\\
2340	0\\
2341	0\\
2342	0\\
2343	0\\
2344	0\\
2345	0\\
2346	0\\
2347	0\\
2348	0\\
2349	0\\
2350	0\\
2351	0\\
2352	0\\
2353	0\\
2354	0\\
2355	0\\
2356	0\\
2357	0\\
2358	0\\
2359	0\\
2360	0\\
2361	0\\
2362	0\\
2363	0\\
2364	0\\
2365	0\\
2366	0\\
2367	0\\
2368	0\\
2369	0\\
2370	0\\
2371	0\\
2372	0\\
2373	0\\
2374	0\\
2375	0\\
2376	0\\
2377	0\\
2378	0\\
2379	0\\
2380	0\\
2381	0\\
2382	0\\
2383	0\\
2384	0\\
2385	0\\
2386	0\\
2387	0\\
2388	0\\
2389	0\\
2390	0\\
2391	0\\
2392	0\\
2393	0\\
2394	0\\
2395	0\\
2396	0\\
2397	0\\
2398	0\\
2399	0\\
2400	0\\
2401	0\\
2402	0\\
2403	0\\
2404	0\\
2405	0\\
2406	0\\
2407	0\\
2408	0\\
2409	0\\
2410	0\\
2411	0\\
2412	0\\
2413	0\\
2414	0\\
2415	0\\
2416	0\\
2417	0\\
2418	0\\
2419	0\\
2420	0\\
2421	0\\
2422	0\\
2423	0\\
2424	0\\
2425	0\\
2426	0\\
2427	0\\
2428	0\\
2429	0\\
2430	0\\
2431	0\\
2432	0\\
2433	0\\
2434	0\\
2435	0\\
2436	0\\
2437	0\\
2438	0\\
2439	0\\
2440	0\\
2441	0\\
2442	0\\
2443	0\\
2444	0\\
2445	0\\
2446	0\\
2447	0\\
2448	0\\
2449	0\\
2450	0\\
2451	0\\
2452	0\\
2453	0\\
2454	0\\
2455	0\\
2456	0\\
2457	0\\
2458	0\\
2459	0\\
2460	0\\
2461	0\\
2462	0\\
2463	0\\
2464	0\\
2465	0\\
2466	0\\
2467	0\\
2468	0\\
2469	0\\
2470	0\\
2471	0\\
2472	0\\
2473	0\\
2474	0\\
2475	0\\
2476	0\\
2477	0\\
2478	0\\
2479	0\\
2480	0\\
2481	0\\
2482	0\\
2483	0\\
2484	0\\
2485	0\\
2486	0\\
2487	0\\
2488	0\\
2489	0\\
2490	0\\
2491	0\\
2492	0\\
2493	0\\
2494	0\\
2495	0\\
2496	0\\
2497	0\\
2498	0\\
2499	0\\
2500	0\\
2501	0\\
2502	0\\
2503	0\\
2504	0\\
2505	0\\
2506	0\\
2507	0\\
2508	0\\
2509	0\\
2510	0\\
2511	0\\
2512	0\\
2513	0\\
2514	0\\
2515	0\\
2516	0\\
2517	0\\
2518	0\\
2519	0\\
2520	0\\
2521	0\\
2522	0\\
2523	0\\
2524	0\\
2525	0\\
2526	0\\
2527	0\\
2528	0\\
2529	0\\
2530	0\\
2531	0\\
2532	0\\
2533	0\\
2534	0\\
2535	0\\
2536	0\\
2537	0\\
2538	0\\
2539	0\\
2540	0\\
2541	0\\
2542	0\\
2543	0\\
2544	0\\
2545	0\\
2546	0\\
2547	0\\
2548	0\\
2549	0\\
2550	0\\
2551	0\\
2552	0\\
2553	0\\
2554	0\\
2555	0\\
2556	0\\
2557	0\\
2558	0\\
2559	0\\
2560	0\\
2561	0\\
2562	0\\
2563	0\\
2564	0\\
2565	0\\
2566	0\\
2567	0\\
2568	0\\
2569	0\\
2570	0\\
2571	0\\
2572	0\\
2573	0\\
2574	0\\
2575	0\\
2576	0\\
2577	0\\
2578	0\\
2579	0\\
2580	0\\
2581	0\\
2582	0\\
2583	0\\
2584	0\\
2585	0\\
2586	0\\
2587	0\\
2588	0\\
2589	0\\
2590	0\\
2591	0\\
2592	0\\
2593	0\\
2594	0\\
2595	0\\
2596	0\\
2597	0\\
2598	0\\
2599	0\\
2600	0\\
2601	0\\
2602	0\\
2603	0\\
2604	0\\
2605	0\\
2606	0\\
2607	0\\
2608	0\\
2609	0\\
2610	0\\
2611	0\\
2612	0\\
2613	0\\
2614	0\\
2615	0\\
2616	0\\
2617	0\\
2618	0\\
2619	0\\
2620	0\\
2621	0\\
2622	0\\
2623	0\\
2624	0\\
2625	0\\
2626	0\\
2627	0\\
2628	0\\
2629	0\\
2630	0\\
2631	0\\
2632	0\\
2633	0\\
2634	0\\
2635	0\\
2636	0\\
2637	0\\
2638	0\\
2639	0\\
2640	0\\
2641	0\\
2642	0\\
2643	0\\
2644	0\\
2645	0\\
2646	0\\
2647	0\\
2648	0\\
2649	0\\
2650	0\\
2651	0\\
2652	0\\
2653	0\\
2654	0\\
2655	0\\
2656	0\\
2657	0\\
2658	0\\
2659	0\\
2660	0\\
2661	0\\
2662	0\\
2663	0\\
2664	0\\
2665	0\\
2666	0\\
2667	0\\
2668	0\\
2669	0\\
2670	0\\
2671	0\\
2672	0\\
2673	0\\
2674	0\\
2675	0\\
2676	0\\
2677	0\\
2678	0\\
2679	0\\
2680	0\\
2681	0\\
2682	0\\
2683	0\\
2684	0\\
2685	0\\
2686	0\\
2687	0\\
2688	0\\
2689	0\\
2690	0\\
2691	0\\
2692	0\\
2693	0\\
2694	0\\
2695	0\\
2696	0\\
2697	0\\
2698	0\\
2699	0\\
2700	0\\
2701	0\\
2702	0\\
2703	0\\
2704	0\\
2705	0\\
2706	0\\
2707	0\\
2708	0\\
2709	0\\
2710	0\\
2711	0\\
2712	0\\
2713	0\\
2714	0\\
2715	0\\
2716	0\\
2717	0\\
2718	0\\
2719	0\\
2720	0\\
2721	0\\
2722	0\\
2723	0\\
2724	0\\
2725	0\\
2726	0\\
2727	0\\
2728	0\\
2729	0\\
2730	0\\
2731	0\\
2732	0\\
2733	0\\
2734	0\\
2735	0\\
2736	0\\
2737	0\\
2738	0\\
2739	0\\
2740	0\\
2741	0\\
2742	0\\
2743	0\\
2744	0\\
2745	0\\
2746	0\\
2747	0\\
2748	0\\
2749	0\\
2750	0\\
2751	0\\
2752	0\\
2753	0\\
2754	0\\
2755	0\\
2756	0\\
2757	0\\
2758	0\\
2759	0\\
2760	0\\
2761	0\\
2762	0\\
2763	0\\
2764	0\\
2765	0\\
2766	0\\
2767	0\\
2768	0\\
2769	0\\
2770	0\\
2771	0\\
2772	0\\
2773	0\\
2774	0\\
2775	0\\
2776	0\\
2777	0\\
2778	0\\
2779	0\\
2780	0\\
2781	0\\
2782	0\\
2783	0\\
2784	0\\
2785	0\\
2786	0\\
2787	0\\
2788	0\\
2789	0\\
2790	0\\
2791	0\\
2792	0\\
2793	0\\
2794	0\\
2795	0\\
2796	0\\
2797	0\\
2798	0\\
2799	0\\
2800	0\\
2801	0\\
2802	0\\
2803	0\\
2804	0\\
2805	0\\
2806	0\\
2807	0\\
2808	0\\
2809	0\\
2810	0\\
2811	0\\
2812	0\\
2813	0\\
2814	0\\
2815	0\\
2816	0\\
2817	0\\
2818	0\\
2819	0\\
2820	0\\
2821	0\\
2822	0\\
2823	0\\
2824	0\\
2825	0\\
2826	0\\
2827	0\\
2828	0\\
2829	0\\
2830	0\\
2831	0\\
2832	0\\
2833	0\\
2834	0\\
2835	0\\
2836	0\\
2837	0\\
2838	0\\
2839	0\\
2840	0\\
2841	0\\
2842	0\\
2843	0\\
2844	0\\
2845	0\\
2846	0\\
2847	0\\
2848	0\\
2849	0\\
2850	0\\
2851	0\\
2852	0\\
2853	0\\
2854	0\\
2855	0\\
2856	0\\
2857	0\\
2858	0\\
2859	0\\
2860	0\\
2861	0\\
2862	0\\
2863	0\\
2864	0\\
2865	0\\
2866	0\\
2867	0\\
2868	0\\
2869	0\\
2870	0\\
2871	0\\
2872	0\\
2873	0\\
2874	0\\
2875	0\\
2876	0\\
2877	0\\
2878	0\\
2879	0\\
2880	0\\
2881	0\\
2882	0\\
2883	0\\
2884	0\\
2885	0\\
2886	0\\
2887	0\\
2888	0\\
2889	0\\
2890	0\\
2891	0\\
2892	0\\
2893	0\\
2894	0\\
2895	0\\
2896	0\\
2897	0\\
2898	0\\
2899	0\\
2900	0\\
2901	0\\
2902	0\\
2903	0\\
2904	0\\
2905	0\\
2906	0\\
2907	0\\
2908	0\\
2909	0\\
2910	0\\
2911	0\\
2912	0\\
2913	0\\
2914	0\\
2915	0\\
2916	0\\
2917	0\\
2918	0\\
2919	0\\
2920	0\\
2921	0\\
2922	0\\
2923	0\\
2924	0\\
2925	0\\
2926	0\\
2927	0\\
2928	0\\
2929	0\\
2930	0\\
2931	0\\
2932	0\\
2933	0\\
2934	0\\
2935	0\\
2936	0\\
2937	0\\
2938	0\\
2939	0\\
2940	0\\
2941	0\\
2942	0\\
2943	0\\
2944	0\\
2945	0\\
2946	0\\
2947	0\\
2948	0\\
2949	0\\
2950	0\\
2951	0\\
2952	0\\
2953	0\\
2954	0\\
2955	0\\
2956	0\\
2957	0\\
2958	0\\
2959	0\\
2960	0\\
2961	0\\
2962	0\\
2963	0\\
2964	0\\
2965	0\\
2966	0\\
2967	0\\
2968	0\\
2969	0\\
2970	0\\
2971	0\\
2972	0\\
2973	0\\
2974	0\\
2975	0\\
2976	0\\
2977	0\\
2978	0\\
2979	0\\
2980	0\\
2981	0\\
2982	0\\
2983	0\\
2984	0\\
2985	0\\
2986	0\\
2987	0\\
2988	0\\
2989	0\\
2990	0\\
2991	0\\
2992	0\\
2993	0\\
2994	0\\
2995	0\\
2996	0\\
2997	0\\
2998	0\\
2999	0\\
3000	0\\
3001	0\\
3002	0\\
3003	0\\
3004	0\\
3005	0\\
3006	0\\
3007	0\\
3008	0\\
3009	0\\
3010	0\\
3011	0\\
3012	0\\
3013	0\\
3014	0\\
3015	0\\
3016	0\\
3017	0\\
3018	0\\
3019	0\\
3020	0\\
3021	0\\
3022	0\\
3023	0\\
3024	0\\
3025	0\\
3026	0\\
3027	0\\
3028	0\\
3029	0\\
3030	0\\
3031	0\\
3032	0\\
3033	0\\
3034	0\\
3035	0\\
3036	0\\
3037	0\\
3038	0\\
3039	0\\
3040	0\\
3041	0\\
3042	0\\
3043	0\\
3044	0\\
3045	0\\
3046	0\\
3047	0\\
3048	0\\
3049	0\\
3050	0\\
3051	0\\
3052	0\\
3053	0\\
3054	0\\
3055	0\\
3056	0\\
3057	0\\
3058	0\\
3059	0\\
3060	0\\
3061	0\\
3062	0\\
3063	0\\
3064	0\\
3065	0\\
3066	0\\
3067	0\\
3068	0\\
3069	0\\
3070	0\\
3071	0\\
3072	0\\
3073	0\\
3074	0\\
3075	0\\
3076	0\\
3077	0\\
3078	0\\
3079	0\\
3080	0\\
3081	0\\
3082	0\\
3083	0\\
3084	0\\
3085	0\\
3086	0\\
3087	0\\
3088	0\\
3089	0\\
3090	0\\
3091	0\\
3092	0\\
3093	0\\
3094	0\\
3095	0\\
3096	0\\
3097	0\\
3098	0\\
3099	0\\
3100	0\\
3101	0\\
3102	0\\
3103	0\\
3104	0\\
3105	0\\
3106	0\\
3107	0\\
3108	0\\
3109	0\\
3110	0\\
3111	0\\
3112	0\\
3113	0\\
3114	0\\
3115	0\\
3116	0\\
3117	0\\
3118	0\\
3119	0\\
3120	0\\
3121	0\\
3122	0\\
3123	0\\
3124	0\\
3125	0\\
3126	0\\
3127	0\\
3128	0\\
3129	0\\
3130	0\\
3131	0\\
3132	0\\
3133	0\\
3134	0\\
3135	0\\
3136	0\\
3137	0\\
3138	0\\
3139	0\\
3140	0\\
3141	0\\
3142	0\\
3143	0\\
3144	0\\
3145	0\\
3146	0\\
3147	0\\
3148	0\\
3149	0\\
3150	0\\
3151	0\\
3152	0\\
3153	0\\
3154	0\\
3155	0\\
3156	0\\
3157	0\\
3158	0\\
3159	0\\
3160	0\\
3161	0\\
3162	0\\
3163	0\\
3164	0\\
3165	0\\
3166	0\\
3167	0\\
3168	0\\
3169	0\\
3170	0\\
3171	0\\
3172	0\\
3173	0\\
3174	0\\
3175	0\\
3176	0\\
3177	0\\
3178	0\\
3179	0\\
3180	0\\
3181	0\\
3182	0\\
3183	0\\
3184	0\\
3185	0\\
3186	0\\
3187	0\\
3188	0\\
3189	0\\
3190	0\\
3191	0\\
3192	0\\
3193	0\\
3194	0\\
3195	0\\
3196	0\\
3197	0\\
3198	0\\
3199	0\\
3200	0\\
3201	0\\
3202	0\\
3203	0\\
3204	0\\
3205	0\\
3206	0\\
3207	0\\
3208	0\\
3209	0\\
3210	0\\
3211	0\\
3212	0\\
3213	0\\
3214	0\\
3215	0\\
3216	0\\
3217	0\\
3218	0\\
3219	0\\
3220	0\\
3221	0\\
3222	0\\
3223	0\\
3224	0\\
3225	0\\
3226	0\\
3227	0\\
3228	0\\
3229	0\\
3230	0\\
3231	0\\
3232	0\\
3233	0\\
3234	0\\
3235	0\\
3236	0\\
3237	0\\
3238	0\\
3239	0\\
3240	0\\
3241	0\\
3242	0\\
3243	0\\
3244	0\\
3245	0\\
3246	0\\
3247	0\\
3248	0\\
3249	0\\
3250	0\\
3251	0\\
3252	0\\
3253	0\\
3254	0\\
3255	0\\
3256	0\\
3257	0\\
3258	0\\
3259	0\\
3260	0\\
3261	0\\
3262	0\\
3263	0\\
3264	0\\
3265	0\\
3266	0\\
3267	0\\
3268	0\\
3269	0\\
3270	0\\
3271	0\\
3272	0\\
3273	0\\
3274	0\\
3275	0\\
3276	0\\
3277	0\\
3278	0\\
3279	0\\
3280	0\\
3281	0\\
3282	0\\
3283	0\\
3284	0\\
3285	0\\
3286	0\\
3287	0\\
3288	0\\
3289	0\\
3290	0\\
3291	0\\
3292	0\\
3293	0\\
3294	0\\
3295	0\\
3296	0\\
3297	0\\
3298	0\\
3299	0\\
3300	0\\
3301	0\\
3302	0\\
3303	0\\
3304	0\\
3305	0\\
3306	0\\
3307	0\\
3308	0\\
3309	0\\
3310	0\\
3311	0\\
3312	0\\
3313	0\\
3314	0\\
3315	0\\
3316	0\\
3317	0\\
3318	0\\
3319	0\\
3320	0\\
3321	0\\
3322	0\\
3323	0\\
3324	0\\
3325	0\\
3326	0\\
3327	0\\
3328	0\\
3329	0\\
3330	0\\
3331	0\\
3332	0\\
3333	0\\
3334	0\\
3335	0\\
3336	0\\
3337	0\\
3338	0\\
3339	0\\
3340	0\\
3341	0\\
3342	0\\
3343	0\\
3344	0\\
3345	0\\
3346	0\\
3347	0\\
3348	0\\
3349	0\\
3350	0\\
3351	0\\
3352	0\\
3353	0\\
3354	0\\
3355	0\\
3356	0\\
3357	0\\
3358	0\\
3359	0\\
3360	0\\
3361	0\\
3362	0\\
3363	0\\
3364	0\\
3365	0\\
3366	0\\
3367	0\\
3368	0\\
3369	0\\
3370	0\\
3371	0\\
3372	0\\
3373	0\\
3374	0\\
3375	0\\
3376	0\\
3377	0\\
3378	0\\
3379	0\\
3380	0\\
3381	0\\
3382	0\\
3383	0\\
3384	0\\
3385	0\\
3386	0\\
3387	0\\
3388	0\\
3389	0\\
3390	0\\
3391	0\\
3392	0\\
3393	0\\
3394	0\\
3395	0\\
3396	0\\
3397	0\\
3398	0\\
3399	0\\
3400	0\\
3401	0\\
3402	0\\
3403	0\\
3404	0\\
3405	0\\
3406	0\\
3407	0\\
3408	0\\
3409	0\\
3410	0\\
3411	0\\
3412	0\\
3413	0\\
3414	0\\
3415	0\\
3416	0\\
3417	0\\
3418	0\\
3419	0\\
3420	0\\
3421	0\\
3422	0\\
3423	0\\
3424	0\\
3425	0\\
3426	0\\
3427	0\\
3428	0\\
3429	0\\
3430	0\\
3431	0\\
3432	0\\
3433	0\\
3434	0\\
3435	0\\
3436	0\\
3437	0\\
3438	0\\
3439	0\\
3440	0\\
3441	0\\
3442	0\\
3443	0\\
3444	0\\
3445	0\\
3446	0\\
3447	0\\
3448	0\\
3449	0\\
3450	0\\
3451	0\\
3452	0\\
3453	0\\
3454	0\\
3455	0\\
3456	0\\
3457	0\\
3458	0\\
3459	0\\
3460	0\\
3461	0\\
3462	0\\
3463	0\\
3464	0\\
3465	0\\
3466	0\\
3467	0\\
3468	0\\
3469	0\\
3470	0\\
3471	0\\
3472	0\\
3473	0\\
3474	0\\
3475	0\\
3476	0\\
3477	0\\
3478	0\\
3479	0\\
3480	0\\
3481	0\\
3482	0\\
3483	0\\
3484	0\\
3485	0\\
3486	0\\
3487	0\\
3488	0\\
3489	0\\
3490	0\\
3491	0\\
3492	0\\
3493	0\\
3494	0\\
3495	0\\
3496	0\\
3497	0\\
3498	0\\
3499	0\\
3500	0\\
3501	0\\
3502	0\\
3503	0\\
3504	0\\
3505	0\\
3506	0\\
3507	0\\
3508	0\\
3509	0\\
3510	0\\
3511	0\\
3512	0\\
3513	0\\
3514	0\\
3515	0\\
3516	0\\
3517	0\\
3518	0\\
3519	0\\
3520	0\\
3521	0\\
3522	0\\
3523	0\\
3524	0\\
3525	0\\
3526	0\\
3527	0\\
3528	0\\
3529	0\\
3530	0\\
3531	0\\
3532	0\\
3533	0\\
3534	0\\
3535	0\\
3536	0\\
3537	0\\
3538	0\\
3539	0\\
3540	0\\
3541	0\\
3542	0\\
3543	0\\
3544	0\\
3545	0\\
3546	0\\
3547	0\\
3548	0\\
3549	0\\
3550	0\\
3551	0\\
3552	0\\
3553	0\\
3554	0\\
3555	0\\
3556	0\\
3557	0\\
3558	0\\
3559	0\\
3560	0\\
3561	0\\
3562	0\\
3563	0\\
3564	0\\
3565	0\\
3566	0\\
3567	0\\
3568	0\\
3569	0\\
3570	0\\
3571	0\\
3572	0\\
3573	0\\
3574	0\\
3575	0\\
3576	0\\
3577	0\\
3578	0\\
3579	0\\
3580	0\\
3581	0\\
3582	0\\
3583	0\\
3584	0\\
3585	0\\
3586	0\\
3587	0\\
3588	0\\
3589	0\\
3590	0\\
3591	0\\
3592	0\\
3593	0\\
3594	0\\
3595	0\\
3596	0\\
3597	0\\
3598	0\\
3599	0\\
3600	0\\
3601	0\\
3602	0\\
3603	0\\
3604	0\\
3605	0\\
3606	0\\
3607	0\\
3608	0\\
3609	0\\
3610	0\\
3611	0\\
3612	0\\
3613	0\\
3614	0\\
3615	0\\
3616	0\\
3617	0\\
3618	0\\
3619	0\\
3620	0\\
3621	0\\
3622	0\\
3623	0\\
3624	0\\
3625	0\\
3626	0\\
3627	0\\
3628	0\\
3629	0\\
3630	0\\
3631	0\\
3632	0\\
3633	0\\
3634	0\\
3635	0\\
3636	0\\
3637	0\\
3638	0\\
3639	0\\
3640	0\\
3641	0\\
3642	0\\
3643	0\\
3644	0\\
3645	0\\
3646	0\\
3647	0\\
3648	0\\
3649	0\\
3650	0\\
3651	0\\
3652	0\\
3653	0\\
3654	0\\
3655	0\\
3656	0\\
3657	0\\
3658	0\\
3659	0\\
3660	0\\
3661	0\\
3662	0\\
3663	0\\
3664	0\\
3665	0\\
3666	0\\
3667	0\\
3668	0\\
3669	0\\
3670	0\\
3671	0\\
3672	0\\
3673	0\\
3674	0\\
3675	0\\
3676	0\\
3677	0\\
3678	0\\
3679	0\\
3680	0\\
3681	0\\
3682	0\\
3683	0\\
3684	0\\
3685	0\\
3686	0\\
3687	0\\
3688	0\\
3689	0\\
3690	0\\
3691	0\\
3692	0\\
3693	0\\
3694	0\\
3695	0\\
3696	0\\
3697	0\\
3698	0\\
3699	0\\
3700	0\\
3701	0\\
3702	0\\
3703	0\\
3704	0\\
3705	0\\
3706	0.764794\\
3707	0.764794\\
3708	0.764794\\
3709	0\\
3710	0\\
3711	0\\
3712	0\\
3713	0\\
3714	0\\
3715	0\\
3716	0\\
3717	0\\
3718	0\\
3719	0\\
3720	0\\
3721	0\\
3722	0\\
3723	0\\
3724	0\\
3725	0\\
3726	0\\
3727	0\\
3728	0\\
3729	0\\
3730	0\\
3731	0\\
3732	0\\
3733	0\\
3734	0\\
3735	0\\
3736	0\\
3737	0\\
3738	0\\
3739	0\\
3740	0\\
3741	0\\
3742	0\\
3743	0\\
3744	0\\
3745	0\\
3746	0\\
3747	0\\
3748	0\\
3749	0\\
3750	0\\
3751	0\\
3752	0\\
3753	0\\
3754	0\\
3755	0\\
3756	0\\
3757	0\\
3758	0\\
3759	0\\
3760	0\\
3761	0\\
3762	0\\
3763	0\\
3764	0\\
3765	0\\
3766	0\\
3767	0\\
3768	0\\
3769	0\\
3770	0\\
3771	0\\
3772	0\\
3773	0\\
3774	0\\
3775	0\\
3776	0\\
3777	0\\
3778	0\\
3779	0\\
3780	0\\
3781	0\\
3782	0\\
3783	0\\
3784	0\\
3785	0\\
3786	0\\
3787	0\\
3788	0\\
3789	0\\
3790	0\\
3791	0\\
3792	0\\
3793	0\\
3794	0\\
3795	0\\
3796	0\\
3797	0\\
3798	0\\
3799	0\\
3800	0\\
3801	0\\
3802	0\\
3803	0\\
3804	0\\
3805	0\\
3806	0\\
3807	0\\
3808	0\\
3809	0\\
3810	0\\
3811	0\\
3812	0\\
3813	0\\
3814	0\\
3815	0\\
3816	0\\
3817	0\\
3818	0\\
3819	0\\
3820	0\\
3821	0\\
3822	0\\
3823	0\\
3824	0\\
3825	0\\
3826	0\\
3827	0\\
3828	0\\
3829	0\\
3830	0\\
3831	0\\
3832	0\\
3833	0\\
3834	0\\
3835	0\\
3836	0\\
3837	0\\
3838	0\\
3839	0\\
3840	0\\
3841	0\\
3842	0\\
3843	0\\
3844	0\\
3845	0\\
3846	0\\
3847	0\\
3848	0\\
3849	0\\
3850	0\\
3851	0\\
3852	0\\
3853	0\\
3854	0\\
3855	0\\
3856	0\\
3857	0\\
3858	0\\
3859	0\\
3860	0\\
3861	0\\
3862	0\\
3863	0\\
3864	0\\
3865	0\\
3866	0\\
3867	0\\
3868	0\\
3869	0\\
3870	0\\
3871	0.979939\\
3872	1.900735\\
3873	2.094992\\
3874	2.094992\\
3875	2.673321\\
3876	2.094992\\
3877	2.094992\\
3878	1.744733\\
3879	1.744733\\
3880	1.900735\\
3881	2.094992\\
3882	0\\
3883	0\\
3884	0\\
3885	0\\
3886	0\\
3887	0\\
3888	0\\
3889	0\\
3890	0\\
3891	0\\
3892	0\\
3893	0\\
3894	0\\
3895	0\\
3896	0\\
3897	0\\
3898	0\\
3899	0\\
3900	0\\
3901	0\\
3902	0\\
3903	0\\
3904	0\\
3905	0\\
3906	0\\
3907	0\\
3908	0\\
3909	0\\
3910	0\\
3911	0\\
3912	0\\
3913	0\\
3914	0\\
3915	0\\
3916	0\\
3917	0\\
3918	0\\
3919	0\\
3920	0\\
3921	0\\
3922	0\\
3923	0\\
3924	0\\
3925	0\\
3926	0\\
3927	0\\
3928	0\\
3929	0\\
3930	0\\
3931	0\\
3932	0\\
3933	0\\
3934	0\\
3935	0\\
3936	0\\
3937	0\\
3938	0\\
3939	0\\
3940	0\\
3941	0\\
3942	0\\
3943	0\\
3944	0\\
3945	0\\
3946	0\\
3947	0\\
3948	0\\
3949	0\\
3950	0\\
3951	0\\
3952	0\\
3953	0\\
3954	0\\
3955	0\\
3956	0\\
3957	0\\
3958	0\\
3959	0\\
3960	0\\
3961	0\\
3962	0\\
3963	0\\
3964	0\\
3965	0\\
3966	0\\
3967	0\\
3968	0\\
3969	0\\
3970	0\\
3971	0\\
3972	0\\
3973	0\\
3974	0\\
3975	0\\
3976	0\\
3977	0\\
3978	0\\
3979	0\\
3980	0\\
3981	0\\
3982	0\\
3983	0\\
3984	0\\
3985	0\\
3986	0\\
3987	0\\
3988	0\\
3989	0\\
3990	0\\
3991	0\\
3992	0\\
3993	0\\
3994	0\\
3995	0\\
3996	0\\
3997	0\\
3998	0\\
3999	0\\
4000	0\\
4001	0\\
};
\addplot [color=mycolor1,solid,line width=1.0pt,forget plot]
  table[row sep=crcr]{%
4001	0\\
4002	0\\
4003	0\\
4004	0\\
4005	0\\
4006	0\\
4007	0\\
4008	0\\
4009	0\\
4010	0\\
4011	0\\
4012	0\\
4013	0\\
4014	0\\
4015	0\\
4016	0\\
4017	0\\
4018	0\\
4019	0\\
4020	0\\
4021	0\\
4022	0\\
4023	0\\
4024	0\\
4025	0\\
4026	0\\
4027	0\\
4028	0\\
4029	0\\
4030	0\\
4031	0\\
4032	0\\
4033	0\\
4034	0\\
4035	0\\
4036	0\\
4037	0\\
4038	0\\
4039	0\\
4040	0\\
4041	0\\
4042	0\\
4043	0\\
4044	0\\
4045	0\\
4046	0\\
4047	0\\
4048	0\\
4049	0\\
4050	0\\
4051	0\\
4052	0\\
4053	0\\
4054	0\\
4055	0\\
4056	0\\
4057	0\\
4058	0\\
4059	0\\
4060	0\\
4061	0\\
4062	0\\
4063	0\\
4064	0\\
4065	0\\
4066	0\\
4067	0\\
4068	0\\
4069	0\\
4070	0\\
4071	0\\
4072	0\\
4073	0\\
4074	0\\
4075	0\\
4076	0\\
4077	0\\
4078	0\\
4079	0\\
4080	0\\
4081	0\\
4082	0\\
4083	0\\
4084	0\\
4085	0\\
4086	0\\
4087	0\\
4088	0\\
4089	0\\
4090	0\\
4091	0\\
4092	0\\
4093	0\\
4094	0\\
4095	0\\
4096	0\\
4097	0\\
4098	0\\
4099	0\\
4100	0\\
4101	0\\
4102	0\\
4103	0\\
4104	0\\
4105	0\\
4106	0\\
4107	0\\
4108	0\\
4109	0\\
4110	0\\
4111	0\\
4112	0\\
4113	0\\
4114	0\\
4115	0\\
4116	0\\
4117	0\\
4118	0\\
4119	0\\
4120	0\\
4121	0\\
4122	0\\
4123	0\\
4124	0\\
4125	0\\
4126	0\\
4127	0\\
4128	0\\
4129	0\\
4130	0\\
4131	0\\
4132	0\\
4133	0\\
4134	0\\
4135	0\\
4136	0\\
4137	0\\
4138	0\\
4139	0\\
4140	0\\
4141	0\\
4142	0\\
4143	0\\
4144	0\\
4145	0\\
4146	0\\
4147	0\\
4148	0\\
4149	0\\
4150	0\\
4151	0\\
4152	0\\
4153	0\\
4154	0\\
4155	0\\
4156	0\\
4157	0\\
4158	0\\
4159	0\\
4160	0\\
4161	0\\
4162	0\\
4163	0\\
4164	0\\
4165	0\\
4166	0\\
4167	0\\
4168	0\\
4169	0\\
4170	0\\
4171	0\\
4172	0\\
4173	0\\
4174	0\\
4175	0\\
4176	0\\
4177	0\\
4178	0\\
4179	0\\
4180	0\\
4181	0\\
4182	0\\
4183	0\\
4184	0\\
4185	0.349566\\
4186	0\\
4187	0\\
4188	0\\
4189	0\\
4190	0\\
4191	0\\
4192	0\\
4193	0\\
4194	0\\
4195	0\\
4196	0\\
4197	0\\
4198	0\\
4199	0\\
4200	0\\
4201	0\\
4202	0\\
4203	0\\
4204	0\\
4205	0\\
4206	0\\
4207	0\\
4208	0\\
4209	0\\
4210	0\\
4211	0\\
4212	0\\
4213	0\\
4214	0\\
4215	0\\
4216	0\\
4217	0\\
4218	0\\
4219	0\\
4220	0\\
4221	0\\
4222	0\\
4223	0\\
4224	0\\
4225	0\\
4226	0\\
4227	0\\
4228	0\\
4229	0\\
4230	0\\
4231	0\\
4232	0\\
4233	0\\
4234	0\\
4235	0\\
4236	0\\
4237	0\\
4238	0\\
4239	0\\
4240	0\\
4241	0\\
4242	0\\
4243	0\\
4244	0\\
4245	0\\
4246	0\\
4247	0\\
4248	0\\
4249	0\\
4250	0\\
4251	0\\
4252	0\\
4253	0\\
4254	0\\
4255	0\\
4256	0\\
4257	0\\
4258	0\\
4259	0\\
4260	0\\
4261	0\\
4262	0\\
4263	0\\
4264	0\\
4265	0\\
4266	0\\
4267	0\\
4268	0\\
4269	0\\
4270	0\\
4271	0\\
4272	0\\
4273	0\\
4274	0\\
4275	0\\
4276	0\\
4277	0\\
4278	0\\
4279	0\\
4280	0\\
4281	0\\
4282	0\\
4283	0\\
4284	0\\
4285	0\\
4286	0\\
4287	0\\
4288	0\\
4289	0\\
4290	0\\
4291	0\\
4292	0\\
4293	0\\
4294	0\\
4295	0\\
4296	0\\
4297	0\\
4298	0\\
4299	0\\
4300	0\\
4301	0\\
4302	0\\
4303	0\\
4304	0\\
4305	0\\
4306	0\\
4307	0\\
4308	0\\
4309	0\\
4310	0\\
4311	0\\
4312	0\\
4313	0\\
4314	0\\
4315	0\\
4316	0\\
4317	0\\
4318	0\\
4319	0\\
4320	0\\
4321	0\\
4322	0\\
4323	0\\
4324	0\\
4325	0\\
4326	0\\
4327	0\\
4328	0\\
4329	0\\
4330	0\\
4331	0\\
4332	0\\
4333	0\\
4334	0\\
4335	0\\
4336	0\\
4337	0\\
4338	0\\
4339	0\\
4340	0\\
4341	0\\
4342	0\\
4343	0\\
4344	0\\
4345	0\\
4346	0\\
4347	0\\
4348	0\\
4349	0\\
4350	0\\
4351	0\\
4352	0\\
4353	0\\
4354	0\\
4355	0\\
4356	0\\
4357	0\\
4358	0\\
4359	0\\
4360	0\\
4361	0\\
4362	0\\
4363	0\\
4364	0\\
4365	0\\
4366	0\\
4367	0\\
4368	0\\
4369	0\\
4370	0\\
4371	0\\
4372	0\\
4373	0\\
4374	0\\
4375	0\\
4376	0\\
4377	0\\
4378	0\\
4379	0\\
4380	0\\
4381	0\\
4382	0\\
4383	0\\
4384	0\\
4385	0\\
4386	0\\
4387	0\\
4388	0\\
4389	0\\
4390	0\\
4391	0\\
4392	0\\
4393	0\\
4394	0\\
4395	0\\
4396	0\\
4397	0\\
4398	0\\
4399	0\\
4400	0\\
4401	0\\
4402	0\\
4403	0\\
4404	0\\
4405	0\\
4406	0\\
4407	0\\
4408	0\\
4409	0\\
4410	0\\
4411	0\\
4412	0\\
4413	0\\
4414	0\\
4415	0\\
4416	0\\
4417	0\\
4418	0\\
4419	0\\
4420	0\\
4421	0\\
4422	0\\
4423	0\\
4424	0\\
4425	0\\
4426	0\\
4427	0\\
4428	0\\
4429	0\\
4430	0\\
4431	0\\
4432	0\\
4433	0\\
4434	0\\
4435	0\\
4436	0\\
4437	0\\
4438	0\\
4439	0\\
4440	0\\
4441	0\\
4442	0\\
4443	0\\
4444	0\\
4445	0\\
4446	0\\
4447	0\\
4448	0\\
4449	0\\
4450	0\\
4451	0\\
4452	0\\
4453	0\\
4454	0\\
4455	0\\
4456	0\\
4457	0\\
4458	0\\
4459	0\\
4460	0\\
4461	0\\
4462	0\\
4463	0\\
4464	0\\
4465	0\\
4466	0\\
4467	0\\
4468	0\\
4469	0\\
4470	0\\
4471	0\\
4472	0\\
4473	0\\
4474	0\\
4475	0\\
4476	0\\
4477	0\\
4478	0\\
4479	0\\
4480	0\\
4481	0\\
4482	0\\
4483	0\\
4484	0\\
4485	0\\
4486	0\\
4487	0\\
4488	0\\
4489	0\\
4490	0\\
4491	0\\
4492	0\\
4493	0\\
4494	0\\
4495	0\\
4496	0\\
4497	0\\
4498	0\\
4499	0\\
4500	0\\
4501	0\\
4502	0\\
4503	0\\
4504	0\\
4505	0\\
4506	0\\
4507	0\\
4508	0\\
4509	0\\
4510	0\\
4511	0\\
4512	0\\
4513	0\\
4514	0\\
4515	0\\
4516	0\\
4517	0\\
4518	0\\
4519	0\\
4520	0\\
4521	0\\
4522	0\\
4523	0\\
4524	0\\
4525	0\\
4526	0\\
4527	0\\
4528	0\\
4529	0\\
4530	0\\
4531	0\\
4532	0\\
4533	0\\
4534	0\\
4535	0\\
4536	0\\
4537	0\\
4538	0\\
4539	0\\
4540	0\\
4541	0\\
4542	0\\
4543	0\\
4544	0\\
4545	0\\
4546	0\\
4547	0\\
4548	0\\
4549	0\\
4550	0\\
4551	0\\
4552	0\\
4553	0\\
4554	0\\
4555	0\\
4556	0\\
4557	0\\
4558	0\\
4559	0\\
4560	0\\
4561	0\\
4562	0\\
4563	0\\
4564	0\\
4565	0\\
4566	0\\
4567	0\\
4568	0\\
4569	0\\
4570	0\\
4571	0\\
4572	0\\
4573	0\\
4574	0\\
4575	0\\
4576	0\\
4577	0\\
4578	0\\
4579	0\\
4580	0\\
4581	0\\
4582	0\\
4583	0\\
4584	0\\
4585	0\\
4586	0\\
4587	0\\
4588	0\\
4589	0\\
4590	0\\
4591	0\\
4592	0\\
4593	0\\
4594	0\\
4595	0\\
4596	0\\
4597	0\\
4598	0\\
4599	0\\
4600	0\\
4601	0\\
4602	0\\
4603	0\\
4604	0\\
4605	0\\
4606	0\\
4607	0\\
4608	0\\
4609	0\\
4610	0\\
4611	0\\
4612	0\\
4613	0\\
4614	0\\
4615	0\\
4616	0\\
4617	0\\
4618	0\\
4619	0\\
4620	0\\
4621	0\\
4622	0\\
4623	0\\
4624	0\\
4625	0\\
4626	0\\
4627	0\\
4628	0\\
4629	0\\
4630	0\\
4631	0\\
4632	0\\
4633	0\\
4634	0\\
4635	0\\
4636	0\\
4637	0\\
4638	0\\
4639	0\\
4640	0\\
4641	0\\
4642	0\\
4643	0\\
4644	0\\
4645	0\\
4646	0\\
4647	0\\
4648	0\\
4649	0\\
4650	0\\
4651	0\\
4652	0\\
4653	0\\
4654	0\\
4655	0\\
4656	0\\
4657	0\\
4658	0\\
4659	0\\
4660	0\\
4661	0\\
4662	0\\
4663	0\\
4664	0\\
4665	0\\
4666	0\\
4667	0\\
4668	0\\
4669	0\\
4670	0\\
4671	0\\
4672	0\\
4673	0\\
4674	0\\
4675	0\\
4676	0\\
4677	0\\
4678	0\\
4679	0\\
4680	0\\
4681	0\\
4682	0\\
4683	0\\
4684	0\\
4685	0\\
4686	0\\
4687	0\\
4688	0\\
4689	0\\
4690	0\\
4691	0\\
4692	0\\
4693	0\\
4694	0\\
4695	0\\
4696	0\\
4697	0\\
4698	0\\
4699	0\\
4700	0\\
4701	0\\
4702	0\\
4703	0\\
4704	0\\
4705	0\\
4706	0\\
4707	0\\
4708	0\\
4709	0\\
4710	0\\
4711	0\\
4712	0\\
4713	0\\
4714	0\\
4715	0\\
4716	0\\
4717	0\\
4718	0\\
4719	0\\
4720	0\\
4721	0\\
4722	0\\
4723	0\\
4724	0\\
4725	0\\
4726	0\\
4727	0\\
4728	0\\
4729	0\\
4730	0\\
4731	0\\
4732	0\\
4733	0\\
4734	0\\
4735	0\\
4736	0\\
4737	0\\
4738	0\\
4739	0\\
4740	0\\
4741	0\\
4742	0\\
4743	0\\
4744	0\\
4745	0\\
4746	0\\
4747	0\\
4748	0\\
4749	0\\
4750	0\\
4751	0\\
4752	0\\
4753	0\\
4754	0\\
4755	0\\
4756	0\\
4757	0\\
4758	0\\
4759	0\\
4760	0\\
4761	0\\
4762	0\\
4763	0\\
4764	0\\
4765	0\\
4766	0\\
4767	0\\
4768	0\\
4769	0\\
4770	0\\
4771	0\\
4772	0\\
4773	0\\
4774	0\\
4775	0\\
4776	0\\
4777	0\\
4778	0\\
4779	0\\
4780	0\\
4781	0\\
4782	0\\
4783	0\\
4784	0\\
4785	0\\
4786	0\\
4787	0\\
4788	0\\
4789	0\\
4790	0\\
4791	0\\
4792	0\\
4793	0\\
4794	0\\
4795	0\\
4796	0\\
4797	0\\
4798	0\\
4799	0\\
4800	0\\
4801	0\\
4802	0\\
4803	0\\
4804	0\\
4805	0\\
4806	0\\
4807	0\\
4808	0\\
4809	0\\
4810	0\\
4811	0\\
4812	0\\
4813	0\\
4814	0\\
4815	0\\
4816	0\\
4817	0\\
4818	0\\
4819	0\\
4820	0\\
4821	0\\
4822	0\\
4823	0\\
4824	0\\
4825	0\\
4826	0\\
4827	0\\
4828	0\\
4829	0\\
4830	0\\
4831	0\\
4832	0\\
4833	0\\
4834	0\\
4835	0\\
4836	0\\
4837	0\\
4838	0\\
4839	0\\
4840	0\\
4841	0\\
4842	0\\
4843	0\\
4844	0\\
4845	0\\
4846	0\\
4847	0\\
4848	0\\
4849	0\\
4850	0\\
4851	0\\
4852	0\\
4853	0\\
4854	0\\
4855	0\\
4856	0\\
4857	0\\
4858	0\\
4859	0\\
4860	0\\
4861	0\\
4862	0\\
4863	0\\
4864	0\\
4865	0\\
4866	0\\
4867	0\\
4868	0\\
4869	0\\
4870	0\\
4871	0\\
4872	0\\
4873	0\\
4874	0\\
4875	0\\
4876	0\\
4877	0\\
4878	0\\
4879	0\\
4880	0\\
4881	0\\
4882	0\\
4883	0\\
4884	0\\
4885	0\\
4886	0\\
4887	0\\
4888	0\\
4889	0\\
4890	0\\
4891	0\\
4892	0\\
4893	0\\
4894	0\\
4895	0\\
4896	0\\
4897	0\\
4898	0\\
4899	0\\
4900	0\\
4901	0\\
4902	0\\
4903	0\\
4904	0\\
4905	0\\
4906	0\\
4907	0\\
4908	0\\
4909	0\\
4910	0\\
4911	0\\
4912	0\\
4913	0\\
4914	0\\
4915	0\\
4916	0\\
4917	0\\
4918	0\\
4919	0\\
4920	0\\
4921	0\\
4922	0\\
4923	0\\
4924	0\\
4925	0\\
4926	0\\
4927	0\\
4928	0\\
4929	0\\
4930	0\\
4931	0\\
4932	0\\
4933	0\\
4934	0\\
4935	0\\
4936	0\\
4937	0\\
4938	0\\
4939	0\\
4940	0\\
4941	0\\
4942	0\\
4943	0\\
4944	0\\
4945	0\\
4946	0\\
4947	0\\
4948	0\\
4949	0\\
4950	0\\
4951	0\\
4952	0\\
4953	0\\
4954	0\\
4955	0\\
4956	0\\
4957	0\\
4958	0\\
4959	0\\
4960	0\\
4961	0\\
4962	0\\
4963	0\\
4964	0\\
4965	0\\
4966	0\\
4967	0\\
4968	0\\
4969	0\\
4970	0\\
4971	0\\
4972	0\\
4973	0\\
4974	0\\
4975	0\\
4976	0\\
4977	0\\
4978	0\\
4979	0\\
4980	0\\
4981	0\\
4982	0\\
4983	0\\
4984	0\\
4985	0\\
4986	0\\
4987	0\\
4988	0\\
4989	0\\
4990	0\\
4991	0\\
4992	0\\
4993	0\\
4994	0\\
4995	0\\
4996	0\\
4997	0\\
4998	0\\
4999	0\\
5000	0\\
5001	0\\
5002	0\\
5003	0\\
5004	0\\
5005	0\\
5006	0\\
5007	0\\
5008	0\\
5009	0\\
5010	0\\
5011	0\\
5012	0\\
5013	0\\
5014	0\\
5015	0\\
5016	0\\
5017	0\\
5018	0\\
5019	0\\
5020	0\\
5021	0\\
5022	0\\
5023	0\\
5024	0\\
5025	0\\
5026	0\\
5027	0\\
5028	0\\
5029	0\\
5030	0\\
5031	0\\
5032	0\\
5033	0\\
5034	0\\
5035	0\\
5036	0\\
5037	0\\
5038	0\\
5039	0\\
5040	0\\
5041	0\\
5042	0\\
5043	0\\
5044	0\\
5045	0\\
5046	0\\
5047	0\\
5048	0\\
5049	0\\
5050	0\\
5051	0\\
5052	0\\
5053	0\\
5054	0\\
5055	0\\
5056	0\\
5057	0\\
5058	0\\
5059	0\\
5060	0\\
5061	0\\
5062	0\\
5063	0\\
5064	0\\
5065	0\\
5066	0\\
5067	0\\
5068	0\\
5069	0\\
5070	0\\
5071	0\\
5072	0\\
5073	0\\
5074	0\\
5075	0\\
5076	0\\
5077	0\\
5078	0\\
5079	0\\
5080	0\\
5081	0\\
5082	0\\
5083	0\\
5084	0\\
5085	0\\
5086	0\\
5087	0\\
5088	0\\
5089	0\\
5090	0\\
5091	0\\
5092	0\\
5093	0\\
5094	0\\
5095	0\\
5096	0\\
5097	0\\
5098	0\\
5099	0\\
5100	0\\
5101	0\\
5102	0\\
5103	0\\
5104	0\\
5105	0\\
5106	0\\
5107	0\\
5108	0\\
5109	0\\
5110	0\\
5111	0\\
5112	0\\
5113	0\\
5114	0\\
5115	0\\
5116	0\\
5117	0\\
5118	0\\
5119	0\\
5120	0\\
5121	0\\
5122	0\\
5123	0\\
5124	0\\
5125	0\\
5126	0\\
5127	0\\
5128	0\\
5129	0\\
5130	0\\
5131	0\\
5132	0\\
5133	0\\
5134	0\\
5135	0\\
5136	0\\
5137	0\\
5138	0\\
5139	0\\
5140	0\\
5141	0\\
5142	0\\
5143	0\\
5144	0\\
5145	0\\
5146	0\\
5147	0\\
5148	0\\
5149	0\\
5150	0\\
5151	0\\
5152	0\\
5153	0\\
5154	0\\
5155	0\\
5156	0\\
5157	0\\
5158	0\\
5159	0\\
5160	0\\
5161	0\\
5162	0\\
5163	0\\
5164	0\\
5165	0\\
5166	0\\
5167	0\\
5168	0\\
5169	0\\
5170	0\\
5171	0\\
5172	0\\
5173	0\\
5174	0\\
5175	0\\
5176	0\\
5177	0\\
5178	0\\
5179	0\\
5180	0\\
5181	0\\
5182	0\\
5183	0\\
5184	0\\
5185	0\\
5186	0\\
5187	0\\
5188	0\\
5189	0\\
5190	0\\
5191	0\\
5192	0\\
5193	0\\
5194	0\\
5195	0\\
5196	0\\
5197	0\\
5198	0\\
5199	0\\
5200	0\\
5201	0\\
5202	0\\
5203	0\\
5204	0\\
5205	0\\
5206	0\\
5207	0\\
5208	0\\
5209	0\\
5210	0\\
5211	0\\
5212	0\\
5213	0\\
5214	0\\
5215	0\\
5216	0\\
5217	0\\
5218	0\\
5219	0\\
5220	0\\
5221	0\\
5222	0\\
5223	0\\
5224	0\\
5225	0\\
5226	0\\
5227	0\\
5228	0\\
5229	0\\
5230	0\\
5231	0\\
5232	0\\
5233	0\\
5234	0\\
5235	0\\
5236	0\\
5237	0\\
5238	0\\
5239	0\\
5240	0\\
5241	0\\
5242	0\\
5243	0\\
5244	0\\
5245	0\\
5246	0\\
5247	0\\
5248	0\\
5249	0\\
5250	0\\
5251	0\\
5252	0\\
5253	0\\
5254	0\\
5255	0\\
5256	0\\
5257	0\\
5258	0\\
5259	0\\
5260	0\\
5261	0\\
5262	0\\
5263	0\\
5264	0\\
5265	0\\
5266	0\\
5267	0\\
5268	0\\
5269	0\\
5270	0\\
5271	0\\
5272	0\\
5273	0\\
5274	0\\
5275	0\\
5276	0\\
5277	0\\
5278	0\\
5279	0\\
5280	0\\
5281	0\\
5282	0\\
5283	0\\
5284	0\\
5285	0\\
5286	0\\
5287	0\\
5288	0\\
5289	0\\
5290	0\\
5291	0\\
5292	0\\
5293	0\\
5294	0\\
5295	0\\
5296	0\\
5297	0\\
5298	0\\
5299	0\\
5300	0\\
5301	0\\
5302	0\\
5303	0\\
5304	0\\
5305	0\\
5306	0\\
5307	0\\
5308	0\\
5309	0\\
5310	0\\
5311	0\\
5312	0\\
5313	0\\
5314	0\\
5315	0\\
5316	0\\
5317	0\\
5318	0\\
5319	0\\
5320	0\\
5321	0\\
5322	0\\
5323	0\\
5324	0\\
5325	0\\
5326	0\\
5327	0\\
5328	0\\
5329	0\\
5330	0\\
5331	0\\
5332	0\\
5333	0\\
5334	0\\
5335	0\\
5336	0\\
5337	0\\
5338	0\\
5339	0\\
5340	0\\
5341	0\\
5342	0\\
5343	0\\
5344	0\\
5345	0\\
5346	0\\
5347	0\\
5348	0\\
5349	0\\
5350	0\\
5351	0\\
5352	0\\
5353	0\\
5354	0\\
5355	0\\
5356	0\\
5357	0\\
5358	0\\
5359	0\\
5360	0\\
5361	0\\
5362	0\\
5363	0\\
5364	0\\
5365	0\\
5366	0\\
5367	0\\
5368	0\\
5369	0\\
5370	0\\
5371	0\\
5372	0\\
5373	0\\
5374	0\\
5375	0\\
5376	0\\
5377	0\\
5378	0\\
5379	0\\
5380	0\\
5381	0\\
5382	0\\
5383	0\\
5384	0\\
5385	0\\
5386	0\\
5387	0\\
5388	0\\
5389	0\\
5390	0\\
5391	0\\
5392	0\\
5393	0\\
5394	0\\
5395	0\\
5396	0\\
5397	0\\
5398	0\\
5399	0\\
5400	0\\
5401	0\\
5402	0\\
5403	0\\
5404	0\\
5405	0\\
5406	0\\
5407	0\\
5408	0\\
5409	0\\
5410	0\\
5411	0\\
5412	0\\
5413	0\\
5414	0\\
5415	0\\
5416	0\\
5417	0\\
5418	0\\
5419	0\\
5420	0\\
5421	0\\
5422	0\\
5423	0\\
5424	0\\
5425	0\\
5426	0\\
5427	0\\
5428	0\\
5429	0\\
5430	0\\
5431	0\\
5432	0\\
5433	0\\
5434	0\\
5435	0\\
5436	0\\
5437	0\\
5438	0\\
5439	0\\
5440	0\\
5441	0\\
5442	0\\
5443	0\\
5444	0\\
5445	0\\
5446	0\\
5447	0\\
5448	0\\
5449	0\\
5450	0\\
5451	0\\
5452	0\\
5453	0\\
5454	0\\
5455	0\\
5456	0\\
5457	0\\
5458	0\\
5459	0\\
5460	0\\
5461	0\\
5462	0\\
5463	0\\
5464	0\\
5465	0\\
5466	0\\
5467	0\\
5468	0\\
5469	0\\
5470	0\\
5471	0\\
5472	0\\
5473	0\\
5474	0\\
5475	0\\
5476	0\\
5477	0\\
5478	0\\
5479	0\\
5480	0\\
5481	0\\
5482	0\\
5483	0\\
5484	0\\
5485	0\\
5486	0\\
5487	0\\
5488	0\\
5489	0\\
5490	0\\
5491	0\\
5492	0\\
5493	0\\
5494	0\\
5495	0\\
5496	0\\
5497	0\\
5498	0\\
5499	0\\
5500	0\\
5501	0\\
5502	0\\
5503	0\\
5504	0\\
5505	0\\
5506	0\\
5507	0\\
5508	0\\
5509	0\\
5510	0\\
5511	0\\
5512	0\\
5513	0\\
5514	0\\
5515	0\\
5516	0\\
5517	0\\
5518	0\\
5519	0\\
5520	0\\
5521	0\\
5522	0\\
5523	0\\
5524	0\\
5525	0\\
5526	0\\
5527	0\\
5528	0\\
5529	0\\
5530	0\\
5531	0\\
5532	0\\
5533	0\\
5534	0\\
5535	0\\
5536	0\\
5537	0\\
5538	0\\
5539	0\\
5540	0\\
5541	0\\
5542	0\\
5543	0\\
5544	0\\
5545	0\\
5546	0\\
5547	0\\
5548	0\\
5549	0\\
5550	0\\
5551	0\\
5552	0\\
5553	0\\
5554	0\\
5555	0\\
5556	0\\
5557	0\\
5558	0\\
5559	0\\
5560	0\\
5561	0\\
5562	0\\
5563	0\\
5564	0\\
5565	0\\
5566	0\\
5567	0\\
5568	0\\
5569	0\\
5570	0\\
5571	0\\
5572	0\\
5573	0\\
5574	0\\
5575	0\\
5576	0\\
5577	0\\
5578	0\\
5579	0\\
5580	0\\
5581	0\\
5582	0\\
5583	0\\
5584	0\\
5585	0\\
5586	0\\
5587	0\\
5588	0\\
5589	0\\
5590	0\\
5591	0\\
5592	0\\
5593	0\\
5594	0\\
5595	0\\
5596	0\\
5597	0\\
5598	0\\
5599	0\\
5600	0\\
5601	0\\
5602	0\\
5603	0\\
5604	0\\
5605	0\\
5606	0\\
5607	0\\
5608	0\\
5609	0\\
5610	0\\
5611	0\\
5612	0\\
5613	0\\
5614	0\\
5615	0\\
5616	0\\
5617	0\\
5618	0\\
5619	0\\
5620	0\\
5621	0\\
5622	0\\
5623	0\\
5624	0\\
5625	0\\
5626	0\\
5627	0\\
5628	0\\
5629	0\\
5630	0\\
5631	0\\
5632	0\\
5633	0\\
5634	0\\
5635	0\\
5636	0\\
5637	0\\
5638	0\\
5639	0\\
5640	0\\
5641	0\\
5642	0\\
5643	0\\
5644	0\\
5645	0\\
5646	0\\
5647	0\\
5648	0\\
5649	0\\
5650	0\\
5651	0\\
5652	0\\
5653	0\\
5654	0\\
5655	0\\
5656	0\\
5657	0\\
5658	0\\
5659	0\\
5660	0\\
5661	0\\
5662	0\\
5663	0\\
5664	0\\
5665	0\\
5666	0\\
5667	0\\
5668	0\\
5669	0\\
5670	0\\
5671	0\\
5672	0\\
5673	0\\
5674	0\\
5675	0\\
5676	0\\
5677	0.576113\\
5678	0\\
5679	0\\
5680	0\\
5681	1.504702\\
5682	1.504702\\
5683	1.504702\\
5684	0.576113\\
5685	0\\
5686	0\\
5687	0\\
5688	0\\
5689	0\\
5690	0\\
5691	0\\
5692	0\\
5693	0\\
5694	0\\
5695	0\\
5696	0\\
5697	1.504702\\
5698	1.970716\\
5699	1.840625\\
5700	1.504702\\
5701	1.504702\\
5702	1.504702\\
5703	1.504702\\
5704	1.504702\\
5705	0.926116\\
5706	1.504702\\
5707	0\\
5708	0\\
5709	0\\
5710	0\\
5711	0\\
5712	0\\
5713	0\\
5714	0\\
5715	0\\
5716	0\\
5717	0\\
5718	0\\
5719	0\\
5720	0\\
5721	0\\
5722	0\\
5723	0\\
5724	0\\
5725	0\\
5726	0\\
5727	0\\
5728	0\\
5729	0\\
5730	0\\
5731	0\\
5732	0\\
5733	0\\
5734	0\\
5735	0\\
5736	0\\
5737	0\\
5738	0\\
5739	0\\
5740	0\\
5741	0\\
5742	0\\
5743	0\\
5744	0\\
5745	0\\
5746	0\\
5747	0\\
5748	0\\
5749	0\\
5750	0\\
5751	0\\
5752	0\\
5753	0\\
5754	0\\
5755	0\\
5756	0\\
5757	0\\
5758	0\\
5759	0\\
5760	0\\
5761	0\\
5762	0\\
5763	0\\
5764	0\\
5765	0\\
5766	0\\
5767	0\\
5768	0\\
5769	0\\
5770	0\\
5771	0\\
5772	0\\
5773	0\\
5774	0\\
5775	0\\
5776	0\\
5777	0\\
5778	0\\
5779	0\\
5780	0\\
5781	0\\
5782	0\\
5783	0\\
5784	0\\
5785	0\\
5786	0\\
5787	0\\
5788	0\\
5789	0\\
5790	0\\
5791	0\\
5792	0\\
5793	0\\
5794	0\\
5795	0\\
5796	0\\
5797	0\\
5798	0\\
5799	0\\
5800	0\\
5801	0\\
5802	0\\
5803	0\\
5804	0\\
5805	0\\
5806	0\\
5807	0\\
5808	0\\
5809	0\\
5810	0\\
5811	0\\
5812	0\\
5813	0\\
5814	0\\
5815	0\\
5816	0\\
5817	0\\
5818	0\\
5819	0\\
5820	0\\
5821	0\\
5822	0\\
5823	0\\
5824	0\\
5825	0\\
5826	0\\
5827	0\\
5828	0\\
5829	0\\
5830	0\\
5831	0\\
5832	0\\
5833	0\\
5834	0\\
5835	0\\
5836	0\\
5837	0\\
5838	0\\
5839	0\\
5840	0\\
5841	0\\
5842	0\\
5843	0\\
5844	0\\
5845	0\\
5846	0\\
5847	0\\
5848	0\\
5849	0\\
5850	0\\
5851	0\\
5852	0\\
5853	0\\
5854	0\\
5855	0\\
5856	0\\
5857	0\\
5858	0\\
5859	0\\
5860	0\\
5861	0\\
5862	0\\
5863	0\\
5864	0.418833\\
5865	2.347652\\
5866	2.347652\\
5867	2.347652\\
5868	2.347652\\
5869	2.347652\\
5870	1.336001\\
5871	1.103252\\
5872	1.336001\\
5873	2.347652\\
5874	1.336001\\
5875	0\\
5876	0\\
5877	0\\
5878	0\\
5879	0\\
5880	0\\
5881	0\\
5882	0\\
5883	0\\
5884	0\\
5885	0\\
5886	0\\
5887	0\\
5888	0.418833\\
5889	0\\
5890	0\\
5891	0\\
5892	0\\
5893	0\\
5894	0\\
5895	0\\
5896	0\\
5897	0\\
5898	0\\
5899	0\\
5900	0\\
5901	0\\
5902	0\\
5903	0\\
5904	0\\
5905	0\\
5906	0\\
5907	0\\
5908	0\\
5909	0\\
5910	0\\
5911	1.336001\\
5912	2.347652\\
5913	2.347652\\
5914	2.347652\\
5915	2.347652\\
5916	1.336001\\
5917	0\\
5918	0\\
5919	0\\
5920	0\\
5921	2.347652\\
5922	1.336001\\
5923	0\\
5924	1.835489\\
5925	1.336001\\
5926	0\\
5927	0\\
5928	0\\
5929	0\\
5930	0\\
5931	0\\
5932	0\\
5933	0\\
5934	0\\
5935	0\\
5936	2.347652\\
5937	2.347652\\
5938	2.347652\\
5939	2.347652\\
5940	1.336001\\
5941	0\\
5942	0\\
5943	0\\
5944	0\\
5945	0\\
5946	0\\
5947	0\\
5948	0\\
5949	0\\
5950	0\\
5951	0\\
5952	0\\
5953	0\\
5954	0\\
5955	0\\
5956	0\\
5957	0\\
5958	0\\
5959	0\\
5960	0\\
5961	0\\
5962	0\\
5963	1.336001\\
5964	1.103252\\
5965	0\\
5966	0\\
5967	0\\
5968	0\\
5969	0\\
5970	0\\
5971	0\\
5972	0\\
5973	0\\
5974	0\\
5975	0\\
5976	0\\
5977	0\\
5978	0\\
5979	0\\
5980	0\\
5981	0\\
5982	0\\
5983	0\\
5984	0\\
5985	0\\
5986	0\\
5987	0\\
5988	0\\
5989	0\\
5990	0\\
5991	0\\
5992	0\\
5993	0\\
5994	0\\
5995	0\\
5996	0\\
5997	0\\
5998	0\\
5999	0\\
6000	0\\
6001	0\\
6002	0\\
6003	0\\
6004	0\\
6005	0\\
6006	0\\
6007	1.336001\\
6008	2.192678\\
6009	1.336001\\
6010	1.336001\\
6011	1.835489\\
6012	1.336001\\
6013	0.418833\\
6014	0\\
6015	0\\
6016	0\\
6017	0\\
6018	0.418833\\
6019	0\\
6020	0\\
6021	0\\
6022	0\\
6023	0\\
6024	0\\
6025	0\\
6026	0\\
6027	0\\
6028	0\\
6029	0\\
6030	0\\
6031	2.347652\\
6032	2.347652\\
6033	2.347652\\
6034	1.103252\\
6035	1.336001\\
6036	1.336001\\
6037	0\\
6038	0\\
6039	0\\
6040	0\\
6041	1.103252\\
6042	0.418833\\
6043	0\\
6044	0\\
6045	0\\
6046	0\\
6047	0\\
6048	0\\
6049	0\\
6050	0\\
6051	0\\
6052	0\\
6053	0\\
6054	0\\
6055	0\\
6056	1.336001\\
6057	0\\
6058	0\\
6059	0\\
6060	0\\
6061	0\\
6062	0\\
6063	0\\
6064	0\\
6065	0\\
6066	0\\
6067	0\\
6068	0.418833\\
6069	0\\
6070	0\\
6071	0\\
6072	0\\
6073	0\\
6074	0\\
6075	0\\
6076	0\\
6077	0\\
6078	0\\
6079	1.336001\\
6080	1.336001\\
6081	1.103252\\
6082	0\\
6083	1.336001\\
6084	0.418833\\
6085	1.336001\\
6086	0\\
6087	0\\
6088	0\\
6089	0\\
6090	0\\
6091	0\\
6092	0\\
6093	0\\
6094	0\\
6095	0\\
6096	0\\
6097	0\\
6098	0\\
6099	0\\
6100	0\\
6101	0\\
6102	0\\
6103	0\\
6104	1.835489\\
6105	0\\
6106	0\\
6107	0\\
6108	0\\
6109	0\\
6110	0\\
6111	0\\
6112	0\\
6113	0\\
6114	0\\
6115	0\\
6116	1.103252\\
6117	0\\
6118	0\\
6119	0\\
6120	0\\
6121	0\\
6122	0\\
6123	0\\
6124	0\\
6125	0\\
6126	0\\
6127	0\\
6128	0\\
6129	0\\
6130	0\\
6131	0\\
6132	0\\
6133	0\\
6134	0\\
6135	0\\
6136	0\\
6137	0\\
6138	0\\
6139	0\\
6140	0\\
6141	0\\
6142	0\\
6143	0\\
6144	0\\
6145	0\\
6146	0\\
6147	0\\
6148	0\\
6149	0\\
6150	0\\
6151	0\\
6152	0\\
6153	0\\
6154	0\\
6155	0\\
6156	0\\
6157	0\\
6158	0\\
6159	0\\
6160	0\\
6161	0\\
6162	0\\
6163	0\\
6164	0\\
6165	0\\
6166	0\\
6167	0\\
6168	0\\
6169	0\\
6170	0\\
6171	0\\
6172	0\\
6173	0\\
6174	0\\
6175	1.336001\\
6176	0.418833\\
6177	1.336001\\
6178	2.347652\\
6179	1.336001\\
6180	0\\
6181	0\\
6182	0\\
6183	0\\
6184	0\\
6185	0\\
6186	0.418833\\
6187	1.336001\\
6188	1.336001\\
6189	0\\
6190	0\\
6191	0\\
6192	0\\
6193	0\\
6194	0\\
6195	0\\
6196	0\\
6197	0\\
6198	0\\
6199	1.336001\\
6200	1.336001\\
6201	1.336001\\
6202	0\\
6203	0\\
6204	0\\
6205	0\\
6206	0\\
6207	0\\
6208	0\\
6209	1.336001\\
6210	1.336001\\
6211	1.103252\\
6212	2.192678\\
6213	0\\
6214	0\\
6215	0\\
6216	0\\
6217	0\\
6218	0\\
6219	0\\
6220	0\\
6221	0\\
6222	0\\
6223	0\\
6224	0\\
6225	0\\
6226	0\\
6227	0\\
6228	0\\
6229	0\\
6230	0\\
6231	0\\
6232	0\\
6233	0\\
6234	0\\
6235	0\\
6236	0\\
6237	0\\
6238	0\\
6239	0\\
6240	0\\
6241	0\\
6242	0\\
6243	0\\
6244	0\\
6245	0\\
6246	0\\
6247	0\\
6248	2.347652\\
6249	2.347652\\
6250	2.347652\\
6251	3.834018\\
6252	2.514148\\
6253	2.347652\\
6254	2.347652\\
6255	2.347652\\
6256	2.347652\\
6257	2.347652\\
6258	2.347652\\
6259	1.336001\\
6260	2.347652\\
6261	0\\
6262	0\\
6263	0\\
6264	0\\
6265	0\\
6266	0\\
6267	0\\
6268	0\\
6269	0\\
6270	0\\
6271	1.336001\\
6272	2.347652\\
6273	2.347652\\
6274	2.347652\\
6275	2.347652\\
6276	2.347652\\
6277	1.336001\\
6278	1.103252\\
6279	0.418833\\
6280	0\\
6281	1.336001\\
6282	0\\
6283	0\\
6284	0\\
6285	0\\
6286	0\\
6287	0\\
6288	0\\
6289	0\\
6290	0\\
6291	0\\
6292	0\\
6293	0\\
6294	0\\
6295	0\\
6296	0\\
6297	0\\
6298	0\\
6299	0\\
6300	0\\
6301	0\\
6302	0\\
6303	0\\
6304	0\\
6305	0\\
6306	0\\
6307	0\\
6308	0\\
6309	0\\
6310	0\\
6311	0\\
6312	0\\
6313	0\\
6314	0\\
6315	0\\
6316	0\\
6317	0\\
6318	0\\
6319	0\\
6320	0\\
6321	0\\
6322	0\\
6323	0\\
6324	0\\
6325	0\\
6326	0\\
6327	0\\
6328	0\\
6329	0\\
6330	0\\
6331	0\\
6332	0\\
6333	0\\
6334	0\\
6335	0\\
6336	0\\
6337	0\\
6338	0\\
6339	0\\
6340	0\\
6341	0\\
6342	0\\
6343	0\\
6344	0\\
6345	0\\
6346	0\\
6347	0\\
6348	0\\
6349	0\\
6350	0\\
6351	0\\
6352	0\\
6353	0\\
6354	0\\
6355	0\\
6356	1.336001\\
6357	0\\
6358	0\\
6359	0\\
6360	0\\
6361	0\\
6362	0\\
6363	0\\
6364	0\\
6365	0\\
6366	0\\
6367	2.347652\\
6368	2.347652\\
6369	0.418833\\
6370	0\\
6371	0\\
6372	0\\
6373	0\\
6374	0\\
6375	0\\
6376	0\\
6377	0\\
6378	1.103252\\
6379	2.347652\\
6380	2.347652\\
6381	0.418833\\
6382	0\\
6383	0\\
6384	0\\
6385	0\\
6386	0\\
6387	0\\
6388	0\\
6389	0\\
6390	0\\
6391	0\\
6392	1.336001\\
6393	1.336001\\
6394	1.835489\\
6395	2.347652\\
6396	2.347652\\
6397	2.347652\\
6398	1.103252\\
6399	1.103252\\
6400	0\\
6401	1.336001\\
6402	1.336001\\
6403	1.103252\\
6404	1.336001\\
6405	0\\
6406	0\\
6407	0\\
6408	0\\
6409	0\\
6410	0\\
6411	0\\
6412	0\\
6413	0\\
6414	0\\
6415	1.336001\\
6416	1.336001\\
6417	1.103252\\
6418	0.418833\\
6419	0\\
6420	0\\
6421	0\\
6422	0\\
6423	0\\
6424	0\\
6425	0\\
6426	0\\
6427	0\\
6428	0\\
6429	0\\
6430	0\\
6431	0\\
6432	0\\
6433	0\\
6434	0\\
6435	0\\
6436	0\\
6437	0\\
6438	0\\
6439	0.418833\\
6440	1.336001\\
6441	2.192678\\
6442	1.336001\\
6443	1.336001\\
6444	1.336001\\
6445	1.103252\\
6446	1.103252\\
6447	0.418833\\
6448	2.347652\\
6449	2.347652\\
6450	2.347652\\
6451	2.347652\\
6452	2.347652\\
6453	2.347652\\
6454	0\\
6455	0\\
6456	0\\
6457	0\\
6458	0\\
6459	0\\
6460	0\\
6461	0\\
6462	0\\
6463	0\\
6464	0\\
6465	0\\
6466	1.336001\\
6467	0.418833\\
6468	0\\
6469	0\\
6470	0\\
6471	0\\
6472	0\\
6473	0\\
6474	0\\
6475	0\\
6476	0\\
6477	0\\
6478	0\\
6479	0\\
6480	0\\
6481	0\\
6482	0\\
6483	0\\
6484	0\\
6485	0\\
6486	0\\
6487	0\\
6488	0\\
6489	0\\
6490	0\\
6491	0\\
6492	0\\
6493	0\\
6494	0\\
6495	0\\
6496	0\\
6497	0\\
6498	0\\
6499	0.418833\\
6500	0.418833\\
6501	0\\
6502	0\\
6503	0\\
6504	0\\
6505	0\\
6506	0\\
6507	0\\
6508	0\\
6509	0\\
6510	0\\
6511	2.347652\\
6512	2.347652\\
6513	1.336001\\
6514	1.336001\\
6515	1.336001\\
6516	0\\
6517	0\\
6518	2.192678\\
6519	2.347652\\
6520	3.834018\\
6521	4.075984\\
6522	4.075984\\
6523	3.840931\\
6524	3.840931\\
6525	2.347652\\
6526	0\\
6527	0\\
6528	0\\
6529	0\\
6530	0\\
6531	0\\
6532	0\\
6533	0\\
6534	0\\
6535	8.675654\\
6536	8.675654\\
6537	4.075984\\
6538	3.840931\\
6539	2.853477\\
6540	2.853477\\
6541	2.347652\\
6542	2.347652\\
6543	2.347652\\
6544	2.347652\\
6545	2.347652\\
6546	3.834018\\
6547	3.840931\\
6548	3.840931\\
6549	2.347652\\
6550	1.336001\\
6551	0\\
6552	0\\
6553	0\\
6554	0\\
6555	0\\
6556	0\\
6557	0\\
6558	0\\
6559	3.955756\\
6560	4.197836\\
6561	4.197836\\
6562	4.197836\\
6563	5.164376\\
6564	3.955756\\
6565	3.955756\\
6566	3.955756\\
6567	2.417835\\
6568	2.417835\\
6569	3.955756\\
6570	4.197836\\
6571	10.040986\\
6572	11.327926\\
6573	3.955756\\
6574	3.955756\\
6575	1.136234\\
6576	2.417835\\
6577	0\\
6578	0\\
6579	0\\
6580	0\\
6581	0\\
6582	3.955756\\
6583	22.442846\\
6584	20.292771\\
6585	28.81516\\
6586	17.89969\\
6587	17.89969\\
6588	17.89969\\
6589	17.89969\\
6590	16.15531\\
6591	16.70181\\
6592	17.89969\\
6593	17.89969\\
6594	17.89969\\
6595	34.33088\\
6596	17.89969\\
6597	13.097841\\
6598	10.040986\\
6599	2.417835\\
6600	0\\
6601	0\\
6602	0\\
6603	0\\
6604	0\\
6605	0\\
6606	0\\
6607	3.955756\\
6608	4.197836\\
6609	3.948636\\
6610	2.938782\\
6611	2.938782\\
6612	2.417835\\
6613	2.417835\\
6614	2.417835\\
6615	1.375941\\
6616	1.375941\\
6617	2.417835\\
6618	3.948636\\
6619	3.955756\\
6620	8.935012\\
6621	2.417835\\
6622	2.589308\\
6623	1.375941\\
6624	0\\
6625	0\\
6626	0\\
6627	0\\
6628	0\\
6629	0\\
6630	0\\
6631	0\\
6632	0\\
6633	0\\
6634	0\\
6635	0\\
6636	0\\
6637	0\\
6638	0\\
6639	0\\
6640	0\\
6641	0\\
6642	0\\
6643	0\\
6644	0\\
6645	0\\
6646	0\\
6647	0\\
6648	0\\
6649	0\\
6650	0\\
6651	0\\
6652	0\\
6653	0\\
6654	0\\
6655	0\\
6656	0\\
6657	0\\
6658	0\\
6659	0\\
6660	0.431354\\
6661	0\\
6662	0\\
6663	0\\
6664	0\\
6665	1.375941\\
6666	2.417835\\
6667	2.589308\\
6668	2.417835\\
6669	2.417835\\
6670	0\\
6671	0\\
6672	0\\
6673	0\\
6674	0\\
6675	0\\
6676	0\\
6677	0\\
6678	0\\
6679	2.938782\\
6680	3.955756\\
6681	3.955756\\
6682	3.955756\\
6683	3.955756\\
6684	3.955756\\
6685	4.197836\\
6686	3.955756\\
6687	2.417835\\
6688	2.417835\\
6689	1.375941\\
6690	2.589308\\
6691	2.938782\\
6692	2.417835\\
6693	0\\
6694	0.431354\\
6695	0\\
6696	0\\
6697	0\\
6698	0\\
6699	0\\
6700	0\\
6701	0\\
6702	0\\
6703	2.417835\\
6704	2.417835\\
6705	2.417835\\
6706	2.417835\\
6707	2.417835\\
6708	2.417835\\
6709	2.417835\\
6710	0\\
6711	0\\
6712	0\\
6713	0\\
6714	2.417835\\
6715	2.938782\\
6716	3.948636\\
6717	1.375941\\
6718	0\\
6719	0\\
6720	0\\
6721	0\\
6722	0\\
6723	0\\
6724	0\\
6725	0\\
6726	0\\
6727	1.375941\\
6728	2.417835\\
6729	2.938782\\
6730	3.955756\\
6731	3.955756\\
6732	3.955756\\
6733	2.417835\\
6734	1.375941\\
6735	0\\
6736	0\\
6737	0\\
6738	0\\
6739	2.417835\\
6740	1.375941\\
6741	0\\
6742	0\\
6743	0\\
6744	0\\
6745	0\\
6746	0\\
6747	0\\
6748	0\\
6749	0\\
6750	0\\
6751	2.417835\\
6752	2.417835\\
6753	2.417835\\
6754	2.417835\\
6755	2.417835\\
6756	1.375941\\
6757	1.375941\\
6758	0.431354\\
6759	0\\
6760	0\\
6761	0\\
6762	2.417835\\
6763	2.417835\\
6764	2.417835\\
6765	0\\
6766	0\\
6767	0\\
6768	0\\
6769	0\\
6770	0\\
6771	0\\
6772	0\\
6773	0\\
6774	0\\
6775	3.955756\\
6776	2.938782\\
6777	2.589308\\
6778	2.589308\\
6779	3.948636\\
6780	3.948636\\
6781	2.417835\\
6782	2.417835\\
6783	1.375941\\
6784	0.431354\\
6785	1.375941\\
6786	3.948636\\
6787	3.955756\\
6788	4.197836\\
6789	2.417835\\
6790	1.136234\\
6791	0\\
6792	0\\
6793	0\\
6794	0\\
6795	0\\
6796	0\\
6797	0\\
6798	0\\
6799	0\\
6800	0\\
6801	0\\
6802	0\\
6803	0\\
6804	0\\
6805	0\\
6806	0\\
6807	0\\
6808	0\\
6809	0\\
6810	0\\
6811	0\\
6812	0\\
6813	0\\
6814	0\\
6815	0\\
6816	0\\
6817	0\\
6818	0\\
6819	0\\
6820	0\\
6821	0\\
6822	0\\
6823	0\\
6824	0\\
6825	0\\
6826	0\\
6827	0\\
6828	0\\
6829	0\\
6830	0\\
6831	0\\
6832	0\\
6833	0\\
6834	0\\
6835	0\\
6836	0\\
6837	0\\
6838	0\\
6839	0\\
6840	0\\
6841	0\\
6842	0\\
6843	0\\
6844	0\\
6845	0\\
6846	0\\
6847	0\\
6848	0\\
6849	0\\
6850	0\\
6851	0\\
6852	0\\
6853	0\\
6854	0\\
6855	0\\
6856	0\\
6857	0\\
6858	2.258228\\
6859	2.589308\\
6860	2.417835\\
6861	0\\
6862	0\\
6863	0\\
6864	0\\
6865	0\\
6866	0\\
6867	0\\
6868	0\\
6869	0\\
6870	0\\
6871	2.417835\\
6872	2.417835\\
6873	1.375941\\
6874	0\\
6875	1.136234\\
6876	0\\
6877	0\\
6878	0\\
6879	0\\
6880	0\\
6881	1.375941\\
6882	2.417835\\
6883	2.589308\\
6884	2.417835\\
6885	0.431354\\
6886	0\\
6887	0\\
6888	0\\
6889	0\\
6890	0\\
6891	0\\
6892	0\\
6893	0\\
6894	2.417835\\
6895	13.097841\\
6896	3.955756\\
6897	2.417835\\
6898	2.417835\\
6899	2.417835\\
6900	2.417835\\
6901	2.417835\\
6902	1.890361\\
6903	2.417835\\
6904	2.417835\\
6905	3.948636\\
6906	4.197836\\
6907	12.438647\\
6908	3.955756\\
6909	2.417835\\
6910	1.375941\\
6911	0\\
6912	0\\
6913	0\\
6914	0\\
6915	0\\
6916	0\\
6917	0\\
6918	0\\
6919	2.417835\\
6920	2.417835\\
6921	2.417835\\
6922	2.417835\\
6923	2.417835\\
6924	2.417835\\
6925	1.375941\\
6926	0\\
6927	0\\
6928	0\\
6929	2.417835\\
6930	3.948636\\
6931	4.197836\\
6932	3.955756\\
6933	1.890361\\
6934	1.136234\\
6935	0\\
6936	0\\
6937	0\\
6938	0\\
6939	0\\
6940	0\\
6941	0\\
6942	0\\
6943	2.417835\\
6944	3.948636\\
6945	2.938782\\
6946	2.417835\\
6947	2.417835\\
6948	0.431354\\
6949	0\\
6950	0\\
6951	0\\
6952	0\\
6953	1.375941\\
6954	1.136234\\
6955	2.417835\\
6956	1.375941\\
6957	0\\
6958	0\\
6959	0\\
6960	0\\
6961	0\\
6962	0\\
6963	0\\
6964	0\\
6965	0\\
6966	0\\
6967	0\\
6968	0\\
6969	0\\
6970	0\\
6971	0\\
6972	0\\
6973	0\\
6974	0\\
6975	0\\
6976	0\\
6977	0\\
6978	0\\
6979	0\\
6980	0\\
6981	0\\
6982	0\\
6983	0\\
6984	0\\
6985	0\\
6986	0\\
6987	0\\
6988	0\\
6989	0\\
6990	0\\
6991	0\\
6992	0\\
6993	0\\
6994	0\\
6995	0\\
6996	0\\
6997	0\\
6998	0\\
6999	0\\
7000	0\\
7001	0\\
7002	0\\
7003	0\\
7004	0\\
7005	0\\
7006	0\\
7007	0\\
7008	0\\
7009	0\\
7010	0\\
7011	0\\
7012	0\\
7013	0\\
7014	0\\
7015	0\\
7016	0\\
7017	0\\
7018	0\\
7019	0\\
7020	0\\
7021	0\\
7022	0\\
7023	0\\
7024	0\\
7025	0\\
7026	0\\
7027	0\\
7028	0\\
7029	0\\
7030	0\\
7031	0\\
7032	0\\
7033	0\\
7034	0\\
7035	0\\
7036	0\\
7037	0\\
7038	0\\
7039	0\\
7040	0\\
7041	0\\
7042	0\\
7043	0\\
7044	0\\
7045	0\\
7046	0\\
7047	0\\
7048	0\\
7049	0\\
7050	0\\
7051	0\\
7052	0\\
7053	0\\
7054	0\\
7055	0\\
7056	0\\
7057	0\\
7058	0\\
7059	0\\
7060	0\\
7061	0\\
7062	0\\
7063	0\\
7064	0\\
7065	0\\
7066	0\\
7067	0\\
7068	0\\
7069	0\\
7070	0\\
7071	0\\
7072	0\\
7073	0\\
7074	1.375941\\
7075	3.955756\\
7076	2.417835\\
7077	0\\
7078	0\\
7079	0\\
7080	0\\
7081	0\\
7082	0\\
7083	0\\
7084	0\\
7085	0\\
7086	0.431354\\
7087	4.197836\\
7088	4.197836\\
7089	3.955756\\
7090	3.955756\\
7091	2.589308\\
7092	1.375941\\
7093	0\\
7094	1.375941\\
7095	1.375941\\
7096	0\\
7097	1.136234\\
7098	3.079995\\
7099	3.955756\\
7100	3.955756\\
7101	2.417835\\
7102	0\\
7103	0\\
7104	0\\
7105	0\\
7106	0\\
7107	0\\
7108	0\\
7109	0\\
7110	0\\
7111	0\\
7112	0\\
7113	0\\
7114	0\\
7115	0\\
7116	0\\
7117	0\\
7118	0\\
7119	0\\
7120	0\\
7121	0\\
7122	0\\
7123	0\\
7124	0\\
7125	0\\
7126	0\\
7127	0\\
7128	0\\
7129	0\\
7130	0\\
7131	0\\
7132	0\\
7133	0\\
7134	0\\
7135	0\\
7136	0\\
7137	0\\
7138	0\\
7139	0\\
7140	0\\
7141	0\\
7142	0\\
7143	0\\
7144	0\\
7145	0\\
7146	0\\
7147	0\\
7148	0\\
7149	0\\
7150	0\\
7151	0\\
7152	0\\
7153	0\\
7154	0\\
7155	0\\
7156	0\\
7157	0\\
7158	0\\
7159	0\\
7160	0\\
7161	0\\
7162	0\\
7163	0\\
7164	0\\
7165	0\\
7166	0\\
7167	0\\
7168	0\\
7169	0\\
7170	0\\
7171	0\\
7172	0\\
7173	0\\
7174	0\\
7175	0\\
7176	0\\
7177	0\\
7178	0\\
7179	0\\
7180	0\\
7181	0\\
7182	0\\
7183	0\\
7184	0\\
7185	0\\
7186	0\\
7187	0\\
7188	0\\
7189	0\\
7190	0\\
7191	0\\
7192	0\\
7193	0\\
7194	1.375941\\
7195	2.417835\\
7196	2.417835\\
7197	0\\
7198	0\\
7199	0\\
7200	0\\
7201	0\\
7202	0\\
7203	0\\
7204	0\\
7205	0\\
7206	0\\
7207	0\\
7208	0\\
7209	0\\
7210	0\\
7211	0\\
7212	0\\
7213	0\\
7214	0\\
7215	0\\
7216	0\\
7217	0\\
7218	3.955756\\
7219	3.955756\\
7220	3.948636\\
7221	2.258228\\
7222	0\\
7223	0\\
7224	0\\
7225	0\\
7226	0\\
7227	0\\
7228	0\\
7229	0\\
7230	0\\
7231	0\\
7232	0\\
7233	0\\
7234	0\\
7235	0\\
7236	0\\
7237	1.136234\\
7238	0.431354\\
7239	0\\
7240	1.136234\\
7241	0\\
7242	2.417835\\
7243	2.938782\\
7244	2.417835\\
7245	1.375941\\
7246	0\\
7247	0\\
7248	0\\
7249	0\\
7250	0\\
7251	0\\
7252	0\\
7253	0\\
7254	0\\
7255	0\\
7256	0\\
7257	0\\
7258	0\\
7259	0\\
7260	0\\
7261	0\\
7262	0\\
7263	0\\
7264	0\\
7265	0\\
7266	0\\
7267	2.938782\\
7268	2.417835\\
7269	0\\
7270	0\\
7271	0\\
7272	0\\
7273	0\\
7274	0\\
7275	0\\
7276	0\\
7277	0\\
7278	0\\
7279	0\\
7280	0\\
7281	0\\
7282	0\\
7283	0\\
7284	0\\
7285	0\\
7286	0\\
7287	0\\
7288	0\\
7289	0\\
7290	0\\
7291	0\\
7292	0\\
7293	0\\
7294	0\\
7295	0\\
7296	0\\
7297	0\\
7298	0\\
7299	0\\
7300	0\\
7301	0\\
7302	0\\
7303	0\\
7304	0\\
7305	0\\
7306	0\\
7307	0\\
7308	0\\
7309	0\\
7310	0\\
7311	0\\
7312	0\\
7313	0\\
7314	0\\
7315	0\\
7316	0\\
7317	0\\
7318	0\\
7319	0\\
7320	0\\
7321	0\\
7322	0\\
7323	0\\
7324	0\\
7325	0\\
7326	0\\
7327	0\\
7328	0\\
7329	0\\
7330	0\\
7331	0\\
7332	0\\
7333	0\\
7334	0\\
7335	0\\
7336	0\\
7337	0\\
7338	0\\
7339	0\\
7340	0\\
7341	0\\
7342	0\\
7343	0\\
7344	0\\
7345	0\\
7346	0\\
7347	0\\
7348	0\\
7349	0\\
7350	0\\
7351	0\\
7352	0\\
7353	0\\
7354	0\\
7355	0\\
7356	0\\
7357	0\\
7358	0\\
7359	0\\
7360	0\\
7361	0\\
7362	0\\
7363	1.453239\\
7364	1.453239\\
7365	0\\
7366	0\\
7367	0\\
7368	0\\
7369	0\\
7370	0\\
7371	0\\
7372	0\\
7373	0\\
7374	0\\
7375	0\\
7376	0\\
7377	0\\
7378	0\\
7379	2.553665\\
7380	2.553665\\
7381	1.996559\\
7382	0.455586\\
7383	0\\
7384	0.455586\\
7385	0\\
7386	2.553665\\
7387	3.023665\\
7388	2.553665\\
7389	0\\
7390	0\\
7391	0\\
7392	0\\
7393	0\\
7394	0\\
7395	0\\
7396	0\\
7397	0\\
7398	0\\
7399	0\\
7400	3.253025\\
7401	9.436969\\
7402	4.433665\\
7403	4.433665\\
7404	9.436969\\
7405	4.177985\\
7406	3.103878\\
7407	3.023665\\
7408	3.103878\\
7409	3.253025\\
7410	9.436969\\
7411	10.807488\\
7412	4.177985\\
7413	2.553665\\
7414	0\\
7415	0\\
7416	0\\
7417	0\\
7418	0\\
7419	0\\
7420	0\\
7421	0\\
7422	0\\
7423	0\\
7424	4.177985\\
7425	2.553665\\
7426	3.023665\\
7427	4.170465\\
7428	3.253025\\
7429	4.170465\\
7430	4.170465\\
7431	2.553665\\
7432	2.553665\\
7433	2.553665\\
7434	3.103878\\
7435	4.433665\\
7436	4.170465\\
7437	2.553665\\
7438	0\\
7439	0\\
7440	0\\
7441	0\\
7442	0\\
7443	0\\
7444	0\\
7445	0\\
7446	0\\
7447	0\\
7448	0\\
7449	0\\
7450	0\\
7451	0\\
7452	0\\
7453	0\\
7454	0\\
7455	0\\
7456	0\\
7457	0\\
7458	0\\
7459	0\\
7460	0\\
7461	0\\
7462	0\\
7463	0\\
7464	0\\
7465	0\\
7466	0\\
7467	0\\
7468	0\\
7469	0\\
7470	0\\
7471	0\\
7472	0\\
7473	0\\
7474	0\\
7475	0\\
7476	0\\
7477	0\\
7478	0\\
7479	0\\
7480	0\\
7481	0\\
7482	0\\
7483	0\\
7484	0\\
7485	0\\
7486	0\\
7487	0\\
7488	0\\
7489	0\\
7490	0\\
7491	0\\
7492	0\\
7493	0\\
7494	0\\
7495	0\\
7496	0\\
7497	0\\
7498	0\\
7499	0\\
7500	0\\
7501	0\\
7502	0\\
7503	0\\
7504	0\\
7505	0\\
7506	0\\
7507	0\\
7508	0\\
7509	0\\
7510	0\\
7511	0\\
7512	0\\
7513	0\\
7514	0\\
7515	0\\
7516	0\\
7517	0\\
7518	0\\
7519	0\\
7520	0\\
7521	0\\
7522	0\\
7523	0\\
7524	0\\
7525	0\\
7526	0\\
7527	0\\
7528	0\\
7529	0\\
7530	0\\
7531	0\\
7532	0\\
7533	0\\
7534	0\\
7535	0\\
7536	0\\
7537	0\\
7538	0\\
7539	0\\
7540	0\\
7541	0\\
7542	0\\
7543	0\\
7544	0\\
7545	0\\
7546	0\\
7547	0\\
7548	0\\
7549	0\\
7550	0\\
7551	0\\
7552	0\\
7553	0\\
7554	0\\
7555	0\\
7556	0\\
7557	0\\
7558	0\\
7559	0\\
7560	0\\
7561	0\\
7562	0\\
7563	0\\
7564	0\\
7565	0\\
7566	0\\
7567	0\\
7568	0\\
7569	0\\
7570	0\\
7571	0\\
7572	0\\
7573	0\\
7574	0\\
7575	0\\
7576	0\\
7577	0\\
7578	0\\
7579	0\\
7580	0\\
7581	0\\
7582	0\\
7583	0\\
7584	0\\
7585	0\\
7586	0\\
7587	0\\
7588	0\\
7589	0\\
7590	0\\
7591	0\\
7592	0\\
7593	0\\
7594	0\\
7595	0\\
7596	0\\
7597	0\\
7598	0\\
7599	0\\
7600	0\\
7601	0\\
7602	0\\
7603	0\\
7604	0\\
7605	0\\
7606	0\\
7607	0\\
7608	0\\
7609	0\\
7610	0\\
7611	0\\
7612	0\\
7613	0\\
7614	0\\
7615	0\\
7616	0\\
7617	0\\
7618	0\\
7619	0\\
7620	0\\
7621	0\\
7622	0\\
7623	0\\
7624	0\\
7625	1.200066\\
7626	0.455586\\
7627	0\\
7628	0\\
7629	0\\
7630	0\\
7631	0\\
7632	0\\
7633	0\\
7634	0\\
7635	0\\
7636	0\\
7637	0\\
7638	0\\
7639	0\\
7640	0\\
7641	0\\
7642	0\\
7643	0\\
7644	0\\
7645	0\\
7646	0\\
7647	0\\
7648	0\\
7649	0\\
7650	0\\
7651	0\\
7652	0\\
7653	0\\
7654	0\\
7655	0\\
7656	0\\
7657	0\\
7658	0\\
7659	0\\
7660	0\\
7661	0\\
7662	0\\
7663	0\\
7664	0\\
7665	0\\
7666	0\\
7667	0\\
7668	0\\
7669	0\\
7670	0\\
7671	0\\
7672	0\\
7673	0\\
7674	0\\
7675	0\\
7676	0\\
7677	0\\
7678	0\\
7679	0\\
7680	0\\
7681	0\\
7682	0\\
7683	0\\
7684	0\\
7685	0\\
7686	0\\
7687	0\\
7688	0\\
7689	0\\
7690	0\\
7691	0\\
7692	0\\
7693	0\\
7694	0\\
7695	0\\
7696	1.200066\\
7697	2.553665\\
7698	4.170465\\
7699	4.170465\\
7700	2.553665\\
7701	0\\
7702	0\\
7703	0\\
7704	0\\
7705	0\\
7706	0\\
7707	0\\
7708	0\\
7709	0\\
7710	0\\
7711	0\\
7712	0\\
7713	2.553665\\
7714	1.453239\\
7715	0.455586\\
7716	0\\
7717	0\\
7718	0\\
7719	0\\
7720	1.453239\\
7721	1.453239\\
7722	2.553665\\
7723	2.553665\\
7724	2.553665\\
7725	0\\
7726	0\\
7727	0\\
7728	0\\
7729	0\\
7730	0\\
7731	0\\
7732	0\\
7733	0\\
7734	0\\
7735	0\\
7736	1.453239\\
7737	1.996559\\
7738	2.553665\\
7739	2.553665\\
7740	2.553665\\
7741	2.553665\\
7742	2.553665\\
7743	2.553665\\
7744	1.453239\\
7745	1.453239\\
7746	2.553665\\
7747	2.553665\\
7748	1.996559\\
7749	0\\
7750	0\\
7751	0\\
7752	0\\
7753	0\\
7754	0\\
7755	0\\
7756	0\\
7757	0\\
7758	0\\
7759	0\\
7760	0\\
7761	1.996559\\
7762	1.200066\\
7763	0\\
7764	0\\
7765	0\\
7766	0\\
7767	0\\
7768	0\\
7769	1.453239\\
7770	2.553665\\
7771	2.553665\\
7772	2.553665\\
7773	0\\
7774	0\\
7775	0\\
7776	0\\
7777	0\\
7778	0\\
7779	0\\
7780	0\\
7781	0\\
7782	0\\
7783	0\\
7784	0\\
7785	0\\
7786	0\\
7787	0\\
7788	0\\
7789	0\\
7790	0\\
7791	0\\
7792	0\\
7793	0\\
7794	0\\
7795	0\\
7796	0\\
7797	0\\
7798	0\\
7799	0\\
7800	0\\
7801	0\\
7802	0\\
7803	0\\
7804	0\\
7805	0\\
7806	0\\
7807	0\\
7808	0\\
7809	0\\
7810	0\\
7811	0\\
7812	0\\
7813	0\\
7814	0\\
7815	0\\
7816	0\\
7817	0\\
7818	0\\
7819	0\\
7820	0\\
7821	0\\
7822	0\\
7823	0\\
7824	0\\
7825	0\\
7826	0\\
7827	0\\
7828	0\\
7829	0\\
7830	0\\
7831	0\\
7832	0\\
7833	0\\
7834	0\\
7835	0\\
7836	0\\
7837	0\\
7838	0\\
7839	0\\
7840	0\\
7841	0\\
7842	0\\
7843	0\\
7844	0\\
7845	0\\
7846	0\\
7847	0\\
7848	0\\
7849	0\\
7850	0\\
7851	0\\
7852	0\\
7853	0\\
7854	0\\
7855	0\\
7856	0\\
7857	0\\
7858	0\\
7859	0\\
7860	0\\
7861	0\\
7862	0\\
7863	0\\
7864	0\\
7865	0\\
7866	0\\
7867	0\\
7868	0\\
7869	0\\
7870	0\\
7871	0\\
7872	0\\
7873	0\\
7874	0\\
7875	0\\
7876	0\\
7877	0\\
7878	0\\
7879	0\\
7880	0\\
7881	0\\
7882	0\\
7883	0\\
7884	0\\
7885	0\\
7886	0\\
7887	0\\
7888	0\\
7889	0\\
7890	1.453239\\
7891	0.455586\\
7892	0\\
7893	0\\
7894	0\\
7895	0\\
7896	0\\
7897	0\\
7898	0\\
7899	0\\
7900	0\\
7901	0\\
7902	0\\
7903	0\\
7904	0\\
7905	0\\
7906	0\\
7907	0\\
7908	0\\
7909	0\\
7910	0\\
7911	0\\
7912	0\\
7913	1.200066\\
7914	1.453239\\
7915	1.200066\\
7916	0\\
7917	0\\
7918	0\\
7919	0\\
7920	0\\
7921	0\\
7922	0\\
7923	0\\
7924	0\\
7925	0\\
7926	0\\
7927	0\\
7928	0\\
7929	1.200066\\
7930	0\\
7931	0\\
7932	0\\
7933	0\\
7934	0\\
7935	0\\
7936	0\\
7937	0\\
7938	0\\
7939	0\\
7940	0\\
7941	0\\
7942	0\\
7943	0\\
7944	0\\
7945	0\\
7946	0\\
7947	0\\
7948	0\\
7949	0\\
7950	0\\
7951	0\\
7952	0\\
7953	0\\
7954	0\\
7955	0\\
7956	0\\
7957	0\\
7958	0\\
7959	0\\
7960	0\\
7961	0\\
7962	0\\
7963	0\\
7964	0\\
7965	0\\
7966	0\\
7967	0\\
7968	0\\
7969	0\\
7970	0\\
7971	0\\
7972	0\\
7973	0\\
7974	0\\
7975	0\\
7976	0\\
7977	0\\
7978	0\\
7979	0\\
7980	0\\
7981	0\\
7982	0\\
7983	0\\
7984	0\\
7985	0\\
7986	0\\
7987	0\\
7988	0\\
7989	0\\
7990	0\\
7991	0\\
7992	0\\
7993	0\\
7994	0\\
7995	0\\
7996	0\\
7997	0\\
7998	0\\
7999	0\\
8000	0\\
8001	0\\
};
\addplot [color=mycolor1,solid,line width=1.0pt,forget plot]
  table[row sep=crcr]{%
8001	0\\
8002	0\\
8003	0\\
8004	0\\
8005	1.453239\\
8006	0\\
8007	0\\
8008	0\\
8009	2.553665\\
8010	2.553665\\
8011	2.553665\\
8012	2.553665\\
8013	1.453239\\
8014	0\\
8015	0\\
8016	0\\
8017	0\\
8018	0\\
8019	0\\
8020	0\\
8021	0\\
8022	0\\
8023	0\\
8024	0\\
8025	0\\
8026	2.58423\\
8027	2.58423\\
8028	4.220381\\
8029	4.227991\\
8030	4.227991\\
8031	3.141029\\
8032	2.58423\\
8033	4.227991\\
8034	9.54992\\
8035	9.54992\\
8036	9.54992\\
8037	2.58423\\
8038	0\\
8039	0\\
8040	0\\
8041	0\\
8042	0\\
8043	0\\
8044	0\\
8045	0\\
8046	0\\
8047	0\\
8048	0\\
8049	2.58423\\
8050	2.58423\\
8051	2.58423\\
8052	2.58423\\
8053	2.58423\\
8054	2.58423\\
8055	2.58423\\
8056	2.58423\\
8057	2.767504\\
8058	4.227991\\
8059	4.227991\\
8060	2.58423\\
8061	0\\
8062	0\\
8063	0\\
8064	0\\
8065	0\\
8066	0\\
8067	0\\
8068	0\\
8069	0\\
8070	0\\
8071	0\\
8072	4.220381\\
8073	9.54992\\
8074	9.54992\\
8075	4.486731\\
8076	4.486731\\
8077	4.613564\\
8078	9.54992\\
8079	12.107515\\
8080	13.381712\\
8081	13.381712\\
8082	13.381712\\
8083	12.107515\\
8084	10.936843\\
8085	4.486731\\
8086	0\\
8087	0\\
8088	0\\
8089	0\\
8090	0\\
8091	0\\
8092	0\\
8093	0\\
8094	0\\
8095	0\\
8096	4.486731\\
8097	10.936843\\
8098	9.54992\\
8099	9.54992\\
8100	9.54992\\
8101	12.107515\\
8102	10.732007\\
8103	10.936843\\
8104	10.732007\\
8105	13.381712\\
8106	13.928212\\
8107	13.928212\\
8108	13.381712\\
8109	4.227991\\
8110	2.58423\\
8111	0\\
8112	0\\
8113	0\\
8114	0\\
8115	0\\
8116	0\\
8117	0\\
8118	0\\
8119	0\\
8120	4.227991\\
8121	10.936843\\
8122	9.54992\\
8123	10.732007\\
8124	10.732007\\
8125	9.54992\\
8126	4.486731\\
8127	4.227991\\
8128	4.227991\\
8129	4.486731\\
8130	9.54992\\
8131	9.54992\\
8132	4.486731\\
8133	4.227991\\
8134	0.461039\\
8135	0\\
8136	0\\
8137	0\\
8138	0\\
8139	0\\
8140	0\\
8141	0\\
8142	0\\
8143	0\\
8144	0\\
8145	0\\
8146	0\\
8147	0\\
8148	0\\
8149	0\\
8150	0\\
8151	0\\
8152	0\\
8153	0\\
8154	0\\
8155	0\\
8156	0\\
8157	0\\
8158	0\\
8159	0\\
8160	0\\
8161	0\\
8162	0\\
8163	0\\
8164	0\\
8165	0\\
8166	0\\
8167	0\\
8168	0\\
8169	0\\
8170	0\\
8171	0\\
8172	0\\
8173	0\\
8174	0\\
8175	0\\
8176	0\\
8177	0\\
8178	0\\
8179	0\\
8180	0\\
8181	0\\
8182	0\\
8183	0\\
8184	0\\
8185	0\\
8186	0\\
8187	0\\
8188	0\\
8189	0\\
8190	0\\
8191	0\\
8192	0\\
8193	0\\
8194	0\\
8195	0\\
8196	0\\
8197	0\\
8198	0\\
8199	1.470633\\
8200	1.470633\\
8201	2.58423\\
8202	2.58423\\
8203	3.141029\\
8204	2.58423\\
8205	1.470633\\
8206	0\\
8207	0\\
8208	0\\
8209	0\\
8210	0\\
8211	0\\
8212	0\\
8213	0\\
8214	0\\
8215	0\\
8216	4.486731\\
8217	12.107515\\
8218	10.936843\\
8219	4.227991\\
8220	4.227991\\
8221	4.227991\\
8222	4.486731\\
8223	4.486731\\
8224	4.486731\\
8225	10.936843\\
8226	13.999235\\
8227	13.999235\\
8228	13.381712\\
8229	4.227991\\
8230	0\\
8231	0\\
8232	0\\
8233	0\\
8234	0\\
8235	0\\
8236	0\\
8237	0\\
8238	0\\
8239	0\\
8240	2.58423\\
8241	4.227991\\
8242	3.29196\\
8243	2.58423\\
8244	2.58423\\
8245	2.413639\\
8246	3.059855\\
8247	4.220381\\
8248	4.486731\\
8249	12.107515\\
8250	13.381712\\
8251	13.928212\\
8252	13.928212\\
8253	12.107515\\
8254	4.227991\\
8255	2.58423\\
8256	0\\
8257	0\\
8258	0\\
8259	0\\
8260	0\\
8261	0\\
8262	0\\
8263	0\\
8264	2.020456\\
8265	2.020456\\
8266	0\\
8267	0\\
8268	0\\
8269	0\\
8270	0\\
8271	0\\
8272	0\\
8273	0\\
8274	1.470633\\
8275	2.020456\\
8276	2.58423\\
8277	0\\
8278	0\\
8279	0\\
8280	0\\
8281	0\\
8282	0\\
8283	0\\
8284	0\\
8285	0\\
8286	0\\
8287	0\\
8288	0\\
8289	0\\
8290	0\\
8291	0\\
8292	0\\
8293	0\\
8294	0\\
8295	0\\
8296	0\\
8297	0\\
8298	2.58423\\
8299	2.413639\\
8300	0\\
8301	0\\
8302	0\\
8303	0\\
8304	0\\
8305	0\\
8306	0\\
8307	0\\
8308	0\\
8309	0\\
8310	0\\
8311	0\\
8312	0\\
8313	0\\
8314	0\\
8315	0\\
8316	0\\
8317	0\\
8318	0\\
8319	0\\
8320	0\\
8321	0\\
8322	0\\
8323	0\\
8324	0\\
8325	0\\
8326	0\\
8327	0\\
8328	0\\
8329	0\\
8330	0\\
8331	0\\
8332	0\\
8333	0\\
8334	0\\
8335	0\\
8336	0\\
8337	0\\
8338	0\\
8339	0\\
8340	0\\
8341	0\\
8342	0\\
8343	0\\
8344	0\\
8345	0\\
8346	0\\
8347	0\\
8348	0\\
8349	0\\
8350	0\\
8351	0\\
8352	0\\
8353	0\\
8354	0\\
8355	0\\
8356	0\\
8357	0\\
8358	0\\
8359	0\\
8360	0\\
8361	0\\
8362	0\\
8363	0\\
8364	0\\
8365	0\\
8366	0\\
8367	0\\
8368	0\\
8369	0\\
8370	0\\
8371	0\\
8372	0\\
8373	0\\
8374	0\\
8375	0\\
8376	0\\
8377	0\\
8378	0\\
8379	0\\
8380	0\\
8381	0\\
8382	0\\
8383	0\\
8384	0\\
8385	2.58423\\
8386	0\\
8387	0\\
8388	0\\
8389	0\\
8390	0\\
8391	0\\
8392	0\\
8393	0\\
8394	2.58423\\
8395	2.58423\\
8396	2.58423\\
8397	0\\
8398	0\\
8399	0\\
8400	0\\
8401	0\\
8402	0\\
8403	0\\
8404	0\\
8405	0\\
8406	0\\
8407	0\\
8408	0\\
8409	0\\
8410	0\\
8411	2.58423\\
8412	2.58423\\
8413	2.58423\\
8414	2.58423\\
8415	2.58423\\
8416	2.58423\\
8417	4.227991\\
8418	4.227991\\
8419	4.227991\\
8420	4.220381\\
8421	0\\
8422	0\\
8423	0\\
8424	0\\
8425	0\\
8426	0\\
8427	0\\
8428	0\\
8429	0\\
8430	0\\
8431	0\\
8432	0\\
8433	0\\
8434	0\\
8435	0\\
8436	0\\
8437	0\\
8438	0\\
8439	0\\
8440	0\\
8441	0\\
8442	0\\
8443	0\\
8444	0\\
8445	0\\
8446	0\\
8447	0\\
8448	0\\
8449	0\\
8450	0\\
8451	0\\
8452	0\\
8453	0\\
8454	0\\
8455	0\\
8456	0\\
8457	0\\
8458	0\\
8459	0\\
8460	0\\
8461	0\\
8462	0\\
8463	0\\
8464	0\\
8465	0\\
8466	0\\
8467	0\\
8468	0\\
8469	0\\
8470	0\\
8471	0\\
8472	0\\
8473	0\\
8474	0\\
8475	0\\
8476	0\\
8477	0\\
8478	0\\
8479	0\\
8480	0\\
8481	0\\
8482	0\\
8483	0\\
8484	0\\
8485	0\\
8486	0\\
8487	0\\
8488	0\\
8489	0\\
8490	0\\
8491	0\\
8492	0\\
8493	0\\
8494	0\\
8495	0\\
8496	0\\
8497	0\\
8498	0\\
8499	0\\
8500	0\\
8501	0\\
8502	0\\
8503	0\\
8504	0\\
8505	0\\
8506	0\\
8507	0\\
8508	0\\
8509	0\\
8510	0\\
8511	0\\
8512	0\\
8513	0\\
8514	0\\
8515	0\\
8516	0\\
8517	0\\
8518	0\\
8519	0\\
8520	0\\
8521	0\\
8522	0\\
8523	0\\
8524	0\\
8525	0\\
8526	0\\
8527	0\\
8528	0\\
8529	0\\
8530	0\\
8531	0\\
8532	0\\
8533	0\\
8534	0\\
8535	0\\
8536	0\\
8537	0\\
8538	0\\
8539	0\\
8540	0\\
8541	0\\
8542	0\\
8543	0\\
8544	0\\
8545	0\\
8546	0\\
8547	0\\
8548	0\\
8549	0\\
8550	0\\
8551	0\\
8552	0\\
8553	0\\
8554	0\\
8555	0\\
8556	0\\
8557	0\\
8558	0\\
8559	0\\
8560	0\\
8561	0\\
8562	0\\
8563	0\\
8564	0\\
8565	0\\
8566	0\\
8567	0\\
8568	0\\
8569	0\\
8570	0\\
8571	0\\
8572	0\\
8573	0\\
8574	0\\
8575	0\\
8576	0\\
8577	0\\
8578	0\\
8579	0\\
8580	0\\
8581	0\\
8582	0\\
8583	0\\
8584	0\\
8585	0\\
8586	0\\
8587	0\\
8588	0\\
8589	0\\
8590	0\\
8591	0\\
8592	0\\
8593	0\\
8594	0\\
8595	0\\
8596	0\\
8597	0\\
8598	0\\
8599	0\\
8600	0\\
8601	0\\
8602	0\\
8603	0\\
8604	0\\
8605	0\\
8606	0\\
8607	0\\
8608	0\\
8609	0\\
8610	0\\
8611	0\\
8612	0\\
8613	0\\
8614	0\\
8615	0\\
8616	0\\
8617	0\\
8618	0\\
8619	0\\
8620	0\\
8621	0\\
8622	0\\
8623	0\\
8624	0\\
8625	0\\
8626	0\\
8627	0\\
8628	0\\
8629	0\\
8630	0\\
8631	0\\
8632	0\\
8633	0\\
8634	0\\
8635	0\\
8636	0\\
8637	0\\
8638	0\\
8639	0\\
8640	0\\
8641	0\\
8642	0\\
8643	0\\
8644	0\\
8645	0\\
8646	0\\
8647	0\\
8648	0\\
8649	0\\
8650	0\\
8651	0\\
8652	0\\
8653	0\\
8654	0\\
8655	0\\
8656	0\\
8657	0\\
8658	0\\
8659	0\\
8660	0\\
8661	0\\
8662	0\\
8663	0\\
8664	0\\
8665	0\\
8666	0\\
8667	0\\
8668	0\\
8669	0\\
8670	0\\
8671	0\\
8672	0\\
8673	0\\
8674	0\\
8675	0\\
8676	0\\
8677	0\\
8678	0\\
8679	0\\
8680	0\\
8681	0\\
8682	0\\
8683	0\\
8684	0\\
8685	0\\
8686	0\\
8687	0\\
8688	0\\
8689	0\\
8690	0\\
8691	0\\
8692	0\\
8693	0\\
8694	0\\
8695	0\\
8696	0\\
8697	0\\
8698	0\\
8699	0\\
8700	0\\
8701	0\\
8702	0\\
8703	0\\
8704	0\\
8705	0\\
8706	0\\
8707	0\\
8708	0\\
8709	0\\
8710	0\\
8711	0\\
8712	0\\
8713	0\\
8714	0\\
8715	0\\
8716	0\\
8717	0\\
8718	0\\
8719	0\\
8720	0\\
8721	0\\
8722	0\\
8723	0\\
8724	0\\
8725	0\\
8726	0\\
8727	0\\
8728	0\\
8729	0\\
8730	0\\
8731	0\\
8732	0\\
8733	0\\
8734	0\\
8735	0\\
8736	0\\
8737	0\\
8738	0\\
8739	0\\
8740	0\\
8741	0\\
8742	0\\
8743	0\\
8744	0\\
8745	0\\
8746	0\\
8747	0\\
8748	0\\
8749	0\\
8750	0\\
8751	0\\
8752	0\\
8753	0\\
8754	0\\
8755	0\\
8756	0\\
8757	0\\
8758	0\\
8759	0\\
8760	0\\
};
\end{axis}
\end{tikzpicture}%
    \caption{Predicted reserve prices for the EDR model}
    \label{fig:EDR_R}
\end{figure}

\begin{figure}[H]
    \centering
    \setlength\fheight{0.3\textwidth}
    \setlength\fwidth{0.85\textwidth}
    % This file was created by matlab2tikz.
% Minimal pgfplots version: 1.3
%
%The latest updates can be retrieved from
%  http://www.mathworks.com/matlabcentral/fileexchange/22022-matlab2tikz
%where you can also make suggestions and rate matlab2tikz.
%
\definecolor{mycolor1}{rgb}{0.04314,0.51765,0.78039}%
%
\begin{tikzpicture}

\begin{axis}[%
width=\fwidth,
height=\fheight,
at={(0\fwidth,0\fheight)},
scale only axis,
separate axis lines,
every outer x axis line/.append style={black},
every x tick label/.append style={font=\color{black}},
xmin=0,
xmax=8760,
xlabel={time [hour]},
xtick={0,1000,2000,3000,4000,5000,6000,7000,8000},
xmajorgrids,
every outer y axis line/.append style={black},
every y tick label/.append style={font=\color{black}},
ymin=0,
ymax=14,
ymajorgrids,
title style={font=\bfseries},
title={ImpExp - Reserve price [\euro/MWh]}
]
\addplot [color=mycolor1,solid,line width=1.0pt,forget plot]
  table[row sep=crcr]{%
1	0\\
2	0\\
3	0\\
4	0\\
5	0\\
6	0\\
7	0\\
8	0\\
9	0\\
10	0\\
11	0\\
12	0\\
13	0\\
14	0\\
15	0\\
16	0\\
17	0\\
18	0\\
19	0\\
20	0\\
21	0\\
22	0\\
23	0\\
24	0\\
25	0\\
26	0\\
27	0\\
28	0\\
29	0\\
30	0\\
31	0\\
32	0\\
33	0\\
34	0\\
35	0\\
36	0\\
37	0\\
38	0\\
39	0\\
40	0\\
41	0\\
42	0\\
43	0\\
44	0\\
45	0\\
46	0\\
47	0\\
48	0\\
49	0\\
50	0\\
51	0\\
52	0\\
53	0\\
54	0\\
55	0\\
56	0\\
57	0\\
58	0\\
59	0\\
60	0\\
61	0\\
62	0\\
63	0\\
64	0\\
65	0\\
66	0\\
67	0\\
68	0\\
69	0\\
70	0\\
71	0\\
72	0\\
73	0\\
74	0\\
75	0\\
76	0\\
77	0\\
78	0\\
79	0\\
80	0\\
81	0\\
82	0\\
83	0\\
84	0\\
85	0\\
86	0\\
87	0\\
88	0\\
89	0\\
90	0\\
91	0\\
92	0\\
93	0\\
94	0\\
95	0\\
96	0\\
97	0\\
98	0\\
99	0\\
100	0\\
101	0\\
102	0\\
103	0\\
104	0\\
105	0\\
106	0\\
107	0\\
108	0\\
109	0\\
110	0\\
111	0\\
112	0\\
113	0\\
114	0\\
115	0\\
116	0\\
117	0\\
118	0\\
119	0\\
120	0\\
121	0\\
122	0\\
123	0\\
124	0\\
125	0\\
126	0\\
127	0\\
128	0\\
129	0\\
130	0\\
131	0\\
132	0\\
133	0\\
134	0\\
135	0\\
136	0\\
137	0\\
138	0\\
139	0\\
140	0\\
141	0\\
142	0\\
143	0\\
144	0\\
145	0\\
146	0\\
147	0\\
148	0\\
149	0\\
150	0\\
151	0\\
152	0\\
153	0\\
154	0\\
155	0\\
156	0\\
157	0\\
158	0\\
159	0\\
160	0\\
161	0\\
162	0\\
163	0\\
164	0\\
165	0\\
166	0\\
167	0\\
168	0\\
169	0\\
170	0\\
171	0\\
172	0\\
173	0\\
174	0\\
175	0\\
176	0\\
177	0\\
178	0\\
179	0\\
180	0\\
181	0\\
182	0\\
183	0\\
184	0\\
185	0\\
186	0\\
187	0\\
188	0\\
189	0\\
190	0\\
191	0\\
192	0\\
193	0\\
194	0\\
195	0\\
196	0\\
197	0\\
198	0\\
199	0\\
200	0\\
201	0\\
202	0\\
203	0\\
204	0\\
205	0\\
206	0\\
207	0\\
208	0\\
209	0\\
210	0\\
211	0\\
212	0\\
213	0\\
214	0\\
215	0\\
216	0\\
217	0\\
218	0\\
219	0\\
220	0\\
221	0\\
222	0\\
223	0\\
224	0\\
225	0\\
226	0\\
227	0\\
228	0\\
229	0\\
230	0\\
231	0\\
232	0\\
233	0\\
234	0\\
235	0\\
236	0\\
237	0\\
238	0\\
239	0\\
240	0\\
241	0\\
242	0\\
243	0\\
244	0\\
245	0\\
246	0\\
247	0\\
248	0\\
249	0\\
250	0\\
251	0\\
252	0\\
253	0\\
254	0\\
255	0\\
256	0\\
257	0\\
258	0\\
259	0\\
260	0\\
261	0\\
262	0\\
263	0\\
264	0\\
265	0\\
266	0\\
267	0\\
268	0\\
269	0\\
270	0\\
271	0\\
272	0\\
273	0\\
274	0\\
275	0\\
276	0\\
277	0\\
278	0\\
279	0\\
280	0\\
281	0\\
282	0\\
283	0\\
284	0\\
285	0\\
286	0\\
287	0\\
288	0\\
289	0\\
290	0\\
291	0\\
292	0\\
293	0\\
294	0\\
295	0\\
296	0\\
297	0\\
298	0\\
299	0\\
300	0\\
301	0\\
302	0\\
303	0\\
304	0\\
305	0\\
306	0\\
307	0\\
308	0\\
309	0\\
310	0\\
311	0\\
312	0\\
313	0\\
314	0\\
315	0\\
316	0\\
317	0\\
318	0\\
319	0\\
320	0\\
321	0\\
322	0\\
323	0\\
324	0\\
325	0\\
326	0\\
327	0\\
328	0\\
329	0\\
330	0\\
331	0\\
332	0\\
333	0\\
334	0\\
335	0\\
336	0\\
337	0\\
338	0\\
339	0\\
340	0\\
341	0\\
342	0\\
343	0\\
344	0\\
345	0\\
346	0\\
347	0\\
348	0\\
349	0\\
350	0\\
351	0\\
352	0\\
353	0\\
354	0\\
355	0\\
356	0\\
357	0\\
358	0\\
359	0\\
360	0\\
361	0\\
362	0\\
363	0\\
364	0\\
365	0\\
366	0\\
367	0\\
368	0\\
369	0\\
370	0\\
371	0\\
372	0\\
373	0\\
374	0\\
375	0\\
376	0\\
377	0\\
378	0\\
379	0\\
380	0\\
381	0\\
382	0\\
383	0\\
384	0\\
385	0\\
386	0\\
387	0\\
388	0\\
389	0\\
390	0\\
391	0\\
392	0\\
393	0\\
394	0\\
395	0\\
396	0\\
397	0\\
398	0\\
399	0\\
400	0\\
401	0\\
402	0\\
403	0\\
404	0\\
405	0\\
406	0\\
407	0\\
408	0\\
409	0\\
410	0\\
411	0\\
412	0\\
413	0\\
414	0\\
415	0\\
416	0\\
417	0\\
418	0\\
419	0\\
420	0\\
421	0\\
422	0\\
423	0\\
424	0\\
425	0\\
426	0\\
427	0\\
428	0\\
429	0\\
430	0\\
431	0\\
432	0\\
433	0\\
434	0\\
435	0\\
436	0\\
437	0\\
438	0\\
439	0\\
440	0\\
441	0\\
442	0\\
443	0\\
444	0\\
445	0\\
446	0\\
447	0\\
448	0\\
449	0\\
450	0\\
451	0\\
452	0\\
453	0\\
454	0\\
455	0\\
456	0\\
457	0\\
458	0\\
459	0\\
460	0\\
461	0\\
462	0\\
463	0\\
464	0\\
465	0\\
466	0\\
467	0\\
468	0\\
469	0\\
470	0\\
471	0\\
472	0\\
473	0\\
474	0\\
475	0\\
476	0\\
477	0\\
478	0\\
479	0\\
480	0\\
481	0\\
482	0\\
483	0\\
484	0\\
485	0\\
486	0\\
487	0\\
488	0\\
489	0\\
490	0\\
491	0\\
492	0\\
493	0\\
494	0\\
495	0\\
496	0\\
497	0\\
498	0\\
499	0\\
500	0\\
501	0\\
502	0\\
503	0\\
504	0\\
505	0\\
506	0\\
507	0\\
508	0\\
509	0\\
510	0\\
511	0\\
512	0\\
513	0\\
514	0\\
515	0\\
516	0\\
517	0\\
518	0\\
519	0\\
520	0\\
521	0\\
522	0\\
523	0\\
524	0\\
525	0\\
526	0\\
527	0\\
528	0\\
529	0\\
530	0\\
531	0\\
532	0\\
533	0\\
534	0\\
535	0\\
536	0\\
537	0\\
538	0\\
539	0\\
540	0\\
541	0\\
542	0\\
543	0\\
544	0\\
545	0\\
546	0\\
547	0\\
548	0\\
549	0\\
550	0\\
551	0\\
552	0\\
553	0\\
554	0\\
555	0\\
556	0\\
557	0\\
558	0\\
559	0\\
560	0\\
561	0\\
562	0\\
563	0\\
564	0\\
565	0\\
566	0\\
567	0\\
568	0\\
569	0\\
570	0\\
571	0\\
572	0\\
573	0\\
574	0\\
575	0\\
576	0\\
577	0\\
578	0\\
579	0\\
580	0\\
581	0\\
582	0\\
583	0\\
584	0\\
585	0\\
586	0\\
587	0\\
588	0\\
589	0\\
590	0\\
591	0\\
592	0\\
593	0\\
594	0\\
595	0\\
596	0\\
597	0\\
598	0\\
599	0\\
600	0\\
601	0\\
602	0\\
603	0\\
604	0\\
605	0\\
606	0\\
607	0\\
608	0\\
609	0\\
610	0\\
611	0\\
612	0\\
613	0\\
614	0\\
615	0\\
616	0\\
617	0\\
618	0\\
619	0\\
620	0\\
621	0\\
622	0\\
623	0\\
624	0\\
625	0\\
626	0\\
627	0\\
628	0\\
629	0\\
630	0\\
631	0\\
632	0\\
633	0\\
634	0\\
635	0\\
636	0\\
637	0\\
638	0\\
639	0\\
640	0\\
641	0\\
642	0\\
643	0\\
644	0\\
645	0\\
646	0\\
647	0\\
648	0\\
649	0\\
650	0\\
651	0\\
652	0\\
653	0\\
654	0\\
655	0\\
656	0\\
657	0\\
658	0\\
659	0\\
660	0\\
661	0\\
662	0\\
663	0\\
664	0\\
665	0\\
666	0\\
667	0\\
668	0\\
669	0\\
670	0\\
671	0\\
672	0\\
673	0\\
674	0\\
675	0\\
676	0\\
677	0\\
678	0\\
679	0\\
680	0\\
681	0\\
682	0\\
683	0\\
684	0\\
685	0\\
686	0\\
687	0\\
688	0\\
689	0\\
690	0\\
691	0\\
692	0\\
693	0\\
694	0\\
695	0\\
696	0\\
697	0\\
698	0\\
699	0\\
700	0\\
701	0\\
702	0\\
703	0\\
704	0\\
705	0\\
706	0\\
707	0\\
708	0\\
709	0\\
710	0\\
711	0\\
712	0\\
713	0\\
714	0\\
715	0\\
716	0\\
717	0\\
718	0\\
719	0\\
720	0\\
721	0\\
722	0\\
723	0\\
724	0\\
725	0\\
726	0\\
727	0\\
728	0\\
729	0\\
730	0\\
731	0\\
732	0\\
733	0\\
734	0\\
735	0\\
736	0\\
737	0\\
738	0\\
739	0\\
740	0\\
741	0\\
742	0\\
743	0\\
744	0\\
745	0\\
746	0\\
747	0\\
748	0\\
749	0\\
750	0\\
751	0\\
752	0\\
753	0\\
754	0\\
755	0\\
756	0\\
757	0\\
758	0\\
759	0\\
760	0\\
761	0\\
762	0\\
763	0\\
764	0\\
765	0\\
766	0\\
767	0\\
768	0\\
769	0\\
770	0\\
771	0\\
772	0\\
773	0\\
774	0\\
775	0\\
776	0\\
777	0\\
778	0\\
779	0\\
780	0\\
781	0\\
782	0\\
783	0\\
784	0\\
785	0\\
786	0\\
787	0\\
788	0\\
789	0\\
790	0\\
791	0\\
792	0\\
793	0\\
794	0\\
795	0\\
796	0\\
797	0\\
798	0\\
799	0\\
800	0\\
801	0\\
802	0\\
803	0\\
804	0\\
805	0\\
806	0\\
807	0\\
808	0\\
809	0\\
810	0\\
811	0\\
812	0\\
813	0\\
814	0\\
815	0\\
816	0\\
817	0\\
818	0\\
819	0\\
820	0\\
821	0\\
822	0\\
823	0\\
824	0\\
825	0\\
826	0\\
827	0\\
828	0\\
829	0\\
830	0\\
831	0\\
832	0\\
833	0\\
834	0\\
835	0\\
836	0\\
837	0\\
838	0\\
839	0\\
840	0\\
841	0\\
842	0\\
843	0\\
844	0\\
845	0\\
846	0\\
847	0\\
848	0\\
849	0\\
850	0\\
851	0\\
852	0\\
853	0\\
854	0\\
855	0\\
856	0\\
857	0\\
858	0\\
859	0\\
860	0\\
861	0\\
862	0\\
863	0\\
864	0\\
865	0\\
866	0\\
867	0\\
868	0\\
869	0\\
870	0\\
871	0\\
872	0\\
873	0\\
874	0\\
875	0\\
876	0\\
877	0\\
878	0\\
879	0\\
880	0\\
881	0\\
882	0\\
883	0\\
884	0\\
885	0\\
886	0\\
887	0\\
888	0\\
889	0\\
890	0\\
891	0\\
892	0\\
893	0\\
894	0\\
895	0\\
896	0\\
897	0\\
898	0\\
899	0\\
900	0\\
901	0\\
902	0\\
903	0\\
904	0\\
905	0\\
906	0\\
907	0\\
908	0\\
909	0\\
910	0\\
911	0\\
912	0\\
913	0\\
914	0\\
915	0\\
916	0\\
917	0\\
918	0\\
919	0\\
920	0\\
921	0\\
922	0\\
923	0\\
924	0\\
925	0\\
926	0\\
927	0\\
928	0\\
929	0\\
930	0\\
931	0\\
932	0\\
933	0\\
934	0\\
935	0\\
936	0\\
937	0\\
938	0\\
939	0\\
940	0\\
941	0\\
942	0\\
943	0\\
944	0\\
945	0\\
946	0\\
947	0\\
948	0\\
949	0\\
950	0\\
951	0\\
952	0\\
953	0\\
954	0\\
955	0\\
956	0\\
957	0\\
958	0\\
959	0\\
960	0\\
961	0\\
962	0\\
963	0\\
964	0\\
965	0\\
966	0\\
967	0\\
968	0\\
969	0\\
970	0\\
971	0\\
972	0\\
973	0\\
974	0\\
975	0\\
976	0\\
977	0\\
978	0\\
979	0\\
980	0\\
981	0\\
982	0\\
983	0\\
984	0\\
985	0\\
986	0\\
987	0\\
988	0\\
989	0\\
990	0\\
991	0\\
992	0\\
993	0\\
994	0\\
995	0\\
996	0\\
997	0\\
998	0\\
999	0\\
1000	0\\
1001	0\\
1002	0\\
1003	0\\
1004	0\\
1005	0\\
1006	0\\
1007	0\\
1008	0\\
1009	0\\
1010	0\\
1011	0\\
1012	0\\
1013	0\\
1014	0\\
1015	0\\
1016	0\\
1017	0\\
1018	0\\
1019	0\\
1020	0\\
1021	0\\
1022	0\\
1023	0\\
1024	0\\
1025	0\\
1026	0\\
1027	0\\
1028	0\\
1029	0\\
1030	0\\
1031	0\\
1032	0\\
1033	0\\
1034	0\\
1035	0\\
1036	0\\
1037	0\\
1038	0\\
1039	0\\
1040	0\\
1041	0\\
1042	0\\
1043	0\\
1044	0\\
1045	0\\
1046	0\\
1047	0\\
1048	0\\
1049	0\\
1050	0\\
1051	0\\
1052	0\\
1053	0\\
1054	0\\
1055	0\\
1056	0\\
1057	0\\
1058	0\\
1059	0\\
1060	0\\
1061	0\\
1062	0\\
1063	0\\
1064	0\\
1065	0\\
1066	0\\
1067	0\\
1068	0\\
1069	0\\
1070	0\\
1071	0\\
1072	0\\
1073	0\\
1074	0\\
1075	0\\
1076	0\\
1077	0\\
1078	0\\
1079	0\\
1080	0\\
1081	0\\
1082	0\\
1083	0\\
1084	0\\
1085	0\\
1086	0\\
1087	0\\
1088	0\\
1089	0\\
1090	0\\
1091	0\\
1092	0\\
1093	0\\
1094	0\\
1095	0\\
1096	0\\
1097	0\\
1098	0\\
1099	0\\
1100	0\\
1101	0\\
1102	0\\
1103	0\\
1104	0\\
1105	0\\
1106	0\\
1107	0\\
1108	0\\
1109	0\\
1110	0\\
1111	0\\
1112	0\\
1113	0\\
1114	0\\
1115	0\\
1116	0\\
1117	0\\
1118	0\\
1119	0\\
1120	0\\
1121	0\\
1122	0\\
1123	0\\
1124	0\\
1125	0\\
1126	0\\
1127	0\\
1128	0\\
1129	0\\
1130	0\\
1131	0\\
1132	0\\
1133	0\\
1134	0\\
1135	0\\
1136	0\\
1137	0\\
1138	0\\
1139	0\\
1140	0\\
1141	0\\
1142	0\\
1143	0\\
1144	0\\
1145	0\\
1146	0\\
1147	0\\
1148	0\\
1149	0\\
1150	0\\
1151	0\\
1152	0\\
1153	0\\
1154	0\\
1155	0\\
1156	0\\
1157	0\\
1158	0\\
1159	0\\
1160	0\\
1161	0\\
1162	0\\
1163	0\\
1164	0\\
1165	0\\
1166	0\\
1167	0\\
1168	0\\
1169	0\\
1170	0\\
1171	0\\
1172	0\\
1173	0\\
1174	0\\
1175	0\\
1176	0\\
1177	0\\
1178	0\\
1179	0\\
1180	0\\
1181	0\\
1182	0\\
1183	0\\
1184	0\\
1185	0\\
1186	0\\
1187	0\\
1188	0\\
1189	0\\
1190	0\\
1191	0\\
1192	0\\
1193	0\\
1194	0\\
1195	0\\
1196	0\\
1197	0\\
1198	0\\
1199	0\\
1200	0\\
1201	0\\
1202	0\\
1203	0\\
1204	0\\
1205	0\\
1206	0\\
1207	0\\
1208	0\\
1209	0\\
1210	0\\
1211	0\\
1212	0\\
1213	0\\
1214	0\\
1215	0\\
1216	0\\
1217	0\\
1218	0\\
1219	0\\
1220	0\\
1221	0\\
1222	0\\
1223	0\\
1224	0\\
1225	0\\
1226	0\\
1227	0\\
1228	0\\
1229	0\\
1230	0\\
1231	0\\
1232	0\\
1233	0\\
1234	0\\
1235	0\\
1236	0\\
1237	0\\
1238	0\\
1239	0\\
1240	0\\
1241	0\\
1242	0\\
1243	0\\
1244	0\\
1245	0\\
1246	0\\
1247	0\\
1248	0\\
1249	0\\
1250	0\\
1251	0\\
1252	0\\
1253	0\\
1254	0\\
1255	0\\
1256	0\\
1257	0\\
1258	0\\
1259	0\\
1260	0\\
1261	0\\
1262	0\\
1263	0\\
1264	0\\
1265	0\\
1266	0\\
1267	0\\
1268	0\\
1269	0\\
1270	0\\
1271	0\\
1272	0\\
1273	0\\
1274	0\\
1275	0\\
1276	0\\
1277	0\\
1278	0\\
1279	0\\
1280	0\\
1281	0\\
1282	0\\
1283	0\\
1284	0\\
1285	0\\
1286	0\\
1287	0\\
1288	0\\
1289	0\\
1290	0\\
1291	0\\
1292	0\\
1293	0\\
1294	0\\
1295	0\\
1296	0\\
1297	0\\
1298	0\\
1299	0\\
1300	0\\
1301	0\\
1302	0\\
1303	0\\
1304	0\\
1305	0\\
1306	0\\
1307	0\\
1308	0\\
1309	0\\
1310	0\\
1311	0\\
1312	0\\
1313	0\\
1314	0\\
1315	0\\
1316	0\\
1317	0\\
1318	0\\
1319	0\\
1320	0\\
1321	0\\
1322	0\\
1323	0\\
1324	0\\
1325	0\\
1326	0\\
1327	0\\
1328	0\\
1329	0\\
1330	0\\
1331	0\\
1332	0\\
1333	0\\
1334	0\\
1335	0\\
1336	0\\
1337	0\\
1338	0\\
1339	0\\
1340	0\\
1341	0\\
1342	0\\
1343	0\\
1344	0\\
1345	0\\
1346	0\\
1347	0\\
1348	0\\
1349	0\\
1350	0\\
1351	0\\
1352	0\\
1353	0\\
1354	0\\
1355	0\\
1356	0\\
1357	0\\
1358	0\\
1359	0\\
1360	0\\
1361	0\\
1362	0\\
1363	0\\
1364	0\\
1365	0\\
1366	0\\
1367	0\\
1368	0\\
1369	0\\
1370	0\\
1371	0\\
1372	0\\
1373	0\\
1374	0\\
1375	0\\
1376	0\\
1377	0\\
1378	0\\
1379	0\\
1380	0\\
1381	0\\
1382	0\\
1383	0\\
1384	0\\
1385	0\\
1386	0\\
1387	0\\
1388	0\\
1389	0\\
1390	0\\
1391	0\\
1392	0\\
1393	0\\
1394	0\\
1395	0\\
1396	0\\
1397	0\\
1398	0\\
1399	0\\
1400	0\\
1401	0\\
1402	0\\
1403	0\\
1404	0\\
1405	0\\
1406	0\\
1407	0\\
1408	0\\
1409	0\\
1410	0\\
1411	0\\
1412	0\\
1413	0\\
1414	0\\
1415	0\\
1416	0\\
1417	0\\
1418	0\\
1419	0\\
1420	0\\
1421	0\\
1422	0\\
1423	0\\
1424	0\\
1425	0\\
1426	0\\
1427	0\\
1428	0\\
1429	0\\
1430	0\\
1431	0\\
1432	0\\
1433	0\\
1434	0\\
1435	0\\
1436	0\\
1437	0\\
1438	0\\
1439	0\\
1440	0\\
1441	0\\
1442	0\\
1443	0\\
1444	0\\
1445	0\\
1446	0\\
1447	0\\
1448	0\\
1449	0\\
1450	0\\
1451	0\\
1452	0\\
1453	0\\
1454	0\\
1455	0\\
1456	0\\
1457	0\\
1458	0\\
1459	0\\
1460	0\\
1461	0\\
1462	0\\
1463	0\\
1464	0\\
1465	0\\
1466	0\\
1467	0\\
1468	0\\
1469	0\\
1470	0\\
1471	0\\
1472	0\\
1473	0\\
1474	0\\
1475	0\\
1476	0\\
1477	0\\
1478	0\\
1479	0\\
1480	0\\
1481	0\\
1482	0\\
1483	0\\
1484	0\\
1485	0\\
1486	0\\
1487	0\\
1488	0\\
1489	0\\
1490	0\\
1491	0\\
1492	0\\
1493	0\\
1494	0\\
1495	0\\
1496	0\\
1497	0\\
1498	0\\
1499	0\\
1500	0\\
1501	0\\
1502	0\\
1503	0\\
1504	0\\
1505	0\\
1506	0\\
1507	0\\
1508	0\\
1509	0\\
1510	0\\
1511	0\\
1512	0\\
1513	0\\
1514	0\\
1515	0\\
1516	0\\
1517	0\\
1518	0\\
1519	0\\
1520	0\\
1521	0\\
1522	0\\
1523	0\\
1524	0\\
1525	0\\
1526	0\\
1527	0\\
1528	0\\
1529	0\\
1530	0\\
1531	0\\
1532	0\\
1533	0\\
1534	0\\
1535	0\\
1536	0\\
1537	0\\
1538	0\\
1539	0\\
1540	0\\
1541	0\\
1542	0\\
1543	0\\
1544	0\\
1545	0\\
1546	0\\
1547	0\\
1548	0\\
1549	0\\
1550	0\\
1551	0\\
1552	0\\
1553	0\\
1554	0\\
1555	0\\
1556	0\\
1557	0\\
1558	0\\
1559	0\\
1560	0\\
1561	0\\
1562	0\\
1563	0\\
1564	0\\
1565	0\\
1566	0\\
1567	0\\
1568	0\\
1569	0\\
1570	0\\
1571	0\\
1572	0\\
1573	0\\
1574	0\\
1575	0\\
1576	0\\
1577	0\\
1578	0\\
1579	0\\
1580	0\\
1581	0\\
1582	0\\
1583	0\\
1584	0\\
1585	0\\
1586	0\\
1587	0\\
1588	0\\
1589	0\\
1590	0\\
1591	0\\
1592	0\\
1593	0\\
1594	0\\
1595	0\\
1596	0\\
1597	0\\
1598	0\\
1599	0\\
1600	0\\
1601	0\\
1602	0\\
1603	0\\
1604	0\\
1605	0\\
1606	0\\
1607	0\\
1608	0\\
1609	0\\
1610	0\\
1611	0\\
1612	0\\
1613	0\\
1614	0\\
1615	0\\
1616	0\\
1617	0\\
1618	0\\
1619	0\\
1620	0\\
1621	0\\
1622	0\\
1623	0\\
1624	0\\
1625	0\\
1626	0\\
1627	0\\
1628	0\\
1629	0\\
1630	0\\
1631	0\\
1632	0\\
1633	0\\
1634	0\\
1635	0\\
1636	0\\
1637	0\\
1638	0\\
1639	0\\
1640	0\\
1641	0\\
1642	0\\
1643	0\\
1644	0\\
1645	0\\
1646	0\\
1647	0\\
1648	0\\
1649	0\\
1650	0\\
1651	0\\
1652	0\\
1653	0\\
1654	0\\
1655	0\\
1656	0\\
1657	0\\
1658	0\\
1659	0\\
1660	0\\
1661	0\\
1662	0\\
1663	0\\
1664	0\\
1665	0\\
1666	0\\
1667	0\\
1668	0\\
1669	0\\
1670	0\\
1671	0\\
1672	0\\
1673	0\\
1674	0\\
1675	0\\
1676	0\\
1677	0\\
1678	0\\
1679	0\\
1680	0\\
1681	0\\
1682	0\\
1683	0\\
1684	0\\
1685	0\\
1686	0\\
1687	0\\
1688	0\\
1689	0\\
1690	0\\
1691	0\\
1692	0\\
1693	0\\
1694	0\\
1695	0\\
1696	0\\
1697	0\\
1698	0\\
1699	0\\
1700	0\\
1701	0\\
1702	0\\
1703	0\\
1704	0\\
1705	0\\
1706	0\\
1707	0\\
1708	0\\
1709	0\\
1710	0\\
1711	0\\
1712	0\\
1713	0\\
1714	0\\
1715	0\\
1716	0\\
1717	0\\
1718	0\\
1719	0\\
1720	0\\
1721	0\\
1722	0\\
1723	0\\
1724	0\\
1725	0\\
1726	0\\
1727	0\\
1728	0\\
1729	0\\
1730	0\\
1731	0\\
1732	0\\
1733	0\\
1734	0\\
1735	0\\
1736	0\\
1737	0\\
1738	0\\
1739	0\\
1740	0\\
1741	0\\
1742	0\\
1743	0\\
1744	0\\
1745	0\\
1746	0\\
1747	0\\
1748	0\\
1749	0\\
1750	0\\
1751	0\\
1752	0\\
1753	0\\
1754	0\\
1755	0\\
1756	0\\
1757	0\\
1758	0\\
1759	0\\
1760	0\\
1761	0\\
1762	0\\
1763	0\\
1764	0\\
1765	0\\
1766	0\\
1767	0\\
1768	0\\
1769	0\\
1770	0\\
1771	0\\
1772	0\\
1773	0\\
1774	0\\
1775	0\\
1776	0\\
1777	0\\
1778	0\\
1779	0\\
1780	0\\
1781	0\\
1782	0\\
1783	0\\
1784	0\\
1785	0\\
1786	0\\
1787	0\\
1788	0\\
1789	0\\
1790	0\\
1791	0\\
1792	0\\
1793	0\\
1794	0\\
1795	0\\
1796	0\\
1797	0\\
1798	0\\
1799	0\\
1800	0\\
1801	0\\
1802	0\\
1803	0\\
1804	0\\
1805	0\\
1806	0\\
1807	0\\
1808	0\\
1809	0\\
1810	0\\
1811	0\\
1812	0\\
1813	0\\
1814	0\\
1815	0\\
1816	0\\
1817	0\\
1818	0\\
1819	0\\
1820	0\\
1821	0\\
1822	0\\
1823	0\\
1824	0\\
1825	0\\
1826	0\\
1827	0\\
1828	0\\
1829	0\\
1830	0\\
1831	0\\
1832	0\\
1833	0\\
1834	0\\
1835	0\\
1836	0\\
1837	0\\
1838	0\\
1839	0\\
1840	0\\
1841	0\\
1842	0\\
1843	0\\
1844	0\\
1845	0\\
1846	0\\
1847	0\\
1848	0\\
1849	0\\
1850	0\\
1851	0\\
1852	0\\
1853	0\\
1854	0\\
1855	0\\
1856	0\\
1857	0\\
1858	0\\
1859	0\\
1860	0\\
1861	0\\
1862	0\\
1863	0\\
1864	0\\
1865	0\\
1866	0\\
1867	0\\
1868	0\\
1869	0\\
1870	0\\
1871	0\\
1872	0\\
1873	0\\
1874	0\\
1875	0\\
1876	0\\
1877	0\\
1878	0\\
1879	0\\
1880	0\\
1881	0\\
1882	0\\
1883	0\\
1884	0\\
1885	0\\
1886	0\\
1887	0\\
1888	0\\
1889	0\\
1890	0\\
1891	0\\
1892	0\\
1893	0\\
1894	0\\
1895	0\\
1896	0\\
1897	0\\
1898	0\\
1899	0\\
1900	0\\
1901	0\\
1902	0\\
1903	0\\
1904	0\\
1905	0\\
1906	0\\
1907	0\\
1908	0\\
1909	0\\
1910	0\\
1911	0\\
1912	0\\
1913	0\\
1914	0\\
1915	0\\
1916	0\\
1917	0\\
1918	0\\
1919	0\\
1920	0\\
1921	0\\
1922	0\\
1923	0\\
1924	0\\
1925	0\\
1926	0\\
1927	0\\
1928	0\\
1929	0\\
1930	0\\
1931	0\\
1932	0\\
1933	0\\
1934	0\\
1935	0\\
1936	0\\
1937	0\\
1938	0\\
1939	0\\
1940	0\\
1941	0\\
1942	0\\
1943	0\\
1944	0\\
1945	0\\
1946	0\\
1947	0\\
1948	0\\
1949	0\\
1950	0\\
1951	0\\
1952	0\\
1953	0\\
1954	0\\
1955	0\\
1956	0\\
1957	0\\
1958	0\\
1959	0\\
1960	0\\
1961	0\\
1962	0\\
1963	0\\
1964	0\\
1965	0\\
1966	0\\
1967	0\\
1968	0\\
1969	0\\
1970	0\\
1971	0\\
1972	0\\
1973	0\\
1974	0\\
1975	0\\
1976	0\\
1977	0\\
1978	0\\
1979	0\\
1980	0\\
1981	0\\
1982	0\\
1983	0\\
1984	0\\
1985	0\\
1986	0\\
1987	0\\
1988	0\\
1989	0\\
1990	0\\
1991	0\\
1992	0\\
1993	0\\
1994	0\\
1995	0\\
1996	0\\
1997	0\\
1998	0\\
1999	0\\
2000	0\\
2001	0\\
2002	0\\
2003	0\\
2004	0\\
2005	0\\
2006	0\\
2007	0\\
2008	0\\
2009	0\\
2010	0\\
2011	0\\
2012	0\\
2013	0\\
2014	0\\
2015	0\\
2016	0\\
2017	0\\
2018	0\\
2019	0\\
2020	0\\
2021	0\\
2022	0\\
2023	0\\
2024	0\\
2025	0\\
2026	0\\
2027	0\\
2028	0\\
2029	0\\
2030	0\\
2031	0\\
2032	0\\
2033	0\\
2034	0\\
2035	0\\
2036	1.469982\\
2037	1.019151\\
2038	0.460767\\
2039	0.460837\\
2040	0.192769\\
2041	0\\
2042	0\\
2043	0\\
2044	0\\
2045	0\\
2046	0\\
2047	0\\
2048	0.297745\\
2049	0.072028\\
2050	0\\
2051	2e-05\\
2052	0\\
2053	0\\
2054	0\\
2055	0\\
2056	0\\
2057	5e-06\\
2058	1.620671\\
2059	2.583097\\
2060	2.583097\\
2061	1.622425\\
2062	0.361106\\
2063	0.7112\\
2064	4e-06\\
2065	0\\
2066	0\\
2067	0\\
2068	0\\
2069	0\\
2070	0\\
2071	1e-06\\
2072	0\\
2073	0\\
2074	0\\
2075	0\\
2076	0\\
2077	0\\
2078	0\\
2079	0\\
2080	0\\
2081	0\\
2082	0\\
2083	0.399228\\
2084	0\\
2085	0\\
2086	0\\
2087	0\\
2088	0\\
2089	0\\
2090	0\\
2091	0\\
2092	0\\
2093	0\\
2094	0\\
2095	0\\
2096	0\\
2097	0\\
2098	0\\
2099	0\\
2100	0\\
2101	0\\
2102	0\\
2103	0\\
2104	0\\
2105	0\\
2106	0\\
2107	0\\
2108	0\\
2109	0\\
2110	0\\
2111	0\\
2112	0\\
2113	0\\
2114	0\\
2115	0\\
2116	0\\
2117	0\\
2118	0\\
2119	0\\
2120	0\\
2121	0\\
2122	0\\
2123	0\\
2124	0\\
2125	0\\
2126	0\\
2127	0\\
2128	0\\
2129	0\\
2130	0\\
2131	0\\
2132	0\\
2133	0\\
2134	0\\
2135	0\\
2136	0\\
2137	0\\
2138	0\\
2139	0\\
2140	0\\
2141	0\\
2142	0\\
2143	0.314531\\
2144	0\\
2145	0\\
2146	0\\
2147	0\\
2148	0\\
2149	0\\
2150	0\\
2151	0\\
2152	0\\
2153	0\\
2154	0\\
2155	0\\
2156	0.905372\\
2157	0\\
2158	0\\
2159	0\\
2160	0\\
2161	0\\
2162	0\\
2163	0\\
2164	0\\
2165	0\\
2166	2.277406\\
2167	2.314827\\
2168	2.314827\\
2169	2.133942\\
2170	1.985004\\
2171	1.830616\\
2172	0.794485\\
2173	0.812134\\
2174	0.7955\\
2175	1.087827\\
2176	1.499705\\
2177	2.314827\\
2178	2.478995\\
2179	2.314827\\
2180	2.314827\\
2181	2.314827\\
2182	2.314827\\
2183	2.298741\\
2184	0\\
2185	0\\
2186	0\\
2187	0\\
2188	0\\
2189	0\\
2190	1.317321\\
2191	2.314827\\
2192	2.314827\\
2193	2.249228\\
2194	2.279606\\
2195	1.720862\\
2196	1.317321\\
2197	1.316786\\
2198	0.954991\\
2199	0.447443\\
2200	0.912811\\
2201	1.489009\\
2202	2.314827\\
2203	2.314827\\
2204	2.314827\\
2205	2.314827\\
2206	2.314827\\
2207	1.087827\\
2208	0\\
2209	0\\
2210	0\\
2211	0\\
2212	0\\
2213	0\\
2214	0.00972\\
2215	2.07283\\
2216	1.677531\\
2217	1.317321\\
2218	1.168145\\
2219	1.317321\\
2220	0.412976\\
2221	0.412977\\
2222	0.412976\\
2223	0.238992\\
2224	1.087826\\
2225	2.314826\\
2226	2.34395\\
2227	2.813579\\
2228	2.54323\\
2229	2.314827\\
2230	2.314827\\
2231	1.500393\\
2232	0\\
2233	0\\
2234	0\\
2235	0\\
2236	0\\
2237	0\\
2238	1.317321\\
2239	2.316032\\
2240	2.813579\\
2241	2.8969\\
2242	2.980422\\
2243	3.596576\\
2244	2.314827\\
2245	2.314827\\
2246	2.095076\\
2247	1.317321\\
2248	1.393765\\
2249	2.314827\\
2250	2.314827\\
2251	2.082137\\
2252	2.314827\\
2253	1.430236\\
2254	2.314827\\
2255	1.657748\\
2256	0\\
2257	0\\
2258	1e-06\\
2259	0\\
2260	0\\
2261	0\\
2262	0\\
2263	0\\
2264	0\\
2265	0\\
2266	0\\
2267	0\\
2268	0\\
2269	0\\
2270	0\\
2271	0\\
2272	0\\
2273	0\\
2274	0\\
2275	0\\
2276	0\\
2277	0\\
2278	0\\
2279	0\\
2280	0\\
2281	0\\
2282	0\\
2283	0\\
2284	0\\
2285	0\\
2286	0\\
2287	0\\
2288	0\\
2289	0\\
2290	0\\
2291	0\\
2292	0\\
2293	0\\
2294	0\\
2295	0\\
2296	0\\
2297	0\\
2298	0\\
2299	0\\
2300	0\\
2301	0\\
2302	0\\
2303	0\\
2304	0\\
2305	0\\
2306	0\\
2307	0\\
2308	0\\
2309	0\\
2310	0\\
2311	0\\
2312	1.317321\\
2313	1.734169\\
2314	2.162625\\
2315	2.01612\\
2316	0.823287\\
2317	0.975175\\
2318	0.510521\\
2319	0.725372\\
2320	0.594864\\
2321	1.317318\\
2322	0.626717\\
2323	0.423494\\
2324	0.985288\\
2325	1.51255\\
2326	1.317321\\
2327	4e-06\\
2328	0\\
2329	0\\
2330	0\\
2331	0\\
2332	0\\
2333	0\\
2334	0\\
2335	0.396392\\
2336	1.317321\\
2337	1.087827\\
2338	0.699051\\
2339	1.115002\\
2340	1e-06\\
2341	0\\
2342	0\\
2343	0\\
2344	7.6e-05\\
2345	1.31732\\
2346	0.100865\\
2347	1.00517\\
2348	2.314827\\
2349	2.314827\\
2350	2.314827\\
2351	2.314618\\
2352	0\\
2353	0\\
2354	0\\
2355	1e-06\\
2356	0\\
2357	0\\
2358	2.314827\\
2359	2.314827\\
2360	2.315994\\
2361	2.349212\\
2362	2.314827\\
2363	2.314827\\
2364	0.552453\\
2365	0.345095\\
2366	0\\
2367	0\\
2368	0.135736\\
2369	1.322791\\
2370	2.314827\\
2371	2.314827\\
2372	2.16202\\
2373	2.314827\\
2374	2.314827\\
2375	2.314827\\
2376	0.13483\\
2377	0\\
2378	0\\
2379	0\\
2380	0\\
2381	0\\
2382	2.314827\\
2383	2.718571\\
2384	2.770039\\
2385	2.346802\\
2386	2.314827\\
2387	2.314827\\
2388	1.570747\\
2389	1.744193\\
2390	1.448217\\
2391	1.566297\\
2392	1.896264\\
2393	2.314827\\
2394	2.595316\\
2395	2.314827\\
2396	2.314827\\
2397	2.314827\\
2398	2.314827\\
2399	2.314827\\
2400	0.100625\\
2401	0\\
2402	0\\
2403	0\\
2404	0\\
2405	0\\
2406	1.622285\\
2407	2.314827\\
2408	2.314906\\
2409	2.314827\\
2410	2.314827\\
2411	2.314827\\
2412	1.809712\\
2413	1.212978\\
2414	0.609824\\
2415	1.055435\\
2416	1.087772\\
2417	1.317321\\
2418	1.924041\\
2419	1.40707\\
2420	1.725015\\
2421	2.249432\\
2422	2.314819\\
2423	1.317317\\
2424	0\\
2425	0\\
2426	0\\
2427	0\\
2428	0\\
2429	0\\
2430	0\\
2431	0\\
2432	0\\
2433	1e-05\\
2434	0\\
2435	0\\
2436	0\\
2437	0\\
2438	0\\
2439	0\\
2440	0\\
2441	0\\
2442	0\\
2443	0\\
2444	0\\
2445	0\\
2446	0\\
2447	0\\
2448	0\\
2449	0\\
2450	0\\
2451	0\\
2452	0\\
2453	0\\
2454	0\\
2455	0\\
2456	0\\
2457	0\\
2458	0\\
2459	0\\
2460	0\\
2461	0\\
2462	0\\
2463	0\\
2464	0\\
2465	0\\
2466	0\\
2467	0\\
2468	0\\
2469	1e-06\\
2470	0\\
2471	0\\
2472	0\\
2473	0\\
2474	0\\
2475	0\\
2476	0\\
2477	0\\
2478	0\\
2479	0\\
2480	0\\
2481	0\\
2482	0\\
2483	0\\
2484	0\\
2485	0\\
2486	0\\
2487	0\\
2488	0\\
2489	0\\
2490	0\\
2491	0\\
2492	0\\
2493	0\\
2494	0\\
2495	0\\
2496	0\\
2497	0\\
2498	0\\
2499	0\\
2500	0\\
2501	0\\
2502	0\\
2503	1e-06\\
2504	0.732802\\
2505	0.412974\\
2506	0\\
2507	0\\
2508	0\\
2509	0\\
2510	0\\
2511	0\\
2512	0\\
2513	0\\
2514	0\\
2515	0\\
2516	0\\
2517	3e-06\\
2518	0.412976\\
2519	0\\
2520	0\\
2521	0\\
2522	0\\
2523	0\\
2524	0\\
2525	0\\
2526	0\\
2527	2.3e-05\\
2528	0\\
2529	0\\
2530	0\\
2531	0\\
2532	0\\
2533	0\\
2534	1e-06\\
2535	0\\
2536	0\\
2537	0\\
2538	0\\
2539	3e-06\\
2540	3.8e-05\\
2541	0\\
2542	0.412788\\
2543	1e-06\\
2544	0\\
2545	0\\
2546	0\\
2547	0\\
2548	0\\
2549	0\\
2550	0\\
2551	0\\
2552	0\\
2553	0\\
2554	0\\
2555	0\\
2556	0\\
2557	0\\
2558	0\\
2559	0\\
2560	0\\
2561	0\\
2562	0\\
2563	0\\
2564	0\\
2565	0\\
2566	0\\
2567	0\\
2568	0\\
2569	0\\
2570	0\\
2571	0\\
2572	0\\
2573	0\\
2574	0\\
2575	0\\
2576	0\\
2577	0\\
2578	0\\
2579	0\\
2580	0\\
2581	0\\
2582	0\\
2583	0\\
2584	0\\
2585	0\\
2586	0\\
2587	0\\
2588	0\\
2589	0\\
2590	0\\
2591	0\\
2592	0\\
2593	0\\
2594	0\\
2595	0\\
2596	0\\
2597	0\\
2598	0\\
2599	0\\
2600	0\\
2601	0\\
2602	0\\
2603	0\\
2604	0\\
2605	0\\
2606	0\\
2607	0\\
2608	0\\
2609	0\\
2610	0\\
2611	0\\
2612	0\\
2613	0\\
2614	0\\
2615	0\\
2616	0\\
2617	0\\
2618	0\\
2619	0\\
2620	0\\
2621	0\\
2622	0\\
2623	0\\
2624	0\\
2625	0\\
2626	0\\
2627	0\\
2628	0\\
2629	0\\
2630	0\\
2631	0\\
2632	0\\
2633	0\\
2634	0\\
2635	0\\
2636	0\\
2637	0\\
2638	0\\
2639	0\\
2640	0\\
2641	0\\
2642	0\\
2643	0\\
2644	0\\
2645	0\\
2646	0\\
2647	0\\
2648	0\\
2649	0\\
2650	0\\
2651	0\\
2652	0\\
2653	0\\
2654	0\\
2655	0\\
2656	0\\
2657	0\\
2658	0\\
2659	0\\
2660	0\\
2661	0\\
2662	0\\
2663	0\\
2664	0\\
2665	0\\
2666	0\\
2667	0\\
2668	0\\
2669	0\\
2670	0\\
2671	1e-06\\
2672	0\\
2673	0\\
2674	0\\
2675	0\\
2676	0\\
2677	0\\
2678	0\\
2679	0\\
2680	0\\
2681	0\\
2682	6e-06\\
2683	0.227885\\
2684	0\\
2685	0\\
2686	0.039218\\
2687	0\\
2688	1e-06\\
2689	0\\
2690	0\\
2691	0\\
2692	0\\
2693	0\\
2694	0\\
2695	0\\
2696	0\\
2697	0\\
2698	0\\
2699	0\\
2700	0\\
2701	0\\
2702	0\\
2703	0\\
2704	0\\
2705	0\\
2706	0.412976\\
2707	0.412976\\
2708	0\\
2709	1.087826\\
2710	0.702618\\
2711	0\\
2712	0\\
2713	0\\
2714	0\\
2715	0\\
2716	0\\
2717	0\\
2718	0\\
2719	0.412977\\
2720	1.155222\\
2721	1.317321\\
2722	1.034625\\
2723	1.317321\\
2724	0.588913\\
2725	0.412976\\
2726	0\\
2727	0.0655\\
2728	0.164454\\
2729	0.445883\\
2730	0.829395\\
2731	0.249396\\
2732	0\\
2733	0\\
2734	0\\
2735	0\\
2736	0\\
2737	0\\
2738	0\\
2739	0\\
2740	0\\
2741	0\\
2742	0\\
2743	0\\
2744	0\\
2745	0\\
2746	1e-06\\
2747	0\\
2748	0\\
2749	0\\
2750	0\\
2751	0\\
2752	0\\
2753	0\\
2754	0\\
2755	0\\
2756	0\\
2757	0\\
2758	0\\
2759	0\\
2760	0\\
2761	0\\
2762	0\\
2763	0\\
2764	0\\
2765	0\\
2766	0\\
2767	0\\
2768	0\\
2769	0\\
2770	0\\
2771	0\\
2772	0\\
2773	0\\
2774	0\\
2775	0\\
2776	0\\
2777	0\\
2778	0\\
2779	0\\
2780	0\\
2781	0\\
2782	0\\
2783	0\\
2784	0\\
2785	0\\
2786	0\\
2787	0\\
2788	0\\
2789	0\\
2790	0\\
2791	0\\
2792	0\\
2793	0\\
2794	0\\
2795	0\\
2796	0\\
2797	0\\
2798	0\\
2799	0\\
2800	0\\
2801	0\\
2802	0\\
2803	0\\
2804	0\\
2805	0\\
2806	0\\
2807	0\\
2808	0\\
2809	0\\
2810	0\\
2811	0\\
2812	0\\
2813	0\\
2814	1e-06\\
2815	0\\
2816	0\\
2817	0\\
2818	0\\
2819	0\\
2820	0\\
2821	0\\
2822	0\\
2823	0\\
2824	0\\
2825	0\\
2826	5e-06\\
2827	0.138669\\
2828	0\\
2829	0\\
2830	0\\
2831	0\\
2832	0\\
2833	0\\
2834	0\\
2835	0\\
2836	0\\
2837	0\\
2838	0\\
2839	0\\
2840	3e-06\\
2841	0.588161\\
2842	0.241214\\
2843	1.087826\\
2844	0.412975\\
2845	1.087826\\
2846	1.089396\\
2847	0.649225\\
2848	0.413334\\
2849	0\\
2850	9e-06\\
2851	0\\
2852	0\\
2853	0\\
2854	0\\
2855	0\\
2856	0\\
2857	0\\
2858	0\\
2859	0\\
2860	0\\
2861	0\\
2862	0\\
2863	0\\
2864	0\\
2865	1.1e-05\\
2866	2e-06\\
2867	0\\
2868	0\\
2869	0\\
2870	0\\
2871	0\\
2872	0\\
2873	0\\
2874	0\\
2875	0\\
2876	0\\
2877	0\\
2878	0\\
2879	0\\
2880	0\\
2881	0\\
2882	0\\
2883	0\\
2884	0\\
2885	0\\
2886	0\\
2887	0\\
2888	0\\
2889	0\\
2890	0\\
2891	0\\
2892	0\\
2893	0\\
2894	0\\
2895	0\\
2896	0\\
2897	0\\
2898	0\\
2899	0\\
2900	0\\
2901	0\\
2902	0\\
2903	0\\
2904	0\\
2905	0\\
2906	0\\
2907	0\\
2908	0\\
2909	0\\
2910	0\\
2911	0\\
2912	0\\
2913	0\\
2914	0\\
2915	0\\
2916	0\\
2917	0\\
2918	0\\
2919	0\\
2920	0\\
2921	0\\
2922	0\\
2923	0\\
2924	0\\
2925	0\\
2926	0\\
2927	0\\
2928	0\\
2929	0\\
2930	0\\
2931	0\\
2932	0\\
2933	0\\
2934	0\\
2935	0\\
2936	0\\
2937	0\\
2938	0\\
2939	0\\
2940	0\\
2941	0\\
2942	0\\
2943	0\\
2944	0\\
2945	0\\
2946	0\\
2947	0\\
2948	0\\
2949	0\\
2950	0\\
2951	0\\
2952	0\\
2953	0\\
2954	0\\
2955	0\\
2956	0\\
2957	0\\
2958	0\\
2959	0\\
2960	0\\
2961	0\\
2962	0\\
2963	0\\
2964	0\\
2965	0\\
2966	0\\
2967	0\\
2968	0\\
2969	0\\
2970	0\\
2971	0\\
2972	0\\
2973	0\\
2974	0\\
2975	0\\
2976	0\\
2977	0\\
2978	0\\
2979	0\\
2980	0\\
2981	0\\
2982	0\\
2983	0\\
2984	0\\
2985	1e-06\\
2986	0\\
2987	0\\
2988	0\\
2989	0\\
2990	0\\
2991	0\\
2992	0\\
2993	0\\
2994	0\\
2995	0\\
2996	0\\
2997	0\\
2998	0\\
2999	0\\
3000	0\\
3001	0\\
3002	0\\
3003	0\\
3004	0\\
3005	0\\
3006	0\\
3007	0\\
3008	0\\
3009	0.167402\\
3010	8e-06\\
3011	0\\
3012	0\\
3013	0\\
3014	0\\
3015	0\\
3016	0\\
3017	0\\
3018	0\\
3019	0\\
3020	0\\
3021	0\\
3022	0\\
3023	0\\
3024	0\\
3025	0\\
3026	0\\
3027	0\\
3028	0\\
3029	0\\
3030	0\\
3031	0\\
3032	0.113309\\
3033	0\\
3034	0\\
3035	1e-06\\
3036	0\\
3037	0\\
3038	0\\
3039	0\\
3040	0\\
3041	0\\
3042	0\\
3043	0\\
3044	0\\
3045	0\\
3046	3e-06\\
3047	0\\
3048	0\\
3049	0\\
3050	0\\
3051	0\\
3052	0\\
3053	0\\
3054	0\\
3055	0\\
3056	0\\
3057	0\\
3058	0.000424\\
3059	0.783973\\
3060	1.1e-05\\
3061	0.386724\\
3062	0.386722\\
3063	0.452283\\
3064	0.472219\\
3065	0.919021\\
3066	0.934026\\
3067	0\\
3068	0\\
3069	0\\
3070	0\\
3071	0\\
3072	0\\
3073	0\\
3074	0\\
3075	0\\
3076	0\\
3077	0\\
3078	0\\
3079	0\\
3080	0\\
3081	0\\
3082	0\\
3083	0\\
3084	0\\
3085	0\\
3086	0\\
3087	0\\
3088	0\\
3089	0\\
3090	0\\
3091	0\\
3092	0\\
3093	0\\
3094	0\\
3095	0\\
3096	0\\
3097	0\\
3098	0\\
3099	0\\
3100	0\\
3101	0\\
3102	0\\
3103	0\\
3104	0\\
3105	0\\
3106	0\\
3107	0\\
3108	0\\
3109	0\\
3110	0\\
3111	0\\
3112	0\\
3113	0\\
3114	0\\
3115	0\\
3116	0\\
3117	0\\
3118	0\\
3119	0\\
3120	0\\
3121	0\\
3122	0\\
3123	0\\
3124	0\\
3125	0\\
3126	0\\
3127	0\\
3128	0\\
3129	0\\
3130	0\\
3131	0\\
3132	0\\
3133	0\\
3134	0\\
3135	0\\
3136	0\\
3137	0\\
3138	0\\
3139	0\\
3140	0\\
3141	0\\
3142	0\\
3143	0\\
3144	0\\
3145	0\\
3146	0\\
3147	0\\
3148	0\\
3149	0\\
3150	0\\
3151	0\\
3152	0\\
3153	0\\
3154	0\\
3155	0\\
3156	0\\
3157	0\\
3158	0\\
3159	0\\
3160	0\\
3161	0\\
3162	0\\
3163	0\\
3164	0.1535\\
3165	0\\
3166	0.386718\\
3167	0\\
3168	0\\
3169	0\\
3170	0\\
3171	0\\
3172	0\\
3173	0\\
3174	0\\
3175	0\\
3176	0.806984\\
3177	1.018673\\
3178	0\\
3179	1e-06\\
3180	0.877499\\
3181	0\\
3182	0\\
3183	0\\
3184	0\\
3185	0\\
3186	0.027219\\
3187	0\\
3188	0\\
3189	0\\
3190	0.571726\\
3191	0\\
3192	0\\
3193	0\\
3194	0\\
3195	0\\
3196	0\\
3197	0\\
3198	0\\
3199	0.482925\\
3200	0.264463\\
3201	0\\
3202	1e-06\\
3203	0\\
3204	0\\
3205	0\\
3206	0\\
3207	0\\
3208	0\\
3209	0\\
3210	0\\
3211	0\\
3212	0\\
3213	0\\
3214	2e-06\\
3215	0\\
3216	0\\
3217	0\\
3218	0\\
3219	0\\
3220	0\\
3221	0\\
3222	0\\
3223	1e-06\\
3224	0.099292\\
3225	1.230007\\
3226	6e-06\\
3227	0\\
3228	0\\
3229	0\\
3230	0\\
3231	0\\
3232	0\\
3233	0\\
3234	0\\
3235	0\\
3236	0\\
3237	0\\
3238	5e-06\\
3239	0\\
3240	0\\
3241	0\\
3242	0\\
3243	0\\
3244	0\\
3245	0\\
3246	0\\
3247	0\\
3248	1e-06\\
3249	0\\
3250	0\\
3251	0\\
3252	0\\
3253	0\\
3254	0\\
3255	0\\
3256	0\\
3257	0\\
3258	0\\
3259	0\\
3260	0\\
3261	0\\
3262	0\\
3263	0\\
3264	0\\
3265	0\\
3266	0\\
3267	0\\
3268	0\\
3269	0\\
3270	0\\
3271	0\\
3272	0\\
3273	0\\
3274	0\\
3275	0\\
3276	0\\
3277	0\\
3278	0\\
3279	0\\
3280	0\\
3281	0\\
3282	0\\
3283	0\\
3284	0\\
3285	0\\
3286	0\\
3287	0\\
3288	0\\
3289	0\\
3290	0\\
3291	0\\
3292	0\\
3293	0\\
3294	0\\
3295	0\\
3296	0\\
3297	0\\
3298	0\\
3299	0\\
3300	0\\
3301	0\\
3302	0\\
3303	0\\
3304	0\\
3305	0\\
3306	0\\
3307	0\\
3308	0\\
3309	0\\
3310	0\\
3311	0\\
3312	0\\
3313	0\\
3314	0\\
3315	0\\
3316	0\\
3317	0\\
3318	0\\
3319	0\\
3320	0\\
3321	0\\
3322	0\\
3323	0\\
3324	0\\
3325	0\\
3326	0\\
3327	0\\
3328	0\\
3329	0\\
3330	0\\
3331	0\\
3332	0\\
3333	0\\
3334	0\\
3335	0\\
3336	0\\
3337	0\\
3338	0\\
3339	0\\
3340	0\\
3341	0\\
3342	0\\
3343	0\\
3344	0\\
3345	0\\
3346	0\\
3347	0\\
3348	0\\
3349	0\\
3350	0\\
3351	0\\
3352	0\\
3353	4e-06\\
3354	0.386724\\
3355	0.894831\\
3356	1e-06\\
3357	0.391413\\
3358	0\\
3359	0\\
3360	0\\
3361	0\\
3362	0\\
3363	0\\
3364	0\\
3365	0\\
3366	0\\
3367	0\\
3368	0\\
3369	0.603811\\
3370	0\\
3371	1.187502\\
3372	0.780669\\
3373	0.258023\\
3374	0.386724\\
3375	0.850344\\
3376	0.547766\\
3377	0.419534\\
3378	0.386723\\
3379	2e-06\\
3380	0\\
3381	0\\
3382	1e-06\\
3383	0\\
3384	0\\
3385	0\\
3386	0\\
3387	0\\
3388	0\\
3389	0\\
3390	0\\
3391	0\\
3392	0\\
3393	0\\
3394	1e-06\\
3395	0\\
3396	0\\
3397	0\\
3398	0\\
3399	0\\
3400	0\\
3401	1e-06\\
3402	0\\
3403	0\\
3404	0\\
3405	0\\
3406	0\\
3407	0\\
3408	0\\
3409	0\\
3410	0\\
3411	0\\
3412	0\\
3413	0\\
3414	0\\
3415	0\\
3416	0\\
3417	0\\
3418	0\\
3419	0\\
3420	0\\
3421	0\\
3422	0\\
3423	0\\
3424	0\\
3425	0\\
3426	0\\
3427	1e-06\\
3428	0\\
3429	0\\
3430	0\\
3431	0\\
3432	0\\
3433	0\\
3434	0\\
3435	0\\
3436	0\\
3437	0\\
3438	0\\
3439	0\\
3440	0\\
3441	0\\
3442	0\\
3443	0\\
3444	0\\
3445	0\\
3446	0\\
3447	0\\
3448	0\\
3449	0\\
3450	0\\
3451	0\\
3452	0\\
3453	0\\
3454	0\\
3455	0\\
3456	0\\
3457	0\\
3458	0\\
3459	0\\
3460	0\\
3461	0\\
3462	0\\
3463	0\\
3464	0\\
3465	0\\
3466	0\\
3467	0\\
3468	0\\
3469	0\\
3470	0\\
3471	0\\
3472	0\\
3473	0\\
3474	0\\
3475	0\\
3476	0\\
3477	0\\
3478	0\\
3479	0\\
3480	0\\
3481	0\\
3482	0\\
3483	0\\
3484	0\\
3485	0\\
3486	0\\
3487	1e-06\\
3488	2e-06\\
3489	0.838923\\
3490	1.034809\\
3491	1.233577\\
3492	0.209656\\
3493	0\\
3494	0\\
3495	0\\
3496	0\\
3497	1e-06\\
3498	0\\
3499	0\\
3500	0\\
3501	0\\
3502	0\\
3503	0\\
3504	0\\
3505	0\\
3506	0\\
3507	0\\
3508	0\\
3509	0\\
3510	0\\
3511	0\\
3512	2e-06\\
3513	0.741001\\
3514	1.233579\\
3515	1.018925\\
3516	0\\
3517	0\\
3518	0\\
3519	0.386724\\
3520	0.386724\\
3521	0.386724\\
3522	0.386724\\
3523	0\\
3524	0\\
3525	0\\
3526	0\\
3527	0\\
3528	0\\
3529	0\\
3530	0\\
3531	0\\
3532	0\\
3533	0\\
3534	0\\
3535	0\\
3536	0\\
3537	0.612381\\
3538	1.233579\\
3539	1.180816\\
3540	1.233579\\
3541	1.018674\\
3542	0.337421\\
3543	0\\
3544	0\\
3545	4.1e-05\\
3546	1e-05\\
3547	0\\
3548	0\\
3549	0\\
3550	0\\
3551	0\\
3552	0\\
3553	0\\
3554	0\\
3555	0\\
3556	0\\
3557	0\\
3558	0\\
3559	0\\
3560	0\\
3561	0\\
3562	0\\
3563	0\\
3564	0\\
3565	0\\
3566	0\\
3567	0\\
3568	0\\
3569	0\\
3570	0\\
3571	0\\
3572	0\\
3573	0\\
3574	0\\
3575	0\\
3576	0\\
3577	0\\
3578	0\\
3579	0\\
3580	0\\
3581	0\\
3582	0\\
3583	0\\
3584	0\\
3585	0\\
3586	0\\
3587	0\\
3588	0\\
3589	0\\
3590	0\\
3591	0\\
3592	0\\
3593	0\\
3594	0\\
3595	0\\
3596	0\\
3597	0\\
3598	0\\
3599	0\\
3600	0\\
3601	0\\
3602	0\\
3603	0\\
3604	0\\
3605	0\\
3606	0\\
3607	0\\
3608	0\\
3609	0\\
3610	0\\
3611	0\\
3612	0\\
3613	0\\
3614	0\\
3615	0\\
3616	0\\
3617	0\\
3618	0\\
3619	0\\
3620	0\\
3621	0\\
3622	0\\
3623	0\\
3624	0\\
3625	0\\
3626	0\\
3627	0\\
3628	0\\
3629	0\\
3630	0\\
3631	0\\
3632	0\\
3633	0\\
3634	0\\
3635	0\\
3636	0\\
3637	0\\
3638	0\\
3639	0\\
3640	0\\
3641	0\\
3642	0\\
3643	0\\
3644	0\\
3645	0\\
3646	0\\
3647	1e-06\\
3648	0\\
3649	0\\
3650	0\\
3651	0\\
3652	0\\
3653	0\\
3654	0\\
3655	1.223764\\
3656	1.693382\\
3657	1.693379\\
3658	1.115053\\
3659	1.115053\\
3660	0.727875\\
3661	0.764794\\
3662	0.764794\\
3663	0.764794\\
3664	1.105196\\
3665	1.693382\\
3666	1.120149\\
3667	1.502566\\
3668	1.079317\\
3669	1.11505\\
3670	1.865811\\
3671	0.920796\\
3672	0\\
3673	0\\
3674	0\\
3675	0\\
3676	0\\
3677	0\\
3678	0\\
3679	1.693382\\
3680	1.959397\\
3681	1.959397\\
3682	1.808165\\
3683	1.693382\\
3684	0.764794\\
3685	0.056884\\
3686	0.488255\\
3687	0.920796\\
3688	1.360236\\
3689	1.342407\\
3690	1.959397\\
3691	1.959397\\
3692	1.640509\\
3693	1.115052\\
3694	1.693382\\
3695	0.806083\\
3696	1e-06\\
3697	0\\
3698	0\\
3699	0\\
3700	0\\
3701	0\\
3702	0.920796\\
3703	1.959397\\
3704	3.205717\\
3705	3.205717\\
3706	3.401897\\
3707	3.332898\\
3708	3.401897\\
3709	3.401897\\
3710	2.381569\\
3711	2.351588\\
3712	2.170991\\
3713	2.381565\\
3714	3.199947\\
3715	2.273837\\
3716	1.959397\\
3717	1.280202\\
3718	0.764794\\
3719	1e-06\\
3720	0\\
3721	0\\
3722	0\\
3723	0\\
3724	0\\
3725	0\\
3726	0\\
3727	0\\
3728	0.764794\\
3729	0.764793\\
3730	0\\
3731	1e-06\\
3732	0\\
3733	0\\
3734	1e-06\\
3735	0\\
3736	0\\
3737	0\\
3738	0.375035\\
3739	0.764794\\
3740	0.393489\\
3741	0.088506\\
3742	1.115053\\
3743	0.764794\\
3744	0\\
3745	0\\
3746	0\\
3747	0\\
3748	0\\
3749	0\\
3750	0.049936\\
3751	1.959397\\
3752	1.959397\\
3753	1.693382\\
3754	1.115053\\
3755	0.824298\\
3756	0.764793\\
3757	0.300332\\
3758	1e-06\\
3759	1e-06\\
3760	0\\
3761	0.764793\\
3762	0.831261\\
3763	0.920796\\
3764	0.349566\\
3765	1e-06\\
3766	0.685846\\
3767	1e-06\\
3768	0\\
3769	0\\
3770	0\\
3771	0\\
3772	0\\
3773	0\\
3774	0\\
3775	0\\
3776	0\\
3777	0\\
3778	0\\
3779	0\\
3780	0\\
3781	0\\
3782	0\\
3783	0\\
3784	0\\
3785	0\\
3786	0\\
3787	0\\
3788	0\\
3789	0\\
3790	0\\
3791	0\\
3792	0\\
3793	0\\
3794	0\\
3795	0\\
3796	0\\
3797	0\\
3798	0\\
3799	0\\
3800	0\\
3801	0\\
3802	0\\
3803	0\\
3804	0\\
3805	0\\
3806	0\\
3807	0\\
3808	0\\
3809	0\\
3810	0\\
3811	0\\
3812	0\\
3813	0\\
3814	0\\
3815	0\\
3816	0\\
3817	0\\
3818	0\\
3819	0\\
3820	0\\
3821	0\\
3822	0\\
3823	0\\
3824	0\\
3825	0\\
3826	0\\
3827	0\\
3828	0\\
3829	0\\
3830	0\\
3831	0\\
3832	0\\
3833	0\\
3834	0\\
3835	0\\
3836	0\\
3837	0\\
3838	0\\
3839	0\\
3840	0\\
3841	0\\
3842	0\\
3843	0\\
3844	0\\
3845	0\\
3846	0\\
3847	1.693382\\
3848	2.098371\\
3849	2.495989\\
3850	2.251793\\
3851	2.381569\\
3852	1.959397\\
3853	1.959397\\
3854	1.959397\\
3855	1.959397\\
3856	1.959397\\
3857	1.959397\\
3858	2.493576\\
3859	2.381568\\
3860	1.959397\\
3861	1.959397\\
3862	1.959397\\
3863	1.959395\\
3864	0.697048\\
3865	0.187077\\
3866	0\\
3867	0\\
3868	0\\
3869	0.045635\\
3870	2.93933\\
3871	3.361507\\
3872	4.185656\\
3873	4.185656\\
3874	3.361506\\
3875	3.09178\\
3876	2.939335\\
3877	2.939335\\
3878	2.939335\\
3879	2.939335\\
3880	2.939335\\
3881	3.475943\\
3882	1.976398\\
3883	2.664675\\
3884	2.537748\\
3885	1.959397\\
3886	2.381572\\
3887	1.959397\\
3888	1.115053\\
3889	0.337629\\
3890	0\\
3891	0\\
3892	0\\
3893	3e-06\\
3894	1.507442\\
3895	2.591301\\
3896	3.205717\\
3897	2.225265\\
3898	2.123857\\
3899	1.959397\\
3900	1.693382\\
3901	1.829554\\
3902	1.693382\\
3903	1.693382\\
3904	1.959397\\
3905	1.959397\\
3906	2.795974\\
3907	2.725953\\
3908	2.381568\\
3909	2.044977\\
3910	2.787278\\
3911	1.959397\\
3912	1.959397\\
3913	1.115053\\
3914	0.742499\\
3915	0.349566\\
3916	0\\
3917	1e-05\\
3918	1.693383\\
3919	3.124015\\
3920	3.205717\\
3921	2.913634\\
3922	2.403187\\
3923	2.376236\\
3924	1.959397\\
3925	1.959397\\
3926	1.959397\\
3927	1.693383\\
3928	1.959397\\
3929	1.959397\\
3930	1.959397\\
3931	1.959397\\
3932	1.959397\\
3933	1.959397\\
3934	1.983972\\
3935	1.959397\\
3936	1.115053\\
3937	0.308663\\
3938	0.649688\\
3939	0.096245\\
3940	0\\
3941	0\\
3942	0\\
3943	0\\
3944	0.764794\\
3945	0.975156\\
3946	0.764792\\
3947	0.679183\\
3948	0\\
3949	0\\
3950	0\\
3951	0\\
3952	0\\
3953	0\\
3954	0\\
3955	0.764791\\
3956	0.402316\\
3957	0.242758\\
3958	1.115053\\
3959	0.764794\\
3960	0\\
3961	0\\
3962	0\\
3963	0\\
3964	0\\
3965	0\\
3966	0\\
3967	0\\
3968	0\\
3969	0\\
3970	0\\
3971	0\\
3972	0\\
3973	0\\
3974	0\\
3975	0\\
3976	0\\
3977	0\\
3978	0\\
3979	0\\
3980	2e-06\\
3981	0\\
3982	0.58917\\
3983	1e-06\\
3984	0\\
3985	0\\
3986	0\\
3987	0\\
3988	0\\
3989	1e-06\\
3990	0\\
3991	1.757027\\
3992	1.959397\\
3993	1.959397\\
3994	1.844944\\
3995	1.959397\\
3996	1.952889\\
3997	1.693382\\
3998	1.149902\\
3999	1.115053\\
4000	1.147688\\
4001	1.54023\\
};
\addplot [color=mycolor1,solid,line width=1.0pt,forget plot]
  table[row sep=crcr]{%
4001	1.54023\\
4002	1.722436\\
4003	1.543806\\
4004	1.115053\\
4005	0.934458\\
4006	1.294275\\
4007	0.764794\\
4008	0\\
4009	0\\
4010	0\\
4011	0\\
4012	0\\
4013	1e-06\\
4014	0.764794\\
4015	1.959396\\
4016	2.24662\\
4017	3.199947\\
4018	1.959397\\
4019	2.098358\\
4020	1.115053\\
4021	0.205704\\
4022	0.303727\\
4023	0.409268\\
4024	0.508245\\
4025	0.764794\\
4026	0.126219\\
4027	0.497841\\
4028	0.764794\\
4029	0.764794\\
4030	1.158209\\
4031	0.873228\\
4032	0\\
4033	0\\
4034	0\\
4035	0\\
4036	0\\
4037	0\\
4038	0.04578\\
4039	1.693382\\
4040	1.693382\\
4041	1.847006\\
4042	1.531936\\
4043	1.524752\\
4044	1.537616\\
4045	0.764794\\
4046	0.349566\\
4047	0.128196\\
4048	0.581163\\
4049	1.044599\\
4050	1.115053\\
4051	1.115053\\
4052	0.920902\\
4053	0.75197\\
4054	1.115053\\
4055	0.916125\\
4056	0\\
4057	0\\
4058	0\\
4059	0\\
4060	0\\
4061	0\\
4062	0.518798\\
4063	1.693382\\
4064	1.959397\\
4065	1.959397\\
4066	1.959397\\
4067	1.692586\\
4068	0.920406\\
4069	0.764794\\
4070	1.031998\\
4071	0.764796\\
4072	1.115053\\
4073	1.922381\\
4074	1.959397\\
4075	1.693382\\
4076	1.033323\\
4077	0.704635\\
4078	1.115053\\
4079	0.449269\\
4080	0\\
4081	0\\
4082	0\\
4083	0\\
4084	0\\
4085	0\\
4086	0\\
4087	0.764794\\
4088	1.305821\\
4089	1.513515\\
4090	1.693382\\
4091	1.514135\\
4092	0.92052\\
4093	0.643998\\
4094	1.115053\\
4095	0.920795\\
4096	1.115053\\
4097	1.507683\\
4098	1.509178\\
4099	1.693382\\
4100	1.531936\\
4101	1.115053\\
4102	1.830052\\
4103	1.526869\\
4104	0.264975\\
4105	0\\
4106	0\\
4107	0\\
4108	0\\
4109	0\\
4110	0\\
4111	0\\
4112	0.71664\\
4113	0.622102\\
4114	0.764794\\
4115	2.4e-05\\
4116	0\\
4117	0\\
4118	0\\
4119	0\\
4120	0\\
4121	0\\
4122	0\\
4123	0.154191\\
4124	0.254102\\
4125	0.134281\\
4126	0.686156\\
4127	0.638047\\
4128	0\\
4129	0\\
4130	0\\
4131	0\\
4132	0\\
4133	0\\
4134	0\\
4135	0\\
4136	0\\
4137	0\\
4138	0\\
4139	0\\
4140	0\\
4141	0\\
4142	0\\
4143	0\\
4144	0\\
4145	0\\
4146	0\\
4147	0\\
4148	1e-06\\
4149	2e-06\\
4150	0.764794\\
4151	0.323345\\
4152	0\\
4153	0\\
4154	0\\
4155	0\\
4156	0\\
4157	0\\
4158	0.331956\\
4159	1.959397\\
4160	1.959397\\
4161	1.959397\\
4162	1.937936\\
4163	1.744797\\
4164	1.551784\\
4165	1.437292\\
4166	1.115053\\
4167	1.115053\\
4168	1.693382\\
4169	1.959397\\
4170	1.959397\\
4171	1.959397\\
4172	1.959397\\
4173	1.115053\\
4174	1.959397\\
4175	1.959397\\
4176	0.764794\\
4177	0\\
4178	0\\
4179	0\\
4180	0\\
4181	0\\
4182	0.687973\\
4183	1.693382\\
4184	1.959397\\
4185	1.959397\\
4186	1.959397\\
4187	2.411319\\
4188	1.959397\\
4189	2.320135\\
4190	1.959397\\
4191	1.959397\\
4192	1.959397\\
4193	1.959397\\
4194	1.453721\\
4195	1.693382\\
4196	1.959397\\
4197	1.693382\\
4198	1.959397\\
4199	1.871152\\
4200	0.815156\\
4201	2e-06\\
4202	0\\
4203	0\\
4204	0\\
4205	0\\
4206	1e-06\\
4207	1.678017\\
4208	1.532155\\
4209	1.52172\\
4210	1.115053\\
4211	1.115053\\
4212	0.764794\\
4213	1.115053\\
4214	1.108899\\
4215	1.036932\\
4216	1.115053\\
4217	1.652166\\
4218	1.693382\\
4219	1.693382\\
4220	1.634843\\
4221	1.115053\\
4222	1.959397\\
4223	1.531936\\
4224	0.209002\\
4225	0\\
4226	0\\
4227	0\\
4228	0\\
4229	0\\
4230	0.783172\\
4231	1.453523\\
4232	1.693382\\
4233	1.959397\\
4234	1.959397\\
4235	1.830053\\
4236	1.693382\\
4237	1.472621\\
4238	1.22745\\
4239	1.243957\\
4240	1.693382\\
4241	1.959397\\
4242	1.959397\\
4243	1.288621\\
4244	1.115053\\
4245	1.115053\\
4246	1.959397\\
4247	1.693383\\
4248	0.75505\\
4249	0\\
4250	0\\
4251	0\\
4252	0\\
4253	0\\
4254	0.349567\\
4255	1.282711\\
4256	1.693382\\
4257	1.959397\\
4258	2.386478\\
4259	1.959397\\
4260	1.959397\\
4261	2.286503\\
4262	1.959397\\
4263	0.764794\\
4264	0.920796\\
4265	1.115053\\
4266	0.840136\\
4267	0.764794\\
4268	0.53073\\
4269	0.464068\\
4270	0.764794\\
4271	0.258006\\
4272	0\\
4273	0\\
4274	0\\
4275	0\\
4276	0\\
4277	0\\
4278	0\\
4279	0\\
4280	0\\
4281	0\\
4282	0\\
4283	0\\
4284	0\\
4285	0\\
4286	0\\
4287	0\\
4288	0\\
4289	0\\
4290	0.192817\\
4291	0\\
4292	0\\
4293	0\\
4294	0\\
4295	0\\
4296	0\\
4297	0\\
4298	0\\
4299	0\\
4300	0\\
4301	0\\
4302	0\\
4303	0\\
4304	0\\
4305	0\\
4306	0\\
4307	0\\
4308	0\\
4309	0\\
4310	0\\
4311	0\\
4312	0\\
4313	0\\
4314	0\\
4315	0\\
4316	0\\
4317	0\\
4318	0\\
4319	0\\
4320	0\\
4321	0\\
4322	0\\
4323	0\\
4324	0\\
4325	0\\
4326	0\\
4327	0.349565\\
4328	0.764794\\
4329	0.985842\\
4330	0.920781\\
4331	1.115053\\
4332	1.0289\\
4333	0.764794\\
4334	0.289984\\
4335	0.151122\\
4336	0.694309\\
4337	0.901202\\
4338	1.693382\\
4339	1.693382\\
4340	1.693382\\
4341	1.120958\\
4342	1.719329\\
4343	1.115053\\
4344	0.871858\\
4345	1e-06\\
4346	0\\
4347	0\\
4348	0\\
4349	0\\
4350	1.05579\\
4351	2.254992\\
4352	2.50001\\
4353	2.199245\\
4354	1.855258\\
4355	1.855258\\
4356	1.163586\\
4357	1.05612\\
4358	0.871858\\
4359	0.888148\\
4360	1.05579\\
4361	1.855258\\
4362	1.855258\\
4363	1.654423\\
4364	1.855258\\
4365	0\\
4366	0.330989\\
4367	1e-06\\
4368	0\\
4369	0\\
4370	0\\
4371	0\\
4372	0\\
4373	0\\
4374	0\\
4375	0.469224\\
4376	1.05579\\
4377	1.05579\\
4378	0.871855\\
4379	0.34343\\
4380	0\\
4381	0\\
4382	0\\
4383	0\\
4384	0\\
4385	0.330987\\
4386	1.05579\\
4387	1.05579\\
4388	1.731488\\
4389	1.056011\\
4390	1.716538\\
4391	0.47674\\
4392	0\\
4393	0\\
4394	0\\
4395	0\\
4396	0\\
4397	0\\
4398	0\\
4399	3e-06\\
4400	1.05579\\
4401	1.055789\\
4402	0.330986\\
4403	5e-06\\
4404	0\\
4405	0\\
4406	0\\
4407	0\\
4408	0\\
4409	1e-06\\
4410	0.170228\\
4411	0\\
4412	1e-06\\
4413	1e-06\\
4414	0.000109\\
4415	0\\
4416	0\\
4417	0\\
4418	0\\
4419	0\\
4420	0\\
4421	0\\
4422	0\\
4423	0.550942\\
4424	1.05579\\
4425	1.069568\\
4426	1.09967\\
4427	1.450549\\
4428	1.05579\\
4429	1.05579\\
4430	1.05579\\
4431	0.892253\\
4432	0.330987\\
4433	0.871858\\
4434	1.05579\\
4435	0.871857\\
4436	0.330989\\
4437	1e-06\\
4438	3.8e-05\\
4439	0\\
4440	0\\
4441	0\\
4442	0\\
4443	0\\
4444	0\\
4445	0\\
4446	0\\
4447	0\\
4448	0\\
4449	0\\
4450	0\\
4451	0\\
4452	0\\
4453	0\\
4454	0\\
4455	0\\
4456	0\\
4457	0\\
4458	0\\
4459	0\\
4460	0\\
4461	0\\
4462	0\\
4463	0\\
4464	0\\
4465	0\\
4466	0\\
4467	0\\
4468	0\\
4469	0\\
4470	0\\
4471	0\\
4472	0\\
4473	0\\
4474	0\\
4475	0\\
4476	0\\
4477	0\\
4478	0\\
4479	0\\
4480	0\\
4481	0\\
4482	0\\
4483	0\\
4484	0\\
4485	0\\
4486	0\\
4487	0\\
4488	0\\
4489	0\\
4490	0\\
4491	0\\
4492	0\\
4493	0\\
4494	0\\
4495	0\\
4496	1e-06\\
4497	1e-06\\
4498	1e-06\\
4499	1e-06\\
4500	0\\
4501	0\\
4502	1e-06\\
4503	0.043052\\
4504	0.871858\\
4505	0.348338\\
4506	1.05579\\
4507	1.016108\\
4508	1.303517\\
4509	1.05579\\
4510	1.739843\\
4511	0.777643\\
4512	0\\
4513	0\\
4514	0\\
4515	0\\
4516	0\\
4517	0\\
4518	0\\
4519	0.903378\\
4520	1.855258\\
4521	2.500648\\
4522	2.72168\\
4523	3.035339\\
4524	2.500648\\
4525	2.500648\\
4526	2.500648\\
4527	2.254992\\
4528	1.982721\\
4529	2.254992\\
4530	1.986799\\
4531	1.855258\\
4532	1.855258\\
4533	1.267088\\
4534	1.855258\\
4535	1.05579\\
4536	0\\
4537	0\\
4538	0\\
4539	0\\
4540	0\\
4541	0\\
4542	0\\
4543	0\\
4544	0.799357\\
4545	1.05579\\
4546	1.83179\\
4547	1.855258\\
4548	1.855258\\
4549	1.855258\\
4550	1.855258\\
4551	1.773967\\
4552	1.618517\\
4553	1.450514\\
4554	1.05579\\
4555	8e-06\\
4556	0\\
4557	0\\
4558	0\\
4559	0\\
4560	0\\
4561	0\\
4562	0\\
4563	0\\
4564	0\\
4565	0\\
4566	0\\
4567	0\\
4568	0.892154\\
4569	1.05579\\
4570	0.871858\\
4571	0.874419\\
4572	1.561745\\
4573	1.05579\\
4574	0.872628\\
4575	0.623119\\
4576	0.270931\\
4577	0.871857\\
4578	0.871858\\
4579	0.629801\\
4580	0\\
4581	0.070293\\
4582	0.617442\\
4583	0\\
4584	0\\
4585	0\\
4586	0\\
4587	0\\
4588	0\\
4589	0\\
4590	0\\
4591	4e-06\\
4592	1.1e-05\\
4593	0\\
4594	0\\
4595	1.84699\\
4596	1.633631\\
4597	1.055789\\
4598	0.330989\\
4599	3e-06\\
4600	0\\
4601	0.443463\\
4602	0.491984\\
4603	2e-06\\
4604	0.247621\\
4605	2e-05\\
4606	0.491956\\
4607	0.871858\\
4608	0\\
4609	0\\
4610	0\\
4611	0\\
4612	0\\
4613	0\\
4614	0\\
4615	0\\
4616	0\\
4617	0\\
4618	3e-06\\
4619	4e-06\\
4620	0\\
4621	0\\
4622	0\\
4623	0\\
4624	0\\
4625	0\\
4626	0\\
4627	1e-06\\
4628	0\\
4629	0\\
4630	0\\
4631	0\\
4632	0\\
4633	0\\
4634	0\\
4635	0\\
4636	0\\
4637	0\\
4638	0\\
4639	0\\
4640	0\\
4641	0\\
4642	0\\
4643	0\\
4644	0\\
4645	0\\
4646	0\\
4647	0\\
4648	0\\
4649	0\\
4650	0\\
4651	0\\
4652	0\\
4653	0\\
4654	0\\
4655	0\\
4656	0\\
4657	0\\
4658	0\\
4659	0\\
4660	0\\
4661	0\\
4662	0\\
4663	0\\
4664	0.331895\\
4665	2.1e-05\\
4666	1.05579\\
4667	1.181338\\
4668	1.240851\\
4669	0.837936\\
4670	0.26434\\
4671	0\\
4672	0.140584\\
4673	0.871857\\
4674	1.05579\\
4675	0.741083\\
4676	0.384091\\
4677	0.000132\\
4678	0.878884\\
4679	0.1587\\
4680	0\\
4681	0\\
4682	0\\
4683	0\\
4684	0\\
4685	0\\
4686	0\\
4687	0\\
4688	0\\
4689	8e-06\\
4690	1.05579\\
4691	1.737425\\
4692	1.855258\\
4693	1.607822\\
4694	1.317898\\
4695	1.494735\\
4696	1.421172\\
4697	1.855258\\
4698	1.855258\\
4699	1.05579\\
4700	1.05579\\
4701	1.055791\\
4702	1.855258\\
4703	1.855258\\
4704	1.05579\\
4705	0.01169\\
4706	1e-06\\
4707	0\\
4708	0\\
4709	0\\
4710	1.05579\\
4711	1.940049\\
4712	2.690348\\
4713	2.851895\\
4714	2.500648\\
4715	2.254992\\
4716	1.855258\\
4717	1.772686\\
4718	1.105199\\
4719	1.764582\\
4720	1.542092\\
4721	1.855258\\
4722	1.9243\\
4723	1.855258\\
4724	1.05579\\
4725	1.055797\\
4726	1.855258\\
4727	1.500168\\
4728	0.000203\\
4729	0\\
4730	0\\
4731	0\\
4732	0\\
4733	0\\
4734	1e-06\\
4735	1.05579\\
4736	1.855258\\
4737	1.855258\\
4738	1.493651\\
4739	1.110026\\
4740	1.05579\\
4741	0.666688\\
4742	0.392913\\
4743	0.871858\\
4744	1.05579\\
4745	1.732789\\
4746	1.855258\\
4747	1.732787\\
4748	1.015946\\
4749	1.055786\\
4750	1.520588\\
4751	0.330984\\
4752	0\\
4753	0\\
4754	0\\
4755	0\\
4756	0\\
4757	0\\
4758	0\\
4759	0\\
4760	0.871858\\
4761	0.594858\\
4762	0.384694\\
4763	1.05579\\
4764	0\\
4765	0\\
4766	0\\
4767	0\\
4768	0\\
4769	1e-06\\
4770	0.330987\\
4771	0.877881\\
4772	0.818447\\
4773	0.492437\\
4774	1.05579\\
4775	7e-06\\
4776	0\\
4777	0\\
4778	0\\
4779	0\\
4780	0\\
4781	0\\
4782	0\\
4783	0\\
4784	0\\
4785	0\\
4786	0\\
4787	0\\
4788	0\\
4789	0\\
4790	0\\
4791	0\\
4792	0\\
4793	0\\
4794	0\\
4795	0\\
4796	0\\
4797	0\\
4798	0\\
4799	0\\
4800	0\\
4801	0\\
4802	0\\
4803	0\\
4804	0\\
4805	0\\
4806	0\\
4807	0\\
4808	0\\
4809	0\\
4810	0\\
4811	0\\
4812	0\\
4813	0\\
4814	0\\
4815	0\\
4816	0\\
4817	0\\
4818	0\\
4819	0\\
4820	0\\
4821	0\\
4822	0\\
4823	0\\
4824	0\\
4825	0\\
4826	0\\
4827	0\\
4828	0\\
4829	0\\
4830	0\\
4831	0\\
4832	0\\
4833	0\\
4834	0\\
4835	0\\
4836	0\\
4837	0\\
4838	0\\
4839	0\\
4840	0\\
4841	0\\
4842	0\\
4843	0\\
4844	0\\
4845	0\\
4846	0\\
4847	0\\
4848	0\\
4849	0\\
4850	0\\
4851	0\\
4852	0\\
4853	0\\
4854	0\\
4855	0\\
4856	0\\
4857	0\\
4858	0\\
4859	0\\
4860	0\\
4861	0\\
4862	0\\
4863	0\\
4864	0\\
4865	0\\
4866	0\\
4867	0\\
4868	0\\
4869	0\\
4870	0\\
4871	0\\
4872	0\\
4873	0\\
4874	0\\
4875	0\\
4876	0\\
4877	0\\
4878	0\\
4879	0\\
4880	0\\
4881	0\\
4882	0\\
4883	0\\
4884	0\\
4885	0\\
4886	0\\
4887	0\\
4888	0\\
4889	0\\
4890	0\\
4891	0\\
4892	0\\
4893	0\\
4894	0\\
4895	0\\
4896	0\\
4897	0\\
4898	0\\
4899	0\\
4900	0\\
4901	0\\
4902	0\\
4903	0\\
4904	0\\
4905	0\\
4906	0\\
4907	0\\
4908	0\\
4909	0\\
4910	0\\
4911	0\\
4912	0\\
4913	0\\
4914	0\\
4915	0\\
4916	0\\
4917	0\\
4918	0\\
4919	0\\
4920	0\\
4921	0\\
4922	0\\
4923	0\\
4924	0\\
4925	0\\
4926	0\\
4927	0\\
4928	0\\
4929	0\\
4930	0\\
4931	0\\
4932	0\\
4933	0\\
4934	0\\
4935	0\\
4936	0\\
4937	0\\
4938	0\\
4939	0\\
4940	0\\
4941	0\\
4942	0\\
4943	0\\
4944	0\\
4945	0\\
4946	0\\
4947	0\\
4948	0\\
4949	0\\
4950	0\\
4951	0\\
4952	0\\
4953	0\\
4954	0\\
4955	0\\
4956	0\\
4957	0\\
4958	0\\
4959	0\\
4960	0\\
4961	0\\
4962	0\\
4963	0\\
4964	0\\
4965	0\\
4966	0\\
4967	0\\
4968	0\\
4969	0\\
4970	0\\
4971	0\\
4972	0\\
4973	0\\
4974	0\\
4975	0\\
4976	0\\
4977	0\\
4978	0\\
4979	0\\
4980	0\\
4981	0\\
4982	0\\
4983	0\\
4984	0\\
4985	0\\
4986	0\\
4987	0\\
4988	0\\
4989	0\\
4990	0\\
4991	0\\
4992	0\\
4993	0\\
4994	0\\
4995	0\\
4996	0\\
4997	0\\
4998	0\\
4999	0\\
5000	0\\
5001	0\\
5002	0\\
5003	0\\
5004	0\\
5005	0\\
5006	0\\
5007	0\\
5008	0\\
5009	0\\
5010	0\\
5011	0\\
5012	0\\
5013	0\\
5014	0\\
5015	0\\
5016	0\\
5017	0\\
5018	0\\
5019	0\\
5020	0\\
5021	0\\
5022	0\\
5023	0\\
5024	0\\
5025	0\\
5026	0\\
5027	0\\
5028	0\\
5029	0\\
5030	0\\
5031	0\\
5032	0\\
5033	0\\
5034	0\\
5035	0\\
5036	0\\
5037	0\\
5038	0\\
5039	0\\
5040	0\\
5041	0\\
5042	0\\
5043	0\\
5044	0\\
5045	0\\
5046	0\\
5047	0\\
5048	0\\
5049	0\\
5050	0\\
5051	0\\
5052	0\\
5053	0\\
5054	0\\
5055	0\\
5056	0\\
5057	0\\
5058	0\\
5059	0\\
5060	0\\
5061	0\\
5062	0\\
5063	0\\
5064	0\\
5065	0\\
5066	0\\
5067	0\\
5068	0\\
5069	0\\
5070	0\\
5071	0\\
5072	0\\
5073	0\\
5074	0\\
5075	0\\
5076	0\\
5077	0\\
5078	0\\
5079	0\\
5080	0\\
5081	0\\
5082	0\\
5083	0\\
5084	0\\
5085	0\\
5086	0\\
5087	0\\
5088	0\\
5089	0\\
5090	0\\
5091	0\\
5092	0\\
5093	0\\
5094	0\\
5095	0\\
5096	0\\
5097	0\\
5098	0\\
5099	0\\
5100	0\\
5101	0\\
5102	0\\
5103	0\\
5104	0\\
5105	0\\
5106	0\\
5107	0\\
5108	0\\
5109	0\\
5110	0\\
5111	0\\
5112	0\\
5113	0\\
5114	0\\
5115	0\\
5116	0\\
5117	0\\
5118	0\\
5119	0\\
5120	0\\
5121	0\\
5122	0\\
5123	0\\
5124	0\\
5125	0\\
5126	0\\
5127	0\\
5128	0\\
5129	0\\
5130	0\\
5131	0\\
5132	0\\
5133	0\\
5134	0\\
5135	0\\
5136	0\\
5137	0\\
5138	0\\
5139	0\\
5140	0\\
5141	0\\
5142	0\\
5143	0\\
5144	0\\
5145	0\\
5146	0\\
5147	0\\
5148	0\\
5149	0\\
5150	0\\
5151	0\\
5152	0\\
5153	0\\
5154	0\\
5155	0\\
5156	0\\
5157	0\\
5158	0\\
5159	0\\
5160	0\\
5161	0\\
5162	0\\
5163	0\\
5164	0\\
5165	0\\
5166	0\\
5167	0\\
5168	0\\
5169	1e-06\\
5170	1e-06\\
5171	0\\
5172	0\\
5173	0\\
5174	0\\
5175	0\\
5176	0\\
5177	0\\
5178	0\\
5179	0\\
5180	0\\
5181	0\\
5182	0\\
5183	0\\
5184	0\\
5185	0\\
5186	0\\
5187	0\\
5188	0\\
5189	0\\
5190	0\\
5191	0\\
5192	0\\
5193	0\\
5194	0\\
5195	0\\
5196	0\\
5197	0\\
5198	0\\
5199	0\\
5200	0\\
5201	0.576113\\
5202	1.102671\\
5203	1.121495\\
5204	1.121495\\
5205	1.288501\\
5206	1.840625\\
5207	0.576113\\
5208	0\\
5209	0\\
5210	0\\
5211	0\\
5212	0\\
5213	0\\
5214	0\\
5215	0\\
5216	0.21691\\
5217	0.329982\\
5218	0.237978\\
5219	0.273179\\
5220	0\\
5221	0\\
5222	0\\
5223	0\\
5224	0.321821\\
5225	0.576113\\
5226	0.576113\\
5227	0.338927\\
5228	2.4e-05\\
5229	3e-06\\
5230	0.54237\\
5231	0\\
5232	0\\
5233	0\\
5234	0\\
5235	0\\
5236	0\\
5237	0\\
5238	0\\
5239	0.576113\\
5240	1.121495\\
5241	1.970716\\
5242	1.864093\\
5243	1.504702\\
5244	1.504702\\
5245	1.288531\\
5246	1.121495\\
5247	1.067415\\
5248	0.845356\\
5249	0.931816\\
5250	1.121495\\
5251	0.773259\\
5252	1.152338\\
5253	1.484478\\
5254	1.504702\\
5255	0.576113\\
5256	0\\
5257	0\\
5258	0\\
5259	0\\
5260	0\\
5261	0\\
5262	0\\
5263	1.064417\\
5264	1.593422\\
5265	1.970716\\
5266	1.504721\\
5267	1.504702\\
5268	1.401801\\
5269	1.504702\\
5270	1.121495\\
5271	1.504702\\
5272	1.504702\\
5273	1.248369\\
5274	1.121495\\
5275	1.121495\\
5276	0.576113\\
5277	0.288254\\
5278	0.082968\\
5279	0\\
5280	0\\
5281	0\\
5282	0\\
5283	0\\
5284	0\\
5285	0\\
5286	0\\
5287	0\\
5288	0\\
5289	0\\
5290	0\\
5291	0\\
5292	0\\
5293	0\\
5294	0\\
5295	0\\
5296	0\\
5297	0\\
5298	0\\
5299	0\\
5300	0\\
5301	0\\
5302	0\\
5303	0\\
5304	0\\
5305	0\\
5306	0\\
5307	0\\
5308	0\\
5309	0\\
5310	0\\
5311	0\\
5312	0\\
5313	0\\
5314	0\\
5315	0\\
5316	0\\
5317	0\\
5318	0\\
5319	0\\
5320	0\\
5321	0\\
5322	0\\
5323	0\\
5324	0\\
5325	0\\
5326	0\\
5327	0\\
5328	0\\
5329	0\\
5330	0\\
5331	0\\
5332	0\\
5333	0\\
5334	0\\
5335	0\\
5336	0\\
5337	0\\
5338	0\\
5339	0\\
5340	0\\
5341	0\\
5342	0\\
5343	0\\
5344	0\\
5345	0\\
5346	0\\
5347	0\\
5348	0\\
5349	0\\
5350	0\\
5351	0\\
5352	0\\
5353	0\\
5354	0\\
5355	0\\
5356	0\\
5357	0\\
5358	0\\
5359	0\\
5360	0\\
5361	0\\
5362	0\\
5363	0\\
5364	0\\
5365	0\\
5366	0\\
5367	0\\
5368	0\\
5369	0\\
5370	0\\
5371	0\\
5372	0\\
5373	0\\
5374	0.351585\\
5375	0\\
5376	0\\
5377	0\\
5378	0\\
5379	0\\
5380	0\\
5381	0\\
5382	0\\
5383	2e-06\\
5384	1.121139\\
5385	1.121495\\
5386	1.121495\\
5387	1.268879\\
5388	0.903267\\
5389	0.576113\\
5390	0.351586\\
5391	0.226428\\
5392	0.139887\\
5393	0.351565\\
5394	0\\
5395	0.576113\\
5396	0.576113\\
5397	0.576113\\
5398	0.926116\\
5399	0\\
5400	0\\
5401	0\\
5402	0\\
5403	0\\
5404	0\\
5405	0\\
5406	0\\
5407	1.121495\\
5408	1.970716\\
5409	1.970716\\
5410	2.087288\\
5411	2.110298\\
5412	1.970716\\
5413	1.970716\\
5414	1.530774\\
5415	1.504702\\
5416	1.588953\\
5417	1.606825\\
5418	1.970716\\
5419	1.743497\\
5420	1.504702\\
5421	1.504732\\
5422	1.504637\\
5423	0.926115\\
5424	0\\
5425	0\\
5426	0\\
5427	0\\
5428	0\\
5429	0\\
5430	0\\
5431	0\\
5432	0\\
5433	0\\
5434	0\\
5435	0\\
5436	0\\
5437	0\\
5438	0\\
5439	0\\
5440	0\\
5441	0\\
5442	0\\
5443	0\\
5444	0\\
5445	0\\
5446	0\\
5447	0\\
5448	0\\
5449	0\\
5450	0\\
5451	0\\
5452	0\\
5453	0\\
5454	0\\
5455	0\\
5456	0\\
5457	0\\
5458	0\\
5459	0\\
5460	0\\
5461	0\\
5462	0\\
5463	0\\
5464	0\\
5465	0\\
5466	0\\
5467	0\\
5468	0\\
5469	0\\
5470	0\\
5471	0\\
5472	0\\
5473	0\\
5474	0\\
5475	0\\
5476	0\\
5477	0\\
5478	0\\
5479	0\\
5480	0\\
5481	0\\
5482	0\\
5483	0\\
5484	0\\
5485	0\\
5486	0\\
5487	0\\
5488	0\\
5489	0\\
5490	0\\
5491	0\\
5492	0\\
5493	0\\
5494	0\\
5495	0\\
5496	0\\
5497	0\\
5498	0\\
5499	0\\
5500	0\\
5501	0\\
5502	0\\
5503	0\\
5504	4e-06\\
5505	0\\
5506	0\\
5507	0\\
5508	0\\
5509	0\\
5510	0\\
5511	1.2e-05\\
5512	0.356395\\
5513	1.121495\\
5514	1.504702\\
5515	1.970716\\
5516	1.970716\\
5517	1.970716\\
5518	1.970716\\
5519	1.504702\\
5520	0.223969\\
5521	0\\
5522	0\\
5523	0\\
5524	0\\
5525	0\\
5526	0.4144\\
5527	1.121495\\
5528	1.970716\\
5529	1.970716\\
5530	1.970716\\
5531	1.840625\\
5532	1.338348\\
5533	0.576113\\
5534	0.899436\\
5535	0.576113\\
5536	0.924883\\
5537	1.119321\\
5538	1.970716\\
5539	1.970716\\
5540	1.970716\\
5541	1.970716\\
5542	1.970716\\
5543	1.504702\\
5544	0.576113\\
5545	0\\
5546	0\\
5547	0\\
5548	0\\
5549	0\\
5550	1.504702\\
5551	2.08137\\
5552	2.135591\\
5553	2.31866\\
5554	1.970716\\
5555	1.970716\\
5556	1.721619\\
5557	1.634908\\
5558	1.455506\\
5559	1.802038\\
5560	1.970716\\
5561	2.358365\\
5562	2.112947\\
5563	2.265378\\
5564	2.33206\\
5565	3.21295\\
5566	2.903523\\
5567	1.970716\\
5568	0.641762\\
5569	0\\
5570	0\\
5571	0\\
5572	0\\
5573	0\\
5574	1.504702\\
5575	2.041644\\
5576	2.460061\\
5577	1.970716\\
5578	1.840624\\
5579	1.769525\\
5580	1.970716\\
5581	1.504706\\
5582	1.970716\\
5583	1.970716\\
5584	1.970716\\
5585	1.844117\\
5586	1.970716\\
5587	1.970716\\
5588	1.970716\\
5589	1.508601\\
5590	1.504702\\
5591	1.12149\\
5592	0\\
5593	0\\
5594	0\\
5595	0\\
5596	0\\
5597	0\\
5598	1e-06\\
5599	0.795157\\
5600	1.970716\\
5601	1.504702\\
5602	1.504702\\
5603	1.5518\\
5604	1.241146\\
5605	1.755208\\
5606	1.504702\\
5607	1.504702\\
5608	1.504702\\
5609	1.504702\\
5610	1.504702\\
5611	1.970716\\
5612	1.970716\\
5613	1.970716\\
5614	1.976021\\
5615	1.803999\\
5616	0.3506\\
5617	0\\
5618	0.679275\\
5619	0.351585\\
5620	1e-06\\
5621	0\\
5622	0.17256\\
5623	0.576113\\
5624	0.991159\\
5625	0.576111\\
5626	0.574795\\
5627	0.000203\\
5628	1e-06\\
5629	0\\
5630	0\\
5631	0\\
5632	0\\
5633	0.351578\\
5634	1.121495\\
5635	1.583036\\
5636	1.970716\\
5637	1.970716\\
5638	1.840626\\
5639	1.504702\\
5640	0.576113\\
5641	0\\
5642	0\\
5643	0\\
5644	0\\
5645	0\\
5646	0\\
5647	0\\
5648	0\\
5649	0\\
5650	0\\
5651	0\\
5652	0\\
5653	0\\
5654	0\\
5655	0\\
5656	0\\
5657	0\\
5658	0\\
5659	0\\
5660	0.576113\\
5661	1.240356\\
5662	1.047399\\
5663	0.518683\\
5664	0\\
5665	0\\
5666	0\\
5667	0\\
5668	0\\
5669	0\\
5670	0.870047\\
5671	1.674739\\
5672	2.110481\\
5673	2.758742\\
5674	3.218433\\
5675	3.224237\\
5676	3.42155\\
5677	3.42155\\
5678	3.42155\\
5679	3.782072\\
5680	4.173354\\
5681	3.630294\\
5682	3.42155\\
5683	3.218433\\
5684	2.510508\\
5685	1.970716\\
5686	1.970716\\
5687	1.073973\\
5688	0\\
5689	0\\
5690	0\\
5691	0\\
5692	0\\
5693	0\\
5694	1.540439\\
5695	2.489969\\
5696	3.329814\\
5697	4.794764\\
5698	5.378813\\
5699	6.014299\\
5700	4.907937\\
5701	4.457156\\
5702	3.810988\\
5703	4.717592\\
5704	4.576995\\
5705	4.710875\\
5706	3.577708\\
5707	3.42155\\
5708	3.224237\\
5709	2.944523\\
5710	2.294953\\
5711	1.970716\\
5712	0.199994\\
5713	0\\
5714	0\\
5715	0\\
5716	0\\
5717	0\\
5718	1.647015\\
5719	1.972288\\
5720	3.315514\\
5721	3.224237\\
5722	2.838511\\
5723	2.223253\\
5724	1.970716\\
5725	1.970716\\
5726	1.970716\\
5727	1.970716\\
5728	2.11048\\
5729	2.351238\\
5730	3.224237\\
5731	3.224237\\
5732	2.857782\\
5733	3.193821\\
5734	3.141166\\
5735	1.970716\\
5736	0.2051\\
5737	0\\
5738	0\\
5739	0\\
5740	0\\
5741	0\\
5742	0.926116\\
5743	1.970716\\
5744	3.122465\\
5745	3.224237\\
5746	3.316643\\
5747	3.279485\\
5748	3.083978\\
5749	2.912116\\
5750	2.582703\\
5751	3.218433\\
5752	2.708482\\
5753	2.282772\\
5754	2.886469\\
5755	2.395327\\
5756	2.170174\\
5757	1.970716\\
5758	1.970716\\
5759	1.121495\\
5760	0\\
5761	0\\
5762	0\\
5763	0\\
5764	0\\
5765	0\\
5766	0.926116\\
5767	1.970716\\
5768	2.11048\\
5769	1.970716\\
5770	1.970716\\
5771	1.540775\\
5772	1.173639\\
5773	0.685383\\
5774	0.576113\\
5775	1e-06\\
5776	1e-06\\
5777	0.478262\\
5778	1.10258\\
5779	1.073375\\
5780	1.121495\\
5781	0.601897\\
5782	0.576114\\
5783	0\\
5784	0\\
5785	0\\
5786	0\\
5787	0\\
5788	0\\
5789	0\\
5790	0\\
5791	0\\
5792	0.260711\\
5793	1.121495\\
5794	1.970716\\
5795	1.970716\\
5796	1.970716\\
5797	1.970716\\
5798	1.790777\\
5799	1.775705\\
5800	1.961007\\
5801	1.970716\\
5802	2.289263\\
5803	2.395328\\
5804	2.11048\\
5805	1.970716\\
5806	1.970716\\
5807	1.504702\\
5808	0.576113\\
5809	0\\
5810	0\\
5811	0\\
5812	0\\
5813	0\\
5814	0\\
5815	0\\
5816	0.286027\\
5817	1.121495\\
5818	1.504702\\
5819	1.865336\\
5820	1.504702\\
5821	0.811116\\
5822	1e-06\\
5823	0\\
5824	0\\
5825	0.356876\\
5826	1.504126\\
5827	2.06555\\
5828	1.971063\\
5829	2.016098\\
5830	2.32324\\
5831	1.970716\\
5832	0\\
5833	0\\
5834	0\\
5835	0\\
5836	0\\
5837	0\\
5838	1.093288\\
5839	2.347652\\
5840	2.347652\\
5841	2.347652\\
5842	2.218066\\
5843	1.835477\\
5844	1.500101\\
5845	1.511326\\
5846	1.835483\\
5847	2.151498\\
5848	2.347652\\
5849	2.347652\\
5850	2.347652\\
5851	2.853476\\
5852	2.347652\\
5853	2.347652\\
5854	2.347652\\
5855	1.269953\\
5856	0\\
5857	0\\
5858	0\\
5859	0\\
5860	0\\
5861	0\\
5862	1.336001\\
5863	2.780283\\
5864	2.853477\\
5865	2.514901\\
5866	2.347652\\
5867	2.347652\\
5868	1.818247\\
5869	1.575055\\
5870	1.734837\\
5871	1.539401\\
5872	2.1878\\
5873	2.347652\\
5874	2.347652\\
5875	2.347652\\
5876	2.347652\\
5877	2.347652\\
5878	2.347652\\
5879	1.336001\\
5880	0\\
5881	0\\
5882	0\\
5883	0\\
5884	0\\
5885	0\\
5886	1.336001\\
5887	2.853477\\
5888	2.857254\\
5889	2.514147\\
5890	2.347652\\
5891	2.088423\\
5892	1.111782\\
5893	0.42498\\
5894	0.04156\\
5895	0.187913\\
5896	1.270523\\
5897	2.192678\\
5898	2.347652\\
5899	2.347652\\
5900	2.347652\\
5901	2.347652\\
5902	2.347652\\
5903	6e-06\\
5904	0\\
5905	0\\
5906	0\\
5907	0\\
5908	0\\
5909	0\\
5910	1.336001\\
5911	2.870813\\
5912	2.961918\\
5913	2.853477\\
5914	2.347652\\
5915	2.347652\\
5916	1.845418\\
5917	1.48264\\
5918	1.336001\\
5919	1.581171\\
5920	2.347652\\
5921	2.347652\\
5922	3.650284\\
5923	3.642845\\
5924	3.840931\\
5925	3.297378\\
5926	2.754128\\
5927	1.649637\\
5928	0.001551\\
5929	0\\
5930	0\\
5931	0\\
5932	0\\
5933	0\\
5934	1.607934\\
5935	3.423292\\
5936	4.070265\\
5937	4.075984\\
5938	3.840931\\
5939	3.72509\\
5940	2.347652\\
5941	2.347652\\
5942	2.347652\\
5943	2.347652\\
5944	2.347652\\
5945	2.347652\\
5946	2.514155\\
5947	2.347652\\
5948	2.440048\\
5949	2.347652\\
5950	2.19268\\
5951	2.190864\\
5952	2.347652\\
5953	0.604296\\
5954	0\\
5955	0\\
5956	0\\
5957	0\\
5958	1e-06\\
5959	0.418423\\
5960	1.529734\\
5961	2.347652\\
5962	2.853477\\
5963	3.500055\\
5964	2.990014\\
5965	2.347652\\
5966	2.347652\\
5967	2.347652\\
5968	2.347652\\
5969	2.876297\\
5970	2.779736\\
5971	2.812629\\
5972	3.46118\\
5973	2.853477\\
5974	2.815104\\
5975	2.347652\\
5976	2.192678\\
5977	1.336001\\
5978	1.207896\\
5979	1.009029\\
5980	0.69702\\
5981	0.945863\\
5982	1.336792\\
5983	1.336001\\
5984	2.050214\\
5985	2.347651\\
5986	2.347652\\
5987	2.347652\\
5988	2.282586\\
5989	1.336001\\
5990	0.713021\\
5991	1.036139\\
5992	1e-06\\
5993	0\\
5994	1.051577\\
5995	1.795148\\
5996	2.347652\\
5997	2.347652\\
5998	2.192678\\
5999	0.521174\\
6000	0\\
6001	0\\
6002	0\\
6003	0\\
6004	0\\
6005	0\\
6006	2.347652\\
6007	4.075984\\
6008	4.264819\\
6009	4.075984\\
6010	4.075985\\
6011	4.075985\\
6012	3.31811\\
6013	2.507155\\
6014	2.347652\\
6015	2.347652\\
6016	2.779735\\
6017	3.840931\\
6018	3.840949\\
6019	3.840931\\
6020	3.464111\\
6021	2.869841\\
6022	2.347652\\
6023	2.34765\\
6024	0.418833\\
6025	1e-06\\
6026	0\\
6027	0\\
6028	0\\
6029	0\\
6030	2.347652\\
6031	3.783624\\
6032	3.764547\\
6033	2.461598\\
6034	2.853481\\
6035	3.357069\\
6036	2.347652\\
6037	2.347652\\
6038	2.347652\\
6039	2.347652\\
6040	2.796483\\
6041	3.834018\\
6042	3.841771\\
6043	3.039129\\
6044	3.525386\\
6045	2.806827\\
6046	2.514149\\
6047	1.597738\\
6048	1e-06\\
6049	1e-06\\
6050	0\\
6051	0\\
6052	0\\
6053	0\\
6054	2.347652\\
6055	3.840931\\
6056	3.968623\\
6057	3.240118\\
6058	2.679164\\
6059	2.347652\\
6060	2.347652\\
6061	2.347651\\
6062	2.026326\\
6063	2.347652\\
6064	2.347652\\
6065	2.347652\\
6066	3.54114\\
6067	2.779725\\
6068	2.879223\\
6069	3.248277\\
6070	2.347652\\
6071	1.396789\\
6072	0\\
6073	0\\
6074	0\\
6075	0\\
6076	0\\
6077	0\\
6078	2.347652\\
6079	3.586268\\
6080	3.758472\\
6081	3.840931\\
6082	3.82198\\
6083	3.840931\\
6084	2.8532\\
6085	2.347652\\
6086	2.347652\\
6087	2.347652\\
6088	2.347652\\
6089	2.941226\\
6090	3.212967\\
6091	2.747879\\
6092	2.811126\\
6093	2.779734\\
6094	2.347652\\
6095	1.336001\\
6096	0\\
6097	0\\
6098	0\\
6099	0\\
6100	0\\
6101	0\\
6102	2.347652\\
6103	2.85111\\
6104	3.688322\\
6105	2.851399\\
6106	2.347652\\
6107	1.570017\\
6108	1.103261\\
6109	0.696759\\
6110	0.512119\\
6111	0.171593\\
6112	1.10325\\
6113	2.030615\\
6114	2.347652\\
6115	2.347652\\
6116	2.853477\\
6117	2.347652\\
6118	2.347652\\
6119	1.336001\\
6120	0\\
6121	0\\
6122	0\\
6123	0\\
6124	0\\
6125	0\\
6126	0\\
6127	0\\
6128	0.300914\\
6129	0.764642\\
6130	0.373367\\
6131	1e-06\\
6132	0\\
6133	0\\
6134	0\\
6135	0\\
6136	0\\
6137	0\\
6138	0\\
6139	2e-06\\
6140	0\\
6141	0\\
6142	0\\
6143	0\\
6144	0\\
6145	0\\
6146	0\\
6147	0\\
6148	0\\
6149	0\\
6150	0\\
6151	0\\
6152	0\\
6153	1e-06\\
6154	0.322105\\
6155	0.599919\\
6156	4e-06\\
6157	0\\
6158	0\\
6159	0\\
6160	0\\
6161	0\\
6162	0\\
6163	0\\
6164	0.5978\\
6165	1.4e-05\\
6166	0\\
6167	0\\
6168	0\\
6169	0\\
6170	0\\
6171	0\\
6172	0\\
6173	0\\
6174	1.336001\\
6175	2.853477\\
6176	3.625754\\
6177	3.834016\\
6178	3.720456\\
6179	3.203208\\
6180	2.347652\\
6181	2.347652\\
6182	2.347652\\
6183	2.347652\\
6184	2.475178\\
6185	2.985426\\
6186	3.840931\\
6187	3.840931\\
6188	3.840931\\
6189	2.990591\\
6190	2.347652\\
6191	1.380351\\
6192	0\\
6193	0\\
6194	0\\
6195	0\\
6196	0\\
6197	0\\
6198	2.347652\\
6199	3.840931\\
6200	4.075984\\
6201	3.896063\\
6202	3.204403\\
6203	2.967015\\
6204	2.347652\\
6205	2.347652\\
6206	2.347652\\
6207	2.347652\\
6208	2.779751\\
6209	3.834018\\
6210	3.935603\\
6211	3.840931\\
6212	4.075984\\
6213	3.814817\\
6214	2.779046\\
6215	2.192701\\
6216	0\\
6217	0\\
6218	0\\
6219	0\\
6220	0\\
6221	0\\
6222	2.347652\\
6223	3.663003\\
6224	3.634459\\
6225	2.852694\\
6226	2.759623\\
6227	2.623019\\
6228	2.347652\\
6229	1.836612\\
6230	1.773363\\
6231	2.267987\\
6232	2.347652\\
6233	2.347652\\
6234	3.32301\\
6235	3.390913\\
6236	3.83699\\
6237	2.783404\\
6238	2.347652\\
6239	1.297735\\
6240	0\\
6241	0\\
6242	0\\
6243	0\\
6244	0\\
6245	0\\
6246	2.192678\\
6247	3.81392\\
6248	4.075984\\
6249	4.78775\\
6250	5.024695\\
6251	6.415296\\
6252	4.281886\\
6253	4.075985\\
6254	4.075984\\
6255	3.840931\\
6256	3.840931\\
6257	4.471077\\
6258	4.191207\\
6259	3.961086\\
6260	4.075984\\
6261	3.387628\\
6262	2.347652\\
6263	1.736196\\
6264	0.185365\\
6265	0\\
6266	0\\
6267	0\\
6268	0\\
6269	0\\
6270	2.347652\\
6271	4.050658\\
6272	4.108465\\
6273	4.74351\\
6274	4.191609\\
6275	4.075984\\
6276	3.840931\\
6277	3.435163\\
6278	3.820696\\
6279	3.530669\\
6280	3.759254\\
6281	3.840931\\
6282	3.840931\\
6283	2.920084\\
6284	3.834018\\
6285	2.593533\\
6286	2.347652\\
6287	2.192678\\
6288	7.5e-05\\
6289	0\\
6290	0\\
6291	0\\
6292	0\\
6293	0\\
6294	0\\
6295	0\\
6296	1.1e-05\\
6297	1.336001\\
6298	1.103252\\
6299	0.737804\\
6300	0.359284\\
6301	0.228466\\
6302	1e-06\\
6303	0\\
6304	0\\
6305	0.418833\\
6306	1.329695\\
6307	1.145665\\
6308	1.835489\\
6309	0.596203\\
6310	0.397927\\
6311	0\\
6312	0\\
6313	0\\
6314	0\\
6315	0\\
6316	0\\
6317	0\\
6318	0\\
6319	0\\
6320	0\\
6321	0\\
6322	0\\
6323	0\\
6324	0\\
6325	0\\
6326	0\\
6327	0\\
6328	0\\
6329	0\\
6330	0\\
6331	0\\
6332	0\\
6333	0\\
6334	0\\
6335	0\\
6336	0\\
6337	0\\
6338	0\\
6339	0\\
6340	0\\
6341	0\\
6342	1.006372\\
6343	2.347652\\
6344	2.882589\\
6345	3.518134\\
6346	2.990552\\
6347	2.853477\\
6348	2.347652\\
6349	2.347652\\
6350	2.347652\\
6351	2.347652\\
6352	2.347652\\
6353	2.347652\\
6354	3.054119\\
6355	3.261092\\
6356	3.497364\\
6357	2.853477\\
6358	2.347652\\
6359	1.336001\\
6360	0\\
6361	0\\
6362	0\\
6363	0\\
6364	0\\
6365	0\\
6366	2.347652\\
6367	4.075984\\
6368	4.075984\\
6369	3.840931\\
6370	3.730336\\
6371	2.95134\\
6372	2.347652\\
6373	2.347652\\
6374	2.347652\\
6375	2.347652\\
6376	2.81912\\
6377	3.812974\\
6378	3.840931\\
6379	4.075985\\
6380	4.414097\\
6381	3.585608\\
6382	2.665177\\
6383	1.909708\\
6384	1e-06\\
6385	0\\
6386	0\\
6387	0\\
6388	0\\
6389	0\\
6390	1.336001\\
6391	2.853477\\
6392	3.13137\\
6393	3.301381\\
6394	2.853478\\
6395	3.834008\\
6396	3.124758\\
6397	3.083853\\
6398	2.853382\\
6399	2.779735\\
6400	2.702713\\
6401	3.467363\\
6402	2.853477\\
6403	2.778957\\
6404	2.930106\\
6405	2.347652\\
6406	2.202159\\
6407	0.447718\\
6408	0\\
6409	0\\
6410	0\\
6411	0\\
6412	0\\
6413	0\\
6414	2.199097\\
6415	3.840931\\
6416	3.840931\\
6417	3.840931\\
6418	3.325157\\
6419	2.853477\\
6420	2.347652\\
6421	2.034438\\
6422	1.485203\\
6423	1.425254\\
6424	1.835759\\
6425	2.347652\\
6426	2.514084\\
6427	2.776798\\
6428	3.128353\\
6429	2.347652\\
6430	1.336001\\
6431	0.282478\\
6432	0\\
6433	0\\
6434	0\\
6435	0\\
6436	0\\
6437	0\\
6438	1.934367\\
6439	3.834018\\
6440	3.834018\\
6441	3.840931\\
6442	3.669765\\
6443	3.758773\\
6444	3.791064\\
6445	3.503453\\
6446	3.834018\\
6447	3.840931\\
6448	4.075984\\
6449	4.711993\\
6450	4.824363\\
6451	5.012128\\
6452	4.075985\\
6453	3.517061\\
6454	3.323351\\
6455	2.347652\\
6456	1.336001\\
6457	1.336001\\
6458	0.418833\\
6459	1e-05\\
6460	0\\
6461	1e-06\\
6462	0.840224\\
6463	2.04081\\
6464	2.347652\\
6465	2.853574\\
6466	3.468036\\
6467	2.993511\\
6468	2.347652\\
6469	2.251695\\
6470	1.359728\\
6471	1.336001\\
6472	1.763095\\
6473	2.347652\\
6474	2.725439\\
6475	2.990595\\
6476	3.598105\\
6477	2.853477\\
6478	2.347652\\
6479	1.919483\\
6480	0.26237\\
6481	0\\
6482	0\\
6483	0\\
6484	0\\
6485	0\\
6486	0\\
6487	0\\
6488	0\\
6489	0\\
6490	0\\
6491	0\\
6492	0\\
6493	0\\
6494	0\\
6495	0\\
6496	0\\
6497	0\\
6498	1.404158\\
6499	3.131715\\
6500	3.834017\\
6501	2.347652\\
6502	1.336001\\
6503	6e-06\\
6504	0\\
6505	0\\
6506	0\\
6507	0\\
6508	0\\
6509	0\\
6510	2.108922\\
6511	3.833974\\
6512	3.840931\\
6513	3.697261\\
6514	3.068082\\
6515	3.021689\\
6516	2.347652\\
6517	2.347652\\
6518	2.514148\\
6519	3.949182\\
6520	4.354794\\
6521	6.723788\\
6522	7.539637\\
6523	6.803105\\
6524	5.6132\\
6525	4.102467\\
6526	3.628162\\
6527	2.347652\\
6528	1.014053\\
6529	0\\
6530	0\\
6531	0\\
6532	0\\
6533	0\\
6534	3.82241\\
6535	6.451505\\
6536	7.21299\\
6537	7.509032\\
6538	6.315264\\
6539	5.344655\\
6540	4.595278\\
6541	4.47604\\
6542	3.84093\\
6543	3.840931\\
6544	4.012381\\
6545	5.468312\\
6546	6.409675\\
6547	7.031753\\
6548	6.820594\\
6549	5.186665\\
6550	3.846923\\
6551	2.477037\\
6552	0.334971\\
6553	0\\
6554	0\\
6555	0\\
6556	0\\
6557	0\\
6558	2.417835\\
6559	4.197836\\
6560	3.948638\\
6561	3.642268\\
6562	2.715909\\
6563	2.938782\\
6564	2.417835\\
6565	2.417835\\
6566	2.417835\\
6567	2.417835\\
6568	2.417871\\
6569	3.949671\\
6570	4.197836\\
6571	4.154065\\
6572	3.955756\\
6573	3.955756\\
6574	3.721476\\
6575	2.417835\\
6576	1.130122\\
6577	1e-06\\
6578	0\\
6579	0\\
6580	0\\
6581	0.340997\\
6582	2.862731\\
6583	5.360661\\
6584	5.621627\\
6585	5.687439\\
6586	5.168222\\
6587	5.236255\\
6588	4.37108\\
6589	4.197836\\
6590	4.382036\\
6591	4.847044\\
6592	5.154687\\
6593	5.694163\\
6594	5.497961\\
6595	5.307567\\
6596	5.061574\\
6597	4.209098\\
6598	3.948635\\
6599	2.417835\\
6600	1.172254\\
6601	2e-06\\
6602	0\\
6603	0\\
6604	0\\
6605	0.431353\\
6606	2.709334\\
6607	5.887218\\
6608	5.095533\\
6609	4.198951\\
6610	3.948635\\
6611	3.955756\\
6612	2.417835\\
6613	2.417835\\
6614	2.417835\\
6615	2.417835\\
6616	2.851418\\
6617	3.948635\\
6618	3.955756\\
6619	3.955756\\
6620	4.197836\\
6621	3.948635\\
6622	2.899575\\
6623	2.417835\\
6624	0.041922\\
6625	0\\
6626	1e-06\\
6627	0\\
6628	0\\
6629	0\\
6630	0\\
6631	0\\
6632	0\\
6633	0\\
6634	0\\
6635	0\\
6636	0\\
6637	1e-06\\
6638	0\\
6639	0\\
6640	0\\
6641	0\\
6642	0.350915\\
6643	6.6e-05\\
6644	0.431264\\
6645	0\\
6646	0\\
6647	0\\
6648	0\\
6649	0\\
6650	0\\
6651	0\\
6652	0\\
6653	0\\
6654	0\\
6655	0\\
6656	0\\
6657	0\\
6658	0.000436\\
6659	0.657602\\
6660	0.626552\\
6661	1.4e-05\\
6662	0\\
6663	0\\
6664	0\\
6665	1.181627\\
6666	2.417835\\
6667	2.417835\\
6668	2.417835\\
6669	2.417835\\
6670	1.890365\\
6671	0.653572\\
6672	0\\
6673	0\\
6674	0\\
6675	0\\
6676	0\\
6677	0\\
6678	2.417835\\
6679	4.439847\\
6680	4.197836\\
6681	4.36155\\
6682	3.955756\\
6683	3.945145\\
6684	2.938782\\
6685	3.807269\\
6686	3.691957\\
6687	2.589308\\
6688	2.938782\\
6689	3.750082\\
6690	3.726836\\
6691	2.417835\\
6692	2.417835\\
6693	1.37594\\
6694	0.36489\\
6695	0\\
6696	0\\
6697	0\\
6698	0\\
6699	0\\
6700	0\\
6701	0\\
6702	0\\
6703	2.417835\\
6704	2.417835\\
6705	1.890261\\
6706	1.344956\\
6707	1.375941\\
6708	0\\
6709	2e-06\\
6710	0\\
6711	0\\
6712	0\\
6713	1.132416\\
6714	2.417835\\
6715	2.417835\\
6716	2.417835\\
6717	1.673204\\
6718	0.851022\\
6719	0\\
6720	0\\
6721	0\\
6722	0\\
6723	0\\
6724	0\\
6725	0\\
6726	0.799463\\
6727	2.417835\\
6728	2.417835\\
6729	3.398687\\
6730	3.146186\\
6731	3.672485\\
6732	3.94862\\
6733	3.028304\\
6734	2.417835\\
6735	2.417835\\
6736	2.043374\\
6737	2.417835\\
6738	2.417835\\
6739	2.417835\\
6740	2.417835\\
6741	1.375941\\
6742	1.136231\\
6743	0\\
6744	0\\
6745	0\\
6746	0\\
6747	0\\
6748	0\\
6749	0\\
6750	0.431354\\
6751	2.417835\\
6752	2.938782\\
6753	2.417835\\
6754	2.417835\\
6755	2.417835\\
6756	0.298111\\
6757	0.431354\\
6758	1.293149\\
6759	1.376917\\
6760	1.790937\\
6761	2.417835\\
6762	2.417835\\
6763	2.417835\\
6764	2.417835\\
6765	1.712004\\
6766	0.789145\\
6767	0\\
6768	0\\
6769	0\\
6770	0\\
6771	0\\
6772	0\\
6773	0\\
6774	1.042664\\
6775	3.142196\\
6776	2.589991\\
6777	2.417835\\
6778	2.417835\\
6779	2.794866\\
6780	2.230703\\
6781	1.932243\\
6782	1.988603\\
6783	1.584667\\
6784	2.226994\\
6785	2.417835\\
6786	3.078947\\
6787	3.85876\\
6788	3.955756\\
6789	2.814228\\
6790	2.417835\\
6791	2.417835\\
6792	1e-05\\
6793	0\\
6794	0\\
6795	0\\
6796	0\\
6797	0\\
6798	0\\
6799	1e-06\\
6800	1.136234\\
6801	1.375941\\
6802	1.586852\\
6803	1.375941\\
6804	1.108873\\
6805	0.647742\\
6806	0\\
6807	0\\
6808	1.3e-05\\
6809	0.687824\\
6810	1.165696\\
6811	2.177137\\
6812	1.375941\\
6813	0.73758\\
6814	0\\
6815	0\\
6816	0\\
6817	0\\
6818	0\\
6819	0\\
6820	0\\
6821	0\\
6822	0\\
6823	0\\
6824	0\\
6825	0\\
6826	0\\
6827	0\\
6828	0\\
6829	0\\
6830	0\\
6831	0\\
6832	0\\
6833	0\\
6834	0\\
6835	1e-06\\
6836	0\\
6837	0\\
6838	0\\
6839	0\\
6840	0\\
6841	0\\
6842	0\\
6843	0\\
6844	0\\
6845	0\\
6846	0.656441\\
6847	2.417835\\
6848	2.377043\\
6849	1.375941\\
6850	1.375941\\
6851	1.815715\\
6852	1.375941\\
6853	1.375941\\
6854	1.375941\\
6855	1.375941\\
6856	1.581389\\
6857	2.417835\\
6858	2.417835\\
6859	2.589308\\
6860	2.417835\\
6861	1.890349\\
6862	1.478708\\
6863	5e-06\\
6864	0\\
6865	0\\
6866	0\\
6867	0\\
6868	0\\
6869	0\\
6870	0.431354\\
6871	2.938782\\
6872	2.417835\\
6873	2.417835\\
6874	1.386328\\
6875	1.49307\\
6876	0.431354\\
6877	1.229024\\
6878	1.375941\\
6879	1.752946\\
6880	2.417835\\
6881	2.938781\\
6882	3.948633\\
6883	4.105603\\
6884	3.948636\\
6885	2.686891\\
6886	2.417835\\
6887	1.481864\\
6888	1.6e-05\\
6889	0\\
6890	0\\
6891	0\\
6892	1e-06\\
6893	1.189537\\
6894	3.634479\\
6895	13.097841\\
6896	4.197836\\
6897	3.955756\\
6898	3.955756\\
6899	3.955756\\
6900	3.343505\\
6901	3.221702\\
6902	3.079992\\
6903	3.425842\\
6904	3.948636\\
6905	4.272857\\
6906	5.511238\\
6907	12.438647\\
6908	4.75256\\
6909	3.877659\\
6910	2.938782\\
6911	2.417835\\
6912	0\\
6913	0\\
6914	0\\
6915	0\\
6916	0\\
6917	0\\
6918	1.375941\\
6919	3.948636\\
6920	3.569374\\
6921	3.831578\\
6922	3.732647\\
6923	3.127489\\
6924	2.724266\\
6925	2.417835\\
6926	2.417835\\
6927	2.417835\\
6928	2.417835\\
6929	3.945684\\
6930	4.320704\\
6931	4.16835\\
6932	3.955756\\
6933	3.07999\\
6934	2.722452\\
6935	2.417835\\
6936	1e-06\\
6937	0\\
6938	0\\
6939	0\\
6940	0\\
6941	0\\
6942	2.359687\\
6943	4.053457\\
6944	4.197836\\
6945	3.955756\\
6946	3.948636\\
6947	2.915463\\
6948	2.417835\\
6949	2.417835\\
6950	2.32992\\
6951	2.417835\\
6952	2.417835\\
6953	2.862834\\
6954	3.955756\\
6955	3.955756\\
6956	3.397409\\
6957	2.417835\\
6958	2.257904\\
6959	0.951278\\
6960	0\\
6961	0\\
6962	0\\
6963	0\\
6964	0\\
6965	0\\
6966	0\\
6967	0\\
6968	0\\
6969	0\\
6970	0\\
6971	0\\
6972	0\\
6973	0\\
6974	0\\
6975	0\\
6976	0\\
6977	0\\
6978	1.653031\\
6979	1.375941\\
6980	0\\
6981	0\\
6982	0\\
6983	0\\
6984	0\\
6985	0\\
6986	0\\
6987	0\\
6988	0\\
6989	0\\
6990	0\\
6991	0\\
6992	0\\
6993	0\\
6994	0\\
6995	0\\
6996	0\\
6997	0\\
6998	0\\
6999	0\\
7000	0\\
7001	0\\
7002	0\\
7003	0\\
7004	0\\
7005	0\\
7006	0\\
7007	0\\
7008	0\\
7009	0\\
7010	0\\
7011	0\\
7012	0\\
7013	0\\
7014	0\\
7015	2.417835\\
7016	2.550169\\
7017	2.417839\\
7018	2.417835\\
7019	2.417835\\
7020	1.981046\\
7021	2.220141\\
7022	2.053777\\
7023	2.053834\\
7024	2.355521\\
7025	2.417835\\
7026	2.417835\\
7027	2.116012\\
7028	1.375941\\
7029	0\\
7030	0\\
7031	0\\
7032	0\\
7033	0\\
7034	0\\
7035	0\\
7036	0\\
7037	0\\
7038	0\\
7039	1.375941\\
7040	1.9594\\
7041	1.890027\\
7042	1.773251\\
7043	2.417835\\
7044	1.375941\\
7045	1.375941\\
7046	1.045865\\
7047	0.304419\\
7048	0.182246\\
7049	1.375941\\
7050	1.25697\\
7051	0.580131\\
7052	0\\
7053	0\\
7054	0\\
7055	0\\
7056	0\\
7057	0\\
7058	0\\
7059	0\\
7060	0\\
7061	0\\
7062	0\\
7063	1.375942\\
7064	1.37594\\
7065	0.622358\\
7066	3e-06\\
7067	2e-06\\
7068	0\\
7069	0\\
7070	0\\
7071	0\\
7072	0.000534\\
7073	1.136234\\
7074	2.456183\\
7075	2.417835\\
7076	2.417835\\
7077	1.375935\\
7078	0.045161\\
7079	0\\
7080	0\\
7081	0\\
7082	0\\
7083	0\\
7084	0\\
7085	0\\
7086	0\\
7087	2.045065\\
7088	1.655144\\
7089	1.375941\\
7090	1.136232\\
7091	1.261663\\
7092	0.207771\\
7093	0.533326\\
7094	0.142925\\
7095	0.950032\\
7096	1.067295\\
7097	1.375941\\
7098	1.220863\\
7099	1.37593\\
7100	0\\
7101	0\\
7102	0\\
7103	0\\
7104	0\\
7105	0\\
7106	0\\
7107	0\\
7108	0\\
7109	0\\
7110	0\\
7111	0\\
7112	5e-06\\
7113	1.136381\\
7114	1.136106\\
7115	1.344343\\
7116	1.494897\\
7117	1.078518\\
7118	0.018726\\
7119	0.155394\\
7120	0.318729\\
7121	0.655772\\
7122	2.022605\\
7123	0.26918\\
7124	0\\
7125	0\\
7126	0\\
7127	0\\
7128	0\\
7129	0\\
7130	0\\
7131	0\\
7132	0\\
7133	0\\
7134	0\\
7135	0\\
7136	3e-06\\
7137	0.431354\\
7138	0.431354\\
7139	0.959298\\
7140	1.136234\\
7141	0\\
7142	0\\
7143	0\\
7144	0\\
7145	0\\
7146	0.823702\\
7147	0\\
7148	0\\
7149	0\\
7150	0\\
7151	0\\
7152	0\\
7153	0\\
7154	0\\
7155	0\\
7156	0\\
7157	0\\
7158	0\\
7159	0\\
7160	0\\
7161	0\\
7162	0\\
7163	0\\
7164	0\\
7165	0\\
7166	0\\
7167	0\\
7168	0\\
7169	0\\
7170	0\\
7171	0\\
7172	0\\
7173	0\\
7174	0\\
7175	0\\
7176	0\\
7177	0\\
7178	0\\
7179	0\\
7180	0\\
7181	0\\
7182	0\\
7183	0\\
7184	0\\
7185	0\\
7186	0\\
7187	0\\
7188	0\\
7189	0\\
7190	0\\
7191	0\\
7192	0\\
7193	1e-06\\
7194	0.533937\\
7195	2.414969\\
7196	1.103051\\
7197	0.392781\\
7198	0\\
7199	0\\
7200	0\\
7201	0\\
7202	0\\
7203	0\\
7204	0\\
7205	0\\
7206	0\\
7207	0\\
7208	0\\
7209	0\\
7210	0\\
7211	0\\
7212	0\\
7213	0\\
7214	0\\
7215	0\\
7216	0\\
7217	0.051437\\
7218	1.464637\\
7219	2.310946\\
7220	1.375941\\
7221	0.003137\\
7222	0\\
7223	0\\
7224	0\\
7225	0\\
7226	0\\
7227	0\\
7228	0\\
7229	0\\
7230	0\\
7231	0\\
7232	0.114299\\
7233	1.52324\\
7234	1.690348\\
7235	2.417835\\
7236	2.417835\\
7237	2.417835\\
7238	2.417835\\
7239	2.417835\\
7240	2.417835\\
7241	2.589308\\
7242	2.93876\\
7243	3.004055\\
7244	2.417835\\
7245	1.883558\\
7246	0.819257\\
7247	0.131212\\
7248	0\\
7249	0\\
7250	0\\
7251	0\\
7252	0\\
7253	0\\
7254	0\\
7255	0\\
7256	0.470201\\
7257	1.375941\\
7258	1.975219\\
7259	1.375941\\
7260	1.375941\\
7261	0.521527\\
7262	0.657471\\
7263	0.431354\\
7264	0.596985\\
7265	0.982677\\
7266	2.258228\\
7267	2.938782\\
7268	2.417835\\
7269	9e-06\\
7270	0\\
7271	0\\
7272	0\\
7273	0\\
7274	0\\
7275	0\\
7276	0\\
7277	0\\
7278	0\\
7279	0\\
7280	0\\
7281	0\\
7282	3e-06\\
7283	0\\
7284	0\\
7285	0\\
7286	0\\
7287	0\\
7288	0\\
7289	0\\
7290	0\\
7291	0.033676\\
7292	0\\
7293	0\\
7294	0\\
7295	0\\
7296	0\\
7297	0\\
7298	0\\
7299	0\\
7300	0\\
7301	0\\
7302	0\\
7303	0\\
7304	0\\
7305	0\\
7306	0\\
7307	0\\
7308	0\\
7309	0\\
7310	0\\
7311	0\\
7312	0\\
7313	0\\
7314	0\\
7315	0\\
7316	0\\
7317	0\\
7318	0\\
7319	0\\
7320	0\\
7321	0\\
7322	0\\
7323	0\\
7324	0\\
7325	0\\
7326	0\\
7327	0\\
7328	0\\
7329	0\\
7330	0\\
7331	0\\
7332	0\\
7333	0\\
7334	0\\
7335	0\\
7336	0\\
7337	0\\
7338	0\\
7339	0\\
7340	0\\
7341	0\\
7342	0\\
7343	0\\
7344	0\\
7345	0\\
7346	0\\
7347	0\\
7348	0\\
7349	0\\
7350	0\\
7351	0\\
7352	0\\
7353	0\\
7354	0\\
7355	0\\
7356	1e-06\\
7357	0\\
7358	0\\
7359	0\\
7360	0\\
7361	0\\
7362	0\\
7363	0\\
7364	0\\
7365	0\\
7366	0\\
7367	0\\
7368	1e-06\\
7369	0\\
7370	0\\
7371	0\\
7372	0\\
7373	0\\
7374	0\\
7375	0\\
7376	0.174467\\
7377	1e-05\\
7378	1.151667\\
7379	2.009662\\
7380	2.32886\\
7381	2.098158\\
7382	2.294026\\
7383	2.385092\\
7384	2.553665\\
7385	2.194789\\
7386	3.103887\\
7387	3.129176\\
7388	2.553665\\
7389	2.553665\\
7390	0.600147\\
7391	3.1e-05\\
7392	1e-06\\
7393	1e-06\\
7394	0\\
7395	0\\
7396	0\\
7397	0\\
7398	0\\
7399	0.134011\\
7400	2.553665\\
7401	2.553665\\
7402	2.553665\\
7403	2.553665\\
7404	2.553665\\
7405	2.553665\\
7406	2.553665\\
7407	2.553665\\
7408	2.553665\\
7409	2.553665\\
7410	2.839109\\
7411	3.103878\\
7412	2.3849\\
7413	2.553665\\
7414	1.062297\\
7415	0.455586\\
7416	0\\
7417	0\\
7418	0\\
7419	0\\
7420	0\\
7421	0\\
7422	0\\
7423	0\\
7424	0.811636\\
7425	0.04488\\
7426	0.540229\\
7427	0.762662\\
7428	5e-06\\
7429	7e-06\\
7430	1e-06\\
7431	0\\
7432	0\\
7433	0\\
7434	2e-06\\
7435	0.455586\\
7436	0\\
7437	0\\
7438	0\\
7439	0\\
7440	0\\
7441	0\\
7442	0\\
7443	0\\
7444	0\\
7445	0\\
7446	0\\
7447	0\\
7448	0\\
7449	0\\
7450	0\\
7451	0\\
7452	0\\
7453	0\\
7454	0\\
7455	0\\
7456	0\\
7457	0\\
7458	0\\
7459	0\\
7460	0\\
7461	0\\
7462	0\\
7463	0\\
7464	0\\
7465	0\\
7466	0\\
7467	0\\
7468	0\\
7469	0\\
7470	0\\
7471	0\\
7472	0\\
7473	0\\
7474	0\\
7475	0\\
7476	0\\
7477	0\\
7478	0\\
7479	0\\
7480	0\\
7481	0\\
7482	0\\
7483	0\\
7484	0\\
7485	0\\
7486	0\\
7487	0\\
7488	0\\
7489	0\\
7490	0\\
7491	0\\
7492	0\\
7493	0\\
7494	0\\
7495	0\\
7496	0\\
7497	0\\
7498	0\\
7499	0\\
7500	0\\
7501	0\\
7502	0\\
7503	0\\
7504	0\\
7505	0\\
7506	0\\
7507	0\\
7508	0\\
7509	0\\
7510	0\\
7511	0\\
7512	0\\
7513	0\\
7514	0\\
7515	0\\
7516	0\\
7517	0\\
7518	0\\
7519	0\\
7520	0\\
7521	0\\
7522	0\\
7523	0\\
7524	0\\
7525	0\\
7526	0\\
7527	0\\
7528	0\\
7529	0\\
7530	0\\
7531	0\\
7532	0\\
7533	0\\
7534	0\\
7535	0\\
7536	0\\
7537	0\\
7538	0\\
7539	0\\
7540	0\\
7541	0\\
7542	0\\
7543	0\\
7544	0\\
7545	0\\
7546	0\\
7547	0\\
7548	0\\
7549	0\\
7550	0\\
7551	0\\
7552	0\\
7553	0\\
7554	0\\
7555	0\\
7556	0\\
7557	0\\
7558	0\\
7559	0\\
7560	0\\
7561	0\\
7562	0\\
7563	0\\
7564	0\\
7565	0\\
7566	0\\
7567	0\\
7568	0.417094\\
7569	5e-06\\
7570	0.093268\\
7571	0.455588\\
7572	0.526319\\
7573	0.455586\\
7574	1.186601\\
7575	1.191874\\
7576	1.231977\\
7577	0.819693\\
7578	2.553665\\
7579	2.553665\\
7580	2.174193\\
7581	1.453239\\
7582	1e-06\\
7583	0\\
7584	0\\
7585	0\\
7586	0\\
7587	0\\
7588	0\\
7589	0\\
7590	0\\
7591	0\\
7592	1.453234\\
7593	1.173069\\
7594	0.043288\\
7595	0\\
7596	0\\
7597	0\\
7598	0\\
7599	0\\
7600	0\\
7601	0.192531\\
7602	2.553665\\
7603	2.553665\\
7604	1.453239\\
7605	0.607926\\
7606	0\\
7607	0\\
7608	0\\
7609	0\\
7610	0\\
7611	0\\
7612	0\\
7613	0\\
7614	0\\
7615	0\\
7616	0.444444\\
7617	0.800112\\
7618	1.302336\\
7619	0.696418\\
7620	1.453239\\
7621	0.882208\\
7622	1.453239\\
7623	1.751423\\
7624	1.996908\\
7625	2.553665\\
7626	2.553665\\
7627	2.553665\\
7628	1.453239\\
7629	1.453239\\
7630	1e-06\\
7631	0.221519\\
7632	0\\
7633	0\\
7634	0\\
7635	0\\
7636	0\\
7637	0\\
7638	0\\
7639	0\\
7640	0\\
7641	0\\
7642	0\\
7643	0\\
7644	0\\
7645	0\\
7646	0\\
7647	0\\
7648	0\\
7649	0\\
7650	0\\
7651	0\\
7652	0\\
7653	0\\
7654	0\\
7655	0\\
7656	0\\
7657	0\\
7658	0\\
7659	0\\
7660	0\\
7661	0\\
7662	0\\
7663	0\\
7664	0\\
7665	0\\
7666	0\\
7667	0\\
7668	0\\
7669	0\\
7670	0\\
7671	0\\
7672	0\\
7673	0\\
7674	0\\
7675	0\\
7676	0\\
7677	0\\
7678	0\\
7679	0\\
7680	0\\
7681	0\\
7682	0\\
7683	0\\
7684	0\\
7685	0\\
7686	0\\
7687	0\\
7688	0.74753\\
7689	1.300982\\
7690	1.54405\\
7691	1.453239\\
7692	1.453239\\
7693	0.740417\\
7694	1.226241\\
7695	2.202786\\
7696	2.553665\\
7697	2.41597\\
7698	3.164687\\
7699	3.103879\\
7700	2.28147\\
7701	1.361242\\
7702	0\\
7703	0\\
7704	0\\
7705	0\\
7706	0\\
7707	0\\
7708	0\\
7709	0\\
7710	0\\
7711	0\\
7712	2.553665\\
7713	2.549678\\
7714	1.973095\\
7715	1.576743\\
7716	1.543075\\
7717	1.55666\\
7718	2.268265\\
7719	2.221402\\
7720	2.553665\\
7721	2.415205\\
7722	2.893882\\
7723	2.734772\\
7724	2.553665\\
7725	2.253801\\
7726	0.426344\\
7727	0.455579\\
7728	0\\
7729	0\\
7730	0\\
7731	0\\
7732	0\\
7733	0\\
7734	0\\
7735	4.2e-05\\
7736	2.626861\\
7737	2.553665\\
7738	2.553665\\
7739	2.553665\\
7740	2.553665\\
7741	2.553665\\
7742	2.55408\\
7743	2.553665\\
7744	2.553665\\
7745	2.553665\\
7746	2.553666\\
7747	2.553665\\
7748	1.983303\\
7749	2.385093\\
7750	0.336574\\
7751	0\\
7752	0\\
7753	0\\
7754	0\\
7755	0\\
7756	0\\
7757	0\\
7758	0\\
7759	0\\
7760	2.553665\\
7761	2.259507\\
7762	1.530595\\
7763	0.930399\\
7764	0.416036\\
7765	0.389939\\
7766	0.817979\\
7767	1.116825\\
7768	1.453237\\
7769	1.77937\\
7770	2.552543\\
7771	2.553665\\
7772	1.453239\\
7773	1.402884\\
7774	0\\
7775	0\\
7776	0\\
7777	0\\
7778	0\\
7779	0\\
7780	0\\
7781	0\\
7782	0\\
7783	0\\
7784	1.453239\\
7785	2.451699\\
7786	2.423125\\
7787	2.462508\\
7788	1.860932\\
7789	1.453238\\
7790	2.114323\\
7791	1.453239\\
7792	1.453239\\
7793	2.121124\\
7794	2.553665\\
7795	2.553656\\
7796	1.453238\\
7797	7e-06\\
7798	0\\
7799	0\\
7800	0\\
7801	0\\
7802	0\\
7803	0\\
7804	0\\
7805	0\\
7806	0\\
7807	0\\
7808	0\\
7809	0\\
7810	0\\
7811	0\\
7812	0\\
7813	0\\
7814	0\\
7815	0\\
7816	0\\
7817	0\\
7818	0\\
7819	0\\
7820	0\\
7821	0\\
7822	0\\
7823	0\\
7824	0\\
7825	0\\
7826	0\\
7827	0\\
7828	0\\
7829	0\\
7830	0\\
7831	0\\
7832	0\\
7833	0\\
7834	0\\
7835	0\\
7836	0\\
7837	0\\
7838	0\\
7839	0\\
7840	0\\
7841	0\\
7842	0\\
7843	0\\
7844	0\\
7845	0\\
7846	0\\
7847	0\\
7848	0\\
7849	0\\
7850	0\\
7851	0\\
7852	0\\
7853	0\\
7854	0\\
7855	0\\
7856	0\\
7857	0\\
7858	0\\
7859	0\\
7860	0\\
7861	0\\
7862	0\\
7863	0\\
7864	0\\
7865	1e-06\\
7866	1.200066\\
7867	1.421857\\
7868	0.538816\\
7869	0\\
7870	0\\
7871	0\\
7872	0\\
7873	0\\
7874	0\\
7875	0\\
7876	0\\
7877	0\\
7878	0\\
7879	0\\
7880	0.888295\\
7881	1.20092\\
7882	0.455587\\
7883	3e-06\\
7884	1e-06\\
7885	0\\
7886	0\\
7887	0\\
7888	1e-06\\
7889	0.005546\\
7890	1.45324\\
7891	1.244902\\
7892	0.839098\\
7893	0\\
7894	0\\
7895	0\\
7896	0\\
7897	0\\
7898	0\\
7899	0\\
7900	0\\
7901	0\\
7902	0\\
7903	0\\
7904	1.150043\\
7905	2.010762\\
7906	2.496121\\
7907	2.553665\\
7908	2.553665\\
7909	2.553665\\
7910	2.578598\\
7911	2.644486\\
7912	2.760655\\
7913	3.103878\\
7914	3.023665\\
7915	3.103878\\
7916	2.553665\\
7917	0.992468\\
7918	1e-06\\
7919	0\\
7920	0\\
7921	0\\
7922	0\\
7923	0\\
7924	0\\
7925	0\\
7926	0\\
7927	1e-05\\
7928	2.553665\\
7929	2.85878\\
7930	2.553665\\
7931	2.553665\\
7932	2.553665\\
7933	2.553665\\
7934	2.553665\\
7935	2.553665\\
7936	2.553665\\
7937	2.553665\\
7938	2.553665\\
7939	2.553665\\
7940	2.553665\\
7941	1.453259\\
7942	0\\
7943	0\\
7944	0\\
7945	0\\
7946	0\\
7947	0\\
7948	0\\
7949	0\\
7950	0\\
7951	0\\
7952	0\\
7953	0\\
7954	0\\
7955	0\\
7956	0\\
7957	0\\
7958	0\\
7959	0\\
7960	0\\
7961	0\\
7962	0\\
7963	0\\
7964	0\\
7965	0\\
7966	0\\
7967	0\\
7968	0\\
7969	0\\
7970	0\\
7971	0\\
7972	0\\
7973	0\\
7974	0\\
7975	0\\
7976	0\\
7977	0\\
7978	0\\
7979	0\\
7980	0\\
7981	0\\
7982	0\\
7983	0\\
7984	0\\
7985	0\\
7986	0\\
7987	0\\
7988	0\\
7989	0\\
7990	0\\
7991	0\\
7992	0\\
7993	0\\
7994	0\\
7995	0\\
7996	0\\
7997	0\\
7998	0\\
7999	0\\
8000	0\\
8001	0\\
};
\addplot [color=mycolor1,solid,line width=1.0pt,forget plot]
  table[row sep=crcr]{%
8001	0\\
8002	0\\
8003	0.029654\\
8004	0.455413\\
8005	0.648703\\
8006	0\\
8007	0\\
8008	0.227616\\
8009	1.453239\\
8010	1.600902\\
8011	1.81544\\
8012	2.355411\\
8013	1.859299\\
8014	0.189265\\
8015	0.193349\\
8016	0\\
8017	0\\
8018	0\\
8019	0\\
8020	0\\
8021	0\\
8022	0\\
8023	0\\
8024	2.450128\\
8025	2.767504\\
8026	3.850431\\
8027	4.220381\\
8028	4.486731\\
8029	4.486731\\
8030	4.227991\\
8031	4.486731\\
8032	4.486731\\
8033	4.486731\\
8034	4.786262\\
8035	4.486731\\
8036	3.578959\\
8037	3.637694\\
8038	2.58423\\
8039	2.58423\\
8040	2.019953\\
8041	1.111572\\
8042	0\\
8043	0\\
8044	0\\
8045	0\\
8046	0\\
8047	0\\
8048	2.58423\\
8049	2.58423\\
8050	2.58423\\
8051	2.58423\\
8052	2.58423\\
8053	2.58423\\
8054	2.115644\\
8055	2.58423\\
8056	2.58423\\
8057	2.58423\\
8058	2.58423\\
8059	2.58423\\
8060	2.58423\\
8061	1.456402\\
8062	0\\
8063	0\\
8064	0\\
8065	0\\
8066	0\\
8067	0\\
8068	0\\
8069	0\\
8070	0\\
8071	0\\
8072	2.20028\\
8073	2.58423\\
8074	2.58423\\
8075	2.58423\\
8076	2.58423\\
8077	2.58423\\
8078	2.58423\\
8079	2.58423\\
8080	3.141029\\
8081	3.386083\\
8082	3.29196\\
8083	3.059854\\
8084	2.58423\\
8085	2.58423\\
8086	1.470633\\
8087	1.470633\\
8088	0.754919\\
8089	3e-06\\
8090	0\\
8091	0\\
8092	0\\
8093	0\\
8094	0\\
8095	1.470633\\
8096	3.141029\\
8097	4.177853\\
8098	4.169591\\
8099	4.09138\\
8100	4.288594\\
8101	4.290434\\
8102	4.321311\\
8103	4.227991\\
8104	4.227991\\
8105	4.220381\\
8106	4.730499\\
8107	4.486731\\
8108	4.227991\\
8109	3.682365\\
8110	2.58423\\
8111	2.58423\\
8112	2.58423\\
8113	0\\
8114	0\\
8115	0\\
8116	0\\
8117	0\\
8118	0\\
8119	1.470633\\
8120	2.767504\\
8121	2.58423\\
8122	3.141028\\
8123	3.141029\\
8124	3.141029\\
8125	2.774505\\
8126	2.584234\\
8127	2.58423\\
8128	2.58423\\
8129	2.58423\\
8130	2.58423\\
8131	2.192914\\
8132	1.826818\\
8133	2e-06\\
8134	0.054671\\
8135	0.46104\\
8136	0.412491\\
8137	0\\
8138	0\\
8139	0\\
8140	0\\
8141	0\\
8142	0\\
8143	0\\
8144	0\\
8145	0\\
8146	0\\
8147	1e-06\\
8148	0\\
8149	0\\
8150	0\\
8151	0\\
8152	0\\
8153	1.067512\\
8154	1.947996\\
8155	2.125732\\
8156	2.113585\\
8157	1.305699\\
8158	1.3e-05\\
8159	2.2e-05\\
8160	1.315305\\
8161	0\\
8162	1e-06\\
8163	0\\
8164	0\\
8165	0\\
8166	0\\
8167	0\\
8168	0\\
8169	0\\
8170	0\\
8171	0\\
8172	0\\
8173	0\\
8174	0\\
8175	0\\
8176	0\\
8177	0\\
8178	0\\
8179	0\\
8180	0\\
8181	0\\
8182	0\\
8183	0\\
8184	0\\
8185	0\\
8186	0\\
8187	0\\
8188	0\\
8189	0\\
8190	0\\
8191	0\\
8192	6e-06\\
8193	1.591686\\
8194	1.470633\\
8195	1.21443\\
8196	1.777332\\
8197	2.561552\\
8198	2.58423\\
8199	2.58423\\
8200	2.58423\\
8201	2.914891\\
8202	2.787961\\
8203	3.382629\\
8204	2.767477\\
8205	1.920006\\
8206	1e-06\\
8207	0.113503\\
8208	0.46104\\
8209	0\\
8210	0\\
8211	0\\
8212	0\\
8213	0\\
8214	0\\
8215	0.461039\\
8216	2.58423\\
8217	3.19418\\
8218	2.58423\\
8219	2.58423\\
8220	2.58423\\
8221	2.58423\\
8222	2.58423\\
8223	2.58423\\
8224	2.58423\\
8225	2.58423\\
8226	2.58423\\
8227	2.58423\\
8228	2.134387\\
8229	1e-06\\
8230	0\\
8231	0\\
8232	0\\
8233	0\\
8234	0\\
8235	0\\
8236	0\\
8237	0\\
8238	0\\
8239	0\\
8240	0\\
8241	1.470632\\
8242	1.444567\\
8243	0.537365\\
8244	0.379563\\
8245	0\\
8246	0.461039\\
8247	0.619491\\
8248	2.169805\\
8249	2.58423\\
8250	2.774466\\
8251	2.58423\\
8252	1.283418\\
8253	0.731644\\
8254	0\\
8255	0\\
8256	0\\
8257	0\\
8258	0\\
8259	0\\
8260	0\\
8261	0\\
8262	0\\
8263	0\\
8264	0\\
8265	0.851568\\
8266	0.556467\\
8267	1.018457\\
8268	1.32163\\
8269	1.094546\\
8270	0.918063\\
8271	0.959626\\
8272	1.214838\\
8273	1.617639\\
8274	2.110352\\
8275	1.470424\\
8276	0.385723\\
8277	1e-06\\
8278	0\\
8279	0\\
8280	0\\
8281	0\\
8282	0\\
8283	0\\
8284	0\\
8285	0\\
8286	0\\
8287	0\\
8288	0\\
8289	0\\
8290	0\\
8291	0\\
8292	0.236546\\
8293	4.5e-05\\
8294	0.291366\\
8295	0.461102\\
8296	0.480944\\
8297	1.744197\\
8298	2.155655\\
8299	2.58423\\
8300	1.21443\\
8301	0\\
8302	0\\
8303	0\\
8304	0\\
8305	0\\
8306	0\\
8307	0\\
8308	0\\
8309	0\\
8310	0\\
8311	0\\
8312	0\\
8313	0\\
8314	0\\
8315	1e-06\\
8316	7e-06\\
8317	2.3e-05\\
8318	7e-06\\
8319	0\\
8320	0.083805\\
8321	0.81056\\
8322	1.091049\\
8323	1.743017\\
8324	0.237414\\
8325	1e-06\\
8326	0\\
8327	0\\
8328	0\\
8329	0\\
8330	0\\
8331	0\\
8332	0\\
8333	0\\
8334	0\\
8335	0\\
8336	0\\
8337	0\\
8338	0\\
8339	0\\
8340	0\\
8341	0\\
8342	0\\
8343	0\\
8344	0\\
8345	0\\
8346	0\\
8347	0\\
8348	0\\
8349	0\\
8350	0\\
8351	0\\
8352	0\\
8353	0\\
8354	0\\
8355	0\\
8356	0\\
8357	0\\
8358	0\\
8359	0\\
8360	0\\
8361	0\\
8362	0\\
8363	0\\
8364	0\\
8365	0\\
8366	0\\
8367	0\\
8368	0\\
8369	0\\
8370	0.315301\\
8371	2e-06\\
8372	0\\
8373	0\\
8374	0\\
8375	0\\
8376	0\\
8377	0\\
8378	0\\
8379	0\\
8380	0\\
8381	0\\
8382	0\\
8383	0\\
8384	1.21443\\
8385	2.58423\\
8386	2.58423\\
8387	2.396343\\
8388	2.425566\\
8389	1.745406\\
8390	1.93188\\
8391	2.2413\\
8392	2.58423\\
8393	2.58423\\
8394	3.322144\\
8395	3.442039\\
8396	2.58423\\
8397	2.58423\\
8398	0.338607\\
8399	0.353238\\
8400	0\\
8401	0\\
8402	0\\
8403	0\\
8404	0\\
8405	0\\
8406	0\\
8407	0\\
8408	0\\
8409	0.461039\\
8410	1.890096\\
8411	2.507219\\
8412	2.58423\\
8413	2.58423\\
8414	2.58423\\
8415	2.58423\\
8416	2.58423\\
8417	3.065261\\
8418	2.765555\\
8419	2.606016\\
8420	2.708492\\
8421	1.407354\\
8422	0\\
8423	0\\
8424	0\\
8425	0\\
8426	0\\
8427	0\\
8428	0\\
8429	0\\
8430	0\\
8431	0\\
8432	1e-06\\
8433	1.429666\\
8434	0\\
8435	1e-06\\
8436	0.46089\\
8437	0\\
8438	1e-06\\
8439	0\\
8440	0\\
8441	0\\
8442	0\\
8443	0\\
8444	0\\
8445	0\\
8446	0\\
8447	0\\
8448	0\\
8449	0\\
8450	0\\
8451	0\\
8452	0\\
8453	0\\
8454	0\\
8455	0\\
8456	0\\
8457	0\\
8458	0\\
8459	0\\
8460	0.432145\\
8461	1.072206\\
8462	0\\
8463	0\\
8464	0\\
8465	0\\
8466	0\\
8467	0\\
8468	0\\
8469	0\\
8470	0\\
8471	0\\
8472	0\\
8473	0\\
8474	0\\
8475	0\\
8476	0\\
8477	0\\
8478	0\\
8479	0\\
8480	0\\
8481	0\\
8482	0\\
8483	0\\
8484	0\\
8485	0\\
8486	0\\
8487	0\\
8488	0\\
8489	0\\
8490	0\\
8491	0\\
8492	0\\
8493	0\\
8494	0\\
8495	0\\
8496	0\\
8497	0\\
8498	0\\
8499	0\\
8500	0\\
8501	0\\
8502	0\\
8503	0\\
8504	0\\
8505	0\\
8506	0\\
8507	0\\
8508	0\\
8509	0\\
8510	0\\
8511	0\\
8512	0\\
8513	0\\
8514	0\\
8515	0\\
8516	0\\
8517	0\\
8518	0\\
8519	0\\
8520	0\\
8521	0\\
8522	0\\
8523	0\\
8524	0\\
8525	0\\
8526	0\\
8527	0\\
8528	0\\
8529	0\\
8530	0\\
8531	0\\
8532	0\\
8533	0\\
8534	0\\
8535	0\\
8536	0\\
8537	0\\
8538	0\\
8539	0\\
8540	0\\
8541	0\\
8542	0\\
8543	0\\
8544	0\\
8545	0\\
8546	0\\
8547	0\\
8548	0\\
8549	0\\
8550	0\\
8551	0\\
8552	0\\
8553	0\\
8554	0\\
8555	0\\
8556	0\\
8557	0\\
8558	0\\
8559	0\\
8560	0\\
8561	0\\
8562	0\\
8563	0\\
8564	0\\
8565	0\\
8566	0\\
8567	0\\
8568	0\\
8569	0\\
8570	0\\
8571	0\\
8572	0\\
8573	0\\
8574	0\\
8575	0\\
8576	0\\
8577	0\\
8578	0\\
8579	0\\
8580	0\\
8581	0\\
8582	0\\
8583	0\\
8584	0\\
8585	0\\
8586	0\\
8587	0\\
8588	0\\
8589	0\\
8590	0\\
8591	0\\
8592	0\\
8593	0\\
8594	0\\
8595	0\\
8596	0\\
8597	0\\
8598	0\\
8599	0\\
8600	0\\
8601	0\\
8602	0\\
8603	0\\
8604	0\\
8605	0\\
8606	0\\
8607	0\\
8608	0\\
8609	0\\
8610	0\\
8611	0\\
8612	0\\
8613	0\\
8614	0\\
8615	0\\
8616	0\\
8617	0\\
8618	0\\
8619	0\\
8620	0\\
8621	0\\
8622	0\\
8623	0\\
8624	0\\
8625	0\\
8626	0\\
8627	0\\
8628	0\\
8629	0\\
8630	0\\
8631	0\\
8632	0\\
8633	0\\
8634	0\\
8635	0\\
8636	0\\
8637	0\\
8638	0\\
8639	0\\
8640	0\\
8641	0\\
8642	0\\
8643	0\\
8644	0\\
8645	0\\
8646	0\\
8647	0\\
8648	0\\
8649	0\\
8650	0\\
8651	0\\
8652	0\\
8653	0\\
8654	0\\
8655	0\\
8656	0\\
8657	0\\
8658	0\\
8659	0\\
8660	0\\
8661	0\\
8662	0\\
8663	0\\
8664	0\\
8665	0\\
8666	0\\
8667	0\\
8668	0\\
8669	0\\
8670	0\\
8671	0\\
8672	0\\
8673	0\\
8674	0\\
8675	0\\
8676	0\\
8677	0\\
8678	0\\
8679	0\\
8680	0\\
8681	0\\
8682	0\\
8683	0\\
8684	0\\
8685	0\\
8686	0\\
8687	0\\
8688	0\\
8689	0\\
8690	0\\
8691	0\\
8692	0\\
8693	0\\
8694	0\\
8695	0\\
8696	0\\
8697	0\\
8698	0\\
8699	0\\
8700	0\\
8701	0\\
8702	0\\
8703	0\\
8704	0\\
8705	0\\
8706	0\\
8707	7e-06\\
8708	0\\
8709	0\\
8710	0\\
8711	0\\
8712	0\\
8713	0\\
8714	0\\
8715	0\\
8716	0\\
8717	0\\
8718	0\\
8719	0\\
8720	0\\
8721	0\\
8722	0\\
8723	0\\
8724	0\\
8725	0\\
8726	0\\
8727	0\\
8728	0\\
8729	0\\
8730	0.491856\\
8731	0.233086\\
8732	0\\
8733	0\\
8734	0\\
8735	0\\
8736	0\\
8737	0\\
8738	0\\
8739	0\\
8740	0\\
8741	0\\
8742	0\\
8743	0\\
8744	0\\
8745	0\\
8746	0\\
8747	0\\
8748	0\\
8749	0\\
8750	0\\
8751	0\\
8752	0\\
8753	0\\
8754	0\\
8755	0\\
8756	0\\
8757	0\\
8758	0\\
8759	0\\
8760	0\\
};
\end{axis}
\end{tikzpicture}%
    \caption{Predicted reserve prices for the ImpExp model}
    \label{fig:ImpExp_R}
\end{figure}

\begin{figure}[H]
    \centering
    \setlength\fheight{0.3\textwidth}
    \setlength\fwidth{0.85\textwidth}
    % This file was created by matlab2tikz.
% Minimal pgfplots version: 1.3
%
%The latest updates can be retrieved from
%  http://www.mathworks.com/matlabcentral/fileexchange/22022-matlab2tikz
%where you can also make suggestions and rate matlab2tikz.
%
\definecolor{mycolor1}{rgb}{0.87059,0.49020,0.00000}%
%
\begin{tikzpicture}

\begin{axis}[%
width=\fwidth,
height=\fheight,
at={(0\fwidth,0\fheight)},
scale only axis,
separate axis lines,
every outer x axis line/.append style={black},
every x tick label/.append style={font=\color{black}},
xmin=0,
xmax=8760,
xlabel={time [hour]},
xtick={0,1000,2000,3000,4000,5000,6000,7000,8000},
xmajorgrids,
every outer y axis line/.append style={black},
every y tick label/.append style={font=\color{black}},
ymin=0,
ymax=800,
ymajorgrids,
title style={font=\bfseries},
title={ORDC - Reserve price [\euro/MWh]}
]
\addplot [color=mycolor1,solid,line width=1.0pt,forget plot]
  table[row sep=crcr]{%
1	2e-06\\
2	2e-06\\
3	6e-06\\
4	1e-06\\
5	0\\
6	0\\
7	0\\
8	2e-06\\
9	2e-06\\
10	2e-06\\
11	5e-06\\
12	4e-06\\
13	0\\
14	0\\
15	0\\
16	0\\
17	0\\
18	0\\
19	1e-06\\
20	1e-06\\
21	1e-06\\
22	1e-06\\
23	0\\
24	2e-06\\
25	1e-06\\
26	0\\
27	1e-06\\
28	0\\
29	1e-06\\
30	0\\
31	2e-06\\
32	1e-06\\
33	2e-06\\
34	3e-06\\
35	9e-06\\
36	2e-06\\
37	2e-06\\
38	2e-06\\
39	2e-06\\
40	2e-06\\
41	1e-06\\
42	2e-06\\
43	1e-06\\
44	4e-06\\
45	3e-06\\
46	1e-06\\
47	2e-06\\
48	6e-06\\
49	3e-06\\
50	0\\
51	4e-06\\
52	0\\
53	0\\
54	2e-06\\
55	1e-06\\
56	8e-06\\
57	3e-06\\
58	9e-06\\
59	3e-06\\
60	3e-06\\
61	3e-06\\
62	2e-06\\
63	4e-06\\
64	6e-06\\
65	3e-06\\
66	4e-06\\
67	4e-06\\
68	2e-06\\
69	3e-06\\
70	5e-06\\
71	7e-06\\
72	5e-06\\
73	9e-06\\
74	3e-06\\
75	1e-06\\
76	2e-06\\
77	0\\
78	6e-06\\
79	1e-06\\
80	4e-06\\
81	3e-06\\
82	1e-05\\
83	0\\
84	4e-06\\
85	9e-06\\
86	4e-06\\
87	1e-05\\
88	0\\
89	3e-06\\
90	1e-06\\
91	3e-06\\
92	2e-06\\
93	5e-06\\
94	2e-06\\
95	2e-06\\
96	4e-06\\
97	4e-06\\
98	4e-06\\
99	0\\
100	3e-06\\
101	5e-06\\
102	3e-06\\
103	5e-06\\
104	3e-06\\
105	2e-06\\
106	4e-06\\
107	0\\
108	0\\
109	0\\
110	1e-05\\
111	8e-06\\
112	0\\
113	4e-06\\
114	3e-06\\
115	1.1e-05\\
116	4e-06\\
117	2e-06\\
118	4e-06\\
119	2e-06\\
120	2e-06\\
121	2e-06\\
122	0\\
123	2e-06\\
124	1e-06\\
125	1e-06\\
126	0\\
127	2e-06\\
128	0\\
129	3e-06\\
130	1e-06\\
131	1.3e-05\\
132	0\\
133	6e-06\\
134	3e-06\\
135	5e-06\\
136	0\\
137	0\\
138	0.011258\\
139	1e-06\\
140	1e-06\\
141	1e-06\\
142	5e-06\\
143	0\\
144	2e-06\\
145	2e-06\\
146	5e-06\\
147	2e-06\\
148	1e-06\\
149	1e-06\\
150	0\\
151	2e-06\\
152	0\\
153	0.224931\\
154	1e-06\\
155	1e-05\\
156	8.2e-05\\
157	7e-06\\
158	0\\
159	2.8e-05\\
160	1.688574\\
161	1.808459\\
162	4.547575\\
163	4.547572\\
164	2.160853\\
165	0.70194\\
166	2.6e-05\\
167	9e-06\\
168	11.017886\\
169	8.19841\\
170	6.073945\\
171	9e-06\\
172	5e-06\\
173	5e-06\\
174	1.4e-05\\
175	2e-06\\
176	1.983045\\
177	4.547572\\
178	3.570045\\
179	2.098037\\
180	1.408328\\
181	3.266787\\
182	6.819754\\
183	6.821725\\
184	6.821726\\
185	6.821726\\
186	7.563612\\
187	8.141963\\
188	6.821726\\
189	7.764469\\
190	7.073953\\
191	7.226651\\
192	5.982189\\
193	0.10759\\
194	3.4e-05\\
195	8.9e-05\\
196	1.4e-05\\
197	3.2e-05\\
198	1e-06\\
199	0.107596\\
200	3.431216\\
201	5.273166\\
202	3.108903\\
203	3.07087\\
204	3.383761\\
205	4.834447\\
206	5.158967\\
207	4.670445\\
208	4.915935\\
209	3.767686\\
210	4.131445\\
211	4.547564\\
212	2.751694\\
213	0.701945\\
214	0.803057\\
215	1.596805\\
216	1.352639\\
217	2e-06\\
218	0\\
219	7e-06\\
220	1e-06\\
221	2e-06\\
222	1.1e-05\\
223	2.303371\\
224	5.687139\\
225	6.684602\\
226	4.987116\\
227	4.547572\\
228	4.956\\
229	3.709624\\
230	4.547548\\
231	4.547576\\
232	4.666527\\
233	5.061395\\
234	6.446448\\
235	5.8579\\
236	4.547573\\
237	3.32127\\
238	2.247857\\
239	2.96533\\
240	3.238165\\
241	1.4e-05\\
242	2e-06\\
243	2e-06\\
244	1.5e-05\\
245	1e-06\\
246	3e-06\\
247	1.4e-05\\
248	0\\
249	8.2e-05\\
250	0.70175\\
251	0.922854\\
252	1.022764\\
253	1.20278\\
254	1.331871\\
255	0.516584\\
256	0.709078\\
257	1.352633\\
258	3.978581\\
259	3.01312\\
260	1.829029\\
261	2.7e-05\\
262	0.48637\\
263	1.688592\\
264	0.13505\\
265	0.591038\\
266	0.107596\\
267	7e-06\\
268	1.1e-05\\
269	2.9e-05\\
270	1.5e-05\\
271	2.9e-05\\
272	1e-06\\
273	0.000577\\
274	0.1076\\
275	0.107599\\
276	2e-05\\
277	8e-06\\
278	3e-06\\
279	1e-06\\
280	2e-06\\
281	0\\
282	0.000108\\
283	1.457164\\
284	0.224879\\
285	2e-06\\
286	2.6e-05\\
287	6e-06\\
288	0\\
289	7e-06\\
290	2e-06\\
291	4e-06\\
292	1e-06\\
293	1e-06\\
294	5e-06\\
295	1.2e-05\\
296	8e-06\\
297	3.3e-05\\
298	1.6e-05\\
299	1.055009\\
300	1.466137\\
301	0.520482\\
302	0.724583\\
303	0.343273\\
304	0.885351\\
305	1.48607\\
306	2.517327\\
307	3.194939\\
308	0.637698\\
309	1e-06\\
310	9e-06\\
311	2e-06\\
312	3e-06\\
313	3e-06\\
314	1e-06\\
315	3e-06\\
316	2e-06\\
317	1e-06\\
318	3e-06\\
319	5e-06\\
320	1.997841\\
321	2.383914\\
322	3.194939\\
323	3.194939\\
324	3.194939\\
325	3.194938\\
326	3.018529\\
327	3.194939\\
328	3.195245\\
329	3.646261\\
330	4.872207\\
331	3.708764\\
332	3.226687\\
333	3.194939\\
334	0.550494\\
335	1.042986\\
336	0.335973\\
337	0\\
338	2e-06\\
339	2e-06\\
340	3e-06\\
341	3e-06\\
342	2e-06\\
343	0\\
344	0.894438\\
345	0.988691\\
346	0.891175\\
347	1.575733\\
348	1.773645\\
349	1.031291\\
350	1.150245\\
351	0.477475\\
352	0.591114\\
353	0.891077\\
354	2.014155\\
355	1.70528\\
356	1.173415\\
357	5e-06\\
358	0\\
359	5e-06\\
360	3e-06\\
361	4e-06\\
362	2e-06\\
363	1e-06\\
364	1e-06\\
365	1e-06\\
366	5e-06\\
367	1e-06\\
368	7e-06\\
369	5e-06\\
370	1e-06\\
371	3e-06\\
372	3e-06\\
373	0.750127\\
374	2.892947\\
375	2.342836\\
376	1.693536\\
377	1.17774\\
378	3.194939\\
379	1.89341\\
380	0.894432\\
381	1e-06\\
382	1.3e-05\\
383	0\\
384	9e-06\\
385	2e-06\\
386	4e-06\\
387	3e-06\\
388	1e-06\\
389	2e-06\\
390	1e-06\\
391	4e-06\\
392	1e-06\\
393	1e-06\\
394	7e-06\\
395	1e-06\\
396	3.7e-05\\
397	1.7e-05\\
398	4e-06\\
399	2e-06\\
400	5.6e-05\\
401	0.262853\\
402	2.108915\\
403	0.667017\\
404	1e-06\\
405	5.5e-05\\
406	2e-06\\
407	4e-06\\
408	3e-06\\
409	2e-06\\
410	4e-06\\
411	3e-06\\
412	1e-06\\
413	1e-06\\
414	1e-06\\
415	1e-06\\
416	1e-06\\
417	2e-06\\
418	2e-06\\
419	1e-06\\
420	3e-06\\
421	5e-06\\
422	7e-06\\
423	6e-06\\
424	2e-06\\
425	4e-06\\
426	0\\
427	2e-06\\
428	2e-06\\
429	1e-06\\
430	0\\
431	4e-06\\
432	5e-06\\
433	3e-06\\
434	1e-06\\
435	0\\
436	3e-06\\
437	3e-06\\
438	3e-06\\
439	4e-06\\
440	3e-06\\
441	4e-06\\
442	3e-06\\
443	2e-06\\
444	1e-06\\
445	0\\
446	1e-06\\
447	0\\
448	0\\
449	6e-06\\
450	8e-06\\
451	0\\
452	1e-05\\
453	0\\
454	0\\
455	1e-06\\
456	6e-06\\
457	4e-06\\
458	3e-06\\
459	5e-06\\
460	1e-06\\
461	8e-06\\
462	2e-06\\
463	3e-05\\
464	2.938922\\
465	2.513503\\
466	3.195058\\
467	3.603385\\
468	3.686806\\
469	3.19494\\
470	3.708764\\
471	3.686806\\
472	3.708764\\
473	3.765364\\
474	4.670758\\
475	3.719842\\
476	3.084152\\
477	3.190196\\
478	2e-06\\
479	4.3e-05\\
480	0\\
481	7e-06\\
482	4e-06\\
483	0\\
484	2e-06\\
485	5e-06\\
486	4e-06\\
487	4e-06\\
488	3.194939\\
489	2.568523\\
490	0.894438\\
491	0.894425\\
492	2.314475\\
493	1.594926\\
494	2.761125\\
495	2.902721\\
496	3.194824\\
497	3.194939\\
498	4.338004\\
499	3.19494\\
500	3.194939\\
501	1.865849\\
502	8e-06\\
503	5e-06\\
504	2.6e-05\\
505	4e-06\\
506	2e-06\\
507	5e-06\\
508	1e-06\\
509	1e-06\\
510	3e-06\\
511	1e-06\\
512	0.335863\\
513	1e-06\\
514	4e-05\\
515	1.9e-05\\
516	0\\
517	1e-06\\
518	3e-06\\
519	2.5e-05\\
520	0.712887\\
521	1.606437\\
522	3.195044\\
523	3.194939\\
524	3.021405\\
525	1.293913\\
526	4e-06\\
527	0.257127\\
528	1e-06\\
529	7e-06\\
530	2e-06\\
531	3e-06\\
532	3e-06\\
533	8e-06\\
534	4e-06\\
535	3e-06\\
536	2.443548\\
537	1.672179\\
538	1.846317\\
539	1.624857\\
540	1.296079\\
541	0.756864\\
542	1.573246\\
543	1.074152\\
544	1.271382\\
545	1.233731\\
546	3.194929\\
547	1.395702\\
548	0.057361\\
549	2.8e-05\\
550	1.4e-05\\
551	4e-06\\
552	0\\
553	2e-06\\
554	1e-06\\
555	3e-06\\
556	2e-06\\
557	3e-06\\
558	3e-06\\
559	2.6e-05\\
560	2.665624\\
561	1.484651\\
562	0.894425\\
563	0.063348\\
564	1e-06\\
565	1e-06\\
566	4.3e-05\\
567	6e-06\\
568	5e-06\\
569	0.015638\\
570	1.912096\\
571	0.894436\\
572	0.006999\\
573	1.3e-05\\
574	6e-06\\
575	0\\
576	2e-05\\
577	1.1e-05\\
578	3e-06\\
579	1e-06\\
580	5e-06\\
581	6e-06\\
582	2e-06\\
583	2e-06\\
584	5e-06\\
585	3e-06\\
586	5e-06\\
587	0\\
588	0\\
589	0\\
590	1e-06\\
591	4e-06\\
592	3e-06\\
593	3e-06\\
594	4e-06\\
595	6e-06\\
596	3e-06\\
597	1e-06\\
598	3e-06\\
599	3e-06\\
600	7e-06\\
601	2e-06\\
602	0\\
603	0\\
604	0\\
605	0\\
606	4e-06\\
607	1e-06\\
608	3e-06\\
609	1e-06\\
610	1e-06\\
611	2e-06\\
612	2e-06\\
613	2e-06\\
614	2e-06\\
615	1e-06\\
616	1e-06\\
617	1e-06\\
618	1e-06\\
619	1e-06\\
620	1e-06\\
621	1e-06\\
622	2e-06\\
623	1e-06\\
624	1e-06\\
625	4e-06\\
626	1e-06\\
627	0\\
628	2e-06\\
629	2e-06\\
630	1e-06\\
631	3e-06\\
632	0\\
633	1e-06\\
634	2e-06\\
635	4e-06\\
636	1e-06\\
637	1e-06\\
638	6e-06\\
639	7e-06\\
640	1e-06\\
641	0.000251\\
642	2e-06\\
643	0.169579\\
644	7e-06\\
645	1.1e-05\\
646	1e-06\\
647	5e-06\\
648	4e-06\\
649	2e-06\\
650	3e-06\\
651	1e-06\\
652	1e-06\\
653	1e-06\\
654	2e-06\\
655	1e-06\\
656	7e-06\\
657	1.2e-05\\
658	2e-06\\
659	0\\
660	3e-06\\
661	2e-06\\
662	0\\
663	6e-06\\
664	6e-06\\
665	2.6e-05\\
666	2e-06\\
667	3e-06\\
668	3e-06\\
669	0\\
670	1e-06\\
671	2e-05\\
672	2e-06\\
673	1e-06\\
674	6e-06\\
675	2e-06\\
676	1e-06\\
677	3e-06\\
678	1e-06\\
679	1e-06\\
680	1e-06\\
681	1e-05\\
682	1e-06\\
683	0\\
684	0\\
685	5e-06\\
686	0\\
687	8e-06\\
688	1e-06\\
689	1e-05\\
690	5e-06\\
691	0.894434\\
692	7e-06\\
693	9e-06\\
694	4e-06\\
695	3e-06\\
696	3e-06\\
697	6e-06\\
698	2e-06\\
699	3e-06\\
700	5e-06\\
701	7e-06\\
702	2e-06\\
703	2e-06\\
704	2.503988\\
705	0.393345\\
706	1.897277\\
707	0.396437\\
708	0.89425\\
709	5.4e-05\\
710	0.078146\\
711	0.262768\\
712	0.821091\\
713	2.370327\\
714	3.275833\\
715	3.194939\\
716	2.05159\\
717	1.876045\\
718	1e-06\\
719	0.772518\\
720	0.447602\\
721	8e-06\\
722	0\\
723	2e-06\\
724	2e-06\\
725	3e-06\\
726	3e-06\\
727	0.000111\\
728	3.194939\\
729	0.583524\\
730	1e-06\\
731	4.9e-05\\
732	0\\
733	7e-06\\
734	1e-06\\
735	3e-06\\
736	0\\
737	5e-06\\
738	3e-06\\
739	7.3e-05\\
740	0\\
741	0\\
742	0\\
743	2e-06\\
744	3e-06\\
745	1e-06\\
746	9e-06\\
747	0\\
748	2e-06\\
749	0\\
750	0\\
751	0\\
752	0\\
753	1e-06\\
754	3e-06\\
755	0\\
756	1e-06\\
757	5e-06\\
758	3e-06\\
759	1e-06\\
760	1e-06\\
761	1e-06\\
762	0\\
763	0\\
764	2e-06\\
765	1e-06\\
766	1e-06\\
767	1e-06\\
768	4e-06\\
769	1e-06\\
770	0\\
771	0\\
772	0\\
773	3e-06\\
774	1e-06\\
775	0\\
776	1e-06\\
777	0\\
778	0\\
779	5e-06\\
780	8e-06\\
781	0\\
782	5e-06\\
783	0\\
784	6e-06\\
785	1e-06\\
786	2e-06\\
787	0\\
788	0\\
789	6e-06\\
790	1e-06\\
791	5e-06\\
792	1e-05\\
793	0\\
794	1e-06\\
795	1e-06\\
796	4e-06\\
797	1e-06\\
798	1e-06\\
799	1.8e-05\\
800	2.910001\\
801	1.643808\\
802	0.381114\\
803	1e-05\\
804	6e-06\\
805	0\\
806	3.3e-05\\
807	1e-05\\
808	0.306032\\
809	1.740324\\
810	3.281994\\
811	4.110747\\
812	2.920389\\
813	2.413893\\
814	0.00829\\
815	0.530802\\
816	0.022366\\
817	6e-06\\
818	0\\
819	0\\
820	3e-06\\
821	3e-06\\
822	1e-06\\
823	1.8e-05\\
824	2.516684\\
825	0.746623\\
826	0.700138\\
827	0.418578\\
828	0.658305\\
829	3.9e-05\\
830	1.6e-05\\
831	1.8e-05\\
832	2.9e-05\\
833	0.362787\\
834	2.151352\\
835	2.910001\\
836	1.315299\\
837	0.632901\\
838	8e-06\\
839	3.2e-05\\
840	6e-06\\
841	1e-06\\
842	1e-06\\
843	0\\
844	0\\
845	0\\
846	9e-06\\
847	8e-06\\
848	1.3e-05\\
849	0.775314\\
850	0.306029\\
851	0.816052\\
852	0.749947\\
853	2e-05\\
854	9.2e-05\\
855	1.2e-05\\
856	4e-06\\
857	0.306055\\
858	2.893872\\
859	2.62514\\
860	0.787793\\
861	5e-06\\
862	1e-06\\
863	0\\
864	3e-06\\
865	1e-06\\
866	0\\
867	1e-06\\
868	0\\
869	1e-06\\
870	1e-06\\
871	0\\
872	1.7e-05\\
873	1e-06\\
874	2.7e-05\\
875	7e-06\\
876	3.1e-05\\
877	3.9e-05\\
878	0.456056\\
879	0.567914\\
880	0.538624\\
881	1.051859\\
882	2.909999\\
883	1.955781\\
884	0.635856\\
885	1e-06\\
886	2e-06\\
887	6e-06\\
888	0\\
889	1e-06\\
890	1e-06\\
891	9e-06\\
892	0\\
893	0\\
894	0\\
895	2e-06\\
896	0\\
897	2.2e-05\\
898	6e-06\\
899	0.30604\\
900	2.512183\\
901	1.301385\\
902	1.540771\\
903	1.136297\\
904	0.486668\\
905	9e-06\\
906	3e-06\\
907	0.515569\\
908	3e-06\\
909	1e-06\\
910	4e-06\\
911	2e-06\\
912	0\\
913	1e-06\\
914	1e-06\\
915	1e-06\\
916	2e-06\\
917	7e-06\\
918	2e-06\\
919	2e-06\\
920	0\\
921	0\\
922	1e-06\\
923	1e-06\\
924	9e-06\\
925	2e-06\\
926	6e-06\\
927	1e-06\\
928	1e-06\\
929	1e-06\\
930	1e-06\\
931	1e-06\\
932	1e-06\\
933	9e-06\\
934	6e-06\\
935	0\\
936	1e-06\\
937	0\\
938	6e-06\\
939	0\\
940	1e-06\\
941	1e-06\\
942	1e-06\\
943	2e-06\\
944	1e-06\\
945	0\\
946	0\\
947	4e-06\\
948	0\\
949	7e-06\\
950	1e-06\\
951	0\\
952	0\\
953	2e-06\\
954	1e-06\\
955	1e-06\\
956	1.2e-05\\
957	1e-06\\
958	1e-06\\
959	1e-06\\
960	1e-06\\
961	1e-06\\
962	1e-06\\
963	0\\
964	0\\
965	6e-06\\
966	1e-06\\
967	3e-06\\
968	3.248924\\
969	3.282019\\
970	3.282043\\
971	2.910001\\
972	3.086409\\
973	2.91\\
974	3.357998\\
975	3.358001\\
976	3.615126\\
977	3.98693\\
978	4.981335\\
979	4.981336\\
980	4.219446\\
981	2.910005\\
982	1.720482\\
983	2.046779\\
984	1.35606\\
985	2e-06\\
986	8e-06\\
987	1e-06\\
988	1e-06\\
989	0\\
990	3e-06\\
991	1.2e-05\\
992	0.595757\\
993	2.2e-05\\
994	5.9e-05\\
995	2e-06\\
996	4.3e-05\\
997	1e-05\\
998	2e-06\\
999	3.8e-05\\
1000	1.1e-05\\
1001	0.306596\\
1002	2.629336\\
1003	2.910001\\
1004	2.195944\\
1005	0.783765\\
1006	6e-05\\
1007	0.305987\\
1008	0.005497\\
1009	4e-06\\
1010	1e-06\\
1011	1e-06\\
1012	3e-06\\
1013	5e-06\\
1014	1e-06\\
1015	1.3e-05\\
1016	3.053448\\
1017	1.723285\\
1018	0.011443\\
1019	3e-06\\
1020	0\\
1021	1e-06\\
1022	0\\
1023	3e-06\\
1024	2e-05\\
1025	2e-06\\
1026	1.291911\\
1027	1.718756\\
1028	0.068331\\
1029	5e-05\\
1030	1e-06\\
1031	7e-06\\
1032	3e-06\\
1033	1e-06\\
1034	0\\
1035	0\\
1036	0\\
1037	0\\
1038	1e-06\\
1039	0\\
1040	4e-06\\
1041	0.002862\\
1042	2.901425\\
1043	2.910001\\
1044	3.378017\\
1045	3.358001\\
1046	3.630718\\
1047	3.358002\\
1048	3.358001\\
1049	3.536167\\
1050	4.158963\\
1051	3.664101\\
1052	3.358001\\
1053	2.3772\\
1054	0.765745\\
1055	1.109988\\
1056	0.509128\\
1057	2e-06\\
1058	0\\
1059	0\\
1060	0\\
1061	8e-06\\
1062	1e-06\\
1063	1.798535\\
1064	2.379916\\
1065	0.823262\\
1066	0.308136\\
1067	3.7e-05\\
1068	0.023844\\
1069	0.305983\\
1070	1.083575\\
1071	1.220152\\
1072	0.819437\\
1073	0.809976\\
1074	2.049086\\
1075	0.291721\\
1076	1e-05\\
1077	5e-06\\
1078	5e-06\\
1079	0\\
1080	0\\
1081	1e-06\\
1082	0\\
1083	0\\
1084	4e-06\\
1085	2e-06\\
1086	2e-06\\
1087	0\\
1088	0\\
1089	0\\
1090	9e-06\\
1091	1e-06\\
1092	1e-06\\
1093	0\\
1094	0\\
1095	7e-06\\
1096	0\\
1097	8e-06\\
1098	0\\
1099	1e-06\\
1100	0\\
1101	0\\
1102	0\\
1103	1e-06\\
1104	1e-06\\
1105	0\\
1106	0\\
1107	0\\
1108	2e-06\\
1109	3e-06\\
1110	2e-06\\
1111	4e-06\\
1112	0\\
1113	0\\
1114	0\\
1115	0\\
1116	7e-06\\
1117	0\\
1118	0\\
1119	0\\
1120	0\\
1121	8e-06\\
1122	1e-06\\
1123	4e-06\\
1124	1e-06\\
1125	1e-06\\
1126	1e-06\\
1127	1e-06\\
1128	1e-06\\
1129	0\\
1130	0\\
1131	8e-06\\
1132	0\\
1133	0\\
1134	0\\
1135	1.7e-05\\
1136	2.908612\\
1137	2.197431\\
1138	0.757512\\
1139	0.306036\\
1140	0.767711\\
1141	5e-06\\
1142	1e-06\\
1143	1.3e-05\\
1144	1.2e-05\\
1145	2e-05\\
1146	0.148303\\
1147	2.910001\\
1148	2.910001\\
1149	2.382677\\
1150	0.294341\\
1151	0.125237\\
1152	3e-06\\
1153	0\\
1154	1e-06\\
1155	5e-06\\
1156	0\\
1157	1.1e-05\\
1158	1e-06\\
1159	3.6e-05\\
1160	0.987208\\
1161	0.423428\\
1162	1.059701\\
1163	2.05043\\
1164	2.910001\\
1165	2.91\\
1166	2.910001\\
1167	2.910001\\
1168	2.910001\\
1169	2.910001\\
1170	3.530949\\
1171	3.98308\\
1172	2.991439\\
1173	2.887698\\
1174	0.008352\\
1175	0.647297\\
1176	0.001832\\
1177	3e-06\\
1178	5e-06\\
1179	1e-06\\
1180	1e-06\\
1181	1e-06\\
1182	1e-06\\
1183	0.30624\\
1184	2.910001\\
1185	2.55182\\
1186	2.874642\\
1187	2.931737\\
1188	3.377989\\
1189	2.910001\\
1190	2.910004\\
1191	2.910001\\
1192	2.910003\\
1193	2.910001\\
1194	3.357995\\
1195	3.743931\\
1196	3.358005\\
1197	3.117159\\
1198	1.870867\\
1199	2.264529\\
1200	0.972408\\
1201	3e-06\\
1202	0\\
1203	1e-06\\
1204	8e-06\\
1205	0\\
1206	1e-06\\
1207	4e-06\\
1208	2.8e-05\\
1209	4e-06\\
1210	8e-06\\
1211	1.8e-05\\
1212	0.604556\\
1213	0.175843\\
1214	0.348222\\
1215	0.306021\\
1216	0.306001\\
1217	0.809518\\
1218	1.660652\\
1219	2.764377\\
1220	1.349112\\
1221	0.116587\\
1222	1.7e-05\\
1223	2.1e-05\\
1224	2e-06\\
1225	0\\
1226	0\\
1227	1e-06\\
1228	2e-06\\
1229	1e-06\\
1230	0\\
1231	6e-06\\
1232	0.780758\\
1233	0.073225\\
1234	8.8e-05\\
1235	0.306027\\
1236	0.305985\\
1237	1e-06\\
1238	8e-06\\
1239	3e-06\\
1240	3e-06\\
1241	1e-06\\
1242	5e-06\\
1243	2.281154\\
1244	0.772019\\
1245	3e-06\\
1246	0\\
1247	2e-06\\
1248	3e-06\\
1249	0\\
1250	2e-06\\
1251	0\\
1252	0\\
1253	0\\
1254	0\\
1255	2e-06\\
1256	0\\
1257	1e-06\\
1258	1e-06\\
1259	8e-06\\
1260	1e-06\\
1261	0\\
1262	1e-06\\
1263	1e-06\\
1264	1e-06\\
1265	1e-06\\
1266	9e-06\\
1267	0\\
1268	3e-06\\
1269	3e-06\\
1270	1e-06\\
1271	1e-06\\
1272	2e-06\\
1273	0\\
1274	0\\
1275	0\\
1276	2e-06\\
1277	1e-06\\
1278	3e-06\\
1279	4e-06\\
1280	2e-06\\
1281	1e-06\\
1282	2e-06\\
1283	2e-06\\
1284	0\\
1285	1e-06\\
1286	2e-06\\
1287	1e-06\\
1288	0\\
1289	0\\
1290	4e-06\\
1291	1e-06\\
1292	0\\
1293	0\\
1294	1e-06\\
1295	1e-06\\
1296	1e-06\\
1297	9e-06\\
1298	0\\
1299	3e-06\\
1300	0\\
1301	3e-06\\
1302	1e-06\\
1303	1.2e-05\\
1304	0\\
1305	1e-06\\
1306	0\\
1307	1.1e-05\\
1308	0\\
1309	0\\
1310	1e-06\\
1311	1e-06\\
1312	2e-06\\
1313	0\\
1314	1e-06\\
1315	0.305734\\
1316	0.312512\\
1317	5e-06\\
1318	2e-06\\
1319	0\\
1320	0\\
1321	1e-06\\
1322	6e-06\\
1323	0\\
1324	0\\
1325	0\\
1326	0\\
1327	8e-06\\
1328	1.7e-05\\
1329	1e-06\\
1330	3e-06\\
1331	6e-06\\
1332	1e-06\\
1333	1e-06\\
1334	1e-06\\
1335	4e-06\\
1336	0.000109\\
1337	6e-06\\
1338	0.498101\\
1339	1.74823\\
1340	1.455472\\
1341	1e-06\\
1342	2e-06\\
1343	0\\
1344	2e-06\\
1345	1e-05\\
1346	1e-06\\
1347	1e-06\\
1348	1e-06\\
1349	1e-06\\
1350	1e-06\\
1351	2.9e-05\\
1352	1.146255\\
1353	1.04349\\
1354	2.17768\\
1355	1.376336\\
1356	0.814649\\
1357	3e-06\\
1358	4e-05\\
1359	2e-06\\
1360	0.038946\\
1361	4e-06\\
1362	0.158223\\
1363	2.910001\\
1364	2.910001\\
1365	1.131842\\
1366	0.013413\\
1367	0.306018\\
1368	1.2e-05\\
1369	6e-06\\
1370	1e-06\\
1371	1e-06\\
1372	0\\
1373	0\\
1374	1e-06\\
1375	1e-06\\
1376	1e-06\\
1377	3e-06\\
1378	0\\
1379	3e-06\\
1380	2e-06\\
1381	3e-06\\
1382	0.302492\\
1383	0.410332\\
1384	0.578779\\
1385	0.814663\\
1386	1.920933\\
1387	2.91\\
1388	2.910001\\
1389	2.836604\\
1390	1.78486\\
1391	2.484364\\
1392	1.684186\\
1393	0\\
1394	1.3e-05\\
1395	1e-06\\
1396	1e-06\\
1397	1e-06\\
1398	2e-06\\
1399	1.436188\\
1400	2.459968\\
1401	1.596261\\
1402	2.540869\\
1403	2.910001\\
1404	3.328104\\
1405	3.378001\\
1406	3.578228\\
1407	3.358042\\
1408	3.357997\\
1409	3.072901\\
1410	3.283204\\
1411	3.281978\\
1412	2.964777\\
1413	2.910002\\
1414	1.194519\\
1415	2.909733\\
1416	3.120666\\
1417	0.000835\\
1418	1e-06\\
1419	2e-06\\
1420	1e-06\\
1421	0\\
1422	0\\
1423	6e-06\\
1424	1e-06\\
1425	2e-06\\
1426	3e-06\\
1427	1.1e-05\\
1428	3.2e-05\\
1429	0.437303\\
1430	1e-06\\
1431	6.4e-05\\
1432	3.3e-05\\
1433	1.7e-05\\
1434	0\\
1435	0.774585\\
1436	0.774602\\
1437	1.5e-05\\
1438	3e-06\\
1439	4e-06\\
1440	3e-06\\
1441	1e-06\\
1442	0\\
1443	0\\
1444	0\\
1445	1e-06\\
1446	1e-06\\
1447	1e-06\\
1448	2e-06\\
1449	0\\
1450	0\\
1451	0\\
1452	4e-06\\
1453	1e-06\\
1454	0\\
1455	0\\
1456	1e-06\\
1457	4e-06\\
1458	5e-06\\
1459	0\\
1460	4e-06\\
1461	0\\
1462	5e-06\\
1463	4e-06\\
1464	3e-06\\
1465	0\\
1466	6e-06\\
1467	5e-06\\
1468	0\\
1469	1e-06\\
1470	2e-06\\
1471	1.2e-05\\
1472	2.6e-05\\
1473	2.2e-05\\
1474	4e-05\\
1475	2.1e-05\\
1476	7e-05\\
1477	0\\
1478	3.6e-05\\
1479	1.4e-05\\
1480	2e-06\\
1481	5.3e-05\\
1482	1.478165\\
1483	1.737808\\
1484	2.384167\\
1485	0.883219\\
1486	0.000111\\
1487	0.980438\\
1488	1.091235\\
1489	1.4e-05\\
1490	1e-06\\
1491	0\\
1492	0\\
1493	4e-06\\
1494	2.9e-05\\
1495	2.683771\\
1496	3.12066\\
1497	2.766925\\
1498	3.017474\\
1499	4.050909\\
1500	4.051174\\
1501	3.697474\\
1502	3.697474\\
1503	3.697481\\
1504	4.070875\\
1505	4.123447\\
1506	4.736418\\
1507	4.736418\\
1508	4.736418\\
1509	4.166542\\
1510	3.214073\\
1511	4.185822\\
1512	3.877884\\
1513	2.5877\\
1514	0.343442\\
1515	2e-06\\
1516	1.8e-05\\
1517	6.6e-05\\
1518	0.117646\\
1519	3.120641\\
1520	4.206064\\
1521	3.702315\\
1522	3.120664\\
1523	3.055142\\
1524	2.766926\\
1525	2.227478\\
1526	1.960536\\
1527	1.977953\\
1528	2.766925\\
1529	2.766926\\
1530	3.476226\\
1531	3.120635\\
1532	4.413129\\
1533	4.736415\\
1534	3.211831\\
1535	3.912693\\
1536	3.192876\\
1537	1.881464\\
1538	2e-06\\
1539	1e-06\\
1540	1e-06\\
1541	0\\
1542	1e-06\\
1543	1.727188\\
1544	0.914534\\
1545	0.731007\\
1546	9.9e-05\\
1547	0.29096\\
1548	4e-06\\
1549	1e-06\\
1550	0\\
1551	1e-06\\
1552	1e-06\\
1553	0.000103\\
1554	2.011314\\
1555	2.495534\\
1556	2.869362\\
1557	3.120597\\
1558	1.08908\\
1559	1.465024\\
1560	0.625508\\
1561	2e-06\\
1562	0\\
1563	1e-06\\
1564	1e-06\\
1565	1e-06\\
1566	2e-06\\
1567	7e-06\\
1568	3e-06\\
1569	1.3e-05\\
1570	1e-06\\
1571	5.3e-05\\
1572	2e-06\\
1573	2e-06\\
1574	1e-06\\
1575	2.3e-05\\
1576	6.3e-05\\
1577	1e-06\\
1578	1.023692\\
1579	0.99249\\
1580	2.19495\\
1581	2.766925\\
1582	0.367735\\
1583	1.101701\\
1584	1.14167\\
1585	3.7e-05\\
1586	1e-06\\
1587	1e-06\\
1588	1e-06\\
1589	1e-06\\
1590	1e-06\\
1591	1e-06\\
1592	1e-06\\
1593	0\\
1594	0\\
1595	1e-06\\
1596	0\\
1597	1e-06\\
1598	0\\
1599	0\\
1600	0\\
1601	0\\
1602	0\\
1603	8e-06\\
1604	0\\
1605	0\\
1606	1e-06\\
1607	0\\
1608	9e-06\\
1609	1e-06\\
1610	0\\
1611	1e-06\\
1612	1e-06\\
1613	1e-06\\
1614	1e-06\\
1615	7e-06\\
1616	0\\
1617	6e-06\\
1618	7e-06\\
1619	1e-06\\
1620	1e-06\\
1621	1e-06\\
1622	1e-06\\
1623	4e-06\\
1624	1e-06\\
1625	1e-06\\
1626	0\\
1627	1e-06\\
1628	7e-06\\
1629	1e-06\\
1630	0\\
1631	1.1e-05\\
1632	1e-06\\
1633	2e-06\\
1634	0\\
1635	2e-06\\
1636	2e-06\\
1637	5e-06\\
1638	1e-06\\
1639	2.234604\\
1640	2.85992\\
1641	2.179978\\
1642	1.332821\\
1643	1.430584\\
1644	1.134551\\
1645	2e-06\\
1646	8e-06\\
1647	2e-06\\
1648	0.204573\\
1649	1.171384\\
1650	3.743967\\
1651	4.364325\\
1652	3.938352\\
1653	3.360317\\
1654	1.715918\\
1655	1.864914\\
1656	1.171424\\
1657	0\\
1658	0\\
1659	4e-06\\
1660	6e-06\\
1661	2e-06\\
1662	0\\
1663	6.1e-05\\
1664	0.094664\\
1665	0.772766\\
1666	1.167866\\
1667	3.229675\\
1668	2.654749\\
1669	0.222262\\
1670	1.9e-05\\
1671	0\\
1672	1e-06\\
1673	2.2e-05\\
1674	0.942998\\
1675	3.138931\\
1676	2.770973\\
1677	2.780109\\
1678	0.34391\\
1679	0.462832\\
1680	0.271862\\
1681	9e-06\\
1682	1e-05\\
1683	1e-06\\
1684	1e-06\\
1685	0\\
1686	0\\
1687	1.807783\\
1688	1.43971\\
1689	0.753695\\
1690	2.452803\\
1691	0.462838\\
1692	0.000879\\
1693	1e-06\\
1694	2e-06\\
1695	7e-06\\
1696	1.9e-05\\
1697	0.100722\\
1698	2.477711\\
1699	3.195315\\
1700	3.655672\\
1701	3.229665\\
1702	2.9026\\
1703	3.22968\\
1704	1.262968\\
1705	6e-06\\
1706	1e-06\\
1707	1e-06\\
1708	1.1e-05\\
1709	0\\
1710	0\\
1711	3.229651\\
1712	3.583404\\
1713	3.229666\\
1714	2.77903\\
1715	2.25964\\
1716	1.030485\\
1717	0.00086\\
1718	0.000188\\
1719	0.000173\\
1720	0.000148\\
1721	0.358324\\
1722	1.73109\\
1723	2.174436\\
1724	3.445358\\
1725	3.229666\\
1726	2.337951\\
1727	2.683367\\
1728	1.237358\\
1729	4e-06\\
1730	1e-06\\
1731	2e-06\\
1732	0\\
1733	2e-06\\
1734	1e-06\\
1735	1.906736\\
1736	2.104039\\
1737	1.396853\\
1738	0.254908\\
1739	1.237358\\
1740	0.506797\\
1741	5.3e-05\\
1742	1e-06\\
1743	1e-05\\
1744	2e-06\\
1745	2.9e-05\\
1746	1.237231\\
1747	2.621203\\
1748	3.22968\\
1749	2.046035\\
1750	3.229692\\
1751	3.229664\\
1752	2.837721\\
1753	1.2e-05\\
1754	1e-06\\
1755	0\\
1756	0\\
1757	0\\
1758	0\\
1759	3e-06\\
1760	0\\
1761	6e-06\\
1762	3e-05\\
1763	2.7e-05\\
1764	0.000639\\
1765	5.4e-05\\
1766	4e-06\\
1767	1e-06\\
1768	0\\
1769	2.1e-05\\
1770	2e-06\\
1771	0.290908\\
1772	1.251716\\
1773	0.291049\\
1774	5.8e-05\\
1775	1e-06\\
1776	1.1e-05\\
1777	0\\
1778	6e-06\\
1779	8e-06\\
1780	3e-06\\
1781	4e-06\\
1782	1e-06\\
1783	1e-06\\
1784	1e-06\\
1785	7e-06\\
1786	1e-06\\
1787	0\\
1788	0\\
1789	3e-06\\
1790	0\\
1791	1e-06\\
1792	1e-06\\
1793	1e-06\\
1794	3e-06\\
1795	3e-06\\
1796	0.777622\\
1797	0.774615\\
1798	1e-06\\
1799	5.1e-05\\
1800	4.2e-05\\
1801	0\\
1802	1e-06\\
1803	0\\
1804	2e-06\\
1805	1e-06\\
1806	0\\
1807	2.766925\\
1808	3.120815\\
1809	3.073706\\
1810	3.292861\\
1811	4.736418\\
1812	4.736418\\
1813	3.766922\\
1814	3.311323\\
1815	3.229304\\
1816	3.873433\\
1817	4.348276\\
1818	4.736418\\
1819	4.736418\\
1820	4.736418\\
1821	4.736418\\
1822	4.288881\\
1823	4.044247\\
1824	3.192899\\
1825	1.297269\\
1826	1.2e-05\\
1827	0\\
1828	2e-06\\
1829	0\\
1830	3e-06\\
1831	3.120649\\
1832	4.736417\\
1833	4.247956\\
1834	4.736416\\
1835	3.988804\\
1836	4.736417\\
1837	4.312165\\
1838	4.247681\\
1839	3.241846\\
1840	3.155934\\
1841	3.150678\\
1842	3.823715\\
1843	4.044211\\
1844	3.316512\\
1845	2.766929\\
1846	2.109406\\
1847	2.620474\\
1848	1.202422\\
1849	0\\
1850	0\\
1851	0\\
1852	6e-06\\
1853	7e-06\\
1854	4e-06\\
1855	2.766929\\
1856	4.377601\\
1857	2.766925\\
1858	2.652362\\
1859	4.466358\\
1860	4.244193\\
1861	3.211941\\
1862	3.092966\\
1863	2.766925\\
1864	2.843634\\
1865	3.192899\\
1866	4.736418\\
1867	4.25431\\
1868	4.736418\\
1869	4.736418\\
1870	4.736419\\
1871	3.597719\\
1872	2.852355\\
1873	1.576119\\
1874	4e-06\\
1875	2e-06\\
1876	4e-06\\
1877	1.5e-05\\
1878	1.7e-05\\
1879	0.040341\\
1880	2.127386\\
1881	4.3e-05\\
1882	3e-06\\
1883	3e-05\\
1884	1.1e-05\\
1885	0\\
1886	8e-06\\
1887	0\\
1888	9e-06\\
1889	7e-06\\
1890	1.8e-05\\
1891	6e-05\\
1892	0.774615\\
1893	2.304263\\
1894	0.77461\\
1895	0.517176\\
1896	7.6e-05\\
1897	0\\
1898	8e-06\\
1899	5e-06\\
1900	1e-06\\
1901	1e-06\\
1902	7e-06\\
1903	1.63212\\
1904	3.211928\\
1905	3.633837\\
1906	4.23914\\
1907	4.618266\\
1908	4.736418\\
1909	4.736418\\
1910	4.04396\\
1911	2.864875\\
1912	2.766925\\
1913	2.766925\\
1914	2.767138\\
1915	2.209592\\
1916	2.766935\\
1917	2.766924\\
1918	2.766925\\
1919	2.766925\\
1920	2.40766\\
1921	1e-05\\
1922	0\\
1923	0\\
1924	1e-06\\
1925	1e-06\\
1926	1e-06\\
1927	1e-06\\
1928	1e-06\\
1929	0\\
1930	1e-06\\
1931	1e-06\\
1932	1.8e-05\\
1933	0\\
1934	0\\
1935	0\\
1936	1e-06\\
1937	1.2e-05\\
1938	3e-06\\
1939	0\\
1940	6.6e-05\\
1941	1.7e-05\\
1942	2.7e-05\\
1943	2.3e-05\\
1944	1e-06\\
1945	0\\
1946	1e-06\\
1947	0\\
1948	1e-06\\
1949	0\\
1950	0\\
1951	0\\
1952	0\\
1953	1e-05\\
1954	1e-06\\
1955	0\\
1956	1e-06\\
1957	1e-06\\
1958	2e-06\\
1959	7e-06\\
1960	0\\
1961	0\\
1962	8e-06\\
1963	0.427517\\
1964	0.290959\\
1965	0.323756\\
1966	5e-06\\
1967	1e-06\\
1968	0.759362\\
1969	7e-06\\
1970	0\\
1971	1e-06\\
1972	1e-06\\
1973	0\\
1974	3e-06\\
1975	2.766935\\
1976	4.64358\\
1977	2.892299\\
1978	2.766925\\
1979	2.766925\\
1980	2.822834\\
1981	2.766926\\
1982	2.766926\\
1983	2.766907\\
1984	2.867983\\
1985	3.192898\\
1986	4.736414\\
1987	4.335996\\
1988	4.736419\\
1989	4.736418\\
1990	3.801943\\
1991	4.603886\\
1992	3.302205\\
1993	1.491037\\
1994	1.3e-05\\
1995	3e-06\\
1996	2.7e-05\\
1997	1e-06\\
1998	0.423996\\
1999	3.19292\\
2000	2.975045\\
2001	3.812145\\
2002	3.192897\\
2003	2.766925\\
2004	2.557748\\
2005	1.559786\\
2006	1.734382\\
2007	2.09597\\
2008	2.766925\\
2009	2.951325\\
2010	3.476758\\
2011	4.734325\\
2012	4.736408\\
2013	4.473075\\
2014	3.120619\\
2015	2.766925\\
2016	2.766926\\
2017	0.727059\\
2018	2.7e-05\\
2019	2e-06\\
2020	1e-06\\
2021	6e-06\\
2022	0.290949\\
2023	3.211881\\
2024	3.701726\\
2025	3.192886\\
2026	3.120639\\
2027	3.192898\\
2028	3.192997\\
2029	3.099167\\
2030	4.736419\\
2031	4.736419\\
2032	4.783027\\
2033	5.595124\\
2034	6.735232\\
2035	6.856049\\
2036	8.962555\\
2037	8.952663\\
2038	8.236019\\
2039	8.382214\\
2040	8.955037\\
2041	7.797634\\
2042	5.865759\\
2043	5.666968\\
2044	5.666966\\
2045	4.736418\\
2046	5.115018\\
2047	7.319515\\
2048	8.059062\\
2049	7.910881\\
2050	7.851227\\
2051	7.87607\\
2052	7.347405\\
2053	6.88949\\
2054	7.263303\\
2055	7.319512\\
2056	7.319515\\
2057	7.791801\\
2058	9.221182\\
2059	12.111623\\
2060	10.979403\\
2061	10.151731\\
2062	8.122239\\
2063	8.663571\\
2064	7.87607\\
2065	6.796293\\
2066	5.197275\\
2067	4.736418\\
2068	4.736418\\
2069	4.736418\\
2070	5.739515\\
2071	7.66187\\
2072	7.869636\\
2073	7.477012\\
2074	7.454123\\
2075	7.319515\\
2076	8.250063\\
2077	8.250061\\
2078	8.250064\\
2079	7.17343\\
2080	7.148995\\
2081	7.296012\\
2082	7.319515\\
2083	8.160263\\
2084	7.319524\\
2085	7.509736\\
2086	7.319515\\
2087	7.319515\\
2088	6.871417\\
2089	4.736418\\
2090	4.631078\\
2091	3.211915\\
2092	3.120648\\
2093	3.187032\\
2094	3.211974\\
2095	3.948583\\
2096	4.454762\\
2097	4.736418\\
2098	4.736418\\
2099	4.171027\\
2100	3.296123\\
2101	3.120635\\
2102	2.766925\\
2103	2.603943\\
2104	2.766925\\
2105	3.211902\\
2106	4.736418\\
2107	4.736418\\
2108	4.736418\\
2109	5.066392\\
2110	4.736418\\
2111	4.736418\\
2112	4.736418\\
2113	3.539816\\
2114	2.980858\\
2115	2.766925\\
2116	2.363928\\
2117	2.766914\\
2118	2.766925\\
2119	2.766925\\
2120	3.120638\\
2121	3.299113\\
2122	3.192898\\
2123	2.874625\\
2124	2.766925\\
2125	1.437761\\
2126	0.516268\\
2127	0.719873\\
2128	1.364898\\
2129	2.766927\\
2130	3.224712\\
2131	4.736418\\
2132	4.736418\\
2133	4.736418\\
2134	4.736418\\
2135	4.736418\\
2136	4.044966\\
2137	3.211979\\
2138	3.098977\\
2139	3.04731\\
2140	3.211913\\
2141	4.736418\\
2142	6.539734\\
2143	8.96254\\
2144	7.319515\\
2145	7.319519\\
2146	7.319519\\
2147	6.379967\\
2148	4.73643\\
2149	4.736418\\
2150	4.736418\\
2151	4.736917\\
2152	5.567349\\
2153	6.983056\\
2154	7.319515\\
2155	7.319516\\
2156	9.221182\\
2157	7.332759\\
2158	7.319517\\
2159	6.206406\\
2160	7.973345\\
2161	7.474611\\
2162	8.129961\\
2163	7.7875\\
2164	8.130012\\
2165	7.474593\\
2166	10.053835\\
2167	10.822495\\
2168	9.245077\\
2169	8.918103\\
2170	9.682344\\
2171	9.528343\\
2172	8.945967\\
2173	8.94698\\
2174	8.946202\\
2175	8.946997\\
2176	9.198503\\
2177	10.971625\\
2178	12.19851\\
2179	10.952136\\
2180	11.868064\\
2181	11.901236\\
2182	11.85788\\
2183	10.075113\\
2184	7.900633\\
2185	7.179101\\
2186	6.093014\\
2187	5.594659\\
2188	5.587513\\
2189	6.96548\\
2190	9.17878\\
2191	10.922583\\
2192	10.247325\\
2193	10.025746\\
2194	10.056024\\
2195	9.365257\\
2196	9.17877\\
2197	8.946992\\
2198	8.946996\\
2199	8.822735\\
2200	8.946996\\
2201	9.189145\\
2202	10.488523\\
2203	10.730427\\
2204	10.898295\\
2205	11.154769\\
2206	10.341382\\
2207	8.946999\\
2208	7.474595\\
2209	6.475757\\
2210	5.170483\\
2211	5.159796\\
2212	5.159809\\
2213	6.270765\\
2214	8.193929\\
2215	9.769932\\
2216	9.322064\\
2217	9.178765\\
2218	8.947032\\
2219	9.177426\\
2220	8.688332\\
2221	8.735857\\
2222	8.659662\\
2223	8.422666\\
2224	8.946996\\
2225	10.121121\\
2226	11.985571\\
2227	13.71412\\
2228	12.324093\\
2229	11.433752\\
2230	10.270278\\
2231	9.199329\\
2232	7.474593\\
2233	6.783195\\
2234	5.719895\\
2235	5.702534\\
2236	6.247607\\
2237	7.474602\\
2238	9.17878\\
2239	11.95536\\
2240	13.714117\\
2241	13.714131\\
2242	14.773528\\
2243	17.699593\\
2244	11.719869\\
2245	10.559524\\
2246	9.792125\\
2247	9.178777\\
2248	9.178765\\
2249	10.493844\\
2250	11.167941\\
2251	9.779215\\
2252	10.278418\\
2253	9.178767\\
2254	10.256929\\
2255	9.302962\\
2256	7.474594\\
2257	6.477089\\
2258	5.555496\\
2259	5.159771\\
2260	5.160017\\
2261	5.572781\\
2262	6.477162\\
2263	7.28297\\
2264	7.4746\\
2265	7.474599\\
2266	7.474598\\
2267	7.638656\\
2268	7.474593\\
2269	6.775641\\
2270	6.219779\\
2271	5.944769\\
2272	6.106729\\
2273	6.477087\\
2274	7.474594\\
2275	7.474597\\
2276	7.474593\\
2277	7.900636\\
2278	7.474593\\
2279	7.474593\\
2280	6.379508\\
2281	5.159784\\
2282	5.159794\\
2283	4.771994\\
2284	4.312289\\
2285	4.647964\\
2286	5.159792\\
2287	4.822592\\
2288	5.159794\\
2289	5.159793\\
2290	5.15979\\
2291	5.159927\\
2292	5.745486\\
2293	5.159772\\
2294	5.159782\\
2295	5.159788\\
2296	5.159787\\
2297	5.189724\\
2298	6.477087\\
2299	7.032287\\
2300	7.474594\\
2301	7.474593\\
2302	7.21339\\
2303	6.477088\\
2304	5.159919\\
2305	5.159791\\
2306	5.158872\\
2307	4.984638\\
2308	5.159794\\
2309	5.159771\\
2310	7.474593\\
2311	7.97311\\
2312	9.178785\\
2313	9.378528\\
2314	9.93342\\
2315	9.713376\\
2316	8.946963\\
2317	8.946994\\
2318	8.885535\\
2319	8.940177\\
2320	8.940136\\
2321	8.965064\\
2322	8.940164\\
2323	8.798845\\
2324	8.946994\\
2325	9.211215\\
2326	9.178622\\
2327	7.973423\\
2328	6.477149\\
2329	5.159777\\
2330	5.15978\\
2331	5.159781\\
2332	5.159778\\
2333	6.247657\\
2334	7.870852\\
2335	8.579695\\
2336	9.08729\\
2337	8.947048\\
2338	8.940177\\
2339	8.947049\\
2340	7.973383\\
2341	7.474656\\
2342	7.474654\\
2343	7.639158\\
2344	8.108558\\
2345	8.998629\\
2346	8.28505\\
2347	8.946994\\
2348	11.742396\\
2349	11.930724\\
2350	12.149589\\
2351	10.523778\\
2352	8.834697\\
2353	7.609105\\
2354	7.501845\\
2355	7.104364\\
2356	7.580045\\
2357	8.274854\\
2358	10.983621\\
2359	11.867426\\
2360	12.410987\\
2361	12.446904\\
2362	11.28143\\
2363	10.931555\\
2364	9.081505\\
2365	9.081504\\
2366	8.337301\\
2367	8.400857\\
2368	9.081434\\
2369	9.42778\\
2370	10.987837\\
2371	10.889569\\
2372	10.333269\\
2373	11.594555\\
2374	11.180922\\
2375	10.907469\\
2376	9.081426\\
2377	7.609105\\
2378	7.609105\\
2379	7.411295\\
2380	7.609104\\
2381	8.087956\\
2382	11.136562\\
2383	13.848627\\
2384	13.848627\\
2385	12.444501\\
2386	11.856335\\
2387	12.223453\\
2388	9.671617\\
2389	9.844599\\
2390	9.549415\\
2391	9.667179\\
2392	10.049922\\
2393	11.636503\\
2394	12.832116\\
2395	11.128998\\
2396	11.774664\\
2397	12.316384\\
2398	11.85309\\
2399	11.009543\\
2400	8.284803\\
2401	7.474593\\
2402	6.922947\\
2403	6.808551\\
2404	7.016935\\
2405	7.496596\\
2406	9.292349\\
2407	11.067072\\
2408	11.948898\\
2409	11.593132\\
2410	11.292228\\
2411	10.765686\\
2412	9.468356\\
2413	8.947\\
2414	8.940174\\
2415	8.947005\\
2416	8.947002\\
2417	9.178768\\
2418	9.621545\\
2419	9.178768\\
2420	9.369398\\
2421	10.025947\\
2422	10.101548\\
2423	8.963708\\
2424	7.553584\\
2425	7.474593\\
2426	6.477108\\
2427	6.277154\\
2428	6.417149\\
2429	6.477104\\
2430	6.969409\\
2431	7.474595\\
2432	7.474595\\
2433	8.002284\\
2434	7.810061\\
2435	7.474596\\
2436	6.477088\\
2437	5.390733\\
2438	5.1598\\
2439	5.159813\\
2440	5.704022\\
2441	6.719911\\
2442	7.474595\\
2443	7.474593\\
2444	7.474593\\
2445	7.474593\\
2446	7.474592\\
2447	7.474597\\
2448	4.780425\\
2449	3.829476\\
2450	3.432931\\
2451	2.970977\\
2452	2.820556\\
2453	3.419645\\
2454	3.529084\\
2455	3.52901\\
2456	4.164958\\
2457	4.101374\\
2458	3.910767\\
2459	4.093421\\
2460	3.911014\\
2461	3.528957\\
2462	1.975595\\
2463	1.743849\\
2464	2.182976\\
2465	3.529005\\
2466	4.782102\\
2467	5.159798\\
2468	5.159919\\
2469	5.505218\\
2470	5.159859\\
2471	5.159794\\
2472	3.846286\\
2473	3.366205\\
2474	2.447841\\
2475	1.800534\\
2476	2.156054\\
2477	3.529346\\
2478	5.159789\\
2479	5.698795\\
2480	6.477087\\
2481	6.477095\\
2482	6.477095\\
2483	6.474087\\
2484	5.591095\\
2485	5.572773\\
2486	5.159831\\
2487	5.159827\\
2488	5.15983\\
2489	6.24758\\
2490	6.83884\\
2491	7.4747\\
2492	7.352001\\
2493	7.474719\\
2494	7.47472\\
2495	7.474593\\
2496	5.568752\\
2497	5.159769\\
2498	5.159799\\
2499	5.159814\\
2500	5.159774\\
2501	5.416695\\
2502	7.474709\\
2503	8.065196\\
2504	8.940191\\
2505	8.62616\\
2506	7.879397\\
2507	7.474593\\
2508	7.474593\\
2509	7.321941\\
2510	6.840744\\
2511	6.543581\\
2512	6.477129\\
2513	7.321788\\
2514	7.474592\\
2515	7.865317\\
2516	7.888431\\
2517	7.973353\\
2518	8.712692\\
2519	7.972762\\
2520	6.96928\\
2521	5.579549\\
2522	5.159777\\
2523	5.159774\\
2524	5.159902\\
2525	6.301206\\
2526	7.708042\\
2527	8.10855\\
2528	7.487672\\
2529	7.474702\\
2530	7.474713\\
2531	7.047706\\
2532	6.477104\\
2533	5.430083\\
2534	5.458207\\
2535	5.625287\\
2536	6.354957\\
2537	7.34972\\
2538	7.695688\\
2539	7.973353\\
2540	8.148821\\
2541	7.973345\\
2542	8.607623\\
2543	7.973348\\
2544	6.582385\\
2545	5.159787\\
2546	5.159825\\
2547	5.159813\\
2548	5.159789\\
2549	5.159773\\
2550	6.694029\\
2551	6.95705\\
2552	7.474686\\
2553	7.474669\\
2554	6.969593\\
2555	7.20907\\
2556	5.680408\\
2557	5.159777\\
2558	5.159807\\
2559	5.159773\\
2560	5.159819\\
2561	6.312878\\
2562	7.010063\\
2563	7.474729\\
2564	7.474702\\
2565	7.474709\\
2566	7.474592\\
2567	7.474715\\
2568	5.843899\\
2569	5.159817\\
2570	5.159758\\
2571	4.718661\\
2572	4.78323\\
2573	5.159782\\
2574	6.262244\\
2575	6.477139\\
2576	7.103542\\
2577	6.709515\\
2578	6.477562\\
2579	6.477092\\
2580	6.104135\\
2581	5.60831\\
2582	5.159816\\
2583	5.159772\\
2584	5.15977\\
2585	5.159863\\
2586	6.221983\\
2587	6.247761\\
2588	5.572758\\
2589	6.477087\\
2590	7.464257\\
2591	7.321787\\
2592	5.159808\\
2593	5.159797\\
2594	4.306693\\
2595	3.923026\\
2596	4.299473\\
2597	4.610871\\
2598	4.671048\\
2599	5.034754\\
2600	5.15982\\
2601	5.159781\\
2602	5.15978\\
2603	5.159835\\
2604	5.159805\\
2605	4.238774\\
2606	3.529255\\
2607	3.274663\\
2608	3.529197\\
2609	3.911006\\
2610	5.128325\\
2611	5.159781\\
2612	5.159767\\
2613	4.992715\\
2614	4.777149\\
2615	4.783217\\
2616	3.846256\\
2617	2.678757\\
2618	1.310196\\
2619	1.062689\\
2620	0.823751\\
2621	1.062689\\
2622	1.493245\\
2623	1.559363\\
2624	3.304469\\
2625	3.529209\\
2626	3.529223\\
2627	3.529075\\
2628	3.529097\\
2629	3.443469\\
2630	2.380925\\
2631	1.743915\\
2632	1.743921\\
2633	3.529236\\
2634	4.168936\\
2635	5.159797\\
2636	5.159818\\
2637	5.159905\\
2638	5.1598\\
2639	5.159816\\
2640	5.096732\\
2641	3.910954\\
2642	3.529292\\
2643	3.072768\\
2644	3.09329\\
2645	3.529325\\
2646	3.528998\\
2647	3.528222\\
2648	3.529\\
2649	3.528973\\
2650	3.635745\\
2651	3.529227\\
2652	3.508631\\
2653	1.860648\\
2654	2.252122\\
2655	2.456721\\
2656	3.529302\\
2657	4.672582\\
2658	5.159884\\
2659	5.159795\\
2660	5.159903\\
2661	5.159797\\
2662	5.159792\\
2663	5.159952\\
2664	5.159779\\
2665	4.31392\\
2666	3.82871\\
2667	3.624408\\
2668	3.910297\\
2669	5.159792\\
2670	6.678662\\
2671	7.973357\\
2672	7.953136\\
2673	7.474606\\
2674	7.4746\\
2675	7.474695\\
2676	6.955327\\
2677	6.86826\\
2678	6.969549\\
2679	6.882028\\
2680	7.474612\\
2681	7.4761\\
2682	8.104195\\
2683	8.411595\\
2684	7.781323\\
2685	7.936699\\
2686	8.223408\\
2687	7.474596\\
2688	6.440197\\
2689	5.159832\\
2690	5.159791\\
2691	5.159792\\
2692	5.159784\\
2693	5.289869\\
2694	7.474592\\
2695	7.798865\\
2696	7.476689\\
2697	7.4746\\
2698	7.474694\\
2699	7.474594\\
2700	7.47462\\
2701	7.474617\\
2702	7.474701\\
2703	7.474594\\
2704	7.474619\\
2705	7.656785\\
2706	8.649962\\
2707	8.690628\\
2708	7.886934\\
2709	8.946997\\
2710	8.940176\\
2711	7.474595\\
2712	6.193821\\
2713	5.159831\\
2714	5.159798\\
2715	5.159813\\
2716	5.159806\\
2717	5.15979\\
2718	7.474729\\
2719	8.707427\\
2720	8.947006\\
2721	9.178808\\
2722	8.947072\\
2723	9.178771\\
2724	8.940171\\
2725	8.66654\\
2726	7.973598\\
2727	8.249706\\
2728	8.348402\\
2729	8.821179\\
2730	8.947005\\
2731	8.433132\\
2732	7.906816\\
2733	7.475235\\
2734	7.474658\\
2735	6.4771\\
2736	5.159798\\
2737	4.191186\\
2738	3.765113\\
2739	3.5292\\
2740	3.559739\\
2741	4.666575\\
2742	5.159794\\
2743	6.806059\\
2744	6.477067\\
2745	6.696496\\
2746	5.614896\\
2747	5.1598\\
2748	5.159806\\
2749	5.159784\\
2750	5.159861\\
2751	5.159909\\
2752	5.321027\\
2753	6.360502\\
2754	7.474697\\
2755	7.474636\\
2756	7.148756\\
2757	7.322411\\
2758	6.736143\\
2759	6.247601\\
2760	5.159806\\
2761	4.1411\\
2762	3.528963\\
2763	3.528626\\
2764	3.529203\\
2765	3.529081\\
2766	3.528931\\
2767	3.603413\\
2768	4.193073\\
2769	4.605946\\
2770	4.006687\\
2771	3.909849\\
2772	3.528337\\
2773	3.528103\\
2774	3.44172\\
2775	3.528936\\
2776	3.910915\\
2777	5.159297\\
2778	5.159801\\
2779	5.159927\\
2780	5.159814\\
2781	5.159914\\
2782	5.159794\\
2783	5.159769\\
2784	3.799668\\
2785	2.850044\\
2786	1.432795\\
2787	1.148549\\
2788	1.063086\\
2789	1.17152\\
2790	1.097702\\
2791	1.310482\\
2792	2.107\\
2793	3.137138\\
2794	3.272159\\
2795	3.529184\\
2796	3.529192\\
2797	2.445956\\
2798	2.170719\\
2799	2.158647\\
2800	2.596685\\
2801	3.529202\\
2802	5.194881\\
2803	5.815186\\
2804	5.815185\\
2805	5.815187\\
2806	5.815187\\
2807	5.815187\\
2808	4.469705\\
2809	4.050239\\
2810	3.862168\\
2811	3.508919\\
2812	4.050236\\
2813	5.273307\\
2814	6.31159\\
2815	7.976688\\
2816	8.130012\\
2817	7.311792\\
2818	7.331655\\
2819	8.130012\\
2820	8.130012\\
2821	8.130012\\
2822	8.130012\\
2823	8.130012\\
2824	8.130014\\
2825	8.271963\\
2826	8.628769\\
2827	8.976253\\
2828	8.611862\\
2829	8.130012\\
2830	8.355127\\
2831	8.130011\\
2832	6.065355\\
2833	5.815185\\
2834	5.815185\\
2835	5.811802\\
2836	5.815185\\
2837	5.815185\\
2838	7.132506\\
2839	8.130012\\
2840	8.773402\\
2841	9.59558\\
2842	9.078544\\
2843	9.602412\\
2844	9.289149\\
2845	9.602412\\
2846	9.602412\\
2847	9.595589\\
2848	9.438169\\
2849	7.973353\\
2850	8.108444\\
2851	7.474593\\
2852	7.474614\\
2853	7.474611\\
2854	7.474594\\
2855	6.477118\\
2856	5.159773\\
2857	5.159792\\
2858	4.782885\\
2859	4.668217\\
2860	4.782809\\
2861	5.159786\\
2862	6.477118\\
2863	7.474598\\
2864	7.638771\\
2865	7.973352\\
2866	7.973455\\
2867	7.620454\\
2868	7.474599\\
2869	7.474598\\
2870	6.915391\\
2871	6.477117\\
2872	6.477116\\
2873	7.034606\\
2874	7.41093\\
2875	7.299585\\
2876	6.891236\\
2877	7.474602\\
2878	7.474612\\
2879	6.791847\\
2880	5.608819\\
2881	4.804204\\
2882	4.802731\\
2883	4.714414\\
2884	4.376137\\
2885	4.376054\\
2886	3.946058\\
2887	3.662366\\
2888	3.716928\\
2889	4.196369\\
2890	4.426675\\
2891	4.802526\\
2892	4.803659\\
2893	4.544061\\
2894	4.801918\\
2895	4.801767\\
2896	4.782466\\
2897	4.789108\\
2898	5.774321\\
2899	5.989181\\
2900	5.931232\\
2901	6.123433\\
2902	6.923258\\
2903	6.442838\\
2904	4.789588\\
2905	4.655829\\
2906	3.671467\\
2907	3.601786\\
2908	3.601787\\
2909	4.015909\\
2910	4.785885\\
2911	4.802985\\
2912	4.803765\\
2913	4.804249\\
2914	4.803563\\
2915	4.802538\\
2916	4.583698\\
2917	3.503987\\
2918	3.304814\\
2919	3.304839\\
2920	3.395265\\
2921	4.384949\\
2922	4.800864\\
2923	4.803368\\
2924	4.803835\\
2925	4.804056\\
2926	4.802152\\
2927	4.804227\\
2928	4.543778\\
2929	3.304799\\
2930	3.29208\\
2931	3.179269\\
2932	3.179218\\
2933	3.304826\\
2934	3.304815\\
2935	3.662409\\
2936	4.544379\\
2937	4.801259\\
2938	4.543367\\
2939	3.678379\\
2940	3.601784\\
2941	3.30482\\
2942	3.30485\\
2943	3.30485\\
2944	3.3048\\
2945	4.395805\\
2946	4.803864\\
2947	4.804213\\
2948	4.80305\\
2949	4.803637\\
2950	4.803993\\
2951	4.803667\\
2952	4.804227\\
2953	4.552819\\
2954	3.725727\\
2955	3.662418\\
2956	3.662417\\
2957	3.756822\\
2958	3.662415\\
2959	4.00547\\
2960	4.033527\\
2961	3.773348\\
2962	3.662346\\
2963	3.304819\\
2964	3.179232\\
2965	2.251599\\
2966	1.465566\\
2967	1.632872\\
2968	2.939892\\
2969	3.66236\\
2970	4.803921\\
2971	4.787457\\
2972	4.914112\\
2973	4.787166\\
2974	5.246982\\
2975	4.788831\\
2976	4.803254\\
2977	4.035988\\
2978	3.413443\\
2979	3.304802\\
2980	3.662404\\
2981	4.790701\\
2982	5.989222\\
2983	7.319737\\
2984	6.923258\\
2985	7.125398\\
2986	6.484172\\
2987	5.709287\\
2988	4.79096\\
2989	4.790554\\
2990	4.812198\\
2991	4.803021\\
2992	5.757757\\
2993	6.923278\\
2994	6.954151\\
2995	7.390306\\
2996	6.923258\\
2997	6.923258\\
2998	6.923259\\
2999	5.989229\\
3000	4.786617\\
3001	3.758676\\
3002	3.431251\\
3003	3.304798\\
3004	3.601787\\
3005	4.764043\\
3006	4.78901\\
3007	6.923258\\
3008	7.076994\\
3009	7.948162\\
3010	7.621828\\
3011	7.503799\\
3012	6.923327\\
3013	6.923283\\
3014	6.3294\\
3015	6.435615\\
3016	6.782568\\
3017	6.923258\\
3018	6.923323\\
3019	7.233145\\
3020	6.923259\\
3021	6.923261\\
3022	6.92326\\
3023	6.893131\\
3024	4.809893\\
3025	4.814544\\
3026	4.781173\\
3027	4.78496\\
3028	4.814651\\
3029	4.811028\\
3030	5.992333\\
3031	7.32754\\
3032	7.894191\\
3033	7.517588\\
3034	7.322222\\
3035	7.39046\\
3036	7.100845\\
3037	6.976686\\
3038	6.923261\\
3039	6.923259\\
3040	6.780274\\
3041	6.923259\\
3042	6.923259\\
3043	7.1636\\
3044	6.923259\\
3045	6.923262\\
3046	7.441749\\
3047	6.923262\\
3048	4.811174\\
3049	4.811736\\
3050	4.777111\\
3051	4.476251\\
3052	4.782767\\
3053	4.815171\\
3054	6.172817\\
3055	6.92326\\
3056	6.92326\\
3057	6.923271\\
3058	7.775666\\
3059	8.302057\\
3060	7.637353\\
3061	8.291518\\
3062	8.200122\\
3063	8.295677\\
3064	8.295679\\
3065	8.302113\\
3066	8.302112\\
3067	7.362089\\
3068	6.92326\\
3069	6.909911\\
3070	6.78853\\
3071	5.989164\\
3072	4.812617\\
3073	4.781261\\
3074	4.089512\\
3075	4.149898\\
3076	4.150313\\
3077	4.811845\\
3078	5.593978\\
3079	6.958129\\
3080	7.04574\\
3081	6.923276\\
3082	6.92326\\
3083	6.92326\\
3084	5.989166\\
3085	4.862361\\
3086	4.817002\\
3087	4.817713\\
3088	4.817438\\
3089	4.789141\\
3090	5.142548\\
3091	5.293617\\
3092	4.803995\\
3093	4.804409\\
3094	6.009025\\
3095	5.796029\\
3096	4.798834\\
3097	4.816348\\
3098	4.767413\\
3099	4.165825\\
3100	4.105654\\
3101	4.150142\\
3102	4.165809\\
3103	4.781289\\
3104	4.813448\\
3105	5.142625\\
3106	5.774948\\
3107	5.10174\\
3108	4.806228\\
3109	4.805317\\
3110	4.805046\\
3111	4.806129\\
3112	4.014282\\
3113	4.250115\\
3114	4.614943\\
3115	4.543752\\
3116	3.662418\\
3117	3.677684\\
3118	4.30482\\
3119	3.723353\\
3120	3.304835\\
3121	1.633016\\
3122	0.301087\\
3123	9e-06\\
3124	0\\
3125	4.5e-05\\
3126	2e-06\\
3127	6e-06\\
3128	0.510113\\
3129	0.881688\\
3130	0.768894\\
3131	1.198699\\
3132	0.6202\\
3133	0.145743\\
3134	6e-05\\
3135	0\\
3136	1e-06\\
3137	1.200232\\
3138	3.179395\\
3139	3.300934\\
3140	3.30497\\
3141	3.304971\\
3142	3.644564\\
3143	3.304971\\
3144	2.386719\\
3145	2.058271\\
3146	1.214216\\
3147	1.092974\\
3148	2.090614\\
3149	3.304967\\
3150	4.957722\\
3151	4.957722\\
3152	5.962266\\
3153	5.976398\\
3154	6.604028\\
3155	6.648254\\
3156	5.557838\\
3157	6.691207\\
3158	6.380725\\
3159	6.661685\\
3160	7.125394\\
3161	7.125404\\
3162	7.27957\\
3163	7.838782\\
3164	8.504195\\
3165	8.497812\\
3166	8.504196\\
3167	7.718616\\
3168	6.164203\\
3169	4.804663\\
3170	4.803579\\
3171	4.803278\\
3172	4.803584\\
3173	4.802875\\
3174	6.62434\\
3175	6.923261\\
3176	8.302061\\
3177	8.302108\\
3178	7.082409\\
3179	7.39031\\
3180	8.30206\\
3181	7.228062\\
3182	6.923258\\
3183	7.104306\\
3184	7.076706\\
3185	6.923258\\
3186	7.808202\\
3187	7.516906\\
3188	7.322217\\
3189	7.318822\\
3190	8.302048\\
3191	7.334515\\
3192	6.296827\\
3193	4.786386\\
3194	4.804396\\
3195	4.803626\\
3196	4.804229\\
3197	5.62562\\
3198	6.923258\\
3199	8.295684\\
3200	8.044965\\
3201	7.076992\\
3202	7.390316\\
3203	7.322219\\
3204	7.076976\\
3205	6.923258\\
3206	6.923269\\
3207	6.790328\\
3208	6.470656\\
3209	6.923353\\
3210	6.923258\\
3211	6.923258\\
3212	6.923258\\
3213	6.923258\\
3214	7.70948\\
3215	6.994835\\
3216	5.232307\\
3217	4.804935\\
3218	4.803722\\
3219	4.803669\\
3220	4.804099\\
3221	4.924247\\
3222	6.923261\\
3223	7.390317\\
3224	8.081892\\
3225	8.622382\\
3226	7.810362\\
3227	7.379793\\
3228	6.596843\\
3229	5.500469\\
3230	5.15733\\
3231	5.142602\\
3232	5.989245\\
3233	6.923264\\
3234	6.923459\\
3235	7.249767\\
3236	6.923264\\
3237	6.923342\\
3238	7.762575\\
3239	6.923265\\
3240	5.69694\\
3241	4.804802\\
3242	4.803589\\
3243	4.803435\\
3244	4.803916\\
3245	4.786159\\
3246	6.644795\\
3247	7.48288\\
3248	7.390306\\
3249	6.977362\\
3250	6.923258\\
3251	6.780167\\
3252	5.552891\\
3253	4.843745\\
3254	4.786202\\
3255	4.804907\\
3256	4.785518\\
3257	5.958899\\
3258	6.780157\\
3259	6.923258\\
3260	6.911617\\
3261	6.646666\\
3262	6.92327\\
3263	6.923268\\
3264	6.192057\\
3265	4.957726\\
3266	4.957923\\
3267	4.95784\\
3268	4.957905\\
3269	4.95791\\
3270	4.957897\\
3271	4.957733\\
3272	4.995506\\
3273	5.344463\\
3274	4.957741\\
3275	4.957837\\
3276	4.957874\\
3277	3.986982\\
3278	3.599647\\
3279	3.6018\\
3280	4.547561\\
3281	4.957839\\
3282	4.960404\\
3283	5.976435\\
3284	6.055622\\
3285	6.191302\\
3286	5.976394\\
3287	6.191304\\
3288	4.957726\\
3289	4.357132\\
3290	3.601791\\
3291	3.305006\\
3292	3.304815\\
3293	3.304826\\
3294	3.304841\\
3295	3.30481\\
3296	3.379591\\
3297	3.304799\\
3298	3.304811\\
3299	3.304843\\
3300	3.08763\\
3301	2.250743\\
3302	1.788647\\
3303	1.633053\\
3304	2.45136\\
3305	3.304831\\
3306	4.326681\\
3307	4.802145\\
3308	4.803205\\
3309	4.803772\\
3310	4.804448\\
3311	4.803825\\
3312	3.678375\\
3313	3.304818\\
3314	3.179283\\
3315	3.09823\\
3316	3.304823\\
3317	3.678377\\
3318	4.804446\\
3319	5.957691\\
3320	6.780171\\
3321	6.785425\\
3322	6.923258\\
3323	6.923258\\
3324	5.989166\\
3325	5.989173\\
3326	5.989166\\
3327	5.989333\\
3328	6.0451\\
3329	6.923342\\
3330	6.923258\\
3331	6.923258\\
3332	6.773169\\
3333	5.989185\\
3334	6.781624\\
3335	5.774338\\
3336	4.957726\\
3337	4.957845\\
3338	3.895739\\
3339	3.675119\\
3340	4.212114\\
3341	4.957749\\
3342	5.219405\\
3343	7.059987\\
3344	7.125397\\
3345	7.125398\\
3346	7.125399\\
3347	7.125396\\
3348	6.585484\\
3349	6.654419\\
3350	6.915926\\
3351	7.125396\\
3352	7.125398\\
3353	7.634723\\
3354	8.485259\\
3355	8.504196\\
3356	7.753183\\
3357	8.497811\\
3358	7.699381\\
3359	7.1254\\
3360	6.651889\\
3361	5.367864\\
3362	4.957743\\
3363	4.957722\\
3364	4.957722\\
3365	4.957727\\
3366	6.981281\\
3367	7.125397\\
3368	7.682677\\
3369	8.504196\\
3370	7.69043\\
3371	8.579625\\
3372	8.504196\\
3373	8.240146\\
3374	8.439461\\
3375	8.504196\\
3376	8.500617\\
3377	8.497814\\
3378	8.42084\\
3379	7.906003\\
3380	7.200235\\
3381	7.279129\\
3382	7.592444\\
3383	7.125396\\
3384	4.957902\\
3385	3.179239\\
3386	2.250743\\
3387	2.250743\\
3388	2.825472\\
3389	3.304816\\
3390	4.629264\\
3391	4.852061\\
3392	5.928901\\
3393	5.989166\\
3394	5.989039\\
3395	5.233641\\
3396	4.785974\\
3397	4.963777\\
3398	5.147721\\
3399	5.390226\\
3400	5.989182\\
3401	6.481809\\
3402	6.92326\\
3403	6.899832\\
3404	6.065181\\
3405	6.913465\\
3406	6.923258\\
3407	5.989348\\
3408	4.804118\\
3409	4.406978\\
3410	3.601785\\
3411	3.545169\\
3412	3.678337\\
3413	4.544353\\
3414	4.786714\\
3415	6.923269\\
3416	6.859377\\
3417	6.923261\\
3418	6.923262\\
3419	6.923263\\
3420	6.748286\\
3421	6.872406\\
3422	6.194608\\
3423	5.989186\\
3424	5.989166\\
3425	6.92327\\
3426	6.747462\\
3427	6.394532\\
3428	5.777075\\
3429	4.853098\\
3430	5.924741\\
3431	5.514718\\
3432	4.79513\\
3433	4.798583\\
3434	3.828218\\
3435	3.311588\\
3436	3.30481\\
3437	3.304694\\
3438	3.304689\\
3439	3.452679\\
3440	4.545312\\
3441	4.545288\\
3442	4.375994\\
3443	3.718811\\
3444	3.662416\\
3445	3.3047\\
3446	2.250757\\
3447	2.250754\\
3448	3.304694\\
3449	3.678518\\
3450	4.376188\\
3451	4.795301\\
3452	4.795007\\
3453	4.271663\\
3454	4.795659\\
3455	3.678369\\
3456	3.678377\\
3457	3.304792\\
3458	3.10775\\
3459	2.892088\\
3460	3.005818\\
3461	2.663278\\
3462	2.884155\\
3463	3.304711\\
3464	3.306948\\
3465	3.600667\\
3466	3.304788\\
3467	3.282566\\
3468	2.91092\\
3469	2.250649\\
3470	1.661159\\
3471	2.250708\\
3472	3.236447\\
3473	3.678352\\
3474	4.795364\\
3475	4.800448\\
3476	4.795785\\
3477	4.798205\\
3478	4.80104\\
3479	4.79734\\
3480	3.96762\\
3481	3.304776\\
3482	3.179232\\
3483	3.179243\\
3484	3.304707\\
3485	3.640237\\
3486	4.801899\\
3487	7.125396\\
3488	7.783678\\
3489	8.504197\\
3490	8.504197\\
3491	8.678527\\
3492	8.191905\\
3493	7.122946\\
3494	6.979987\\
3495	7.034168\\
3496	7.100978\\
3497	7.390305\\
3498	7.322217\\
3499	6.923258\\
3500	6.518234\\
3501	5.989183\\
3502	6.273591\\
3503	5.749805\\
3504	4.801576\\
3505	4.795621\\
3506	4.37618\\
3507	4.2758\\
3508	4.546156\\
3509	4.797207\\
3510	5.142361\\
3511	6.923271\\
3512	7.390307\\
3513	8.302061\\
3514	8.519108\\
3515	8.302113\\
3516	6.92326\\
3517	7.345575\\
3518	7.322306\\
3519	8.269399\\
3520	8.295673\\
3521	8.295672\\
3522	8.295642\\
3523	7.319166\\
3524	6.923261\\
3525	6.923265\\
3526	6.459754\\
3527	5.989299\\
3528	4.799729\\
3529	4.800684\\
3530	4.780874\\
3531	4.651578\\
3532	4.72792\\
3533	4.800572\\
3534	5.774463\\
3535	6.977511\\
3536	7.516924\\
3537	8.302062\\
3538	8.51908\\
3539	8.371363\\
3540	8.519101\\
3541	8.302063\\
3542	8.117751\\
3543	7.541503\\
3544	7.390305\\
3545	7.745416\\
3546	7.735617\\
3547	7.306627\\
3548	6.923258\\
3549	6.923266\\
3550	7.127118\\
3551	6.923258\\
3552	5.14236\\
3553	4.801843\\
3554	4.794971\\
3555	4.311215\\
3556	3.678342\\
3557	3.541819\\
3558	3.304804\\
3559	3.601786\\
3560	3.761004\\
3561	4.546555\\
3562	4.795063\\
3563	4.798704\\
3564	4.797196\\
3565	4.376247\\
3566	3.601787\\
3567	3.306046\\
3568	3.304809\\
3569	3.601786\\
3570	4.624504\\
3571	4.801289\\
3572	4.801161\\
3573	4.802632\\
3574	4.800892\\
3575	4.800462\\
3576	3.91595\\
3577	3.304801\\
3578	3.179199\\
3579	3.179214\\
3580	3.179234\\
3581	3.30472\\
3582	3.662419\\
3583	4.79562\\
3584	4.795346\\
3585	4.796409\\
3586	4.79523\\
3587	4.799688\\
3588	4.036323\\
3589	3.601787\\
3590	3.612015\\
3591	3.304807\\
3592	3.607284\\
3593	3.662415\\
3594	4.797966\\
3595	4.802456\\
3596	4.801609\\
3597	4.799564\\
3598	4.802202\\
3599	4.801577\\
3600	4.802221\\
3601	3.678319\\
3602	3.30484\\
3603	3.304744\\
3604	3.304784\\
3605	3.304816\\
3606	3.304795\\
3607	3.963003\\
3608	4.548082\\
3609	4.800939\\
3610	4.019911\\
3611	3.976586\\
3612	3.927536\\
3613	3.662417\\
3614	3.381111\\
3615	3.360525\\
3616	3.69741\\
3617	4.797545\\
3618	4.802726\\
3619	4.810658\\
3620	4.81145\\
3621	4.801655\\
3622	4.787254\\
3623	4.788169\\
3624	4.940134\\
3625	4.47772\\
3626	4.479638\\
3627	4.922293\\
3628	4.922293\\
3629	4.922293\\
3630	4.922293\\
3631	4.922293\\
3632	4.922293\\
3633	5.271747\\
3634	5.38384\\
3635	5.272429\\
3636	4.922293\\
3637	4.922293\\
3638	4.922293\\
3639	4.660322\\
3640	4.922293\\
3641	5.110125\\
3642	6.52151\\
3643	6.752432\\
3644	6.881689\\
3645	6.881692\\
3646	6.881689\\
3647	6.881692\\
3648	6.037346\\
3649	5.27186\\
3650	4.922293\\
3651	4.922293\\
3652	4.966136\\
3653	5.819333\\
3654	6.881691\\
3655	8.32419\\
3656	8.918808\\
3657	8.644668\\
3658	8.32419\\
3659	8.168888\\
3660	7.786387\\
3661	8.064058\\
3662	8.12224\\
3663	8.12224\\
3664	8.128012\\
3665	8.985563\\
3666	8.32419\\
3667	8.420175\\
3668	8.128012\\
3669	8.128018\\
3670	9.364566\\
3671	8.128009\\
3672	6.881692\\
3673	6.615675\\
3674	6.037346\\
3675	5.865886\\
3676	5.84297\\
3677	6.037353\\
3678	6.881691\\
3679	8.854804\\
3680	9.875465\\
3681	10.165934\\
3682	9.2272\\
3683	9.075084\\
3684	8.121935\\
3685	7.303639\\
3686	7.557392\\
3687	8.128009\\
3688	8.324231\\
3689	8.324196\\
3690	9.838646\\
3691	9.815219\\
3692	8.574311\\
3693	8.128022\\
3694	9.077744\\
3695	8.128002\\
3696	6.881692\\
3697	13.522295\\
3698	13.266926\\
3699	13.256288\\
3700	13.522293\\
3701	13.522294\\
3702	14.760149\\
3703	14.909328\\
3704	15.964878\\
3705	16.091113\\
3706	16.907074\\
3707	16.462464\\
3708	16.907019\\
3709	16.872605\\
3710	12.163165\\
3711	11.965799\\
3712	11.70602\\
3713	12.018917\\
3714	15.762363\\
3715	14.914432\\
3716	14.909266\\
3717	14.837069\\
3718	7.910721\\
3719	6.881692\\
3720	5.686808\\
3721	4.922293\\
3722	4.922293\\
3723	4.922293\\
3724	4.922293\\
3725	4.922293\\
3726	6.03735\\
3727	7.059421\\
3728	8.122188\\
3729	7.836641\\
3730	7.020657\\
3731	6.881692\\
3732	6.881689\\
3733	6.615676\\
3734	6.394005\\
3735	6.037346\\
3736	6.615675\\
3737	6.881692\\
3738	7.444609\\
3739	8.122237\\
3740	7.46288\\
3741	7.303847\\
3742	8.129129\\
3743	7.876838\\
3744	6.881691\\
3745	6.454234\\
3746	5.98508\\
3747	5.864398\\
3748	6.037346\\
3749	6.615675\\
3750	7.303444\\
3751	9.602122\\
3752	10.200849\\
3753	8.911155\\
3754	8.277728\\
3755	8.128003\\
3756	7.850268\\
3757	7.303873\\
3758	6.881692\\
3759	6.881692\\
3760	7.020694\\
3761	7.85851\\
3762	8.128004\\
3763	8.12801\\
3764	7.41822\\
3765	6.881692\\
3766	7.744471\\
3767	6.881692\\
3768	6.264454\\
3769	5.533759\\
3770	4.922293\\
3771	4.922293\\
3772	4.922293\\
3773	4.322566\\
3774	4.551552\\
3775	4.922293\\
3776	5.457805\\
3777	6.037345\\
3778	6.19049\\
3779	6.118857\\
3780	6.013033\\
3781	5.687085\\
3782	4.922294\\
3783	5.088646\\
3784	5.687092\\
3785	6.037347\\
3786	6.615678\\
3787	6.88169\\
3788	6.881648\\
3789	6.88169\\
3790	6.88169\\
3791	6.88169\\
3792	6.067996\\
3793	5.340139\\
3794	4.922293\\
3795	4.922293\\
3796	4.922293\\
3797	4.922293\\
3798	4.922038\\
3799	4.922293\\
3800	5.260072\\
3801	5.687107\\
3802	5.84309\\
3803	5.687087\\
3804	5.271857\\
3805	4.922293\\
3806	4.551564\\
3807	4.285255\\
3808	4.922223\\
3809	4.922293\\
3810	5.687087\\
3811	5.843085\\
3812	6.037346\\
3813	6.14928\\
3814	6.037346\\
3815	6.037346\\
3816	5.271864\\
3817	4.922293\\
3818	4.548158\\
3819	4.413465\\
3820	4.08264\\
3821	3.751463\\
3822	3.765908\\
3823	4.770781\\
3824	4.922293\\
3825	4.922293\\
3826	4.922293\\
3827	4.92246\\
3828	4.922293\\
3829	4.922293\\
3830	4.922293\\
3831	4.922293\\
3832	4.922293\\
3833	5.610983\\
3834	6.037369\\
3835	6.615688\\
3836	6.881689\\
3837	6.87914\\
3838	6.881692\\
3839	6.881684\\
3840	9.447867\\
3841	8.842052\\
3842	8.647599\\
3843	8.061733\\
3844	8.076538\\
3845	8.731891\\
3846	9.786342\\
3847	11.579579\\
3848	14.417049\\
3849	14.96789\\
3850	14.583797\\
3851	14.967861\\
3852	13.987893\\
3853	13.729711\\
3854	13.810376\\
3855	13.065034\\
3856	13.008994\\
3857	13.859589\\
3858	14.967885\\
3859	14.944808\\
3860	13.510643\\
3861	12.506579\\
3862	13.769715\\
3863	12.290619\\
3864	10.552843\\
3865	13.944478\\
3866	13.522306\\
3867	13.522306\\
3868	13.651604\\
3869	13.941989\\
3870	23.430258\\
3871	26.142084\\
3872	27.237366\\
3873	27.123103\\
3874	25.995432\\
3875	25.600756\\
3876	25.131344\\
3877	24.885417\\
3878	24.04857\\
3879	23.998645\\
3880	25.106948\\
3881	26.154451\\
3882	14.169368\\
3883	15.108618\\
3884	14.980736\\
3885	13.829421\\
3886	14.967869\\
3887	13.719262\\
3888	10.932731\\
3889	10.108799\\
3890	9.686398\\
3891	9.686417\\
3892	9.686399\\
3893	9.970381\\
3894	11.224919\\
3895	15.033911\\
3896	15.889805\\
3897	14.55734\\
3898	14.456199\\
3899	13.566721\\
3900	11.757358\\
3901	12.05091\\
3902	11.561737\\
3903	11.80811\\
3904	13.173668\\
3905	13.916034\\
3906	15.248018\\
3907	15.178185\\
3908	14.915364\\
3909	14.23777\\
3910	15.239346\\
3911	13.599626\\
3912	23.494407\\
3913	22.119313\\
3914	21.754771\\
3915	21.35414\\
3916	9.686392\\
3917	9.987853\\
3918	15.521763\\
3919	19.40082\\
3920	19.79145\\
3921	19.191003\\
3922	18.803785\\
3923	18.61322\\
3924	17.619743\\
3925	17.34481\\
3926	16.435327\\
3927	15.652748\\
3928	16.599205\\
3929	17.002507\\
3930	17.621636\\
3931	17.755736\\
3932	17.255387\\
3933	16.598714\\
3934	18.002565\\
3935	16.459263\\
3936	14.964807\\
3937	13.944478\\
3938	14.33126\\
3939	13.944476\\
3940	13.522306\\
3941	13.522306\\
3942	13.522307\\
3943	13.812533\\
3944	14.66578\\
3945	14.768627\\
3946	14.508328\\
3947	14.360672\\
3948	13.624155\\
3949	13.522307\\
3950	13.094849\\
3951	12.677963\\
3952	13.070543\\
3953	13.522307\\
3954	13.522306\\
3955	14.492731\\
3956	14.09453\\
3957	13.944481\\
3958	14.768627\\
3959	14.598331\\
3960	13.661267\\
3961	9.686391\\
3962	8.984165\\
3963	8.842019\\
3964	8.841564\\
3965	8.491816\\
3966	8.491803\\
3967	8.842051\\
3968	9.304478\\
3969	9.347928\\
3970	9.42039\\
3971	9.375116\\
3972	8.737267\\
3973	8.452943\\
3974	7.726996\\
3975	7.726997\\
3976	7.726996\\
3977	8.491795\\
3978	9.480009\\
3979	9.686397\\
3980	10.045868\\
3981	9.829182\\
3982	10.445271\\
3983	9.871057\\
3984	13.522306\\
3985	5.865475\\
3986	5.450251\\
3987	5.531166\\
3988	5.865485\\
3989	6.506394\\
3990	7.344935\\
3991	9.354144\\
3992	10.516988\\
3993	9.794109\\
3994	9.521644\\
3995	9.775636\\
3996	9.629735\\
3997	8.992188\\
3998	8.502573\\
3999	8.502572\\
4000	8.502573\\
4001	8.652208\\
};
\addplot [color=mycolor1,solid,line width=1.0pt,forget plot]
  table[row sep=crcr]{%
4001	8.652208\\
4002	9.319868\\
4003	8.655777\\
4004	8.502579\\
4005	8.306399\\
4006	8.502578\\
4007	8.30063\\
4008	7.060074\\
4009	7.044386\\
4010	6.794064\\
4011	6.794065\\
4012	6.973845\\
4013	7.060079\\
4014	8.300628\\
4015	9.675213\\
4016	11.959344\\
4017	13.041261\\
4018	9.714491\\
4019	11.156391\\
4020	7.62205\\
4021	6.68022\\
4022	6.68022\\
4023	6.856645\\
4024	6.955348\\
4025	7.291945\\
4026	6.68022\\
4027	7.127208\\
4028	7.478162\\
4029	7.467503\\
4030	7.883266\\
4031	7.687083\\
4032	6.440765\\
4033	5.596427\\
4034	5.440382\\
4035	5.386403\\
4036	5.596421\\
4037	6.058943\\
4038	6.862876\\
4039	8.416522\\
4040	8.623723\\
4041	8.906068\\
4042	8.004982\\
4043	7.979676\\
4044	8.031715\\
4045	6.722465\\
4046	6.085206\\
4047	5.974356\\
4048	6.314148\\
4049	6.798511\\
4050	6.994686\\
4051	6.994678\\
4052	6.798507\\
4053	6.48448\\
4054	6.994686\\
4055	6.798503\\
4056	5.552185\\
4057	5.286176\\
4058	4.707841\\
4059	4.707162\\
4060	4.780435\\
4061	5.332568\\
4062	6.257618\\
4063	7.565078\\
4064	9.138534\\
4065	8.748774\\
4066	8.465545\\
4067	7.302004\\
4068	6.798503\\
4069	6.650084\\
4070	6.798511\\
4071	6.798393\\
4072	6.994687\\
4073	8.095042\\
4074	9.119055\\
4075	7.738487\\
4076	6.798511\\
4077	6.43727\\
4078	6.875573\\
4079	6.192571\\
4080	5.552185\\
4081	5.286171\\
4082	4.614301\\
4083	4.513609\\
4084	4.513604\\
4085	4.707841\\
4086	5.552185\\
4087	6.792735\\
4088	6.994688\\
4089	7.090825\\
4090	7.415985\\
4091	7.090834\\
4092	6.798503\\
4093	6.376802\\
4094	6.798511\\
4095	6.798505\\
4096	6.994567\\
4097	7.090601\\
4098	7.090686\\
4099	7.337405\\
4100	7.124086\\
4101	6.994685\\
4102	7.96932\\
4103	7.091641\\
4104	5.974357\\
4105	5.552188\\
4106	5.284661\\
4107	4.70785\\
4108	4.513584\\
4109	4.357581\\
4110	4.357582\\
4111	5.124672\\
4112	7.335438\\
4113	7.241154\\
4114	7.681236\\
4115	6.584526\\
4116	6.258048\\
4117	6.258048\\
4118	5.992037\\
4119	5.992034\\
4120	6.128705\\
4121	6.258048\\
4122	6.258048\\
4123	6.68022\\
4124	6.68022\\
4125	6.68022\\
4126	7.122812\\
4127	7.074831\\
4128	6.258048\\
4129	5.413705\\
4130	5.063446\\
4131	4.648219\\
4132	4.79201\\
4133	4.298666\\
4134	4.298653\\
4135	4.931021\\
4136	5.092967\\
4137	5.413705\\
4138	5.413705\\
4139	5.413702\\
4140	5.063445\\
4141	4.298663\\
4142	4.298652\\
4143	4.298652\\
4144	4.596854\\
4145	5.413705\\
4146	5.992034\\
4147	6.258048\\
4148	6.258048\\
4149	6.618674\\
4150	7.498599\\
4151	6.68023\\
4152	6.257827\\
4153	5.413705\\
4154	5.063445\\
4155	4.978212\\
4156	5.063445\\
4157	5.413789\\
4158	6.680316\\
4159	8.967248\\
4160	9.724116\\
4161	10.03097\\
4162	8.81453\\
4163	8.542032\\
4164	7.863859\\
4165	7.749671\\
4166	7.611963\\
4167	7.700548\\
4168	8.164338\\
4169	9.097425\\
4170	10.388897\\
4171	9.935379\\
4172	8.928712\\
4173	7.70054\\
4174	9.151076\\
4175	9.207234\\
4176	7.458742\\
4177	6.258048\\
4178	6.253827\\
4179	5.995811\\
4180	6.209042\\
4181	6.258048\\
4182	7.124624\\
4183	8.419739\\
4184	9.017897\\
4185	9.357394\\
4186	10.397369\\
4187	11.722241\\
4188	10.536494\\
4189	11.486641\\
4190	10.539195\\
4191	9.647144\\
4192	9.406598\\
4193	10.119395\\
4194	7.94827\\
4195	8.478049\\
4196	9.628607\\
4197	8.468252\\
4198	10.494104\\
4199	8.93015\\
4200	7.68708\\
4201	6.57936\\
4202	5.721833\\
4203	5.596423\\
4204	5.958271\\
4205	6.174749\\
4206	6.80139\\
4207	8.171987\\
4208	8.021106\\
4209	7.979465\\
4210	7.883266\\
4211	7.687088\\
4212	7.664775\\
4213	7.882995\\
4214	7.687086\\
4215	7.687093\\
4216	7.834813\\
4217	8.146201\\
4218	8.355078\\
4219	8.526728\\
4220	8.128924\\
4221	7.883266\\
4222	9.075263\\
4223	8.003427\\
4224	6.862937\\
4225	6.440765\\
4226	6.174751\\
4227	6.14306\\
4228	6.440765\\
4229	6.440772\\
4230	7.687078\\
4231	7.948073\\
4232	8.440216\\
4233	9.453878\\
4234	9.851829\\
4235	8.869018\\
4236	8.357305\\
4237	7.966721\\
4238	7.883273\\
4239	7.883274\\
4240	8.534796\\
4241	9.460677\\
4242	9.613316\\
4243	7.883271\\
4244	7.883266\\
4245	7.8832\\
4246	9.161323\\
4247	8.666539\\
4248	7.373765\\
4249	6.440767\\
4250	6.291031\\
4251	6.013422\\
4252	6.145466\\
4253	6.380267\\
4254	6.977376\\
4255	7.883271\\
4256	8.306754\\
4257	9.981902\\
4258	12.341553\\
4259	11.036113\\
4260	11.290821\\
4261	11.999134\\
4262	9.146733\\
4263	6.792706\\
4264	6.798505\\
4265	6.823347\\
4266	6.798501\\
4267	6.605141\\
4268	6.264066\\
4269	6.207332\\
4270	6.732354\\
4271	5.974361\\
4272	5.124727\\
4273	4.357381\\
4274	3.592789\\
4275	3.592789\\
4276	3.592789\\
4277	3.592789\\
4278	3.592789\\
4279	3.655227\\
4280	4.638083\\
4281	5.286171\\
4282	5.552185\\
4283	5.552186\\
4284	5.552186\\
4285	5.549989\\
4286	5.286171\\
4287	5.49612\\
4288	5.552185\\
4289	5.552185\\
4290	5.974358\\
4291	5.552185\\
4292	5.413874\\
4293	5.552185\\
4294	5.552185\\
4295	5.552185\\
4296	4.681332\\
4297	3.795576\\
4298	3.592789\\
4299	3.592789\\
4300	3.592789\\
4301	4.481369\\
4302	4.481369\\
4303	4.481369\\
4304	3.592789\\
4305	3.592789\\
4306	3.592791\\
4307	4.322201\\
4308	5.860464\\
4309	5.402166\\
4310	5.107458\\
4311	4.482616\\
4312	4.481369\\
4313	5.221779\\
4314	5.596422\\
4315	6.174751\\
4316	5.603651\\
4317	6.440765\\
4318	6.066398\\
4319	5.625582\\
4320	5.401726\\
4321	4.830939\\
4322	4.481369\\
4323	4.481369\\
4324	4.481369\\
4325	4.857175\\
4326	6.247568\\
4327	6.919206\\
4328	7.526756\\
4329	7.687086\\
4330	7.687085\\
4331	7.883266\\
4332	7.687086\\
4333	7.49337\\
4334	6.862938\\
4335	6.862937\\
4336	7.313183\\
4337	7.687085\\
4338	8.317003\\
4339	8.11071\\
4340	8.475823\\
4341	7.700565\\
4342	8.516633\\
4343	7.700568\\
4344	7.105561\\
4345	6.570913\\
4346	6.090327\\
4347	5.925472\\
4348	6.057161\\
4349	6.32519\\
4350	7.291318\\
4351	10.050875\\
4352	10.596487\\
4353	9.893623\\
4354	8.549774\\
4355	7.638434\\
4356	7.291327\\
4357	7.291341\\
4358	7.105561\\
4359	7.105562\\
4360	7.245211\\
4361	7.615526\\
4362	8.671907\\
4363	6.27473\\
4364	6.452114\\
4365	5.269887\\
4366	5.728813\\
4367	5.230635\\
4368	4.700753\\
4369	4.579014\\
4370	3.726046\\
4371	3.637747\\
4372	3.637686\\
4373	3.901855\\
4374	4.700753\\
4375	5.875372\\
4376	5.930591\\
4377	6.000531\\
4378	5.880844\\
4379	5.859285\\
4380	5.04221\\
4381	4.930696\\
4382	4.700753\\
4383	4.832327\\
4384	5.100486\\
4385	5.76915\\
4386	6.036384\\
4387	6.066517\\
4388	6.274729\\
4389	6.066589\\
4390	6.27473\\
4391	5.875377\\
4392	4.700753\\
4393	4.600869\\
4394	3.896983\\
4395	3.63866\\
4396	3.648133\\
4397	3.788312\\
4398	4.700752\\
4399	5.346143\\
4400	6.06172\\
4401	5.880895\\
4402	5.746547\\
4403	5.346146\\
4404	4.769535\\
4405	4.700753\\
4406	4.700753\\
4407	4.700753\\
4408	4.700753\\
4409	5.247743\\
4410	5.494435\\
4411	5.346147\\
4412	5.345444\\
4413	5.277288\\
4414	5.346141\\
4415	5.100487\\
4416	4.659807\\
4417	3.755845\\
4418	3.637334\\
4419	3.466979\\
4420	3.637734\\
4421	3.806731\\
4422	4.700753\\
4423	5.875387\\
4424	6.066589\\
4425	6.066588\\
4426	6.066588\\
4427	6.274368\\
4428	6.066588\\
4429	6.066588\\
4430	6.022107\\
4431	5.880862\\
4432	5.727294\\
4433	5.880837\\
4434	6.066586\\
4435	5.880837\\
4436	5.812405\\
4437	5.235347\\
4438	5.346172\\
4439	4.700797\\
4440	4.137039\\
4441	3.607933\\
4442	3.400026\\
4443	3.39722\\
4444	3.392536\\
4445	3.401723\\
4446	3.28189\\
4447	3.400004\\
4448	3.392537\\
4449	3.401828\\
4450	3.401828\\
4451	3.401829\\
4452	3.401626\\
4453	3.401832\\
4454	3.399937\\
4455	3.40183\\
4456	3.399958\\
4457	3.39971\\
4458	3.394924\\
4459	3.40178\\
4460	3.40183\\
4461	3.401721\\
4462	3.706104\\
4463	3.401785\\
4464	3.40183\\
4465	3.401773\\
4466	3.303951\\
4467	2.404658\\
4468	1.987286\\
4469	1.987287\\
4470	1.987287\\
4471	2.262661\\
4472	3.050671\\
4473	3.401746\\
4474	3.397062\\
4475	3.39316\\
4476	3.398766\\
4477	3.399163\\
4478	3.401767\\
4479	3.401767\\
4480	3.399401\\
4481	3.40183\\
4482	3.753891\\
4483	3.901499\\
4484	3.718819\\
4485	3.749437\\
4486	3.665229\\
4487	3.432995\\
4488	3.401764\\
4489	3.39745\\
4490	2.904645\\
4491	2.856435\\
4492	3.238555\\
4493	3.397873\\
4494	3.637735\\
4495	4.832331\\
4496	5.193992\\
4497	5.346137\\
4498	5.174127\\
4499	5.34481\\
4500	4.832364\\
4501	5.100495\\
4502	5.346141\\
4503	5.367586\\
4504	5.880843\\
4505	5.864024\\
4506	6.043673\\
4507	5.880842\\
4508	6.143016\\
4509	6.066587\\
4510	6.274727\\
4511	5.880854\\
4512	4.700753\\
4513	4.519965\\
4514	3.789925\\
4515	3.638283\\
4516	3.638522\\
4517	3.901415\\
4518	4.700753\\
4519	5.880843\\
4520	7.79162\\
4521	9.436613\\
4522	9.777886\\
4523	10.414756\\
4524	9.633488\\
4525	9.555402\\
4526	9.464808\\
4527	8.804728\\
4528	8.239028\\
4529	8.807256\\
4530	8.254876\\
4531	7.217275\\
4532	6.420372\\
4533	6.106693\\
4534	6.380362\\
4535	5.888611\\
4536	4.700754\\
4537	3.901452\\
4538	3.646329\\
4539	3.401761\\
4540	3.401669\\
4541	3.401757\\
4542	3.901462\\
4543	4.930245\\
4544	5.880854\\
4545	6.009954\\
4546	6.274728\\
4547	7.610028\\
4548	7.116499\\
4549	6.875075\\
4550	6.66185\\
4551	6.274877\\
4552	6.274727\\
4553	6.258514\\
4554	6.066587\\
4555	5.346172\\
4556	4.832325\\
4557	4.701187\\
4558	5.04221\\
4559	4.700754\\
4560	3.637737\\
4561	3.401652\\
4562	3.399671\\
4563	3.397172\\
4564	3.399869\\
4565	3.401662\\
4566	3.901406\\
4567	4.700788\\
4568	5.880837\\
4569	6.066587\\
4570	5.880837\\
4571	5.880837\\
4572	6.274727\\
4573	6.066587\\
4574	5.880841\\
4575	5.880864\\
4576	5.594875\\
4577	5.880837\\
4578	5.880837\\
4579	5.880838\\
4580	5.100487\\
4581	5.3948\\
4582	5.880843\\
4583	5.102247\\
4584	4.486848\\
4585	3.64271\\
4586	3.492976\\
4587	3.401763\\
4588	3.401754\\
4589	3.672221\\
4590	4.700754\\
4591	5.346141\\
4592	5.346141\\
4593	5.208677\\
4594	5.346141\\
4595	6.274728\\
4596	6.274732\\
4597	5.880869\\
4598	5.806319\\
4599	5.346141\\
4600	5.346141\\
4601	5.875375\\
4602	5.875377\\
4603	5.346141\\
4604	5.571622\\
4605	5.346141\\
4606	5.875377\\
4607	5.881006\\
4608	5.100744\\
4609	4.70076\\
4610	3.901304\\
4611	3.865364\\
4612	3.638099\\
4613	3.592351\\
4614	3.540995\\
4615	3.901307\\
4616	4.700758\\
4617	5.100292\\
4618	5.346313\\
4619	5.346339\\
4620	5.13095\\
4621	4.701126\\
4622	4.700761\\
4623	4.296001\\
4624	4.624126\\
4625	4.701173\\
4626	4.81544\\
4627	5.209062\\
4628	5.042604\\
4629	5.042504\\
4630	5.346386\\
4631	4.701179\\
4632	4.578779\\
4633	3.645209\\
4634	3.401716\\
4635	3.401631\\
4636	3.400826\\
4637	3.392012\\
4638	3.399587\\
4639	3.398545\\
4640	3.401693\\
4641	3.638239\\
4642	3.901305\\
4643	4.368588\\
4644	4.320678\\
4645	3.637535\\
4646	3.401705\\
4647	3.401681\\
4648	3.401695\\
4649	3.437381\\
4650	3.901303\\
4651	4.701178\\
4652	4.701174\\
4653	4.701173\\
4654	4.701174\\
4655	4.701173\\
4656	3.926403\\
4657	3.399217\\
4658	3.401746\\
4659	3.401732\\
4660	3.401785\\
4661	3.401828\\
4662	3.756801\\
4663	4.700753\\
4664	5.84567\\
4665	5.346144\\
4666	7.136295\\
4667	7.291324\\
4668	7.301962\\
4669	7.105551\\
4670	6.809747\\
4671	6.433833\\
4672	6.686325\\
4673	7.105553\\
4674	7.291303\\
4675	7.105548\\
4676	7.100041\\
4677	6.57089\\
4678	7.105554\\
4679	6.704389\\
4680	5.925467\\
4681	5.126031\\
4682	4.243186\\
4683	4.243183\\
4684	4.243184\\
4685	4.269168\\
4686	5.63216\\
4687	5.925468\\
4688	6.394396\\
4689	6.570887\\
4690	7.287332\\
4691	7.499438\\
4692	7.597688\\
4693	7.499444\\
4694	7.375621\\
4695	7.499435\\
4696	7.428146\\
4697	7.948939\\
4698	8.16154\\
4699	7.291314\\
4700	7.288432\\
4701	7.291333\\
4702	7.596423\\
4703	9.631975\\
4704	7.881774\\
4705	7.161352\\
4706	6.647609\\
4707	6.51595\\
4708	6.515994\\
4709	7.157761\\
4710	7.806103\\
4711	10.005811\\
4712	11.557026\\
4713	11.718085\\
4714	11.516718\\
4715	10.61072\\
4716	8.709721\\
4717	8.089921\\
4718	7.881812\\
4719	8.08992\\
4720	8.089942\\
4721	8.520437\\
4722	9.990104\\
4723	9.442928\\
4724	7.291281\\
4725	7.291281\\
4726	8.367443\\
4727	7.499421\\
4728	6.570834\\
4729	5.925444\\
4730	5.925445\\
4731	5.731859\\
4732	5.756867\\
4733	5.925444\\
4734	6.433546\\
4735	7.291278\\
4736	8.355035\\
4737	7.537178\\
4738	7.499421\\
4739	7.291282\\
4740	7.291279\\
4741	7.105522\\
4742	7.100056\\
4743	7.105525\\
4744	7.124106\\
4745	7.499435\\
4746	7.934627\\
4747	7.499435\\
4748	7.105528\\
4749	7.105534\\
4750	7.499422\\
4751	6.9112\\
4752	5.925445\\
4753	5.125976\\
4754	4.401176\\
4755	4.22263\\
4756	4.264758\\
4757	5.07245\\
4758	5.925444\\
4759	6.325179\\
4760	7.105525\\
4761	7.105066\\
4762	7.100046\\
4763	7.105533\\
4764	6.325159\\
4765	5.925444\\
4766	5.925444\\
4767	5.925444\\
4768	5.968336\\
4769	6.570834\\
4770	6.995076\\
4771	7.105525\\
4772	7.105526\\
4773	7.100062\\
4774	7.291279\\
4775	6.570834\\
4776	5.925444\\
4777	3.401755\\
4778	3.399482\\
4779	3.401677\\
4780	3.294977\\
4781	2.590708\\
4782	2.306808\\
4783	2.90464\\
4784	3.401664\\
4785	3.401677\\
4786	3.401776\\
4787	2.42578\\
4788	2.271485\\
4789	2.293215\\
4790	2.241321\\
4791	2.362624\\
4792	3.401736\\
4793	3.40183\\
4794	3.669285\\
4795	4.219549\\
4796	4.589194\\
4797	3.901458\\
4798	3.901822\\
4799	3.901415\\
4800	3.401798\\
4801	3.401829\\
4802	3.395963\\
4803	3.392137\\
4804	3.401737\\
4805	3.401729\\
4806	3.218412\\
4807	3.401751\\
4808	3.391926\\
4809	3.401829\\
4810	3.401687\\
4811	3.401832\\
4812	3.395213\\
4813	3.396767\\
4814	3.401756\\
4815	3.132563\\
4816	3.272985\\
4817	3.401649\\
4818	3.401829\\
4819	3.401758\\
4820	3.473933\\
4821	3.629283\\
4822	3.648198\\
4823	3.401769\\
4824	3.40183\\
4825	3.401751\\
4826	3.059085\\
4827	2.529041\\
4828	2.306818\\
4829	2.258311\\
4830	2.238613\\
4831	2.512398\\
4832	2.632808\\
4833	3.401697\\
4834	3.392781\\
4835	3.401745\\
4836	3.401732\\
4837	3.394367\\
4838	3.401739\\
4839	3.358409\\
4840	3.246768\\
4841	3.401729\\
4842	3.396896\\
4843	3.398263\\
4844	3.399383\\
4845	3.399689\\
4846	3.401829\\
4847	3.397881\\
4848	3.159764\\
4849	2.293182\\
4850	1.987288\\
4851	1.987286\\
4852	1.987286\\
4853	2.243115\\
4854	3.401728\\
4855	3.401759\\
4856	3.901504\\
4857	4.700761\\
4858	4.700761\\
4859	4.578305\\
4860	3.845699\\
4861	3.63769\\
4862	3.401762\\
4863	3.401673\\
4864	3.401831\\
4865	3.710841\\
4866	3.90142\\
4867	3.954892\\
4868	4.387637\\
4869	3.901491\\
4870	4.503123\\
4871	3.915439\\
4872	3.401673\\
4873	3.394675\\
4874	3.401671\\
4875	3.401666\\
4876	3.401664\\
4877	3.401786\\
4878	3.40183\\
4879	3.821993\\
4880	4.648449\\
4881	4.545211\\
4882	4.040347\\
4883	3.901421\\
4884	3.401762\\
4885	3.401758\\
4886	3.401716\\
4887	3.401762\\
4888	3.428102\\
4889	3.647177\\
4890	3.847251\\
4891	3.90142\\
4892	3.901427\\
4893	3.901421\\
4894	4.700758\\
4895	3.901452\\
4896	3.401718\\
4897	3.394323\\
4898	3.401736\\
4899	3.40171\\
4900	3.40176\\
4901	3.396922\\
4902	3.401744\\
4903	4.058479\\
4904	4.365225\\
4905	3.901512\\
4906	3.650758\\
4907	3.637745\\
4908	3.637733\\
4909	3.401762\\
4910	3.401755\\
4911	3.401757\\
4912	3.402705\\
4913	3.720484\\
4914	3.901468\\
4915	3.901483\\
4916	4.083889\\
4917	4.300035\\
4918	4.700758\\
4919	4.422085\\
4920	3.576677\\
4921	3.401829\\
4922	3.397461\\
4923	3.396866\\
4924	3.398554\\
4925	3.40183\\
4926	3.401765\\
4927	3.82106\\
4928	4.700753\\
4929	4.700758\\
4930	4.674117\\
4931	4.387654\\
4932	3.994403\\
4933	3.901503\\
4934	3.901504\\
4935	3.901423\\
4936	3.89938\\
4937	3.901421\\
4938	3.94778\\
4939	4.289596\\
4940	3.974527\\
4941	4.218404\\
4942	4.700758\\
4943	3.901479\\
4944	3.648142\\
4945	3.40165\\
4946	3.398153\\
4947	3.394237\\
4948	3.401772\\
4949	3.401764\\
4950	3.40177\\
4951	3.399815\\
4952	3.401712\\
4953	3.638335\\
4954	3.650625\\
4955	3.65566\\
4956	3.53429\\
4957	3.40183\\
4958	3.398044\\
4959	3.397391\\
4960	3.399755\\
4961	3.401761\\
4962	3.637852\\
4963	3.887965\\
4964	3.901447\\
4965	3.901448\\
4966	4.249381\\
4967	3.863936\\
4968	3.401626\\
4969	3.39893\\
4970	3.401761\\
4971	3.401645\\
4972	3.093648\\
4973	3.275514\\
4974	2.90464\\
4975	3.401661\\
4976	3.40176\\
4977	3.391642\\
4978	3.401677\\
4979	3.062113\\
4980	2.483642\\
4981	2.241402\\
4982	2.127882\\
4983	2.241329\\
4984	2.530767\\
4985	3.401753\\
4986	3.401829\\
4987	3.637742\\
4988	3.719123\\
4989	3.901425\\
4990	3.901427\\
4991	3.647372\\
4992	3.401829\\
4993	3.399582\\
4994	3.401775\\
4995	3.401755\\
4996	3.40177\\
4997	3.399722\\
4998	3.401672\\
4999	4.229829\\
5000	4.700755\\
5001	4.832223\\
5002	4.873525\\
5003	4.832329\\
5004	4.832323\\
5005	4.700755\\
5006	4.700766\\
5007	4.700764\\
5008	4.700768\\
5009	4.700755\\
5010	4.700755\\
5011	4.553886\\
5012	3.901536\\
5013	3.901438\\
5014	4.700765\\
5015	3.758422\\
5016	3.401826\\
5017	3.396874\\
5018	3.401651\\
5019	3.24122\\
5020	3.351518\\
5021	3.391712\\
5022	3.400249\\
5023	3.640593\\
5024	4.205082\\
5025	3.901439\\
5026	3.901874\\
5027	3.901878\\
5028	3.901526\\
5029	3.80462\\
5030	3.637847\\
5031	3.55722\\
5032	3.639089\\
5033	3.901479\\
5034	4.199411\\
5035	3.901957\\
5036	4.271281\\
5037	4.217469\\
5038	4.700754\\
5039	4.417744\\
5040	3.637131\\
5041	3.401826\\
5042	3.398939\\
5043	3.396669\\
5044	3.396505\\
5045	3.399956\\
5046	3.401703\\
5047	4.30391\\
5048	4.700766\\
5049	4.700764\\
5050	4.408553\\
5051	4.013749\\
5052	3.758506\\
5053	3.642237\\
5054	3.734336\\
5055	3.782101\\
5056	3.901537\\
5057	4.700767\\
5058	4.70076\\
5059	4.700762\\
5060	4.700768\\
5061	4.700764\\
5062	4.849781\\
5063	4.320864\\
5064	3.638157\\
5065	3.401828\\
5066	3.396764\\
5067	3.40178\\
5068	3.394091\\
5069	3.3997\\
5070	3.401625\\
5071	4.063099\\
5072	4.700764\\
5073	4.70075\\
5074	4.700769\\
5075	4.295795\\
5076	4.298813\\
5077	3.901526\\
5078	3.901448\\
5079	3.901443\\
5080	3.901536\\
5081	3.901442\\
5082	4.294942\\
5083	4.235922\\
5084	4.700766\\
5085	4.700768\\
5086	4.963807\\
5087	4.578294\\
5088	3.593077\\
5089	3.590901\\
5090	3.38284\\
5091	3.244578\\
5092	3.221713\\
5093	3.591468\\
5094	3.593157\\
5095	3.685162\\
5096	4.04592\\
5097	3.686039\\
5098	3.639572\\
5099	3.600612\\
5100	3.60689\\
5101	3.600193\\
5102	3.596687\\
5103	3.59965\\
5104	3.6019\\
5105	3.639598\\
5106	3.77979\\
5107	4.144073\\
5108	4.144153\\
5109	4.144098\\
5110	4.563398\\
5111	3.948885\\
5112	3.60094\\
5113	3.593585\\
5114	3.575799\\
5115	2.823054\\
5116	2.450157\\
5117	2.43592\\
5118	2.380813\\
5119	2.951397\\
5120	3.59099\\
5121	3.592538\\
5122	3.593415\\
5123	3.593083\\
5124	3.592457\\
5125	3.591901\\
5126	3.582579\\
5127	3.591816\\
5128	3.594706\\
5129	3.627458\\
5130	3.642633\\
5131	3.644563\\
5132	3.692025\\
5133	3.773056\\
5134	3.928628\\
5135	3.639224\\
5136	3.592114\\
5137	3.598818\\
5138	2.497923\\
5139	2.435893\\
5140	2.425664\\
5141	2.435887\\
5142	2.429772\\
5143	2.6042\\
5144	3.240583\\
5145	2.435927\\
5146	2.435892\\
5147	2.382247\\
5148	2.110821\\
5149	1.905611\\
5150	1.004416\\
5151	0.921281\\
5152	1.819505\\
5153	2.290383\\
5154	2.46727\\
5155	3.591292\\
5156	3.591563\\
5157	3.590971\\
5158	3.592635\\
5159	3.578402\\
5160	3.577693\\
5161	2.948517\\
5162	2.381084\\
5163	2.380813\\
5164	2.435901\\
5165	3.280446\\
5166	4.323521\\
5167	5.44498\\
5168	6.294201\\
5169	6.294201\\
5170	6.294201\\
5171	4.863131\\
5172	4.527327\\
5173	4.144147\\
5174	3.685404\\
5175	3.688225\\
5176	3.685576\\
5177	4.144102\\
5178	4.144145\\
5179	4.144138\\
5180	4.527329\\
5181	4.527327\\
5182	4.993332\\
5183	4.144082\\
5184	3.604105\\
5185	3.593618\\
5186	3.590796\\
5187	3.578687\\
5188	3.576858\\
5189	3.59277\\
5190	3.607935\\
5191	4.144159\\
5192	4.527328\\
5193	4.240682\\
5194	4.144134\\
5195	3.686707\\
5196	4.993519\\
5197	4.993511\\
5198	4.99349\\
5199	4.993492\\
5200	4.993532\\
5201	5.844163\\
5202	6.246813\\
5203	6.444127\\
5204	6.444128\\
5205	6.444123\\
5206	7.417046\\
5207	6.195414\\
5208	4.993545\\
5209	4.145357\\
5210	3.651265\\
5211	3.637956\\
5212	3.570991\\
5213	3.666075\\
5214	4.145004\\
5215	4.99352\\
5216	5.41794\\
5217	5.417958\\
5218	5.417941\\
5219	5.417943\\
5220	4.993789\\
5221	4.99353\\
5222	4.993522\\
5223	5.006718\\
5224	5.417951\\
5225	6.165175\\
5226	6.241124\\
5227	5.417981\\
5228	5.292297\\
5229	5.356013\\
5230	5.706755\\
5231	4.993521\\
5232	4.527279\\
5233	3.82792\\
5234	3.653539\\
5235	3.638289\\
5236	3.666026\\
5237	4.145729\\
5238	4.993511\\
5239	5.916918\\
5240	6.259416\\
5241	8.067782\\
5242	7.468279\\
5243	6.777769\\
5244	6.775688\\
5245	6.444126\\
5246	6.428319\\
5247	6.246815\\
5248	6.246822\\
5249	6.246815\\
5250	6.353568\\
5251	6.246801\\
5252	6.444126\\
5253	6.524056\\
5254	6.744883\\
5255	6.14709\\
5256	4.99353\\
5257	4.527559\\
5258	4.145453\\
5259	3.930927\\
5260	3.948707\\
5261	4.527587\\
5262	4.993522\\
5263	6.246813\\
5264	7.150386\\
5265	7.763583\\
5266	7.000482\\
5267	6.985798\\
5268	6.446228\\
5269	6.55755\\
5270	6.444127\\
5271	6.822204\\
5272	6.986621\\
5273	6.444123\\
5274	6.444128\\
5275	6.444129\\
5276	6.21125\\
5277	5.417944\\
5278	5.417966\\
5279	4.993528\\
5280	4.042419\\
5281	3.602912\\
5282	3.57254\\
5283	3.574374\\
5284	3.604002\\
5285	3.600239\\
5286	3.602033\\
5287	3.574333\\
5288	3.572009\\
5289	3.638238\\
5290	3.638345\\
5291	3.638358\\
5292	3.572155\\
5293	3.574164\\
5294	3.574209\\
5295	3.574452\\
5296	3.573158\\
5297	3.570353\\
5298	3.722475\\
5299	4.528059\\
5300	4.144079\\
5301	4.527976\\
5302	4.883575\\
5303	4.860677\\
5304	3.948719\\
5305	3.63805\\
5306	3.572075\\
5307	3.57421\\
5308	3.57485\\
5309	3.5746\\
5310	3.603544\\
5311	3.60355\\
5312	3.574563\\
5313	3.572125\\
5314	3.571338\\
5315	3.638298\\
5316	3.635636\\
5317	3.574917\\
5318	3.601806\\
5319	3.343214\\
5320	3.598415\\
5321	3.604194\\
5322	3.573513\\
5323	3.57292\\
5324	3.572952\\
5325	3.572114\\
5326	3.602982\\
5327	3.572447\\
5328	3.240468\\
5329	2.435968\\
5330	2.11085\\
5331	2.11085\\
5332	2.399928\\
5333	3.579859\\
5334	3.571188\\
5335	3.684984\\
5336	4.565012\\
5337	4.1947\\
5338	4.057221\\
5339	3.954761\\
5340	3.818784\\
5341	3.67348\\
5342	3.665106\\
5343	3.665229\\
5344	3.666126\\
5345	3.766999\\
5346	4.14599\\
5347	4.52728\\
5348	4.52728\\
5349	4.527281\\
5350	4.993503\\
5351	4.14408\\
5352	3.569486\\
5353	3.571274\\
5354	3.57224\\
5355	3.573387\\
5356	3.57207\\
5357	3.571167\\
5358	3.72165\\
5359	4.857306\\
5360	4.993518\\
5361	4.993516\\
5362	4.881303\\
5363	4.993519\\
5364	4.52728\\
5365	4.52728\\
5366	4.145912\\
5367	4.527281\\
5368	4.86214\\
5369	4.993528\\
5370	5.347188\\
5371	4.993521\\
5372	4.99352\\
5373	5.133427\\
5374	5.533001\\
5375	4.993517\\
5376	4.144082\\
5377	3.638295\\
5378	3.570816\\
5379	3.571226\\
5380	3.570464\\
5381	3.665177\\
5382	4.667093\\
5383	5.034826\\
5384	6.246815\\
5385	6.246825\\
5386	6.442658\\
5387	6.444126\\
5388	6.246817\\
5389	5.974854\\
5390	5.533005\\
5391	5.417944\\
5392	5.417945\\
5393	5.439923\\
5394	5.345813\\
5395	6.088924\\
5396	5.829894\\
5397	6.081157\\
5398	6.246815\\
5399	4.993526\\
5400	4.52728\\
5401	3.948725\\
5402	3.665166\\
5403	3.666024\\
5404	3.665119\\
5405	4.344444\\
5406	4.993529\\
5407	6.339293\\
5408	7.732743\\
5409	8.533131\\
5410	10.177541\\
5411	10.212214\\
5412	8.885355\\
5413	8.261539\\
5414	7.623612\\
5415	7.57396\\
5416	7.735286\\
5417	7.753115\\
5418	8.770807\\
5419	7.857467\\
5420	7.336989\\
5421	7.590087\\
5422	7.130274\\
5423	6.83778\\
5424	5.584262\\
5425	5.017386\\
5426	4.539722\\
5427	4.258708\\
5428	4.189654\\
5429	4.189655\\
5430	3.965245\\
5431	4.189656\\
5432	4.460072\\
5433	4.735038\\
5434	4.795693\\
5435	4.735038\\
5436	4.413323\\
5437	3.964352\\
5438	3.613548\\
5439	3.613549\\
5440	3.613549\\
5441	3.613548\\
5442	4.189656\\
5443	4.735028\\
5444	5.118245\\
5445	5.584261\\
5446	5.584264\\
5447	5.348629\\
5448	4.356779\\
5449	3.613566\\
5450	3.613549\\
5451	3.61355\\
5452	3.61355\\
5453	3.613548\\
5454	3.613547\\
5455	4.189655\\
5456	4.822071\\
5457	5.118245\\
5458	5.118247\\
5459	5.118245\\
5460	4.735039\\
5461	4.189657\\
5462	3.618171\\
5463	3.613549\\
5464	3.86824\\
5465	4.294328\\
5466	5.118245\\
5467	5.27291\\
5468	4.735039\\
5469	4.735039\\
5470	5.036176\\
5471	4.529379\\
5472	3.613549\\
5473	3.613544\\
5474	3.400379\\
5475	2.762244\\
5476	2.440824\\
5477	2.898832\\
5478	2.53455\\
5479	2.939391\\
5480	3.351957\\
5481	3.613544\\
5482	3.613546\\
5483	3.613547\\
5484	3.613547\\
5485	3.613543\\
5486	3.407128\\
5487	3.312072\\
5488	3.613544\\
5489	3.613549\\
5490	3.613548\\
5491	4.086161\\
5492	4.170219\\
5493	4.189657\\
5494	4.110294\\
5495	3.844945\\
5496	3.613548\\
5497	3.613543\\
5498	3.295651\\
5499	3.167316\\
5500	3.337066\\
5501	3.613548\\
5502	4.520895\\
5503	5.584262\\
5504	5.864046\\
5505	5.584263\\
5506	5.584264\\
5507	5.584265\\
5508	5.584263\\
5509	5.584264\\
5510	5.584264\\
5511	5.916499\\
5512	6.124321\\
5513	7.025958\\
5514	7.406406\\
5515	8.492537\\
5516	8.755354\\
5517	8.473613\\
5518	8.64623\\
5519	7.458792\\
5520	6.008871\\
5521	5.584262\\
5522	5.118245\\
5523	4.738199\\
5524	4.774463\\
5525	5.456209\\
5526	6.168575\\
5527	7.035094\\
5528	8.722359\\
5529	9.12996\\
5530	8.578452\\
5531	7.988319\\
5532	7.035098\\
5533	6.418481\\
5534	6.837778\\
5535	6.817879\\
5536	6.837779\\
5537	6.837787\\
5538	8.566082\\
5539	8.954982\\
5540	9.817785\\
5541	9.223844\\
5542	9.32126\\
5543	7.307531\\
5544	6.731793\\
5545	5.584265\\
5546	5.58426\\
5547	5.527985\\
5548	5.584224\\
5549	5.747844\\
5550	7.251014\\
5551	10.171642\\
5552	10.36548\\
5553	10.548074\\
5554	9.648765\\
5555	9.475844\\
5556	7.835765\\
5557	7.781119\\
5558	7.084561\\
5559	7.915846\\
5560	8.791617\\
5561	10.667323\\
5562	10.341042\\
5563	10.494921\\
5564	10.564046\\
5565	11.561904\\
5566	11.252857\\
5567	8.990352\\
5568	6.831978\\
5569	5.584622\\
5570	5.584263\\
5571	5.584261\\
5572	5.584262\\
5573	5.723908\\
5574	7.162873\\
5575	10.132022\\
5576	10.896257\\
5577	8.875077\\
5578	7.976394\\
5579	7.883421\\
5580	8.595624\\
5581	7.586322\\
5582	8.548078\\
5583	8.203806\\
5584	8.237902\\
5585	8.037114\\
5586	9.989198\\
5587	8.789434\\
5588	8.564159\\
5589	7.601299\\
5590	7.552776\\
5591	6.837794\\
5592	5.584262\\
5593	5.044906\\
5594	4.608176\\
5595	4.189657\\
5596	4.189658\\
5597	5.118245\\
5598	5.80556\\
5599	6.837773\\
5600	8.566633\\
5601	7.134753\\
5602	7.183112\\
5603	7.69823\\
5604	7.035096\\
5605	7.869144\\
5606	7.246639\\
5607	7.299318\\
5608	7.136565\\
5609	7.437294\\
5610	7.496618\\
5611	8.217155\\
5612	9.371694\\
5613	9.964329\\
5614	10.065897\\
5615	7.917803\\
5616	6.009451\\
5617	5.584263\\
5618	6.831979\\
5619	6.083462\\
5620	5.796522\\
5621	5.756758\\
5622	6.00887\\
5623	6.573303\\
5624	6.837784\\
5625	6.357363\\
5626	6.330995\\
5627	5.963581\\
5628	5.592583\\
5629	5.58426\\
5630	5.58426\\
5631	5.558469\\
5632	5.58426\\
5633	6.035842\\
5634	6.837782\\
5635	7.729391\\
5636	9.04519\\
5637	8.756428\\
5638	8.002898\\
5639	7.558017\\
5640	6.697439\\
5641	4.379838\\
5642	3.613548\\
5643	3.61355\\
5644	3.613551\\
5645	3.61355\\
5646	3.613548\\
5647	4.189655\\
5648	4.189659\\
5649	4.203941\\
5650	4.189657\\
5651	4.330704\\
5652	4.189658\\
5653	3.613549\\
5654	3.613551\\
5655	3.61355\\
5656	3.613549\\
5657	4.189658\\
5658	5.584155\\
5659	5.724113\\
5660	6.621659\\
5661	7.035096\\
5662	6.837783\\
5663	6.272516\\
5664	5.584258\\
5665	4.757167\\
5666	4.49523\\
5667	4.249031\\
5668	4.539792\\
5669	5.577276\\
5670	6.837778\\
5671	7.819376\\
5672	10.317778\\
5673	11.108457\\
5674	11.699993\\
5675	11.953902\\
5676	12.747116\\
5677	12.846652\\
5678	12.846649\\
5679	13.238993\\
5680	13.629227\\
5681	13.087643\\
5682	12.846659\\
5683	11.737808\\
5684	10.896571\\
5685	9.290769\\
5686	8.920773\\
5687	7.035095\\
5688	5.848009\\
5689	5.584265\\
5690	5.291501\\
5691	5.358635\\
5692	5.584262\\
5693	5.587328\\
5694	7.928145\\
5695	10.896254\\
5696	12.485702\\
5697	14.227259\\
5698	14.289264\\
5699	14.289264\\
5700	14.28924\\
5701	13.890562\\
5702	13.267831\\
5703	14.150299\\
5704	14.010079\\
5705	14.1436\\
5706	13.035173\\
5707	12.846601\\
5708	11.886005\\
5709	11.29374\\
5710	10.846315\\
5711	8.864783\\
5712	6.630793\\
5713	5.58426\\
5714	5.454398\\
5715	5.394317\\
5716	5.58426\\
5717	5.807746\\
5718	8.083173\\
5719	10.381671\\
5720	12.471441\\
5721	12.18636\\
5722	11.188012\\
5723	10.774834\\
5724	10.014722\\
5725	10.185061\\
5726	9.905654\\
5727	10.272274\\
5728	10.572454\\
5729	10.896245\\
5730	12.083709\\
5731	11.885904\\
5732	11.207232\\
5733	11.542372\\
5734	11.489856\\
5735	8.792627\\
5736	6.635886\\
5737	5.58426\\
5738	5.39564\\
5739	5.118232\\
5740	5.118246\\
5741	5.58426\\
5742	7.035094\\
5743	9.866614\\
5744	11.471207\\
5745	11.8697\\
5746	12.472566\\
5747	12.435507\\
5748	11.432821\\
5749	11.26142\\
5750	10.932989\\
5751	11.636263\\
5752	11.058331\\
5753	10.834186\\
5754	10.646462\\
5755	10.222683\\
5756	9.810584\\
5757	9.250866\\
5758	8.176922\\
5759	6.444129\\
5760	4.993493\\
5761	4.70776\\
5762	4.144081\\
5763	4.144083\\
5764	4.144079\\
5765	4.993534\\
5766	6.246815\\
5767	7.974494\\
5768	9.668441\\
5769	8.838711\\
5770	7.667029\\
5771	7.060232\\
5772	6.444127\\
5773	6.241027\\
5774	6.021534\\
5775	5.180399\\
5776	5.178069\\
5777	5.642819\\
5778	6.246815\\
5779	6.246815\\
5780	6.246826\\
5781	6.241028\\
5782	6.241021\\
5783	5.148496\\
5784	4.145786\\
5785	3.66598\\
5786	4.859623\\
5787	4.527308\\
5788	4.258773\\
5789	4.144118\\
5790	4.637008\\
5791	4.993294\\
5792	5.417969\\
5793	6.44413\\
5794	7.656123\\
5795	8.620964\\
5796	8.203671\\
5797	7.646326\\
5798	7.315228\\
5799	7.300196\\
5800	7.565046\\
5801	8.971778\\
5802	9.929349\\
5803	10.305282\\
5804	9.650569\\
5805	9.084733\\
5806	7.985772\\
5807	6.840913\\
5808	6.136878\\
5809	4.993292\\
5810	4.993302\\
5811	4.89232\\
5812	4.863587\\
5813	4.922471\\
5814	4.993301\\
5815	4.993294\\
5816	5.417999\\
5817	6.24683\\
5818	6.707455\\
5819	7.469461\\
5820	6.703419\\
5821	6.24683\\
5822	4.993291\\
5823	4.993301\\
5824	4.993294\\
5825	5.533034\\
5826	6.540163\\
5827	9.5665\\
5828	9.467429\\
5829	9.517184\\
5830	9.963264\\
5831	8.502701\\
5832	5.640702\\
5833	4.662008\\
5834	4.304711\\
5835	4.30471\\
5836	4.304701\\
5837	5.689137\\
5838	8.145631\\
5839	10.32681\\
5840	10.545585\\
5841	9.672679\\
5842	9.141871\\
5843	8.649833\\
5844	8.380825\\
5845	8.381246\\
5846	8.652308\\
5847	8.995612\\
5848	9.532679\\
5849	10.258634\\
5850	10.878093\\
5851	11.920475\\
5852	11.078155\\
5853	10.49039\\
5854	10.149479\\
5855	8.145632\\
5856	6.894069\\
5857	6.652352\\
5858	5.815145\\
5859	5.640702\\
5860	5.745563\\
5861	6.652353\\
5862	8.222732\\
5863	12.980354\\
5864	14.054218\\
5865	11.440064\\
5866	9.883878\\
5867	9.483622\\
5868	8.609595\\
5869	8.420714\\
5870	8.526411\\
5871	8.386668\\
5872	9.032374\\
5873	9.882937\\
5874	10.623586\\
5875	10.580582\\
5876	10.586247\\
5877	10.689934\\
5878	10.08383\\
5879	8.23265\\
5880	6.792904\\
5881	6.652352\\
5882	6.092636\\
5883	5.640702\\
5884	6.140543\\
5885	6.652353\\
5886	8.380685\\
5887	12.980359\\
5888	14.054226\\
5889	11.367225\\
5890	10.204707\\
5891	8.932704\\
5892	8.145632\\
5893	7.947282\\
5894	7.372976\\
5895	7.518946\\
5896	8.145634\\
5897	9.081795\\
5898	9.524776\\
5899	9.40324\\
5900	10.995394\\
5901	11.029005\\
5902	9.371891\\
5903	7.158178\\
5904	6.652353\\
5905	5.786864\\
5906	5.683882\\
5907	5.640702\\
5908	5.641349\\
5909	6.652354\\
5910	8.380685\\
5911	14.054228\\
5912	14.240307\\
5913	12.980354\\
5914	11.057379\\
5915	10.483608\\
5916	8.690233\\
5917	8.380686\\
5918	8.380685\\
5919	8.426812\\
5920	9.981573\\
5921	10.941139\\
5922	17.022347\\
5923	17.022346\\
5924	17.022348\\
5925	16.38229\\
5926	11.681643\\
5927	8.49184\\
5928	7.329782\\
5929	6.652354\\
5930	6.652353\\
5931	6.652349\\
5932	6.652353\\
5933	6.652354\\
5934	8.453497\\
5935	17.022348\\
5936	21.629869\\
5937	21.684814\\
5938	17.731991\\
5939	17.022348\\
5940	10.973088\\
5941	10.461222\\
5942	9.992064\\
5943	9.458918\\
5944	10.075026\\
5945	11.00813\\
5946	11.431046\\
5947	10.540899\\
5948	11.228636\\
5949	10.695615\\
5950	9.097365\\
5951	9.880065\\
5952	11.029842\\
5953	9.712767\\
5954	8.411833\\
5955	8.245299\\
5956	8.245299\\
5957	8.245299\\
5958	8.751087\\
5959	9.345939\\
5960	9.975518\\
5961	11.249007\\
5962	13.590539\\
5963	14.492719\\
5964	13.908956\\
5965	12.568371\\
5966	12.02948\\
5967	11.486331\\
5968	11.806079\\
5969	13.793762\\
5970	13.366292\\
5971	13.40833\\
5972	14.45395\\
5973	13.550703\\
5974	13.410799\\
5975	12.266239\\
5976	10.671572\\
5977	9.767924\\
5978	9.738578\\
5979	9.738572\\
5980	9.731664\\
5981	9.738568\\
5982	9.973632\\
5983	9.973631\\
5984	10.483277\\
5985	10.912072\\
5986	11.502626\\
5987	11.674296\\
5988	10.794903\\
5989	9.785841\\
5990	9.731664\\
5991	9.738573\\
5992	8.755037\\
5993	7.657907\\
5994	8.99142\\
5995	9.430075\\
5996	10.597634\\
5997	10.62077\\
5998	9.93017\\
5999	8.886986\\
6000	7.579261\\
6001	7.529694\\
6002	7.20441\\
6003	6.995208\\
6004	7.579664\\
6005	7.930258\\
6006	11.312488\\
6007	17.868118\\
6008	22.475634\\
6009	17.868085\\
6010	17.868098\\
6011	17.86808\\
6012	13.826045\\
6013	12.139561\\
6014	11.497083\\
6015	11.211443\\
6016	12.580404\\
6017	16.149548\\
6018	17.86808\\
6019	17.86808\\
6020	13.826111\\
6021	13.042083\\
6022	11.917003\\
6023	10.163438\\
6024	8.750078\\
6025	7.633668\\
6026	7.579395\\
6027	7.579423\\
6028	7.579409\\
6029	8.026183\\
6030	11.15836\\
6031	14.899957\\
6032	14.899956\\
6033	12.093608\\
6034	13.004918\\
6035	13.826085\\
6036	11.671596\\
6037	11.201984\\
6038	11.527648\\
6039	11.696385\\
6040	12.647015\\
6041	16.14954\\
6042	17.868079\\
6043	13.287837\\
6044	13.826086\\
6045	12.657342\\
6046	12.263242\\
6047	9.286801\\
6048	8.071606\\
6049	7.579009\\
6050	7.579664\\
6051	7.578814\\
6052	7.57978\\
6053	7.760177\\
6054	10.329326\\
6055	17.868082\\
6056	17.868082\\
6057	14.899957\\
6058	12.450343\\
6059	11.877044\\
6060	10.894154\\
6061	10.162713\\
6062	9.714267\\
6063	10.383444\\
6064	11.679722\\
6065	11.564251\\
6066	16.149559\\
6067	12.563282\\
6068	13.826096\\
6069	14.899957\\
6070	11.094669\\
6071	9.226459\\
6072	8.001655\\
6073	7.579361\\
6074	7.574503\\
6075	7.533949\\
6076	7.579592\\
6077	7.918994\\
6078	11.156452\\
6079	16.149556\\
6080	17.82816\\
6081	17.86808\\
6082	17.868079\\
6083	17.868079\\
6084	12.714696\\
6085	10.984999\\
6086	11.695215\\
6087	11.070573\\
6088	11.366894\\
6089	13.826085\\
6090	14.899956\\
6091	12.518875\\
6092	12.661619\\
6093	12.572875\\
6094	10.917694\\
6095	9.226469\\
6096	7.678587\\
6097	7.579635\\
6098	7.382229\\
6099	7.222794\\
6100	7.536544\\
6101	7.580744\\
6102	10.26071\\
6103	12.705064\\
6104	16.706039\\
6105	13.260807\\
6106	11.935255\\
6107	9.259417\\
6108	8.991395\\
6109	8.984452\\
6110	8.877986\\
6111	8.346142\\
6112	8.991402\\
6113	9.718545\\
6114	10.934428\\
6115	11.462633\\
6116	12.830937\\
6117	11.164758\\
6118	11.506979\\
6119	9.973631\\
6120	8.410033\\
6121	8.244757\\
6122	7.524987\\
6123	7.233607\\
6124	6.915917\\
6125	7.198463\\
6126	8.013413\\
6127	8.245299\\
6128	9.220326\\
6129	9.731666\\
6130	9.292585\\
6131	8.751116\\
6132	8.245299\\
6133	8.24507\\
6134	7.223903\\
6135	6.962617\\
6136	7.233648\\
6137	8.245299\\
6138	8.411795\\
6139	8.751123\\
6140	8.751136\\
6141	8.416414\\
6142	8.411819\\
6143	8.245299\\
6144	7.376048\\
6145	7.098775\\
6146	6.314774\\
6147	5.897647\\
6148	5.897648\\
6149	6.055071\\
6150	7.002168\\
6151	7.23365\\
6152	8.245299\\
6153	8.751122\\
6154	9.241461\\
6155	9.709127\\
6156	8.222831\\
6157	7.717005\\
6158	7.23317\\
6159	6.705354\\
6160	6.705356\\
6161	7.717005\\
6162	7.717005\\
6163	8.188898\\
6164	9.179315\\
6165	8.322799\\
6166	8.222832\\
6167	7.717005\\
6168	6.705355\\
6169	6.109227\\
6170	5.772571\\
6171	5.550802\\
6172	5.935161\\
6173	7.616618\\
6174	9.445338\\
6175	13.143787\\
6176	16.368457\\
6177	18.086996\\
6178	17.446941\\
6179	15.118877\\
6180	11.635534\\
6181	11.183117\\
6182	11.307744\\
6183	11.555999\\
6184	12.325469\\
6185	14.045007\\
6186	18.087\\
6187	18.086993\\
6188	18.087\\
6189	14.045007\\
6190	11.890107\\
6191	9.445338\\
6192	7.883577\\
6193	7.717005\\
6194	7.562049\\
6195	7.59647\\
6196	7.717005\\
6197	8.148939\\
6198	11.494953\\
6199	18.086997\\
6200	22.694506\\
6201	18.086997\\
6202	15.118877\\
6203	14.045007\\
6204	11.245012\\
6205	11.212761\\
6206	10.698417\\
6207	11.046459\\
6208	12.83675\\
6209	18.086978\\
6210	18.087001\\
6211	18.086998\\
6212	22.694521\\
6213	18.08699\\
6214	12.772489\\
6215	10.163502\\
6216	8.153868\\
6217	7.717005\\
6218	7.502779\\
6219	7.562029\\
6220	7.717003\\
6221	7.722373\\
6222	10.875447\\
6223	16.368467\\
6224	16.368456\\
6225	12.928157\\
6226	12.748925\\
6227	12.612683\\
6228	10.39648\\
6229	9.740252\\
6230	9.626629\\
6231	10.253463\\
6232	10.53108\\
6233	11.876698\\
6234	15.304961\\
6235	15.304962\\
6236	18.086992\\
6237	12.851554\\
6238	11.033428\\
6239	9.210285\\
6240	7.962883\\
6241	7.495402\\
6242	6.702686\\
6243	6.689577\\
6244	7.151806\\
6245	7.498083\\
6246	9.9089\\
6247	17.868072\\
6248	22.4756\\
6249	22.530544\\
6250	22.530545\\
6251	38.484788\\
6252	22.530545\\
6253	22.4756\\
6254	22.475607\\
6255	17.868079\\
6256	17.868079\\
6257	22.530547\\
6258	22.53056\\
6259	17.86808\\
6260	22.530544\\
6261	15.08604\\
6262	11.665727\\
6263	9.371231\\
6264	8.359867\\
6265	7.498085\\
6266	7.498083\\
6267	7.490008\\
6268	7.498083\\
6269	7.911336\\
6270	10.996583\\
6271	22.47553\\
6272	22.530543\\
6273	22.530544\\
6274	22.530544\\
6275	22.475601\\
6276	17.868079\\
6277	16.149537\\
6278	17.868071\\
6279	16.149541\\
6280	17.867783\\
6281	17.868083\\
6282	17.868081\\
6283	13.826085\\
6284	17.868081\\
6285	12.364942\\
6286	11.780094\\
6287	9.920214\\
6288	8.137849\\
6289	7.498083\\
6290	6.916303\\
6291	6.486433\\
6292	6.351762\\
6293	6.480194\\
6294	6.913928\\
6295	7.498083\\
6296	8.142128\\
6297	9.226416\\
6298	8.991363\\
6299	8.98445\\
6300	8.53333\\
6301	8.402857\\
6302	8.003366\\
6303	7.94716\\
6304	8.003922\\
6305	8.735414\\
6306	8.991368\\
6307	8.991363\\
6308	9.505007\\
6309	8.961548\\
6310	8.57187\\
6311	7.498083\\
6312	6.986202\\
6313	6.124486\\
6314	5.150434\\
6315	5.150432\\
6316	5.150438\\
6317	5.150434\\
6318	5.150438\\
6319	5.150432\\
6320	5.427057\\
6321	6.824233\\
6322	7.498083\\
6323	7.493623\\
6324	6.283228\\
6325	5.150432\\
6326	5.150432\\
6327	5.150447\\
6328	5.150432\\
6329	5.150432\\
6330	6.981718\\
6331	7.498083\\
6332	7.520965\\
6333	7.498083\\
6334	7.498083\\
6335	6.643151\\
6336	7.233645\\
6337	6.31648\\
6338	5.897683\\
6339	5.897647\\
6340	5.897647\\
6341	7.233647\\
6342	9.738564\\
6343	12.319983\\
6344	14.5733\\
6345	16.896755\\
6346	14.5733\\
6347	13.636252\\
6348	11.651496\\
6349	11.392017\\
6350	11.310482\\
6351	11.344686\\
6352	12.079105\\
6353	11.667235\\
6354	13.826087\\
6355	14.89996\\
6356	16.149535\\
6357	12.856558\\
6358	11.738744\\
6359	9.226415\\
6360	8.004062\\
6361	7.498083\\
6362	7.34339\\
6363	6.981334\\
6364	7.498082\\
6365	7.688141\\
6366	11.259538\\
6367	22.4756\\
6368	22.475602\\
6369	17.868079\\
6370	17.228023\\
6371	13.826085\\
6372	11.359947\\
6373	10.737366\\
6374	10.733689\\
6375	11.3477\\
6376	12.66959\\
6377	17.868072\\
6378	17.868079\\
6379	22.4756\\
6380	22.530545\\
6381	16.149542\\
6382	12.436393\\
6383	9.597942\\
6384	8.003901\\
6385	7.498083\\
6386	6.486527\\
6387	6.484417\\
6388	6.486432\\
6389	7.498083\\
6390	9.226416\\
6391	12.759965\\
6392	14.899944\\
6393	15.086038\\
6394	12.997684\\
6395	17.868077\\
6396	13.826222\\
6397	13.826086\\
6398	12.718957\\
6399	12.581248\\
6400	12.473828\\
6401	16.14954\\
6402	12.872912\\
6403	12.554302\\
6404	13.826084\\
6405	11.034893\\
6406	9.969431\\
6407	8.813688\\
6408	7.498083\\
6409	6.919475\\
6410	6.486435\\
6411	6.486432\\
6412	6.486441\\
6413	7.498083\\
6414	9.966218\\
6415	17.868079\\
6416	17.868082\\
6417	17.868079\\
6418	15.086039\\
6419	12.87448\\
6420	10.387615\\
6421	9.722329\\
6422	9.226417\\
6423	9.226416\\
6424	9.519726\\
6425	10.508403\\
6426	12.159167\\
6427	12.549235\\
6428	14.275516\\
6429	11.168518\\
6430	9.226416\\
6431	8.456719\\
6432	7.664582\\
6433	7.498083\\
6434	6.648348\\
6435	6.486432\\
6436	6.63346\\
6437	7.498083\\
6438	9.622529\\
6439	17.868077\\
6440	17.868079\\
6441	17.868079\\
6442	16.149613\\
6443	17.840724\\
6444	17.868079\\
6445	16.14954\\
6446	17.868077\\
6447	17.868079\\
6448	22.475608\\
6449	22.530544\\
6450	22.530546\\
6451	22.530548\\
6452	22.530533\\
6453	16.14954\\
6454	15.086038\\
6455	11.552676\\
6456	9.226416\\
6457	13.334126\\
6458	12.816716\\
6459	12.11159\\
6460	12.078845\\
6461	12.111585\\
6462	13.092211\\
6463	13.825364\\
6464	15.721159\\
6465	17.816904\\
6466	20.257216\\
6467	17.933757\\
6468	15.223861\\
6469	14.115699\\
6470	13.334121\\
6471	13.245628\\
6472	13.494805\\
6473	15.575092\\
6474	16.593204\\
6475	17.933756\\
6476	20.257211\\
6477	16.95232\\
6478	15.562914\\
6479	13.704355\\
6480	12.533354\\
6481	7.962131\\
6482	7.19751\\
6483	6.575098\\
6484	6.359315\\
6485	6.668388\\
6486	7.278072\\
6487	8.184388\\
6488	8.245299\\
6489	8.245299\\
6490	8.245299\\
6491	8.245299\\
6492	8.089679\\
6493	6.877994\\
6494	6.249789\\
6495	6.31648\\
6496	7.038521\\
6497	8.245299\\
6498	9.973632\\
6499	17.93376\\
6500	21.975872\\
6501	11.084411\\
6502	9.973575\\
6503	8.751124\\
6504	8.245299\\
6505	7.235043\\
6506	6.896669\\
6507	6.853073\\
6508	7.233153\\
6509	8.245299\\
6510	10.541828\\
6511	18.615295\\
6512	18.615295\\
6513	16.896765\\
6514	14.573301\\
6515	14.573301\\
6516	12.513299\\
6517	12.654781\\
6518	12.984204\\
6519	18.615295\\
6520	23.27776\\
6521	145.217096\\
6522	189.961798\\
6523	145.217116\\
6524	24.244632\\
6525	23.277759\\
6526	16.896759\\
6527	11.895696\\
6528	9.738574\\
6529	8.355426\\
6530	8.245299\\
6531	8.245299\\
6532	8.245299\\
6533	8.677358\\
6534	21.975749\\
6535	49.338843\\
6536	190.7038\\
6537	190.703828\\
6538	39.243108\\
6539	27.12977\\
6540	26.638213\\
6541	26.638214\\
6542	18.615295\\
6543	18.615295\\
6544	18.615296\\
6545	24.244459\\
6546	39.232\\
6547	159.131475\\
6548	145.217116\\
6549	23.277767\\
6550	18.615295\\
6551	12.854206\\
6552	9.43013\\
6553	8.491792\\
6554	8.435123\\
6555	7.959598\\
6556	8.412563\\
6557	8.492555\\
6558	11.748285\\
6559	23.973647\\
6560	22.229267\\
6561	19.171798\\
6562	16.306602\\
6563	19.171797\\
6564	11.910437\\
6565	12.005797\\
6566	11.894952\\
6567	12.116825\\
6568	15.008969\\
6569	22.229267\\
6570	23.973647\\
6571	23.973647\\
6572	22.775767\\
6573	23.973647\\
6574	19.171798\\
6575	11.813162\\
6576	10.029715\\
6577	8.936803\\
6578	8.491792\\
6579	8.491797\\
6580	8.491798\\
6581	9.43614\\
6582	17.401883\\
6583	78.749884\\
6584	144.626093\\
6585	144.626071\\
6586	39.690301\\
6587	40.404839\\
6588	24.913421\\
6589	23.973647\\
6590	24.913427\\
6591	28.516773\\
6592	36.360082\\
6593	144.626081\\
6594	129.856998\\
6595	58.986872\\
6596	36.360081\\
6597	23.973647\\
6598	22.229267\\
6599	12.000331\\
6600	10.029713\\
6601	9.094527\\
6602	8.522976\\
6603	8.491798\\
6604	8.491792\\
6605	9.635745\\
6606	16.306591\\
6607	164.342673\\
6608	36.360081\\
6609	23.973647\\
6610	22.229267\\
6611	22.775766\\
6612	15.008969\\
6613	12.344236\\
6614	12.274242\\
6615	12.250017\\
6616	17.401882\\
6617	22.229267\\
6618	23.973647\\
6619	23.973647\\
6620	23.973647\\
6621	22.229267\\
6622	17.401911\\
6623	11.460607\\
6624	9.155058\\
6625	8.491785\\
6626	7.493216\\
6627	7.209271\\
6628	6.07586\\
6629	6.073941\\
6630	6.809894\\
6631	7.937764\\
6632	8.411696\\
6633	8.2089\\
6634	7.788518\\
6635	7.476668\\
6636	7.182996\\
6637	6.697816\\
6638	6.505292\\
6639	7.449782\\
6640	8.306548\\
6641	8.788126\\
6642	9.446028\\
6643	9.139923\\
6644	9.541477\\
6645	9.012738\\
6646	8.491786\\
6647	8.491771\\
6648	7.179034\\
6649	6.073947\\
6650	6.073939\\
6651	6.073938\\
6652	6.073938\\
6653	6.073944\\
6654	6.591118\\
6655	7.658999\\
6656	8.49179\\
6657	8.491792\\
6658	9.153165\\
6659	9.94387\\
6660	9.913031\\
6661	9.012738\\
6662	8.491791\\
6663	8.491791\\
6664	9.012705\\
6665	10.029676\\
6666	11.91186\\
6667	11.773916\\
6668	11.354754\\
6669	11.855366\\
6670	10.485951\\
6671	9.939884\\
6672	8.49179\\
6673	7.849768\\
6674	7.313034\\
6675	7.285128\\
6676	7.449897\\
6677	8.49179\\
6678	11.443374\\
6679	22.229184\\
6680	19.513736\\
6681	22.229268\\
6682	19.171798\\
6683	16.306589\\
6684	13.872458\\
6685	16.114884\\
6686	15.008968\\
6687	12.017993\\
6688	12.612699\\
6689	13.801599\\
6690	13.801569\\
6691	11.667345\\
6692	10.122262\\
6693	8.822768\\
6694	8.349818\\
6695	7.378683\\
6696	6.097095\\
6697	4.43339\\
6698	4.433392\\
6699	4.433392\\
6700	4.433392\\
6701	4.452686\\
6702	7.110764\\
6703	10.941791\\
6704	9.668663\\
6705	8.816106\\
6706	8.389146\\
6707	8.498111\\
6708	6.866088\\
6709	7.372172\\
6710	6.963445\\
6711	6.851226\\
6712	7.008034\\
6713	8.389142\\
6714	10.016704\\
6715	9.848938\\
6716	9.790978\\
6717	8.647731\\
6718	8.382025\\
6719	7.030142\\
6720	6.136175\\
6721	5.109187\\
6722	4.43339\\
6723	4.43339\\
6724	4.841901\\
6725	5.970089\\
6726	8.381955\\
6727	10.734265\\
6728	11.29563\\
6729	12.9313\\
6730	12.679464\\
6731	13.368405\\
6732	14.666004\\
6733	12.485028\\
6734	11.089078\\
6735	9.902356\\
6736	9.01612\\
6737	9.897807\\
6738	10.749924\\
6739	9.996485\\
6740	9.716565\\
6741	8.560417\\
6742	8.389144\\
6743	6.902563\\
6744	5.978018\\
6745	4.884945\\
6746	4.43339\\
6747	4.433389\\
6748	4.433389\\
6749	5.783558\\
6750	8.010589\\
6751	10.71019\\
6752	12.177476\\
6753	10.46924\\
6754	10.522385\\
6755	9.605565\\
6756	7.757196\\
6757	8.053486\\
6758	8.389145\\
6759	8.631225\\
6760	8.749599\\
6761	10.433105\\
6762	9.879597\\
6763	10.794434\\
6764	9.680437\\
6765	8.68564\\
6766	8.381946\\
6767	7.14713\\
6768	5.850695\\
6769	4.760482\\
6770	4.43339\\
6771	4.43339\\
6772	4.433389\\
6773	5.80866\\
6774	8.389137\\
6775	12.675483\\
6776	11.642129\\
6777	11.251595\\
6778	10.520204\\
6779	11.850614\\
6780	9.202963\\
6781	8.905286\\
6782	8.961496\\
6783	8.631228\\
6784	9.199256\\
6785	10.792454\\
6786	12.536886\\
6787	14.474385\\
6788	16.872033\\
6789	11.869924\\
6790	11.283672\\
6791	9.877426\\
6792	7.420861\\
6793	6.851167\\
6794	5.80933\\
6795	5.444526\\
6796	5.21588\\
6797	5.777659\\
6798	6.851083\\
6799	7.372169\\
6800	8.389145\\
6801	8.631225\\
6802	8.631229\\
6803	8.631225\\
6804	8.389142\\
6805	8.298061\\
6806	7.296223\\
6807	7.116267\\
6808	7.372171\\
6809	8.33792\\
6810	8.389145\\
6811	9.149528\\
6812	8.631224\\
6813	8.378653\\
6814	7.197284\\
6815	6.851224\\
6816	6.323323\\
6817	5.23084\\
6818	4.538094\\
6819	4.433394\\
6820	4.433393\\
6821	4.894997\\
6822	5.80124\\
6823	6.845755\\
6824	6.851224\\
6825	6.933666\\
6826	7.144229\\
6827	6.851224\\
6828	6.851224\\
6829	6.447771\\
6830	5.80933\\
6831	5.80933\\
6832	5.989082\\
6833	6.851224\\
6834	6.851224\\
6835	7.214234\\
6836	6.851224\\
6837	6.684683\\
6838	6.691663\\
6839	5.80933\\
6840	5.35619\\
6841	4.866599\\
6842	4.866524\\
6843	4.866696\\
6844	4.866959\\
6845	6.347744\\
6846	8.738718\\
6847	10.446826\\
6848	9.860751\\
6849	9.064385\\
6850	9.064392\\
6851	9.183043\\
6852	8.876331\\
6853	9.064368\\
6854	8.955529\\
6855	9.058708\\
6856	9.064492\\
6857	10.468854\\
6858	17.96436\\
6859	17.964362\\
6860	10.478104\\
6861	9.26214\\
6862	9.064501\\
6863	7.402087\\
6864	6.514269\\
6865	4.863523\\
6866	4.433391\\
6867	4.43339\\
6868	4.43339\\
6869	5.80933\\
6870	8.054565\\
6871	17.53123\\
6872	15.761316\\
6873	9.816815\\
6874	8.631225\\
6875	8.631226\\
6876	8.061838\\
6877	8.389145\\
6878	8.424564\\
6879	8.726231\\
6880	9.728742\\
6881	17.53123\\
6882	22.333079\\
6883	23.328693\\
6884	22.333079\\
6885	17.53123\\
6886	15.761365\\
6887	8.631227\\
6888	7.372171\\
6889	9.01279\\
6890	8.491791\\
6891	8.49179\\
6892	8.940893\\
6893	10.029672\\
6894	23.973433\\
6895	718.036701\\
6896	33.248531\\
6897	22.333085\\
6898	22.333107\\
6899	22.333086\\
6900	20.588701\\
6901	20.5887\\
6902	20.5887\\
6903	21.135199\\
6904	22.33308\\
6905	38.764268\\
6906	187.130922\\
6907	567.89465\\
6908	142.386241\\
6909	22.333079\\
6910	17.53123\\
6911	9.76549\\
6912	7.372172\\
6913	6.851224\\
6914	5.810037\\
6915	5.80933\\
6916	5.80933\\
6917	6.851223\\
6918	8.631223\\
6919	22.33308\\
6920	21.1352\\
6921	22.333079\\
6922	22.333079\\
6923	20.5887\\
6924	17.53123\\
6925	17.53123\\
6926	14.474377\\
6927	15.761315\\
6928	15.761316\\
6929	22.333079\\
6930	52.261719\\
6931	23.328694\\
6932	22.33308\\
6933	20.5887\\
6934	17.53123\\
6935	9.975048\\
6936	7.372145\\
6937	6.851224\\
6938	5.959021\\
6939	5.809347\\
6940	6.428546\\
6941	6.851224\\
6942	9.411461\\
6943	23.328693\\
6944	26.876236\\
6945	22.358989\\
6946	22.33308\\
6947	17.53123\\
6948	13.368401\\
6949	9.934424\\
6950	9.381774\\
6951	9.603427\\
6952	14.858628\\
6953	17.53123\\
6954	14.474382\\
6955	15.761371\\
6956	12.930016\\
6957	9.910515\\
6958	9.234031\\
6959	8.382499\\
6960	6.851224\\
6961	6.157225\\
6962	4.960861\\
6963	4.960856\\
6964	4.960861\\
6965	4.960853\\
6966	4.960861\\
6967	6.336802\\
6968	7.378684\\
6969	7.378684\\
6970	8.523602\\
6971	8.654729\\
6972	8.49179\\
6973	7.599434\\
6974	7.449684\\
6975	7.449898\\
6976	7.997477\\
6977	8.828078\\
6978	10.271966\\
6979	9.064449\\
6980	7.899593\\
6981	7.378683\\
6982	7.378683\\
6983	7.172467\\
6984	5.223428\\
6985	4.960857\\
6986	4.960862\\
6987	4.463902\\
6988	4.503402\\
6989	4.613925\\
6990	4.960862\\
6991	4.960861\\
6992	4.970925\\
6993	5.291189\\
6994	5.392216\\
6995	5.388665\\
6996	4.960861\\
6997	4.43339\\
6998	4.433391\\
6999	4.433391\\
7000	4.43339\\
7001	4.43339\\
7002	5.903665\\
7003	6.851224\\
7004	6.851224\\
7005	6.708651\\
7006	5.80933\\
7007	5.088897\\
7008	4.433391\\
7009	4.431111\\
7010	4.099082\\
7011	3.97486\\
7012	4.433384\\
7013	4.433389\\
7014	6.851224\\
7015	10.551389\\
7016	11.466807\\
7017	11.313588\\
7018	10.777135\\
7019	10.548845\\
7020	8.95396\\
7021	9.192415\\
7022	9.026496\\
7023	9.026553\\
7024	9.407308\\
7025	11.122309\\
7026	9.93804\\
7027	9.088564\\
7028	8.631226\\
7029	6.851225\\
7030	6.851322\\
7031	6.778106\\
7032	5.34263\\
7033	4.433391\\
7034	4.433396\\
7035	4.43339\\
7036	4.433389\\
7037	4.433392\\
7038	6.851226\\
7039	8.631223\\
7040	8.932392\\
7041	8.814777\\
7042	8.743862\\
7043	10.16018\\
7044	8.631225\\
7045	8.466691\\
7046	8.389143\\
7047	7.763487\\
7048	7.641642\\
7049	8.550012\\
7050	8.389145\\
7051	8.230638\\
7052	6.851224\\
7053	5.809371\\
7054	5.613998\\
7055	4.433401\\
7056	5.304405\\
7057	5.304255\\
7058	4.468355\\
7059	4.052996\\
7060	4.667247\\
7061	5.304403\\
7062	7.562625\\
7063	9.502247\\
7064	8.389146\\
7065	8.272752\\
7066	7.372171\\
7067	7.37217\\
7068	7.022551\\
7069	6.851229\\
7070	6.851229\\
7071	7.129525\\
7072	7.511567\\
7073	8.389145\\
7074	11.37308\\
7075	10.329844\\
7076	10.547042\\
7077	8.389146\\
7078	7.51362\\
7079	6.52617\\
7080	4.433604\\
7081	4.433389\\
7082	4.433389\\
7083	4.433389\\
7084	4.433389\\
7085	4.433389\\
7086	6.851225\\
7087	9.017808\\
7088	8.63698\\
7089	8.456044\\
7090	8.389145\\
7091	8.38915\\
7092	7.667096\\
7093	8.183957\\
7094	7.602416\\
7095	8.382439\\
7096	8.389142\\
7097	8.555712\\
7098	8.38915\\
7099	8.38916\\
7100	7.178273\\
7101	6.851224\\
7102	6.573643\\
7103	5.616935\\
7104	4.433389\\
7105	4.433384\\
7106	3.217174\\
7107	2.988622\\
7108	3.00643\\
7109	4.433389\\
7110	5.705244\\
7111	6.851247\\
7112	7.487522\\
7113	8.389145\\
7114	8.389144\\
7115	8.389145\\
7116	8.631225\\
7117	8.38914\\
7118	7.513384\\
7119	7.614856\\
7120	7.777754\\
7121	8.30595\\
7122	8.995407\\
7123	7.728343\\
7124	7.271018\\
7125	6.851224\\
7126	6.851224\\
7127	7.022699\\
7128	6.197421\\
7129	4.864747\\
7130	4.433389\\
7131	4.433389\\
7132	4.433389\\
7133	4.433389\\
7134	5.5692\\
7135	6.816851\\
7136	7.405456\\
7137	8.058403\\
7138	8.050027\\
7139	8.386923\\
7140	8.389145\\
7141	6.851224\\
7142	6.191175\\
7143	5.809337\\
7144	6.620322\\
7145	6.96902\\
7146	8.382024\\
7147	7.282192\\
7148	6.851228\\
7149	6.648384\\
7150	6.830332\\
7151	6.381387\\
7152	4.43339\\
7153	4.433389\\
7154	3.785472\\
7155	3.396949\\
7156	3.038769\\
7157	3.179694\\
7158	3.006443\\
7159	3.972636\\
7160	4.433389\\
7161	4.43339\\
7162	4.43339\\
7163	4.433389\\
7164	4.433389\\
7165	4.433389\\
7166	4.29442\\
7167	4.185365\\
7168	5.304403\\
7169	5.304403\\
7170	5.61683\\
7171	7.605171\\
7172	6.680345\\
7173	4.43339\\
7174	4.43339\\
7175	4.43339\\
7176	4.43339\\
7177	3.785298\\
7178	2.589927\\
7179	2.589901\\
7180	2.589901\\
7181	2.589905\\
7182	4.091642\\
7183	5.09075\\
7184	6.851224\\
7185	6.851227\\
7186	6.851224\\
7187	6.85122\\
7188	5.72345\\
7189	5.80933\\
7190	5.80933\\
7191	5.80933\\
7192	6.752257\\
7193	7.089118\\
7194	8.184568\\
7195	9.468916\\
7196	8.389143\\
7197	7.851611\\
7198	6.851224\\
7199	6.297307\\
7200	5.569624\\
7201	4.43339\\
7202	4.433325\\
7203	3.498689\\
7204	3.00642\\
7205	3.674861\\
7206	4.433389\\
7207	5.876266\\
7208	6.853926\\
7209	6.851228\\
7210	6.851224\\
7211	6.805663\\
7212	6.429061\\
7213	5.80933\\
7214	5.809333\\
7215	5.933859\\
7216	6.851224\\
7217	7.516748\\
7218	8.631225\\
7219	9.362855\\
7220	8.522868\\
7221	7.513223\\
7222	6.851227\\
7223	6.851224\\
7224	5.814456\\
7225	4.43339\\
7226	4.433385\\
7227	3.785472\\
7228	3.687005\\
7229	4.05194\\
7230	4.43339\\
7231	6.549598\\
7232	7.573853\\
7233	8.631226\\
7234	8.664043\\
7235	9.75051\\
7236	10.89764\\
7237	11.129131\\
7238	11.199479\\
7239	11.008119\\
7240	11.027368\\
7241	11.551881\\
7242	13.368401\\
7243	13.368401\\
7244	10.109069\\
7245	8.803309\\
7246	8.381977\\
7247	7.590732\\
7248	7.141119\\
7249	5.878543\\
7250	4.43339\\
7251	4.43339\\
7252	4.433389\\
7253	4.43339\\
7254	4.693298\\
7255	6.851231\\
7256	8.121002\\
7257	8.59332\\
7258	8.948153\\
7259	8.631224\\
7260	8.631227\\
7261	8.172191\\
7262	8.307748\\
7263	8.014872\\
7264	8.247445\\
7265	8.389072\\
7266	9.259314\\
7267	12.151969\\
7268	9.619563\\
7269	7.372171\\
7270	7.722238\\
7271	7.722235\\
7272	7.156565\\
7273	5.304406\\
7274	5.304403\\
7275	4.366754\\
7276	3.877443\\
7277	4.642399\\
7278	5.304403\\
7279	6.769681\\
7280	7.893711\\
7281	8.243192\\
7282	8.243174\\
7283	7.722238\\
7284	7.722239\\
7285	7.62268\\
7286	7.037056\\
7287	6.75202\\
7288	7.12363\\
7289	6.851204\\
7290	7.023014\\
7291	7.514257\\
7292	6.851231\\
7293	6.106692\\
7294	4.43339\\
7295	4.43339\\
7296	5.840531\\
7297	4.377665\\
7298	3.893476\\
7299	2.460283\\
7300	1.856514\\
7301	1.472052\\
7302	1.78175\\
7303	2.964855\\
7304	3.893478\\
7305	3.893481\\
7306	3.893477\\
7307	3.750026\\
7308	3.651463\\
7309	3.893476\\
7310	3.893319\\
7311	3.656582\\
7312	3.893476\\
7313	4.472373\\
7314	5.840531\\
7315	5.840531\\
7316	5.840531\\
7317	5.83523\\
7318	5.840514\\
7319	5.84053\\
7320	5.84053\\
7321	3.893478\\
7322	2.26507\\
7323	1.15808\\
7324	0.170531\\
7325	0.000326\\
7326	0.093339\\
7327	1.158078\\
7328	1.445718\\
7329	1.543226\\
7330	1.445373\\
7331	1.206841\\
7332	1.166334\\
7333	1.856526\\
7334	1.566052\\
7335	1.941131\\
7336	2.770054\\
7337	3.858059\\
7338	4.31727\\
7339	5.477684\\
7340	5.160498\\
7341	4.263288\\
7342	4.333392\\
7343	4.314598\\
7344	4.078718\\
7345	2.006772\\
7346	0.600973\\
7347	1e-06\\
7348	7.8e-05\\
7349	0.000321\\
7350	1.873582\\
7351	5.1474\\
7352	6.141911\\
7353	6.165691\\
7354	5.840545\\
7355	6.677097\\
7356	6.333899\\
7357	6.296089\\
7358	6.849316\\
7359	7.041813\\
7360	7.191604\\
7361	8.022586\\
7362	8.394198\\
7363	8.394196\\
7364	8.394196\\
7365	7.41022\\
7366	6.535874\\
7367	6.727965\\
7368	6.296121\\
7369	5.84053\\
7370	5.357579\\
7371	4.458721\\
7372	4.71611\\
7373	5.245517\\
7374	5.84053\\
7375	8.234305\\
7376	9.093537\\
7377	8.944403\\
7378	10.011007\\
7379	10.974187\\
7380	11.277326\\
7381	11.047243\\
7382	11.242586\\
7383	11.354542\\
7384	12.158307\\
7385	10.570511\\
7386	19.67419\\
7387	19.67419\\
7388	11.41604\\
7389	11.06822\\
7390	9.653967\\
7391	8.944629\\
7392	8.533891\\
7393	7.318989\\
7394	6.019987\\
7395	5.840531\\
7396	5.840531\\
7397	5.840531\\
7398	6.78533\\
7399	9.073661\\
7400	12.42575\\
7401	12.341261\\
7402	11.674924\\
7403	12.117727\\
7404	11.281049\\
7405	11.287655\\
7406	11.286098\\
7407	11.086678\\
7408	11.9333\\
7409	15.277499\\
7410	17.804843\\
7411	19.67419\\
7412	10.764494\\
7413	11.257102\\
7414	10.010993\\
7415	9.38661\\
7416	8.394196\\
7417	7.280767\\
7418	5.840531\\
7419	5.84053\\
7420	5.84053\\
7421	5.84053\\
7422	5.840531\\
7423	8.394196\\
7424	9.864881\\
7425	7.830089\\
7426	8.439239\\
7427	8.661069\\
7428	7.786329\\
7429	7.712651\\
7430	7.591971\\
7431	7.249631\\
7432	7.24976\\
7433	7.504812\\
7434	7.698559\\
7435	8.334255\\
7436	7.236117\\
7437	7.117494\\
7438	5.120843\\
7439	5.138\\
7440	4.682451\\
7441	4.056969\\
7442	2.735399\\
7443	1.723973\\
7444	0.772378\\
7445	1.060768\\
7446	2.735399\\
7447	4.682451\\
7448	5.830183\\
7449	6.135691\\
7450	6.677073\\
7451	6.818309\\
7452	7.236117\\
7453	7.236116\\
7454	7.167235\\
7455	6.690288\\
7456	6.212499\\
7457	5.882501\\
7458	6.400144\\
7459	7.236116\\
7460	6.679013\\
7461	5.01835\\
7462	4.682455\\
7463	4.682451\\
7464	4.819444\\
7465	4.682452\\
7466	3.175332\\
7467	2.735404\\
7468	2.069396\\
7469	1.930087\\
7470	3.893478\\
7471	4.332852\\
7472	5.840389\\
7473	5.840531\\
7474	5.840531\\
7475	5.649724\\
7476	5.553342\\
7477	5.808636\\
7478	5.464375\\
7479	5.750039\\
7480	4.682446\\
7481	4.682452\\
7482	4.682451\\
7483	5.60302\\
7484	4.682453\\
7485	4.682452\\
7486	4.682452\\
7487	4.682452\\
7488	4.682451\\
7489	3.648722\\
7490	2.735399\\
7491	1.951239\\
7492	1.460069\\
7493	1.204212\\
7494	1.866976\\
7495	2.713388\\
7496	2.735402\\
7497	3.085078\\
7498	3.687073\\
7499	4.301709\\
7500	4.682453\\
7501	4.682451\\
7502	4.682451\\
7503	4.682451\\
7504	4.682451\\
7505	4.682452\\
7506	5.13803\\
7507	6.135434\\
7508	6.13569\\
7509	5.138039\\
7510	4.927993\\
7511	4.683532\\
7512	4.682451\\
7513	4.682452\\
7514	4.164415\\
7515	3.156493\\
7516	2.798268\\
7517	2.984069\\
7518	3.648654\\
7519	4.682452\\
7520	4.682451\\
7521	4.765122\\
7522	4.682451\\
7523	4.682451\\
7524	4.682451\\
7525	4.682452\\
7526	4.682452\\
7527	4.682453\\
7528	4.682451\\
7529	4.682451\\
7530	6.617561\\
7531	7.037371\\
7532	5.877942\\
7533	5.138439\\
7534	4.682451\\
7535	5.160237\\
7536	4.682451\\
7537	4.68245\\
7538	3.156519\\
7539	2.735405\\
7540	2.703927\\
7541	2.579075\\
7542	2.735399\\
7543	3.085078\\
7544	2.735403\\
7545	2.735404\\
7546	3.085079\\
7547	3.918553\\
7548	4.44799\\
7549	4.682449\\
7550	4.208699\\
7551	3.6392\\
7552	4.141865\\
7553	4.682451\\
7554	4.682452\\
7555	5.128669\\
7556	4.766646\\
7557	4.781928\\
7558	4.682452\\
7559	4.682455\\
7560	4.682453\\
7561	3.998204\\
7562	2.735402\\
7563	2.562969\\
7564	2.530222\\
7565	2.735404\\
7566	4.08819\\
7567	5.943622\\
7568	8.124254\\
7569	7.70616\\
7570	7.878188\\
7571	8.343097\\
7572	8.425366\\
7573	8.335072\\
7574	8.859382\\
7575	8.860047\\
7576	8.860435\\
7577	8.71795\\
7578	11.031181\\
7579	14.11942\\
7580	9.394985\\
7581	8.90417\\
7582	7.613246\\
7583	7.786327\\
7584	7.236122\\
7585	6.13569\\
7586	4.682451\\
7587	4.682452\\
7588	4.682453\\
7589	4.682453\\
7590	4.868593\\
7591	7.359983\\
7592	8.860439\\
7593	8.853191\\
7594	7.828261\\
7595	7.236116\\
7596	7.236117\\
7597	7.23611\\
7598	7.236117\\
7599	7.236136\\
7600	7.236117\\
7601	7.935476\\
7602	10.644325\\
7603	11.343618\\
7604	9.03279\\
7605	8.506754\\
7606	7.236123\\
7607	7.236117\\
7608	7.030138\\
7609	4.682452\\
7610	4.682451\\
7611	3.686614\\
7612	3.177174\\
7613	3.300996\\
7614	4.682453\\
7615	6.144596\\
7616	8.151697\\
7617	8.698423\\
7618	8.860436\\
7619	8.595007\\
7620	8.899509\\
7621	8.780282\\
7622	9.021684\\
7623	9.116117\\
7624	9.241446\\
7625	10.123264\\
7626	10.05993\\
7627	10.129643\\
7628	8.936417\\
7629	9.113695\\
7630	7.632881\\
7631	7.937419\\
7632	7.236116\\
7633	4.682457\\
7634	4.682451\\
7635	4.50695\\
7636	3.761952\\
7637	3.551259\\
7638	4.147561\\
7639	4.682451\\
7640	4.682451\\
7641	5.882521\\
7642	6.864744\\
7643	7.236117\\
7644	7.236117\\
7645	7.236117\\
7646	7.236116\\
7647	6.828295\\
7648	6.347245\\
7649	6.135691\\
7650	6.549943\\
7651	7.131758\\
7652	7.236117\\
7653	6.135692\\
7654	4.682457\\
7655	5.137572\\
7656	4.976053\\
7657	4.682451\\
7658	4.199291\\
7659	3.156508\\
7660	2.748362\\
7661	2.737563\\
7662	2.735404\\
7663	3.156519\\
7664	4.010885\\
7665	4.682451\\
7666	4.682451\\
7667	4.999869\\
7668	5.753284\\
7669	6.135691\\
7670	6.052645\\
7671	5.290876\\
7672	5.660068\\
7673	5.773567\\
7674	6.833882\\
7675	7.236117\\
7676	7.236117\\
7677	6.66946\\
7678	6.135513\\
7679	5.882529\\
7680	5.138032\\
7681	4.682451\\
7682	4.273855\\
7683	3.443566\\
7684	3.156518\\
7685	3.175316\\
7686	4.682451\\
7687	6.372751\\
7688	8.645978\\
7689	8.860436\\
7690	9.116116\\
7691	8.860439\\
7692	8.875136\\
7693	8.638883\\
7694	8.860436\\
7695	10.340997\\
7696	11.155313\\
7697	10.633473\\
7698	17.804834\\
7699	16.802447\\
7700	9.501979\\
7701	8.860436\\
7702	7.236117\\
7703	7.403753\\
7704	7.236116\\
7705	5.59807\\
7706	4.682451\\
7707	4.682451\\
7708	4.682451\\
7709	4.682451\\
7710	4.682452\\
7711	7.236162\\
7712	10.286491\\
7713	9.850568\\
7714	9.19442\\
7715	9.116116\\
7716	9.116116\\
7717	9.116116\\
7718	9.488807\\
7719	9.442071\\
7720	10.741954\\
7721	9.715229\\
7722	14.119419\\
7723	11.965419\\
7724	10.726908\\
7725	9.474383\\
7726	8.133485\\
7727	8.185843\\
7728	7.236117\\
7729	6.052817\\
7730	4.682452\\
7731	4.682451\\
7732	4.682451\\
7733	4.682451\\
7734	4.682452\\
7735	8.706276\\
7736	12.709175\\
7737	10.765351\\
7738	11.297729\\
7739	11.10729\\
7740	11.080454\\
7741	10.766289\\
7742	11.711149\\
7743	11.068891\\
7744	11.503576\\
7745	10.226205\\
7746	11.683304\\
7747	11.216948\\
7748	9.204695\\
7749	9.661947\\
7750	8.04396\\
7751	7.417222\\
7752	7.236117\\
7753	6.135691\\
7754	4.682452\\
7755	4.682451\\
7756	4.682451\\
7757	4.682451\\
7758	5.138038\\
7759	7.455065\\
7760	10.444328\\
7761	9.480073\\
7762	9.116116\\
7763	8.826014\\
7764	8.123208\\
7765	8.09718\\
7766	8.71623\\
7767	8.852919\\
7768	8.860436\\
7769	9.116122\\
7770	9.856441\\
7771	10.080539\\
7772	8.860791\\
7773	8.860436\\
7774	7.236117\\
7775	7.236117\\
7776	7.236117\\
7777	5.882517\\
7778	4.682452\\
7779	4.682451\\
7780	4.682451\\
7781	4.682451\\
7782	4.682458\\
7783	7.236117\\
7784	8.982788\\
7785	9.751626\\
7786	9.723128\\
7787	9.762406\\
7788	9.116119\\
7789	8.860436\\
7790	9.335275\\
7791	9.053014\\
7792	9.002951\\
7793	9.342058\\
7794	9.932455\\
7795	9.871755\\
7796	8.860436\\
7797	7.707035\\
7798	7.207821\\
7799	7.236117\\
7800	7.234962\\
7801	4.820239\\
7802	4.682451\\
7803	3.906271\\
7804	3.15652\\
7805	2.945239\\
7806	3.08644\\
7807	3.751145\\
7808	5.602398\\
7809	5.662313\\
7810	6.562668\\
7811	4.682451\\
7812	4.682451\\
7813	4.682451\\
7814	4.682451\\
7815	4.682451\\
7816	4.682451\\
7817	5.56995\\
7818	6.135676\\
7819	6.781225\\
7820	7.067542\\
7821	5.882518\\
7822	4.682451\\
7823	4.68247\\
7824	4.682467\\
7825	4.682451\\
7826	3.175318\\
7827	2.735399\\
7828	2.505574\\
7829	2.529763\\
7830	2.735399\\
7831	2.847323\\
7832	3.156519\\
7833	3.691303\\
7834	4.207362\\
7835	4.306203\\
7836	4.682451\\
7837	4.682451\\
7838	4.602561\\
7839	4.312726\\
7840	4.68199\\
7841	4.682451\\
7842	4.682451\\
7843	4.682451\\
7844	4.886174\\
7845	4.682451\\
7846	4.682451\\
7847	4.682451\\
7848	4.68245\\
7849	3.156519\\
7850	2.735399\\
7851	2.392333\\
7852	2.283846\\
7853	2.735399\\
7854	3.175334\\
7855	4.682451\\
7856	7.236117\\
7857	7.438305\\
7858	7.417223\\
7859	7.236117\\
7860	7.236117\\
7861	7.236116\\
7862	6.680025\\
7863	7.236116\\
7864	7.236118\\
7865	7.568076\\
7866	8.860432\\
7867	8.860436\\
7868	8.43783\\
7869	7.236117\\
7870	6.135694\\
7871	6.679323\\
7872	6.135629\\
7873	4.682451\\
7874	4.682451\\
7875	4.199333\\
7876	3.809341\\
7877	4.027153\\
7878	4.682451\\
7879	6.511812\\
7880	8.786261\\
7881	8.860435\\
7882	8.314517\\
7883	7.786328\\
7884	7.650159\\
7885	7.236117\\
7886	7.236117\\
7887	7.404719\\
7888	7.331062\\
7889	7.793549\\
7890	9.1161\\
7891	8.860436\\
7892	8.737293\\
7893	7.417215\\
7894	6.67901\\
7895	7.065224\\
7896	6.141017\\
7897	4.682451\\
7898	4.682451\\
7899	4.199494\\
7900	3.998126\\
7901	4.456521\\
7902	4.682451\\
7903	6.966073\\
7904	8.852928\\
7905	9.24291\\
7906	9.795925\\
7907	10.799746\\
7908	15.287525\\
7909	15.287526\\
7910	16.646765\\
7911	16.646767\\
7912	18.516111\\
7913	18.516111\\
7914	18.516111\\
7915	18.516111\\
7916	10.546259\\
7917	8.852897\\
7918	7.697788\\
7919	7.581007\\
7920	7.236117\\
7921	6.060048\\
7922	4.682452\\
7923	4.682451\\
7924	5.602397\\
7925	5.602397\\
7926	6.029401\\
7927	8.649955\\
7928	11.413324\\
7929	15.039368\\
7930	10.419488\\
7931	9.944291\\
7932	11.494397\\
7933	10.132934\\
7934	10.534644\\
7935	10.161089\\
7936	10.317861\\
7937	10.91335\\
7938	11.047628\\
7939	10.714217\\
7940	10.648983\\
7941	9.116115\\
7942	7.409207\\
7943	7.463325\\
7944	7.236117\\
7945	4.682452\\
7946	4.682451\\
7947	4.199427\\
7948	3.48284\\
7949	3.685409\\
7950	4.682451\\
7951	6.146253\\
7952	7.236117\\
7953	7.236117\\
7954	7.236117\\
7955	7.236117\\
7956	7.362971\\
7957	7.236117\\
7958	7.236117\\
7959	7.236116\\
7960	7.236116\\
7961	7.41722\\
7962	7.236117\\
7963	7.236117\\
7964	7.236117\\
7965	6.485118\\
7966	5.138038\\
7967	5.974247\\
7968	5.883141\\
7969	4.682451\\
7970	4.682451\\
7971	3.635735\\
7972	3.085092\\
7973	3.085074\\
7974	3.156516\\
7975	4.413925\\
7976	4.682451\\
7977	4.682452\\
7978	5.285436\\
7979	4.87016\\
7980	4.682452\\
7981	4.682452\\
7982	4.682452\\
7983	4.682452\\
7984	5.250181\\
7985	5.882517\\
7986	6.909099\\
7987	6.965513\\
7988	5.710634\\
7989	4.921968\\
7990	4.682455\\
7991	5.568472\\
7992	6.13563\\
7993	4.999461\\
7994	4.682451\\
7995	4.682451\\
7996	4.204267\\
7997	3.969439\\
7998	4.010153\\
7999	4.68245\\
8000	4.682451\\
8001	4.682451\\
};
\addplot [color=mycolor1,solid,line width=1.0pt,forget plot]
  table[row sep=crcr]{%
8001	4.682451\\
8002	5.46217\\
8003	7.815303\\
8004	8.175164\\
8005	8.547418\\
8006	7.774924\\
8007	7.539024\\
8008	7.939906\\
8009	10.027498\\
8010	10.036062\\
8011	10.036064\\
8012	10.49322\\
8013	10.036105\\
8014	8.855394\\
8015	8.85542\\
8016	8.253683\\
8017	7.140098\\
8018	6.213373\\
8019	5.669452\\
8020	5.669452\\
8021	5.669452\\
8022	5.669453\\
8023	8.253683\\
8024	10.734413\\
8025	16.606313\\
8026	19.668687\\
8027	19.668688\\
8028	24.801001\\
8029	24.801001\\
8030	24.801\\
8031	24.801\\
8032	24.801\\
8033	24.801\\
8034	25.865132\\
8035	24.801001\\
8036	19.668687\\
8037	19.668687\\
8038	10.300796\\
8039	10.068739\\
8040	9.300552\\
8041	8.958876\\
8042	7.505992\\
8043	7.322728\\
8044	6.885416\\
8045	4.738496\\
8046	4.738498\\
8047	7.506\\
8048	9.987296\\
8049	10.584976\\
8050	11.43972\\
8051	11.552561\\
8052	11.132064\\
8053	11.411439\\
8054	9.39249\\
8055	10.187088\\
8056	10.795025\\
8057	11.302152\\
8058	11.696954\\
8059	11.199982\\
8060	10.094854\\
8061	8.966486\\
8062	7.322726\\
8063	7.322725\\
8064	7.322726\\
8065	6.464895\\
8066	5.197337\\
8067	4.738495\\
8068	4.738495\\
8069	4.738495\\
8070	4.738496\\
8071	7.322725\\
8072	9.476894\\
8073	10.35519\\
8074	10.392513\\
8075	9.989418\\
8076	10.242377\\
8077	10.281719\\
8078	11.120219\\
8079	11.367265\\
8080	16.846011\\
8081	18.120207\\
8082	18.120207\\
8083	15.470506\\
8084	11.42756\\
8085	10.677363\\
8086	9.041058\\
8087	8.966486\\
8088	8.709244\\
8089	7.879523\\
8090	7.322725\\
8091	6.422817\\
8092	5.868171\\
8093	5.952926\\
8094	6.831351\\
8095	8.966517\\
8096	18.73773\\
8097	23.870043\\
8098	23.870043\\
8099	23.870043\\
8100	24.934175\\
8101	24.934175\\
8102	24.934175\\
8103	23.870043\\
8104	23.870043\\
8105	18.737718\\
8106	23.870043\\
8107	22.257683\\
8108	18.73773\\
8109	18.120202\\
8110	9.983576\\
8111	10.257344\\
8112	10.207576\\
8113	7.434216\\
8114	7.322725\\
8115	6.29112\\
8116	5.836769\\
8117	5.67897\\
8118	6.759711\\
8119	8.966486\\
8120	14.288415\\
8121	11.485833\\
8122	16.84601\\
8123	16.846013\\
8124	16.846011\\
8125	14.288415\\
8126	11.785243\\
8127	10.959498\\
8128	10.785211\\
8129	11.553896\\
8130	10.734378\\
8131	9.469548\\
8132	9.225227\\
8133	7.879513\\
8134	7.895678\\
8135	8.36511\\
8136	8.175556\\
8137	7.322726\\
8138	6.209126\\
8139	4.738495\\
8140	4.738496\\
8141	4.738495\\
8142	4.738495\\
8143	4.738495\\
8144	5.82893\\
8145	7.152113\\
8146	7.412809\\
8147	7.690344\\
8148	7.506492\\
8149	7.323447\\
8150	7.322729\\
8151	7.322725\\
8152	7.873324\\
8153	8.958703\\
8154	9.231187\\
8155	9.402561\\
8156	9.390468\\
8157	8.966485\\
8158	7.810187\\
8159	7.879524\\
8160	10.719954\\
8161	9.261512\\
8162	8.445541\\
8163	6.981146\\
8164	6.491967\\
8165	6.491966\\
8166	6.491966\\
8167	6.491967\\
8168	6.491967\\
8169	6.491967\\
8170	7.4384\\
8171	8.487758\\
8172	9.076195\\
8173	9.076197\\
8174	6.209129\\
8175	5.952903\\
8176	5.796414\\
8177	6.241108\\
8178	7.290824\\
8179	7.303133\\
8180	7.322726\\
8181	7.322728\\
8182	7.222828\\
8183	7.322726\\
8184	5.996087\\
8185	4.738496\\
8186	4.738495\\
8187	3.663831\\
8188	3.097117\\
8189	3.122032\\
8190	4.605949\\
8191	6.03884\\
8192	7.798237\\
8193	9.225226\\
8194	9.149696\\
8195	8.966482\\
8196	9.225226\\
8197	9.917076\\
8198	10.260553\\
8199	11.262191\\
8200	14.288415\\
8201	16.846011\\
8202	16.84601\\
8203	18.737729\\
8204	15.675338\\
8205	9.225323\\
8206	7.683098\\
8207	7.954299\\
8208	8.348275\\
8209	7.054412\\
8210	4.925853\\
8211	4.738496\\
8212	4.738495\\
8213	4.738496\\
8214	5.487786\\
8215	8.311147\\
8216	10.880919\\
8217	18.120206\\
8218	11.054057\\
8219	11.500011\\
8220	10.108026\\
8221	10.448424\\
8222	10.945523\\
8223	10.518198\\
8224	10.738354\\
8225	11.019485\\
8226	11.415479\\
8227	10.232345\\
8228	9.411179\\
8229	7.667754\\
8230	6.761594\\
8231	6.384058\\
8232	6.209125\\
8233	4.738495\\
8234	4.738495\\
8235	3.367334\\
8236	2.924284\\
8237	2.781129\\
8238	4.200863\\
8239	5.423951\\
8240	7.322726\\
8241	8.966486\\
8242	8.966486\\
8243	8.492277\\
8244	8.142721\\
8245	7.324468\\
8246	8.966482\\
8247	8.966486\\
8248	9.76845\\
8249	14.288415\\
8250	18.120207\\
8251	11.369485\\
8252	9.225226\\
8253	8.966486\\
8254	7.322726\\
8255	7.322726\\
8256	7.322726\\
8257	4.811915\\
8258	4.738497\\
8259	4.364027\\
8260	3.705065\\
8261	3.979011\\
8262	4.738497\\
8263	5.684331\\
8264	8.157763\\
8265	8.966486\\
8266	8.966487\\
8267	8.966487\\
8268	9.225226\\
8269	8.967809\\
8270	8.966486\\
8271	8.966513\\
8272	9.225219\\
8273	9.227158\\
8274	9.709156\\
8275	9.225227\\
8276	8.958871\\
8277	8.03104\\
8278	7.152145\\
8279	6.950533\\
8280	7.092446\\
8281	4.738496\\
8282	4.737379\\
8283	3.555764\\
8284	3.122031\\
8285	3.194289\\
8286	4.738496\\
8287	5.900659\\
8288	7.322726\\
8289	7.322726\\
8290	7.322726\\
8291	7.953433\\
8292	8.86831\\
8293	8.519507\\
8294	8.921419\\
8295	8.966482\\
8296	8.966484\\
8297	9.342149\\
8298	9.754324\\
8299	10.732048\\
8300	9.191086\\
8301	7.79835\\
8302	7.322726\\
8303	7.506986\\
8304	7.505996\\
8305	7.322726\\
8306	5.507399\\
8307	4.738496\\
8308	4.738497\\
8309	4.738497\\
8310	4.738498\\
8311	4.738496\\
8312	5.660601\\
8313	7.322726\\
8314	7.614207\\
8315	8.369858\\
8316	8.193879\\
8317	8.505114\\
8318	8.195668\\
8319	7.830922\\
8320	8.716046\\
8321	8.966486\\
8322	8.967111\\
8323	9.341201\\
8324	8.869173\\
8325	8.028347\\
8326	7.322726\\
8327	7.322726\\
8328	7.322726\\
8329	7.205452\\
8330	5.458946\\
8331	4.738498\\
8332	4.7385\\
8333	4.738499\\
8334	4.738498\\
8335	4.738499\\
8336	4.7385\\
8337	4.738498\\
8338	5.057407\\
8339	5.528841\\
8340	6.066567\\
8341	5.952932\\
8342	4.738496\\
8343	4.738496\\
8344	4.738496\\
8345	4.738496\\
8346	6.209125\\
8347	6.034498\\
8348	4.759382\\
8349	4.738517\\
8350	4.738501\\
8351	4.738501\\
8352	4.738497\\
8353	4.144504\\
8354	2.8712\\
8355	1.916247\\
8356	0.458607\\
8357	0.083471\\
8358	0.618532\\
8359	2.768142\\
8360	3.555489\\
8361	4.738497\\
8362	5.17598\\
8363	7.101779\\
8364	7.322726\\
8365	7.322726\\
8366	7.322726\\
8367	7.322725\\
8368	7.322726\\
8369	7.505995\\
8370	8.940979\\
8371	8.419047\\
8372	7.322726\\
8373	5.392392\\
8374	4.738496\\
8375	4.845066\\
8376	5.588221\\
8377	4.738496\\
8378	4.731872\\
8379	3.249229\\
8380	2.916537\\
8381	3.007528\\
8382	4.737418\\
8383	8.253663\\
8384	10.139513\\
8385	11.311742\\
8386	11.744745\\
8387	9.994369\\
8388	10.103383\\
8389	9.343091\\
8390	9.47749\\
8391	9.839739\\
8392	10.567195\\
8393	11.141831\\
8394	18.73773\\
8395	18.73773\\
8396	11.084041\\
8397	10.601612\\
8398	8.953963\\
8399	8.958354\\
8400	7.797744\\
8401	5.874179\\
8402	4.738496\\
8403	4.666098\\
8404	3.300517\\
8405	3.122071\\
8406	3.996915\\
8407	4.738496\\
8408	7.322726\\
8409	8.958887\\
8410	9.435819\\
8411	10.184827\\
8412	11.1037\\
8413	10.796081\\
8414	10.711927\\
8415	10.965146\\
8416	11.345617\\
8417	18.120208\\
8418	18.120202\\
8419	16.846011\\
8420	18.033063\\
8421	9.225227\\
8422	7.40477\\
8423	7.322726\\
8424	7.322726\\
8425	5.199532\\
8426	4.738496\\
8427	4.738495\\
8428	4.219194\\
8429	6.491947\\
8430	6.491966\\
8431	9.076196\\
8432	8.064241\\
8433	9.225227\\
8434	7.322726\\
8435	8.369143\\
8436	8.958879\\
8437	7.322726\\
8438	8.051742\\
8439	7.588069\\
8440	7.453612\\
8441	7.505999\\
8442	7.341738\\
8443	8.326174\\
8444	7.322726\\
8445	5.995077\\
8446	4.738496\\
8447	4.738497\\
8448	4.738496\\
8449	4.732199\\
8450	2.907682\\
8451	1.920786\\
8452	1.029048\\
8453	1.00358\\
8454	2.768142\\
8455	4.738496\\
8456	5.952745\\
8457	7.322726\\
8458	7.322726\\
8459	7.522209\\
8460	8.958876\\
8461	8.966517\\
8462	7.979556\\
8463	7.322726\\
8464	7.322726\\
8465	7.322726\\
8466	7.343582\\
8467	7.322822\\
8468	6.900213\\
8469	5.152626\\
8470	4.738496\\
8471	4.738496\\
8472	4.738497\\
8473	2.768545\\
8474	1.25491\\
8475	3.4e-05\\
8476	5e-05\\
8477	1.2e-05\\
8478	1.1e-05\\
8479	2.3e-05\\
8480	0.00017\\
8481	2.563427\\
8482	2.327059\\
8483	2.768139\\
8484	2.768139\\
8485	2.767191\\
8486	2.768143\\
8487	2.768139\\
8488	2.768141\\
8489	3.056915\\
8490	3.194299\\
8491	4.093708\\
8492	4.422341\\
8493	3.290011\\
8494	3.122022\\
8495	3.357547\\
8496	4.249488\\
8497	2.769495\\
8498	1.847536\\
8499	0.291084\\
8500	9e-06\\
8501	2e-06\\
8502	4e-06\\
8503	3e-05\\
8504	2.4e-05\\
8505	1.042292\\
8506	2.614923\\
8507	2.768141\\
8508	3.122004\\
8509	3.194299\\
8510	3.213205\\
8511	2.768139\\
8512	2.768141\\
8513	2.76814\\
8514	2.768139\\
8515	2.768139\\
8516	2.768139\\
8517	2.768139\\
8518	2.089764\\
8519	2.766914\\
8520	2.504916\\
8521	0.000143\\
8522	2e-06\\
8523	0\\
8524	1.3e-05\\
8525	3e-06\\
8526	1e-06\\
8527	0.546432\\
8528	2.768141\\
8529	4.738495\\
8530	4.7385\\
8531	4.738497\\
8532	4.738499\\
8533	4.738495\\
8534	4.738495\\
8535	4.204226\\
8536	3.680212\\
8537	3.837516\\
8538	3.723306\\
8539	4.307782\\
8540	2.81222\\
8541	2.747057\\
8542	2.157327\\
8543	2.768139\\
8544	2.767985\\
8545	1e-06\\
8546	8e-06\\
8547	5e-06\\
8548	0\\
8549	0\\
8550	0\\
8551	1.831974\\
8552	3.940077\\
8553	5.716072\\
8554	5.910432\\
8555	5.910433\\
8556	5.910434\\
8557	5.910434\\
8558	5.910434\\
8559	5.9093\\
8560	5.910433\\
8561	5.909065\\
8562	5.910434\\
8563	4.738495\\
8564	4.738495\\
8565	3.194294\\
8566	2.768124\\
8567	2.786626\\
8568	2.768145\\
8569	0.008868\\
8570	1e-06\\
8571	0\\
8572	0\\
8573	0\\
8574	0\\
8575	0\\
8576	6.9e-05\\
8577	1.763831\\
8578	2.768139\\
8579	4.125256\\
8580	4.144292\\
8581	4.189022\\
8582	5.66945\\
8583	5.589851\\
8584	5.669188\\
8585	5.244081\\
8586	3.657249\\
8587	3.66308\\
8588	3.231953\\
8589	2.510614\\
8590	1.815348\\
8591	3.231081\\
8592	5.201437\\
8593	4.698879\\
8594	3.231089\\
8595	1.559086\\
8596	0.147215\\
8597	3e-05\\
8598	3e-05\\
8599	0.000407\\
8600	0.000135\\
8601	0.565899\\
8602	1.237894\\
8603	2.414453\\
8604	3.231088\\
8605	3.231089\\
8606	3.231089\\
8607	3.231089\\
8608	3.231092\\
8609	3.560722\\
8610	3.657241\\
8611	4.048794\\
8612	4.987283\\
8613	5.201432\\
8614	5.201438\\
8615	5.201492\\
8616	5.201493\\
8617	5.201438\\
8618	4.044009\\
8619	3.231098\\
8620	3.231087\\
8621	3.231082\\
8622	3.668813\\
8623	5.201439\\
8624	5.20144\\
8625	5.641262\\
8626	6.672104\\
8627	7.2864\\
8628	7.194218\\
8629	6.884305\\
8630	6.730469\\
8631	7.330143\\
8632	7.411723\\
8633	7.696357\\
8634	6.964808\\
8635	6.705537\\
8636	5.646964\\
8637	5.201492\\
8638	5.201438\\
8639	5.201437\\
8640	5.201437\\
8641	4.591984\\
8642	3.231092\\
8643	1.579939\\
8644	0.463073\\
8645	0.322495\\
8646	0.729053\\
8647	1.688744\\
8648	3.048246\\
8649	3.676275\\
8650	5.201438\\
8651	5.201468\\
8652	5.201473\\
8653	5.201477\\
8654	5.201482\\
8655	5.201456\\
8656	5.201457\\
8657	5.201465\\
8658	5.201453\\
8659	5.201454\\
8660	5.201457\\
8661	5.201463\\
8662	5.20147\\
8663	5.201466\\
8664	5.201465\\
8665	4.712496\\
8666	3.328653\\
8667	2.866031\\
8668	1.23791\\
8669	0.753266\\
8670	0.753949\\
8671	1.473314\\
8672	1.727983\\
8673	3.23111\\
8674	3.889616\\
8675	4.511943\\
8676	5.001383\\
8677	5.201438\\
8678	5.201414\\
8679	5.201481\\
8680	5.201463\\
8681	5.68621\\
8682	7.459211\\
8683	7.785712\\
8684	7.615265\\
8685	7.7857\\
8686	7.490724\\
8687	7.289244\\
8688	8.494671\\
8689	7.38107\\
8690	5.910437\\
8691	5.910437\\
8692	4.889169\\
8693	4.366241\\
8694	4.596575\\
8695	5.910436\\
8696	5.910437\\
8697	7.38107\\
8698	6.672065\\
8699	6.672072\\
8700	6.672075\\
8701	6.582059\\
8702	6.671947\\
8703	6.440441\\
8704	6.672074\\
8705	7.287043\\
8706	7.785699\\
8707	8.342479\\
8708	8.02475\\
8709	7.785669\\
8710	7.308166\\
8711	7.785668\\
8712	7.785668\\
8713	6.188912\\
8714	5.201438\\
8715	5.201439\\
8716	4.711394\\
8717	4.560286\\
8718	5.201439\\
8719	5.201438\\
8720	5.662478\\
8721	7.344082\\
8722	7.785668\\
8723	7.785668\\
8724	7.785668\\
8725	7.785668\\
8726	7.785671\\
8727	7.785668\\
8728	7.785671\\
8729	7.904682\\
8730	8.90859\\
8731	8.493398\\
8732	7.785667\\
8733	7.785668\\
8734	6.672085\\
8735	7.785668\\
8736	7.785668\\
8737	5.66249\\
8738	5.20144\\
8739	5.20141\\
8740	4.508931\\
8741	4.149758\\
8742	4.712501\\
8743	5.20144\\
8744	5.201483\\
8745	6.672085\\
8746	7.604462\\
8747	6.672086\\
8748	6.672102\\
8749	6.377036\\
8750	6.672086\\
8751	6.447919\\
8752	6.672009\\
8753	6.971629\\
8754	6.910869\\
8755	6.672098\\
8756	5.201483\\
8757	5.201489\\
8758	5.201437\\
8759	5.201473\\
8760	5.948071\\
};
\end{axis}
\end{tikzpicture}%
    \caption{Predicted reserve prices for the ORDC model}
    \label{fig:ORDC_R1}
\end{figure}

\begin{figure}[H]
    \centering
    \setlength\fheight{0.3\textwidth}
    \setlength\fwidth{0.85\textwidth}
    % This file was created by matlab2tikz.
% Minimal pgfplots version: 1.3
%
%The latest updates can be retrieved from
%  http://www.mathworks.com/matlabcentral/fileexchange/22022-matlab2tikz
%where you can also make suggestions and rate matlab2tikz.
%
\definecolor{mycolor1}{rgb}{0.87059,0.49020,0.00000}%
%
\begin{tikzpicture}

\begin{axis}[%
width=\fwidth,
height=\fheight,
at={(0\fwidth,0\fheight)},
scale only axis,
separate axis lines,
every outer x axis line/.append style={black},
every x tick label/.append style={font=\color{black}},
xmin=0,
xmax=8760,
xlabel={time [hour]},
xtick={0,1000,2000,3000,4000,5000,6000,7000,8000},
xmajorgrids,
every outer y axis line/.append style={black},
every y tick label/.append style={font=\color{black}},
ymin=0,
ymax=70,
ymajorgrids,
title style={font=\bfseries},
title={ORDC - Reserve price [\euro/MWh]}
]
\addplot [color=mycolor1,solid,line width=1.0pt,forget plot]
  table[row sep=crcr]{%
1	2e-06\\
2	2e-06\\
3	6e-06\\
4	1e-06\\
5	0\\
6	0\\
7	0\\
8	2e-06\\
9	2e-06\\
10	2e-06\\
11	5e-06\\
12	4e-06\\
13	0\\
14	0\\
15	0\\
16	0\\
17	0\\
18	0\\
19	1e-06\\
20	1e-06\\
21	1e-06\\
22	1e-06\\
23	0\\
24	2e-06\\
25	1e-06\\
26	0\\
27	1e-06\\
28	0\\
29	1e-06\\
30	0\\
31	2e-06\\
32	1e-06\\
33	2e-06\\
34	3e-06\\
35	9e-06\\
36	2e-06\\
37	2e-06\\
38	2e-06\\
39	2e-06\\
40	2e-06\\
41	1e-06\\
42	2e-06\\
43	1e-06\\
44	4e-06\\
45	3e-06\\
46	1e-06\\
47	2e-06\\
48	6e-06\\
49	3e-06\\
50	0\\
51	4e-06\\
52	0\\
53	0\\
54	2e-06\\
55	1e-06\\
56	8e-06\\
57	3e-06\\
58	9e-06\\
59	3e-06\\
60	3e-06\\
61	3e-06\\
62	2e-06\\
63	4e-06\\
64	6e-06\\
65	3e-06\\
66	4e-06\\
67	4e-06\\
68	2e-06\\
69	3e-06\\
70	5e-06\\
71	7e-06\\
72	5e-06\\
73	9e-06\\
74	3e-06\\
75	1e-06\\
76	2e-06\\
77	0\\
78	6e-06\\
79	1e-06\\
80	4e-06\\
81	3e-06\\
82	1e-05\\
83	0\\
84	4e-06\\
85	9e-06\\
86	4e-06\\
87	1e-05\\
88	0\\
89	3e-06\\
90	1e-06\\
91	3e-06\\
92	2e-06\\
93	5e-06\\
94	2e-06\\
95	2e-06\\
96	4e-06\\
97	4e-06\\
98	4e-06\\
99	0\\
100	3e-06\\
101	5e-06\\
102	3e-06\\
103	5e-06\\
104	3e-06\\
105	2e-06\\
106	4e-06\\
107	0\\
108	0\\
109	0\\
110	1e-05\\
111	8e-06\\
112	0\\
113	4e-06\\
114	3e-06\\
115	1.1e-05\\
116	4e-06\\
117	2e-06\\
118	4e-06\\
119	2e-06\\
120	2e-06\\
121	2e-06\\
122	0\\
123	2e-06\\
124	1e-06\\
125	1e-06\\
126	0\\
127	2e-06\\
128	0\\
129	3e-06\\
130	1e-06\\
131	1.3e-05\\
132	0\\
133	6e-06\\
134	3e-06\\
135	5e-06\\
136	0\\
137	0\\
138	0.011258\\
139	1e-06\\
140	1e-06\\
141	1e-06\\
142	5e-06\\
143	0\\
144	2e-06\\
145	2e-06\\
146	5e-06\\
147	2e-06\\
148	1e-06\\
149	1e-06\\
150	0\\
151	2e-06\\
152	0\\
153	0.224931\\
154	1e-06\\
155	1e-05\\
156	8.2e-05\\
157	7e-06\\
158	0\\
159	2.8e-05\\
160	1.688574\\
161	1.808459\\
162	4.547575\\
163	4.547572\\
164	2.160853\\
165	0.70194\\
166	2.6e-05\\
167	9e-06\\
168	11.017886\\
169	8.19841\\
170	6.073945\\
171	9e-06\\
172	5e-06\\
173	5e-06\\
174	1.4e-05\\
175	2e-06\\
176	1.983045\\
177	4.547572\\
178	3.570045\\
179	2.098037\\
180	1.408328\\
181	3.266787\\
182	6.819754\\
183	6.821725\\
184	6.821726\\
185	6.821726\\
186	7.563612\\
187	8.141963\\
188	6.821726\\
189	7.764469\\
190	7.073953\\
191	7.226651\\
192	5.982189\\
193	0.10759\\
194	3.4e-05\\
195	8.9e-05\\
196	1.4e-05\\
197	3.2e-05\\
198	1e-06\\
199	0.107596\\
200	3.431216\\
201	5.273166\\
202	3.108903\\
203	3.07087\\
204	3.383761\\
205	4.834447\\
206	5.158967\\
207	4.670445\\
208	4.915935\\
209	3.767686\\
210	4.131445\\
211	4.547564\\
212	2.751694\\
213	0.701945\\
214	0.803057\\
215	1.596805\\
216	1.352639\\
217	2e-06\\
218	0\\
219	7e-06\\
220	1e-06\\
221	2e-06\\
222	1.1e-05\\
223	2.303371\\
224	5.687139\\
225	6.684602\\
226	4.987116\\
227	4.547572\\
228	4.956\\
229	3.709624\\
230	4.547548\\
231	4.547576\\
232	4.666527\\
233	5.061395\\
234	6.446448\\
235	5.8579\\
236	4.547573\\
237	3.32127\\
238	2.247857\\
239	2.96533\\
240	3.238165\\
241	1.4e-05\\
242	2e-06\\
243	2e-06\\
244	1.5e-05\\
245	1e-06\\
246	3e-06\\
247	1.4e-05\\
248	0\\
249	8.2e-05\\
250	0.70175\\
251	0.922854\\
252	1.022764\\
253	1.20278\\
254	1.331871\\
255	0.516584\\
256	0.709078\\
257	1.352633\\
258	3.978581\\
259	3.01312\\
260	1.829029\\
261	2.7e-05\\
262	0.48637\\
263	1.688592\\
264	0.13505\\
265	0.591038\\
266	0.107596\\
267	7e-06\\
268	1.1e-05\\
269	2.9e-05\\
270	1.5e-05\\
271	2.9e-05\\
272	1e-06\\
273	0.000577\\
274	0.1076\\
275	0.107599\\
276	2e-05\\
277	8e-06\\
278	3e-06\\
279	1e-06\\
280	2e-06\\
281	0\\
282	0.000108\\
283	1.457164\\
284	0.224879\\
285	2e-06\\
286	2.6e-05\\
287	6e-06\\
288	0\\
289	7e-06\\
290	2e-06\\
291	4e-06\\
292	1e-06\\
293	1e-06\\
294	5e-06\\
295	1.2e-05\\
296	8e-06\\
297	3.3e-05\\
298	1.6e-05\\
299	1.055009\\
300	1.466137\\
301	0.520482\\
302	0.724583\\
303	0.343273\\
304	0.885351\\
305	1.48607\\
306	2.517327\\
307	3.194939\\
308	0.637698\\
309	1e-06\\
310	9e-06\\
311	2e-06\\
312	3e-06\\
313	3e-06\\
314	1e-06\\
315	3e-06\\
316	2e-06\\
317	1e-06\\
318	3e-06\\
319	5e-06\\
320	1.997841\\
321	2.383914\\
322	3.194939\\
323	3.194939\\
324	3.194939\\
325	3.194938\\
326	3.018529\\
327	3.194939\\
328	3.195245\\
329	3.646261\\
330	4.872207\\
331	3.708764\\
332	3.226687\\
333	3.194939\\
334	0.550494\\
335	1.042986\\
336	0.335973\\
337	0\\
338	2e-06\\
339	2e-06\\
340	3e-06\\
341	3e-06\\
342	2e-06\\
343	0\\
344	0.894438\\
345	0.988691\\
346	0.891175\\
347	1.575733\\
348	1.773645\\
349	1.031291\\
350	1.150245\\
351	0.477475\\
352	0.591114\\
353	0.891077\\
354	2.014155\\
355	1.70528\\
356	1.173415\\
357	5e-06\\
358	0\\
359	5e-06\\
360	3e-06\\
361	4e-06\\
362	2e-06\\
363	1e-06\\
364	1e-06\\
365	1e-06\\
366	5e-06\\
367	1e-06\\
368	7e-06\\
369	5e-06\\
370	1e-06\\
371	3e-06\\
372	3e-06\\
373	0.750127\\
374	2.892947\\
375	2.342836\\
376	1.693536\\
377	1.17774\\
378	3.194939\\
379	1.89341\\
380	0.894432\\
381	1e-06\\
382	1.3e-05\\
383	0\\
384	9e-06\\
385	2e-06\\
386	4e-06\\
387	3e-06\\
388	1e-06\\
389	2e-06\\
390	1e-06\\
391	4e-06\\
392	1e-06\\
393	1e-06\\
394	7e-06\\
395	1e-06\\
396	3.7e-05\\
397	1.7e-05\\
398	4e-06\\
399	2e-06\\
400	5.6e-05\\
401	0.262853\\
402	2.108915\\
403	0.667017\\
404	1e-06\\
405	5.5e-05\\
406	2e-06\\
407	4e-06\\
408	3e-06\\
409	2e-06\\
410	4e-06\\
411	3e-06\\
412	1e-06\\
413	1e-06\\
414	1e-06\\
415	1e-06\\
416	1e-06\\
417	2e-06\\
418	2e-06\\
419	1e-06\\
420	3e-06\\
421	5e-06\\
422	7e-06\\
423	6e-06\\
424	2e-06\\
425	4e-06\\
426	0\\
427	2e-06\\
428	2e-06\\
429	1e-06\\
430	0\\
431	4e-06\\
432	5e-06\\
433	3e-06\\
434	1e-06\\
435	0\\
436	3e-06\\
437	3e-06\\
438	3e-06\\
439	4e-06\\
440	3e-06\\
441	4e-06\\
442	3e-06\\
443	2e-06\\
444	1e-06\\
445	0\\
446	1e-06\\
447	0\\
448	0\\
449	6e-06\\
450	8e-06\\
451	0\\
452	1e-05\\
453	0\\
454	0\\
455	1e-06\\
456	6e-06\\
457	4e-06\\
458	3e-06\\
459	5e-06\\
460	1e-06\\
461	8e-06\\
462	2e-06\\
463	3e-05\\
464	2.938922\\
465	2.513503\\
466	3.195058\\
467	3.603385\\
468	3.686806\\
469	3.19494\\
470	3.708764\\
471	3.686806\\
472	3.708764\\
473	3.765364\\
474	4.670758\\
475	3.719842\\
476	3.084152\\
477	3.190196\\
478	2e-06\\
479	4.3e-05\\
480	0\\
481	7e-06\\
482	4e-06\\
483	0\\
484	2e-06\\
485	5e-06\\
486	4e-06\\
487	4e-06\\
488	3.194939\\
489	2.568523\\
490	0.894438\\
491	0.894425\\
492	2.314475\\
493	1.594926\\
494	2.761125\\
495	2.902721\\
496	3.194824\\
497	3.194939\\
498	4.338004\\
499	3.19494\\
500	3.194939\\
501	1.865849\\
502	8e-06\\
503	5e-06\\
504	2.6e-05\\
505	4e-06\\
506	2e-06\\
507	5e-06\\
508	1e-06\\
509	1e-06\\
510	3e-06\\
511	1e-06\\
512	0.335863\\
513	1e-06\\
514	4e-05\\
515	1.9e-05\\
516	0\\
517	1e-06\\
518	3e-06\\
519	2.5e-05\\
520	0.712887\\
521	1.606437\\
522	3.195044\\
523	3.194939\\
524	3.021405\\
525	1.293913\\
526	4e-06\\
527	0.257127\\
528	1e-06\\
529	7e-06\\
530	2e-06\\
531	3e-06\\
532	3e-06\\
533	8e-06\\
534	4e-06\\
535	3e-06\\
536	2.443548\\
537	1.672179\\
538	1.846317\\
539	1.624857\\
540	1.296079\\
541	0.756864\\
542	1.573246\\
543	1.074152\\
544	1.271382\\
545	1.233731\\
546	3.194929\\
547	1.395702\\
548	0.057361\\
549	2.8e-05\\
550	1.4e-05\\
551	4e-06\\
552	0\\
553	2e-06\\
554	1e-06\\
555	3e-06\\
556	2e-06\\
557	3e-06\\
558	3e-06\\
559	2.6e-05\\
560	2.665624\\
561	1.484651\\
562	0.894425\\
563	0.063348\\
564	1e-06\\
565	1e-06\\
566	4.3e-05\\
567	6e-06\\
568	5e-06\\
569	0.015638\\
570	1.912096\\
571	0.894436\\
572	0.006999\\
573	1.3e-05\\
574	6e-06\\
575	0\\
576	2e-05\\
577	1.1e-05\\
578	3e-06\\
579	1e-06\\
580	5e-06\\
581	6e-06\\
582	2e-06\\
583	2e-06\\
584	5e-06\\
585	3e-06\\
586	5e-06\\
587	0\\
588	0\\
589	0\\
590	1e-06\\
591	4e-06\\
592	3e-06\\
593	3e-06\\
594	4e-06\\
595	6e-06\\
596	3e-06\\
597	1e-06\\
598	3e-06\\
599	3e-06\\
600	7e-06\\
601	2e-06\\
602	0\\
603	0\\
604	0\\
605	0\\
606	4e-06\\
607	1e-06\\
608	3e-06\\
609	1e-06\\
610	1e-06\\
611	2e-06\\
612	2e-06\\
613	2e-06\\
614	2e-06\\
615	1e-06\\
616	1e-06\\
617	1e-06\\
618	1e-06\\
619	1e-06\\
620	1e-06\\
621	1e-06\\
622	2e-06\\
623	1e-06\\
624	1e-06\\
625	4e-06\\
626	1e-06\\
627	0\\
628	2e-06\\
629	2e-06\\
630	1e-06\\
631	3e-06\\
632	0\\
633	1e-06\\
634	2e-06\\
635	4e-06\\
636	1e-06\\
637	1e-06\\
638	6e-06\\
639	7e-06\\
640	1e-06\\
641	0.000251\\
642	2e-06\\
643	0.169579\\
644	7e-06\\
645	1.1e-05\\
646	1e-06\\
647	5e-06\\
648	4e-06\\
649	2e-06\\
650	3e-06\\
651	1e-06\\
652	1e-06\\
653	1e-06\\
654	2e-06\\
655	1e-06\\
656	7e-06\\
657	1.2e-05\\
658	2e-06\\
659	0\\
660	3e-06\\
661	2e-06\\
662	0\\
663	6e-06\\
664	6e-06\\
665	2.6e-05\\
666	2e-06\\
667	3e-06\\
668	3e-06\\
669	0\\
670	1e-06\\
671	2e-05\\
672	2e-06\\
673	1e-06\\
674	6e-06\\
675	2e-06\\
676	1e-06\\
677	3e-06\\
678	1e-06\\
679	1e-06\\
680	1e-06\\
681	1e-05\\
682	1e-06\\
683	0\\
684	0\\
685	5e-06\\
686	0\\
687	8e-06\\
688	1e-06\\
689	1e-05\\
690	5e-06\\
691	0.894434\\
692	7e-06\\
693	9e-06\\
694	4e-06\\
695	3e-06\\
696	3e-06\\
697	6e-06\\
698	2e-06\\
699	3e-06\\
700	5e-06\\
701	7e-06\\
702	2e-06\\
703	2e-06\\
704	2.503988\\
705	0.393345\\
706	1.897277\\
707	0.396437\\
708	0.89425\\
709	5.4e-05\\
710	0.078146\\
711	0.262768\\
712	0.821091\\
713	2.370327\\
714	3.275833\\
715	3.194939\\
716	2.05159\\
717	1.876045\\
718	1e-06\\
719	0.772518\\
720	0.447602\\
721	8e-06\\
722	0\\
723	2e-06\\
724	2e-06\\
725	3e-06\\
726	3e-06\\
727	0.000111\\
728	3.194939\\
729	0.583524\\
730	1e-06\\
731	4.9e-05\\
732	0\\
733	7e-06\\
734	1e-06\\
735	3e-06\\
736	0\\
737	5e-06\\
738	3e-06\\
739	7.3e-05\\
740	0\\
741	0\\
742	0\\
743	2e-06\\
744	3e-06\\
745	1e-06\\
746	9e-06\\
747	0\\
748	2e-06\\
749	0\\
750	0\\
751	0\\
752	0\\
753	1e-06\\
754	3e-06\\
755	0\\
756	1e-06\\
757	5e-06\\
758	3e-06\\
759	1e-06\\
760	1e-06\\
761	1e-06\\
762	0\\
763	0\\
764	2e-06\\
765	1e-06\\
766	1e-06\\
767	1e-06\\
768	4e-06\\
769	1e-06\\
770	0\\
771	0\\
772	0\\
773	3e-06\\
774	1e-06\\
775	0\\
776	1e-06\\
777	0\\
778	0\\
779	5e-06\\
780	8e-06\\
781	0\\
782	5e-06\\
783	0\\
784	6e-06\\
785	1e-06\\
786	2e-06\\
787	0\\
788	0\\
789	6e-06\\
790	1e-06\\
791	5e-06\\
792	1e-05\\
793	0\\
794	1e-06\\
795	1e-06\\
796	4e-06\\
797	1e-06\\
798	1e-06\\
799	1.8e-05\\
800	2.910001\\
801	1.643808\\
802	0.381114\\
803	1e-05\\
804	6e-06\\
805	0\\
806	3.3e-05\\
807	1e-05\\
808	0.306032\\
809	1.740324\\
810	3.281994\\
811	4.110747\\
812	2.920389\\
813	2.413893\\
814	0.00829\\
815	0.530802\\
816	0.022366\\
817	6e-06\\
818	0\\
819	0\\
820	3e-06\\
821	3e-06\\
822	1e-06\\
823	1.8e-05\\
824	2.516684\\
825	0.746623\\
826	0.700138\\
827	0.418578\\
828	0.658305\\
829	3.9e-05\\
830	1.6e-05\\
831	1.8e-05\\
832	2.9e-05\\
833	0.362787\\
834	2.151352\\
835	2.910001\\
836	1.315299\\
837	0.632901\\
838	8e-06\\
839	3.2e-05\\
840	6e-06\\
841	1e-06\\
842	1e-06\\
843	0\\
844	0\\
845	0\\
846	9e-06\\
847	8e-06\\
848	1.3e-05\\
849	0.775314\\
850	0.306029\\
851	0.816052\\
852	0.749947\\
853	2e-05\\
854	9.2e-05\\
855	1.2e-05\\
856	4e-06\\
857	0.306055\\
858	2.893872\\
859	2.62514\\
860	0.787793\\
861	5e-06\\
862	1e-06\\
863	0\\
864	3e-06\\
865	1e-06\\
866	0\\
867	1e-06\\
868	0\\
869	1e-06\\
870	1e-06\\
871	0\\
872	1.7e-05\\
873	1e-06\\
874	2.7e-05\\
875	7e-06\\
876	3.1e-05\\
877	3.9e-05\\
878	0.456056\\
879	0.567914\\
880	0.538624\\
881	1.051859\\
882	2.909999\\
883	1.955781\\
884	0.635856\\
885	1e-06\\
886	2e-06\\
887	6e-06\\
888	0\\
889	1e-06\\
890	1e-06\\
891	9e-06\\
892	0\\
893	0\\
894	0\\
895	2e-06\\
896	0\\
897	2.2e-05\\
898	6e-06\\
899	0.30604\\
900	2.512183\\
901	1.301385\\
902	1.540771\\
903	1.136297\\
904	0.486668\\
905	9e-06\\
906	3e-06\\
907	0.515569\\
908	3e-06\\
909	1e-06\\
910	4e-06\\
911	2e-06\\
912	0\\
913	1e-06\\
914	1e-06\\
915	1e-06\\
916	2e-06\\
917	7e-06\\
918	2e-06\\
919	2e-06\\
920	0\\
921	0\\
922	1e-06\\
923	1e-06\\
924	9e-06\\
925	2e-06\\
926	6e-06\\
927	1e-06\\
928	1e-06\\
929	1e-06\\
930	1e-06\\
931	1e-06\\
932	1e-06\\
933	9e-06\\
934	6e-06\\
935	0\\
936	1e-06\\
937	0\\
938	6e-06\\
939	0\\
940	1e-06\\
941	1e-06\\
942	1e-06\\
943	2e-06\\
944	1e-06\\
945	0\\
946	0\\
947	4e-06\\
948	0\\
949	7e-06\\
950	1e-06\\
951	0\\
952	0\\
953	2e-06\\
954	1e-06\\
955	1e-06\\
956	1.2e-05\\
957	1e-06\\
958	1e-06\\
959	1e-06\\
960	1e-06\\
961	1e-06\\
962	1e-06\\
963	0\\
964	0\\
965	6e-06\\
966	1e-06\\
967	3e-06\\
968	3.248924\\
969	3.282019\\
970	3.282043\\
971	2.910001\\
972	3.086409\\
973	2.91\\
974	3.357998\\
975	3.358001\\
976	3.615126\\
977	3.98693\\
978	4.981335\\
979	4.981336\\
980	4.219446\\
981	2.910005\\
982	1.720482\\
983	2.046779\\
984	1.35606\\
985	2e-06\\
986	8e-06\\
987	1e-06\\
988	1e-06\\
989	0\\
990	3e-06\\
991	1.2e-05\\
992	0.595757\\
993	2.2e-05\\
994	5.9e-05\\
995	2e-06\\
996	4.3e-05\\
997	1e-05\\
998	2e-06\\
999	3.8e-05\\
1000	1.1e-05\\
1001	0.306596\\
1002	2.629336\\
1003	2.910001\\
1004	2.195944\\
1005	0.783765\\
1006	6e-05\\
1007	0.305987\\
1008	0.005497\\
1009	4e-06\\
1010	1e-06\\
1011	1e-06\\
1012	3e-06\\
1013	5e-06\\
1014	1e-06\\
1015	1.3e-05\\
1016	3.053448\\
1017	1.723285\\
1018	0.011443\\
1019	3e-06\\
1020	0\\
1021	1e-06\\
1022	0\\
1023	3e-06\\
1024	2e-05\\
1025	2e-06\\
1026	1.291911\\
1027	1.718756\\
1028	0.068331\\
1029	5e-05\\
1030	1e-06\\
1031	7e-06\\
1032	3e-06\\
1033	1e-06\\
1034	0\\
1035	0\\
1036	0\\
1037	0\\
1038	1e-06\\
1039	0\\
1040	4e-06\\
1041	0.002862\\
1042	2.901425\\
1043	2.910001\\
1044	3.378017\\
1045	3.358001\\
1046	3.630718\\
1047	3.358002\\
1048	3.358001\\
1049	3.536167\\
1050	4.158963\\
1051	3.664101\\
1052	3.358001\\
1053	2.3772\\
1054	0.765745\\
1055	1.109988\\
1056	0.509128\\
1057	2e-06\\
1058	0\\
1059	0\\
1060	0\\
1061	8e-06\\
1062	1e-06\\
1063	1.798535\\
1064	2.379916\\
1065	0.823262\\
1066	0.308136\\
1067	3.7e-05\\
1068	0.023844\\
1069	0.305983\\
1070	1.083575\\
1071	1.220152\\
1072	0.819437\\
1073	0.809976\\
1074	2.049086\\
1075	0.291721\\
1076	1e-05\\
1077	5e-06\\
1078	5e-06\\
1079	0\\
1080	0\\
1081	1e-06\\
1082	0\\
1083	0\\
1084	4e-06\\
1085	2e-06\\
1086	2e-06\\
1087	0\\
1088	0\\
1089	0\\
1090	9e-06\\
1091	1e-06\\
1092	1e-06\\
1093	0\\
1094	0\\
1095	7e-06\\
1096	0\\
1097	8e-06\\
1098	0\\
1099	1e-06\\
1100	0\\
1101	0\\
1102	0\\
1103	1e-06\\
1104	1e-06\\
1105	0\\
1106	0\\
1107	0\\
1108	2e-06\\
1109	3e-06\\
1110	2e-06\\
1111	4e-06\\
1112	0\\
1113	0\\
1114	0\\
1115	0\\
1116	7e-06\\
1117	0\\
1118	0\\
1119	0\\
1120	0\\
1121	8e-06\\
1122	1e-06\\
1123	4e-06\\
1124	1e-06\\
1125	1e-06\\
1126	1e-06\\
1127	1e-06\\
1128	1e-06\\
1129	0\\
1130	0\\
1131	8e-06\\
1132	0\\
1133	0\\
1134	0\\
1135	1.7e-05\\
1136	2.908612\\
1137	2.197431\\
1138	0.757512\\
1139	0.306036\\
1140	0.767711\\
1141	5e-06\\
1142	1e-06\\
1143	1.3e-05\\
1144	1.2e-05\\
1145	2e-05\\
1146	0.148303\\
1147	2.910001\\
1148	2.910001\\
1149	2.382677\\
1150	0.294341\\
1151	0.125237\\
1152	3e-06\\
1153	0\\
1154	1e-06\\
1155	5e-06\\
1156	0\\
1157	1.1e-05\\
1158	1e-06\\
1159	3.6e-05\\
1160	0.987208\\
1161	0.423428\\
1162	1.059701\\
1163	2.05043\\
1164	2.910001\\
1165	2.91\\
1166	2.910001\\
1167	2.910001\\
1168	2.910001\\
1169	2.910001\\
1170	3.530949\\
1171	3.98308\\
1172	2.991439\\
1173	2.887698\\
1174	0.008352\\
1175	0.647297\\
1176	0.001832\\
1177	3e-06\\
1178	5e-06\\
1179	1e-06\\
1180	1e-06\\
1181	1e-06\\
1182	1e-06\\
1183	0.30624\\
1184	2.910001\\
1185	2.55182\\
1186	2.874642\\
1187	2.931737\\
1188	3.377989\\
1189	2.910001\\
1190	2.910004\\
1191	2.910001\\
1192	2.910003\\
1193	2.910001\\
1194	3.357995\\
1195	3.743931\\
1196	3.358005\\
1197	3.117159\\
1198	1.870867\\
1199	2.264529\\
1200	0.972408\\
1201	3e-06\\
1202	0\\
1203	1e-06\\
1204	8e-06\\
1205	0\\
1206	1e-06\\
1207	4e-06\\
1208	2.8e-05\\
1209	4e-06\\
1210	8e-06\\
1211	1.8e-05\\
1212	0.604556\\
1213	0.175843\\
1214	0.348222\\
1215	0.306021\\
1216	0.306001\\
1217	0.809518\\
1218	1.660652\\
1219	2.764377\\
1220	1.349112\\
1221	0.116587\\
1222	1.7e-05\\
1223	2.1e-05\\
1224	2e-06\\
1225	0\\
1226	0\\
1227	1e-06\\
1228	2e-06\\
1229	1e-06\\
1230	0\\
1231	6e-06\\
1232	0.780758\\
1233	0.073225\\
1234	8.8e-05\\
1235	0.306027\\
1236	0.305985\\
1237	1e-06\\
1238	8e-06\\
1239	3e-06\\
1240	3e-06\\
1241	1e-06\\
1242	5e-06\\
1243	2.281154\\
1244	0.772019\\
1245	3e-06\\
1246	0\\
1247	2e-06\\
1248	3e-06\\
1249	0\\
1250	2e-06\\
1251	0\\
1252	0\\
1253	0\\
1254	0\\
1255	2e-06\\
1256	0\\
1257	1e-06\\
1258	1e-06\\
1259	8e-06\\
1260	1e-06\\
1261	0\\
1262	1e-06\\
1263	1e-06\\
1264	1e-06\\
1265	1e-06\\
1266	9e-06\\
1267	0\\
1268	3e-06\\
1269	3e-06\\
1270	1e-06\\
1271	1e-06\\
1272	2e-06\\
1273	0\\
1274	0\\
1275	0\\
1276	2e-06\\
1277	1e-06\\
1278	3e-06\\
1279	4e-06\\
1280	2e-06\\
1281	1e-06\\
1282	2e-06\\
1283	2e-06\\
1284	0\\
1285	1e-06\\
1286	2e-06\\
1287	1e-06\\
1288	0\\
1289	0\\
1290	4e-06\\
1291	1e-06\\
1292	0\\
1293	0\\
1294	1e-06\\
1295	1e-06\\
1296	1e-06\\
1297	9e-06\\
1298	0\\
1299	3e-06\\
1300	0\\
1301	3e-06\\
1302	1e-06\\
1303	1.2e-05\\
1304	0\\
1305	1e-06\\
1306	0\\
1307	1.1e-05\\
1308	0\\
1309	0\\
1310	1e-06\\
1311	1e-06\\
1312	2e-06\\
1313	0\\
1314	1e-06\\
1315	0.305734\\
1316	0.312512\\
1317	5e-06\\
1318	2e-06\\
1319	0\\
1320	0\\
1321	1e-06\\
1322	6e-06\\
1323	0\\
1324	0\\
1325	0\\
1326	0\\
1327	8e-06\\
1328	1.7e-05\\
1329	1e-06\\
1330	3e-06\\
1331	6e-06\\
1332	1e-06\\
1333	1e-06\\
1334	1e-06\\
1335	4e-06\\
1336	0.000109\\
1337	6e-06\\
1338	0.498101\\
1339	1.74823\\
1340	1.455472\\
1341	1e-06\\
1342	2e-06\\
1343	0\\
1344	2e-06\\
1345	1e-05\\
1346	1e-06\\
1347	1e-06\\
1348	1e-06\\
1349	1e-06\\
1350	1e-06\\
1351	2.9e-05\\
1352	1.146255\\
1353	1.04349\\
1354	2.17768\\
1355	1.376336\\
1356	0.814649\\
1357	3e-06\\
1358	4e-05\\
1359	2e-06\\
1360	0.038946\\
1361	4e-06\\
1362	0.158223\\
1363	2.910001\\
1364	2.910001\\
1365	1.131842\\
1366	0.013413\\
1367	0.306018\\
1368	1.2e-05\\
1369	6e-06\\
1370	1e-06\\
1371	1e-06\\
1372	0\\
1373	0\\
1374	1e-06\\
1375	1e-06\\
1376	1e-06\\
1377	3e-06\\
1378	0\\
1379	3e-06\\
1380	2e-06\\
1381	3e-06\\
1382	0.302492\\
1383	0.410332\\
1384	0.578779\\
1385	0.814663\\
1386	1.920933\\
1387	2.91\\
1388	2.910001\\
1389	2.836604\\
1390	1.78486\\
1391	2.484364\\
1392	1.684186\\
1393	0\\
1394	1.3e-05\\
1395	1e-06\\
1396	1e-06\\
1397	1e-06\\
1398	2e-06\\
1399	1.436188\\
1400	2.459968\\
1401	1.596261\\
1402	2.540869\\
1403	2.910001\\
1404	3.328104\\
1405	3.378001\\
1406	3.578228\\
1407	3.358042\\
1408	3.357997\\
1409	3.072901\\
1410	3.283204\\
1411	3.281978\\
1412	2.964777\\
1413	2.910002\\
1414	1.194519\\
1415	2.909733\\
1416	3.120666\\
1417	0.000835\\
1418	1e-06\\
1419	2e-06\\
1420	1e-06\\
1421	0\\
1422	0\\
1423	6e-06\\
1424	1e-06\\
1425	2e-06\\
1426	3e-06\\
1427	1.1e-05\\
1428	3.2e-05\\
1429	0.437303\\
1430	1e-06\\
1431	6.4e-05\\
1432	3.3e-05\\
1433	1.7e-05\\
1434	0\\
1435	0.774585\\
1436	0.774602\\
1437	1.5e-05\\
1438	3e-06\\
1439	4e-06\\
1440	3e-06\\
1441	1e-06\\
1442	0\\
1443	0\\
1444	0\\
1445	1e-06\\
1446	1e-06\\
1447	1e-06\\
1448	2e-06\\
1449	0\\
1450	0\\
1451	0\\
1452	4e-06\\
1453	1e-06\\
1454	0\\
1455	0\\
1456	1e-06\\
1457	4e-06\\
1458	5e-06\\
1459	0\\
1460	4e-06\\
1461	0\\
1462	5e-06\\
1463	4e-06\\
1464	3e-06\\
1465	0\\
1466	6e-06\\
1467	5e-06\\
1468	0\\
1469	1e-06\\
1470	2e-06\\
1471	1.2e-05\\
1472	2.6e-05\\
1473	2.2e-05\\
1474	4e-05\\
1475	2.1e-05\\
1476	7e-05\\
1477	0\\
1478	3.6e-05\\
1479	1.4e-05\\
1480	2e-06\\
1481	5.3e-05\\
1482	1.478165\\
1483	1.737808\\
1484	2.384167\\
1485	0.883219\\
1486	0.000111\\
1487	0.980438\\
1488	1.091235\\
1489	1.4e-05\\
1490	1e-06\\
1491	0\\
1492	0\\
1493	4e-06\\
1494	2.9e-05\\
1495	2.683771\\
1496	3.12066\\
1497	2.766925\\
1498	3.017474\\
1499	4.050909\\
1500	4.051174\\
1501	3.697474\\
1502	3.697474\\
1503	3.697481\\
1504	4.070875\\
1505	4.123447\\
1506	4.736418\\
1507	4.736418\\
1508	4.736418\\
1509	4.166542\\
1510	3.214073\\
1511	4.185822\\
1512	3.877884\\
1513	2.5877\\
1514	0.343442\\
1515	2e-06\\
1516	1.8e-05\\
1517	6.6e-05\\
1518	0.117646\\
1519	3.120641\\
1520	4.206064\\
1521	3.702315\\
1522	3.120664\\
1523	3.055142\\
1524	2.766926\\
1525	2.227478\\
1526	1.960536\\
1527	1.977953\\
1528	2.766925\\
1529	2.766926\\
1530	3.476226\\
1531	3.120635\\
1532	4.413129\\
1533	4.736415\\
1534	3.211831\\
1535	3.912693\\
1536	3.192876\\
1537	1.881464\\
1538	2e-06\\
1539	1e-06\\
1540	1e-06\\
1541	0\\
1542	1e-06\\
1543	1.727188\\
1544	0.914534\\
1545	0.731007\\
1546	9.9e-05\\
1547	0.29096\\
1548	4e-06\\
1549	1e-06\\
1550	0\\
1551	1e-06\\
1552	1e-06\\
1553	0.000103\\
1554	2.011314\\
1555	2.495534\\
1556	2.869362\\
1557	3.120597\\
1558	1.08908\\
1559	1.465024\\
1560	0.625508\\
1561	2e-06\\
1562	0\\
1563	1e-06\\
1564	1e-06\\
1565	1e-06\\
1566	2e-06\\
1567	7e-06\\
1568	3e-06\\
1569	1.3e-05\\
1570	1e-06\\
1571	5.3e-05\\
1572	2e-06\\
1573	2e-06\\
1574	1e-06\\
1575	2.3e-05\\
1576	6.3e-05\\
1577	1e-06\\
1578	1.023692\\
1579	0.99249\\
1580	2.19495\\
1581	2.766925\\
1582	0.367735\\
1583	1.101701\\
1584	1.14167\\
1585	3.7e-05\\
1586	1e-06\\
1587	1e-06\\
1588	1e-06\\
1589	1e-06\\
1590	1e-06\\
1591	1e-06\\
1592	1e-06\\
1593	0\\
1594	0\\
1595	1e-06\\
1596	0\\
1597	1e-06\\
1598	0\\
1599	0\\
1600	0\\
1601	0\\
1602	0\\
1603	8e-06\\
1604	0\\
1605	0\\
1606	1e-06\\
1607	0\\
1608	9e-06\\
1609	1e-06\\
1610	0\\
1611	1e-06\\
1612	1e-06\\
1613	1e-06\\
1614	1e-06\\
1615	7e-06\\
1616	0\\
1617	6e-06\\
1618	7e-06\\
1619	1e-06\\
1620	1e-06\\
1621	1e-06\\
1622	1e-06\\
1623	4e-06\\
1624	1e-06\\
1625	1e-06\\
1626	0\\
1627	1e-06\\
1628	7e-06\\
1629	1e-06\\
1630	0\\
1631	1.1e-05\\
1632	1e-06\\
1633	2e-06\\
1634	0\\
1635	2e-06\\
1636	2e-06\\
1637	5e-06\\
1638	1e-06\\
1639	2.234604\\
1640	2.85992\\
1641	2.179978\\
1642	1.332821\\
1643	1.430584\\
1644	1.134551\\
1645	2e-06\\
1646	8e-06\\
1647	2e-06\\
1648	0.204573\\
1649	1.171384\\
1650	3.743967\\
1651	4.364325\\
1652	3.938352\\
1653	3.360317\\
1654	1.715918\\
1655	1.864914\\
1656	1.171424\\
1657	0\\
1658	0\\
1659	4e-06\\
1660	6e-06\\
1661	2e-06\\
1662	0\\
1663	6.1e-05\\
1664	0.094664\\
1665	0.772766\\
1666	1.167866\\
1667	3.229675\\
1668	2.654749\\
1669	0.222262\\
1670	1.9e-05\\
1671	0\\
1672	1e-06\\
1673	2.2e-05\\
1674	0.942998\\
1675	3.138931\\
1676	2.770973\\
1677	2.780109\\
1678	0.34391\\
1679	0.462832\\
1680	0.271862\\
1681	9e-06\\
1682	1e-05\\
1683	1e-06\\
1684	1e-06\\
1685	0\\
1686	0\\
1687	1.807783\\
1688	1.43971\\
1689	0.753695\\
1690	2.452803\\
1691	0.462838\\
1692	0.000879\\
1693	1e-06\\
1694	2e-06\\
1695	7e-06\\
1696	1.9e-05\\
1697	0.100722\\
1698	2.477711\\
1699	3.195315\\
1700	3.655672\\
1701	3.229665\\
1702	2.9026\\
1703	3.22968\\
1704	1.262968\\
1705	6e-06\\
1706	1e-06\\
1707	1e-06\\
1708	1.1e-05\\
1709	0\\
1710	0\\
1711	3.229651\\
1712	3.583404\\
1713	3.229666\\
1714	2.77903\\
1715	2.25964\\
1716	1.030485\\
1717	0.00086\\
1718	0.000188\\
1719	0.000173\\
1720	0.000148\\
1721	0.358324\\
1722	1.73109\\
1723	2.174436\\
1724	3.445358\\
1725	3.229666\\
1726	2.337951\\
1727	2.683367\\
1728	1.237358\\
1729	4e-06\\
1730	1e-06\\
1731	2e-06\\
1732	0\\
1733	2e-06\\
1734	1e-06\\
1735	1.906736\\
1736	2.104039\\
1737	1.396853\\
1738	0.254908\\
1739	1.237358\\
1740	0.506797\\
1741	5.3e-05\\
1742	1e-06\\
1743	1e-05\\
1744	2e-06\\
1745	2.9e-05\\
1746	1.237231\\
1747	2.621203\\
1748	3.22968\\
1749	2.046035\\
1750	3.229692\\
1751	3.229664\\
1752	2.837721\\
1753	1.2e-05\\
1754	1e-06\\
1755	0\\
1756	0\\
1757	0\\
1758	0\\
1759	3e-06\\
1760	0\\
1761	6e-06\\
1762	3e-05\\
1763	2.7e-05\\
1764	0.000639\\
1765	5.4e-05\\
1766	4e-06\\
1767	1e-06\\
1768	0\\
1769	2.1e-05\\
1770	2e-06\\
1771	0.290908\\
1772	1.251716\\
1773	0.291049\\
1774	5.8e-05\\
1775	1e-06\\
1776	1.1e-05\\
1777	0\\
1778	6e-06\\
1779	8e-06\\
1780	3e-06\\
1781	4e-06\\
1782	1e-06\\
1783	1e-06\\
1784	1e-06\\
1785	7e-06\\
1786	1e-06\\
1787	0\\
1788	0\\
1789	3e-06\\
1790	0\\
1791	1e-06\\
1792	1e-06\\
1793	1e-06\\
1794	3e-06\\
1795	3e-06\\
1796	0.777622\\
1797	0.774615\\
1798	1e-06\\
1799	5.1e-05\\
1800	4.2e-05\\
1801	0\\
1802	1e-06\\
1803	0\\
1804	2e-06\\
1805	1e-06\\
1806	0\\
1807	2.766925\\
1808	3.120815\\
1809	3.073706\\
1810	3.292861\\
1811	4.736418\\
1812	4.736418\\
1813	3.766922\\
1814	3.311323\\
1815	3.229304\\
1816	3.873433\\
1817	4.348276\\
1818	4.736418\\
1819	4.736418\\
1820	4.736418\\
1821	4.736418\\
1822	4.288881\\
1823	4.044247\\
1824	3.192899\\
1825	1.297269\\
1826	1.2e-05\\
1827	0\\
1828	2e-06\\
1829	0\\
1830	3e-06\\
1831	3.120649\\
1832	4.736417\\
1833	4.247956\\
1834	4.736416\\
1835	3.988804\\
1836	4.736417\\
1837	4.312165\\
1838	4.247681\\
1839	3.241846\\
1840	3.155934\\
1841	3.150678\\
1842	3.823715\\
1843	4.044211\\
1844	3.316512\\
1845	2.766929\\
1846	2.109406\\
1847	2.620474\\
1848	1.202422\\
1849	0\\
1850	0\\
1851	0\\
1852	6e-06\\
1853	7e-06\\
1854	4e-06\\
1855	2.766929\\
1856	4.377601\\
1857	2.766925\\
1858	2.652362\\
1859	4.466358\\
1860	4.244193\\
1861	3.211941\\
1862	3.092966\\
1863	2.766925\\
1864	2.843634\\
1865	3.192899\\
1866	4.736418\\
1867	4.25431\\
1868	4.736418\\
1869	4.736418\\
1870	4.736419\\
1871	3.597719\\
1872	2.852355\\
1873	1.576119\\
1874	4e-06\\
1875	2e-06\\
1876	4e-06\\
1877	1.5e-05\\
1878	1.7e-05\\
1879	0.040341\\
1880	2.127386\\
1881	4.3e-05\\
1882	3e-06\\
1883	3e-05\\
1884	1.1e-05\\
1885	0\\
1886	8e-06\\
1887	0\\
1888	9e-06\\
1889	7e-06\\
1890	1.8e-05\\
1891	6e-05\\
1892	0.774615\\
1893	2.304263\\
1894	0.77461\\
1895	0.517176\\
1896	7.6e-05\\
1897	0\\
1898	8e-06\\
1899	5e-06\\
1900	1e-06\\
1901	1e-06\\
1902	7e-06\\
1903	1.63212\\
1904	3.211928\\
1905	3.633837\\
1906	4.23914\\
1907	4.618266\\
1908	4.736418\\
1909	4.736418\\
1910	4.04396\\
1911	2.864875\\
1912	2.766925\\
1913	2.766925\\
1914	2.767138\\
1915	2.209592\\
1916	2.766935\\
1917	2.766924\\
1918	2.766925\\
1919	2.766925\\
1920	2.40766\\
1921	1e-05\\
1922	0\\
1923	0\\
1924	1e-06\\
1925	1e-06\\
1926	1e-06\\
1927	1e-06\\
1928	1e-06\\
1929	0\\
1930	1e-06\\
1931	1e-06\\
1932	1.8e-05\\
1933	0\\
1934	0\\
1935	0\\
1936	1e-06\\
1937	1.2e-05\\
1938	3e-06\\
1939	0\\
1940	6.6e-05\\
1941	1.7e-05\\
1942	2.7e-05\\
1943	2.3e-05\\
1944	1e-06\\
1945	0\\
1946	1e-06\\
1947	0\\
1948	1e-06\\
1949	0\\
1950	0\\
1951	0\\
1952	0\\
1953	1e-05\\
1954	1e-06\\
1955	0\\
1956	1e-06\\
1957	1e-06\\
1958	2e-06\\
1959	7e-06\\
1960	0\\
1961	0\\
1962	8e-06\\
1963	0.427517\\
1964	0.290959\\
1965	0.323756\\
1966	5e-06\\
1967	1e-06\\
1968	0.759362\\
1969	7e-06\\
1970	0\\
1971	1e-06\\
1972	1e-06\\
1973	0\\
1974	3e-06\\
1975	2.766935\\
1976	4.64358\\
1977	2.892299\\
1978	2.766925\\
1979	2.766925\\
1980	2.822834\\
1981	2.766926\\
1982	2.766926\\
1983	2.766907\\
1984	2.867983\\
1985	3.192898\\
1986	4.736414\\
1987	4.335996\\
1988	4.736419\\
1989	4.736418\\
1990	3.801943\\
1991	4.603886\\
1992	3.302205\\
1993	1.491037\\
1994	1.3e-05\\
1995	3e-06\\
1996	2.7e-05\\
1997	1e-06\\
1998	0.423996\\
1999	3.19292\\
2000	2.975045\\
2001	3.812145\\
2002	3.192897\\
2003	2.766925\\
2004	2.557748\\
2005	1.559786\\
2006	1.734382\\
2007	2.09597\\
2008	2.766925\\
2009	2.951325\\
2010	3.476758\\
2011	4.734325\\
2012	4.736408\\
2013	4.473075\\
2014	3.120619\\
2015	2.766925\\
2016	2.766926\\
2017	0.727059\\
2018	2.7e-05\\
2019	2e-06\\
2020	1e-06\\
2021	6e-06\\
2022	0.290949\\
2023	3.211881\\
2024	3.701726\\
2025	3.192886\\
2026	3.120639\\
2027	3.192898\\
2028	3.192997\\
2029	3.099167\\
2030	4.736419\\
2031	4.736419\\
2032	4.783027\\
2033	5.595124\\
2034	6.735232\\
2035	6.856049\\
2036	8.962555\\
2037	8.952663\\
2038	8.236019\\
2039	8.382214\\
2040	8.955037\\
2041	7.797634\\
2042	5.865759\\
2043	5.666968\\
2044	5.666966\\
2045	4.736418\\
2046	5.115018\\
2047	7.319515\\
2048	8.059062\\
2049	7.910881\\
2050	7.851227\\
2051	7.87607\\
2052	7.347405\\
2053	6.88949\\
2054	7.263303\\
2055	7.319512\\
2056	7.319515\\
2057	7.791801\\
2058	9.221182\\
2059	12.111623\\
2060	10.979403\\
2061	10.151731\\
2062	8.122239\\
2063	8.663571\\
2064	7.87607\\
2065	6.796293\\
2066	5.197275\\
2067	4.736418\\
2068	4.736418\\
2069	4.736418\\
2070	5.739515\\
2071	7.66187\\
2072	7.869636\\
2073	7.477012\\
2074	7.454123\\
2075	7.319515\\
2076	8.250063\\
2077	8.250061\\
2078	8.250064\\
2079	7.17343\\
2080	7.148995\\
2081	7.296012\\
2082	7.319515\\
2083	8.160263\\
2084	7.319524\\
2085	7.509736\\
2086	7.319515\\
2087	7.319515\\
2088	6.871417\\
2089	4.736418\\
2090	4.631078\\
2091	3.211915\\
2092	3.120648\\
2093	3.187032\\
2094	3.211974\\
2095	3.948583\\
2096	4.454762\\
2097	4.736418\\
2098	4.736418\\
2099	4.171027\\
2100	3.296123\\
2101	3.120635\\
2102	2.766925\\
2103	2.603943\\
2104	2.766925\\
2105	3.211902\\
2106	4.736418\\
2107	4.736418\\
2108	4.736418\\
2109	5.066392\\
2110	4.736418\\
2111	4.736418\\
2112	4.736418\\
2113	3.539816\\
2114	2.980858\\
2115	2.766925\\
2116	2.363928\\
2117	2.766914\\
2118	2.766925\\
2119	2.766925\\
2120	3.120638\\
2121	3.299113\\
2122	3.192898\\
2123	2.874625\\
2124	2.766925\\
2125	1.437761\\
2126	0.516268\\
2127	0.719873\\
2128	1.364898\\
2129	2.766927\\
2130	3.224712\\
2131	4.736418\\
2132	4.736418\\
2133	4.736418\\
2134	4.736418\\
2135	4.736418\\
2136	4.044966\\
2137	3.211979\\
2138	3.098977\\
2139	3.04731\\
2140	3.211913\\
2141	4.736418\\
2142	6.539734\\
2143	8.96254\\
2144	7.319515\\
2145	7.319519\\
2146	7.319519\\
2147	6.379967\\
2148	4.73643\\
2149	4.736418\\
2150	4.736418\\
2151	4.736917\\
2152	5.567349\\
2153	6.983056\\
2154	7.319515\\
2155	7.319516\\
2156	9.221182\\
2157	7.332759\\
2158	7.319517\\
2159	6.206406\\
2160	7.973345\\
2161	7.474611\\
2162	8.129961\\
2163	7.7875\\
2164	8.130012\\
2165	7.474593\\
2166	10.053835\\
2167	10.822495\\
2168	9.245077\\
2169	8.918103\\
2170	9.682344\\
2171	9.528343\\
2172	8.945967\\
2173	8.94698\\
2174	8.946202\\
2175	8.946997\\
2176	9.198503\\
2177	10.971625\\
2178	12.19851\\
2179	10.952136\\
2180	11.868064\\
2181	11.901236\\
2182	11.85788\\
2183	10.075113\\
2184	7.900633\\
2185	7.179101\\
2186	6.093014\\
2187	5.594659\\
2188	5.587513\\
2189	6.96548\\
2190	9.17878\\
2191	10.922583\\
2192	10.247325\\
2193	10.025746\\
2194	10.056024\\
2195	9.365257\\
2196	9.17877\\
2197	8.946992\\
2198	8.946996\\
2199	8.822735\\
2200	8.946996\\
2201	9.189145\\
2202	10.488523\\
2203	10.730427\\
2204	10.898295\\
2205	11.154769\\
2206	10.341382\\
2207	8.946999\\
2208	7.474595\\
2209	6.475757\\
2210	5.170483\\
2211	5.159796\\
2212	5.159809\\
2213	6.270765\\
2214	8.193929\\
2215	9.769932\\
2216	9.322064\\
2217	9.178765\\
2218	8.947032\\
2219	9.177426\\
2220	8.688332\\
2221	8.735857\\
2222	8.659662\\
2223	8.422666\\
2224	8.946996\\
2225	10.121121\\
2226	11.985571\\
2227	13.71412\\
2228	12.324093\\
2229	11.433752\\
2230	10.270278\\
2231	9.199329\\
2232	7.474593\\
2233	6.783195\\
2234	5.719895\\
2235	5.702534\\
2236	6.247607\\
2237	7.474602\\
2238	9.17878\\
2239	11.95536\\
2240	13.714117\\
2241	13.714131\\
2242	14.773528\\
2243	17.699593\\
2244	11.719869\\
2245	10.559524\\
2246	9.792125\\
2247	9.178777\\
2248	9.178765\\
2249	10.493844\\
2250	11.167941\\
2251	9.779215\\
2252	10.278418\\
2253	9.178767\\
2254	10.256929\\
2255	9.302962\\
2256	7.474594\\
2257	6.477089\\
2258	5.555496\\
2259	5.159771\\
2260	5.160017\\
2261	5.572781\\
2262	6.477162\\
2263	7.28297\\
2264	7.4746\\
2265	7.474599\\
2266	7.474598\\
2267	7.638656\\
2268	7.474593\\
2269	6.775641\\
2270	6.219779\\
2271	5.944769\\
2272	6.106729\\
2273	6.477087\\
2274	7.474594\\
2275	7.474597\\
2276	7.474593\\
2277	7.900636\\
2278	7.474593\\
2279	7.474593\\
2280	6.379508\\
2281	5.159784\\
2282	5.159794\\
2283	4.771994\\
2284	4.312289\\
2285	4.647964\\
2286	5.159792\\
2287	4.822592\\
2288	5.159794\\
2289	5.159793\\
2290	5.15979\\
2291	5.159927\\
2292	5.745486\\
2293	5.159772\\
2294	5.159782\\
2295	5.159788\\
2296	5.159787\\
2297	5.189724\\
2298	6.477087\\
2299	7.032287\\
2300	7.474594\\
2301	7.474593\\
2302	7.21339\\
2303	6.477088\\
2304	5.159919\\
2305	5.159791\\
2306	5.158872\\
2307	4.984638\\
2308	5.159794\\
2309	5.159771\\
2310	7.474593\\
2311	7.97311\\
2312	9.178785\\
2313	9.378528\\
2314	9.93342\\
2315	9.713376\\
2316	8.946963\\
2317	8.946994\\
2318	8.885535\\
2319	8.940177\\
2320	8.940136\\
2321	8.965064\\
2322	8.940164\\
2323	8.798845\\
2324	8.946994\\
2325	9.211215\\
2326	9.178622\\
2327	7.973423\\
2328	6.477149\\
2329	5.159777\\
2330	5.15978\\
2331	5.159781\\
2332	5.159778\\
2333	6.247657\\
2334	7.870852\\
2335	8.579695\\
2336	9.08729\\
2337	8.947048\\
2338	8.940177\\
2339	8.947049\\
2340	7.973383\\
2341	7.474656\\
2342	7.474654\\
2343	7.639158\\
2344	8.108558\\
2345	8.998629\\
2346	8.28505\\
2347	8.946994\\
2348	11.742396\\
2349	11.930724\\
2350	12.149589\\
2351	10.523778\\
2352	8.834697\\
2353	7.609105\\
2354	7.501845\\
2355	7.104364\\
2356	7.580045\\
2357	8.274854\\
2358	10.983621\\
2359	11.867426\\
2360	12.410987\\
2361	12.446904\\
2362	11.28143\\
2363	10.931555\\
2364	9.081505\\
2365	9.081504\\
2366	8.337301\\
2367	8.400857\\
2368	9.081434\\
2369	9.42778\\
2370	10.987837\\
2371	10.889569\\
2372	10.333269\\
2373	11.594555\\
2374	11.180922\\
2375	10.907469\\
2376	9.081426\\
2377	7.609105\\
2378	7.609105\\
2379	7.411295\\
2380	7.609104\\
2381	8.087956\\
2382	11.136562\\
2383	13.848627\\
2384	13.848627\\
2385	12.444501\\
2386	11.856335\\
2387	12.223453\\
2388	9.671617\\
2389	9.844599\\
2390	9.549415\\
2391	9.667179\\
2392	10.049922\\
2393	11.636503\\
2394	12.832116\\
2395	11.128998\\
2396	11.774664\\
2397	12.316384\\
2398	11.85309\\
2399	11.009543\\
2400	8.284803\\
2401	7.474593\\
2402	6.922947\\
2403	6.808551\\
2404	7.016935\\
2405	7.496596\\
2406	9.292349\\
2407	11.067072\\
2408	11.948898\\
2409	11.593132\\
2410	11.292228\\
2411	10.765686\\
2412	9.468356\\
2413	8.947\\
2414	8.940174\\
2415	8.947005\\
2416	8.947002\\
2417	9.178768\\
2418	9.621545\\
2419	9.178768\\
2420	9.369398\\
2421	10.025947\\
2422	10.101548\\
2423	8.963708\\
2424	7.553584\\
2425	7.474593\\
2426	6.477108\\
2427	6.277154\\
2428	6.417149\\
2429	6.477104\\
2430	6.969409\\
2431	7.474595\\
2432	7.474595\\
2433	8.002284\\
2434	7.810061\\
2435	7.474596\\
2436	6.477088\\
2437	5.390733\\
2438	5.1598\\
2439	5.159813\\
2440	5.704022\\
2441	6.719911\\
2442	7.474595\\
2443	7.474593\\
2444	7.474593\\
2445	7.474593\\
2446	7.474592\\
2447	7.474597\\
2448	4.780425\\
2449	3.829476\\
2450	3.432931\\
2451	2.970977\\
2452	2.820556\\
2453	3.419645\\
2454	3.529084\\
2455	3.52901\\
2456	4.164958\\
2457	4.101374\\
2458	3.910767\\
2459	4.093421\\
2460	3.911014\\
2461	3.528957\\
2462	1.975595\\
2463	1.743849\\
2464	2.182976\\
2465	3.529005\\
2466	4.782102\\
2467	5.159798\\
2468	5.159919\\
2469	5.505218\\
2470	5.159859\\
2471	5.159794\\
2472	3.846286\\
2473	3.366205\\
2474	2.447841\\
2475	1.800534\\
2476	2.156054\\
2477	3.529346\\
2478	5.159789\\
2479	5.698795\\
2480	6.477087\\
2481	6.477095\\
2482	6.477095\\
2483	6.474087\\
2484	5.591095\\
2485	5.572773\\
2486	5.159831\\
2487	5.159827\\
2488	5.15983\\
2489	6.24758\\
2490	6.83884\\
2491	7.4747\\
2492	7.352001\\
2493	7.474719\\
2494	7.47472\\
2495	7.474593\\
2496	5.568752\\
2497	5.159769\\
2498	5.159799\\
2499	5.159814\\
2500	5.159774\\
2501	5.416695\\
2502	7.474709\\
2503	8.065196\\
2504	8.940191\\
2505	8.62616\\
2506	7.879397\\
2507	7.474593\\
2508	7.474593\\
2509	7.321941\\
2510	6.840744\\
2511	6.543581\\
2512	6.477129\\
2513	7.321788\\
2514	7.474592\\
2515	7.865317\\
2516	7.888431\\
2517	7.973353\\
2518	8.712692\\
2519	7.972762\\
2520	6.96928\\
2521	5.579549\\
2522	5.159777\\
2523	5.159774\\
2524	5.159902\\
2525	6.301206\\
2526	7.708042\\
2527	8.10855\\
2528	7.487672\\
2529	7.474702\\
2530	7.474713\\
2531	7.047706\\
2532	6.477104\\
2533	5.430083\\
2534	5.458207\\
2535	5.625287\\
2536	6.354957\\
2537	7.34972\\
2538	7.695688\\
2539	7.973353\\
2540	8.148821\\
2541	7.973345\\
2542	8.607623\\
2543	7.973348\\
2544	6.582385\\
2545	5.159787\\
2546	5.159825\\
2547	5.159813\\
2548	5.159789\\
2549	5.159773\\
2550	6.694029\\
2551	6.95705\\
2552	7.474686\\
2553	7.474669\\
2554	6.969593\\
2555	7.20907\\
2556	5.680408\\
2557	5.159777\\
2558	5.159807\\
2559	5.159773\\
2560	5.159819\\
2561	6.312878\\
2562	7.010063\\
2563	7.474729\\
2564	7.474702\\
2565	7.474709\\
2566	7.474592\\
2567	7.474715\\
2568	5.843899\\
2569	5.159817\\
2570	5.159758\\
2571	4.718661\\
2572	4.78323\\
2573	5.159782\\
2574	6.262244\\
2575	6.477139\\
2576	7.103542\\
2577	6.709515\\
2578	6.477562\\
2579	6.477092\\
2580	6.104135\\
2581	5.60831\\
2582	5.159816\\
2583	5.159772\\
2584	5.15977\\
2585	5.159863\\
2586	6.221983\\
2587	6.247761\\
2588	5.572758\\
2589	6.477087\\
2590	7.464257\\
2591	7.321787\\
2592	5.159808\\
2593	5.159797\\
2594	4.306693\\
2595	3.923026\\
2596	4.299473\\
2597	4.610871\\
2598	4.671048\\
2599	5.034754\\
2600	5.15982\\
2601	5.159781\\
2602	5.15978\\
2603	5.159835\\
2604	5.159805\\
2605	4.238774\\
2606	3.529255\\
2607	3.274663\\
2608	3.529197\\
2609	3.911006\\
2610	5.128325\\
2611	5.159781\\
2612	5.159767\\
2613	4.992715\\
2614	4.777149\\
2615	4.783217\\
2616	3.846256\\
2617	2.678757\\
2618	1.310196\\
2619	1.062689\\
2620	0.823751\\
2621	1.062689\\
2622	1.493245\\
2623	1.559363\\
2624	3.304469\\
2625	3.529209\\
2626	3.529223\\
2627	3.529075\\
2628	3.529097\\
2629	3.443469\\
2630	2.380925\\
2631	1.743915\\
2632	1.743921\\
2633	3.529236\\
2634	4.168936\\
2635	5.159797\\
2636	5.159818\\
2637	5.159905\\
2638	5.1598\\
2639	5.159816\\
2640	5.096732\\
2641	3.910954\\
2642	3.529292\\
2643	3.072768\\
2644	3.09329\\
2645	3.529325\\
2646	3.528998\\
2647	3.528222\\
2648	3.529\\
2649	3.528973\\
2650	3.635745\\
2651	3.529227\\
2652	3.508631\\
2653	1.860648\\
2654	2.252122\\
2655	2.456721\\
2656	3.529302\\
2657	4.672582\\
2658	5.159884\\
2659	5.159795\\
2660	5.159903\\
2661	5.159797\\
2662	5.159792\\
2663	5.159952\\
2664	5.159779\\
2665	4.31392\\
2666	3.82871\\
2667	3.624408\\
2668	3.910297\\
2669	5.159792\\
2670	6.678662\\
2671	7.973357\\
2672	7.953136\\
2673	7.474606\\
2674	7.4746\\
2675	7.474695\\
2676	6.955327\\
2677	6.86826\\
2678	6.969549\\
2679	6.882028\\
2680	7.474612\\
2681	7.4761\\
2682	8.104195\\
2683	8.411595\\
2684	7.781323\\
2685	7.936699\\
2686	8.223408\\
2687	7.474596\\
2688	6.440197\\
2689	5.159832\\
2690	5.159791\\
2691	5.159792\\
2692	5.159784\\
2693	5.289869\\
2694	7.474592\\
2695	7.798865\\
2696	7.476689\\
2697	7.4746\\
2698	7.474694\\
2699	7.474594\\
2700	7.47462\\
2701	7.474617\\
2702	7.474701\\
2703	7.474594\\
2704	7.474619\\
2705	7.656785\\
2706	8.649962\\
2707	8.690628\\
2708	7.886934\\
2709	8.946997\\
2710	8.940176\\
2711	7.474595\\
2712	6.193821\\
2713	5.159831\\
2714	5.159798\\
2715	5.159813\\
2716	5.159806\\
2717	5.15979\\
2718	7.474729\\
2719	8.707427\\
2720	8.947006\\
2721	9.178808\\
2722	8.947072\\
2723	9.178771\\
2724	8.940171\\
2725	8.66654\\
2726	7.973598\\
2727	8.249706\\
2728	8.348402\\
2729	8.821179\\
2730	8.947005\\
2731	8.433132\\
2732	7.906816\\
2733	7.475235\\
2734	7.474658\\
2735	6.4771\\
2736	5.159798\\
2737	4.191186\\
2738	3.765113\\
2739	3.5292\\
2740	3.559739\\
2741	4.666575\\
2742	5.159794\\
2743	6.806059\\
2744	6.477067\\
2745	6.696496\\
2746	5.614896\\
2747	5.1598\\
2748	5.159806\\
2749	5.159784\\
2750	5.159861\\
2751	5.159909\\
2752	5.321027\\
2753	6.360502\\
2754	7.474697\\
2755	7.474636\\
2756	7.148756\\
2757	7.322411\\
2758	6.736143\\
2759	6.247601\\
2760	5.159806\\
2761	4.1411\\
2762	3.528963\\
2763	3.528626\\
2764	3.529203\\
2765	3.529081\\
2766	3.528931\\
2767	3.603413\\
2768	4.193073\\
2769	4.605946\\
2770	4.006687\\
2771	3.909849\\
2772	3.528337\\
2773	3.528103\\
2774	3.44172\\
2775	3.528936\\
2776	3.910915\\
2777	5.159297\\
2778	5.159801\\
2779	5.159927\\
2780	5.159814\\
2781	5.159914\\
2782	5.159794\\
2783	5.159769\\
2784	3.799668\\
2785	2.850044\\
2786	1.432795\\
2787	1.148549\\
2788	1.063086\\
2789	1.17152\\
2790	1.097702\\
2791	1.310482\\
2792	2.107\\
2793	3.137138\\
2794	3.272159\\
2795	3.529184\\
2796	3.529192\\
2797	2.445956\\
2798	2.170719\\
2799	2.158647\\
2800	2.596685\\
2801	3.529202\\
2802	5.194881\\
2803	5.815186\\
2804	5.815185\\
2805	5.815187\\
2806	5.815187\\
2807	5.815187\\
2808	4.469705\\
2809	4.050239\\
2810	3.862168\\
2811	3.508919\\
2812	4.050236\\
2813	5.273307\\
2814	6.31159\\
2815	7.976688\\
2816	8.130012\\
2817	7.311792\\
2818	7.331655\\
2819	8.130012\\
2820	8.130012\\
2821	8.130012\\
2822	8.130012\\
2823	8.130012\\
2824	8.130014\\
2825	8.271963\\
2826	8.628769\\
2827	8.976253\\
2828	8.611862\\
2829	8.130012\\
2830	8.355127\\
2831	8.130011\\
2832	6.065355\\
2833	5.815185\\
2834	5.815185\\
2835	5.811802\\
2836	5.815185\\
2837	5.815185\\
2838	7.132506\\
2839	8.130012\\
2840	8.773402\\
2841	9.59558\\
2842	9.078544\\
2843	9.602412\\
2844	9.289149\\
2845	9.602412\\
2846	9.602412\\
2847	9.595589\\
2848	9.438169\\
2849	7.973353\\
2850	8.108444\\
2851	7.474593\\
2852	7.474614\\
2853	7.474611\\
2854	7.474594\\
2855	6.477118\\
2856	5.159773\\
2857	5.159792\\
2858	4.782885\\
2859	4.668217\\
2860	4.782809\\
2861	5.159786\\
2862	6.477118\\
2863	7.474598\\
2864	7.638771\\
2865	7.973352\\
2866	7.973455\\
2867	7.620454\\
2868	7.474599\\
2869	7.474598\\
2870	6.915391\\
2871	6.477117\\
2872	6.477116\\
2873	7.034606\\
2874	7.41093\\
2875	7.299585\\
2876	6.891236\\
2877	7.474602\\
2878	7.474612\\
2879	6.791847\\
2880	5.608819\\
2881	4.804204\\
2882	4.802731\\
2883	4.714414\\
2884	4.376137\\
2885	4.376054\\
2886	3.946058\\
2887	3.662366\\
2888	3.716928\\
2889	4.196369\\
2890	4.426675\\
2891	4.802526\\
2892	4.803659\\
2893	4.544061\\
2894	4.801918\\
2895	4.801767\\
2896	4.782466\\
2897	4.789108\\
2898	5.774321\\
2899	5.989181\\
2900	5.931232\\
2901	6.123433\\
2902	6.923258\\
2903	6.442838\\
2904	4.789588\\
2905	4.655829\\
2906	3.671467\\
2907	3.601786\\
2908	3.601787\\
2909	4.015909\\
2910	4.785885\\
2911	4.802985\\
2912	4.803765\\
2913	4.804249\\
2914	4.803563\\
2915	4.802538\\
2916	4.583698\\
2917	3.503987\\
2918	3.304814\\
2919	3.304839\\
2920	3.395265\\
2921	4.384949\\
2922	4.800864\\
2923	4.803368\\
2924	4.803835\\
2925	4.804056\\
2926	4.802152\\
2927	4.804227\\
2928	4.543778\\
2929	3.304799\\
2930	3.29208\\
2931	3.179269\\
2932	3.179218\\
2933	3.304826\\
2934	3.304815\\
2935	3.662409\\
2936	4.544379\\
2937	4.801259\\
2938	4.543367\\
2939	3.678379\\
2940	3.601784\\
2941	3.30482\\
2942	3.30485\\
2943	3.30485\\
2944	3.3048\\
2945	4.395805\\
2946	4.803864\\
2947	4.804213\\
2948	4.80305\\
2949	4.803637\\
2950	4.803993\\
2951	4.803667\\
2952	4.804227\\
2953	4.552819\\
2954	3.725727\\
2955	3.662418\\
2956	3.662417\\
2957	3.756822\\
2958	3.662415\\
2959	4.00547\\
2960	4.033527\\
2961	3.773348\\
2962	3.662346\\
2963	3.304819\\
2964	3.179232\\
2965	2.251599\\
2966	1.465566\\
2967	1.632872\\
2968	2.939892\\
2969	3.66236\\
2970	4.803921\\
2971	4.787457\\
2972	4.914112\\
2973	4.787166\\
2974	5.246982\\
2975	4.788831\\
2976	4.803254\\
2977	4.035988\\
2978	3.413443\\
2979	3.304802\\
2980	3.662404\\
2981	4.790701\\
2982	5.989222\\
2983	7.319737\\
2984	6.923258\\
2985	7.125398\\
2986	6.484172\\
2987	5.709287\\
2988	4.79096\\
2989	4.790554\\
2990	4.812198\\
2991	4.803021\\
2992	5.757757\\
2993	6.923278\\
2994	6.954151\\
2995	7.390306\\
2996	6.923258\\
2997	6.923258\\
2998	6.923259\\
2999	5.989229\\
3000	4.786617\\
3001	3.758676\\
3002	3.431251\\
3003	3.304798\\
3004	3.601787\\
3005	4.764043\\
3006	4.78901\\
3007	6.923258\\
3008	7.076994\\
3009	7.948162\\
3010	7.621828\\
3011	7.503799\\
3012	6.923327\\
3013	6.923283\\
3014	6.3294\\
3015	6.435615\\
3016	6.782568\\
3017	6.923258\\
3018	6.923323\\
3019	7.233145\\
3020	6.923259\\
3021	6.923261\\
3022	6.92326\\
3023	6.893131\\
3024	4.809893\\
3025	4.814544\\
3026	4.781173\\
3027	4.78496\\
3028	4.814651\\
3029	4.811028\\
3030	5.992333\\
3031	7.32754\\
3032	7.894191\\
3033	7.517588\\
3034	7.322222\\
3035	7.39046\\
3036	7.100845\\
3037	6.976686\\
3038	6.923261\\
3039	6.923259\\
3040	6.780274\\
3041	6.923259\\
3042	6.923259\\
3043	7.1636\\
3044	6.923259\\
3045	6.923262\\
3046	7.441749\\
3047	6.923262\\
3048	4.811174\\
3049	4.811736\\
3050	4.777111\\
3051	4.476251\\
3052	4.782767\\
3053	4.815171\\
3054	6.172817\\
3055	6.92326\\
3056	6.92326\\
3057	6.923271\\
3058	7.775666\\
3059	8.302057\\
3060	7.637353\\
3061	8.291518\\
3062	8.200122\\
3063	8.295677\\
3064	8.295679\\
3065	8.302113\\
3066	8.302112\\
3067	7.362089\\
3068	6.92326\\
3069	6.909911\\
3070	6.78853\\
3071	5.989164\\
3072	4.812617\\
3073	4.781261\\
3074	4.089512\\
3075	4.149898\\
3076	4.150313\\
3077	4.811845\\
3078	5.593978\\
3079	6.958129\\
3080	7.04574\\
3081	6.923276\\
3082	6.92326\\
3083	6.92326\\
3084	5.989166\\
3085	4.862361\\
3086	4.817002\\
3087	4.817713\\
3088	4.817438\\
3089	4.789141\\
3090	5.142548\\
3091	5.293617\\
3092	4.803995\\
3093	4.804409\\
3094	6.009025\\
3095	5.796029\\
3096	4.798834\\
3097	4.816348\\
3098	4.767413\\
3099	4.165825\\
3100	4.105654\\
3101	4.150142\\
3102	4.165809\\
3103	4.781289\\
3104	4.813448\\
3105	5.142625\\
3106	5.774948\\
3107	5.10174\\
3108	4.806228\\
3109	4.805317\\
3110	4.805046\\
3111	4.806129\\
3112	4.014282\\
3113	4.250115\\
3114	4.614943\\
3115	4.543752\\
3116	3.662418\\
3117	3.677684\\
3118	4.30482\\
3119	3.723353\\
3120	3.304835\\
3121	1.633016\\
3122	0.301087\\
3123	9e-06\\
3124	0\\
3125	4.5e-05\\
3126	2e-06\\
3127	6e-06\\
3128	0.510113\\
3129	0.881688\\
3130	0.768894\\
3131	1.198699\\
3132	0.6202\\
3133	0.145743\\
3134	6e-05\\
3135	0\\
3136	1e-06\\
3137	1.200232\\
3138	3.179395\\
3139	3.300934\\
3140	3.30497\\
3141	3.304971\\
3142	3.644564\\
3143	3.304971\\
3144	2.386719\\
3145	2.058271\\
3146	1.214216\\
3147	1.092974\\
3148	2.090614\\
3149	3.304967\\
3150	4.957722\\
3151	4.957722\\
3152	5.962266\\
3153	5.976398\\
3154	6.604028\\
3155	6.648254\\
3156	5.557838\\
3157	6.691207\\
3158	6.380725\\
3159	6.661685\\
3160	7.125394\\
3161	7.125404\\
3162	7.27957\\
3163	7.838782\\
3164	8.504195\\
3165	8.497812\\
3166	8.504196\\
3167	7.718616\\
3168	6.164203\\
3169	4.804663\\
3170	4.803579\\
3171	4.803278\\
3172	4.803584\\
3173	4.802875\\
3174	6.62434\\
3175	6.923261\\
3176	8.302061\\
3177	8.302108\\
3178	7.082409\\
3179	7.39031\\
3180	8.30206\\
3181	7.228062\\
3182	6.923258\\
3183	7.104306\\
3184	7.076706\\
3185	6.923258\\
3186	7.808202\\
3187	7.516906\\
3188	7.322217\\
3189	7.318822\\
3190	8.302048\\
3191	7.334515\\
3192	6.296827\\
3193	4.786386\\
3194	4.804396\\
3195	4.803626\\
3196	4.804229\\
3197	5.62562\\
3198	6.923258\\
3199	8.295684\\
3200	8.044965\\
3201	7.076992\\
3202	7.390316\\
3203	7.322219\\
3204	7.076976\\
3205	6.923258\\
3206	6.923269\\
3207	6.790328\\
3208	6.470656\\
3209	6.923353\\
3210	6.923258\\
3211	6.923258\\
3212	6.923258\\
3213	6.923258\\
3214	7.70948\\
3215	6.994835\\
3216	5.232307\\
3217	4.804935\\
3218	4.803722\\
3219	4.803669\\
3220	4.804099\\
3221	4.924247\\
3222	6.923261\\
3223	7.390317\\
3224	8.081892\\
3225	8.622382\\
3226	7.810362\\
3227	7.379793\\
3228	6.596843\\
3229	5.500469\\
3230	5.15733\\
3231	5.142602\\
3232	5.989245\\
3233	6.923264\\
3234	6.923459\\
3235	7.249767\\
3236	6.923264\\
3237	6.923342\\
3238	7.762575\\
3239	6.923265\\
3240	5.69694\\
3241	4.804802\\
3242	4.803589\\
3243	4.803435\\
3244	4.803916\\
3245	4.786159\\
3246	6.644795\\
3247	7.48288\\
3248	7.390306\\
3249	6.977362\\
3250	6.923258\\
3251	6.780167\\
3252	5.552891\\
3253	4.843745\\
3254	4.786202\\
3255	4.804907\\
3256	4.785518\\
3257	5.958899\\
3258	6.780157\\
3259	6.923258\\
3260	6.911617\\
3261	6.646666\\
3262	6.92327\\
3263	6.923268\\
3264	6.192057\\
3265	4.957726\\
3266	4.957923\\
3267	4.95784\\
3268	4.957905\\
3269	4.95791\\
3270	4.957897\\
3271	4.957733\\
3272	4.995506\\
3273	5.344463\\
3274	4.957741\\
3275	4.957837\\
3276	4.957874\\
3277	3.986982\\
3278	3.599647\\
3279	3.6018\\
3280	4.547561\\
3281	4.957839\\
3282	4.960404\\
3283	5.976435\\
3284	6.055622\\
3285	6.191302\\
3286	5.976394\\
3287	6.191304\\
3288	4.957726\\
3289	4.357132\\
3290	3.601791\\
3291	3.305006\\
3292	3.304815\\
3293	3.304826\\
3294	3.304841\\
3295	3.30481\\
3296	3.379591\\
3297	3.304799\\
3298	3.304811\\
3299	3.304843\\
3300	3.08763\\
3301	2.250743\\
3302	1.788647\\
3303	1.633053\\
3304	2.45136\\
3305	3.304831\\
3306	4.326681\\
3307	4.802145\\
3308	4.803205\\
3309	4.803772\\
3310	4.804448\\
3311	4.803825\\
3312	3.678375\\
3313	3.304818\\
3314	3.179283\\
3315	3.09823\\
3316	3.304823\\
3317	3.678377\\
3318	4.804446\\
3319	5.957691\\
3320	6.780171\\
3321	6.785425\\
3322	6.923258\\
3323	6.923258\\
3324	5.989166\\
3325	5.989173\\
3326	5.989166\\
3327	5.989333\\
3328	6.0451\\
3329	6.923342\\
3330	6.923258\\
3331	6.923258\\
3332	6.773169\\
3333	5.989185\\
3334	6.781624\\
3335	5.774338\\
3336	4.957726\\
3337	4.957845\\
3338	3.895739\\
3339	3.675119\\
3340	4.212114\\
3341	4.957749\\
3342	5.219405\\
3343	7.059987\\
3344	7.125397\\
3345	7.125398\\
3346	7.125399\\
3347	7.125396\\
3348	6.585484\\
3349	6.654419\\
3350	6.915926\\
3351	7.125396\\
3352	7.125398\\
3353	7.634723\\
3354	8.485259\\
3355	8.504196\\
3356	7.753183\\
3357	8.497811\\
3358	7.699381\\
3359	7.1254\\
3360	6.651889\\
3361	5.367864\\
3362	4.957743\\
3363	4.957722\\
3364	4.957722\\
3365	4.957727\\
3366	6.981281\\
3367	7.125397\\
3368	7.682677\\
3369	8.504196\\
3370	7.69043\\
3371	8.579625\\
3372	8.504196\\
3373	8.240146\\
3374	8.439461\\
3375	8.504196\\
3376	8.500617\\
3377	8.497814\\
3378	8.42084\\
3379	7.906003\\
3380	7.200235\\
3381	7.279129\\
3382	7.592444\\
3383	7.125396\\
3384	4.957902\\
3385	3.179239\\
3386	2.250743\\
3387	2.250743\\
3388	2.825472\\
3389	3.304816\\
3390	4.629264\\
3391	4.852061\\
3392	5.928901\\
3393	5.989166\\
3394	5.989039\\
3395	5.233641\\
3396	4.785974\\
3397	4.963777\\
3398	5.147721\\
3399	5.390226\\
3400	5.989182\\
3401	6.481809\\
3402	6.92326\\
3403	6.899832\\
3404	6.065181\\
3405	6.913465\\
3406	6.923258\\
3407	5.989348\\
3408	4.804118\\
3409	4.406978\\
3410	3.601785\\
3411	3.545169\\
3412	3.678337\\
3413	4.544353\\
3414	4.786714\\
3415	6.923269\\
3416	6.859377\\
3417	6.923261\\
3418	6.923262\\
3419	6.923263\\
3420	6.748286\\
3421	6.872406\\
3422	6.194608\\
3423	5.989186\\
3424	5.989166\\
3425	6.92327\\
3426	6.747462\\
3427	6.394532\\
3428	5.777075\\
3429	4.853098\\
3430	5.924741\\
3431	5.514718\\
3432	4.79513\\
3433	4.798583\\
3434	3.828218\\
3435	3.311588\\
3436	3.30481\\
3437	3.304694\\
3438	3.304689\\
3439	3.452679\\
3440	4.545312\\
3441	4.545288\\
3442	4.375994\\
3443	3.718811\\
3444	3.662416\\
3445	3.3047\\
3446	2.250757\\
3447	2.250754\\
3448	3.304694\\
3449	3.678518\\
3450	4.376188\\
3451	4.795301\\
3452	4.795007\\
3453	4.271663\\
3454	4.795659\\
3455	3.678369\\
3456	3.678377\\
3457	3.304792\\
3458	3.10775\\
3459	2.892088\\
3460	3.005818\\
3461	2.663278\\
3462	2.884155\\
3463	3.304711\\
3464	3.306948\\
3465	3.600667\\
3466	3.304788\\
3467	3.282566\\
3468	2.91092\\
3469	2.250649\\
3470	1.661159\\
3471	2.250708\\
3472	3.236447\\
3473	3.678352\\
3474	4.795364\\
3475	4.800448\\
3476	4.795785\\
3477	4.798205\\
3478	4.80104\\
3479	4.79734\\
3480	3.96762\\
3481	3.304776\\
3482	3.179232\\
3483	3.179243\\
3484	3.304707\\
3485	3.640237\\
3486	4.801899\\
3487	7.125396\\
3488	7.783678\\
3489	8.504197\\
3490	8.504197\\
3491	8.678527\\
3492	8.191905\\
3493	7.122946\\
3494	6.979987\\
3495	7.034168\\
3496	7.100978\\
3497	7.390305\\
3498	7.322217\\
3499	6.923258\\
3500	6.518234\\
3501	5.989183\\
3502	6.273591\\
3503	5.749805\\
3504	4.801576\\
3505	4.795621\\
3506	4.37618\\
3507	4.2758\\
3508	4.546156\\
3509	4.797207\\
3510	5.142361\\
3511	6.923271\\
3512	7.390307\\
3513	8.302061\\
3514	8.519108\\
3515	8.302113\\
3516	6.92326\\
3517	7.345575\\
3518	7.322306\\
3519	8.269399\\
3520	8.295673\\
3521	8.295672\\
3522	8.295642\\
3523	7.319166\\
3524	6.923261\\
3525	6.923265\\
3526	6.459754\\
3527	5.989299\\
3528	4.799729\\
3529	4.800684\\
3530	4.780874\\
3531	4.651578\\
3532	4.72792\\
3533	4.800572\\
3534	5.774463\\
3535	6.977511\\
3536	7.516924\\
3537	8.302062\\
3538	8.51908\\
3539	8.371363\\
3540	8.519101\\
3541	8.302063\\
3542	8.117751\\
3543	7.541503\\
3544	7.390305\\
3545	7.745416\\
3546	7.735617\\
3547	7.306627\\
3548	6.923258\\
3549	6.923266\\
3550	7.127118\\
3551	6.923258\\
3552	5.14236\\
3553	4.801843\\
3554	4.794971\\
3555	4.311215\\
3556	3.678342\\
3557	3.541819\\
3558	3.304804\\
3559	3.601786\\
3560	3.761004\\
3561	4.546555\\
3562	4.795063\\
3563	4.798704\\
3564	4.797196\\
3565	4.376247\\
3566	3.601787\\
3567	3.306046\\
3568	3.304809\\
3569	3.601786\\
3570	4.624504\\
3571	4.801289\\
3572	4.801161\\
3573	4.802632\\
3574	4.800892\\
3575	4.800462\\
3576	3.91595\\
3577	3.304801\\
3578	3.179199\\
3579	3.179214\\
3580	3.179234\\
3581	3.30472\\
3582	3.662419\\
3583	4.79562\\
3584	4.795346\\
3585	4.796409\\
3586	4.79523\\
3587	4.799688\\
3588	4.036323\\
3589	3.601787\\
3590	3.612015\\
3591	3.304807\\
3592	3.607284\\
3593	3.662415\\
3594	4.797966\\
3595	4.802456\\
3596	4.801609\\
3597	4.799564\\
3598	4.802202\\
3599	4.801577\\
3600	4.802221\\
3601	3.678319\\
3602	3.30484\\
3603	3.304744\\
3604	3.304784\\
3605	3.304816\\
3606	3.304795\\
3607	3.963003\\
3608	4.548082\\
3609	4.800939\\
3610	4.019911\\
3611	3.976586\\
3612	3.927536\\
3613	3.662417\\
3614	3.381111\\
3615	3.360525\\
3616	3.69741\\
3617	4.797545\\
3618	4.802726\\
3619	4.810658\\
3620	4.81145\\
3621	4.801655\\
3622	4.787254\\
3623	4.788169\\
3624	4.940134\\
3625	4.47772\\
3626	4.479638\\
3627	4.922293\\
3628	4.922293\\
3629	4.922293\\
3630	4.922293\\
3631	4.922293\\
3632	4.922293\\
3633	5.271747\\
3634	5.38384\\
3635	5.272429\\
3636	4.922293\\
3637	4.922293\\
3638	4.922293\\
3639	4.660322\\
3640	4.922293\\
3641	5.110125\\
3642	6.52151\\
3643	6.752432\\
3644	6.881689\\
3645	6.881692\\
3646	6.881689\\
3647	6.881692\\
3648	6.037346\\
3649	5.27186\\
3650	4.922293\\
3651	4.922293\\
3652	4.966136\\
3653	5.819333\\
3654	6.881691\\
3655	8.32419\\
3656	8.918808\\
3657	8.644668\\
3658	8.32419\\
3659	8.168888\\
3660	7.786387\\
3661	8.064058\\
3662	8.12224\\
3663	8.12224\\
3664	8.128012\\
3665	8.985563\\
3666	8.32419\\
3667	8.420175\\
3668	8.128012\\
3669	8.128018\\
3670	9.364566\\
3671	8.128009\\
3672	6.881692\\
3673	6.615675\\
3674	6.037346\\
3675	5.865886\\
3676	5.84297\\
3677	6.037353\\
3678	6.881691\\
3679	8.854804\\
3680	9.875465\\
3681	10.165934\\
3682	9.2272\\
3683	9.075084\\
3684	8.121935\\
3685	7.303639\\
3686	7.557392\\
3687	8.128009\\
3688	8.324231\\
3689	8.324196\\
3690	9.838646\\
3691	9.815219\\
3692	8.574311\\
3693	8.128022\\
3694	9.077744\\
3695	8.128002\\
3696	6.881692\\
3697	13.522295\\
3698	13.266926\\
3699	13.256288\\
3700	13.522293\\
3701	13.522294\\
3702	14.760149\\
3703	14.909328\\
3704	15.964878\\
3705	16.091113\\
3706	16.907074\\
3707	16.462464\\
3708	16.907019\\
3709	16.872605\\
3710	12.163165\\
3711	11.965799\\
3712	11.70602\\
3713	12.018917\\
3714	15.762363\\
3715	14.914432\\
3716	14.909266\\
3717	14.837069\\
3718	7.910721\\
3719	6.881692\\
3720	5.686808\\
3721	4.922293\\
3722	4.922293\\
3723	4.922293\\
3724	4.922293\\
3725	4.922293\\
3726	6.03735\\
3727	7.059421\\
3728	8.122188\\
3729	7.836641\\
3730	7.020657\\
3731	6.881692\\
3732	6.881689\\
3733	6.615676\\
3734	6.394005\\
3735	6.037346\\
3736	6.615675\\
3737	6.881692\\
3738	7.444609\\
3739	8.122237\\
3740	7.46288\\
3741	7.303847\\
3742	8.129129\\
3743	7.876838\\
3744	6.881691\\
3745	6.454234\\
3746	5.98508\\
3747	5.864398\\
3748	6.037346\\
3749	6.615675\\
3750	7.303444\\
3751	9.602122\\
3752	10.200849\\
3753	8.911155\\
3754	8.277728\\
3755	8.128003\\
3756	7.850268\\
3757	7.303873\\
3758	6.881692\\
3759	6.881692\\
3760	7.020694\\
3761	7.85851\\
3762	8.128004\\
3763	8.12801\\
3764	7.41822\\
3765	6.881692\\
3766	7.744471\\
3767	6.881692\\
3768	6.264454\\
3769	5.533759\\
3770	4.922293\\
3771	4.922293\\
3772	4.922293\\
3773	4.322566\\
3774	4.551552\\
3775	4.922293\\
3776	5.457805\\
3777	6.037345\\
3778	6.19049\\
3779	6.118857\\
3780	6.013033\\
3781	5.687085\\
3782	4.922294\\
3783	5.088646\\
3784	5.687092\\
3785	6.037347\\
3786	6.615678\\
3787	6.88169\\
3788	6.881648\\
3789	6.88169\\
3790	6.88169\\
3791	6.88169\\
3792	6.067996\\
3793	5.340139\\
3794	4.922293\\
3795	4.922293\\
3796	4.922293\\
3797	4.922293\\
3798	4.922038\\
3799	4.922293\\
3800	5.260072\\
3801	5.687107\\
3802	5.84309\\
3803	5.687087\\
3804	5.271857\\
3805	4.922293\\
3806	4.551564\\
3807	4.285255\\
3808	4.922223\\
3809	4.922293\\
3810	5.687087\\
3811	5.843085\\
3812	6.037346\\
3813	6.14928\\
3814	6.037346\\
3815	6.037346\\
3816	5.271864\\
3817	4.922293\\
3818	4.548158\\
3819	4.413465\\
3820	4.08264\\
3821	3.751463\\
3822	3.765908\\
3823	4.770781\\
3824	4.922293\\
3825	4.922293\\
3826	4.922293\\
3827	4.92246\\
3828	4.922293\\
3829	4.922293\\
3830	4.922293\\
3831	4.922293\\
3832	4.922293\\
3833	5.610983\\
3834	6.037369\\
3835	6.615688\\
3836	6.881689\\
3837	6.87914\\
3838	6.881692\\
3839	6.881684\\
3840	9.447867\\
3841	8.842052\\
3842	8.647599\\
3843	8.061733\\
3844	8.076538\\
3845	8.731891\\
3846	9.786342\\
3847	11.579579\\
3848	14.417049\\
3849	14.96789\\
3850	14.583797\\
3851	14.967861\\
3852	13.987893\\
3853	13.729711\\
3854	13.810376\\
3855	13.065034\\
3856	13.008994\\
3857	13.859589\\
3858	14.967885\\
3859	14.944808\\
3860	13.510643\\
3861	12.506579\\
3862	13.769715\\
3863	12.290619\\
3864	10.552843\\
3865	13.944478\\
3866	13.522306\\
3867	13.522306\\
3868	13.651604\\
3869	13.941989\\
3870	23.430258\\
3871	26.142084\\
3872	27.237366\\
3873	27.123103\\
3874	25.995432\\
3875	25.600756\\
3876	25.131344\\
3877	24.885417\\
3878	24.04857\\
3879	23.998645\\
3880	25.106948\\
3881	26.154451\\
3882	14.169368\\
3883	15.108618\\
3884	14.980736\\
3885	13.829421\\
3886	14.967869\\
3887	13.719262\\
3888	10.932731\\
3889	10.108799\\
3890	9.686398\\
3891	9.686417\\
3892	9.686399\\
3893	9.970381\\
3894	11.224919\\
3895	15.033911\\
3896	15.889805\\
3897	14.55734\\
3898	14.456199\\
3899	13.566721\\
3900	11.757358\\
3901	12.05091\\
3902	11.561737\\
3903	11.80811\\
3904	13.173668\\
3905	13.916034\\
3906	15.248018\\
3907	15.178185\\
3908	14.915364\\
3909	14.23777\\
3910	15.239346\\
3911	13.599626\\
3912	23.494407\\
3913	22.119313\\
3914	21.754771\\
3915	21.35414\\
3916	9.686392\\
3917	9.987853\\
3918	15.521763\\
3919	19.40082\\
3920	19.79145\\
3921	19.191003\\
3922	18.803785\\
3923	18.61322\\
3924	17.619743\\
3925	17.34481\\
3926	16.435327\\
3927	15.652748\\
3928	16.599205\\
3929	17.002507\\
3930	17.621636\\
3931	17.755736\\
3932	17.255387\\
3933	16.598714\\
3934	18.002565\\
3935	16.459263\\
3936	14.964807\\
3937	13.944478\\
3938	14.33126\\
3939	13.944476\\
3940	13.522306\\
3941	13.522306\\
3942	13.522307\\
3943	13.812533\\
3944	14.66578\\
3945	14.768627\\
3946	14.508328\\
3947	14.360672\\
3948	13.624155\\
3949	13.522307\\
3950	13.094849\\
3951	12.677963\\
3952	13.070543\\
3953	13.522307\\
3954	13.522306\\
3955	14.492731\\
3956	14.09453\\
3957	13.944481\\
3958	14.768627\\
3959	14.598331\\
3960	13.661267\\
3961	9.686391\\
3962	8.984165\\
3963	8.842019\\
3964	8.841564\\
3965	8.491816\\
3966	8.491803\\
3967	8.842051\\
3968	9.304478\\
3969	9.347928\\
3970	9.42039\\
3971	9.375116\\
3972	8.737267\\
3973	8.452943\\
3974	7.726996\\
3975	7.726997\\
3976	7.726996\\
3977	8.491795\\
3978	9.480009\\
3979	9.686397\\
3980	10.045868\\
3981	9.829182\\
3982	10.445271\\
3983	9.871057\\
3984	13.522306\\
3985	5.865475\\
3986	5.450251\\
3987	5.531166\\
3988	5.865485\\
3989	6.506394\\
3990	7.344935\\
3991	9.354144\\
3992	10.516988\\
3993	9.794109\\
3994	9.521644\\
3995	9.775636\\
3996	9.629735\\
3997	8.992188\\
3998	8.502573\\
3999	8.502572\\
4000	8.502573\\
4001	8.652208\\
};
\addplot [color=mycolor1,solid,line width=1.0pt,forget plot]
  table[row sep=crcr]{%
4001	8.652208\\
4002	9.319868\\
4003	8.655777\\
4004	8.502579\\
4005	8.306399\\
4006	8.502578\\
4007	8.30063\\
4008	7.060074\\
4009	7.044386\\
4010	6.794064\\
4011	6.794065\\
4012	6.973845\\
4013	7.060079\\
4014	8.300628\\
4015	9.675213\\
4016	11.959344\\
4017	13.041261\\
4018	9.714491\\
4019	11.156391\\
4020	7.62205\\
4021	6.68022\\
4022	6.68022\\
4023	6.856645\\
4024	6.955348\\
4025	7.291945\\
4026	6.68022\\
4027	7.127208\\
4028	7.478162\\
4029	7.467503\\
4030	7.883266\\
4031	7.687083\\
4032	6.440765\\
4033	5.596427\\
4034	5.440382\\
4035	5.386403\\
4036	5.596421\\
4037	6.058943\\
4038	6.862876\\
4039	8.416522\\
4040	8.623723\\
4041	8.906068\\
4042	8.004982\\
4043	7.979676\\
4044	8.031715\\
4045	6.722465\\
4046	6.085206\\
4047	5.974356\\
4048	6.314148\\
4049	6.798511\\
4050	6.994686\\
4051	6.994678\\
4052	6.798507\\
4053	6.48448\\
4054	6.994686\\
4055	6.798503\\
4056	5.552185\\
4057	5.286176\\
4058	4.707841\\
4059	4.707162\\
4060	4.780435\\
4061	5.332568\\
4062	6.257618\\
4063	7.565078\\
4064	9.138534\\
4065	8.748774\\
4066	8.465545\\
4067	7.302004\\
4068	6.798503\\
4069	6.650084\\
4070	6.798511\\
4071	6.798393\\
4072	6.994687\\
4073	8.095042\\
4074	9.119055\\
4075	7.738487\\
4076	6.798511\\
4077	6.43727\\
4078	6.875573\\
4079	6.192571\\
4080	5.552185\\
4081	5.286171\\
4082	4.614301\\
4083	4.513609\\
4084	4.513604\\
4085	4.707841\\
4086	5.552185\\
4087	6.792735\\
4088	6.994688\\
4089	7.090825\\
4090	7.415985\\
4091	7.090834\\
4092	6.798503\\
4093	6.376802\\
4094	6.798511\\
4095	6.798505\\
4096	6.994567\\
4097	7.090601\\
4098	7.090686\\
4099	7.337405\\
4100	7.124086\\
4101	6.994685\\
4102	7.96932\\
4103	7.091641\\
4104	5.974357\\
4105	5.552188\\
4106	5.284661\\
4107	4.70785\\
4108	4.513584\\
4109	4.357581\\
4110	4.357582\\
4111	5.124672\\
4112	7.335438\\
4113	7.241154\\
4114	7.681236\\
4115	6.584526\\
4116	6.258048\\
4117	6.258048\\
4118	5.992037\\
4119	5.992034\\
4120	6.128705\\
4121	6.258048\\
4122	6.258048\\
4123	6.68022\\
4124	6.68022\\
4125	6.68022\\
4126	7.122812\\
4127	7.074831\\
4128	6.258048\\
4129	5.413705\\
4130	5.063446\\
4131	4.648219\\
4132	4.79201\\
4133	4.298666\\
4134	4.298653\\
4135	4.931021\\
4136	5.092967\\
4137	5.413705\\
4138	5.413705\\
4139	5.413702\\
4140	5.063445\\
4141	4.298663\\
4142	4.298652\\
4143	4.298652\\
4144	4.596854\\
4145	5.413705\\
4146	5.992034\\
4147	6.258048\\
4148	6.258048\\
4149	6.618674\\
4150	7.498599\\
4151	6.68023\\
4152	6.257827\\
4153	5.413705\\
4154	5.063445\\
4155	4.978212\\
4156	5.063445\\
4157	5.413789\\
4158	6.680316\\
4159	8.967248\\
4160	9.724116\\
4161	10.03097\\
4162	8.81453\\
4163	8.542032\\
4164	7.863859\\
4165	7.749671\\
4166	7.611963\\
4167	7.700548\\
4168	8.164338\\
4169	9.097425\\
4170	10.388897\\
4171	9.935379\\
4172	8.928712\\
4173	7.70054\\
4174	9.151076\\
4175	9.207234\\
4176	7.458742\\
4177	6.258048\\
4178	6.253827\\
4179	5.995811\\
4180	6.209042\\
4181	6.258048\\
4182	7.124624\\
4183	8.419739\\
4184	9.017897\\
4185	9.357394\\
4186	10.397369\\
4187	11.722241\\
4188	10.536494\\
4189	11.486641\\
4190	10.539195\\
4191	9.647144\\
4192	9.406598\\
4193	10.119395\\
4194	7.94827\\
4195	8.478049\\
4196	9.628607\\
4197	8.468252\\
4198	10.494104\\
4199	8.93015\\
4200	7.68708\\
4201	6.57936\\
4202	5.721833\\
4203	5.596423\\
4204	5.958271\\
4205	6.174749\\
4206	6.80139\\
4207	8.171987\\
4208	8.021106\\
4209	7.979465\\
4210	7.883266\\
4211	7.687088\\
4212	7.664775\\
4213	7.882995\\
4214	7.687086\\
4215	7.687093\\
4216	7.834813\\
4217	8.146201\\
4218	8.355078\\
4219	8.526728\\
4220	8.128924\\
4221	7.883266\\
4222	9.075263\\
4223	8.003427\\
4224	6.862937\\
4225	6.440765\\
4226	6.174751\\
4227	6.14306\\
4228	6.440765\\
4229	6.440772\\
4230	7.687078\\
4231	7.948073\\
4232	8.440216\\
4233	9.453878\\
4234	9.851829\\
4235	8.869018\\
4236	8.357305\\
4237	7.966721\\
4238	7.883273\\
4239	7.883274\\
4240	8.534796\\
4241	9.460677\\
4242	9.613316\\
4243	7.883271\\
4244	7.883266\\
4245	7.8832\\
4246	9.161323\\
4247	8.666539\\
4248	7.373765\\
4249	6.440767\\
4250	6.291031\\
4251	6.013422\\
4252	6.145466\\
4253	6.380267\\
4254	6.977376\\
4255	7.883271\\
4256	8.306754\\
4257	9.981902\\
4258	12.341553\\
4259	11.036113\\
4260	11.290821\\
4261	11.999134\\
4262	9.146733\\
4263	6.792706\\
4264	6.798505\\
4265	6.823347\\
4266	6.798501\\
4267	6.605141\\
4268	6.264066\\
4269	6.207332\\
4270	6.732354\\
4271	5.974361\\
4272	5.124727\\
4273	4.357381\\
4274	3.592789\\
4275	3.592789\\
4276	3.592789\\
4277	3.592789\\
4278	3.592789\\
4279	3.655227\\
4280	4.638083\\
4281	5.286171\\
4282	5.552185\\
4283	5.552186\\
4284	5.552186\\
4285	5.549989\\
4286	5.286171\\
4287	5.49612\\
4288	5.552185\\
4289	5.552185\\
4290	5.974358\\
4291	5.552185\\
4292	5.413874\\
4293	5.552185\\
4294	5.552185\\
4295	5.552185\\
4296	4.681332\\
4297	3.795576\\
4298	3.592789\\
4299	3.592789\\
4300	3.592789\\
4301	4.481369\\
4302	4.481369\\
4303	4.481369\\
4304	3.592789\\
4305	3.592789\\
4306	3.592791\\
4307	4.322201\\
4308	5.860464\\
4309	5.402166\\
4310	5.107458\\
4311	4.482616\\
4312	4.481369\\
4313	5.221779\\
4314	5.596422\\
4315	6.174751\\
4316	5.603651\\
4317	6.440765\\
4318	6.066398\\
4319	5.625582\\
4320	5.401726\\
4321	4.830939\\
4322	4.481369\\
4323	4.481369\\
4324	4.481369\\
4325	4.857175\\
4326	6.247568\\
4327	6.919206\\
4328	7.526756\\
4329	7.687086\\
4330	7.687085\\
4331	7.883266\\
4332	7.687086\\
4333	7.49337\\
4334	6.862938\\
4335	6.862937\\
4336	7.313183\\
4337	7.687085\\
4338	8.317003\\
4339	8.11071\\
4340	8.475823\\
4341	7.700565\\
4342	8.516633\\
4343	7.700568\\
4344	7.105561\\
4345	6.570913\\
4346	6.090327\\
4347	5.925472\\
4348	6.057161\\
4349	6.32519\\
4350	7.291318\\
4351	10.050875\\
4352	10.596487\\
4353	9.893623\\
4354	8.549774\\
4355	7.638434\\
4356	7.291327\\
4357	7.291341\\
4358	7.105561\\
4359	7.105562\\
4360	7.245211\\
4361	7.615526\\
4362	8.671907\\
4363	6.27473\\
4364	6.452114\\
4365	5.269887\\
4366	5.728813\\
4367	5.230635\\
4368	4.700753\\
4369	4.579014\\
4370	3.726046\\
4371	3.637747\\
4372	3.637686\\
4373	3.901855\\
4374	4.700753\\
4375	5.875372\\
4376	5.930591\\
4377	6.000531\\
4378	5.880844\\
4379	5.859285\\
4380	5.04221\\
4381	4.930696\\
4382	4.700753\\
4383	4.832327\\
4384	5.100486\\
4385	5.76915\\
4386	6.036384\\
4387	6.066517\\
4388	6.274729\\
4389	6.066589\\
4390	6.27473\\
4391	5.875377\\
4392	4.700753\\
4393	4.600869\\
4394	3.896983\\
4395	3.63866\\
4396	3.648133\\
4397	3.788312\\
4398	4.700752\\
4399	5.346143\\
4400	6.06172\\
4401	5.880895\\
4402	5.746547\\
4403	5.346146\\
4404	4.769535\\
4405	4.700753\\
4406	4.700753\\
4407	4.700753\\
4408	4.700753\\
4409	5.247743\\
4410	5.494435\\
4411	5.346147\\
4412	5.345444\\
4413	5.277288\\
4414	5.346141\\
4415	5.100487\\
4416	4.659807\\
4417	3.755845\\
4418	3.637334\\
4419	3.466979\\
4420	3.637734\\
4421	3.806731\\
4422	4.700753\\
4423	5.875387\\
4424	6.066589\\
4425	6.066588\\
4426	6.066588\\
4427	6.274368\\
4428	6.066588\\
4429	6.066588\\
4430	6.022107\\
4431	5.880862\\
4432	5.727294\\
4433	5.880837\\
4434	6.066586\\
4435	5.880837\\
4436	5.812405\\
4437	5.235347\\
4438	5.346172\\
4439	4.700797\\
4440	4.137039\\
4441	3.607933\\
4442	3.400026\\
4443	3.39722\\
4444	3.392536\\
4445	3.401723\\
4446	3.28189\\
4447	3.400004\\
4448	3.392537\\
4449	3.401828\\
4450	3.401828\\
4451	3.401829\\
4452	3.401626\\
4453	3.401832\\
4454	3.399937\\
4455	3.40183\\
4456	3.399958\\
4457	3.39971\\
4458	3.394924\\
4459	3.40178\\
4460	3.40183\\
4461	3.401721\\
4462	3.706104\\
4463	3.401785\\
4464	3.40183\\
4465	3.401773\\
4466	3.303951\\
4467	2.404658\\
4468	1.987286\\
4469	1.987287\\
4470	1.987287\\
4471	2.262661\\
4472	3.050671\\
4473	3.401746\\
4474	3.397062\\
4475	3.39316\\
4476	3.398766\\
4477	3.399163\\
4478	3.401767\\
4479	3.401767\\
4480	3.399401\\
4481	3.40183\\
4482	3.753891\\
4483	3.901499\\
4484	3.718819\\
4485	3.749437\\
4486	3.665229\\
4487	3.432995\\
4488	3.401764\\
4489	3.39745\\
4490	2.904645\\
4491	2.856435\\
4492	3.238555\\
4493	3.397873\\
4494	3.637735\\
4495	4.832331\\
4496	5.193992\\
4497	5.346137\\
4498	5.174127\\
4499	5.34481\\
4500	4.832364\\
4501	5.100495\\
4502	5.346141\\
4503	5.367586\\
4504	5.880843\\
4505	5.864024\\
4506	6.043673\\
4507	5.880842\\
4508	6.143016\\
4509	6.066587\\
4510	6.274727\\
4511	5.880854\\
4512	4.700753\\
4513	4.519965\\
4514	3.789925\\
4515	3.638283\\
4516	3.638522\\
4517	3.901415\\
4518	4.700753\\
4519	5.880843\\
4520	7.79162\\
4521	9.436613\\
4522	9.777886\\
4523	10.414756\\
4524	9.633488\\
4525	9.555402\\
4526	9.464808\\
4527	8.804728\\
4528	8.239028\\
4529	8.807256\\
4530	8.254876\\
4531	7.217275\\
4532	6.420372\\
4533	6.106693\\
4534	6.380362\\
4535	5.888611\\
4536	4.700754\\
4537	3.901452\\
4538	3.646329\\
4539	3.401761\\
4540	3.401669\\
4541	3.401757\\
4542	3.901462\\
4543	4.930245\\
4544	5.880854\\
4545	6.009954\\
4546	6.274728\\
4547	7.610028\\
4548	7.116499\\
4549	6.875075\\
4550	6.66185\\
4551	6.274877\\
4552	6.274727\\
4553	6.258514\\
4554	6.066587\\
4555	5.346172\\
4556	4.832325\\
4557	4.701187\\
4558	5.04221\\
4559	4.700754\\
4560	3.637737\\
4561	3.401652\\
4562	3.399671\\
4563	3.397172\\
4564	3.399869\\
4565	3.401662\\
4566	3.901406\\
4567	4.700788\\
4568	5.880837\\
4569	6.066587\\
4570	5.880837\\
4571	5.880837\\
4572	6.274727\\
4573	6.066587\\
4574	5.880841\\
4575	5.880864\\
4576	5.594875\\
4577	5.880837\\
4578	5.880837\\
4579	5.880838\\
4580	5.100487\\
4581	5.3948\\
4582	5.880843\\
4583	5.102247\\
4584	4.486848\\
4585	3.64271\\
4586	3.492976\\
4587	3.401763\\
4588	3.401754\\
4589	3.672221\\
4590	4.700754\\
4591	5.346141\\
4592	5.346141\\
4593	5.208677\\
4594	5.346141\\
4595	6.274728\\
4596	6.274732\\
4597	5.880869\\
4598	5.806319\\
4599	5.346141\\
4600	5.346141\\
4601	5.875375\\
4602	5.875377\\
4603	5.346141\\
4604	5.571622\\
4605	5.346141\\
4606	5.875377\\
4607	5.881006\\
4608	5.100744\\
4609	4.70076\\
4610	3.901304\\
4611	3.865364\\
4612	3.638099\\
4613	3.592351\\
4614	3.540995\\
4615	3.901307\\
4616	4.700758\\
4617	5.100292\\
4618	5.346313\\
4619	5.346339\\
4620	5.13095\\
4621	4.701126\\
4622	4.700761\\
4623	4.296001\\
4624	4.624126\\
4625	4.701173\\
4626	4.81544\\
4627	5.209062\\
4628	5.042604\\
4629	5.042504\\
4630	5.346386\\
4631	4.701179\\
4632	4.578779\\
4633	3.645209\\
4634	3.401716\\
4635	3.401631\\
4636	3.400826\\
4637	3.392012\\
4638	3.399587\\
4639	3.398545\\
4640	3.401693\\
4641	3.638239\\
4642	3.901305\\
4643	4.368588\\
4644	4.320678\\
4645	3.637535\\
4646	3.401705\\
4647	3.401681\\
4648	3.401695\\
4649	3.437381\\
4650	3.901303\\
4651	4.701178\\
4652	4.701174\\
4653	4.701173\\
4654	4.701174\\
4655	4.701173\\
4656	3.926403\\
4657	3.399217\\
4658	3.401746\\
4659	3.401732\\
4660	3.401785\\
4661	3.401828\\
4662	3.756801\\
4663	4.700753\\
4664	5.84567\\
4665	5.346144\\
4666	7.136295\\
4667	7.291324\\
4668	7.301962\\
4669	7.105551\\
4670	6.809747\\
4671	6.433833\\
4672	6.686325\\
4673	7.105553\\
4674	7.291303\\
4675	7.105548\\
4676	7.100041\\
4677	6.57089\\
4678	7.105554\\
4679	6.704389\\
4680	5.925467\\
4681	5.126031\\
4682	4.243186\\
4683	4.243183\\
4684	4.243184\\
4685	4.269168\\
4686	5.63216\\
4687	5.925468\\
4688	6.394396\\
4689	6.570887\\
4690	7.287332\\
4691	7.499438\\
4692	7.597688\\
4693	7.499444\\
4694	7.375621\\
4695	7.499435\\
4696	7.428146\\
4697	7.948939\\
4698	8.16154\\
4699	7.291314\\
4700	7.288432\\
4701	7.291333\\
4702	7.596423\\
4703	9.631975\\
4704	7.881774\\
4705	7.161352\\
4706	6.647609\\
4707	6.51595\\
4708	6.515994\\
4709	7.157761\\
4710	7.806103\\
4711	10.005811\\
4712	11.557026\\
4713	11.718085\\
4714	11.516718\\
4715	10.61072\\
4716	8.709721\\
4717	8.089921\\
4718	7.881812\\
4719	8.08992\\
4720	8.089942\\
4721	8.520437\\
4722	9.990104\\
4723	9.442928\\
4724	7.291281\\
4725	7.291281\\
4726	8.367443\\
4727	7.499421\\
4728	6.570834\\
4729	5.925444\\
4730	5.925445\\
4731	5.731859\\
4732	5.756867\\
4733	5.925444\\
4734	6.433546\\
4735	7.291278\\
4736	8.355035\\
4737	7.537178\\
4738	7.499421\\
4739	7.291282\\
4740	7.291279\\
4741	7.105522\\
4742	7.100056\\
4743	7.105525\\
4744	7.124106\\
4745	7.499435\\
4746	7.934627\\
4747	7.499435\\
4748	7.105528\\
4749	7.105534\\
4750	7.499422\\
4751	6.9112\\
4752	5.925445\\
4753	5.125976\\
4754	4.401176\\
4755	4.22263\\
4756	4.264758\\
4757	5.07245\\
4758	5.925444\\
4759	6.325179\\
4760	7.105525\\
4761	7.105066\\
4762	7.100046\\
4763	7.105533\\
4764	6.325159\\
4765	5.925444\\
4766	5.925444\\
4767	5.925444\\
4768	5.968336\\
4769	6.570834\\
4770	6.995076\\
4771	7.105525\\
4772	7.105526\\
4773	7.100062\\
4774	7.291279\\
4775	6.570834\\
4776	5.925444\\
4777	3.401755\\
4778	3.399482\\
4779	3.401677\\
4780	3.294977\\
4781	2.590708\\
4782	2.306808\\
4783	2.90464\\
4784	3.401664\\
4785	3.401677\\
4786	3.401776\\
4787	2.42578\\
4788	2.271485\\
4789	2.293215\\
4790	2.241321\\
4791	2.362624\\
4792	3.401736\\
4793	3.40183\\
4794	3.669285\\
4795	4.219549\\
4796	4.589194\\
4797	3.901458\\
4798	3.901822\\
4799	3.901415\\
4800	3.401798\\
4801	3.401829\\
4802	3.395963\\
4803	3.392137\\
4804	3.401737\\
4805	3.401729\\
4806	3.218412\\
4807	3.401751\\
4808	3.391926\\
4809	3.401829\\
4810	3.401687\\
4811	3.401832\\
4812	3.395213\\
4813	3.396767\\
4814	3.401756\\
4815	3.132563\\
4816	3.272985\\
4817	3.401649\\
4818	3.401829\\
4819	3.401758\\
4820	3.473933\\
4821	3.629283\\
4822	3.648198\\
4823	3.401769\\
4824	3.40183\\
4825	3.401751\\
4826	3.059085\\
4827	2.529041\\
4828	2.306818\\
4829	2.258311\\
4830	2.238613\\
4831	2.512398\\
4832	2.632808\\
4833	3.401697\\
4834	3.392781\\
4835	3.401745\\
4836	3.401732\\
4837	3.394367\\
4838	3.401739\\
4839	3.358409\\
4840	3.246768\\
4841	3.401729\\
4842	3.396896\\
4843	3.398263\\
4844	3.399383\\
4845	3.399689\\
4846	3.401829\\
4847	3.397881\\
4848	3.159764\\
4849	2.293182\\
4850	1.987288\\
4851	1.987286\\
4852	1.987286\\
4853	2.243115\\
4854	3.401728\\
4855	3.401759\\
4856	3.901504\\
4857	4.700761\\
4858	4.700761\\
4859	4.578305\\
4860	3.845699\\
4861	3.63769\\
4862	3.401762\\
4863	3.401673\\
4864	3.401831\\
4865	3.710841\\
4866	3.90142\\
4867	3.954892\\
4868	4.387637\\
4869	3.901491\\
4870	4.503123\\
4871	3.915439\\
4872	3.401673\\
4873	3.394675\\
4874	3.401671\\
4875	3.401666\\
4876	3.401664\\
4877	3.401786\\
4878	3.40183\\
4879	3.821993\\
4880	4.648449\\
4881	4.545211\\
4882	4.040347\\
4883	3.901421\\
4884	3.401762\\
4885	3.401758\\
4886	3.401716\\
4887	3.401762\\
4888	3.428102\\
4889	3.647177\\
4890	3.847251\\
4891	3.90142\\
4892	3.901427\\
4893	3.901421\\
4894	4.700758\\
4895	3.901452\\
4896	3.401718\\
4897	3.394323\\
4898	3.401736\\
4899	3.40171\\
4900	3.40176\\
4901	3.396922\\
4902	3.401744\\
4903	4.058479\\
4904	4.365225\\
4905	3.901512\\
4906	3.650758\\
4907	3.637745\\
4908	3.637733\\
4909	3.401762\\
4910	3.401755\\
4911	3.401757\\
4912	3.402705\\
4913	3.720484\\
4914	3.901468\\
4915	3.901483\\
4916	4.083889\\
4917	4.300035\\
4918	4.700758\\
4919	4.422085\\
4920	3.576677\\
4921	3.401829\\
4922	3.397461\\
4923	3.396866\\
4924	3.398554\\
4925	3.40183\\
4926	3.401765\\
4927	3.82106\\
4928	4.700753\\
4929	4.700758\\
4930	4.674117\\
4931	4.387654\\
4932	3.994403\\
4933	3.901503\\
4934	3.901504\\
4935	3.901423\\
4936	3.89938\\
4937	3.901421\\
4938	3.94778\\
4939	4.289596\\
4940	3.974527\\
4941	4.218404\\
4942	4.700758\\
4943	3.901479\\
4944	3.648142\\
4945	3.40165\\
4946	3.398153\\
4947	3.394237\\
4948	3.401772\\
4949	3.401764\\
4950	3.40177\\
4951	3.399815\\
4952	3.401712\\
4953	3.638335\\
4954	3.650625\\
4955	3.65566\\
4956	3.53429\\
4957	3.40183\\
4958	3.398044\\
4959	3.397391\\
4960	3.399755\\
4961	3.401761\\
4962	3.637852\\
4963	3.887965\\
4964	3.901447\\
4965	3.901448\\
4966	4.249381\\
4967	3.863936\\
4968	3.401626\\
4969	3.39893\\
4970	3.401761\\
4971	3.401645\\
4972	3.093648\\
4973	3.275514\\
4974	2.90464\\
4975	3.401661\\
4976	3.40176\\
4977	3.391642\\
4978	3.401677\\
4979	3.062113\\
4980	2.483642\\
4981	2.241402\\
4982	2.127882\\
4983	2.241329\\
4984	2.530767\\
4985	3.401753\\
4986	3.401829\\
4987	3.637742\\
4988	3.719123\\
4989	3.901425\\
4990	3.901427\\
4991	3.647372\\
4992	3.401829\\
4993	3.399582\\
4994	3.401775\\
4995	3.401755\\
4996	3.40177\\
4997	3.399722\\
4998	3.401672\\
4999	4.229829\\
5000	4.700755\\
5001	4.832223\\
5002	4.873525\\
5003	4.832329\\
5004	4.832323\\
5005	4.700755\\
5006	4.700766\\
5007	4.700764\\
5008	4.700768\\
5009	4.700755\\
5010	4.700755\\
5011	4.553886\\
5012	3.901536\\
5013	3.901438\\
5014	4.700765\\
5015	3.758422\\
5016	3.401826\\
5017	3.396874\\
5018	3.401651\\
5019	3.24122\\
5020	3.351518\\
5021	3.391712\\
5022	3.400249\\
5023	3.640593\\
5024	4.205082\\
5025	3.901439\\
5026	3.901874\\
5027	3.901878\\
5028	3.901526\\
5029	3.80462\\
5030	3.637847\\
5031	3.55722\\
5032	3.639089\\
5033	3.901479\\
5034	4.199411\\
5035	3.901957\\
5036	4.271281\\
5037	4.217469\\
5038	4.700754\\
5039	4.417744\\
5040	3.637131\\
5041	3.401826\\
5042	3.398939\\
5043	3.396669\\
5044	3.396505\\
5045	3.399956\\
5046	3.401703\\
5047	4.30391\\
5048	4.700766\\
5049	4.700764\\
5050	4.408553\\
5051	4.013749\\
5052	3.758506\\
5053	3.642237\\
5054	3.734336\\
5055	3.782101\\
5056	3.901537\\
5057	4.700767\\
5058	4.70076\\
5059	4.700762\\
5060	4.700768\\
5061	4.700764\\
5062	4.849781\\
5063	4.320864\\
5064	3.638157\\
5065	3.401828\\
5066	3.396764\\
5067	3.40178\\
5068	3.394091\\
5069	3.3997\\
5070	3.401625\\
5071	4.063099\\
5072	4.700764\\
5073	4.70075\\
5074	4.700769\\
5075	4.295795\\
5076	4.298813\\
5077	3.901526\\
5078	3.901448\\
5079	3.901443\\
5080	3.901536\\
5081	3.901442\\
5082	4.294942\\
5083	4.235922\\
5084	4.700766\\
5085	4.700768\\
5086	4.963807\\
5087	4.578294\\
5088	3.593077\\
5089	3.590901\\
5090	3.38284\\
5091	3.244578\\
5092	3.221713\\
5093	3.591468\\
5094	3.593157\\
5095	3.685162\\
5096	4.04592\\
5097	3.686039\\
5098	3.639572\\
5099	3.600612\\
5100	3.60689\\
5101	3.600193\\
5102	3.596687\\
5103	3.59965\\
5104	3.6019\\
5105	3.639598\\
5106	3.77979\\
5107	4.144073\\
5108	4.144153\\
5109	4.144098\\
5110	4.563398\\
5111	3.948885\\
5112	3.60094\\
5113	3.593585\\
5114	3.575799\\
5115	2.823054\\
5116	2.450157\\
5117	2.43592\\
5118	2.380813\\
5119	2.951397\\
5120	3.59099\\
5121	3.592538\\
5122	3.593415\\
5123	3.593083\\
5124	3.592457\\
5125	3.591901\\
5126	3.582579\\
5127	3.591816\\
5128	3.594706\\
5129	3.627458\\
5130	3.642633\\
5131	3.644563\\
5132	3.692025\\
5133	3.773056\\
5134	3.928628\\
5135	3.639224\\
5136	3.592114\\
5137	3.598818\\
5138	2.497923\\
5139	2.435893\\
5140	2.425664\\
5141	2.435887\\
5142	2.429772\\
5143	2.6042\\
5144	3.240583\\
5145	2.435927\\
5146	2.435892\\
5147	2.382247\\
5148	2.110821\\
5149	1.905611\\
5150	1.004416\\
5151	0.921281\\
5152	1.819505\\
5153	2.290383\\
5154	2.46727\\
5155	3.591292\\
5156	3.591563\\
5157	3.590971\\
5158	3.592635\\
5159	3.578402\\
5160	3.577693\\
5161	2.948517\\
5162	2.381084\\
5163	2.380813\\
5164	2.435901\\
5165	3.280446\\
5166	4.323521\\
5167	5.44498\\
5168	6.294201\\
5169	6.294201\\
5170	6.294201\\
5171	4.863131\\
5172	4.527327\\
5173	4.144147\\
5174	3.685404\\
5175	3.688225\\
5176	3.685576\\
5177	4.144102\\
5178	4.144145\\
5179	4.144138\\
5180	4.527329\\
5181	4.527327\\
5182	4.993332\\
5183	4.144082\\
5184	3.604105\\
5185	3.593618\\
5186	3.590796\\
5187	3.578687\\
5188	3.576858\\
5189	3.59277\\
5190	3.607935\\
5191	4.144159\\
5192	4.527328\\
5193	4.240682\\
5194	4.144134\\
5195	3.686707\\
5196	4.993519\\
5197	4.993511\\
5198	4.99349\\
5199	4.993492\\
5200	4.993532\\
5201	5.844163\\
5202	6.246813\\
5203	6.444127\\
5204	6.444128\\
5205	6.444123\\
5206	7.417046\\
5207	6.195414\\
5208	4.993545\\
5209	4.145357\\
5210	3.651265\\
5211	3.637956\\
5212	3.570991\\
5213	3.666075\\
5214	4.145004\\
5215	4.99352\\
5216	5.41794\\
5217	5.417958\\
5218	5.417941\\
5219	5.417943\\
5220	4.993789\\
5221	4.99353\\
5222	4.993522\\
5223	5.006718\\
5224	5.417951\\
5225	6.165175\\
5226	6.241124\\
5227	5.417981\\
5228	5.292297\\
5229	5.356013\\
5230	5.706755\\
5231	4.993521\\
5232	4.527279\\
5233	3.82792\\
5234	3.653539\\
5235	3.638289\\
5236	3.666026\\
5237	4.145729\\
5238	4.993511\\
5239	5.916918\\
5240	6.259416\\
5241	8.067782\\
5242	7.468279\\
5243	6.777769\\
5244	6.775688\\
5245	6.444126\\
5246	6.428319\\
5247	6.246815\\
5248	6.246822\\
5249	6.246815\\
5250	6.353568\\
5251	6.246801\\
5252	6.444126\\
5253	6.524056\\
5254	6.744883\\
5255	6.14709\\
5256	4.99353\\
5257	4.527559\\
5258	4.145453\\
5259	3.930927\\
5260	3.948707\\
5261	4.527587\\
5262	4.993522\\
5263	6.246813\\
5264	7.150386\\
5265	7.763583\\
5266	7.000482\\
5267	6.985798\\
5268	6.446228\\
5269	6.55755\\
5270	6.444127\\
5271	6.822204\\
5272	6.986621\\
5273	6.444123\\
5274	6.444128\\
5275	6.444129\\
5276	6.21125\\
5277	5.417944\\
5278	5.417966\\
5279	4.993528\\
5280	4.042419\\
5281	3.602912\\
5282	3.57254\\
5283	3.574374\\
5284	3.604002\\
5285	3.600239\\
5286	3.602033\\
5287	3.574333\\
5288	3.572009\\
5289	3.638238\\
5290	3.638345\\
5291	3.638358\\
5292	3.572155\\
5293	3.574164\\
5294	3.574209\\
5295	3.574452\\
5296	3.573158\\
5297	3.570353\\
5298	3.722475\\
5299	4.528059\\
5300	4.144079\\
5301	4.527976\\
5302	4.883575\\
5303	4.860677\\
5304	3.948719\\
5305	3.63805\\
5306	3.572075\\
5307	3.57421\\
5308	3.57485\\
5309	3.5746\\
5310	3.603544\\
5311	3.60355\\
5312	3.574563\\
5313	3.572125\\
5314	3.571338\\
5315	3.638298\\
5316	3.635636\\
5317	3.574917\\
5318	3.601806\\
5319	3.343214\\
5320	3.598415\\
5321	3.604194\\
5322	3.573513\\
5323	3.57292\\
5324	3.572952\\
5325	3.572114\\
5326	3.602982\\
5327	3.572447\\
5328	3.240468\\
5329	2.435968\\
5330	2.11085\\
5331	2.11085\\
5332	2.399928\\
5333	3.579859\\
5334	3.571188\\
5335	3.684984\\
5336	4.565012\\
5337	4.1947\\
5338	4.057221\\
5339	3.954761\\
5340	3.818784\\
5341	3.67348\\
5342	3.665106\\
5343	3.665229\\
5344	3.666126\\
5345	3.766999\\
5346	4.14599\\
5347	4.52728\\
5348	4.52728\\
5349	4.527281\\
5350	4.993503\\
5351	4.14408\\
5352	3.569486\\
5353	3.571274\\
5354	3.57224\\
5355	3.573387\\
5356	3.57207\\
5357	3.571167\\
5358	3.72165\\
5359	4.857306\\
5360	4.993518\\
5361	4.993516\\
5362	4.881303\\
5363	4.993519\\
5364	4.52728\\
5365	4.52728\\
5366	4.145912\\
5367	4.527281\\
5368	4.86214\\
5369	4.993528\\
5370	5.347188\\
5371	4.993521\\
5372	4.99352\\
5373	5.133427\\
5374	5.533001\\
5375	4.993517\\
5376	4.144082\\
5377	3.638295\\
5378	3.570816\\
5379	3.571226\\
5380	3.570464\\
5381	3.665177\\
5382	4.667093\\
5383	5.034826\\
5384	6.246815\\
5385	6.246825\\
5386	6.442658\\
5387	6.444126\\
5388	6.246817\\
5389	5.974854\\
5390	5.533005\\
5391	5.417944\\
5392	5.417945\\
5393	5.439923\\
5394	5.345813\\
5395	6.088924\\
5396	5.829894\\
5397	6.081157\\
5398	6.246815\\
5399	4.993526\\
5400	4.52728\\
5401	3.948725\\
5402	3.665166\\
5403	3.666024\\
5404	3.665119\\
5405	4.344444\\
5406	4.993529\\
5407	6.339293\\
5408	7.732743\\
5409	8.533131\\
5410	10.177541\\
5411	10.212214\\
5412	8.885355\\
5413	8.261539\\
5414	7.623612\\
5415	7.57396\\
5416	7.735286\\
5417	7.753115\\
5418	8.770807\\
5419	7.857467\\
5420	7.336989\\
5421	7.590087\\
5422	7.130274\\
5423	6.83778\\
5424	5.584262\\
5425	5.017386\\
5426	4.539722\\
5427	4.258708\\
5428	4.189654\\
5429	4.189655\\
5430	3.965245\\
5431	4.189656\\
5432	4.460072\\
5433	4.735038\\
5434	4.795693\\
5435	4.735038\\
5436	4.413323\\
5437	3.964352\\
5438	3.613548\\
5439	3.613549\\
5440	3.613549\\
5441	3.613548\\
5442	4.189656\\
5443	4.735028\\
5444	5.118245\\
5445	5.584261\\
5446	5.584264\\
5447	5.348629\\
5448	4.356779\\
5449	3.613566\\
5450	3.613549\\
5451	3.61355\\
5452	3.61355\\
5453	3.613548\\
5454	3.613547\\
5455	4.189655\\
5456	4.822071\\
5457	5.118245\\
5458	5.118247\\
5459	5.118245\\
5460	4.735039\\
5461	4.189657\\
5462	3.618171\\
5463	3.613549\\
5464	3.86824\\
5465	4.294328\\
5466	5.118245\\
5467	5.27291\\
5468	4.735039\\
5469	4.735039\\
5470	5.036176\\
5471	4.529379\\
5472	3.613549\\
5473	3.613544\\
5474	3.400379\\
5475	2.762244\\
5476	2.440824\\
5477	2.898832\\
5478	2.53455\\
5479	2.939391\\
5480	3.351957\\
5481	3.613544\\
5482	3.613546\\
5483	3.613547\\
5484	3.613547\\
5485	3.613543\\
5486	3.407128\\
5487	3.312072\\
5488	3.613544\\
5489	3.613549\\
5490	3.613548\\
5491	4.086161\\
5492	4.170219\\
5493	4.189657\\
5494	4.110294\\
5495	3.844945\\
5496	3.613548\\
5497	3.613543\\
5498	3.295651\\
5499	3.167316\\
5500	3.337066\\
5501	3.613548\\
5502	4.520895\\
5503	5.584262\\
5504	5.864046\\
5505	5.584263\\
5506	5.584264\\
5507	5.584265\\
5508	5.584263\\
5509	5.584264\\
5510	5.584264\\
5511	5.916499\\
5512	6.124321\\
5513	7.025958\\
5514	7.406406\\
5515	8.492537\\
5516	8.755354\\
5517	8.473613\\
5518	8.64623\\
5519	7.458792\\
5520	6.008871\\
5521	5.584262\\
5522	5.118245\\
5523	4.738199\\
5524	4.774463\\
5525	5.456209\\
5526	6.168575\\
5527	7.035094\\
5528	8.722359\\
5529	9.12996\\
5530	8.578452\\
5531	7.988319\\
5532	7.035098\\
5533	6.418481\\
5534	6.837778\\
5535	6.817879\\
5536	6.837779\\
5537	6.837787\\
5538	8.566082\\
5539	8.954982\\
5540	9.817785\\
5541	9.223844\\
5542	9.32126\\
5543	7.307531\\
5544	6.731793\\
5545	5.584265\\
5546	5.58426\\
5547	5.527985\\
5548	5.584224\\
5549	5.747844\\
5550	7.251014\\
5551	10.171642\\
5552	10.36548\\
5553	10.548074\\
5554	9.648765\\
5555	9.475844\\
5556	7.835765\\
5557	7.781119\\
5558	7.084561\\
5559	7.915846\\
5560	8.791617\\
5561	10.667323\\
5562	10.341042\\
5563	10.494921\\
5564	10.564046\\
5565	11.561904\\
5566	11.252857\\
5567	8.990352\\
5568	6.831978\\
5569	5.584622\\
5570	5.584263\\
5571	5.584261\\
5572	5.584262\\
5573	5.723908\\
5574	7.162873\\
5575	10.132022\\
5576	10.896257\\
5577	8.875077\\
5578	7.976394\\
5579	7.883421\\
5580	8.595624\\
5581	7.586322\\
5582	8.548078\\
5583	8.203806\\
5584	8.237902\\
5585	8.037114\\
5586	9.989198\\
5587	8.789434\\
5588	8.564159\\
5589	7.601299\\
5590	7.552776\\
5591	6.837794\\
5592	5.584262\\
5593	5.044906\\
5594	4.608176\\
5595	4.189657\\
5596	4.189658\\
5597	5.118245\\
5598	5.80556\\
5599	6.837773\\
5600	8.566633\\
5601	7.134753\\
5602	7.183112\\
5603	7.69823\\
5604	7.035096\\
5605	7.869144\\
5606	7.246639\\
5607	7.299318\\
5608	7.136565\\
5609	7.437294\\
5610	7.496618\\
5611	8.217155\\
5612	9.371694\\
5613	9.964329\\
5614	10.065897\\
5615	7.917803\\
5616	6.009451\\
5617	5.584263\\
5618	6.831979\\
5619	6.083462\\
5620	5.796522\\
5621	5.756758\\
5622	6.00887\\
5623	6.573303\\
5624	6.837784\\
5625	6.357363\\
5626	6.330995\\
5627	5.963581\\
5628	5.592583\\
5629	5.58426\\
5630	5.58426\\
5631	5.558469\\
5632	5.58426\\
5633	6.035842\\
5634	6.837782\\
5635	7.729391\\
5636	9.04519\\
5637	8.756428\\
5638	8.002898\\
5639	7.558017\\
5640	6.697439\\
5641	4.379838\\
5642	3.613548\\
5643	3.61355\\
5644	3.613551\\
5645	3.61355\\
5646	3.613548\\
5647	4.189655\\
5648	4.189659\\
5649	4.203941\\
5650	4.189657\\
5651	4.330704\\
5652	4.189658\\
5653	3.613549\\
5654	3.613551\\
5655	3.61355\\
5656	3.613549\\
5657	4.189658\\
5658	5.584155\\
5659	5.724113\\
5660	6.621659\\
5661	7.035096\\
5662	6.837783\\
5663	6.272516\\
5664	5.584258\\
5665	4.757167\\
5666	4.49523\\
5667	4.249031\\
5668	4.539792\\
5669	5.577276\\
5670	6.837778\\
5671	7.819376\\
5672	10.317778\\
5673	11.108457\\
5674	11.699993\\
5675	11.953902\\
5676	12.747116\\
5677	12.846652\\
5678	12.846649\\
5679	13.238993\\
5680	13.629227\\
5681	13.087643\\
5682	12.846659\\
5683	11.737808\\
5684	10.896571\\
5685	9.290769\\
5686	8.920773\\
5687	7.035095\\
5688	5.848009\\
5689	5.584265\\
5690	5.291501\\
5691	5.358635\\
5692	5.584262\\
5693	5.587328\\
5694	7.928145\\
5695	10.896254\\
5696	12.485702\\
5697	14.227259\\
5698	14.289264\\
5699	14.289264\\
5700	14.28924\\
5701	13.890562\\
5702	13.267831\\
5703	14.150299\\
5704	14.010079\\
5705	14.1436\\
5706	13.035173\\
5707	12.846601\\
5708	11.886005\\
5709	11.29374\\
5710	10.846315\\
5711	8.864783\\
5712	6.630793\\
5713	5.58426\\
5714	5.454398\\
5715	5.394317\\
5716	5.58426\\
5717	5.807746\\
5718	8.083173\\
5719	10.381671\\
5720	12.471441\\
5721	12.18636\\
5722	11.188012\\
5723	10.774834\\
5724	10.014722\\
5725	10.185061\\
5726	9.905654\\
5727	10.272274\\
5728	10.572454\\
5729	10.896245\\
5730	12.083709\\
5731	11.885904\\
5732	11.207232\\
5733	11.542372\\
5734	11.489856\\
5735	8.792627\\
5736	6.635886\\
5737	5.58426\\
5738	5.39564\\
5739	5.118232\\
5740	5.118246\\
5741	5.58426\\
5742	7.035094\\
5743	9.866614\\
5744	11.471207\\
5745	11.8697\\
5746	12.472566\\
5747	12.435507\\
5748	11.432821\\
5749	11.26142\\
5750	10.932989\\
5751	11.636263\\
5752	11.058331\\
5753	10.834186\\
5754	10.646462\\
5755	10.222683\\
5756	9.810584\\
5757	9.250866\\
5758	8.176922\\
5759	6.444129\\
5760	4.993493\\
5761	4.70776\\
5762	4.144081\\
5763	4.144083\\
5764	4.144079\\
5765	4.993534\\
5766	6.246815\\
5767	7.974494\\
5768	9.668441\\
5769	8.838711\\
5770	7.667029\\
5771	7.060232\\
5772	6.444127\\
5773	6.241027\\
5774	6.021534\\
5775	5.180399\\
5776	5.178069\\
5777	5.642819\\
5778	6.246815\\
5779	6.246815\\
5780	6.246826\\
5781	6.241028\\
5782	6.241021\\
5783	5.148496\\
5784	4.145786\\
5785	3.66598\\
5786	4.859623\\
5787	4.527308\\
5788	4.258773\\
5789	4.144118\\
5790	4.637008\\
5791	4.993294\\
5792	5.417969\\
5793	6.44413\\
5794	7.656123\\
5795	8.620964\\
5796	8.203671\\
5797	7.646326\\
5798	7.315228\\
5799	7.300196\\
5800	7.565046\\
5801	8.971778\\
5802	9.929349\\
5803	10.305282\\
5804	9.650569\\
5805	9.084733\\
5806	7.985772\\
5807	6.840913\\
5808	6.136878\\
5809	4.993292\\
5810	4.993302\\
5811	4.89232\\
5812	4.863587\\
5813	4.922471\\
5814	4.993301\\
5815	4.993294\\
5816	5.417999\\
5817	6.24683\\
5818	6.707455\\
5819	7.469461\\
5820	6.703419\\
5821	6.24683\\
5822	4.993291\\
5823	4.993301\\
5824	4.993294\\
5825	5.533034\\
5826	6.540163\\
5827	9.5665\\
5828	9.467429\\
5829	9.517184\\
5830	9.963264\\
5831	8.502701\\
5832	5.640702\\
5833	4.662008\\
5834	4.304711\\
5835	4.30471\\
5836	4.304701\\
5837	5.689137\\
5838	8.145631\\
5839	10.32681\\
5840	10.545585\\
5841	9.672679\\
5842	9.141871\\
5843	8.649833\\
5844	8.380825\\
5845	8.381246\\
5846	8.652308\\
5847	8.995612\\
5848	9.532679\\
5849	10.258634\\
5850	10.878093\\
5851	11.920475\\
5852	11.078155\\
5853	10.49039\\
5854	10.149479\\
5855	8.145632\\
5856	6.894069\\
5857	6.652352\\
5858	5.815145\\
5859	5.640702\\
5860	5.745563\\
5861	6.652353\\
5862	8.222732\\
5863	12.980354\\
5864	14.054218\\
5865	11.440064\\
5866	9.883878\\
5867	9.483622\\
5868	8.609595\\
5869	8.420714\\
5870	8.526411\\
5871	8.386668\\
5872	9.032374\\
5873	9.882937\\
5874	10.623586\\
5875	10.580582\\
5876	10.586247\\
5877	10.689934\\
5878	10.08383\\
5879	8.23265\\
5880	6.792904\\
5881	6.652352\\
5882	6.092636\\
5883	5.640702\\
5884	6.140543\\
5885	6.652353\\
5886	8.380685\\
5887	12.980359\\
5888	14.054226\\
5889	11.367225\\
5890	10.204707\\
5891	8.932704\\
5892	8.145632\\
5893	7.947282\\
5894	7.372976\\
5895	7.518946\\
5896	8.145634\\
5897	9.081795\\
5898	9.524776\\
5899	9.40324\\
5900	10.995394\\
5901	11.029005\\
5902	9.371891\\
5903	7.158178\\
5904	6.652353\\
5905	5.786864\\
5906	5.683882\\
5907	5.640702\\
5908	5.641349\\
5909	6.652354\\
5910	8.380685\\
5911	14.054228\\
5912	14.240307\\
5913	12.980354\\
5914	11.057379\\
5915	10.483608\\
5916	8.690233\\
5917	8.380686\\
5918	8.380685\\
5919	8.426812\\
5920	9.981573\\
5921	10.941139\\
5922	17.022347\\
5923	17.022346\\
5924	17.022348\\
5925	16.38229\\
5926	11.681643\\
5927	8.49184\\
5928	7.329782\\
5929	6.652354\\
5930	6.652353\\
5931	6.652349\\
5932	6.652353\\
5933	6.652354\\
5934	8.453497\\
5935	17.022348\\
5936	21.629869\\
5937	21.684814\\
5938	17.731991\\
5939	17.022348\\
5940	10.973088\\
5941	10.461222\\
5942	9.992064\\
5943	9.458918\\
5944	10.075026\\
5945	11.00813\\
5946	11.431046\\
5947	10.540899\\
5948	11.228636\\
5949	10.695615\\
5950	9.097365\\
5951	9.880065\\
5952	11.029842\\
5953	9.712767\\
5954	8.411833\\
5955	8.245299\\
5956	8.245299\\
5957	8.245299\\
5958	8.751087\\
5959	9.345939\\
5960	9.975518\\
5961	11.249007\\
5962	13.590539\\
5963	14.492719\\
5964	13.908956\\
5965	12.568371\\
5966	12.02948\\
5967	11.486331\\
5968	11.806079\\
5969	13.793762\\
5970	13.366292\\
5971	13.40833\\
5972	14.45395\\
5973	13.550703\\
5974	13.410799\\
5975	12.266239\\
5976	10.671572\\
5977	9.767924\\
5978	9.738578\\
5979	9.738572\\
5980	9.731664\\
5981	9.738568\\
5982	9.973632\\
5983	9.973631\\
5984	10.483277\\
5985	10.912072\\
5986	11.502626\\
5987	11.674296\\
5988	10.794903\\
5989	9.785841\\
5990	9.731664\\
5991	9.738573\\
5992	8.755037\\
5993	7.657907\\
5994	8.99142\\
5995	9.430075\\
5996	10.597634\\
5997	10.62077\\
5998	9.93017\\
5999	8.886986\\
6000	7.579261\\
6001	7.529694\\
6002	7.20441\\
6003	6.995208\\
6004	7.579664\\
6005	7.930258\\
6006	11.312488\\
6007	17.868118\\
6008	22.475634\\
6009	17.868085\\
6010	17.868098\\
6011	17.86808\\
6012	13.826045\\
6013	12.139561\\
6014	11.497083\\
6015	11.211443\\
6016	12.580404\\
6017	16.149548\\
6018	17.86808\\
6019	17.86808\\
6020	13.826111\\
6021	13.042083\\
6022	11.917003\\
6023	10.163438\\
6024	8.750078\\
6025	7.633668\\
6026	7.579395\\
6027	7.579423\\
6028	7.579409\\
6029	8.026183\\
6030	11.15836\\
6031	14.899957\\
6032	14.899956\\
6033	12.093608\\
6034	13.004918\\
6035	13.826085\\
6036	11.671596\\
6037	11.201984\\
6038	11.527648\\
6039	11.696385\\
6040	12.647015\\
6041	16.14954\\
6042	17.868079\\
6043	13.287837\\
6044	13.826086\\
6045	12.657342\\
6046	12.263242\\
6047	9.286801\\
6048	8.071606\\
6049	7.579009\\
6050	7.579664\\
6051	7.578814\\
6052	7.57978\\
6053	7.760177\\
6054	10.329326\\
6055	17.868082\\
6056	17.868082\\
6057	14.899957\\
6058	12.450343\\
6059	11.877044\\
6060	10.894154\\
6061	10.162713\\
6062	9.714267\\
6063	10.383444\\
6064	11.679722\\
6065	11.564251\\
6066	16.149559\\
6067	12.563282\\
6068	13.826096\\
6069	14.899957\\
6070	11.094669\\
6071	9.226459\\
6072	8.001655\\
6073	7.579361\\
6074	7.574503\\
6075	7.533949\\
6076	7.579592\\
6077	7.918994\\
6078	11.156452\\
6079	16.149556\\
6080	17.82816\\
6081	17.86808\\
6082	17.868079\\
6083	17.868079\\
6084	12.714696\\
6085	10.984999\\
6086	11.695215\\
6087	11.070573\\
6088	11.366894\\
6089	13.826085\\
6090	14.899956\\
6091	12.518875\\
6092	12.661619\\
6093	12.572875\\
6094	10.917694\\
6095	9.226469\\
6096	7.678587\\
6097	7.579635\\
6098	7.382229\\
6099	7.222794\\
6100	7.536544\\
6101	7.580744\\
6102	10.26071\\
6103	12.705064\\
6104	16.706039\\
6105	13.260807\\
6106	11.935255\\
6107	9.259417\\
6108	8.991395\\
6109	8.984452\\
6110	8.877986\\
6111	8.346142\\
6112	8.991402\\
6113	9.718545\\
6114	10.934428\\
6115	11.462633\\
6116	12.830937\\
6117	11.164758\\
6118	11.506979\\
6119	9.973631\\
6120	8.410033\\
6121	8.244757\\
6122	7.524987\\
6123	7.233607\\
6124	6.915917\\
6125	7.198463\\
6126	8.013413\\
6127	8.245299\\
6128	9.220326\\
6129	9.731666\\
6130	9.292585\\
6131	8.751116\\
6132	8.245299\\
6133	8.24507\\
6134	7.223903\\
6135	6.962617\\
6136	7.233648\\
6137	8.245299\\
6138	8.411795\\
6139	8.751123\\
6140	8.751136\\
6141	8.416414\\
6142	8.411819\\
6143	8.245299\\
6144	7.376048\\
6145	7.098775\\
6146	6.314774\\
6147	5.897647\\
6148	5.897648\\
6149	6.055071\\
6150	7.002168\\
6151	7.23365\\
6152	8.245299\\
6153	8.751122\\
6154	9.241461\\
6155	9.709127\\
6156	8.222831\\
6157	7.717005\\
6158	7.23317\\
6159	6.705354\\
6160	6.705356\\
6161	7.717005\\
6162	7.717005\\
6163	8.188898\\
6164	9.179315\\
6165	8.322799\\
6166	8.222832\\
6167	7.717005\\
6168	6.705355\\
6169	6.109227\\
6170	5.772571\\
6171	5.550802\\
6172	5.935161\\
6173	7.616618\\
6174	9.445338\\
6175	13.143787\\
6176	16.368457\\
6177	18.086996\\
6178	17.446941\\
6179	15.118877\\
6180	11.635534\\
6181	11.183117\\
6182	11.307744\\
6183	11.555999\\
6184	12.325469\\
6185	14.045007\\
6186	18.087\\
6187	18.086993\\
6188	18.087\\
6189	14.045007\\
6190	11.890107\\
6191	9.445338\\
6192	7.883577\\
6193	7.717005\\
6194	7.562049\\
6195	7.59647\\
6196	7.717005\\
6197	8.148939\\
6198	11.494953\\
6199	18.086997\\
6200	22.694506\\
6201	18.086997\\
6202	15.118877\\
6203	14.045007\\
6204	11.245012\\
6205	11.212761\\
6206	10.698417\\
6207	11.046459\\
6208	12.83675\\
6209	18.086978\\
6210	18.087001\\
6211	18.086998\\
6212	22.694521\\
6213	18.08699\\
6214	12.772489\\
6215	10.163502\\
6216	8.153868\\
6217	7.717005\\
6218	7.502779\\
6219	7.562029\\
6220	7.717003\\
6221	7.722373\\
6222	10.875447\\
6223	16.368467\\
6224	16.368456\\
6225	12.928157\\
6226	12.748925\\
6227	12.612683\\
6228	10.39648\\
6229	9.740252\\
6230	9.626629\\
6231	10.253463\\
6232	10.53108\\
6233	11.876698\\
6234	15.304961\\
6235	15.304962\\
6236	18.086992\\
6237	12.851554\\
6238	11.033428\\
6239	9.210285\\
6240	7.962883\\
6241	7.495402\\
6242	6.702686\\
6243	6.689577\\
6244	7.151806\\
6245	7.498083\\
6246	9.9089\\
6247	17.868072\\
6248	22.4756\\
6249	22.530544\\
6250	22.530545\\
6251	38.484788\\
6252	22.530545\\
6253	22.4756\\
6254	22.475607\\
6255	17.868079\\
6256	17.868079\\
6257	22.530547\\
6258	22.53056\\
6259	17.86808\\
6260	22.530544\\
6261	15.08604\\
6262	11.665727\\
6263	9.371231\\
6264	8.359867\\
6265	7.498085\\
6266	7.498083\\
6267	7.490008\\
6268	7.498083\\
6269	7.911336\\
6270	10.996583\\
6271	22.47553\\
6272	22.530543\\
6273	22.530544\\
6274	22.530544\\
6275	22.475601\\
6276	17.868079\\
6277	16.149537\\
6278	17.868071\\
6279	16.149541\\
6280	17.867783\\
6281	17.868083\\
6282	17.868081\\
6283	13.826085\\
6284	17.868081\\
6285	12.364942\\
6286	11.780094\\
6287	9.920214\\
6288	8.137849\\
6289	7.498083\\
6290	6.916303\\
6291	6.486433\\
6292	6.351762\\
6293	6.480194\\
6294	6.913928\\
6295	7.498083\\
6296	8.142128\\
6297	9.226416\\
6298	8.991363\\
6299	8.98445\\
6300	8.53333\\
6301	8.402857\\
6302	8.003366\\
6303	7.94716\\
6304	8.003922\\
6305	8.735414\\
6306	8.991368\\
6307	8.991363\\
6308	9.505007\\
6309	8.961548\\
6310	8.57187\\
6311	7.498083\\
6312	6.986202\\
6313	6.124486\\
6314	5.150434\\
6315	5.150432\\
6316	5.150438\\
6317	5.150434\\
6318	5.150438\\
6319	5.150432\\
6320	5.427057\\
6321	6.824233\\
6322	7.498083\\
6323	7.493623\\
6324	6.283228\\
6325	5.150432\\
6326	5.150432\\
6327	5.150447\\
6328	5.150432\\
6329	5.150432\\
6330	6.981718\\
6331	7.498083\\
6332	7.520965\\
6333	7.498083\\
6334	7.498083\\
6335	6.643151\\
6336	7.233645\\
6337	6.31648\\
6338	5.897683\\
6339	5.897647\\
6340	5.897647\\
6341	7.233647\\
6342	9.738564\\
6343	12.319983\\
6344	14.5733\\
6345	16.896755\\
6346	14.5733\\
6347	13.636252\\
6348	11.651496\\
6349	11.392017\\
6350	11.310482\\
6351	11.344686\\
6352	12.079105\\
6353	11.667235\\
6354	13.826087\\
6355	14.89996\\
6356	16.149535\\
6357	12.856558\\
6358	11.738744\\
6359	9.226415\\
6360	8.004062\\
6361	7.498083\\
6362	7.34339\\
6363	6.981334\\
6364	7.498082\\
6365	7.688141\\
6366	11.259538\\
6367	22.4756\\
6368	22.475602\\
6369	17.868079\\
6370	17.228023\\
6371	13.826085\\
6372	11.359947\\
6373	10.737366\\
6374	10.733689\\
6375	11.3477\\
6376	12.66959\\
6377	17.868072\\
6378	17.868079\\
6379	22.4756\\
6380	22.530545\\
6381	16.149542\\
6382	12.436393\\
6383	9.597942\\
6384	8.003901\\
6385	7.498083\\
6386	6.486527\\
6387	6.484417\\
6388	6.486432\\
6389	7.498083\\
6390	9.226416\\
6391	12.759965\\
6392	14.899944\\
6393	15.086038\\
6394	12.997684\\
6395	17.868077\\
6396	13.826222\\
6397	13.826086\\
6398	12.718957\\
6399	12.581248\\
6400	12.473828\\
6401	16.14954\\
6402	12.872912\\
6403	12.554302\\
6404	13.826084\\
6405	11.034893\\
6406	9.969431\\
6407	8.813688\\
6408	7.498083\\
6409	6.919475\\
6410	6.486435\\
6411	6.486432\\
6412	6.486441\\
6413	7.498083\\
6414	9.966218\\
6415	17.868079\\
6416	17.868082\\
6417	17.868079\\
6418	15.086039\\
6419	12.87448\\
6420	10.387615\\
6421	9.722329\\
6422	9.226417\\
6423	9.226416\\
6424	9.519726\\
6425	10.508403\\
6426	12.159167\\
6427	12.549235\\
6428	14.275516\\
6429	11.168518\\
6430	9.226416\\
6431	8.456719\\
6432	7.664582\\
6433	7.498083\\
6434	6.648348\\
6435	6.486432\\
6436	6.63346\\
6437	7.498083\\
6438	9.622529\\
6439	17.868077\\
6440	17.868079\\
6441	17.868079\\
6442	16.149613\\
6443	17.840724\\
6444	17.868079\\
6445	16.14954\\
6446	17.868077\\
6447	17.868079\\
6448	22.475608\\
6449	22.530544\\
6450	22.530546\\
6451	22.530548\\
6452	22.530533\\
6453	16.14954\\
6454	15.086038\\
6455	11.552676\\
6456	9.226416\\
6457	13.334126\\
6458	12.816716\\
6459	12.11159\\
6460	12.078845\\
6461	12.111585\\
6462	13.092211\\
6463	13.825364\\
6464	15.721159\\
6465	17.816904\\
6466	20.257216\\
6467	17.933757\\
6468	15.223861\\
6469	14.115699\\
6470	13.334121\\
6471	13.245628\\
6472	13.494805\\
6473	15.575092\\
6474	16.593204\\
6475	17.933756\\
6476	20.257211\\
6477	16.95232\\
6478	15.562914\\
6479	13.704355\\
6480	12.533354\\
6481	7.962131\\
6482	7.19751\\
6483	6.575098\\
6484	6.359315\\
6485	6.668388\\
6486	7.278072\\
6487	8.184388\\
6488	8.245299\\
6489	8.245299\\
6490	8.245299\\
6491	8.245299\\
6492	8.089679\\
6493	6.877994\\
6494	6.249789\\
6495	6.31648\\
6496	7.038521\\
6497	8.245299\\
6498	9.973632\\
6499	17.93376\\
6500	21.975872\\
6501	11.084411\\
6502	9.973575\\
6503	8.751124\\
6504	8.245299\\
6505	7.235043\\
6506	6.896669\\
6507	6.853073\\
6508	7.233153\\
6509	8.245299\\
6510	10.541828\\
6511	18.615295\\
6512	18.615295\\
6513	16.896765\\
6514	14.573301\\
6515	14.573301\\
6516	12.513299\\
6517	12.654781\\
6518	12.984204\\
6519	18.615295\\
6520	23.27776\\
6521	145.217096\\
6522	189.961798\\
6523	145.217116\\
6524	24.244632\\
6525	23.277759\\
6526	16.896759\\
6527	11.895696\\
6528	9.738574\\
6529	8.355426\\
6530	8.245299\\
6531	8.245299\\
6532	8.245299\\
6533	8.677358\\
6534	21.975749\\
6535	49.338843\\
6536	190.7038\\
6537	190.703828\\
6538	39.243108\\
6539	27.12977\\
6540	26.638213\\
6541	26.638214\\
6542	18.615295\\
6543	18.615295\\
6544	18.615296\\
6545	24.244459\\
6546	39.232\\
6547	159.131475\\
6548	145.217116\\
6549	23.277767\\
6550	18.615295\\
6551	12.854206\\
6552	9.43013\\
6553	8.491792\\
6554	8.435123\\
6555	7.959598\\
6556	8.412563\\
6557	8.492555\\
6558	11.748285\\
6559	23.973647\\
6560	22.229267\\
6561	19.171798\\
6562	16.306602\\
6563	19.171797\\
6564	11.910437\\
6565	12.005797\\
6566	11.894952\\
6567	12.116825\\
6568	15.008969\\
6569	22.229267\\
6570	23.973647\\
6571	23.973647\\
6572	22.775767\\
6573	23.973647\\
6574	19.171798\\
6575	11.813162\\
6576	10.029715\\
6577	8.936803\\
6578	8.491792\\
6579	8.491797\\
6580	8.491798\\
6581	9.43614\\
6582	17.401883\\
6583	78.749884\\
6584	144.626093\\
6585	144.626071\\
6586	39.690301\\
6587	40.404839\\
6588	24.913421\\
6589	23.973647\\
6590	24.913427\\
6591	28.516773\\
6592	36.360082\\
6593	144.626081\\
6594	129.856998\\
6595	58.986872\\
6596	36.360081\\
6597	23.973647\\
6598	22.229267\\
6599	12.000331\\
6600	10.029713\\
6601	9.094527\\
6602	8.522976\\
6603	8.491798\\
6604	8.491792\\
6605	9.635745\\
6606	16.306591\\
6607	164.342673\\
6608	36.360081\\
6609	23.973647\\
6610	22.229267\\
6611	22.775766\\
6612	15.008969\\
6613	12.344236\\
6614	12.274242\\
6615	12.250017\\
6616	17.401882\\
6617	22.229267\\
6618	23.973647\\
6619	23.973647\\
6620	23.973647\\
6621	22.229267\\
6622	17.401911\\
6623	11.460607\\
6624	9.155058\\
6625	8.491785\\
6626	7.493216\\
6627	7.209271\\
6628	6.07586\\
6629	6.073941\\
6630	6.809894\\
6631	7.937764\\
6632	8.411696\\
6633	8.2089\\
6634	7.788518\\
6635	7.476668\\
6636	7.182996\\
6637	6.697816\\
6638	6.505292\\
6639	7.449782\\
6640	8.306548\\
6641	8.788126\\
6642	9.446028\\
6643	9.139923\\
6644	9.541477\\
6645	9.012738\\
6646	8.491786\\
6647	8.491771\\
6648	7.179034\\
6649	6.073947\\
6650	6.073939\\
6651	6.073938\\
6652	6.073938\\
6653	6.073944\\
6654	6.591118\\
6655	7.658999\\
6656	8.49179\\
6657	8.491792\\
6658	9.153165\\
6659	9.94387\\
6660	9.913031\\
6661	9.012738\\
6662	8.491791\\
6663	8.491791\\
6664	9.012705\\
6665	10.029676\\
6666	11.91186\\
6667	11.773916\\
6668	11.354754\\
6669	11.855366\\
6670	10.485951\\
6671	9.939884\\
6672	8.49179\\
6673	7.849768\\
6674	7.313034\\
6675	7.285128\\
6676	7.449897\\
6677	8.49179\\
6678	11.443374\\
6679	22.229184\\
6680	19.513736\\
6681	22.229268\\
6682	19.171798\\
6683	16.306589\\
6684	13.872458\\
6685	16.114884\\
6686	15.008968\\
6687	12.017993\\
6688	12.612699\\
6689	13.801599\\
6690	13.801569\\
6691	11.667345\\
6692	10.122262\\
6693	8.822768\\
6694	8.349818\\
6695	7.378683\\
6696	6.097095\\
6697	4.43339\\
6698	4.433392\\
6699	4.433392\\
6700	4.433392\\
6701	4.452686\\
6702	7.110764\\
6703	10.941791\\
6704	9.668663\\
6705	8.816106\\
6706	8.389146\\
6707	8.498111\\
6708	6.866088\\
6709	7.372172\\
6710	6.963445\\
6711	6.851226\\
6712	7.008034\\
6713	8.389142\\
6714	10.016704\\
6715	9.848938\\
6716	9.790978\\
6717	8.647731\\
6718	8.382025\\
6719	7.030142\\
6720	6.136175\\
6721	5.109187\\
6722	4.43339\\
6723	4.43339\\
6724	4.841901\\
6725	5.970089\\
6726	8.381955\\
6727	10.734265\\
6728	11.29563\\
6729	12.9313\\
6730	12.679464\\
6731	13.368405\\
6732	14.666004\\
6733	12.485028\\
6734	11.089078\\
6735	9.902356\\
6736	9.01612\\
6737	9.897807\\
6738	10.749924\\
6739	9.996485\\
6740	9.716565\\
6741	8.560417\\
6742	8.389144\\
6743	6.902563\\
6744	5.978018\\
6745	4.884945\\
6746	4.43339\\
6747	4.433389\\
6748	4.433389\\
6749	5.783558\\
6750	8.010589\\
6751	10.71019\\
6752	12.177476\\
6753	10.46924\\
6754	10.522385\\
6755	9.605565\\
6756	7.757196\\
6757	8.053486\\
6758	8.389145\\
6759	8.631225\\
6760	8.749599\\
6761	10.433105\\
6762	9.879597\\
6763	10.794434\\
6764	9.680437\\
6765	8.68564\\
6766	8.381946\\
6767	7.14713\\
6768	5.850695\\
6769	4.760482\\
6770	4.43339\\
6771	4.43339\\
6772	4.433389\\
6773	5.80866\\
6774	8.389137\\
6775	12.675483\\
6776	11.642129\\
6777	11.251595\\
6778	10.520204\\
6779	11.850614\\
6780	9.202963\\
6781	8.905286\\
6782	8.961496\\
6783	8.631228\\
6784	9.199256\\
6785	10.792454\\
6786	12.536886\\
6787	14.474385\\
6788	16.872033\\
6789	11.869924\\
6790	11.283672\\
6791	9.877426\\
6792	7.420861\\
6793	6.851167\\
6794	5.80933\\
6795	5.444526\\
6796	5.21588\\
6797	5.777659\\
6798	6.851083\\
6799	7.372169\\
6800	8.389145\\
6801	8.631225\\
6802	8.631229\\
6803	8.631225\\
6804	8.389142\\
6805	8.298061\\
6806	7.296223\\
6807	7.116267\\
6808	7.372171\\
6809	8.33792\\
6810	8.389145\\
6811	9.149528\\
6812	8.631224\\
6813	8.378653\\
6814	7.197284\\
6815	6.851224\\
6816	6.323323\\
6817	5.23084\\
6818	4.538094\\
6819	4.433394\\
6820	4.433393\\
6821	4.894997\\
6822	5.80124\\
6823	6.845755\\
6824	6.851224\\
6825	6.933666\\
6826	7.144229\\
6827	6.851224\\
6828	6.851224\\
6829	6.447771\\
6830	5.80933\\
6831	5.80933\\
6832	5.989082\\
6833	6.851224\\
6834	6.851224\\
6835	7.214234\\
6836	6.851224\\
6837	6.684683\\
6838	6.691663\\
6839	5.80933\\
6840	5.35619\\
6841	4.866599\\
6842	4.866524\\
6843	4.866696\\
6844	4.866959\\
6845	6.347744\\
6846	8.738718\\
6847	10.446826\\
6848	9.860751\\
6849	9.064385\\
6850	9.064392\\
6851	9.183043\\
6852	8.876331\\
6853	9.064368\\
6854	8.955529\\
6855	9.058708\\
6856	9.064492\\
6857	10.468854\\
6858	17.96436\\
6859	17.964362\\
6860	10.478104\\
6861	9.26214\\
6862	9.064501\\
6863	7.402087\\
6864	6.514269\\
6865	4.863523\\
6866	4.433391\\
6867	4.43339\\
6868	4.43339\\
6869	5.80933\\
6870	8.054565\\
6871	17.53123\\
6872	15.761316\\
6873	9.816815\\
6874	8.631225\\
6875	8.631226\\
6876	8.061838\\
6877	8.389145\\
6878	8.424564\\
6879	8.726231\\
6880	9.728742\\
6881	17.53123\\
6882	22.333079\\
6883	23.328693\\
6884	22.333079\\
6885	17.53123\\
6886	15.761365\\
6887	8.631227\\
6888	7.372171\\
6889	9.01279\\
6890	8.491791\\
6891	8.49179\\
6892	8.940893\\
6893	10.029672\\
6894	23.973433\\
6895	718.036701\\
6896	33.248531\\
6897	22.333085\\
6898	22.333107\\
6899	22.333086\\
6900	20.588701\\
6901	20.5887\\
6902	20.5887\\
6903	21.135199\\
6904	22.33308\\
6905	38.764268\\
6906	187.130922\\
6907	567.89465\\
6908	142.386241\\
6909	22.333079\\
6910	17.53123\\
6911	9.76549\\
6912	7.372172\\
6913	6.851224\\
6914	5.810037\\
6915	5.80933\\
6916	5.80933\\
6917	6.851223\\
6918	8.631223\\
6919	22.33308\\
6920	21.1352\\
6921	22.333079\\
6922	22.333079\\
6923	20.5887\\
6924	17.53123\\
6925	17.53123\\
6926	14.474377\\
6927	15.761315\\
6928	15.761316\\
6929	22.333079\\
6930	52.261719\\
6931	23.328694\\
6932	22.33308\\
6933	20.5887\\
6934	17.53123\\
6935	9.975048\\
6936	7.372145\\
6937	6.851224\\
6938	5.959021\\
6939	5.809347\\
6940	6.428546\\
6941	6.851224\\
6942	9.411461\\
6943	23.328693\\
6944	26.876236\\
6945	22.358989\\
6946	22.33308\\
6947	17.53123\\
6948	13.368401\\
6949	9.934424\\
6950	9.381774\\
6951	9.603427\\
6952	14.858628\\
6953	17.53123\\
6954	14.474382\\
6955	15.761371\\
6956	12.930016\\
6957	9.910515\\
6958	9.234031\\
6959	8.382499\\
6960	6.851224\\
6961	6.157225\\
6962	4.960861\\
6963	4.960856\\
6964	4.960861\\
6965	4.960853\\
6966	4.960861\\
6967	6.336802\\
6968	7.378684\\
6969	7.378684\\
6970	8.523602\\
6971	8.654729\\
6972	8.49179\\
6973	7.599434\\
6974	7.449684\\
6975	7.449898\\
6976	7.997477\\
6977	8.828078\\
6978	10.271966\\
6979	9.064449\\
6980	7.899593\\
6981	7.378683\\
6982	7.378683\\
6983	7.172467\\
6984	5.223428\\
6985	4.960857\\
6986	4.960862\\
6987	4.463902\\
6988	4.503402\\
6989	4.613925\\
6990	4.960862\\
6991	4.960861\\
6992	4.970925\\
6993	5.291189\\
6994	5.392216\\
6995	5.388665\\
6996	4.960861\\
6997	4.43339\\
6998	4.433391\\
6999	4.433391\\
7000	4.43339\\
7001	4.43339\\
7002	5.903665\\
7003	6.851224\\
7004	6.851224\\
7005	6.708651\\
7006	5.80933\\
7007	5.088897\\
7008	4.433391\\
7009	4.431111\\
7010	4.099082\\
7011	3.97486\\
7012	4.433384\\
7013	4.433389\\
7014	6.851224\\
7015	10.551389\\
7016	11.466807\\
7017	11.313588\\
7018	10.777135\\
7019	10.548845\\
7020	8.95396\\
7021	9.192415\\
7022	9.026496\\
7023	9.026553\\
7024	9.407308\\
7025	11.122309\\
7026	9.93804\\
7027	9.088564\\
7028	8.631226\\
7029	6.851225\\
7030	6.851322\\
7031	6.778106\\
7032	5.34263\\
7033	4.433391\\
7034	4.433396\\
7035	4.43339\\
7036	4.433389\\
7037	4.433392\\
7038	6.851226\\
7039	8.631223\\
7040	8.932392\\
7041	8.814777\\
7042	8.743862\\
7043	10.16018\\
7044	8.631225\\
7045	8.466691\\
7046	8.389143\\
7047	7.763487\\
7048	7.641642\\
7049	8.550012\\
7050	8.389145\\
7051	8.230638\\
7052	6.851224\\
7053	5.809371\\
7054	5.613998\\
7055	4.433401\\
7056	5.304405\\
7057	5.304255\\
7058	4.468355\\
7059	4.052996\\
7060	4.667247\\
7061	5.304403\\
7062	7.562625\\
7063	9.502247\\
7064	8.389146\\
7065	8.272752\\
7066	7.372171\\
7067	7.37217\\
7068	7.022551\\
7069	6.851229\\
7070	6.851229\\
7071	7.129525\\
7072	7.511567\\
7073	8.389145\\
7074	11.37308\\
7075	10.329844\\
7076	10.547042\\
7077	8.389146\\
7078	7.51362\\
7079	6.52617\\
7080	4.433604\\
7081	4.433389\\
7082	4.433389\\
7083	4.433389\\
7084	4.433389\\
7085	4.433389\\
7086	6.851225\\
7087	9.017808\\
7088	8.63698\\
7089	8.456044\\
7090	8.389145\\
7091	8.38915\\
7092	7.667096\\
7093	8.183957\\
7094	7.602416\\
7095	8.382439\\
7096	8.389142\\
7097	8.555712\\
7098	8.38915\\
7099	8.38916\\
7100	7.178273\\
7101	6.851224\\
7102	6.573643\\
7103	5.616935\\
7104	4.433389\\
7105	4.433384\\
7106	3.217174\\
7107	2.988622\\
7108	3.00643\\
7109	4.433389\\
7110	5.705244\\
7111	6.851247\\
7112	7.487522\\
7113	8.389145\\
7114	8.389144\\
7115	8.389145\\
7116	8.631225\\
7117	8.38914\\
7118	7.513384\\
7119	7.614856\\
7120	7.777754\\
7121	8.30595\\
7122	8.995407\\
7123	7.728343\\
7124	7.271018\\
7125	6.851224\\
7126	6.851224\\
7127	7.022699\\
7128	6.197421\\
7129	4.864747\\
7130	4.433389\\
7131	4.433389\\
7132	4.433389\\
7133	4.433389\\
7134	5.5692\\
7135	6.816851\\
7136	7.405456\\
7137	8.058403\\
7138	8.050027\\
7139	8.386923\\
7140	8.389145\\
7141	6.851224\\
7142	6.191175\\
7143	5.809337\\
7144	6.620322\\
7145	6.96902\\
7146	8.382024\\
7147	7.282192\\
7148	6.851228\\
7149	6.648384\\
7150	6.830332\\
7151	6.381387\\
7152	4.43339\\
7153	4.433389\\
7154	3.785472\\
7155	3.396949\\
7156	3.038769\\
7157	3.179694\\
7158	3.006443\\
7159	3.972636\\
7160	4.433389\\
7161	4.43339\\
7162	4.43339\\
7163	4.433389\\
7164	4.433389\\
7165	4.433389\\
7166	4.29442\\
7167	4.185365\\
7168	5.304403\\
7169	5.304403\\
7170	5.61683\\
7171	7.605171\\
7172	6.680345\\
7173	4.43339\\
7174	4.43339\\
7175	4.43339\\
7176	4.43339\\
7177	3.785298\\
7178	2.589927\\
7179	2.589901\\
7180	2.589901\\
7181	2.589905\\
7182	4.091642\\
7183	5.09075\\
7184	6.851224\\
7185	6.851227\\
7186	6.851224\\
7187	6.85122\\
7188	5.72345\\
7189	5.80933\\
7190	5.80933\\
7191	5.80933\\
7192	6.752257\\
7193	7.089118\\
7194	8.184568\\
7195	9.468916\\
7196	8.389143\\
7197	7.851611\\
7198	6.851224\\
7199	6.297307\\
7200	5.569624\\
7201	4.43339\\
7202	4.433325\\
7203	3.498689\\
7204	3.00642\\
7205	3.674861\\
7206	4.433389\\
7207	5.876266\\
7208	6.853926\\
7209	6.851228\\
7210	6.851224\\
7211	6.805663\\
7212	6.429061\\
7213	5.80933\\
7214	5.809333\\
7215	5.933859\\
7216	6.851224\\
7217	7.516748\\
7218	8.631225\\
7219	9.362855\\
7220	8.522868\\
7221	7.513223\\
7222	6.851227\\
7223	6.851224\\
7224	5.814456\\
7225	4.43339\\
7226	4.433385\\
7227	3.785472\\
7228	3.687005\\
7229	4.05194\\
7230	4.43339\\
7231	6.549598\\
7232	7.573853\\
7233	8.631226\\
7234	8.664043\\
7235	9.75051\\
7236	10.89764\\
7237	11.129131\\
7238	11.199479\\
7239	11.008119\\
7240	11.027368\\
7241	11.551881\\
7242	13.368401\\
7243	13.368401\\
7244	10.109069\\
7245	8.803309\\
7246	8.381977\\
7247	7.590732\\
7248	7.141119\\
7249	5.878543\\
7250	4.43339\\
7251	4.43339\\
7252	4.433389\\
7253	4.43339\\
7254	4.693298\\
7255	6.851231\\
7256	8.121002\\
7257	8.59332\\
7258	8.948153\\
7259	8.631224\\
7260	8.631227\\
7261	8.172191\\
7262	8.307748\\
7263	8.014872\\
7264	8.247445\\
7265	8.389072\\
7266	9.259314\\
7267	12.151969\\
7268	9.619563\\
7269	7.372171\\
7270	7.722238\\
7271	7.722235\\
7272	7.156565\\
7273	5.304406\\
7274	5.304403\\
7275	4.366754\\
7276	3.877443\\
7277	4.642399\\
7278	5.304403\\
7279	6.769681\\
7280	7.893711\\
7281	8.243192\\
7282	8.243174\\
7283	7.722238\\
7284	7.722239\\
7285	7.62268\\
7286	7.037056\\
7287	6.75202\\
7288	7.12363\\
7289	6.851204\\
7290	7.023014\\
7291	7.514257\\
7292	6.851231\\
7293	6.106692\\
7294	4.43339\\
7295	4.43339\\
7296	5.840531\\
7297	4.377665\\
7298	3.893476\\
7299	2.460283\\
7300	1.856514\\
7301	1.472052\\
7302	1.78175\\
7303	2.964855\\
7304	3.893478\\
7305	3.893481\\
7306	3.893477\\
7307	3.750026\\
7308	3.651463\\
7309	3.893476\\
7310	3.893319\\
7311	3.656582\\
7312	3.893476\\
7313	4.472373\\
7314	5.840531\\
7315	5.840531\\
7316	5.840531\\
7317	5.83523\\
7318	5.840514\\
7319	5.84053\\
7320	5.84053\\
7321	3.893478\\
7322	2.26507\\
7323	1.15808\\
7324	0.170531\\
7325	0.000326\\
7326	0.093339\\
7327	1.158078\\
7328	1.445718\\
7329	1.543226\\
7330	1.445373\\
7331	1.206841\\
7332	1.166334\\
7333	1.856526\\
7334	1.566052\\
7335	1.941131\\
7336	2.770054\\
7337	3.858059\\
7338	4.31727\\
7339	5.477684\\
7340	5.160498\\
7341	4.263288\\
7342	4.333392\\
7343	4.314598\\
7344	4.078718\\
7345	2.006772\\
7346	0.600973\\
7347	1e-06\\
7348	7.8e-05\\
7349	0.000321\\
7350	1.873582\\
7351	5.1474\\
7352	6.141911\\
7353	6.165691\\
7354	5.840545\\
7355	6.677097\\
7356	6.333899\\
7357	6.296089\\
7358	6.849316\\
7359	7.041813\\
7360	7.191604\\
7361	8.022586\\
7362	8.394198\\
7363	8.394196\\
7364	8.394196\\
7365	7.41022\\
7366	6.535874\\
7367	6.727965\\
7368	6.296121\\
7369	5.84053\\
7370	5.357579\\
7371	4.458721\\
7372	4.71611\\
7373	5.245517\\
7374	5.84053\\
7375	8.234305\\
7376	9.093537\\
7377	8.944403\\
7378	10.011007\\
7379	10.974187\\
7380	11.277326\\
7381	11.047243\\
7382	11.242586\\
7383	11.354542\\
7384	12.158307\\
7385	10.570511\\
7386	19.67419\\
7387	19.67419\\
7388	11.41604\\
7389	11.06822\\
7390	9.653967\\
7391	8.944629\\
7392	8.533891\\
7393	7.318989\\
7394	6.019987\\
7395	5.840531\\
7396	5.840531\\
7397	5.840531\\
7398	6.78533\\
7399	9.073661\\
7400	12.42575\\
7401	12.341261\\
7402	11.674924\\
7403	12.117727\\
7404	11.281049\\
7405	11.287655\\
7406	11.286098\\
7407	11.086678\\
7408	11.9333\\
7409	15.277499\\
7410	17.804843\\
7411	19.67419\\
7412	10.764494\\
7413	11.257102\\
7414	10.010993\\
7415	9.38661\\
7416	8.394196\\
7417	7.280767\\
7418	5.840531\\
7419	5.84053\\
7420	5.84053\\
7421	5.84053\\
7422	5.840531\\
7423	8.394196\\
7424	9.864881\\
7425	7.830089\\
7426	8.439239\\
7427	8.661069\\
7428	7.786329\\
7429	7.712651\\
7430	7.591971\\
7431	7.249631\\
7432	7.24976\\
7433	7.504812\\
7434	7.698559\\
7435	8.334255\\
7436	7.236117\\
7437	7.117494\\
7438	5.120843\\
7439	5.138\\
7440	4.682451\\
7441	4.056969\\
7442	2.735399\\
7443	1.723973\\
7444	0.772378\\
7445	1.060768\\
7446	2.735399\\
7447	4.682451\\
7448	5.830183\\
7449	6.135691\\
7450	6.677073\\
7451	6.818309\\
7452	7.236117\\
7453	7.236116\\
7454	7.167235\\
7455	6.690288\\
7456	6.212499\\
7457	5.882501\\
7458	6.400144\\
7459	7.236116\\
7460	6.679013\\
7461	5.01835\\
7462	4.682455\\
7463	4.682451\\
7464	4.819444\\
7465	4.682452\\
7466	3.175332\\
7467	2.735404\\
7468	2.069396\\
7469	1.930087\\
7470	3.893478\\
7471	4.332852\\
7472	5.840389\\
7473	5.840531\\
7474	5.840531\\
7475	5.649724\\
7476	5.553342\\
7477	5.808636\\
7478	5.464375\\
7479	5.750039\\
7480	4.682446\\
7481	4.682452\\
7482	4.682451\\
7483	5.60302\\
7484	4.682453\\
7485	4.682452\\
7486	4.682452\\
7487	4.682452\\
7488	4.682451\\
7489	3.648722\\
7490	2.735399\\
7491	1.951239\\
7492	1.460069\\
7493	1.204212\\
7494	1.866976\\
7495	2.713388\\
7496	2.735402\\
7497	3.085078\\
7498	3.687073\\
7499	4.301709\\
7500	4.682453\\
7501	4.682451\\
7502	4.682451\\
7503	4.682451\\
7504	4.682451\\
7505	4.682452\\
7506	5.13803\\
7507	6.135434\\
7508	6.13569\\
7509	5.138039\\
7510	4.927993\\
7511	4.683532\\
7512	4.682451\\
7513	4.682452\\
7514	4.164415\\
7515	3.156493\\
7516	2.798268\\
7517	2.984069\\
7518	3.648654\\
7519	4.682452\\
7520	4.682451\\
7521	4.765122\\
7522	4.682451\\
7523	4.682451\\
7524	4.682451\\
7525	4.682452\\
7526	4.682452\\
7527	4.682453\\
7528	4.682451\\
7529	4.682451\\
7530	6.617561\\
7531	7.037371\\
7532	5.877942\\
7533	5.138439\\
7534	4.682451\\
7535	5.160237\\
7536	4.682451\\
7537	4.68245\\
7538	3.156519\\
7539	2.735405\\
7540	2.703927\\
7541	2.579075\\
7542	2.735399\\
7543	3.085078\\
7544	2.735403\\
7545	2.735404\\
7546	3.085079\\
7547	3.918553\\
7548	4.44799\\
7549	4.682449\\
7550	4.208699\\
7551	3.6392\\
7552	4.141865\\
7553	4.682451\\
7554	4.682452\\
7555	5.128669\\
7556	4.766646\\
7557	4.781928\\
7558	4.682452\\
7559	4.682455\\
7560	4.682453\\
7561	3.998204\\
7562	2.735402\\
7563	2.562969\\
7564	2.530222\\
7565	2.735404\\
7566	4.08819\\
7567	5.943622\\
7568	8.124254\\
7569	7.70616\\
7570	7.878188\\
7571	8.343097\\
7572	8.425366\\
7573	8.335072\\
7574	8.859382\\
7575	8.860047\\
7576	8.860435\\
7577	8.71795\\
7578	11.031181\\
7579	14.11942\\
7580	9.394985\\
7581	8.90417\\
7582	7.613246\\
7583	7.786327\\
7584	7.236122\\
7585	6.13569\\
7586	4.682451\\
7587	4.682452\\
7588	4.682453\\
7589	4.682453\\
7590	4.868593\\
7591	7.359983\\
7592	8.860439\\
7593	8.853191\\
7594	7.828261\\
7595	7.236116\\
7596	7.236117\\
7597	7.23611\\
7598	7.236117\\
7599	7.236136\\
7600	7.236117\\
7601	7.935476\\
7602	10.644325\\
7603	11.343618\\
7604	9.03279\\
7605	8.506754\\
7606	7.236123\\
7607	7.236117\\
7608	7.030138\\
7609	4.682452\\
7610	4.682451\\
7611	3.686614\\
7612	3.177174\\
7613	3.300996\\
7614	4.682453\\
7615	6.144596\\
7616	8.151697\\
7617	8.698423\\
7618	8.860436\\
7619	8.595007\\
7620	8.899509\\
7621	8.780282\\
7622	9.021684\\
7623	9.116117\\
7624	9.241446\\
7625	10.123264\\
7626	10.05993\\
7627	10.129643\\
7628	8.936417\\
7629	9.113695\\
7630	7.632881\\
7631	7.937419\\
7632	7.236116\\
7633	4.682457\\
7634	4.682451\\
7635	4.50695\\
7636	3.761952\\
7637	3.551259\\
7638	4.147561\\
7639	4.682451\\
7640	4.682451\\
7641	5.882521\\
7642	6.864744\\
7643	7.236117\\
7644	7.236117\\
7645	7.236117\\
7646	7.236116\\
7647	6.828295\\
7648	6.347245\\
7649	6.135691\\
7650	6.549943\\
7651	7.131758\\
7652	7.236117\\
7653	6.135692\\
7654	4.682457\\
7655	5.137572\\
7656	4.976053\\
7657	4.682451\\
7658	4.199291\\
7659	3.156508\\
7660	2.748362\\
7661	2.737563\\
7662	2.735404\\
7663	3.156519\\
7664	4.010885\\
7665	4.682451\\
7666	4.682451\\
7667	4.999869\\
7668	5.753284\\
7669	6.135691\\
7670	6.052645\\
7671	5.290876\\
7672	5.660068\\
7673	5.773567\\
7674	6.833882\\
7675	7.236117\\
7676	7.236117\\
7677	6.66946\\
7678	6.135513\\
7679	5.882529\\
7680	5.138032\\
7681	4.682451\\
7682	4.273855\\
7683	3.443566\\
7684	3.156518\\
7685	3.175316\\
7686	4.682451\\
7687	6.372751\\
7688	8.645978\\
7689	8.860436\\
7690	9.116116\\
7691	8.860439\\
7692	8.875136\\
7693	8.638883\\
7694	8.860436\\
7695	10.340997\\
7696	11.155313\\
7697	10.633473\\
7698	17.804834\\
7699	16.802447\\
7700	9.501979\\
7701	8.860436\\
7702	7.236117\\
7703	7.403753\\
7704	7.236116\\
7705	5.59807\\
7706	4.682451\\
7707	4.682451\\
7708	4.682451\\
7709	4.682451\\
7710	4.682452\\
7711	7.236162\\
7712	10.286491\\
7713	9.850568\\
7714	9.19442\\
7715	9.116116\\
7716	9.116116\\
7717	9.116116\\
7718	9.488807\\
7719	9.442071\\
7720	10.741954\\
7721	9.715229\\
7722	14.119419\\
7723	11.965419\\
7724	10.726908\\
7725	9.474383\\
7726	8.133485\\
7727	8.185843\\
7728	7.236117\\
7729	6.052817\\
7730	4.682452\\
7731	4.682451\\
7732	4.682451\\
7733	4.682451\\
7734	4.682452\\
7735	8.706276\\
7736	12.709175\\
7737	10.765351\\
7738	11.297729\\
7739	11.10729\\
7740	11.080454\\
7741	10.766289\\
7742	11.711149\\
7743	11.068891\\
7744	11.503576\\
7745	10.226205\\
7746	11.683304\\
7747	11.216948\\
7748	9.204695\\
7749	9.661947\\
7750	8.04396\\
7751	7.417222\\
7752	7.236117\\
7753	6.135691\\
7754	4.682452\\
7755	4.682451\\
7756	4.682451\\
7757	4.682451\\
7758	5.138038\\
7759	7.455065\\
7760	10.444328\\
7761	9.480073\\
7762	9.116116\\
7763	8.826014\\
7764	8.123208\\
7765	8.09718\\
7766	8.71623\\
7767	8.852919\\
7768	8.860436\\
7769	9.116122\\
7770	9.856441\\
7771	10.080539\\
7772	8.860791\\
7773	8.860436\\
7774	7.236117\\
7775	7.236117\\
7776	7.236117\\
7777	5.882517\\
7778	4.682452\\
7779	4.682451\\
7780	4.682451\\
7781	4.682451\\
7782	4.682458\\
7783	7.236117\\
7784	8.982788\\
7785	9.751626\\
7786	9.723128\\
7787	9.762406\\
7788	9.116119\\
7789	8.860436\\
7790	9.335275\\
7791	9.053014\\
7792	9.002951\\
7793	9.342058\\
7794	9.932455\\
7795	9.871755\\
7796	8.860436\\
7797	7.707035\\
7798	7.207821\\
7799	7.236117\\
7800	7.234962\\
7801	4.820239\\
7802	4.682451\\
7803	3.906271\\
7804	3.15652\\
7805	2.945239\\
7806	3.08644\\
7807	3.751145\\
7808	5.602398\\
7809	5.662313\\
7810	6.562668\\
7811	4.682451\\
7812	4.682451\\
7813	4.682451\\
7814	4.682451\\
7815	4.682451\\
7816	4.682451\\
7817	5.56995\\
7818	6.135676\\
7819	6.781225\\
7820	7.067542\\
7821	5.882518\\
7822	4.682451\\
7823	4.68247\\
7824	4.682467\\
7825	4.682451\\
7826	3.175318\\
7827	2.735399\\
7828	2.505574\\
7829	2.529763\\
7830	2.735399\\
7831	2.847323\\
7832	3.156519\\
7833	3.691303\\
7834	4.207362\\
7835	4.306203\\
7836	4.682451\\
7837	4.682451\\
7838	4.602561\\
7839	4.312726\\
7840	4.68199\\
7841	4.682451\\
7842	4.682451\\
7843	4.682451\\
7844	4.886174\\
7845	4.682451\\
7846	4.682451\\
7847	4.682451\\
7848	4.68245\\
7849	3.156519\\
7850	2.735399\\
7851	2.392333\\
7852	2.283846\\
7853	2.735399\\
7854	3.175334\\
7855	4.682451\\
7856	7.236117\\
7857	7.438305\\
7858	7.417223\\
7859	7.236117\\
7860	7.236117\\
7861	7.236116\\
7862	6.680025\\
7863	7.236116\\
7864	7.236118\\
7865	7.568076\\
7866	8.860432\\
7867	8.860436\\
7868	8.43783\\
7869	7.236117\\
7870	6.135694\\
7871	6.679323\\
7872	6.135629\\
7873	4.682451\\
7874	4.682451\\
7875	4.199333\\
7876	3.809341\\
7877	4.027153\\
7878	4.682451\\
7879	6.511812\\
7880	8.786261\\
7881	8.860435\\
7882	8.314517\\
7883	7.786328\\
7884	7.650159\\
7885	7.236117\\
7886	7.236117\\
7887	7.404719\\
7888	7.331062\\
7889	7.793549\\
7890	9.1161\\
7891	8.860436\\
7892	8.737293\\
7893	7.417215\\
7894	6.67901\\
7895	7.065224\\
7896	6.141017\\
7897	4.682451\\
7898	4.682451\\
7899	4.199494\\
7900	3.998126\\
7901	4.456521\\
7902	4.682451\\
7903	6.966073\\
7904	8.852928\\
7905	9.24291\\
7906	9.795925\\
7907	10.799746\\
7908	15.287525\\
7909	15.287526\\
7910	16.646765\\
7911	16.646767\\
7912	18.516111\\
7913	18.516111\\
7914	18.516111\\
7915	18.516111\\
7916	10.546259\\
7917	8.852897\\
7918	7.697788\\
7919	7.581007\\
7920	7.236117\\
7921	6.060048\\
7922	4.682452\\
7923	4.682451\\
7924	5.602397\\
7925	5.602397\\
7926	6.029401\\
7927	8.649955\\
7928	11.413324\\
7929	15.039368\\
7930	10.419488\\
7931	9.944291\\
7932	11.494397\\
7933	10.132934\\
7934	10.534644\\
7935	10.161089\\
7936	10.317861\\
7937	10.91335\\
7938	11.047628\\
7939	10.714217\\
7940	10.648983\\
7941	9.116115\\
7942	7.409207\\
7943	7.463325\\
7944	7.236117\\
7945	4.682452\\
7946	4.682451\\
7947	4.199427\\
7948	3.48284\\
7949	3.685409\\
7950	4.682451\\
7951	6.146253\\
7952	7.236117\\
7953	7.236117\\
7954	7.236117\\
7955	7.236117\\
7956	7.362971\\
7957	7.236117\\
7958	7.236117\\
7959	7.236116\\
7960	7.236116\\
7961	7.41722\\
7962	7.236117\\
7963	7.236117\\
7964	7.236117\\
7965	6.485118\\
7966	5.138038\\
7967	5.974247\\
7968	5.883141\\
7969	4.682451\\
7970	4.682451\\
7971	3.635735\\
7972	3.085092\\
7973	3.085074\\
7974	3.156516\\
7975	4.413925\\
7976	4.682451\\
7977	4.682452\\
7978	5.285436\\
7979	4.87016\\
7980	4.682452\\
7981	4.682452\\
7982	4.682452\\
7983	4.682452\\
7984	5.250181\\
7985	5.882517\\
7986	6.909099\\
7987	6.965513\\
7988	5.710634\\
7989	4.921968\\
7990	4.682455\\
7991	5.568472\\
7992	6.13563\\
7993	4.999461\\
7994	4.682451\\
7995	4.682451\\
7996	4.204267\\
7997	3.969439\\
7998	4.010153\\
7999	4.68245\\
8000	4.682451\\
8001	4.682451\\
};
\addplot [color=mycolor1,solid,line width=1.0pt,forget plot]
  table[row sep=crcr]{%
8001	4.682451\\
8002	5.46217\\
8003	7.815303\\
8004	8.175164\\
8005	8.547418\\
8006	7.774924\\
8007	7.539024\\
8008	7.939906\\
8009	10.027498\\
8010	10.036062\\
8011	10.036064\\
8012	10.49322\\
8013	10.036105\\
8014	8.855394\\
8015	8.85542\\
8016	8.253683\\
8017	7.140098\\
8018	6.213373\\
8019	5.669452\\
8020	5.669452\\
8021	5.669452\\
8022	5.669453\\
8023	8.253683\\
8024	10.734413\\
8025	16.606313\\
8026	19.668687\\
8027	19.668688\\
8028	24.801001\\
8029	24.801001\\
8030	24.801\\
8031	24.801\\
8032	24.801\\
8033	24.801\\
8034	25.865132\\
8035	24.801001\\
8036	19.668687\\
8037	19.668687\\
8038	10.300796\\
8039	10.068739\\
8040	9.300552\\
8041	8.958876\\
8042	7.505992\\
8043	7.322728\\
8044	6.885416\\
8045	4.738496\\
8046	4.738498\\
8047	7.506\\
8048	9.987296\\
8049	10.584976\\
8050	11.43972\\
8051	11.552561\\
8052	11.132064\\
8053	11.411439\\
8054	9.39249\\
8055	10.187088\\
8056	10.795025\\
8057	11.302152\\
8058	11.696954\\
8059	11.199982\\
8060	10.094854\\
8061	8.966486\\
8062	7.322726\\
8063	7.322725\\
8064	7.322726\\
8065	6.464895\\
8066	5.197337\\
8067	4.738495\\
8068	4.738495\\
8069	4.738495\\
8070	4.738496\\
8071	7.322725\\
8072	9.476894\\
8073	10.35519\\
8074	10.392513\\
8075	9.989418\\
8076	10.242377\\
8077	10.281719\\
8078	11.120219\\
8079	11.367265\\
8080	16.846011\\
8081	18.120207\\
8082	18.120207\\
8083	15.470506\\
8084	11.42756\\
8085	10.677363\\
8086	9.041058\\
8087	8.966486\\
8088	8.709244\\
8089	7.879523\\
8090	7.322725\\
8091	6.422817\\
8092	5.868171\\
8093	5.952926\\
8094	6.831351\\
8095	8.966517\\
8096	18.73773\\
8097	23.870043\\
8098	23.870043\\
8099	23.870043\\
8100	24.934175\\
8101	24.934175\\
8102	24.934175\\
8103	23.870043\\
8104	23.870043\\
8105	18.737718\\
8106	23.870043\\
8107	22.257683\\
8108	18.73773\\
8109	18.120202\\
8110	9.983576\\
8111	10.257344\\
8112	10.207576\\
8113	7.434216\\
8114	7.322725\\
8115	6.29112\\
8116	5.836769\\
8117	5.67897\\
8118	6.759711\\
8119	8.966486\\
8120	14.288415\\
8121	11.485833\\
8122	16.84601\\
8123	16.846013\\
8124	16.846011\\
8125	14.288415\\
8126	11.785243\\
8127	10.959498\\
8128	10.785211\\
8129	11.553896\\
8130	10.734378\\
8131	9.469548\\
8132	9.225227\\
8133	7.879513\\
8134	7.895678\\
8135	8.36511\\
8136	8.175556\\
8137	7.322726\\
8138	6.209126\\
8139	4.738495\\
8140	4.738496\\
8141	4.738495\\
8142	4.738495\\
8143	4.738495\\
8144	5.82893\\
8145	7.152113\\
8146	7.412809\\
8147	7.690344\\
8148	7.506492\\
8149	7.323447\\
8150	7.322729\\
8151	7.322725\\
8152	7.873324\\
8153	8.958703\\
8154	9.231187\\
8155	9.402561\\
8156	9.390468\\
8157	8.966485\\
8158	7.810187\\
8159	7.879524\\
8160	10.719954\\
8161	9.261512\\
8162	8.445541\\
8163	6.981146\\
8164	6.491967\\
8165	6.491966\\
8166	6.491966\\
8167	6.491967\\
8168	6.491967\\
8169	6.491967\\
8170	7.4384\\
8171	8.487758\\
8172	9.076195\\
8173	9.076197\\
8174	6.209129\\
8175	5.952903\\
8176	5.796414\\
8177	6.241108\\
8178	7.290824\\
8179	7.303133\\
8180	7.322726\\
8181	7.322728\\
8182	7.222828\\
8183	7.322726\\
8184	5.996087\\
8185	4.738496\\
8186	4.738495\\
8187	3.663831\\
8188	3.097117\\
8189	3.122032\\
8190	4.605949\\
8191	6.03884\\
8192	7.798237\\
8193	9.225226\\
8194	9.149696\\
8195	8.966482\\
8196	9.225226\\
8197	9.917076\\
8198	10.260553\\
8199	11.262191\\
8200	14.288415\\
8201	16.846011\\
8202	16.84601\\
8203	18.737729\\
8204	15.675338\\
8205	9.225323\\
8206	7.683098\\
8207	7.954299\\
8208	8.348275\\
8209	7.054412\\
8210	4.925853\\
8211	4.738496\\
8212	4.738495\\
8213	4.738496\\
8214	5.487786\\
8215	8.311147\\
8216	10.880919\\
8217	18.120206\\
8218	11.054057\\
8219	11.500011\\
8220	10.108026\\
8221	10.448424\\
8222	10.945523\\
8223	10.518198\\
8224	10.738354\\
8225	11.019485\\
8226	11.415479\\
8227	10.232345\\
8228	9.411179\\
8229	7.667754\\
8230	6.761594\\
8231	6.384058\\
8232	6.209125\\
8233	4.738495\\
8234	4.738495\\
8235	3.367334\\
8236	2.924284\\
8237	2.781129\\
8238	4.200863\\
8239	5.423951\\
8240	7.322726\\
8241	8.966486\\
8242	8.966486\\
8243	8.492277\\
8244	8.142721\\
8245	7.324468\\
8246	8.966482\\
8247	8.966486\\
8248	9.76845\\
8249	14.288415\\
8250	18.120207\\
8251	11.369485\\
8252	9.225226\\
8253	8.966486\\
8254	7.322726\\
8255	7.322726\\
8256	7.322726\\
8257	4.811915\\
8258	4.738497\\
8259	4.364027\\
8260	3.705065\\
8261	3.979011\\
8262	4.738497\\
8263	5.684331\\
8264	8.157763\\
8265	8.966486\\
8266	8.966487\\
8267	8.966487\\
8268	9.225226\\
8269	8.967809\\
8270	8.966486\\
8271	8.966513\\
8272	9.225219\\
8273	9.227158\\
8274	9.709156\\
8275	9.225227\\
8276	8.958871\\
8277	8.03104\\
8278	7.152145\\
8279	6.950533\\
8280	7.092446\\
8281	4.738496\\
8282	4.737379\\
8283	3.555764\\
8284	3.122031\\
8285	3.194289\\
8286	4.738496\\
8287	5.900659\\
8288	7.322726\\
8289	7.322726\\
8290	7.322726\\
8291	7.953433\\
8292	8.86831\\
8293	8.519507\\
8294	8.921419\\
8295	8.966482\\
8296	8.966484\\
8297	9.342149\\
8298	9.754324\\
8299	10.732048\\
8300	9.191086\\
8301	7.79835\\
8302	7.322726\\
8303	7.506986\\
8304	7.505996\\
8305	7.322726\\
8306	5.507399\\
8307	4.738496\\
8308	4.738497\\
8309	4.738497\\
8310	4.738498\\
8311	4.738496\\
8312	5.660601\\
8313	7.322726\\
8314	7.614207\\
8315	8.369858\\
8316	8.193879\\
8317	8.505114\\
8318	8.195668\\
8319	7.830922\\
8320	8.716046\\
8321	8.966486\\
8322	8.967111\\
8323	9.341201\\
8324	8.869173\\
8325	8.028347\\
8326	7.322726\\
8327	7.322726\\
8328	7.322726\\
8329	7.205452\\
8330	5.458946\\
8331	4.738498\\
8332	4.7385\\
8333	4.738499\\
8334	4.738498\\
8335	4.738499\\
8336	4.7385\\
8337	4.738498\\
8338	5.057407\\
8339	5.528841\\
8340	6.066567\\
8341	5.952932\\
8342	4.738496\\
8343	4.738496\\
8344	4.738496\\
8345	4.738496\\
8346	6.209125\\
8347	6.034498\\
8348	4.759382\\
8349	4.738517\\
8350	4.738501\\
8351	4.738501\\
8352	4.738497\\
8353	4.144504\\
8354	2.8712\\
8355	1.916247\\
8356	0.458607\\
8357	0.083471\\
8358	0.618532\\
8359	2.768142\\
8360	3.555489\\
8361	4.738497\\
8362	5.17598\\
8363	7.101779\\
8364	7.322726\\
8365	7.322726\\
8366	7.322726\\
8367	7.322725\\
8368	7.322726\\
8369	7.505995\\
8370	8.940979\\
8371	8.419047\\
8372	7.322726\\
8373	5.392392\\
8374	4.738496\\
8375	4.845066\\
8376	5.588221\\
8377	4.738496\\
8378	4.731872\\
8379	3.249229\\
8380	2.916537\\
8381	3.007528\\
8382	4.737418\\
8383	8.253663\\
8384	10.139513\\
8385	11.311742\\
8386	11.744745\\
8387	9.994369\\
8388	10.103383\\
8389	9.343091\\
8390	9.47749\\
8391	9.839739\\
8392	10.567195\\
8393	11.141831\\
8394	18.73773\\
8395	18.73773\\
8396	11.084041\\
8397	10.601612\\
8398	8.953963\\
8399	8.958354\\
8400	7.797744\\
8401	5.874179\\
8402	4.738496\\
8403	4.666098\\
8404	3.300517\\
8405	3.122071\\
8406	3.996915\\
8407	4.738496\\
8408	7.322726\\
8409	8.958887\\
8410	9.435819\\
8411	10.184827\\
8412	11.1037\\
8413	10.796081\\
8414	10.711927\\
8415	10.965146\\
8416	11.345617\\
8417	18.120208\\
8418	18.120202\\
8419	16.846011\\
8420	18.033063\\
8421	9.225227\\
8422	7.40477\\
8423	7.322726\\
8424	7.322726\\
8425	5.199532\\
8426	4.738496\\
8427	4.738495\\
8428	4.219194\\
8429	6.491947\\
8430	6.491966\\
8431	9.076196\\
8432	8.064241\\
8433	9.225227\\
8434	7.322726\\
8435	8.369143\\
8436	8.958879\\
8437	7.322726\\
8438	8.051742\\
8439	7.588069\\
8440	7.453612\\
8441	7.505999\\
8442	7.341738\\
8443	8.326174\\
8444	7.322726\\
8445	5.995077\\
8446	4.738496\\
8447	4.738497\\
8448	4.738496\\
8449	4.732199\\
8450	2.907682\\
8451	1.920786\\
8452	1.029048\\
8453	1.00358\\
8454	2.768142\\
8455	4.738496\\
8456	5.952745\\
8457	7.322726\\
8458	7.322726\\
8459	7.522209\\
8460	8.958876\\
8461	8.966517\\
8462	7.979556\\
8463	7.322726\\
8464	7.322726\\
8465	7.322726\\
8466	7.343582\\
8467	7.322822\\
8468	6.900213\\
8469	5.152626\\
8470	4.738496\\
8471	4.738496\\
8472	4.738497\\
8473	2.768545\\
8474	1.25491\\
8475	3.4e-05\\
8476	5e-05\\
8477	1.2e-05\\
8478	1.1e-05\\
8479	2.3e-05\\
8480	0.00017\\
8481	2.563427\\
8482	2.327059\\
8483	2.768139\\
8484	2.768139\\
8485	2.767191\\
8486	2.768143\\
8487	2.768139\\
8488	2.768141\\
8489	3.056915\\
8490	3.194299\\
8491	4.093708\\
8492	4.422341\\
8493	3.290011\\
8494	3.122022\\
8495	3.357547\\
8496	4.249488\\
8497	2.769495\\
8498	1.847536\\
8499	0.291084\\
8500	9e-06\\
8501	2e-06\\
8502	4e-06\\
8503	3e-05\\
8504	2.4e-05\\
8505	1.042292\\
8506	2.614923\\
8507	2.768141\\
8508	3.122004\\
8509	3.194299\\
8510	3.213205\\
8511	2.768139\\
8512	2.768141\\
8513	2.76814\\
8514	2.768139\\
8515	2.768139\\
8516	2.768139\\
8517	2.768139\\
8518	2.089764\\
8519	2.766914\\
8520	2.504916\\
8521	0.000143\\
8522	2e-06\\
8523	0\\
8524	1.3e-05\\
8525	3e-06\\
8526	1e-06\\
8527	0.546432\\
8528	2.768141\\
8529	4.738495\\
8530	4.7385\\
8531	4.738497\\
8532	4.738499\\
8533	4.738495\\
8534	4.738495\\
8535	4.204226\\
8536	3.680212\\
8537	3.837516\\
8538	3.723306\\
8539	4.307782\\
8540	2.81222\\
8541	2.747057\\
8542	2.157327\\
8543	2.768139\\
8544	2.767985\\
8545	1e-06\\
8546	8e-06\\
8547	5e-06\\
8548	0\\
8549	0\\
8550	0\\
8551	1.831974\\
8552	3.940077\\
8553	5.716072\\
8554	5.910432\\
8555	5.910433\\
8556	5.910434\\
8557	5.910434\\
8558	5.910434\\
8559	5.9093\\
8560	5.910433\\
8561	5.909065\\
8562	5.910434\\
8563	4.738495\\
8564	4.738495\\
8565	3.194294\\
8566	2.768124\\
8567	2.786626\\
8568	2.768145\\
8569	0.008868\\
8570	1e-06\\
8571	0\\
8572	0\\
8573	0\\
8574	0\\
8575	0\\
8576	6.9e-05\\
8577	1.763831\\
8578	2.768139\\
8579	4.125256\\
8580	4.144292\\
8581	4.189022\\
8582	5.66945\\
8583	5.589851\\
8584	5.669188\\
8585	5.244081\\
8586	3.657249\\
8587	3.66308\\
8588	3.231953\\
8589	2.510614\\
8590	1.815348\\
8591	3.231081\\
8592	5.201437\\
8593	4.698879\\
8594	3.231089\\
8595	1.559086\\
8596	0.147215\\
8597	3e-05\\
8598	3e-05\\
8599	0.000407\\
8600	0.000135\\
8601	0.565899\\
8602	1.237894\\
8603	2.414453\\
8604	3.231088\\
8605	3.231089\\
8606	3.231089\\
8607	3.231089\\
8608	3.231092\\
8609	3.560722\\
8610	3.657241\\
8611	4.048794\\
8612	4.987283\\
8613	5.201432\\
8614	5.201438\\
8615	5.201492\\
8616	5.201493\\
8617	5.201438\\
8618	4.044009\\
8619	3.231098\\
8620	3.231087\\
8621	3.231082\\
8622	3.668813\\
8623	5.201439\\
8624	5.20144\\
8625	5.641262\\
8626	6.672104\\
8627	7.2864\\
8628	7.194218\\
8629	6.884305\\
8630	6.730469\\
8631	7.330143\\
8632	7.411723\\
8633	7.696357\\
8634	6.964808\\
8635	6.705537\\
8636	5.646964\\
8637	5.201492\\
8638	5.201438\\
8639	5.201437\\
8640	5.201437\\
8641	4.591984\\
8642	3.231092\\
8643	1.579939\\
8644	0.463073\\
8645	0.322495\\
8646	0.729053\\
8647	1.688744\\
8648	3.048246\\
8649	3.676275\\
8650	5.201438\\
8651	5.201468\\
8652	5.201473\\
8653	5.201477\\
8654	5.201482\\
8655	5.201456\\
8656	5.201457\\
8657	5.201465\\
8658	5.201453\\
8659	5.201454\\
8660	5.201457\\
8661	5.201463\\
8662	5.20147\\
8663	5.201466\\
8664	5.201465\\
8665	4.712496\\
8666	3.328653\\
8667	2.866031\\
8668	1.23791\\
8669	0.753266\\
8670	0.753949\\
8671	1.473314\\
8672	1.727983\\
8673	3.23111\\
8674	3.889616\\
8675	4.511943\\
8676	5.001383\\
8677	5.201438\\
8678	5.201414\\
8679	5.201481\\
8680	5.201463\\
8681	5.68621\\
8682	7.459211\\
8683	7.785712\\
8684	7.615265\\
8685	7.7857\\
8686	7.490724\\
8687	7.289244\\
8688	8.494671\\
8689	7.38107\\
8690	5.910437\\
8691	5.910437\\
8692	4.889169\\
8693	4.366241\\
8694	4.596575\\
8695	5.910436\\
8696	5.910437\\
8697	7.38107\\
8698	6.672065\\
8699	6.672072\\
8700	6.672075\\
8701	6.582059\\
8702	6.671947\\
8703	6.440441\\
8704	6.672074\\
8705	7.287043\\
8706	7.785699\\
8707	8.342479\\
8708	8.02475\\
8709	7.785669\\
8710	7.308166\\
8711	7.785668\\
8712	7.785668\\
8713	6.188912\\
8714	5.201438\\
8715	5.201439\\
8716	4.711394\\
8717	4.560286\\
8718	5.201439\\
8719	5.201438\\
8720	5.662478\\
8721	7.344082\\
8722	7.785668\\
8723	7.785668\\
8724	7.785668\\
8725	7.785668\\
8726	7.785671\\
8727	7.785668\\
8728	7.785671\\
8729	7.904682\\
8730	8.90859\\
8731	8.493398\\
8732	7.785667\\
8733	7.785668\\
8734	6.672085\\
8735	7.785668\\
8736	7.785668\\
8737	5.66249\\
8738	5.20144\\
8739	5.20141\\
8740	4.508931\\
8741	4.149758\\
8742	4.712501\\
8743	5.20144\\
8744	5.201483\\
8745	6.672085\\
8746	7.604462\\
8747	6.672086\\
8748	6.672102\\
8749	6.377036\\
8750	6.672086\\
8751	6.447919\\
8752	6.672009\\
8753	6.971629\\
8754	6.910869\\
8755	6.672098\\
8756	5.201483\\
8757	5.201489\\
8758	5.201437\\
8759	5.201473\\
8760	5.948071\\
};
\end{axis}
\end{tikzpicture}%
    \caption{Predicted reserve prices for the ORDC model}
    \label{fig:ORDC_R2}
\end{figure}