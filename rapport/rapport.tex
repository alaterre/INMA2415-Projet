 \documentclass[a4paper,11pt]{article}
\usepackage{float}
\usepackage{graphicx}
\usepackage{caption}
\usepackage{eurosym}
\usepackage{textcomp}
%\usepackage{minitoc}
\usepackage{pdfpages}

\usepackage{pgfplots}
\usetikzlibrary{external}
\tikzexternalize[prefix=tikz/] % activate!

% ANALYSE NUMERIQUE - DEVOIR 2
%% packages - color - newcommand 
\input{miscellaneous/preambule.tex}

\title{\vspace{-2cm}
\centering \bf
\noindent\makebox[\linewidth]{\rule{\linewidth}{0.1cm}}
LINMA2415 - Quantitative Energy Economics\\
\textsc{Pricing Scarce Capacity in Belgium}
\noindent\makebox[\linewidth]{\rule{\linewidth}{0.1cm}}}
\date{\vspace{-1cm} \today \vspace{-0.5cm}}
\author{}
\newcommand{\norme}[1]{\left\Vert #1\right\Vert}

\setcounter{tocdepth}{2}
\renewcommand*\contentsname{Table of Contents}
\newcommand{\R}{\mathcal{R}}
\newcommand{\G}{\mathcal{G}}
\newcommand{\D}{\mathcal{D}}

\begin{document}

%% page de gare
%\begin{titlepage}
\centering
\BgThispage
\newgeometry{left=1cm,right=1cm,bottom=1cm}
\vspace*{2.5cm}
\noindent
\textcolor{white}{\bigsf Quantitative Energy Economics\\ {\bf\vspace{0.5cm} \Huge INMA2415 - Project}}
\vspace*{2.3cm}\par
\noindent
%
\begin{center}
\textbf{\Large Université catholique de Louvain}  \\
Faculté des Sciences Appliquées \\
Département d’Ingénierie Mathématique \\
\vspace*{0.5cm}
\begin{figure}[!h]
	\centering
	\includegraphics[width=5.5cm]{images/logo.jpg}
\end{figure}
\end{center}
\vfill
\begin{figure}[b]
\centering
\textbf{\large Laterre Alexandre (49-25-11-00)} \\
\textbf{\large Laurent Quentin (XX-XX-11-00)} \\
\textbf{\large Année Académique 2014-2015} \\
\textbf{\large Professeur : Anthony Papavasiliou} \\
\begin{minipage}{0.2\linewidth}
	\includegraphics[scale=0.2]{images/logo_ucl.jpg}
\end{minipage} \hspace{-10pt}
\begin{minipage}{0.2\linewidth}
\centering
    \includegraphics[scale=0.2]{images/logo_epl.jpg}
\end{minipage}
\end{figure}
\end{titlepage}
\restoregeometry

\setcounter{page}{1}
\maketitle
%\tableofcontents
\section{Introduction}
The primary goal of this work is to investigate the basic energy and reserve dispatch co-optimization. We analyze how the price of energy and reserve vary under a variety of market designs.

\section{Description of the Models}
\subsection{Model 1}
% MODEL 1
The first model described in this project is based on simultaneous auction for energy and reserve with fixed procurement of reserves. 
For this first model, some assumptions are made :

\begin{description}
\item[Demand] The demand is assumed completely inelastic, with a valuation of $5000$ \euro $/MWH$. The welfare maximization is therefore, equivalent to a cost minimization.
\item[Reserve] The reserve requirement is fixed at $\R$. Moreover, the reserves are not assumed substituable. Only one type of reserve is considered and needs to be available within the next 15 minutes.
\item[Imports/Exports] The import/export of energy is assumed fixed and independent from the price of energy. Since the import/export quantities are considered as input data, there are no transmission constraints.
\item[Producer bids] We assumed that producers submit bids that truthfully represent their cost to the market operator.
\end{description}

Once those assumptions made, we can describe the economic dispatch optimization problem. \\

\begin{center}
\boxput*(0,1){\colorbox{white}{\textbf{ Economic Dispatch with Reserve [EDR]}}}{
\setlength{\fboxsep}{10pt}
\fbox{\begin{minipage}{10cm} \vspace{0.2cm}
$$ \min\limits_{r_g, p_g \geq 0} \quad  \sum_{g \in \G} \int_0^{p_g} MC_g(x) dx $$
$$
\begin{array}{llll}
(\lambda)				& \sum_{g \in \G} p_g + f^{t}_{in} - f^{t}_{out} + p^{t}_{r} \geq \D^{t} & & (1) \\
(\mu)					& \sum_{g \in \G} r_g \geq \R & & (2) \\
							& r_g \leq 15\: R^{t}_g & & (3) \\
							& p_g + r_g \leq P^{t}_g & & (4) \\
\end{array}
$$
\vspace{0.1cm}
\end{minipage}}}
\end{center} \vspace{0.5cm}

Where $MC_g$ is the marginal cost of a generator $g$ to supply power\footnote{The gas price modulator is included in the computation of the objectif function.}. $p_r$ is the power production coming from renewable energies (wind, solar, hydro). The amount of available reserves depends not only on the maximal capacity (4) of a generator, but also on its ramp constraint (3). Any reserve that needs to be available within the next 15 minutes is limited by 15 times the ramp rate. \newpage

The EDR problem is solve for each hour of the year (the $t$ exponent marks the time-dependency, \: $t \in \left[ 1 \ldots 8760 \right]$). Notice that there are no linking constraints between two consecutive hours. Since the time step is one hour and that generators have generally a ramp rate above $1\%$ of their capacity. Therefore, a generator can almost reach any production rate within the hour.\\

Since the maximum production rate of generators may vary from one hour to another because of possible outages, and that the ramp constraints depend on this maximum production rate, The parameter $R_g$ may change over time, in accord with the production rate. Indeed, it's quite logical that partial generators are not able to react as fast as if they had been in good condition. \\

Even if the goal of this model isn't to meet the demand in every possible occurrence of contingency, while minimizing the cost of the base case, it takes into account the variations of the capacity of the generator under reported failures. \\

Once the market operator solves [EDR], $\lambda$ and $\mu$ are announced as the uniform price for energy and reserve respectively.
\subsection{Model 2}
% MODEL 2

Throughout this second model, we will analyze the influence of neighboring markets on generator profitability. The first task is to represent the neighboring markets with a supply function just like any other generators. The figures [\ref{fig:France}] [\ref{fig:Netherlands}] show the pairs quantity exchange between Belgium and France/Netherlands, at a given market price. The marginal cost of Netherlands behave as expected, i.e. the change in the total cost that arises when the quantity produced has an increment by unit, becomes more and more important. On the contrary, the marginal cost of the 'french' generator decreases when more power is sent to Belgium. This leads to a non-intuitive behavior. Indeed, more the generator produces, less the next unit of power will be expensive. Unfortunately, if both generators are represented with those marginal cost functions, our problem will become non-convex. An other approach which makes sense, is to combined the two import supply functions into one. Meaning that the importations are considered as coming from a unique external generator. The result is shown on figure [\ref{France_Netherlands}]. The supply function is approximated with a linear regression : $\text{MC}(p)= 31.2424 p + 0.005$. Notice that supply function seems quite responsive to the market price. For a small difference in the Belgian price, the importation drastically vary. \\

\vspace{0.5cm}

\begin{minipage}{0.495\textwidth} 
\begin{figure}[H]
    \centering
    \newlength\fheight 
    \newlength\fwidth 
    \setlength\fheight{4cm}
    \setlength\fwidth{0.75\textwidth}
    % This file was created by matlab2tikz.
% Minimal pgfplots version: 1.3
%
%The latest updates can be retrieved from
%  http://www.mathworks.com/matlabcentral/fileexchange/22022-matlab2tikz
%where you can also make suggestions and rate matlab2tikz.
%
\definecolor{mycolor1}{rgb}{0.04314,0.51765,0.78039}%
\definecolor{mycolor2}{rgb}{0.84706,0.16078,0.00000}%
%
\begin{tikzpicture}

\begin{axis}[%
width=\fwidth,
height=\fheight,
at={(0\fwidth,0\fheight)},
scale only axis,
clip mode=individual,
separate axis lines,
every outer x axis line/.append style={black},
every x tick label/.append style={font=\color{black}},
xmin=-1500,
xmax=3100,
xlabel={Power Exchange},
xmajorgrids,
every outer y axis line/.append style={black},
every y tick label/.append style={font=\color{black}},
ymin=-50,
ymax=200,
ylabel={Market Price [\euro/MWh]},
ymajorgrids,
title style={font=\bfseries},
title={France}
]
\addplot [color=mycolor1,line width=1.0pt,mark size=0.3pt,only marks,mark=*,mark options={solid},forget plot]
  table[row sep=crcr]{%
1640	15.15\\
1218	12.96\\
1135	12.09\\
1332	11.7\\
1036	11.66\\
918	11.35\\
926	9.85\\
155	9.54\\
-706	9.49\\
-510	11.64\\
-278	11.94\\
-319	13.15\\
-276	15.24\\
-3	13.69\\
-84	12.43\\
-47	9.92\\
48	12.12\\
268	15.24\\
-87	15.73\\
-121	17.73\\
294	15.63\\
498	13.93\\
876	15.1\\
1307	12.95\\
1035	9.62\\
816	7.64\\
608	4.96\\
633	0.06\\
655	1.05\\
695	7.08\\
1069	12.5\\
1087	21.31\\
1435	30.44\\
1566	35.48\\
1734	33.06\\
1863	33.78\\
1798	37.97\\
1830	37.42\\
1812	36.24\\
1929	32.18\\
2158	33.56\\
1770	52.94\\
1645	66.7\\
1557	53.53\\
1688	39.54\\
1457	35.9\\
1311	35.47\\
1544	30.64\\
1371	27.4\\
1425	25.23\\
1371	15.63\\
1165	8.74\\
1192	11.28\\
1490	12.34\\
1178	26.02\\
1566	31.66\\
1816	31.96\\
1768	31.94\\
2029	30.96\\
2096	31.45\\
1909	39.05\\
1839	30.99\\
1704	30.46\\
1720	30.43\\
1807	31.12\\
2181	36.98\\
1601	34.97\\
1731	42.18\\
1467	30.39\\
1273	26.32\\
1562	31.82\\
1321	32.22\\
1906	11.94\\
2232	10.5\\
1798	7.93\\
1630	5.23\\
1556	4.86\\
1612	8.96\\
842	8.72\\
1013	9.91\\
1533	11.5\\
1873	12.76\\
1918	13.36\\
2024	14.03\\
1679	15.64\\
1983	14.24\\
2488	14.48\\
3000	30\\
2742	12.86\\
2560	16.79\\
2046	21.08\\
1924	18.1\\
1636	16.14\\
2078	13.84\\
2220	14.89\\
2413	17.34\\
2458	15.46\\
2270	14.69\\
2472	13.33\\
2355	10.96\\
2288	9.83\\
2307	11.66\\
1055	11\\
1047	11.38\\
1269	13.77\\
1696	27.42\\
1799	30.9\\
1859	32.45\\
1866	32.07\\
2052	29.91\\
2238	28.41\\
2293	16.89\\
2281	15.39\\
3000	46.14\\
2673	40.13\\
2548	38.49\\
2249	34.67\\
1866	29.4\\
1984	28.81\\
2107	22.08\\
1736	13.84\\
1377	11.89\\
1219	9.85\\
1078	5.09\\
1161	3.79\\
1267	8.24\\
2179	17.29\\
2645	48.6\\
2405	46.6\\
2623	40\\
3000	67.8\\
3000	67.3\\
3000	46.08\\
3000	46.08\\
2978	34.5\\
3000	46.08\\
3000	48.99\\
3000	84.71\\
2927	54.57\\
2456	50.7\\
2750	41.75\\
2331	28.34\\
2700	38.52\\
2472	33.69\\
1965	9.26\\
2259	9.98\\
1937	3.88\\
1823	0.07\\
1887	2.05\\
2291	6.52\\
2527	26.07\\
3000	34.99\\
2800	49.42\\
2154	52\\
1997	52\\
1760	54.36\\
1516	52.78\\
2090	49.99\\
2334	45.93\\
2581	43\\
2574	46\\
2552	54.44\\
2360	61.93\\
1646	56.29\\
2399	40.02\\
2560	31.58\\
2618	36.08\\
2414	33.51\\
2247	22.01\\
2469	14.61\\
2357	12.67\\
2227	11.34\\
2351	9.99\\
2650	22.41\\
2330	30.31\\
1174	46.42\\
2054	57.96\\
1739	57.82\\
1350	53.7\\
1418	54.51\\
1845	56.46\\
2302	54.3\\
2535	51.11\\
3000	47\\
3000	52.06\\
3000	70.01\\
2769	65\\
2233	61.89\\
2843	48.39\\
2814	38.97\\
2962	39.86\\
2637	38.79\\
2188	27.18\\
2392	26.5\\
2061	24.83\\
1841	18.34\\
1874	12.96\\
2221	25.04\\
1521	34.65\\
1211	48\\
1987	56.97\\
1749	48.54\\
1776	46.99\\
1771	47.38\\
2073	48.92\\
2262	46.65\\
2237	41.82\\
2303	37.84\\
2467	43.83\\
2132	52.66\\
1996	62.06\\
1167	59.78\\
1339	47.89\\
2323	40.66\\
2362	45.35\\
2216	44.29\\
1923	42.64\\
1703	38\\
2406	32.2\\
2564	16.27\\
2661	15.49\\
2741	33.77\\
2425	46.7\\
2142	62.87\\
2849	59.51\\
2720	61.62\\
2607	59.99\\
2677	58.32\\
2015	57.73\\
2552	53.76\\
2863	50.84\\
2953	46.87\\
3000	53.65\\
3000	74.86\\
2571	66.02\\
1883	61.5\\
1879	57.3\\
2089	46.82\\
1188	50\\
1203	47.33\\
1238	40.36\\
1382	28.17\\
1172	35.03\\
1822	28.26\\
2377	26.76\\
1973	28.27\\
1037	34.12\\
418	38.34\\
1890	43.65\\
1914	45\\
2264	46.36\\
2201	46.34\\
1783	49.72\\
2150	41.89\\
2111	37.94\\
2228	38.34\\
2513	40.28\\
2938	51.79\\
2631	59.12\\
2304	58.65\\
1830	50.94\\
2302	44.8\\
2118	52.81\\
1346	48.69\\
2333	43.47\\
1767	40\\
2116	25.86\\
2012	16.97\\
2774	14.51\\
2657	16.47\\
2793	16.18\\
2805	23.1\\
2719	31.5\\
3000	44.33\\
2188	25.75\\
2025	23.98\\
1923	29.73\\
1806	24.67\\
2560	25.7\\
2624	27.07\\
2235	31.11\\
2625	39.66\\
2364	41.96\\
2001	46.09\\
1752	37.35\\
1567	32.85\\
919	37.62\\
1125	32.17\\
1507	26\\
2464	27.86\\
2291	27.6\\
2134	13.45\\
2239	14.24\\
2247	26.71\\
1770	33.11\\
1289	50.84\\
1910	51.45\\
2110	53.89\\
2600	64.36\\
2600	65.14\\
2484	55.89\\
2600	55\\
2600	54.94\\
2600	54.97\\
2280	57.93\\
2600	67.48\\
2303	64.41\\
1463	60.71\\
1491	51.62\\
2600	44.67\\
2600	43.21\\
2600	43.21\\
1257	31.14\\
991	29.65\\
1560	29.56\\
2281	24.17\\
2361	19.72\\
2003	30.34\\
1062	36.98\\
927	60.09\\
764	65.94\\
878	66.06\\
1131	66.95\\
1264	67.51\\
1005	60.02\\
1078	55.01\\
1354	52.54\\
1598	51.86\\
1914	55.88\\
1842	70.68\\
824	68\\
511	53.7\\
591	54.94\\
1935	45.14\\
1785	44.94\\
1585	39.99\\
1761	33.03\\
1609	30.66\\
1759	29.92\\
2137	29.63\\
1993	29.54\\
1782	30.65\\
1096	39.94\\
1645	55.67\\
1401	62.65\\
1511	58.07\\
1676	55.47\\
1634	53.9\\
1245	54.16\\
1225	54.35\\
1371	50.75\\
1895	50.51\\
2219	52.1\\
2149	56.49\\
1073	62.96\\
705	61.29\\
1685	47\\
2505	38.6\\
1828	39.46\\
1334	35.75\\
1673	30.06\\
2303	28.92\\
2415	25.6\\
2242	21.42\\
2292	18.42\\
2262	28.66\\
1409	33.35\\
1694	50.09\\
1999	48.34\\
1999	51.57\\
1932	50.84\\
2027	52.2\\
2291	51.52\\
2800	55\\
2800	54.94\\
2800	53.62\\
2800	55\\
2800	79.94\\
2765	55.55\\
2542	55\\
2642	46.02\\
2433	35.23\\
2596	37.69\\
2800	38\\
2091	29.53\\
2800	28.17\\
2474	19.58\\
2313	11.23\\
2385	11.18\\
2318	20.74\\
1726	34.03\\
1784	48.97\\
1963	49.96\\
2592	47.71\\
2570	47.91\\
2428	47.96\\
2449	48.68\\
2610	46.17\\
2740	42.32\\
2800	47.44\\
2800	48.87\\
2800	68.99\\
2538	52.75\\
1894	49.43\\
2522	43.74\\
2549	33.33\\
2676	34.45\\
2107	35.9\\
2113	30.5\\
1990	29.79\\
2511	28.58\\
2719	26.79\\
2610	18.63\\
2455	22.25\\
1692	22\\
1707	30.25\\
2009	33.34\\
2397	39.96\\
2360	35.17\\
2222	32.44\\
1857	31.28\\
1928	30.89\\
2141	30.68\\
2302	30.88\\
2763	33.17\\
2800	54.53\\
2800	45.05\\
2552	39.72\\
2579	30.21\\
2326	28.63\\
2445	30.4\\
1968	38.63\\
1462	36.1\\
1407	30.14\\
2475	26.15\\
2272	9.48\\
2248	12.8\\
2304	16.41\\
1384	12.18\\
1478	15.76\\
1842	29.99\\
1965	37.57\\
2526	44.38\\
2629	46.47\\
2666	49.96\\
2761	44.12\\
2818	35.64\\
2917	27.42\\
2919	29.67\\
3000	60.71\\
3000	60\\
2925	46.54\\
2690	42.51\\
2780	40.02\\
2931	41.86\\
2611	38\\
2288	34.75\\
3000	29.46\\
3000	29.21\\
3000	14.99\\
3000	29.33\\
2721	25.5\\
2022	45.93\\
1738	71.97\\
1047	70.92\\
1436	64.5\\
1149	62.93\\
1337	64.44\\
1097	59.61\\
1696	58.04\\
1896	54.16\\
2189	49.7\\
2319	52.53\\
2319	64.91\\
1370	77.99\\
785	63.97\\
1517	62.7\\
2894	50.4\\
2285	58.17\\
2469	49.38\\
2146	44.79\\
2243	34.35\\
2973	30.65\\
3000	30.3\\
3000	30.65\\
2620	33.68\\
1778	45.86\\
1907	59.91\\
1567	62.98\\
1183	60.19\\
1043	63.32\\
1231	65.65\\
1176	62.91\\
1583	61.46\\
1411	57.41\\
1826	50.81\\
1959	54.09\\
2204	60.23\\
783	60.57\\
535	62.97\\
628	52.97\\
983	48.03\\
1080	51.45\\
1611	47.44\\
1518	45.67\\
1722	43.09\\
1634	34.97\\
1786	31.02\\
2314	31.14\\
1066	35.41\\
432	48.5\\
323	67.49\\
-511	66.49\\
-602	65.66\\
-656	60.88\\
-836	62.79\\
-950	61.44\\
-543	59.37\\
-415	58.96\\
-284	54.76\\
47	61.62\\
262	71.58\\
-686	80.06\\
-871	64.04\\
-852	62\\
-679	53.5\\
-847	61.58\\
-590	53.5\\
1322	41.81\\
1613	41.5\\
1780	39.22\\
2514	31.32\\
2711	31.17\\
1824	34.03\\
1435	43.13\\
857	60.59\\
-86	65.1\\
-54	65.05\\
77	64.92\\
146	62.17\\
257	60.92\\
715	60.91\\
977	54.77\\
1453	49.69\\
1629	48.08\\
1572	55.17\\
387	63.96\\
-240	57.53\\
897	53.78\\
2051	45.45\\
1678	53.8\\
1934	46.91\\
1206	35.05\\
1831	32.29\\
2398	30.34\\
2440	29.99\\
2538	30.53\\
1881	32.17\\
1262	43.42\\
180	61.6\\
15	62.5\\
324	63.3\\
293	61.59\\
287	63.28\\
255	60.39\\
492	59.33\\
206	58.97\\
389	53.19\\
700	53.03\\
1060	64.41\\
27	64.94\\
-244	63.71\\
211	54.64\\
1419	50.15\\
2220	47.09\\
1852	49\\
2697	37.68\\
1851	32.05\\
2168	31.1\\
2749	30.32\\
2671	29.83\\
2577	29.04\\
2701	30.9\\
1969	31.58\\
2306	39.27\\
2434	42.04\\
2563	43.63\\
2517	44.11\\
2329	45.82\\
2399	37.68\\
2230	32.69\\
2211	32.69\\
2249	33.57\\
2520	38.13\\
2323	48.04\\
2317	43.14\\
1927	32.57\\
1589	30.05\\
1814	32.86\\
1268	38.23\\
1130	22.11\\
1338	19.14\\
1359	14.85\\
1405	13.14\\
1377	13.22\\
1575	12.73\\
1401	9.53\\
1407	14.05\\
1454	14.04\\
1884	22.19\\
1971	22.69\\
1833	22.85\\
1456	23.42\\
1706	19.08\\
1472	15.23\\
1248	12.89\\
1258	8.26\\
1745	17.66\\
1354	22.42\\
1340	26.7\\
1868	31.96\\
1645	23.75\\
1966	28.24\\
2000	20.29\\
1402	12.63\\
1529	11.74\\
1649	12.81\\
1495	7.1\\
1606	8.77\\
2034	13.09\\
1853	30.14\\
1534	47.43\\
1649	51.07\\
1509	52.47\\
1338	53.47\\
1256	52.73\\
1138	56.05\\
1318	50.71\\
1357	51.92\\
1330	51.92\\
1214	51.47\\
1228	55.53\\
252	79.92\\
-114	64.25\\
953	47.25\\
1112	38.71\\
501	40.2\\
89	38.51\\
819	36.41\\
1190	32.92\\
1335	30.28\\
1647	29.57\\
1793	30.16\\
1468	31.21\\
483	41.47\\
178	55.05\\
457	56.58\\
743	54.52\\
800	52\\
727	52.44\\
515	49.94\\
1081	47.44\\
1232	46.5\\
1497	48\\
1644	49.77\\
1335	57.75\\
631	70\\
236	62.09\\
508	56.94\\
1443	47.4\\
459	53.42\\
54	48.28\\
1198	38.65\\
1424	30.54\\
1556	29.63\\
1918	28.98\\
1532	29.55\\
1306	31.08\\
182	37.1\\
-440	55.38\\
-678	52.95\\
-718	52.9\\
-587	57.44\\
-693	56.78\\
-846	53.55\\
-892	54\\
-667	50.01\\
-30	47.44\\
253	47.02\\
-466	61.52\\
-629	83.27\\
-906	72.99\\
-1075	54.43\\
-767	49.5\\
-615	54.11\\
-513	52.05\\
-34	37.75\\
597	34.94\\
450	34.29\\
1204	29.92\\
1501	30.32\\
300	34.43\\
400	45.2\\
406	53.07\\
-925	54.73\\
-347	64.78\\
-283	59.99\\
-212	64.16\\
-345	59.72\\
115	54.48\\
348	51.89\\
248	51.5\\
535	51.56\\
733	55.12\\
-279	67.98\\
-641	59.67\\
-206	50.5\\
350	45.81\\
702	46.52\\
58	47.1\\
727	42.96\\
499	42.5\\
768	36.54\\
800	33.06\\
1099	31.95\\
965	40.03\\
1006	46.9\\
282	69.94\\
-504	71.95\\
15	72.19\\
325	65.88\\
124	60.93\\
-169	52.93\\
305	48.14\\
874	45\\
833	44.1\\
1238	44.23\\
873	52.92\\
-423	61.76\\
-1009	56.35\\
-31	49.08\\
318	40\\
-161	41.99\\
-528	39.17\\
629	31.04\\
812	28.94\\
1261	26.63\\
1403	20.87\\
1286	20.18\\
1324	20.71\\
1523	26.82\\
721	28.7\\
972	37.27\\
1311	42.65\\
1418	43.28\\
1417	44.55\\
1187	48.59\\
1151	40.24\\
1017	32.87\\
1154	32.8\\
1254	33.28\\
1477	43.56\\
1131	50.02\\
1134	44.94\\
1458	41.24\\
970	35.81\\
897	35.01\\
563	35.96\\
-562	41.62\\
-480	31.96\\
-6	25.8\\
1223	19.38\\
1107	13.18\\
1231	14.07\\
987	16.54\\
540	18.85\\
508	22.07\\
469	30.83\\
750	31.96\\
905	31.82\\
908	34.15\\
1041	32.14\\
964	30.8\\
1635	31.03\\
2169	32.95\\
1661	36.15\\
1659	48.68\\
1644	54.57\\
1386	53.78\\
1216	45.76\\
817	49.81\\
53	48.84\\
422	44.37\\
182	37.8\\
938	36\\
1134	31.56\\
1058	31.52\\
631	36.19\\
298	54.1\\
504	71.89\\
3	72.94\\
-8	67.5\\
50	57.18\\
19	54.63\\
-27	52.9\\
326	51.96\\
647	46.82\\
906	47.14\\
1098	46.76\\
1190	54.09\\
395	67.8\\
75	59.96\\
1158	57.98\\
1838	48.72\\
1251	52.2\\
718	52.27\\
1076	44.42\\
286	44.96\\
716	40.9\\
1210	30.63\\
1145	30\\
379	32.38\\
29	47.13\\
139	60.3\\
-8	58.49\\
404	53.88\\
948	51.99\\
1007	52.83\\
621	54.68\\
993	51.68\\
1349	48.05\\
1447	47.59\\
1614	48.03\\
1847	53.05\\
1027	63.99\\
507	55.73\\
987	48\\
2055	41.26\\
1689	44.08\\
1313	42.46\\
762	30.53\\
511	29.98\\
1187	28.99\\
1701	25.71\\
1733	18.11\\
1557	28.08\\
953	33.51\\
1012	45.16\\
1070	49.94\\
1041	52.61\\
1313	58.57\\
1306	54.16\\
1064	48.08\\
1233	52.2\\
1327	50.09\\
1915	43.92\\
2099	42.55\\
2049	49.69\\
1116	67.38\\
651	53.32\\
1323	36.79\\
-1030	57.1\\
-1358	52\\
-1296	45\\
-559	42.28\\
-623	39.94\\
-326	34.79\\
551	28.27\\
615	28.67\\
-903	32.2\\
-915	44\\
-1230	55.87\\
-1197	56.7\\
-1301	58.84\\
-1122	61.77\\
-1145	60.43\\
-1275	57.28\\
-736	55.91\\
-18	55.29\\
157	49.94\\
670	47.89\\
639	55.89\\
-72	66.01\\
-958	60.02\\
-396	48.05\\
993	37.12\\
1063	33.71\\
895	39.61\\
1657	23.64\\
2034	19.51\\
1743	14.66\\
1560	12.18\\
1627	8.54\\
1588	15.42\\
1906	30.68\\
1560	47.79\\
1691	45.71\\
1956	47.63\\
2201	47.94\\
2441	51.34\\
2217	48.37\\
2255	42.94\\
2252	36.86\\
2232	32.31\\
2317	31.3\\
2606	36.16\\
2285	59.92\\
1969	54.71\\
2077	43.31\\
1778	30.49\\
2265	38.37\\
2310	40.84\\
1726	28\\
1160	26.43\\
1670	25.9\\
1383	24.93\\
1268	10.52\\
1262	16.2\\
1474	24.04\\
649	27.13\\
889	29.51\\
1241	29.48\\
1591	28.07\\
1671	26.76\\
1611	28.03\\
1693	25.5\\
1577	25.96\\
2287	28.81\\
2088	28.03\\
1612	31.17\\
1656	42.58\\
1379	33.88\\
982	27.57\\
1077	24.29\\
1241	27.55\\
1636	23.6\\
1162	14.23\\
663	13.02\\
691	12.87\\
467	11.77\\
278	9.72\\
384	9.36\\
616	10.63\\
116	10.54\\
375	11.04\\
727	19.16\\
1002	17.7\\
1225	17.29\\
1463	14.68\\
1595	13.16\\
1376	10.5\\
1374	8.97\\
1319	12.42\\
1416	14.69\\
1805	35.79\\
1541	36.81\\
1593	31.23\\
1241	27.97\\
1278	30.04\\
1916	25.31\\
775	36.13\\
1215	33.87\\
1887	27.22\\
1931	23.86\\
1001	23.08\\
1188	33.18\\
25	45.43\\
482	59.42\\
425	60.99\\
213	58.79\\
-421	60.02\\
-254	54.96\\
-414	56.12\\
404	56.82\\
343	55.63\\
601	51.47\\
967	47.44\\
924	54.94\\
-47	79.49\\
-357	54.32\\
-75	50.06\\
821	46.57\\
753	47.59\\
1349	42.83\\
1762	30.75\\
1705	32.07\\
1467	29.45\\
1922	28.27\\
2226	28.05\\
1641	30.72\\
761	41.12\\
646	53.8\\
643	55.34\\
1003	56.62\\
1003	54.1\\
923	51.74\\
614	50.38\\
880	49.25\\
1493	42.49\\
1768	39.56\\
1560	41\\
786	51.9\\
604	58.97\\
95	67.49\\
244	50\\
1152	46.48\\
627	49.48\\
760	44.09\\
764	45\\
749	41.94\\
666	37.84\\
813	28.43\\
880	29.36\\
24	36.72\\
-464	45.54\\
-149	59.3\\
-82	65.77\\
-84	68.42\\
103	60.87\\
-128	55.9\\
-147	50.33\\
505	49.42\\
848	45.92\\
1165	43.34\\
1721	40.93\\
1049	44.99\\
254	58.94\\
-436	54.37\\
189	52.03\\
1429	42.78\\
483	46.13\\
437	44.03\\
693	42.97\\
909	34.54\\
1071	32.23\\
1838	18.33\\
1899	12.47\\
1304	26.67\\
980	42.44\\
122	55.88\\
244	53.54\\
606	57.79\\
1016	56.72\\
1459	57.98\\
1284	55\\
1772	52.89\\
1771	52.35\\
2186	45.25\\
2803	42.91\\
2121	48.66\\
975	68.92\\
730	58.69\\
1662	52.81\\
1786	42\\
2049	40.63\\
1272	38.04\\
2455	33\\
2347	32.72\\
2141	29.33\\
2638	24.42\\
2730	25.63\\
2243	29.44\\
1474	46.4\\
1615	61.12\\
1149	64.8\\
1435	69.25\\
1751	62.12\\
1702	55\\
1793	52.08\\
2397	47.1\\
2421	44.22\\
2480	36.2\\
2516	40\\
2420	46.18\\
1643	53.48\\
1661	48.73\\
2356	36.97\\
2095	30.39\\
2657	30.61\\
2948	42.44\\
1654	27.73\\
1191	17.99\\
1549	12.38\\
1302	10.36\\
1198	10.11\\
1244	10.33\\
1544	11.36\\
801	13.34\\
1026	18.26\\
1004	22.66\\
1436	23.11\\
1402	23.08\\
1145	23.76\\
1246	22.48\\
1594	21.11\\
1665	19.33\\
1658	20.46\\
1550	27.73\\
1591	39.29\\
1277	44.85\\
1131	28.93\\
1073	29.36\\
1193	41.5\\
1207	41.56\\
1401	25.49\\
1101	14.46\\
1379	12.35\\
1228	9.06\\
1244	5.3\\
1319	8.08\\
1388	11.12\\
926	9.53\\
904	10.73\\
1100	13.37\\
1159	16.81\\
1153	17.86\\
992	18.23\\
987	13.15\\
898	11.38\\
1100	10.48\\
1359	12.22\\
1723	15.26\\
2119	40.15\\
2214	47.98\\
2134	46.83\\
2047	34.73\\
2368	39.08\\
2487	31.53\\
2157	31.59\\
2491	25.69\\
2404	23.33\\
2290	17.52\\
2321	19.63\\
2715	26.47\\
1814	43.79\\
1503	58.75\\
1737	53.31\\
1954	54\\
2018	49.32\\
1989	50\\
1975	44.22\\
2450	42.07\\
2485	34.96\\
2816	37.25\\
2057	40.33\\
2080	43.03\\
786	82.04\\
643	71.74\\
388	56.51\\
1011	47.19\\
674	49.94\\
1333	43.43\\
1530	31.95\\
1780	31.66\\
1537	30.09\\
1488	28.86\\
1623	29.2\\
1324	31.3\\
824	42.03\\
-252	56.19\\
-424	54.9\\
-444	53.79\\
-314	51.9\\
-112	50.74\\
-253	51.42\\
51	48.48\\
785	43.27\\
1051	43.44\\
1196	45.72\\
1417	53.91\\
525	73.68\\
63	59.98\\
906	45.94\\
2081	39.68\\
1756	41.9\\
1590	36.47\\
1784	32.16\\
1840	29.65\\
1962	28.46\\
2266	27.67\\
2472	28.51\\
1538	29.34\\
1496	38.19\\
961	47.45\\
1164	49.44\\
1544	52.21\\
1475	53.93\\
1669	53.96\\
1361	50.76\\
1859	49.48\\
1781	49.37\\
1826	47.95\\
2021	44.71\\
1743	47.44\\
504	70\\
218	59.54\\
524	54.27\\
1741	47.02\\
2417	44.18\\
1727	44.96\\
1577	31.28\\
1916	29.47\\
2211	28.87\\
2016	27.75\\
2318	25.59\\
1226	27.85\\
1113	34.51\\
1064	46.28\\
953	48.34\\
1346	48.28\\
2074	44.93\\
2193	46.95\\
1932	47.58\\
2667	44.94\\
2825	40.02\\
2906	34.96\\
2898	32.45\\
3000	41.78\\
2565	55\\
2242	52.17\\
2357	47.44\\
2323	39.95\\
2667	41.78\\
2565	41.49\\
2727	32.91\\
2789	22.15\\
2552	20.84\\
2427	19.42\\
2509	21.02\\
1935	26.64\\
2712	35.83\\
2506	46.31\\
2436	49.03\\
2508	52.95\\
2718	50.07\\
2874	48.45\\
2518	42.88\\
2665	37.61\\
2544	32.16\\
2556	32.1\\
2628	31.68\\
2594	42.9\\
2493	49.44\\
1437	49.96\\
1998	49.08\\
1848	40.71\\
1740	40.93\\
1294	41\\
652	35.11\\
522	29.96\\
952	26.09\\
1848	21.02\\
1763	18.67\\
1757	21.15\\
1526	27.93\\
876	29.38\\
1570	32.92\\
2002	37.43\\
2305	37.35\\
2446	36.86\\
2372	38.85\\
2198	35.47\\
2172	29.59\\
2021	31.99\\
1987	31.33\\
2247	37.71\\
2124	50.91\\
1957	53.1\\
1578	44.22\\
1591	36.72\\
1324	40.34\\
1393	36.96\\
1837	26.11\\
1264	21.63\\
1499	22.07\\
1558	19.48\\
1441	16.28\\
1427	19.86\\
1420	15.73\\
463	15.73\\
-8	17.1\\
309	16.63\\
713	17.46\\
843	16.77\\
880	14.68\\
1783	12.98\\
1650	11.26\\
1736	7.45\\
1511	10.84\\
1438	11.28\\
1779	24.31\\
1875	35.49\\
2004	27.81\\
1799	23.11\\
1925	27.1\\
1861	25.71\\
1361	14.5\\
1816	15.09\\
1598	17.22\\
1499	13.68\\
1568	14\\
1167	18.12\\
1233	35.66\\
1402	46.95\\
1348	50\\
1837	46.65\\
1978	45.98\\
1853	41.79\\
1426	40.46\\
1979	39.32\\
2233	35.99\\
2457	31.07\\
2661	32.97\\
2700	46.16\\
2700	64.95\\
2444	64.73\\
2223	48.99\\
1987	35.43\\
1871	37.44\\
1435	33.87\\
742	26.22\\
1193	23.87\\
1472	22.77\\
1638	20.29\\
1735	18.97\\
1328	25.94\\
857	36.97\\
945	46.95\\
1054	46.66\\
1603	46.12\\
1353	44.59\\
1385	44\\
1574	44.94\\
2023	40.49\\
2201	36.09\\
2477	35.3\\
2590	33.71\\
2423	42.61\\
1762	72.68\\
1792	62.52\\
1382	46.87\\
2007	39.88\\
1345	42.89\\
1080	39.59\\
1732	36.05\\
1820	30\\
1705	27.29\\
1480	26.91\\
1528	27.86\\
1033	28.48\\
1471	39.22\\
909	45.05\\
1391	49.24\\
2166	51.3\\
2353	50.42\\
2092	50\\
1957	46.91\\
1962	46.42\\
1998	44.96\\
1960	43.35\\
1842	41.81\\
1536	43.99\\
1115	71.19\\
539	72.95\\
224	49.85\\
580	45.59\\
1179	42.44\\
1262	40.05\\
1319	33.99\\
1430	32.03\\
1137	30.28\\
970	29.61\\
973	29.48\\
413	32.18\\
1074	41.54\\
1060	53.34\\
718	50.1\\
1290	47.1\\
1285	48.14\\
929	46\\
1101	44.24\\
1290	44.81\\
1389	43.22\\
1826	41.94\\
2088	40.62\\
1477	51.76\\
1369	64.96\\
1772	57.83\\
1784	48.66\\
1977	43.21\\
1947	41.45\\
1845	40.16\\
1575	36.25\\
1803	34.94\\
1757	30.19\\
1785	28.03\\
1709	28.54\\
1102	29.94\\
1718	39.6\\
1235	47.45\\
1853	49.45\\
2032	49.77\\
2366	49.12\\
2477	47.42\\
2635	46.72\\
2667	44.7\\
2637	42.31\\
2618	41.92\\
2579	42.98\\
2487	47.3\\
1405	64.68\\
951	59.92\\
1561	47.28\\
1625	39.75\\
1246	42.95\\
997	42\\
1191	40.61\\
1190	35.72\\
1186	33.79\\
1171	30.22\\
1331	29.28\\
1351	29.33\\
715	28.98\\
1027	33.71\\
1129	39.51\\
1228	47.4\\
1510	50.34\\
1524	48.18\\
1737	47.97\\
1787	43.25\\
1672	38.63\\
1629	36.46\\
1527	33.77\\
1346	41.27\\
929	63.21\\
598	74.76\\
837	45.93\\
905	42.66\\
167	48.38\\
460	44.97\\
-55	44.94\\
-14	39.4\\
29	32.28\\
384	29.31\\
221	27.57\\
166	27.86\\
492	27.33\\
341	27.04\\
-306	27.49\\
313	32.39\\
336	33.69\\
468	33.67\\
399	34.38\\
957	27.45\\
790	23.55\\
697	24.59\\
713	25.42\\
798	28.22\\
678	33.96\\
575	42.62\\
1273	39.44\\
799	31.89\\
44	36.16\\
-537	32.59\\
-165	26.52\\
440	25.11\\
609	24.08\\
936	22.14\\
1104	24.03\\
222	27.01\\
-317	43.93\\
-226	49.94\\
-232	51.33\\
123	53.68\\
514	51.43\\
636	57.13\\
459	47.83\\
598	49.41\\
735	46.48\\
1626	38.95\\
1481	40.98\\
1434	44.96\\
165	72.91\\
-264	80.69\\
99	47.81\\
275	43.99\\
258	45.3\\
498	43.34\\
301	45.03\\
634	39.19\\
955	33.29\\
1293	28.99\\
1447	29.68\\
818	33.28\\
-78	47.18\\
565	53.9\\
-76	57.9\\
25	57.32\\
-140	51.02\\
-181	51.98\\
-300	52.3\\
-22	50\\
71	48.59\\
439	46.51\\
749	45.92\\
956	45.87\\
37	75.64\\
-15	79.38\\
28	54.77\\
620	50.68\\
167	49.01\\
-180	50.54\\
343	40.21\\
-165	39.94\\
12	42.1\\
312	34.72\\
628	33.37\\
555	41.28\\
326	47.32\\
991	62.16\\
780	67.76\\
1288	68.08\\
1488	53.01\\
1480	44.96\\
1392	43.85\\
1176	44.19\\
2080	40.21\\
2391	38.02\\
2516	38.08\\
1769	46.54\\
1100	58.03\\
1005	68.37\\
1797	49.04\\
2065	42.58\\
1451	45.99\\
1021	44.1\\
1233	37.43\\
1238	36.72\\
958	34.14\\
1296	28.81\\
1372	29.19\\
1311	33.98\\
723	42.85\\
-132	55.33\\
-329	56.97\\
241	56.22\\
1056	50\\
968	48.58\\
549	46.41\\
826	44.92\\
1666	40.51\\
2070	34.11\\
2138	37\\
1604	45\\
238	61.59\\
-675	69.75\\
466	53.9\\
1005	42.17\\
52	45.73\\
-202	40.97\\
-805	40.01\\
130	33.98\\
-102	32.22\\
408	28.9\\
391	29.1\\
112	32.06\\
-241	40.9\\
-250	48.26\\
143	53.76\\
492	62\\
1180	49.17\\
967	45.36\\
899	44.82\\
1186	37.93\\
1298	35.64\\
1661	31.97\\
1838	31.94\\
1953	39.36\\
1396	47.09\\
1468	49.26\\
1845	45.95\\
1525	41.92\\
1067	44.28\\
1061	43.68\\
1019	39.97\\
906	33.31\\
537	30.44\\
776	25.71\\
576	24.96\\
625	25.82\\
666	26.92\\
637	29.6\\
1259	31.96\\
1354	31.51\\
1204	28.19\\
1022	26.17\\
939	24.42\\
773	20.02\\
997	15.91\\
1208	12.03\\
1621	16.02\\
2068	22.55\\
2063	35.58\\
1697	42.77\\
1293	30.03\\
1138	25.14\\
1084	25.47\\
622	23.65\\
514	15.67\\
1148	13.75\\
826	12.1\\
657	11.4\\
-99	10.35\\
-90	11.4\\
407	11.49\\
-161	11.4\\
-72	11.55\\
29	12.05\\
-39	11.54\\
449	12.11\\
591	12.26\\
814	10.29\\
840	6.2\\
319	1.75\\
420	1.87\\
822	9.83\\
1459	22.41\\
1755	41.73\\
1658	34.86\\
1366	28.07\\
1421	30.09\\
643	26.92\\
1199	24.26\\
1580	22.62\\
1474	22.73\\
1545	22.76\\
1808	22.11\\
1166	25.57\\
1594	43.5\\
1910	48.28\\
1927	48.12\\
1765	44.76\\
1766	40.9\\
1604	37.42\\
1477	36.46\\
1573	35.85\\
1641	33.8\\
1832	30.13\\
2058	30.19\\
2356	36.36\\
1711	56.07\\
896	75.46\\
867	46.32\\
1707	38.68\\
1474	35.17\\
1118	30.93\\
1017	25.83\\
1130	25.9\\
1053	24.63\\
1572	24.27\\
1707	23.78\\
936	26.5\\
1287	35.08\\
1804	42.55\\
1888	44.58\\
1844	40.03\\
2416	40.2\\
2274	36.07\\
1920	31.88\\
1846	30.65\\
2024	28.18\\
2156	28.72\\
2293	30.91\\
2584	36.26\\
2600	49.94\\
2141	54.1\\
2459	42.56\\
2110	34.04\\
1826	40.03\\
1764	32.99\\
1490	28.87\\
1697	28.04\\
1517	26.65\\
1793	25.15\\
1764	25.69\\
1733	28.09\\
2267	35.8\\
1875	44.54\\
1717	43.73\\
2206	41.27\\
1953	32.59\\
1730	33.67\\
1591	30\\
1661	28.78\\
1863	28.42\\
2092	29.25\\
2277	30.79\\
2700	37.04\\
2415	49\\
2304	75.98\\
2254	48.43\\
2217	42.01\\
2477	38.28\\
2448	32.08\\
2533	31.35\\
1629	30.65\\
2197	29.96\\
2010	30\\
1885	30.11\\
1665	31.43\\
2749	43.03\\
2115	53.46\\
2292	49.84\\
1842	41.27\\
1801	34.74\\
1553	29.77\\
1328	30.58\\
1370	28.42\\
1410	28.78\\
1561	30.95\\
1693	35.54\\
1968	30.84\\
1485	52.76\\
1541	75.55\\
1596	48.69\\
2294	43.63\\
2662	42.87\\
2542	40.69\\
2177	34.96\\
2324	32.01\\
2583	30.99\\
2498	29.57\\
2644	29.65\\
1885	32.89\\
2585	42.04\\
1534	49.94\\
2158	47.93\\
2236	42.94\\
2788	41.26\\
2592	38.83\\
2313	37.01\\
2289	34.91\\
2182	30.96\\
2279	29.53\\
2374	30.91\\
2640	37.91\\
2445	45.07\\
1965	53.95\\
2800	49.94\\
2796	38.64\\
2800	42.44\\
2800	38.99\\
1776	28.64\\
1818	18.86\\
1837	21.1\\
1820	11.3\\
1784	7.58\\
1836	11.81\\
1847	13.36\\
1937	14.02\\
2271	22.83\\
2754	20.01\\
2795	16.52\\
2824	15.8\\
2667	21.57\\
2558	15.2\\
2497	12.55\\
2508	11.78\\
2570	11.26\\
2867	18.81\\
3000	44.94\\
3000	48\\
2870	25.54\\
2623	23.29\\
2634	20.33\\
2378	19.29\\
2040	15.89\\
2455	14.23\\
2149	13.29\\
1972	11.77\\
2050	11.11\\
963	10.12\\
1076	11.23\\
1012	10.91\\
1239	11.44\\
1462	10.77\\
1550	12.34\\
1568	13.02\\
1481	13.95\\
1947	11.07\\
1866	9.33\\
1910	3.07\\
2139	5.41\\
2564	9.43\\
2523	22.78\\
3000	52\\
3000	49\\
2667	21.31\\
2718	20.51\\
2844	28.62\\
2312	29.73\\
1916	26.33\\
2057	23.93\\
2540	20.73\\
2600	21.96\\
1837	26.41\\
2600	42\\
2467	50.81\\
2122	49\\
2600	56.84\\
2600	59.16\\
2600	65.27\\
2600	50\\
2600	50\\
2600	49.96\\
2600	49.08\\
2600	51.87\\
2600	67.78\\
2600	47.44\\
2483	69.94\\
2451	54.86\\
2600	44\\
2600	45\\
2600	44.25\\
2400	31.57\\
2400	32.18\\
2400	29.99\\
2340	24.07\\
2103	24.89\\
2372	28.06\\
2139	40.44\\
2400	49.49\\
2400	50.12\\
2400	63.77\\
2400	55\\
2400	61.99\\
2400	58.13\\
2400	52.24\\
2400	45.13\\
2400	42\\
2400	41.36\\
2400	42.44\\
2400	39.68\\
1405	46.28\\
1675	41.55\\
2400	39\\
2340	32.73\\
2313	30.33\\
2465	22.2\\
2600	22.46\\
2593	14.3\\
2600	22.46\\
2600	29.66\\
2600	27.44\\
2600	42.36\\
2600	49.79\\
2600	47.99\\
2600	48\\
2600	39.01\\
2590	34.72\\
2365	33.41\\
2313	32.59\\
2284	31.48\\
2584	28.19\\
2600	42.44\\
2600	47.59\\
2600	42.44\\
2600	54.97\\
2600	49\\
2600	45\\
2600	44\\
2600	41.93\\
1868	26.47\\
1932	26.72\\
2335	24.54\\
2541	17.18\\
2572	18.05\\
2257	24.57\\
1620	37.54\\
1785	46.15\\
2257	41.33\\
2318	35.3\\
2344	31.3\\
2287	29.01\\
1956	26.8\\
1922	26.56\\
1940	25.76\\
2251	26.52\\
2425	26.84\\
2624	28.95\\
2465	36.94\\
2540	45.07\\
2800	32\\
2592	27.84\\
2790	31.24\\
2246	30.47\\
2265	21.24\\
2396	12.08\\
2236	11.32\\
2267	10.25\\
2527	10.65\\
2379	18.63\\
2800	29.62\\
2800	40\\
2800	39.18\\
2674	38.33\\
2800	44.54\\
2800	63.62\\
2800	57.66\\
2800	55.89\\
2800	45\\
2800	39.96\\
2800	39.94\\
2800	38.93\\
1633	42.43\\
1611	55\\
1441	47.39\\
1549	43.08\\
1328	43.08\\
772	36.5\\
1339	33.94\\
1114	28.54\\
1459	24.08\\
2204	19.07\\
2109	13.66\\
1964	17.92\\
1496	22.76\\
876	24.31\\
1178	29.83\\
1417	32.37\\
1883	32.33\\
1906	32.48\\
1945	31.22\\
2087	27.59\\
1995	27.84\\
1863	26.45\\
1765	27.99\\
2086	31.39\\
1660	38.04\\
1256	45.01\\
1642	36.93\\
1793	31.7\\
1661	30.99\\
1549	24.69\\
1788	22.91\\
1605	20.54\\
1624	20.14\\
1790	20.59\\
2148	17.65\\
2078	19.82\\
1814	17.05\\
1435	16.07\\
1418	21.36\\
1577	24.42\\
1691	26.9\\
1770	30.58\\
1737	29.26\\
2172	24.04\\
2227	23.46\\
2460	18.55\\
2522	17.34\\
2595	19.29\\
2800	40\\
2800	45.12\\
2800	42\\
2636	37.1\\
1945	40.51\\
2111	35.55\\
1737	32.64\\
2061	30\\
1934	27.8\\
1995	25.08\\
1934	25.3\\
1982	28.82\\
2200	40.58\\
2200	57.32\\
1808	57.52\\
1965	54.94\\
2115	42.91\\
2075	42.6\\
1641	43.18\\
1537	44.64\\
1648	43.91\\
1961	41.46\\
2453	40.91\\
2800	40.93\\
1160	47.23\\
719	96.69\\
1857	48.73\\
2744	41.12\\
2388	44.62\\
2640	41.69\\
751	37.3\\
-163	36.95\\
329	33.3\\
583	30.65\\
348	31\\
147	35.52\\
144	42.44\\
-315	45\\
-326	49.47\\
-385	49.99\\
-366	51.15\\
-701	45.85\\
-921	45.62\\
-796	42.97\\
-600	42.92\\
104	39.82\\
-25	40.1\\
234	44.07\\
-58	49.08\\
-652	81.94\\
198	48.12\\
244	43.42\\
-284	42.8\\
-236	46.99\\
-237	39.53\\
-227	41.02\\
-5	40.42\\
543	32.22\\
839	31.13\\
394	34.68\\
445	44.49\\
134	48.93\\
633	52.06\\
870	52.07\\
1085	54.01\\
1133	48.32\\
808	47\\
1173	43.52\\
1250	41.95\\
1626	41.09\\
1831	40.24\\
1631	42.24\\
588	48.68\\
-44	86.74\\
314	46.5\\
1334	42.37\\
605	43.59\\
329	42.44\\
2081	41.11\\
1851	42.44\\
2183	38.41\\
2287	35.02\\
2442	34.24\\
1520	38.08\\
2044	46.51\\
2083	51.57\\
1967	59.9\\
1853	52.44\\
2155	51.19\\
2034	50.58\\
1649	48.79\\
1651	49.4\\
1688	46.77\\
1973	43.27\\
2151	42.07\\
2345	44.45\\
1778	50\\
1278	81.51\\
1280	62.32\\
1187	46.92\\
1786	48.57\\
2118	45.4\\
1829	40.77\\
1247	41.97\\
1940	41.04\\
2028	37.01\\
2194	35.1\\
2492	40.21\\
2072	45.87\\
1677	51.76\\
1720	58.1\\
1642	57.39\\
1742	47.4\\
1774	45\\
2430	42.06\\
2485	40.59\\
2601	40.81\\
2800	39.99\\
2800	40\\
2800	43\\
2205	44.29\\
1529	70\\
1877	45\\
2800	42.44\\
2630	42.44\\
2299	40.06\\
2300	43.99\\
2257	32.81\\
2295	31.02\\
2300	31.11\\
2300	32\\
2300	31.32\\
2300	36.89\\
2300	39.94\\
2300	43\\
2300	47.44\\
2300	45.62\\
2300	42.62\\
2300	40\\
2300	38\\
2300	34.99\\
2300	34.96\\
2300	39.96\\
2300	43\\
2300	43\\
2300	46.17\\
2300	54.96\\
2300	46\\
2300	44.94\\
2300	40.56\\
1850	44\\
1850	34.94\\
1850	29.49\\
1850	29.49\\
1850	30.7\\
1850	35.04\\
1850	35.04\\
1850	39.33\\
1850	40\\
1850	41\\
1850	38.83\\
1850	37.44\\
1850	36.99\\
1850	32.44\\
1850	30.81\\
1850	31.8\\
1850	39.96\\
1850	39.96\\
1850	45\\
1850	69.97\\
1850	59.94\\
1850	55\\
1850	42.44\\
1180	30.91\\
1850	32.46\\
1850	31.28\\
1850	32.2\\
1850	35.77\\
1850	28.14\\
1850	40.69\\
1850	69.53\\
1850	55.47\\
1459	56.01\\
1361	48.33\\
1355	46.08\\
1291	46.7\\
1695	43.73\\
1647	41.79\\
1850	40\\
1850	42.22\\
1850	43.22\\
1850	40.47\\
1659	48.47\\
1850	54.61\\
1850	50\\
1850	44.94\\
1850	44\\
1850	54.48\\
1850	43.64\\
1850	41.06\\
1850	40.5\\
1850	36.26\\
1850	44.94\\
1850	48\\
1850	57.44\\
1662	62.19\\
1386	57.89\\
1643	47.29\\
1592	46.64\\
1791	37.77\\
1850	40\\
1850	39.96\\
1850	38.76\\
1850	39.96\\
1850	42.11\\
1850	73.81\\
1850	50.12\\
1850	45\\
1850	47\\
1850	120\\
1850	120\\
1850	65\\
1850	37.69\\
1850	37.53\\
1850	37.53\\
1850	37.75\\
1850	36.93\\
1850	44\\
1850	49.22\\
1380	62.17\\
1406	54.35\\
1850	48\\
1850	45\\
1850	45\\
1850	44\\
1850	42.12\\
1850	41.19\\
1850	44\\
1850	47.44\\
1850	46\\
1850	52.92\\
1850	58.31\\
1850	53.12\\
1850	44.94\\
1315	40.98\\
1850	34.99\\
1850	31.27\\
1850	31.27\\
1850	32.81\\
1850	28.3\\
1850	29.55\\
1850	38.64\\
1850	55.5\\
1850	52.09\\
1850	49.94\\
1850	50\\
1850	51\\
1850	47.44\\
1850	47\\
1850	44.94\\
1850	44.42\\
1850	45\\
1850	50\\
1850	69.97\\
1850	44.94\\
1850	50\\
1850	47.44\\
1850	63.41\\
1850	64.08\\
1437	37.23\\
1850	34.37\\
1850	34.24\\
1850	30.51\\
1850	30.51\\
1850	31.4\\
1850	37.77\\
1612	46.07\\
1796	51.91\\
1813	56.15\\
1850	63.46\\
1850	120\\
1850	58.9\\
1850	47.44\\
1850	44.94\\
1850	40\\
1850	40\\
1850	40\\
1850	39.94\\
1751	40.99\\
1707	41.76\\
1735	38.72\\
1850	65.37\\
1850	43\\
1850	34.3\\
1850	37.84\\
1850	34\\
1850	34\\
1850	30.74\\
1850	30.74\\
1850	29.5\\
1850	34.44\\
1850	43.18\\
1850	46\\
1850	45\\
1850	47.44\\
1850	42.8\\
1850	35.87\\
1850	34.44\\
1850	34.44\\
1850	33.61\\
1850	35\\
1850	42.44\\
1850	42.44\\
1850	49.99\\
1850	47.44\\
1850	42\\
1850	44.01\\
1850	35.16\\
1850	42.16\\
1850	34.94\\
1850	34.21\\
1850	31.5\\
1850	33.97\\
1850	24.25\\
1850	24.3\\
1850	25.31\\
1850	34.94\\
1850	38.99\\
1850	41.96\\
1850	47.44\\
1850	39.94\\
1850	35.15\\
1850	34.9\\
1850	34.94\\
1850	39.96\\
1850	42.17\\
1850	44.94\\
1850	59.94\\
1850	54.96\\
1850	48\\
1850	35.99\\
1850	29.88\\
1850	27.73\\
1850	28.85\\
1850	29.85\\
1850	29.75\\
1850	29.99\\
1850	41\\
1828	49.94\\
1781	53.05\\
1850	51.06\\
1850	47.45\\
1850	44.61\\
1850	38.94\\
1786	37.09\\
1850	36.16\\
1850	36.26\\
1850	36.31\\
1850	39.12\\
1850	45.31\\
1850	46.99\\
1850	48.75\\
1850	38.58\\
1850	34.86\\
1850	30.07\\
1850	28.01\\
1850	28.01\\
1850	21.92\\
1850	21.44\\
1850	22.69\\
1850	24.02\\
1749	31.84\\
1842	38.99\\
1683	43.96\\
1488	45.8\\
1580	48.41\\
1696	53.7\\
1493	43.88\\
1353	44.6\\
1006	45.32\\
1615	40.42\\
1850	39.94\\
1850	44.46\\
1830	39.94\\
1850	39.96\\
1850	66.56\\
1850	63.54\\
1850	65.06\\
1850	55\\
1676	41.06\\
1850	37.77\\
1850	35.97\\
1850	35.74\\
1850	37.06\\
1850	73.11\\
1850	80.78\\
1783	50.12\\
1592	55.91\\
1725	59.81\\
1850	52.44\\
1850	47.12\\
1850	44\\
1774	44.8\\
1568	41.22\\
1850	42.98\\
1850	42.99\\
1850	45.87\\
1850	55\\
1850	50.12\\
1850	54.25\\
1850	42.44\\
1850	42.44\\
1850	70\\
1850	81.93\\
1850	39.75\\
1850	60.01\\
1850	39.15\\
1850	38.88\\
1850	38.44\\
1850	69.9\\
1850	59.94\\
1784	63.98\\
1669	60.89\\
1850	54.91\\
1850	60\\
1850	45\\
1850	44.94\\
1850	72.04\\
1850	71.86\\
1850	69.9\\
1850	75.33\\
1850	75.67\\
1850	50\\
1850	50\\
1850	55\\
1850	69.9\\
1850	39.78\\
1850	50\\
1850	100\\
1850	55\\
1850	40.1\\
1850	59.94\\
1850	37.34\\
1850	42.91\\
1850	65\\
1850	70.93\\
1850	69.71\\
1850	69.54\\
1850	52.44\\
1850	68.76\\
1850	50\\
1850	42.86\\
1850	39.99\\
1850	38\\
1850	39.94\\
1850	39.94\\
1850	44.44\\
1850	42.91\\
1850	42.5\\
1850	45\\
1850	42.02\\
1850	79.9\\
1850	41.06\\
1850	39.05\\
1850	39\\
1850	38.27\\
1850	39.3\\
1850	39.54\\
1850	42.44\\
1850	46.24\\
1850	50.4\\
1850	60.52\\
1850	56.09\\
1850	45.12\\
1850	40\\
1850	38.97\\
1850	37.84\\
1850	37.66\\
1850	39.94\\
1850	42.44\\
1850	42.44\\
1850	44.94\\
1850	53.12\\
1850	50.31\\
1850	50\\
1850	40.88\\
1850	30\\
1850	19.99\\
1850	30\\
1850	34.61\\
1850	33.86\\
1850	34.92\\
1850	34.99\\
1850	39.65\\
1850	40.86\\
1850	39.94\\
1850	39.96\\
1850	39.99\\
1850	38.38\\
1850	35\\
1850	32.96\\
1850	34.94\\
1850	38.37\\
1850	41.55\\
1850	42.44\\
1850	49.96\\
1850	48\\
1850	40\\
1850	31.42\\
1850	20.12\\
1809	14.94\\
1850	14.92\\
1850	15.78\\
1850	19.96\\
1850	22.54\\
1850	37.14\\
1850	43.2\\
1197	44.35\\
1097	48.65\\
1483	48.66\\
1538	46.67\\
1476	42.9\\
1850	42.44\\
1850	45\\
1850	44.96\\
1850	44.96\\
1850	55.74\\
1850	48\\
1850	48\\
1850	50\\
1850	55\\
1850	49.04\\
1850	42.44\\
1850	38.11\\
1850	36.22\\
1850	34.31\\
1850	30\\
1850	29.27\\
1850	27.44\\
1820	38.07\\
1604	44.28\\
1758	50.97\\
1740	58.5\\
1850	51\\
1850	50.12\\
1850	47.44\\
1850	49.94\\
1850	48.66\\
1850	43\\
1850	41.33\\
1850	41.06\\
1850	40\\
1850	40\\
1850	42.32\\
1850	45\\
1850	45\\
1850	42.14\\
1779	36.96\\
1693	34.48\\
1850	33.9\\
1850	33.94\\
1850	35.15\\
1850	37.36\\
1850	40.92\\
1850	55.1\\
1384	58.87\\
1209	63.9\\
1795	47\\
1850	42.44\\
1847	39.94\\
1850	39.47\\
1850	39.94\\
1850	42\\
1850	44\\
1850	49.31\\
1850	45\\
1850	44.94\\
1850	47.44\\
1850	46.73\\
1850	72.29\\
1850	42.39\\
1850	37.73\\
1850	35.49\\
1850	34.99\\
1850	33.04\\
1850	29.25\\
1850	31.76\\
1850	38.36\\
1540	46.11\\
1438	51.52\\
1850	50\\
1850	49.94\\
1850	55.02\\
1850	44.94\\
1850	45\\
1850	42.44\\
1850	42.13\\
1850	42.44\\
1850	42.44\\
1850	39.99\\
1850	47.44\\
1850	47.44\\
1850	54.25\\
1850	45\\
1850	41.45\\
1850	36.83\\
1850	31.73\\
1850	31.23\\
1850	29.95\\
1850	28.68\\
1850	29.95\\
1850	36.83\\
1850	38.03\\
1850	42.48\\
1850	46.72\\
1850	48\\
1850	51.9\\
1850	43.45\\
1850	42.12\\
1850	39.94\\
1850	38.92\\
1850	39.94\\
1850	42.12\\
1850	39.99\\
1850	39.09\\
1850	39.94\\
1850	42.44\\
1850	43.84\\
1850	43.84\\
1803	35.91\\
1703	29.1\\
1900	27.72\\
1900	27.98\\
1900	27.59\\
1900	24.71\\
1900	27.72\\
1900	29.99\\
1900	35.65\\
1900	39.94\\
1900	39.94\\
1900	38.12\\
1900	36\\
1900	34.94\\
1900	32.99\\
1900	35\\
1900	35\\
1900	35.65\\
1900	38.38\\
1900	37.85\\
1900	37.84\\
1900	38.19\\
1558	38.39\\
1089	37.97\\
1900	22.76\\
1796	18.19\\
1900	19.79\\
1900	19.75\\
1900	19.78\\
1900	19.9\\
1900	19.79\\
1737	12.92\\
1900	27.21\\
1900	28.74\\
1900	30.4\\
1900	31.3\\
1900	31.3\\
1900	27.46\\
1900	21.68\\
1900	21.47\\
1900	24.96\\
1900	32.16\\
1900	39.94\\
1900	43\\
1900	50\\
1900	54.94\\
1900	53.22\\
1900	42\\
1900	27.46\\
1900	23\\
1900	19.94\\
1900	19.88\\
1900	20.09\\
1900	28.74\\
1900	29.45\\
1900	20.61\\
1900	29.79\\
1900	32.44\\
1900	34.99\\
1900	35\\
1900	33.86\\
1900	30.65\\
1900	29.79\\
1899	14.99\\
1900	27.99\\
1900	34.94\\
1900	40\\
1900	40\\
1900	42.44\\
1900	47.44\\
1900	49.94\\
1900	40.41\\
1900	36.1\\
1900	35.94\\
1900	31.81\\
1900	31.15\\
1900	29.7\\
1900	31.4\\
1900	40.67\\
1549	50.24\\
1427	54.16\\
1704	55.12\\
1900	51\\
1900	48\\
1900	46.72\\
1900	46.72\\
1900	44.26\\
1900	43.94\\
1900	44.16\\
1900	47.44\\
1900	44.94\\
1900	45.15\\
1900	45.67\\
1900	44.94\\
1900	56.96\\
1845	43.75\\
1900	31.96\\
1730	31.04\\
1900	31.96\\
1900	31.95\\
1900	28.94\\
1900	34.33\\
1895	40.59\\
1836	46.49\\
1458	50.8\\
1900	44\\
1900	44.94\\
1900	47.44\\
1900	44.94\\
1900	44.94\\
1900	44.94\\
1900	43.82\\
1900	44.94\\
1900	47.44\\
1900	45\\
1900	46.32\\
1900	47.44\\
1900	47.44\\
1900	44\\
1844	37\\
1334	37.32\\
1900	32.29\\
1900	31.63\\
1900	31.38\\
1900	31.9\\
1900	32.16\\
1797	39.83\\
1800	45.42\\
1532	46.73\\
1854	42.56\\
1900	45\\
1900	48.28\\
1900	47.44\\
1900	44.94\\
1900	41.33\\
1900	40.36\\
1900	40.87\\
1900	39.94\\
1900	39.97\\
1900	39.95\\
1900	39.45\\
1900	42.44\\
1900	43.38\\
1900	35.98\\
1838	29.5\\
1900	30.92\\
1900	29.66\\
1900	29.99\\
1900	29.99\\
1900	31.95\\
1900	35.82\\
1900	44.4\\
1900	44.94\\
1900	57.24\\
1900	53.99\\
1900	44.94\\
1900	37.44\\
1900	39.94\\
1900	40.01\\
1900	43.67\\
1900	43.64\\
1900	50.11\\
1900	49.33\\
1900	45\\
1900	44.99\\
1900	48.9\\
1900	39.94\\
1900	42.39\\
1080	40.16\\
1403	33.49\\
1746	31.75\\
2002	28.02\\
2129	26.5\\
2239	28.08\\
1670	29.16\\
1718	30.35\\
1962	32.79\\
1918	35\\
2086	34.81\\
1784	35.95\\
1671	34.94\\
2083	29.7\\
1565	31.75\\
2046	27.9\\
2292	28.42\\
2300	34.58\\
2300	40\\
2300	42.44\\
2300	44.94\\
2300	50.48\\
2300	54.94\\
2276	45\\
2300	35.28\\
2300	29.68\\
2300	30.94\\
2300	27.15\\
2300	25.23\\
2300	19.19\\
2300	17.49\\
2300	27.9\\
2300	22.51\\
2300	32.5\\
2300	34.96\\
2300	38.46\\
2300	38\\
2300	40\\
2300	35.23\\
2200	19.79\\
2255	16.22\\
2300	42\\
2300	42.44\\
2300	41\\
2300	38\\
2300	44.96\\
2300	41\\
2300	39.94\\
1900	33.28\\
1900	29.68\\
1900	30.94\\
1900	29.68\\
1900	28.42\\
1900	28.25\\
1900	38.1\\
1146	55.52\\
1059	52.44\\
1077	55.54\\
1123	55.56\\
1366	54.02\\
1470	52.57\\
1610	42.57\\
1707	40.96\\
1843	39.27\\
1900	36.13\\
1833	38.23\\
1634	43.38\\
1214	53.12\\
1536	44.95\\
1265	50.07\\
1900	47.44\\
1305	44.94\\
1579	35.84\\
1900	36.03\\
1900	34.99\\
1900	33.53\\
1900	33.53\\
1900	33.59\\
1631	38.97\\
1370	47.92\\
1526	50.38\\
1613	50.79\\
1602	47.44\\
1852	49.32\\
1665	46.74\\
1900	50.58\\
1900	48\\
1900	44.34\\
1900	43.37\\
1900	48\\
1900	45\\
1900	45.96\\
1900	46.16\\
1900	44.96\\
1900	49.96\\
1900	43.59\\
1900	31.85\\
1900	30.01\\
1900	31.23\\
1900	31.23\\
1900	34.51\\
1900	33.27\\
1900	35.03\\
1812	44.93\\
1622	52.26\\
1869	49.11\\
1900	50\\
1900	49.94\\
1900	43.59\\
1900	48.28\\
1900	48\\
1900	44\\
1900	40.56\\
1900	44.94\\
1900	40.12\\
1900	37.44\\
1900	37.51\\
1900	42.44\\
1900	45\\
1900	39.94\\
2300	34.99\\
2300	38.29\\
2300	33.38\\
2300	31.84\\
2300	33.38\\
2300	30.76\\
2300	29.27\\
2300	31.05\\
2300	34.78\\
2300	40\\
2300	42.44\\
2300	45\\
2300	48\\
2300	43\\
2300	25\\
2300	35\\
2300	42.44\\
2300	42\\
2300	42.44\\
2300	42.44\\
2300	40\\
2300	47.44\\
2300	50\\
2300	37.99\\
2300	47.44\\
2300	36.49\\
2300	32.07\\
2300	30.78\\
2300	29.15\\
2300	31.5\\
2300	33.78\\
2300	38\\
2300	42\\
2300	51.27\\
2300	56.13\\
2300	50.88\\
2300	46\\
2300	35\\
2300	33.41\\
2249	26.06\\
2296	25.38\\
2300	35\\
2300	33.38\\
2300	33.26\\
2300	33.78\\
2300	35\\
2300	36\\
2300	34.99\\
2300	34.96\\
2300	26.73\\
2300	23.06\\
2300	30.43\\
2300	31.72\\
2300	30.43\\
2300	28.29\\
2300	31.77\\
2300	38\\
2300	42\\
2300	41.8\\
2300	39.96\\
2300	35.61\\
2240	25.47\\
2076	23.28\\
2163	22.2\\
2300	32.44\\
2300	37.44\\
2300	44.96\\
2300	42.44\\
2300	40\\
2300	49.94\\
2300	53.76\\
2300	49.94\\
2300	50.76\\
2300	41.19\\
2300	32.44\\
2300	30.7\\
2300	35.92\\
2300	37.74\\
2300	33.16\\
2300	25.68\\
2300	30.7\\
2300	33.07\\
2300	34\\
2300	32.44\\
2300	31.44\\
2166	11.22\\
2028	5.34\\
2069	4.73\\
2277	7.3\\
2300	32.75\\
2300	40\\
2300	40.12\\
2300	42.44\\
2300	49.94\\
2300	54.96\\
2300	45.99\\
1900	30.83\\
1900	30.28\\
1900	34.33\\
1900	34.96\\
1900	36.45\\
1900	37.5\\
1900	40.09\\
1900	44.08\\
1900	48.09\\
1900	43.94\\
1900	44.94\\
1900	47\\
1900	39.99\\
1900	47.44\\
1900	42.44\\
1900	44\\
1900	43.96\\
1900	47.63\\
1900	52.44\\
1900	48\\
1900	46.99\\
1900	53.94\\
1900	49.94\\
1900	35.02\\
1900	36.98\\
1900	34.44\\
1900	31.05\\
1900	24.89\\
1900	30.65\\
1900	27.45\\
1900	34.37\\
1900	37.98\\
1754	38.94\\
1900	44.94\\
1900	57.71\\
1900	57.71\\
1900	44.94\\
1900	42.44\\
1900	40\\
1900	41.44\\
1900	41.27\\
1900	47\\
1900	37.04\\
1900	38.94\\
1900	39.94\\
1900	40.12\\
1900	45\\
1900	41.45\\
1900	32.58\\
1900	30.22\\
1900	29.88\\
1900	28.89\\
1900	28.89\\
1900	30.74\\
1900	38.02\\
1900	44.82\\
1900	47.07\\
1900	59.49\\
1900	57.46\\
1900	57.46\\
1900	54.39\\
1900	47\\
1900	44.94\\
1900	45\\
1900	42.44\\
1900	47.25\\
1900	42.41\\
1900	38.44\\
1900	38.51\\
1900	45\\
1900	48.87\\
1900	32\\
1900	29.67\\
1900	26.87\\
1900	26.13\\
1900	26.07\\
1900	25.96\\
1900	27.03\\
1900	35.78\\
1900	38.09\\
1900	38\\
1900	44.94\\
1900	47.44\\
1900	48.87\\
1900	47.44\\
1900	48.87\\
1900	46.56\\
1900	44.93\\
1900	45.26\\
1900	53.26\\
1900	47\\
1900	40\\
1900	47.6\\
1900	43.18\\
1900	36.69\\
1900	34.43\\
2000	36.43\\
2000	30.11\\
2000	22.82\\
2000	27.23\\
2000	23.15\\
2000	29.35\\
2000	34.86\\
2000	39.94\\
2000	39.35\\
2000	47\\
2000	48.88\\
2000	45\\
2000	42.44\\
2000	37.44\\
2000	34.94\\
2000	34.29\\
2000	34.3\\
2000	34.31\\
2000	33.28\\
2000	34.74\\
2000	35.27\\
2000	35\\
2000	47.44\\
2000	40.83\\
2300	40\\
2300	30.52\\
2300	30.52\\
2300	30.06\\
2300	25.29\\
2300	27.99\\
2300	23.53\\
2300	28.35\\
2300	34.94\\
2300	45\\
2300	55.62\\
2300	50\\
2300	45\\
2300	40\\
2300	33.7\\
2300	32.25\\
2300	32.25\\
2300	35\\
2300	33.7\\
2300	33.99\\
2300	32.31\\
2300	31.63\\
2300	32.06\\
2300	23.55\\
2300	22.77\\
2300	21.56\\
2300	21.09\\
2069	6.38\\
2081	4.97\\
1925	2.8\\
2300	-0.01\\
2300	-0.01\\
1993	2.28\\
1957	8.06\\
2108	10.09\\
2242	10.46\\
2154	10.82\\
2052	9.8\\
2300	20.01\\
2249	3.51\\
2300	20.01\\
2127	2.47\\
2300	35\\
2300	33.71\\
2300	33.73\\
2300	33.63\\
2300	32.76\\
2300	24.48\\
2000	23.15\\
2000	23.01\\
2000	22.7\\
2000	22.7\\
2000	23.05\\
2000	31.8\\
2000	37.69\\
1897	39.94\\
2000	42.44\\
2000	48\\
2000	57.71\\
2000	58.48\\
2000	44.99\\
2000	52\\
2000	53.71\\
2000	48.4\\
2000	48.87\\
2000	64.96\\
2000	42.44\\
1770	42.44\\
2000	42.92\\
2000	48.87\\
2000	57.44\\
2000	49.01\\
2000	50\\
2000	35.16\\
2000	32.75\\
2000	31.6\\
2000	31.91\\
2000	29.26\\
2000	37.17\\
2000	50.3\\
2000	58.51\\
2000	63.46\\
2000	51.85\\
2000	52.02\\
2000	48.87\\
2000	42.65\\
2000	39.78\\
2000	44.17\\
2000	38\\
1989	36.95\\
2000	39.96\\
2000	41.79\\
2000	42.27\\
2000	41.03\\
2000	42.44\\
1899	34.07\\
2000	31.83\\
2000	27.44\\
2000	31.12\\
2000	32.56\\
2000	31.57\\
2000	58.74\\
2000	33.71\\
2000	59.35\\
2000	49.11\\
2000	44.94\\
2000	42.44\\
1980	40.33\\
1885	39.94\\
2000	39.94\\
2000	42.44\\
2000	44\\
2000	46.67\\
2000	49.11\\
2000	43\\
2000	39.99\\
2000	39\\
2000	40.66\\
2000	72.25\\
2000	47\\
1710	31.77\\
2000	30.89\\
2000	31.89\\
2000	30.95\\
2000	30\\
1734	30.55\\
1999	38.64\\
1973	47.42\\
1787	55.64\\
2000	51.69\\
1899	47.39\\
2000	46.74\\
1831	40.37\\
1764	39.99\\
1568	37.98\\
1854	35.2\\
2000	35.18\\
2000	37.44\\
2000	40\\
2000	40.57\\
2000	40\\
2000	43.2\\
2000	70.58\\
2000	40\\
2000	32.34\\
2000	33.91\\
2000	33.72\\
2000	32\\
2000	30.65\\
2000	29.1\\
2000	34.09\\
2000	55.13\\
2000	44.96\\
2000	47\\
2000	45\\
2000	41\\
2000	41\\
2000	40\\
2000	41\\
2000	38\\
2000	39.94\\
2000	42.44\\
2000	44.16\\
2000	40\\
2000	39.47\\
2000	39.94\\
2000	39.94\\
2000	40\\
2300	31.5\\
2300	27.88\\
2300	32.98\\
2300	31.68\\
2300	33.67\\
2300	31.68\\
2300	26.31\\
2300	30.39\\
2300	34.96\\
2300	42.44\\
2300	37.44\\
2300	34.83\\
2300	34.27\\
2300	31.68\\
2300	30.39\\
2300	29.99\\
2300	31.98\\
2300	36.4\\
2300	44.94\\
2300	46.83\\
2300	47.12\\
2300	50\\
2300	49.94\\
2300	48.06\\
2300	52.35\\
2300	32\\
2300	35.13\\
2300	33.33\\
2300	33.19\\
2300	30.75\\
2300	17.97\\
2300	20.01\\
2300	21.22\\
2300	29\\
2300	30.75\\
2300	30.75\\
2300	29.96\\
2259	22\\
2300	31.03\\
2300	33.2\\
2300	33.2\\
2300	31.77\\
2300	37.44\\
2300	37.44\\
2300	39.94\\
2300	45\\
2300	49\\
2300	39.94\\
2000	30.9\\
2000	31.47\\
2000	30.57\\
2000	30.1\\
2000	28.12\\
2000	29.43\\
2000	38.44\\
2000	49.95\\
2000	53.11\\
2000	50.39\\
2000	46.96\\
2000	47.2\\
1505	44\\
1624	42.43\\
1753	39.45\\
2000	37.44\\
2000	37.92\\
2000	41.18\\
2000	46.5\\
2000	53.46\\
2000	49.47\\
2000	47.44\\
2000	39.98\\
1800	33.29\\
2000	31.59\\
2000	30.59\\
2000	29.02\\
2000	29.42\\
2000	29.62\\
2000	31.5\\
2000	42.77\\
2000	53.94\\
2000	52.39\\
2000	46.82\\
2000	42.7\\
2000	41.22\\
1858	37.99\\
1825	38.06\\
1931	36.08\\
2000	36.97\\
2000	41\\
2000	57.21\\
2000	48\\
2000	48\\
2000	53.11\\
2000	46.04\\
2000	44.94\\
2000	48.01\\
2000	45.85\\
2000	35.72\\
2000	33.52\\
2000	32.84\\
2000	32.7\\
2000	31.95\\
2000	41.19\\
2000	48.83\\
2000	52.91\\
2000	52.99\\
2000	44\\
2000	68.83\\
2000	39.94\\
2000	44.94\\
2000	42.3\\
2000	39.94\\
2000	44.94\\
2000	53.11\\
2000	43.12\\
2000	47.27\\
2000	43.46\\
2000	37.97\\
2000	44.94\\
2000	44.18\\
1744	28.25\\
1502	24\\
1591	20.41\\
1944	15.13\\
2000	18.41\\
2000	22.91\\
1929	29.25\\
2000	36.55\\
1874	39.66\\
1745	38.58\\
1884	40.5\\
2000	42.29\\
1715	41.26\\
2000	39.94\\
2000	42.29\\
2000	42.29\\
2000	48\\
2000	48\\
2000	54.24\\
2000	44.7\\
2000	43.82\\
2000	46.63\\
2000	55\\
1599	47.01\\
1739	30.05\\
2000	34.8\\
2000	31.81\\
2000	31.03\\
2000	31.81\\
2000	30.75\\
2000	34.04\\
2000	49.95\\
2000	51.06\\
2000	52.97\\
2000	52.41\\
2000	57.04\\
2000	47.98\\
2000	45.23\\
2000	40.53\\
2000	37.38\\
2000	35\\
2000	37.54\\
2000	42\\
2000	43.98\\
2000	41.92\\
2000	40.49\\
2000	42.58\\
2000	49.78\\
2300	47.24\\
2300	34.57\\
2300	34.29\\
2300	33.19\\
2300	32.08\\
2300	30.58\\
2300	29.99\\
2300	32.44\\
2300	39.94\\
2300	40.77\\
2300	48\\
2300	47.44\\
2300	44.94\\
2300	35\\
2300	34.09\\
2300	32.44\\
2300	32\\
2300	37.44\\
2300	40\\
2300	42.44\\
2300	45\\
2300	40\\
2300	49.96\\
2300	44.96\\
2300	32.94\\
2300	24.96\\
2002	20.34\\
2130	16.18\\
2300	15.45\\
2300	15.79\\
2300	18.94\\
2300	27\\
2300	30.85\\
2300	37.99\\
2300	37.44\\
2300	38.99\\
2300	38.3\\
2300	32.94\\
2220	13.43\\
2225	13.88\\
2300	30\\
2300	42.44\\
2300	44.94\\
2300	50\\
2300	49.99\\
2300	50\\
2300	55\\
2300	44.94\\
1900	50\\
1900	32.75\\
1900	32.77\\
1900	29.99\\
1900	32.75\\
1900	29.99\\
1900	37.12\\
1900	46.58\\
1900	42.44\\
1797	42.2\\
1900	42.4\\
1900	60\\
1900	52.08\\
1900	70.42\\
1900	70.83\\
1900	69.99\\
1900	49.04\\
1900	60\\
1900	55\\
1754	45.21\\
1869	42.29\\
1900	44.94\\
1900	69.99\\
1900	41.75\\
1614	30.03\\
1900	31.82\\
1900	33.35\\
1900	33.85\\
1900	34.01\\
1900	50\\
1900	33.63\\
1900	43.97\\
1900	47.44\\
1900	50.01\\
1900	49.93\\
1900	50.01\\
1715	44.71\\
1900	44\\
1900	45.42\\
1900	41.89\\
1900	41.2\\
1900	45\\
1900	50.12\\
1900	40.94\\
1900	42.44\\
1900	41.67\\
1900	39.99\\
1900	37.58\\
1900	200\\
1900	34.62\\
1900	30.85\\
1900	29.63\\
1900	31.63\\
1900	31.72\\
1900	33.85\\
1900	44.94\\
1900	44.1\\
1900	45.17\\
1900	54\\
1900	50\\
1900	60\\
1900	50\\
1900	45.06\\
1900	43.01\\
1900	46.4\\
1900	54.31\\
1900	44.94\\
1900	41.2\\
1900	38.13\\
1900	41.96\\
1900	54\\
1900	44.94\\
1900	39.94\\
1900	33.94\\
1900	34.09\\
1900	33.58\\
1900	32\\
1900	29.42\\
1900	30.43\\
1900	29.96\\
1900	28.61\\
1900	32.44\\
1900	34.37\\
1900	37.44\\
1900	39.94\\
1900	34.94\\
1900	33.89\\
1900	33\\
1900	33.25\\
1900	34.33\\
1900	44.94\\
1900	45\\
1900	49.83\\
1900	44.53\\
1900	47.94\\
1900	45\\
1256	33.05\\
1900	30.06\\
1900	31.35\\
1900	31.35\\
1900	30.66\\
1900	31.35\\
1900	32.37\\
1900	38.73\\
1900	42.42\\
1900	44.94\\
1900	42.44\\
1900	42.44\\
1335	40.95\\
1457	35.63\\
1561	37.17\\
1751	34.69\\
1794	33.95\\
1769	37.11\\
1900	47.44\\
1900	45.24\\
1900	45.84\\
1900	42.76\\
1796	42.32\\
1900	39.94\\
1900	36.47\\
1900	44.25\\
1900	35\\
1900	32.87\\
1900	32.51\\
1900	32.87\\
1900	32.51\\
1900	35\\
1900	35\\
1900	43.15\\
1900	37.12\\
1900	36\\
1900	34.94\\
1900	38.75\\
1900	34.25\\
1900	34.99\\
1900	34.94\\
1900	38.75\\
1900	41\\
1900	42.44\\
1900	39.94\\
1900	39.94\\
1900	44.94\\
1900	44.94\\
2000	38.23\\
2000	34.99\\
2000	38.79\\
2000	34.29\\
2000	32.29\\
2000	31.35\\
2000	29.3\\
2000	29.96\\
2000	31.35\\
2000	30.9\\
2000	33.56\\
2000	36\\
2000	38\\
2000	33.67\\
2000	31.68\\
2000	35\\
2000	30.37\\
2000	31.68\\
2000	36\\
2000	39\\
2000	44\\
2000	44\\
2000	45\\
2000	46.9\\
1900	49.23\\
1900	41.08\\
1900	37.04\\
1900	34.04\\
1900	34.04\\
1900	30.58\\
1900	38.33\\
1900	42.04\\
1900	43.01\\
1900	43.5\\
1900	43.72\\
1900	44.99\\
1639	41.01\\
1900	50\\
1900	47.7\\
1900	43.58\\
1900	42.44\\
1900	54.85\\
1900	40.76\\
1900	42.89\\
1900	42.89\\
1900	41.91\\
1900	48.93\\
1900	45.32\\
1696	32.46\\
1900	32.55\\
1900	31.47\\
1900	30.13\\
1900	30.79\\
1900	30.11\\
1900	40.94\\
1900	48.29\\
1900	53.15\\
1900	50.94\\
1900	47.86\\
1900	51.25\\
1900	47.69\\
1813	44.99\\
1544	45.84\\
1728	44.37\\
1900	44.88\\
1900	47.02\\
1900	48.47\\
1900	48.97\\
1900	47.61\\
1771	44.48\\
1613	45.61\\
1639	37.5\\
1900	32.49\\
1900	31.54\\
1900	33.2\\
1900	33.2\\
1900	33.25\\
1900	31.54\\
1900	37.66\\
1850	47.44\\
1634	50.44\\
1850	50.01\\
1850	53.15\\
1850	55.55\\
1850	54.94\\
1850	55.12\\
1850	48.37\\
1850	45.17\\
1850	47\\
1900	54.94\\
1900	48\\
1900	41.91\\
1900	39.67\\
1900	37.24\\
1900	45.51\\
1900	39.94\\
1964	28.95\\
2000	28\\
2000	27.58\\
2000	27.58\\
2000	24.07\\
2000	25.03\\
2000	30.97\\
2000	37.67\\
2000	42\\
2000	46.44\\
2000	49\\
2000	47.96\\
2000	42\\
2000	44.94\\
2000	40.12\\
2000	37.44\\
2000	37.44\\
2000	43.56\\
2000	39.94\\
2000	40\\
2000	37.2\\
2000	36.83\\
2000	45.41\\
2000	41.99\\
2000	31.38\\
2000	31.4\\
2000	31.41\\
2000	31.37\\
2000	31.39\\
2000	32.86\\
2000	40.01\\
2000	41.71\\
2000	44.94\\
2000	47.18\\
2000	45\\
2000	44.94\\
2000	40\\
2000	39.94\\
2000	39.94\\
2000	38\\
2000	38.49\\
2000	42.44\\
2000	44\\
2000	39.44\\
2000	37.68\\
2000	36.01\\
2000	43.7\\
2000	32.3\\
2300	28.34\\
2300	26.07\\
2300	25.06\\
2300	24.73\\
2300	24.99\\
2300	24.09\\
2271	10.35\\
2287	15.98\\
2300	29.19\\
2300	31.86\\
2300	34.94\\
2300	32.14\\
2300	31.21\\
2296	20.01\\
2284	17.38\\
2300	29.58\\
2300	30.43\\
2300	33.86\\
2300	38.43\\
2300	40\\
2300	38\\
2300	33.87\\
2300	39.94\\
2231	30.57\\
2300	29.51\\
2300	27.3\\
2200	20.12\\
2300	25.04\\
2300	24.99\\
2300	16.32\\
2300	14.91\\
2141	15\\
1992	15.98\\
2216	18.13\\
2300	27.41\\
2300	27.46\\
2178	18.45\\
1845	17.1\\
1790	14.06\\
1695	13.7\\
1670	16.55\\
2008	20.08\\
2300	27.51\\
2300	37\\
2300	37\\
2300	34.93\\
2300	37.11\\
2300	37\\
2200	31.37\\
2200	28.36\\
2200	25.04\\
2200	24.99\\
2200	24.99\\
2200	22.28\\
2200	24.33\\
2200	27.99\\
2128	26.09\\
2154	26.93\\
2014	26.46\\
2026	28.89\\
2082	29.87\\
1945	26.71\\
2005	25.25\\
2041	21.61\\
2200	31.09\\
2200	36\\
2200	37\\
2200	39.94\\
2200	45\\
2200	39.94\\
2200	45.37\\
2200	35.99\\
2000	30.47\\
2000	31.37\\
2000	30.01\\
2000	29.25\\
2000	28.66\\
2000	21.77\\
2000	30.01\\
2000	40.12\\
2000	41.27\\
2000	45.39\\
2000	48\\
2000	49.88\\
2000	45\\
2000	45.1\\
2000	44.94\\
2000	44.57\\
2000	44.49\\
2000	49.69\\
2000	43.84\\
2000	43.03\\
2000	39.91\\
2000	36.92\\
1896	41.84\\
2000	43.16\\
2000	31.07\\
2000	30.71\\
2000	30.68\\
2000	30.52\\
2000	30.71\\
2000	30.52\\
2000	34.92\\
2000	44.99\\
2000	49.54\\
2000	49.51\\
2000	49.88\\
1801	49.94\\
1650	49.4\\
1547	44.98\\
1467	43.77\\
1591	41.81\\
1818	39.57\\
2000	40.96\\
2000	47.96\\
2000	44.94\\
2000	39.6\\
2000	36.24\\
2000	43\\
2000	44.01\\
2000	30.95\\
2000	30.57\\
2000	29.39\\
2000	28.74\\
2000	28.74\\
2000	29.66\\
2000	33.09\\
2000	44.07\\
2000	46.37\\
2000	46.58\\
2000	46.12\\
1939	46.68\\
1537	42.94\\
1866	41.05\\
2000	39.99\\
1938	38.35\\
1939	37.07\\
2000	39.4\\
1850	42\\
2000	44.77\\
2000	43.05\\
2000	40.51\\
2000	39.39\\
2000	39\\
2000	32.37\\
2000	32.39\\
2000	31.39\\
2000	30.49\\
2000	31.11\\
2000	31.39\\
2000	36\\
2000	44.82\\
2000	45.55\\
2000	53.88\\
2000	51.42\\
2000	55\\
2000	45.6\\
2000	43\\
2000	47.75\\
2000	39.94\\
2000	38\\
2000	44\\
2000	40\\
2000	35.12\\
2000	47\\
2000	32.36\\
2000	49.52\\
2000	45\\
2000	44.99\\
2000	32\\
2000	30.34\\
2000	29.73\\
2000	27.34\\
2000	20\\
2000	22.33\\
2000	29.34\\
2000	32.03\\
2000	48.57\\
2000	50\\
2000	49.94\\
2000	42.44\\
2000	37.44\\
2000	33.32\\
2000	31.91\\
2000	31.87\\
2000	33.32\\
2000	36\\
2000	39.94\\
2000	36\\
2000	34.35\\
2000	45.1\\
2000	48.33\\
2000	31.61\\
2000	31.3\\
2000	28.82\\
2000	28.62\\
2000	27.46\\
2000	23.12\\
2000	16.43\\
2000	22.89\\
2000	27.44\\
2000	31.24\\
2000	32.44\\
2000	32.91\\
2000	33.11\\
2000	31.6\\
2000	30\\
2000	22.89\\
2000	22.89\\
2000	30.65\\
2000	31.54\\
2000	33.11\\
2000	35.25\\
2000	31.59\\
2000	34.96\\
2000	31.63\\
2000	30.04\\
2000	27.8\\
2000	25.84\\
2000	21.49\\
2000	27.46\\
2000	24.89\\
2000	33.93\\
2000	41.17\\
2000	41.93\\
2000	40\\
2000	42.44\\
2000	47.44\\
2000	45\\
2000	44.94\\
2000	37.49\\
2000	34.99\\
2000	33.09\\
};
\addplot [color=mycolor1,line width=1.0pt,mark size=0.3pt,only marks,mark=*,mark options={solid},forget plot]
  table[row sep=crcr]{%
2000	33.61\\
2000	35.11\\
2000	36.82\\
2000	35.99\\
2000	34.53\\
1962	36.05\\
2000	30.9\\
2000	30.84\\
2000	30.87\\
2000	30.87\\
2000	31.36\\
2000	30.87\\
2000	28.57\\
2000	36.96\\
2000	44.71\\
2000	51.99\\
2000	52.44\\
2000	50\\
2000	49.99\\
1884	42.98\\
1809	41.83\\
2000	41.45\\
2000	39.94\\
2000	37.44\\
2000	38.34\\
2000	39.84\\
2000	41.21\\
2000	41.22\\
2000	36.31\\
2000	40.48\\
1729	35.65\\
2000	29.87\\
2000	31.03\\
2000	31.32\\
2000	31.2\\
2000	31.05\\
2000	28.22\\
2000	34.97\\
2000	48.51\\
2000	43.23\\
2000	48.51\\
2000	48.51\\
2000	48.51\\
2000	62.13\\
2000	39\\
2000	40\\
2000	37.44\\
2000	37.44\\
2000	60.01\\
2000	65.5\\
2000	57.01\\
2000	60\\
2000	44.99\\
2000	69.47\\
2000	47.44\\
2000	31.33\\
2000	30.52\\
2000	31.22\\
2000	30.68\\
2000	30.86\\
2000	29.14\\
2000	30.52\\
2000	38\\
2000	39.94\\
2000	43.73\\
2000	49.47\\
2000	55\\
2000	40\\
2000	42\\
2000	44.57\\
2000	39.94\\
2000	37.44\\
2000	39.94\\
2000	50.01\\
2000	36.07\\
2000	45.95\\
2000	36.09\\
2000	60\\
2000	35.5\\
2000	30.84\\
2000	31.61\\
2000	30.99\\
2000	30.48\\
2000	30.25\\
2000	27.02\\
2000	30.43\\
1977	35\\
2000	35.96\\
2000	40\\
2000	42.44\\
2000	49.88\\
2000	42.44\\
2000	44.94\\
2000	42.44\\
2000	35\\
2000	34.11\\
2000	35\\
2000	36.01\\
1922	35.42\\
2000	35.52\\
2000	33.51\\
2000	43\\
2000	44.83\\
2000	31.49\\
2000	30.97\\
2000	31.49\\
2000	30.68\\
2000	26.71\\
2000	18.63\\
2000	27.4\\
2000	26.71\\
2000	31.01\\
2000	31.44\\
2000	34.09\\
2000	35\\
2000	34.31\\
2000	34.94\\
2000	33\\
2000	32.21\\
2000	31.41\\
2000	34.94\\
2000	36\\
2000	39.94\\
2000	40\\
2000	32.44\\
2000	42\\
2000	36\\
2000	31.49\\
2000	30.13\\
2000	31.01\\
2000	17.58\\
2000	15.91\\
2000	14.59\\
2000	13.55\\
2000	14.24\\
2000	13.56\\
2000	25.04\\
2000	29\\
2000	30\\
2000	30.55\\
2000	30\\
2000	29.75\\
2000	29.75\\
2000	29.96\\
2000	32.85\\
2000	34.94\\
2000	44.94\\
2000	50\\
2000	40\\
2000	50.5\\
2000	45.01\\
2000	31.24\\
2000	30.33\\
2000	30.98\\
2000	27.4\\
2000	20.18\\
2000	23.88\\
2000	31.21\\
2000	39.01\\
2000	35.96\\
1752	34.26\\
1408	33.99\\
1382	35.9\\
1018	37\\
968	38.02\\
937	39.47\\
1019	39.23\\
1167	38.36\\
1355	41.94\\
1507	47.91\\
1387	47.84\\
1487	43.98\\
1545	38.48\\
1424	41.92\\
1390	36.31\\
1591	30.36\\
2000	32.08\\
2000	28.54\\
2000	28.01\\
2000	27.87\\
2000	29.03\\
1429	36.12\\
1164	44.9\\
847	48\\
577	49.68\\
645	50.61\\
871	54.11\\
481	47.85\\
494	45.23\\
654	41.98\\
732	39.94\\
954	38.73\\
1199	40.79\\
755	41.99\\
816	42.93\\
1277	41.46\\
1761	38.86\\
1834	39.82\\
1695	36.02\\
1675	30.98\\
2000	28.71\\
2000	28.23\\
2000	26.9\\
2000	26.81\\
2000	28.87\\
1846	35.06\\
1745	44.4\\
941	45\\
1012	44.81\\
1051	43.67\\
1074	45.98\\
870	41.2\\
1246	40.85\\
993	39.47\\
955	38.91\\
855	38.41\\
1166	40.45\\
1117	41.68\\
1237	43.22\\
1505	41.23\\
1517	38.86\\
1408	40.63\\
1296	34.96\\
1553	31.98\\
2000	29.69\\
2000	28.85\\
2000	27.76\\
2000	27.71\\
2000	29.11\\
1455	36.71\\
1127	44.69\\
554	47.55\\
609	44.06\\
706	43.07\\
423	45.33\\
527	42.85\\
635	41.2\\
481	44.98\\
478	43.62\\
561	41.9\\
854	44.83\\
703	48.59\\
216	49.07\\
687	45.59\\
852	41.96\\
1141	42.49\\
1405	36.74\\
1486	32.48\\
2000	29.67\\
2000	28.99\\
2000	28.14\\
2000	28.27\\
2000	29.08\\
2000	35.39\\
1918	41.97\\
1355	45.41\\
1660	44.55\\
1825	43.79\\
1424	43.04\\
1635	40.96\\
1945	38.98\\
1738	36.07\\
1868	34.24\\
2000	34.94\\
2000	37.93\\
1911	40.47\\
1933	41.97\\
2000	38.9\\
2000	36\\
1958	41.96\\
1991	38.54\\
2100	46.71\\
2100	30.33\\
2100	28.98\\
2100	28.02\\
2100	27.15\\
2100	27.07\\
2100	27.99\\
2100	29.99\\
2100	31.78\\
2100	45.12\\
2100	56.38\\
2100	57.97\\
2100	54.25\\
2100	42.44\\
2100	38.79\\
2100	35.33\\
2100	36.39\\
2100	40\\
2100	50\\
2100	40\\
2100	36\\
2100	39.96\\
2100	48.46\\
2100	46.09\\
2100	31.22\\
2100	29.91\\
2100	24.58\\
2100	22.95\\
2100	28.47\\
2100	22.51\\
2100	22.47\\
2100	22.78\\
2100	29\\
2100	30.35\\
2100	35\\
2100	39.04\\
2100	39.94\\
2100	33.97\\
2100	30\\
2100	29.73\\
2100	29.94\\
2100	36\\
2100	35.2\\
2100	35.2\\
2100	33.96\\
2100	37\\
2100	33.98\\
2013	31.37\\
1877	28.08\\
2000	29\\
2000	25\\
2000	24.01\\
2000	24.4\\
2000	26.78\\
2000	34.08\\
2000	41.26\\
1813	41.93\\
1985	42.29\\
2000	43\\
2000	47\\
1948	42.55\\
1982	41.25\\
1825	39.94\\
2000	36\\
2000	33.87\\
2000	35\\
1664	37\\
1548	40.37\\
2000	40.27\\
2000	36.23\\
2000	37\\
2000	37\\
2000	31.75\\
2000	30.54\\
2000	30.1\\
2000	29.32\\
2000	25.97\\
2000	26.9\\
2000	33.59\\
2000	41.06\\
2000	40.99\\
2000	42.44\\
2000	40\\
2000	39.94\\
1899	35.69\\
2000	36\\
2000	36\\
2000	35.17\\
2000	34.94\\
2000	35.18\\
2000	39\\
2000	41.92\\
2000	40.09\\
2000	38.87\\
2000	39.38\\
2000	36\\
2000	30.26\\
2000	29.29\\
2000	29.37\\
2000	25.15\\
2000	25.81\\
2000	27.97\\
2000	31.63\\
2000	39.91\\
2000	39.99\\
2000	39.61\\
1886	39.13\\
1825	40.43\\
1451	38.41\\
1455	36.96\\
1363	36\\
1475	34.95\\
1762	33.82\\
1753	36.08\\
1705	38.59\\
1687	39.91\\
2000	39.02\\
2000	36.9\\
1755	38.88\\
1596	32.59\\
2000	30.03\\
2000	28.3\\
2000	29.39\\
2000	29.59\\
2000	29.52\\
2000	26.87\\
2000	31.4\\
2000	36.92\\
2000	38.73\\
2000	36.95\\
1929	35.46\\
1811	36.3\\
1659	33.97\\
1765	32.45\\
1795	33.79\\
2000	31.83\\
2000	32.44\\
2000	36\\
2000	38.79\\
2000	39.96\\
2000	39.3\\
2000	38.94\\
2000	39.43\\
2000	33.12\\
2000	29.63\\
2000	26.87\\
2000	25.69\\
2000	25.06\\
2000	24.99\\
2000	26.04\\
2000	30.28\\
2000	37\\
2000	38.5\\
2000	37.44\\
2000	39.1\\
2000	41\\
2000	37.44\\
2000	37\\
2000	38\\
2000	37.44\\
2000	37.61\\
2000	44.98\\
2000	36.94\\
2000	37.96\\
2000	36.93\\
2000	32.9\\
2000	33.22\\
2000	30.15\\
2300	30.34\\
2300	34.8\\
2300	30.23\\
2300	29.3\\
2300	28.75\\
2300	24.99\\
2300	24.43\\
2300	27.6\\
2300	30\\
2300	34\\
2300	30\\
2300	30.18\\
2300	30\\
2300	30.26\\
2259	20.03\\
2234	20.31\\
2293	19.35\\
2300	28.32\\
2300	28.33\\
2300	28.32\\
2300	33.2\\
2300	31.6\\
2300	31.85\\
2300	28.86\\
2300	47\\
2300	18\\
2300	23.55\\
2300	22.6\\
2300	22.02\\
2000	4.79\\
2080	0.29\\
2300	22.79\\
2217	14.21\\
2300	26.1\\
2300	28.23\\
2300	28.89\\
2300	30.32\\
2300	32.44\\
2300	30\\
2300	28.89\\
2300	29.94\\
2300	30.32\\
2300	43\\
2300	47.27\\
2300	46.22\\
2300	40.96\\
2300	42.95\\
2300	31.6\\
2000	28.23\\
2000	28.25\\
2000	25.35\\
2000	25.35\\
2000	24.14\\
2000	26.02\\
2000	33.09\\
2000	36.93\\
2000	37.16\\
2000	36.94\\
2000	37.26\\
2000	38.13\\
1813	37.98\\
1919	36.97\\
2000	36.19\\
2000	35.29\\
2000	36.15\\
2000	37.97\\
2000	39.23\\
2000	40\\
2000	38.91\\
2000	36.17\\
2000	36\\
2000	29.64\\
2000	28.07\\
2000	27.55\\
2000	25.97\\
2000	25.07\\
2000	25.08\\
2000	25.93\\
2000	30.87\\
2000	40.26\\
2000	42.98\\
2000	43.07\\
2000	45.09\\
2000	44.99\\
2000	42.03\\
2000	40.49\\
2000	38.55\\
2000	36.99\\
2000	36.18\\
2000	35\\
2000	33.98\\
2000	33.03\\
2000	30.83\\
2000	29.64\\
2000	29.15\\
2000	27.8\\
2100	25.22\\
2100	23.99\\
2100	22.04\\
2100	20.92\\
2100	21.7\\
2100	24.2\\
2100	27.61\\
2100	34.95\\
2100	39.08\\
2100	39.75\\
2100	40\\
2100	41.91\\
2100	38.64\\
2100	39.74\\
2100	36.94\\
2100	35.1\\
2100	32\\
2100	34.9\\
2100	35.61\\
2100	35.19\\
2100	33.41\\
2100	32.14\\
2100	30.74\\
2100	29.11\\
2100	25.07\\
2100	21.85\\
2100	21.04\\
2100	21.37\\
2100	22.95\\
2100	25.56\\
2100	29.4\\
2100	36.46\\
2100	40.18\\
2100	39.91\\
2100	39.68\\
2100	39\\
2100	34.58\\
2100	32.86\\
2100	31.88\\
2100	31.85\\
2100	32.77\\
2100	34.94\\
2100	35.85\\
2100	36\\
2100	33.8\\
2100	32.11\\
2100	39.99\\
2100	29.48\\
2300	25.95\\
2300	25\\
2300	23.84\\
2300	23.24\\
2300	23.59\\
2300	25.86\\
2300	30.09\\
2300	36.74\\
2300	37.47\\
2300	36.48\\
2283	35.94\\
2300	36.27\\
2300	39.1\\
2300	35\\
2300	38\\
2300	35\\
2300	35\\
2300	35\\
2300	33.53\\
2300	35\\
2300	35.97\\
2300	35.93\\
2300	35.98\\
2300	29.88\\
1900	30.13\\
1900	26.5\\
1900	25.1\\
1900	25.07\\
1900	24.88\\
1900	24.58\\
1900	25.1\\
1900	29.79\\
1900	32.93\\
1900	34.65\\
1900	34.96\\
1900	47.44\\
1900	40\\
1900	34\\
1900	29.94\\
1900	27.55\\
1900	29.94\\
1900	31.53\\
1900	32.86\\
1900	35.96\\
1900	35.98\\
1900	35.91\\
1900	35.97\\
1900	29.89\\
2300	27.72\\
2300	25.27\\
2300	24.02\\
2300	22.69\\
2300	21.04\\
2300	19.61\\
2300	18.09\\
2300	19.99\\
2300	23.66\\
2300	25.62\\
2300	28.08\\
2300	34.94\\
2300	39.94\\
2300	34\\
2300	29.89\\
2300	28.04\\
2300	28.03\\
2300	34.94\\
2300	35\\
2300	35\\
2300	35\\
2300	32.99\\
2300	42.44\\
2300	34\\
2200	35\\
2200	27.4\\
2195	17.57\\
2200	24.01\\
2200	26.88\\
2200	26.71\\
2200	34\\
2200	41.46\\
2200	41.96\\
2200	40.78\\
2200	38.01\\
2200	39.94\\
2200	33\\
2200	32.44\\
2200	29.94\\
2200	29.7\\
2200	29.94\\
2200	33.97\\
2200	37.5\\
2200	38.87\\
2200	38.08\\
2200	35.93\\
2200	35.91\\
2200	30\\
2200	28.21\\
2200	26.1\\
2200	25.22\\
2200	25.07\\
2200	25.23\\
2200	27.05\\
2200	29.62\\
2200	28.24\\
2200	36.1\\
2200	34.92\\
2200	35.23\\
2170	36.94\\
1905	35.93\\
1954	34.92\\
2156	34\\
2200	34.94\\
2200	40\\
2200	42.44\\
2200	37.49\\
2200	39.7\\
2200	39.81\\
2200	38.98\\
2044	38.04\\
1799	33.54\\
2200	29.35\\
2200	28.93\\
2200	28.94\\
2200	28.91\\
2200	26.79\\
2200	28.33\\
2200	33.74\\
2200	40.3\\
2200	41.8\\
2200	44.94\\
2200	46.01\\
2200	46.42\\
2200	40\\
2200	42.44\\
2200	39.94\\
2200	39\\
2169	34.94\\
2200	36.12\\
2200	38.47\\
2200	42.28\\
2200	42.06\\
2200	39.24\\
2200	39.81\\
1966	34.45\\
2300	31.36\\
2300	28.56\\
2300	28.01\\
2300	28.31\\
2300	28.21\\
2300	29.59\\
2300	35.85\\
2300	42.1\\
2300	43.05\\
2300	46.73\\
2300	46.54\\
2300	41\\
2300	49.67\\
2300	54.94\\
2300	50.11\\
2300	48.99\\
2300	50\\
2300	50.11\\
2300	42.36\\
2300	42.64\\
2300	40.65\\
2300	38.94\\
2269	38.89\\
2259	33.15\\
2300	31.09\\
2300	29.44\\
2300	28.4\\
2300	28.09\\
2300	27.69\\
2300	29.44\\
2300	35.44\\
2300	41.75\\
2300	42.44\\
2247	40.72\\
2010	39.77\\
2079	38.1\\
1669	36.99\\
1632	34.26\\
2153	33.93\\
2300	34.84\\
2300	34.94\\
2288	37.93\\
2300	41.16\\
2300	43.03\\
2300	40.73\\
2300	36.93\\
2161	39.5\\
1697	37.92\\
2300	34.47\\
2300	31.42\\
2300	30.68\\
2300	30.68\\
2300	29.23\\
2300	26.24\\
2118	9.74\\
1990	14.54\\
2011	22.03\\
2175	27.3\\
2278	29.32\\
1726	27.31\\
1316	26.41\\
1462	23.27\\
1487	23.94\\
1793	26\\
2105	26.02\\
2300	30.91\\
2300	35\\
2300	35.02\\
2300	37.44\\
2300	32.35\\
2300	46.7\\
2300	46.92\\
2300	47.76\\
2300	32.87\\
2300	30.24\\
2300	21.62\\
2300	18.42\\
2300	15.67\\
2300	16.76\\
2300	20.38\\
2300	19.96\\
2300	31.13\\
2300	35\\
2300	41.63\\
2300	35\\
2300	35\\
2300	33\\
2300	33.75\\
2300	34.02\\
2300	34.94\\
2300	45\\
2300	40\\
2300	42.44\\
2300	39.1\\
2300	37.99\\
2300	31.16\\
2300	34.94\\
2300	23.86\\
2300	22.81\\
2300	22.99\\
2300	22.69\\
1776	15.17\\
2300	30.6\\
1927	24.77\\
1403	31.35\\
1566	37.19\\
1725	42.86\\
1485	44.29\\
1419	43.51\\
1768	41.78\\
1570	38.44\\
1470	35.96\\
1533	34.14\\
1712	33.95\\
1856	35.52\\
1938	35.92\\
2183	32.87\\
2300	30.33\\
2300	29.1\\
2300	29.04\\
2200	28\\
2200	26.17\\
2096	21.92\\
2200	25.09\\
2148	11.67\\
2187	19.67\\
2200	29.73\\
2200	37.92\\
2200	42.69\\
2200	41.95\\
2200	42.92\\
2121	42.97\\
1973	38.04\\
2108	35.03\\
2165	33.65\\
2200	32.39\\
2191	32.18\\
2200	35\\
2200	35.97\\
2200	37.94\\
2200	35\\
2200	33.07\\
2034	35.95\\
2200	31.88\\
2200	29.98\\
2200	29.05\\
2200	25.81\\
2200	25.32\\
2200	25.94\\
2200	27.3\\
2200	30.9\\
2200	38.67\\
2200	38.07\\
2127	36.5\\
2158	35.94\\
2168	36.67\\
1899	34\\
1919	34.23\\
1980	32.23\\
2200	32.44\\
2200	34.03\\
2200	44.94\\
2200	36.4\\
2200	39.08\\
2200	35.93\\
2200	32.88\\
2200	44.92\\
2200	34\\
2200	29.63\\
2200	26.64\\
2200	25.79\\
2200	25.05\\
2200	25.62\\
2200	27.32\\
2200	29.62\\
2200	36.49\\
2130	36.9\\
1952	35.45\\
1732	35.23\\
1856	36.41\\
1858	36.97\\
1813	36.55\\
1770	35\\
1822	34.99\\
2075	35.4\\
2200	39.94\\
2200	39.93\\
2166	41.36\\
2200	40.07\\
2200	39.95\\
2200	40.34\\
2200	40.5\\
2200	31.13\\
2200	30.14\\
2200	30.45\\
2200	30.14\\
2200	30.14\\
2200	28.85\\
2200	32.66\\
2200	40\\
2200	41.42\\
2200	46.03\\
2200	48.96\\
2200	50.01\\
2200	48.96\\
2200	49.99\\
2200	48.96\\
2200	46.46\\
2200	45.86\\
2200	39.99\\
2200	38.87\\
2200	38.16\\
2200	36.12\\
2200	35.01\\
2200	36.91\\
2200	35\\
2200	36.5\\
2200	32.09\\
2200	30.11\\
2200	27.98\\
2200	27.99\\
2200	27.67\\
2200	29.31\\
2200	30.1\\
2200	32.07\\
2200	36\\
2200	40\\
2200	46.56\\
2200	40\\
2200	36\\
2200	32.46\\
2200	32.44\\
2200	31.91\\
2200	32.96\\
2200	35.91\\
2200	38.05\\
2200	38.9\\
2200	38.1\\
2200	47.03\\
2200	36\\
2200	31.04\\
2200	29.23\\
2200	28.56\\
2200	27.95\\
2200	27.12\\
2200	26.83\\
2200	26.91\\
2200	26.04\\
2200	26.07\\
2200	29.17\\
2158	17.03\\
2188	19.28\\
2066	22.48\\
2012	18.59\\
2139	15.91\\
2177	15.73\\
2143	13.87\\
2200	32.46\\
2200	34.94\\
2200	37.11\\
2200	36\\
2200	44.11\\
2200	48.96\\
2200	37.11\\
2100	36\\
2100	30.64\\
2100	29.08\\
2100	27.67\\
2100	27.84\\
2100	28.97\\
2100	36.98\\
2100	41.97\\
2100	43\\
2100	41.28\\
2100	39.99\\
2100	47\\
2100	42.44\\
2100	42.44\\
2100	40.23\\
2100	37.44\\
2100	37.94\\
2100	41.02\\
2100	41.58\\
2100	42.95\\
2100	39.93\\
2100	35.99\\
2100	36\\
2100	30.44\\
2100	29.73\\
2100	28.58\\
2089	11.2\\
2095	5.23\\
2100	25.85\\
2100	29.31\\
2100	32.48\\
2100	39.41\\
2100	42.5\\
2100	42.31\\
2100	42.5\\
2100	42.5\\
2100	42.5\\
2100	42.5\\
2100	36\\
2100	34.97\\
2100	34.94\\
2100	36.48\\
2100	36.5\\
2100	35.78\\
2100	35.94\\
2100	36\\
2100	48.97\\
2100	34.1\\
2100	33.1\\
2100	33.7\\
2100	33.7\\
2100	31.97\\
2100	31.57\\
2100	31.54\\
2100	32.17\\
2100	35.98\\
2100	41.03\\
2100	40.59\\
2100	39.98\\
2100	39.99\\
2100	37.44\\
2100	38\\
2100	38\\
2100	38\\
2100	38\\
2100	48.96\\
2100	50\\
2100	44.94\\
2100	39.94\\
2100	34.1\\
2100	49.94\\
2100	44.58\\
2100	32.59\\
2100	32.68\\
2100	27.19\\
2100	31.41\\
2100	30\\
2100	31.71\\
2100	32.42\\
2100	36.93\\
2100	38\\
2100	34.94\\
2100	38\\
2100	34.94\\
2100	34.71\\
2100	35.81\\
2100	38\\
2100	39.94\\
2100	43.05\\
2100	44.94\\
2100	35\\
2100	38\\
2100	40\\
2100	35.97\\
2100	41.15\\
2100	38\\
2100	31.93\\
2100	31.62\\
2100	31.58\\
2100	31.57\\
2100	31.61\\
2100	31.64\\
2100	37.78\\
2100	40.3\\
2100	39.96\\
2100	44.03\\
2100	48.97\\
2100	46.44\\
2100	39.99\\
2100	38.74\\
2100	37.44\\
2100	35.02\\
2100	42.97\\
2100	47.51\\
2100	46.44\\
2100	45.26\\
2100	48.9\\
2100	48.77\\
2100	57.44\\
2100	44.13\\
2100	32.03\\
2100	32.03\\
2100	34.77\\
2100	32.03\\
2100	31.67\\
2100	29.99\\
2100	27.27\\
2100	29.06\\
2100	29.94\\
2100	38.1\\
2100	41\\
2100	44.94\\
2100	30.01\\
2100	42.44\\
2100	42.44\\
2100	37.73\\
2100	37.68\\
2100	41.45\\
2100	49\\
2100	39.78\\
2100	45.63\\
2100	44.58\\
2100	49.28\\
2100	39.78\\
2100	31.31\\
2100	28.48\\
2100	28.09\\
2100	29.2\\
2100	28.91\\
2100	27.15\\
2097	10.59\\
2042	10.9\\
1925	12.93\\
1798	13.47\\
1823	13.73\\
1832	14.92\\
1701	15.73\\
1467	14.6\\
1657	14.44\\
1485	13.9\\
1704	13.5\\
1688	14.51\\
1779	17.39\\
2100	30.79\\
2100	38\\
2100	46.48\\
2100	52.44\\
2100	39.8\\
2200	27.59\\
2200	27.59\\
2157	17.91\\
2200	27.56\\
2200	29.4\\
2200	29.56\\
2200	32.6\\
2200	38.09\\
2200	37.4\\
2200	38.7\\
2200	39.5\\
2200	43.4\\
2200	39.41\\
2200	38\\
2200	35.4\\
2200	35\\
2200	35\\
2200	40\\
2200	36.94\\
2200	38.3\\
2200	38.09\\
2200	39.94\\
2200	47.44\\
2200	32.73\\
2200	29.24\\
2200	29.14\\
2200	29.19\\
2200	29.93\\
2200	29.95\\
2200	29.95\\
2200	32.58\\
2200	38.6\\
2200	41.04\\
2200	45.08\\
2200	49.69\\
2200	45.11\\
2200	47\\
2200	45.19\\
2200	44.94\\
2200	39.02\\
2200	42\\
2200	52.44\\
2200	39.49\\
2200	39.93\\
2200	38.99\\
2200	38.89\\
2200	35\\
2200	30.75\\
2200	28.95\\
2200	28.67\\
2200	28.22\\
2200	28.08\\
2200	27.78\\
2200	28.18\\
2200	30.76\\
2200	37.73\\
2200	40.19\\
2200	36.82\\
2200	37.44\\
2200	40\\
2200	36.82\\
2200	34.94\\
2200	32.65\\
2200	31.22\\
2200	32.44\\
2200	33.01\\
2200	37.9\\
2200	39.37\\
2200	38.76\\
2200	37.78\\
2200	32.75\\
2200	30.35\\
2000	31.21\\
2000	28.64\\
2000	28.41\\
2000	29.06\\
2000	29.04\\
2000	28.5\\
2000	30.61\\
2000	35.62\\
2000	37.95\\
2000	41\\
2000	47\\
2000	44.94\\
2000	39.94\\
2000	39.94\\
2000	39.5\\
2000	37\\
2000	42.44\\
2000	47.55\\
2000	46.44\\
2000	39.22\\
2000	38.63\\
2000	38.3\\
2000	47\\
2000	39.99\\
2200	33.71\\
2200	31.2\\
2200	30.44\\
2200	30.45\\
2200	30\\
2200	29.94\\
2200	39.97\\
2200	44\\
2200	42.03\\
2200	44.44\\
2200	42.83\\
2200	44.99\\
2200	42.44\\
2200	59.94\\
2200	52.63\\
2200	52.63\\
2200	52.63\\
2200	52.27\\
2200	44.94\\
2200	41.07\\
2200	36.87\\
2200	34.53\\
2200	32.8\\
2200	38.16\\
2100	47.52\\
2100	30\\
2100	27.44\\
2100	27.17\\
2100	25.3\\
2100	24.6\\
2100	24.1\\
2100	26.03\\
2100	29.98\\
2100	34\\
2100	34.94\\
2100	35\\
2100	33.1\\
2100	31.9\\
2100	30.62\\
2100	30.6\\
2100	32.7\\
2100	35\\
2100	34.96\\
2100	39.94\\
2100	35.66\\
2100	43.78\\
2100	47.44\\
2100	42.44\\
2100	34.41\\
2100	34\\
2100	34.41\\
2100	30.43\\
2100	29.68\\
2100	27.27\\
2100	26.18\\
2100	27.2\\
2100	28.03\\
2100	29.94\\
2100	32.44\\
2100	33.7\\
2100	33\\
2100	32.44\\
2100	30.43\\
2100	29.93\\
2100	29.94\\
2100	32.44\\
2100	32.7\\
2100	33\\
2100	32.7\\
2100	43.26\\
2100	47.36\\
2100	29.19\\
2200	26.17\\
2200	13.45\\
2137	8.99\\
2120	5.37\\
2200	21.95\\
2200	24.8\\
2200	29.4\\
2200	31.97\\
2200	34.96\\
2200	32.9\\
2200	32.96\\
2200	34.94\\
2200	35\\
2200	35.37\\
2200	33\\
2200	33.7\\
2200	35.7\\
2200	37.87\\
2200	34.94\\
2200	37.27\\
2200	36.91\\
2200	38.32\\
2200	35.7\\
2200	29.94\\
2200	30.85\\
2200	29.87\\
2200	26\\
2200	25.1\\
2200	27.48\\
2200	26.96\\
2200	35.01\\
2200	41.05\\
2200	39.18\\
2200	38.28\\
2200	35.12\\
2200	37.2\\
2200	33.3\\
2200	34.94\\
2200	34.94\\
2200	34.94\\
2200	37.2\\
2200	43\\
2200	41\\
2200	41.1\\
2200	42.44\\
2200	39.86\\
2200	40\\
2200	37.2\\
2200	30.86\\
2200	30.12\\
2200	30.54\\
2200	30.45\\
2200	30.05\\
2200	30.56\\
2200	33.95\\
2200	38.19\\
2200	39.98\\
2200	41.78\\
2200	44.8\\
2200	48\\
2200	42.11\\
2200	42.98\\
2200	41.97\\
2200	41.26\\
2200	40.1\\
2200	45.51\\
2200	42.97\\
2200	39.97\\
2200	37.23\\
2200	38.65\\
2200	37\\
2200	34\\
2200	30.82\\
2200	30.86\\
2200	29.86\\
2200	29.08\\
2200	28.97\\
2200	29.85\\
2200	33.99\\
2200	39.03\\
2200	40.63\\
2200	37.27\\
2200	42.44\\
2200	42.44\\
2200	39.94\\
2200	39.94\\
2200	39.94\\
2200	39.94\\
2200	42.44\\
2200	47.44\\
2200	39.94\\
2200	41\\
2200	41.08\\
2200	40.97\\
2200	37.43\\
2200	35.99\\
2200	31.15\\
2200	31.21\\
2200	30.3\\
2200	29.78\\
2200	29.79\\
2200	29.92\\
2200	31.36\\
2200	36.18\\
2200	35.91\\
2200	37.97\\
2200	40.07\\
2200	41.05\\
2200	37.97\\
2200	36.4\\
2200	35\\
2200	33.17\\
2200	32.54\\
2200	34.99\\
2200	37.91\\
2200	40\\
2200	40\\
2200	37.97\\
2200	42.44\\
2200	39.94\\
2200	46.95\\
2200	32.5\\
2200	32.27\\
2200	31.9\\
2200	30.96\\
2200	31.64\\
2200	29.35\\
2200	31.9\\
2200	34.99\\
2200	45\\
2200	49.94\\
2200	49.94\\
2200	45.12\\
2200	37.44\\
2200	32.2\\
2200	31.9\\
2200	31.72\\
2200	32.5\\
2200	40\\
2200	44\\
2200	41.11\\
2200	47.19\\
2200	50.56\\
2200	32.5\\
2200	30.96\\
2200	29.64\\
2200	28.03\\
2200	27.08\\
2177	5.51\\
2200	8.06\\
2200	18.23\\
2200	24.33\\
2200	25\\
2200	31.31\\
2200	29.96\\
2200	30.11\\
2200	31.9\\
2200	30.97\\
2200	30.12\\
2200	17.44\\
2200	31.49\\
2200	32.92\\
2200	39.96\\
2200	47\\
2200	44.94\\
2200	40\\
2200	39.94\\
2200	32.45\\
2200	41.99\\
2200	28.51\\
2200	28.07\\
2200	28.09\\
2200	28.28\\
2200	30.03\\
2200	33.24\\
2200	35\\
2200	40\\
2200	43.78\\
2200	41.89\\
2200	45.85\\
2200	41.76\\
2200	48.53\\
2200	44.72\\
2200	42.22\\
2200	42.8\\
2200	47.44\\
2200	43.78\\
2200	43.78\\
2200	45\\
2200	48.97\\
2200	48.53\\
2200	48.53\\
2200	38.38\\
2200	36.11\\
2200	34.28\\
2200	33.45\\
2200	33.45\\
2200	33.59\\
2200	33.96\\
2200	35.6\\
2200	40.64\\
2200	52.44\\
2200	53.89\\
2200	55.69\\
2200	52.44\\
2200	40\\
2200	44.96\\
2200	39.99\\
2200	39.65\\
2200	35.03\\
2200	38\\
2200	39.94\\
2200	47.44\\
2200	37.18\\
2200	35.29\\
2200	30.63\\
2200	31.22\\
2200	29.94\\
2200	29.46\\
2200	29.46\\
2200	31.43\\
2200	32.18\\
2200	33.36\\
2200	40\\
2200	39.5\\
2200	52.44\\
2200	52.44\\
2200	55.04\\
2200	39.94\\
2200	49.99\\
2200	39.94\\
2200	40.78\\
2200	44.94\\
2200	58.25\\
2200	38.77\\
2200	42.33\\
2200	41.05\\
2200	45\\
2200	44.96\\
2200	39.5\\
2200	34.01\\
2200	32.66\\
2200	32.11\\
2200	27.96\\
2200	27.79\\
2200	29.16\\
2200	39.37\\
2200	66.87\\
2200	43.35\\
2200	52.2\\
2200	55\\
2200	59.3\\
2200	55\\
2200	40.17\\
2200	48\\
2200	43.78\\
2200	44.31\\
2200	45\\
2200	38.71\\
2200	41.02\\
2200	41.01\\
2200	39.98\\
2200	37.09\\
2200	32.94\\
2200	29.92\\
2200	28.99\\
2200	28.47\\
2200	28.05\\
2200	28.09\\
2200	29.37\\
2200	32.82\\
2200	38.91\\
2200	41.43\\
2200	38.2\\
2200	35.7\\
2200	37.44\\
2200	34.96\\
2200	39.37\\
2200	37.75\\
2200	38.3\\
2200	39.94\\
2200	44.94\\
2200	43.78\\
2200	40\\
2200	40\\
2200	42.5\\
2200	39.94\\
2200	40\\
2200	37.3\\
2200	39.97\\
2200	38.3\\
2200	32.7\\
2200	32.04\\
2200	33.37\\
2200	34.94\\
2200	33.37\\
2200	34.99\\
2200	35\\
2200	40\\
2200	39.38\\
2200	34.94\\
2200	33.37\\
2200	34.99\\
2200	32.54\\
2200	32.72\\
2200	39.94\\
2200	35.09\\
2200	35\\
2200	40\\
2200	44.94\\
2200	44.73\\
2200	48.75\\
2200	50.28\\
2200	39.45\\
2200	36.94\\
2200	32.7\\
2200	32.43\\
2200	32.42\\
2200	32.7\\
2200	31.74\\
2200	32.43\\
2200	33.59\\
2200	39.36\\
2200	42.44\\
2200	48\\
2200	40.5\\
2200	37.44\\
2200	35\\
2200	37.12\\
2200	44.94\\
2200	39.94\\
2200	34.94\\
2200	39.94\\
2200	57.44\\
2200	50.23\\
2200	47.06\\
2200	39.38\\
2200	33.87\\
2200	32.83\\
2200	29.3\\
2200	29.42\\
2200	30.13\\
2200	37.91\\
2200	42\\
2200	40.81\\
2200	43.24\\
2200	49.45\\
2200	60\\
2200	64.44\\
2200	64.35\\
2200	48.96\\
2200	43.52\\
2200	43.29\\
2200	46.39\\
2200	54.79\\
2181	44.01\\
2050	44.52\\
1767	40.45\\
2048	38.37\\
2072	37.26\\
2200	32.28\\
2200	32.01\\
2200	31.69\\
2200	32.86\\
2200	33.29\\
2200	33.54\\
2200	36.46\\
2200	43.9\\
2200	48.86\\
2017	51.98\\
2053	53.59\\
2200	54.27\\
2053	48.96\\
2168	47.17\\
2022	44.08\\
2200	42.94\\
2200	42.09\\
2200	42.28\\
1778	43.74\\
1903	42.59\\
1846	41.97\\
2200	45\\
2200	45.67\\
2200	35.99\\
2300	32.49\\
2300	32.6\\
2300	32.5\\
2300	32.37\\
2300	32.64\\
2300	32.96\\
2300	35.99\\
2300	41.99\\
2300	45.03\\
2300	49.5\\
2300	48\\
2300	49.94\\
2300	47.44\\
2300	48.96\\
2300	48.75\\
2300	43.27\\
2300	42.13\\
2300	50.95\\
2300	47.44\\
2300	43.6\\
2048	42.02\\
2063	40.65\\
2300	40\\
2300	37.33\\
2300	38.91\\
2300	32.55\\
2300	31.05\\
2300	29.47\\
2300	27.9\\
2300	31.84\\
2300	35.77\\
2300	44\\
2300	41.63\\
2300	44.96\\
2300	49.41\\
2300	51.42\\
2300	44\\
2300	46\\
2300	46\\
2300	48.96\\
2300	49.41\\
2300	49.41\\
2300	46\\
2300	41.99\\
2300	46\\
2300	51.42\\
2300	39.99\\
2300	44.59\\
2300	35.14\\
2300	32.31\\
2300	32.71\\
2300	31.14\\
2300	31.5\\
2300	29.96\\
2300	37.46\\
2300	44.5\\
2300	47.81\\
2300	54.44\\
2300	55.84\\
2300	54.94\\
2300	45\\
2300	39.94\\
2300	38.73\\
2300	37.46\\
2300	35.88\\
2300	37.44\\
2300	38\\
2141	37.71\\
2203	37.52\\
2300	36.72\\
2300	40\\
2300	35.14\\
2300	36.32\\
2300	33\\
2300	29.78\\
2300	28.67\\
2300	25.1\\
2300	27.7\\
2300	32.99\\
2300	34\\
2300	34.96\\
2300	44.96\\
2300	47.44\\
2300	49.5\\
2300	45\\
2300	38.49\\
2300	36.25\\
2300	34.96\\
2300	35\\
2300	39.94\\
2300	42.44\\
2300	44.96\\
2300	51.99\\
2300	54.94\\
2300	49.94\\
2300	42.48\\
2300	35.03\\
2300	34\\
2300	33.9\\
2300	31.41\\
2300	31.31\\
2300	31.27\\
2300	31.38\\
2300	31.74\\
2300	32.15\\
2300	34.06\\
2300	34.9\\
2300	35.05\\
2300	35.02\\
2300	34.9\\
2300	32.64\\
2300	32.44\\
2300	32.4\\
2300	34.9\\
2300	36.3\\
2300	45\\
2300	52.44\\
2300	52.44\\
2300	50\\
2300	42.68\\
2300	34.89\\
2300	33.64\\
2300	34.22\\
2300	34.47\\
2300	34.88\\
2300	34.89\\
2300	41.03\\
2300	41.03\\
2300	43.42\\
2300	44.94\\
2286	43.76\\
2300	47.44\\
2300	50.44\\
2300	47.44\\
2300	44.94\\
2300	44.94\\
2300	48.96\\
2300	44\\
2300	40.21\\
2300	50\\
2300	44.94\\
2300	48.96\\
2300	50.44\\
2300	44.99\\
2000	37.88\\
2000	38.35\\
2000	37.55\\
2000	37.08\\
2000	37.07\\
2000	36.81\\
2000	41.2\\
2000	74.36\\
2000	55.39\\
2000	59.94\\
2000	62.29\\
2000	63.66\\
2000	59.96\\
2000	60.01\\
2000	58.2\\
2000	55.24\\
2000	52.21\\
2000	64.94\\
2000	42.44\\
2000	42.21\\
2000	53.11\\
2000	45.2\\
2000	52\\
2000	47.44\\
2000	49.99\\
2000	41\\
2000	38.96\\
2000	38.55\\
2000	38.62\\
2000	37.81\\
2000	42.54\\
2000	57\\
2000	57.86\\
2000	60.5\\
2000	58.9\\
2000	55.78\\
2000	50\\
2000	50\\
2000	47.82\\
2000	52.07\\
2000	55.78\\
2000	69.9\\
2000	41\\
2000	39.96\\
2000	55.78\\
2000	55.78\\
2000	60.39\\
2000	42.44\\
2000	37.18\\
2000	35.96\\
2000	35.5\\
2000	35.03\\
2000	35.03\\
2000	35.94\\
2000	39.3\\
2000	47.44\\
2000	45.68\\
2000	54.7\\
2000	44.94\\
2000	55.78\\
2000	44.96\\
2000	50\\
2000	49.94\\
2000	47.03\\
2000	54.96\\
2000	55\\
2000	55.31\\
2000	50\\
2000	52.44\\
2000	44.12\\
2000	39.97\\
2000	49.99\\
2000	39.5\\
2000	39.94\\
2000	35.28\\
2000	35.28\\
2000	37.5\\
2000	50\\
2000	38.26\\
2000	55.78\\
2000	49.48\\
2000	60.75\\
2000	63.99\\
2000	61\\
2000	55\\
2000	60\\
2000	54.94\\
2000	49.05\\
2000	47.44\\
2000	55.83\\
2000	54.94\\
2000	42.36\\
2000	47.44\\
2000	45.86\\
2000	55\\
2000	46.9\\
2000	60.12\\
2000	49.99\\
2000	46.01\\
2000	42.44\\
2000	39.92\\
2000	39.46\\
2000	41.49\\
2000	43.07\\
2000	45.71\\
2000	49.91\\
2000	60\\
2000	58.69\\
2000	52.44\\
2000	39\\
2000	37.81\\
2000	36.57\\
2000	36.57\\
2000	39.81\\
2000	39.94\\
2000	39.94\\
2000	55\\
2000	49.99\\
2000	47.44\\
2000	40\\
2000	37.13\\
2000	36.29\\
2000	36.29\\
2000	31.97\\
2000	31.74\\
2000	31.68\\
2000	32.09\\
2000	31.83\\
2000	32.56\\
2000	36.91\\
2000	39.94\\
2000	39.99\\
2000	40\\
2000	40\\
2000	37.07\\
2000	36.91\\
2000	36.53\\
2000	42.44\\
2000	49.78\\
2000	48.19\\
2000	59.94\\
2000	57\\
2000	46\\
2000	36.67\\
2200	31.9\\
2200	30.99\\
2200	30.83\\
2200	30.8\\
2200	31.34\\
2200	32.58\\
2200	54.99\\
2200	60\\
2200	53.11\\
2200	53.11\\
2200	53.11\\
2114	43.21\\
1738	43.53\\
1611	42.77\\
1622	45.13\\
1687	45.07\\
1845	45.54\\
2192	43.93\\
2200	54\\
2200	54\\
2200	65.15\\
2200	70\\
2200	57.11\\
2200	36.92\\
2200	35.45\\
2200	39\\
2200	41.42\\
2200	39\\
2200	35.47\\
2200	35.47\\
2200	37.95\\
2200	45.76\\
2200	50.95\\
2200	54\\
2200	53.75\\
2200	53.1\\
2200	45\\
2200	44.94\\
2200	48\\
2200	45.9\\
2200	44.19\\
2200	60\\
2200	45.38\\
1921	45.9\\
2200	53.78\\
2154	41.89\\
2200	44.17\\
2200	37.17\\
2200	35.24\\
2200	31.53\\
2200	31.33\\
2200	31.27\\
2200	31.73\\
2200	32.61\\
2200	43.04\\
2200	57.11\\
2200	57.11\\
2041	46.17\\
2040	44.01\\
2200	45\\
1971	42.45\\
2200	50\\
2200	45\\
2200	43.24\\
2200	46.09\\
2174	42.98\\
2200	55.1\\
1883	46.24\\
1908	56.23\\
2190	42.88\\
2200	72.09\\
2200	46\\
2200	35.39\\
2200	34.66\\
2200	31.55\\
2200	31.51\\
2200	34.71\\
2200	34.91\\
2200	56\\
2200	45.96\\
2163	47.67\\
2200	65.4\\
2131	45.85\\
2083	45.96\\
1890	44.42\\
1641	42.47\\
1925	41.05\\
1862	39.66\\
1944	38.48\\
2088	41.46\\
2075	45.91\\
1938	47.1\\
1850	46.18\\
1966	42.64\\
1969	37.8\\
1833	36.29\\
2200	32.54\\
2200	32.21\\
2200	32.14\\
2200	32.2\\
2200	32.22\\
2200	32.46\\
2200	40.52\\
1979	44.94\\
1972	47.39\\
2156	47.8\\
1860	47.32\\
1724	47.09\\
1652	46.24\\
1520	42.91\\
1651	39.34\\
1601	38.1\\
2185	35.85\\
2200	42\\
2200	39.38\\
1823	40.97\\
2200	50\\
2200	43.74\\
2200	50\\
2200	44\\
2200	36.27\\
2200	34.34\\
2200	34.09\\
2200	32.81\\
2200	32.77\\
2200	32.84\\
2200	33.96\\
2200	34.29\\
2200	41.52\\
2200	44.94\\
2200	45\\
2200	40\\
2200	37.93\\
2200	37.85\\
2200	35.2\\
2200	34.94\\
2200	35.18\\
2200	38.52\\
2200	43\\
2200	49.94\\
2200	59.94\\
2200	50\\
2200	40.88\\
2200	37.99\\
2200	34.33\\
2200	32.7\\
2200	31.95\\
2200	32.18\\
2200	32.22\\
2200	32.11\\
2200	32.51\\
2200	32.71\\
2200	33.38\\
2200	37.44\\
2200	40\\
2200	44.94\\
2200	49.99\\
2200	38.5\\
2200	36.87\\
2200	34.96\\
2200	35.87\\
2200	39.94\\
2200	38.5\\
2200	49.11\\
2200	59.94\\
2200	49.94\\
2200	48\\
2200	46\\
2200	47.46\\
2200	36.87\\
2200	36.65\\
2200	35.41\\
2200	35.23\\
2200	35.23\\
2200	43.27\\
1729	45.01\\
2082	47.92\\
1958	48.37\\
1809	49\\
1962	49.2\\
1858	47.75\\
1930	48.95\\
1954	48.56\\
2183	46.74\\
2200	59.58\\
2200	69.94\\
2200	53.11\\
2199	49.95\\
2098	48.41\\
1995	41.48\\
1763	41.93\\
1813	38.43\\
2200	34.99\\
2200	33.44\\
2200	33.24\\
2200	33.24\\
2200	33.57\\
2200	34.05\\
2200	45.01\\
2200	74.87\\
2200	57.11\\
2200	57.11\\
2200	57.11\\
2200	50.55\\
1792	47.23\\
1837	47.44\\
1857	46.91\\
2200	57.11\\
2200	49.94\\
2200	57.05\\
2200	57.11\\
2200	57.11\\
2200	74.95\\
2200	60\\
2200	57.11\\
2200	40\\
2200	37.56\\
2200	33.14\\
2200	32.5\\
2200	32.53\\
2200	33.44\\
2200	34.01\\
2200	58.13\\
2200	55.78\\
2200	69.21\\
2200	71\\
2200	55.78\\
2200	54.94\\
1881	52.5\\
2200	57.04\\
2200	57.05\\
2200	57.04\\
2200	57.04\\
2200	57.57\\
2200	47.74\\
2200	55.78\\
2200	55.78\\
2200	53.12\\
2200	46.46\\
2200	45.65\\
2200	37.61\\
2200	34.21\\
2200	34.21\\
2200	34.21\\
2200	37.61\\
2200	37.61\\
2200	42\\
2178	49.71\\
2200	55.78\\
2200	55.78\\
2200	57.04\\
2200	55.78\\
1910	48.7\\
1849	47.95\\
1878	47.93\\
1950	47.31\\
2050	45.74\\
2200	55.78\\
2200	55.78\\
2200	55.78\\
2200	55.78\\
2200	55\\
2200	55.12\\
2200	42\\
2200	70\\
2200	37.95\\
2200	37.95\\
2200	37.95\\
2200	37.95\\
2200	42\\
2200	60\\
2200	51.78\\
2200	51.78\\
2200	51.78\\
2200	58.95\\
2200	62\\
2191	48.51\\
2200	49.94\\
2200	57.04\\
2200	46.38\\
2200	50\\
2200	57.04\\
2200	51.78\\
2200	51.78\\
2200	60.01\\
2200	46.1\\
2200	52.44\\
2200	59.94\\
2200	53\\
2200	37.85\\
2200	39.01\\
2200	33.59\\
2200	33.37\\
2200	36.57\\
2200	39.99\\
2200	38.84\\
2200	46.44\\
2200	57.05\\
2200	53\\
2200	53\\
2200	48.24\\
2200	42\\
1966	40.24\\
2020	38.16\\
2200	38\\
2200	46.44\\
2200	52.44\\
2200	57.05\\
2200	69.96\\
2200	52.44\\
2200	44.94\\
2125	43.54\\
2200	42\\
2200	36.63\\
2200	35.18\\
2200	32.31\\
2200	32.05\\
2200	31.92\\
2200	34\\
2200	32.33\\
2200	31.7\\
2200	35.28\\
2200	34.17\\
2200	34.99\\
2200	37.08\\
2200	33.27\\
2200	32.82\\
2200	33\\
2200	32.69\\
2200	35.18\\
2200	42\\
2200	51\\
2200	60\\
2200	51\\
2200	42\\
2170	35.47\\
2200	36.63\\
2200	36.63\\
2200	33.73\\
2200	35.18\\
2200	35.59\\
2200	33.73\\
2200	44.99\\
2200	76.99\\
2200	54\\
2200	54\\
2139	49.31\\
2077	50.47\\
1880	51.24\\
1970	49.14\\
1954	48.76\\
1958	47.95\\
2105	47.34\\
2192	46.66\\
2200	54\\
2200	54\\
2200	70\\
2200	54\\
2093	44.58\\
2200	45\\
2200	59.99\\
2200	45\\
2200	38.4\\
2200	38.4\\
2200	38.4\\
2200	44.99\\
2200	72.2\\
2200	74.92\\
2200	60.27\\
2200	52.44\\
2200	50.44\\
2200	47.49\\
1892	44.06\\
2116	42.56\\
2048	43\\
2200	45.94\\
2200	45\\
2200	60.22\\
2200	72.2\\
2183	69.3\\
2184	64.94\\
2072	47.33\\
2200	65\\
2200	76.66\\
1968	40.43\\
2200	37.1\\
2200	33.3\\
2200	33.3\\
2200	33.14\\
2200	33.51\\
2200	42\\
1947	52.43\\
1859	54.01\\
1875	57.78\\
1706	55.49\\
1779	57.84\\
1599	51.56\\
1834	53.65\\
1793	50.26\\
1787	48.26\\
1854	47.05\\
1854	46.91\\
1751	59.17\\
1784	59.92\\
1832	52.1\\
1670	48.36\\
1932	48.43\\
1936	44.38\\
2200	47.7\\
2200	37.87\\
2200	32.61\\
2200	32.37\\
2200	32.9\\
2200	35.76\\
2200	51.7\\
2200	55.5\\
2200	61\\
2200	69.32\\
2200	60.21\\
2047	57.33\\
1820	49.95\\
1667	50.07\\
1614	48.5\\
1867	47.78\\
2015	46.05\\
2147	44.89\\
2200	51.78\\
2199	53.93\\
2200	74.57\\
2200	49.96\\
2200	49.94\\
2074	45.03\\
2200	44.16\\
2200	37.87\\
2200	37.87\\
2200	35.6\\
2200	34.29\\
2200	34.33\\
2191	40.06\\
2133	52.48\\
1966	52.37\\
2200	57.04\\
2200	59.94\\
2200	56\\
2200	52.13\\
2200	53.44\\
2200	48.5\\
2200	48.5\\
2200	48.5\\
2200	52.06\\
2200	50.44\\
2200	62.13\\
2200	57.44\\
2200	55\\
2200	50.44\\
2200	48.5\\
2200	59.46\\
2200	45.91\\
2200	40.36\\
2200	38.87\\
2200	38.87\\
2200	38.87\\
2200	40.11\\
2200	46.93\\
2200	50.84\\
2200	55\\
2200	59.94\\
2200	57.05\\
2200	50.44\\
2200	42.44\\
2200	39.99\\
2200	39.55\\
2200	40.69\\
2200	48.76\\
2200	55\\
2200	61\\
2200	61\\
2200	50.44\\
2200	43.95\\
2200	41.43\\
2200	43\\
2200	35.3\\
2200	35.24\\
2200	33.95\\
2200	32.67\\
2200	32.35\\
2200	33.32\\
2200	32.35\\
2200	37.67\\
2200	43.4\\
2200	47.44\\
2200	52.44\\
2200	51.03\\
2200	41.12\\
2200	37.83\\
2200	36.21\\
2200	36.53\\
2200	44.94\\
2200	53.95\\
2200	59.92\\
2200	75\\
2200	59.94\\
2200	50\\
2200	42.44\\
2200	32.74\\
2200	32.67\\
2200	32.74\\
2200	32.23\\
2200	32.61\\
2200	33.47\\
2200	44.95\\
1820	53.97\\
1931	57.54\\
2127	55.88\\
2082	50.12\\
2176	51.06\\
2058	51.67\\
2200	51.78\\
2065	48.23\\
2200	47.5\\
2200	48.34\\
2200	60.01\\
2200	60\\
2190	63.24\\
2200	57.05\\
2200	47.5\\
2200	47.5\\
2200	40.7\\
2200	36.51\\
2200	35.54\\
2200	36.57\\
2200	37.6\\
2200	37.6\\
2200	34.15\\
2200	44.49\\
1888	54.76\\
2013	58.41\\
2194	56.6\\
2112	54.56\\
2037	55.93\\
1928	52\\
1966	50.6\\
1928	48.59\\
2009	46.31\\
2095	48.67\\
2200	50.97\\
2200	53.11\\
2200	66.93\\
2200	64\\
2200	52\\
2200	52\\
2200	43\\
1354	46.44\\
2115	38.09\\
2200	32.01\\
2200	31.58\\
2200	32.13\\
2187	35.71\\
1773	48.11\\
1737	58\\
1443	60.88\\
1313	65\\
1068	65.08\\
1086	65.19\\
900	63.65\\
861	64.81\\
899	60.9\\
1180	58\\
1431	52.7\\
1651	54.74\\
1738	56.75\\
1488	69.45\\
1270	70.98\\
1657	50.59\\
1448	54\\
1674	48\\
1710	38.29\\
2200	40.81\\
2200	38.29\\
2200	38.29\\
2200	39.88\\
2200	40.46\\
2200	54.3\\
1919	59.98\\
2009	59.64\\
1975	55.15\\
2200	54.3\\
2200	54.3\\
1970	55.98\\
1958	54.37\\
2200	56\\
2200	56\\
2200	54.3\\
2200	70\\
2200	54.3\\
1837	63.38\\
1969	61.04\\
2200	54.3\\
2200	54.3\\
2200	54.3\\
1823	50.98\\
2200	45\\
2200	41.47\\
2200	41.34\\
2200	41.52\\
2200	49\\
2200	50\\
2168	56.03\\
1981	56.42\\
1958	59.53\\
1708	59.88\\
1779	60.43\\
1448	60.01\\
1331	56.48\\
1444	54\\
1672	50.27\\
1905	49.47\\
2166	51.88\\
1774	54\\
1665	74.6\\
1580	79.03\\
1809	54\\
1474	56.96\\
1526	55.94\\
1859	38.65\\
2200	37.59\\
2200	38.69\\
2200	33.2\\
2200	31.35\\
2200	30.6\\
2200	30.83\\
2136	34.02\\
2200	38.19\\
2200	44.42\\
2200	42.44\\
2200	40.95\\
1859	41.86\\
1582	40.6\\
1773	38.41\\
2047	37.9\\
2200	38.16\\
2200	44.94\\
2200	52.44\\
2200	64.94\\
2200	53.11\\
2200	48.8\\
2200	46.5\\
1720	44.78\\
1559	38.91\\
2023	36.45\\
2200	32.43\\
2200	35.2\\
2200	34.99\\
2200	35.8\\
2200	32.8\\
2200	34.48\\
2200	38.64\\
2200	43\\
2200	42.44\\
2200	45.2\\
2200	45.2\\
2200	43\\
2200	39.94\\
2200	39.8\\
2200	41.32\\
2200	48.72\\
2200	46\\
2200	60\\
2200	59.94\\
2200	53.11\\
2200	45.97\\
2200	44.95\\
1391	43.52\\
2200	37.27\\
2200	37.95\\
2200	37.6\\
2200	37.18\\
2200	32.43\\
1984	43.24\\
1721	52.78\\
1543	56.79\\
1496	55.89\\
1343	57.6\\
1242	58.94\\
907	59.31\\
1066	62.6\\
1066	59\\
1118	54.61\\
1374	51.78\\
1846	50.27\\
1349	59.3\\
878	84.15\\
793	68.49\\
1199	48.66\\
572	51.71\\
731	49.15\\
648	40.5\\
2200	36.63\\
2200	32.74\\
2200	32.33\\
2200	32.9\\
2200	38.9\\
2147	38.99\\
1049	56.79\\
751	60.08\\
549	61.66\\
309	58.12\\
292	57.85\\
-262	57.21\\
37	54.92\\
323	48.29\\
586	46.47\\
751	44.42\\
1052	44.39\\
893	47.62\\
637	65.01\\
534	57.46\\
913	49.99\\
847	50\\
982	48.69\\
1215	41.26\\
2200	37.88\\
2200	37.16\\
2200	33.16\\
2200	37.02\\
2200	37.98\\
2162	38.51\\
1397	51.96\\
1121	55.31\\
1240	54.84\\
1108	53.77\\
1126	52.87\\
1104	51.29\\
1483	46.07\\
1615	44.96\\
1563	42.26\\
2143	40\\
2200	40.53\\
1867	41.14\\
881	54.95\\
1088	51\\
1542	44.94\\
1438	42\\
1471	38.47\\
2289	32.53\\
2300	34.94\\
2300	33.12\\
2300	32.56\\
2300	33\\
2300	37.23\\
2300	37.92\\
1280	51.74\\
1282	53.11\\
1130	57.44\\
1230	51.17\\
1002	50.6\\
473	52.94\\
595	50.67\\
887	48\\
1088	43.3\\
1400	41.08\\
1592	43.79\\
1079	51.83\\
1079	61.68\\
919	53.11\\
1424	42.8\\
1053	46.48\\
1174	46.18\\
1615	37.88\\
2199	32.27\\
2179	28.93\\
2300	28.92\\
2300	29.64\\
2300	30.86\\
1845	37.41\\
1070	51.94\\
944	54.15\\
942	55.82\\
816	53.23\\
933	52.35\\
404	52.73\\
654	51.5\\
716	49.55\\
917	47.09\\
1148	45\\
1451	49.89\\
1185	52.22\\
1104	58.41\\
1106	52.06\\
1453	42.94\\
1639	42.1\\
1696	40.76\\
2200	38.85\\
2200	38.96\\
2200	36.05\\
2200	33.54\\
2200	33.54\\
2200	35.25\\
2200	39.15\\
2200	41\\
2200	46.36\\
2200	56.8\\
2200	62.23\\
2200	61.55\\
2200	49.5\\
2200	44.94\\
2200	42\\
2200	42\\
2200	44.94\\
2200	46.42\\
2200	43\\
2200	53.11\\
2200	47\\
2017	39.31\\
1562	39.66\\
1486	38.62\\
2091	34.75\\
2300	30.96\\
2300	33.85\\
2300	33.54\\
2300	32.69\\
2300	33.54\\
2300	31.16\\
2300	31.13\\
2300	34.94\\
2300	38.7\\
2300	40.12\\
2300	40.25\\
2300	40.63\\
2300	39.94\\
2300	38.8\\
2300	37.64\\
2300	38.99\\
2300	44.94\\
2300	47.44\\
2300	54.99\\
2108	49\\
2156	40\\
1718	39.25\\
1178	35.52\\
2200	29.4\\
2200	30.94\\
2200	30.71\\
2200	30.82\\
2200	32.16\\
2200	32.69\\
2000	51.8\\
2000	54.94\\
1821	55.95\\
1448	56.66\\
1630	51.84\\
1632	51.8\\
1343	51.2\\
1448	47.06\\
1689	43.26\\
1616	43.95\\
1769	44.92\\
2000	47.24\\
1711	51.8\\
1383	62.47\\
1362	59.91\\
1654	49.37\\
1653	47.67\\
1927	46.46\\
2000	36.68\\
2000	39.06\\
2000	31.83\\
2000	31.07\\
2000	36.67\\
2000	39.62\\
2000	40.41\\
1761	54.87\\
1368	57.34\\
1486	55.95\\
1361	53.79\\
1301	53.99\\
1040	54.22\\
1412	52.87\\
1488	50.16\\
1775	47.32\\
1979	42.86\\
2000	53.6\\
1811	57.77\\
1986	67.8\\
1844	59.89\\
2000	49.5\\
2000	48.5\\
2000	51.3\\
2000	40.28\\
2000	39.45\\
2000	41.3\\
2000	41.33\\
2000	51.36\\
2000	51.36\\
2000	51.36\\
2000	97.81\\
1857	54.15\\
2000	150\\
2000	77.1\\
2000	60.74\\
1813	54.66\\
1949	53.12\\
2000	51.5\\
2000	60.74\\
2000	70\\
2000	80\\
2000	55\\
2000	120\\
1967	52.32\\
2000	47.5\\
2000	47.5\\
2000	47.5\\
2000	40.23\\
2000	51.1\\
2000	44.99\\
2000	41\\
2000	39.17\\
2000	32.55\\
2000	40.98\\
1888	52.98\\
1740	54.94\\
1770	55.04\\
1768	56.55\\
1740	55.52\\
1737	52.66\\
1864	51.07\\
2000	49\\
2000	49\\
2000	49\\
2000	51.4\\
2000	50\\
1679	57.98\\
1658	51.78\\
2000	41\\
2000	40.28\\
2000	48.4\\
2000	47.1\\
2000	41.5\\
2000	47.1\\
2000	45.33\\
2000	47.1\\
2000	47.1\\
2000	51.2\\
2000	55.78\\
1904	53.87\\
1897	55.1\\
1844	54.11\\
1708	51.93\\
1726	49.18\\
1910	45.35\\
1927	43.63\\
2000	41.5\\
2000	48\\
2000	78.4\\
2200	45.94\\
1745	52.29\\
1676	45.15\\
2200	38.28\\
2200	45.5\\
2200	39.94\\
2200	38.34\\
2200	36.6\\
2200	32.8\\
2200	31\\
2200	30.63\\
2200	30.67\\
2200	29.99\\
2000	38.15\\
2000	40.25\\
2000	46.3\\
2000	46.3\\
2000	45\\
2000	42\\
2000	40.2\\
2000	38.09\\
2000	38.09\\
2000	40.3\\
2000	46.3\\
2000	46.3\\
2000	54.94\\
2000	51.77\\
2000	46.3\\
2000	46.28\\
2000	44.88\\
2200	26.25\\
2200	24.38\\
2200	26.55\\
2200	26.55\\
2200	22.76\\
2200	23.08\\
2200	26.31\\
2200	27.32\\
2200	28.89\\
2200	36.6\\
2200	36\\
2200	34.94\\
2200	30.92\\
2200	30.09\\
2200	29.87\\
2200	29.64\\
2200	29.83\\
2200	34.8\\
2200	42\\
2200	52.6\\
2200	49.8\\
2200	43\\
2200	37.99\\
2200	32.02\\
1818	20.59\\
1154	18.1\\
1418	16.37\\
1803	10.05\\
1999	11.52\\
1569	15.38\\
2000	34.94\\
2000	60\\
1942	45.61\\
1852	48.65\\
1913	46.76\\
1870	49.72\\
1413	53.61\\
1713	54.81\\
1676	52.95\\
1831	49.67\\
1929	48.21\\
2000	60.6\\
2000	50\\
1722	57.98\\
2000	50\\
1910	39.99\\
1790	41.64\\
1754	39.96\\
1830	35.14\\
1900	35.14\\
1900	34.68\\
1900	31.95\\
1900	30\\
1900	30.07\\
1900	37.44\\
1584	46.44\\
1762	48.93\\
2000	46.19\\
2000	44.94\\
2000	50.12\\
2000	46.54\\
2000	49\\
2000	47.01\\
2000	44.62\\
2000	42.44\\
2000	47.71\\
2000	42.12\\
1188	49.53\\
1267	39.96\\
2000	37.1\\
1979	39.71\\
1776	33.92\\
2000	42.44\\
2000	35.06\\
2000	35.06\\
2000	31.88\\
2000	32.22\\
2000	34.47\\
2000	42.94\\
2000	50.5\\
1828	50.15\\
1714	51.78\\
2000	59\\
1818	50\\
1789	49.21\\
1647	45.84\\
1973	45.24\\
2000	45\\
2000	49.94\\
2000	54.53\\
2000	59\\
1572	66.4\\
2000	59\\
2000	50\\
1873	53.12\\
2000	51.3\\
859	48.41\\
1197	45.7\\
1444	41.49\\
1653	36.97\\
1573	36.06\\
1104	42.94\\
40	52.58\\
169	63.64\\
28	70\\
-5	68.06\\
-10	61.05\\
216	63.64\\
368	57.02\\
647	52.15\\
373	52.9\\
579	50.52\\
825	48.91\\
846	52.03\\
97	61.05\\
-16	69\\
-445	69.36\\
-454	53.57\\
343	47.75\\
651	42.48\\
1669	38.5\\
1953	32.91\\
1968	30.79\\
2000	29.17\\
2000	24.99\\
2000	29.99\\
2000	40\\
1661	53.49\\
1376	53.11\\
2000	53.11\\
2000	54.53\\
2000	59.94\\
2000	52.44\\
2000	51\\
2000	50.25\\
2000	49.52\\
2000	49.51\\
2000	50.39\\
1893	49.93\\
873	50.81\\
1569	50.5\\
1958	47.07\\
1452	49.99\\
1403	48.23\\
1834	47.02\\
1713	39.21\\
2219	33.32\\
2300	28.98\\
2300	33.15\\
2300	34.4\\
2203	23.34\\
2300	42\\
2300	46.59\\
2300	52.44\\
2300	55.78\\
2300	55.78\\
2300	57.5\\
2300	46.68\\
2300	39.94\\
2300	40\\
2300	42.44\\
2300	50.97\\
2300	57.5\\
2300	57.5\\
2300	55.78\\
2287	42.09\\
2291	45\\
2107	46.21\\
1981	44.99\\
2080	38.37\\
2300	33.7\\
2238	35.74\\
2300	28.39\\
2300	20\\
2092	18.3\\
1996	23.36\\
2271	25.56\\
2131	25.43\\
2300	37.44\\
2300	38.4\\
2300	42\\
2300	44.94\\
2300	38.3\\
2300	34.94\\
2300	34.93\\
2300	38.41\\
2300	48.29\\
2300	59.37\\
2300	59.37\\
2300	46.44\\
2300	43.71\\
2049	43.91\\
1564	43.9\\
1217	39.27\\
1610	32.75\\
2100	27.82\\
2100	22.1\\
2098	18.6\\
2087	22.93\\
1507	39.53\\
1491	44.8\\
1493	50.03\\
1656	51.86\\
1837	49.97\\
1239	48.86\\
1273	50\\
1573	49.12\\
1668	47.85\\
1944	48.76\\
2064	52.04\\
2100	60.51\\
1555	62.07\\
919	56.95\\
1553	47.65\\
2100	44.94\\
1656	45.75\\
1659	40.5\\
649	44.46\\
936	38.6\\
1163	32.74\\
1749	31.54\\
1741	25.48\\
1507	31.84\\
1186	41.58\\
896	48.66\\
338	52.6\\
348	52.6\\
532	51.69\\
476	51.7\\
418	47.15\\
437	48.69\\
507	49.24\\
693	46.32\\
1058	49.42\\
856	58.4\\
203	64.02\\
67	62.19\\
181	52.6\\
822	45\\
690	49.98\\
492	49\\
623	42.68\\
935	38.17\\
1132	28.48\\
1590	24.8\\
1892	24.36\\
1838	28.42\\
1577	47.61\\
1271	50.38\\
1355	51.78\\
1321	52.87\\
1529	51.78\\
1634	52\\
1512	52.59\\
1550	52.79\\
1611	51.7\\
1571	52.25\\
1812	47.99\\
1608	54.35\\
1066	69.96\\
693	60.97\\
795	52.44\\
1214	46.93\\
1688	46.59\\
1482	47.11\\
1123	42.28\\
1211	32.36\\
1569	32.73\\
1981	29.94\\
1889	30.19\\
1865	34.29\\
1543	47\\
1085	54.56\\
1027	58.32\\
1233	57.53\\
1185	56.97\\
1303	56.71\\
1000	54.18\\
1133	52.99\\
1159	53.45\\
1239	52.93\\
1622	52\\
1621	56.42\\
1177	64.1\\
735	54.91\\
959	49.62\\
1086	45.92\\
1162	49.97\\
1129	43.83\\
911	37.59\\
1692	30.64\\
2000	30.22\\
2000	28.83\\
2000	28.83\\
2000	30.64\\
2000	39.96\\
2000	50.03\\
2000	50.03\\
2000	49.11\\
2000	47.44\\
2000	50\\
1846	49.11\\
1969	48.38\\
2000	47.44\\
2000	45\\
2000	49.11\\
2000	58.49\\
2000	60\\
1827	46.14\\
2000	49.94\\
2000	49.94\\
2000	44.94\\
1870	36.17\\
1428	27.24\\
1330	19.67\\
1293	17.25\\
1786	15.34\\
2200	14.29\\
2200	13.52\\
1613	16.2\\
1816	17.61\\
2136	20.83\\
2200	30.97\\
2200	33.62\\
2200	33.6\\
2200	33.67\\
2200	33.56\\
2200	32.78\\
2200	34.68\\
2200	39.99\\
2200	50.22\\
2200	63.4\\
2200	51.37\\
2200	42.44\\
2200	37\\
2200	37.78\\
2200	32\\
1227	16.11\\
1318	14.91\\
1047	13.03\\
1232	10.78\\
1118	11.65\\
918	11.88\\
792	12.98\\
1280	13.55\\
1402	14.31\\
1639	15.1\\
2007	15.9\\
2132	16.9\\
2200	21\\
2200	19.96\\
2200	14.69\\
2200	21\\
2200	31.18\\
2200	42.44\\
2200	46.25\\
2200	40\\
2200	34.94\\
2200	32.76\\
2133	22.47\\
1404	19.16\\
1294	18.77\\
1080	14.5\\
812	6.33\\
1609	0.12\\
1667	2.58\\
1293	13.48\\
1452	28.97\\
1394	39.8\\
623	43.38\\
391	46.64\\
572	48.05\\
468	48.09\\
435	47.64\\
545	46.54\\
588	43.06\\
1001	43.12\\
1394	40.77\\
1828	49.23\\
534	62.74\\
311	58.03\\
158	50.46\\
-244	54.69\\
10	49.05\\
425	46.8\\
430	51.02\\
709	42.57\\
1188	40.21\\
1273	34.18\\
1402	35.38\\
1452	39.44\\
1693	51.32\\
1672	62.7\\
1084	58.79\\
1225	58.9\\
1205	55.96\\
1222	54.2\\
1289	52\\
1518	49.52\\
1738	45.51\\
1812	46.41\\
2121	48.93\\
2088	57.47\\
1552	62.19\\
1157	63.95\\
1734	57.1\\
1799	44.52\\
1797	46.96\\
2076	43\\
2115	35.81\\
2200	34.85\\
2200	33.91\\
2200	31.65\\
2200	33\\
2189	30.71\\
1858	43.36\\
1172	53.75\\
808	56.83\\
721	56.28\\
810	55.31\\
599	55.87\\
737	53.52\\
984	53.5\\
977	51.15\\
1098	51.46\\
1500	53.5\\
1247	61.3\\
1005	75.05\\
643	67.79\\
1333	52.43\\
1311	46.78\\
1064	49.41\\
764	49.77\\
1412	43.3\\
1112	40.63\\
1546	35.08\\
2130	30.21\\
1992	31.02\\
1792	38.74\\
942	46.31\\
348	55.66\\
415	60.56\\
488	60.06\\
749	60.42\\
661	60.08\\
656	55.41\\
744	53.75\\
701	53.5\\
1056	53.96\\
1213	55.98\\
1439	63.72\\
943	77.92\\
144	63.5\\
211	54.92\\
1036	45.94\\
1218	47.44\\
1087	43.3\\
1811	29.88\\
2200	27.86\\
2200	26.7\\
2200	25.91\\
2200	26.81\\
2200	26.7\\
1883	34.76\\
1259	49.04\\
1514	50.4\\
1922	47.31\\
2037	44.96\\
2140	45.33\\
2123	43.3\\
2200	42.8\\
2200	41.37\\
2200	40.49\\
2200	43.3\\
2200	61.3\\
2200	61.3\\
1898	52.44\\
1785	44.96\\
2200	39.89\\
1852	42.15\\
1507	43.7\\
1333	34.94\\
1362	30.33\\
1919	29.75\\
2184	23.15\\
2300	23.32\\
2300	23.43\\
1683	28.75\\
803	35\\
1093	39.98\\
1556	44.02\\
1508	43.99\\
1723	42.33\\
1633	44.75\\
1695	41.77\\
2300	37.7\\
2300	37.21\\
2300	39.23\\
2300	52.79\\
2152	53.6\\
1414	51.18\\
1510	44\\
1803	38.79\\
1142	41.43\\
787	41.8\\
1005	38.38\\
1337	30.95\\
2139	20.78\\
2300	21.06\\
2300	21.11\\
2300	22.39\\
2300	22.7\\
2300	22.75\\
2300	23.34\\
2300	26\\
1837	33\\
2108	34.94\\
1917	40.46\\
2300	37.44\\
2300	33.47\\
2300	37.32\\
2300	39.94\\
2300	54.94\\
2300	64.99\\
2300	55\\
1912	48.06\\
2300	41.76\\
1804	44.22\\
1412	40.85\\
1164	38.43\\
1796	28.71\\
2243	25.14\\
2300	29.1\\
2300	24.08\\
1755	29.98\\
1125	42.92\\
979	50.13\\
805	50.43\\
559	50.06\\
683	49.97\\
469	49.73\\
221	50.04\\
488	48.72\\
716	48.44\\
1223	48.62\\
1482	47.99\\
1358	53.53\\
921	60.33\\
658	53.36\\
1310	44.96\\
1951	37.89\\
1426	40.38\\
1457	34.26\\
1357	34.22\\
1895	28.81\\
2088	27.25\\
2300	25.6\\
2300	25.44\\
2300	27.98\\
2070	31.29\\
1747	41.13\\
1820	42.72\\
1970	42.87\\
2300	43.5\\
2300	44.94\\
2199	43.11\\
2300	44.5\\
2300	43.5\\
2300	44\\
2300	57.44\\
2300	60.99\\
2300	57.8\\
1988	51.91\\
2300	55\\
2300	42.36\\
1585	43.01\\
1460	35.38\\
2200	25.14\\
2200	23.88\\
2200	23.74\\
2200	23.7\\
2200	29.66\\
2200	25.85\\
2200	37.01\\
2149	46.68\\
1467	47.14\\
1496	49.77\\
1644	50.5\\
1589	50.58\\
1572	49.2\\
1695	46.98\\
1665	47.12\\
2023	49.15\\
2157	49.94\\
2200	60.7\\
1967	68.13\\
1519	57.81\\
1939	49.51\\
2200	50.8\\
2200	47.44\\
2200	43.13\\
2200	39.96\\
2200	35.13\\
2200	34.28\\
2200	33.32\\
2200	31.71\\
2200	31.71\\
2200	39.97\\
2167	54.09\\
2066	53.9\\
2184	55\\
1913	54.27\\
1992	53.8\\
2037	51.62\\
2200	49.91\\
2200	47.42\\
2200	47.01\\
2200	48.05\\
2200	58.01\\
2200	58.93\\
1565	52.97\\
1938	47.18\\
1884	37.93\\
2060	36.13\\
2200	36\\
2200	41.63\\
2200	40.22\\
2200	39.71\\
2200	32.44\\
2200	28.14\\
2200	29.2\\
2200	37.07\\
1984	47.3\\
1490	47.91\\
1546	46.21\\
1706	45.2\\
1808	44.03\\
2200	44.77\\
2200	47.44\\
2200	56.41\\
2200	58\\
2200	56.43\\
2200	71.2\\
2200	70\\
1714	56.1\\
2200	51.37\\
2200	50\\
2200	52.44\\
2200	44\\
2300	35.91\\
2300	39.99\\
2300	37.48\\
2300	35.17\\
2300	33.69\\
2300	33.5\\
2300	37.66\\
2300	39.94\\
2300	49.94\\
2300	44.99\\
2300	44.5\\
2300	50.5\\
2300	49.99\\
2300	45.2\\
2300	44.5\\
2300	43.27\\
2300	45.35\\
2300	68.9\\
2300	68.9\\
2300	57\\
2300	40.44\\
2300	42.21\\
2300	47.44\\
2300	43.91\\
2400	40.44\\
2400	41.17\\
2400	39.16\\
2400	34.99\\
2400	33.69\\
2400	33.69\\
2400	32.17\\
2400	31.31\\
2400	39.16\\
2400	43.75\\
2400	49.9\\
2400	53.49\\
2400	52\\
2400	44.94\\
2400	40.71\\
2400	41\\
2400	40.66\\
2400	59.96\\
2400	68.9\\
2400	51.78\\
2400	40.79\\
2400	41.17\\
2400	44.75\\
2400	40.83\\
2300	35.17\\
2300	34.02\\
2300	33.43\\
2300	33.8\\
2300	33.43\\
2300	33.21\\
2300	44.94\\
2101	51.32\\
1660	54.29\\
2086	53.4\\
1912	52.43\\
1949	53.79\\
2007	51.98\\
1989	54.15\\
2300	56.25\\
2300	54.97\\
2300	57.41\\
2300	70.11\\
1778	67.52\\
1326	54.99\\
2002	49.78\\
2040	42.07\\
2066	46.57\\
2200	46.2\\
2300	42.69\\
2300	44\\
2300	39.03\\
2300	36.55\\
2300	38.28\\
2300	37.17\\
2300	44.83\\
2083	53.5\\
1460	55.5\\
1406	54.99\\
1782	54.76\\
2043	54.96\\
2229	52.88\\
2300	53.03\\
2245	54.5\\
2283	54.75\\
2300	57.02\\
2300	78.61\\
1974	63.3\\
1459	58.3\\
2041	50.89\\
2300	50.8\\
2300	50.8\\
2300	49.99\\
2005	42.02\\
2300	38.43\\
2300	41.43\\
2300	40\\
2300	41.42\\
2300	41.47\\
2300	50.5\\
1767	55.19\\
1983	59.94\\
1988	60\\
2231	58.75\\
2300	57.78\\
2139	54.41\\
2300	54.32\\
2300	52.98\\
2300	50.5\\
2300	54.96\\
2300	75\\
2218	66.97\\
1596	59.45\\
1533	52.26\\
2109	47.61\\
2300	50.5\\
2300	49.99\\
2300	42.44\\
2300	41.49\\
2300	40.6\\
2300	40.39\\
2300	40.6\\
2300	41.57\\
2300	44.75\\
2300	58.69\\
2238	58.13\\
2300	56.37\\
2300	54.06\\
2300	56.15\\
2300	51\\
2300	50.45\\
2300	49.97\\
2300	50.57\\
2300	56.48\\
2300	75\\
2040	61.47\\
1569	58.34\\
1744	52.74\\
2261	46.85\\
2300	47.7\\
2300	46.4\\
2300	49.99\\
2300	41.74\\
2300	41.23\\
2300	38.63\\
2300	40.76\\
2300	41.45\\
2300	44.36\\
2300	54.94\\
2266	55.32\\
2300	56.37\\
2300	60\\
2300	64\\
2214	53.94\\
2300	56.36\\
2300	50.29\\
2300	50.38\\
2300	53.5\\
2300	65.12\\
2288	55.37\\
1980	52.85\\
2300	50.61\\
2300	49.88\\
2300	53.5\\
2300	52.32\\
2300	40.34\\
2300	35.88\\
2300	34.22\\
2300	35.13\\
2300	32.64\\
2300	31.49\\
2300	31.56\\
2300	39.96\\
2300	44.7\\
2300	46.3\\
2300	52.38\\
2300	54.94\\
2300	55.12\\
2300	44.94\\
2300	50\\
2300	49.21\\
2300	48.94\\
2300	66.4\\
2300	59.94\\
2300	47.39\\
2300	49.99\\
2300	45.23\\
2300	41.92\\
2300	40.8\\
2300	31.94\\
2300	28.91\\
2300	30.34\\
2300	31.87\\
2300	28.12\\
2300	28.74\\
2300	30\\
2300	30\\
2300	31\\
2300	36.57\\
2300	41.64\\
2300	50.21\\
2300	57.39\\
2300	46.79\\
2300	42.11\\
2300	42.84\\
2300	52.62\\
2300	69\\
2300	64.94\\
2300	57.39\\
2300	57.39\\
2300	44.5\\
2300	40.46\\
2280	37.93\\
2300	31.69\\
2300	31.71\\
2300	34.2\\
2300	34.2\\
2300	34.62\\
2300	30.34\\
2255	38.11\\
1783	51.34\\
1745	51\\
2153	49.44\\
2035	48.97\\
1690	50.32\\
1485	48.5\\
1513	49\\
1556	49.95\\
1769	50.54\\
2036	51.51\\
2158	60.16\\
1422	63.34\\
1149	57.47\\
1629	51\\
2200	48.58\\
2000	47.44\\
1885	44.88\\
2000	46.7\\
2000	41.79\\
2000	41.3\\
2000	37.7\\
2000	39.94\\
2000	35.7\\
2078	46.9\\
2130	56.27\\
1900	56.03\\
1681	55.94\\
1700	53.99\\
1648	54.33\\
1334	53.01\\
1503	54\\
1600	55\\
1811	57.29\\
2160	57.99\\
2200	69.44\\
1191	68.03\\
1096	57.72\\
1248	52.22\\
1570	47\\
1757	50.7\\
1982	49.27\\
2200	35.33\\
2200	36.09\\
2200	34.72\\
2200	34.45\\
2200	34.45\\
2200	34.72\\
2200	46.1\\
2200	59.99\\
1997	54.49\\
2200	52.5\\
2200	64.63\\
2200	78.9\\
2200	69.01\\
2200	69.06\\
2200	63\\
2200	69.06\\
2200	78.9\\
2200	84.63\\
2200	78.9\\
2042	54.35\\
1900	49.48\\
2200	78.9\\
2200	78.9\\
2200	55.72\\
1670	49.64\\
2013	43.01\\
2200	39.3\\
2200	35.7\\
2200	37.25\\
2200	42.94\\
2189	46.54\\
2200	60\\
1926	57.47\\
2018	58.93\\
1812	58\\
2062	58.1\\
1894	57.92\\
2200	55.5\\
2200	55.5\\
2200	55.5\\
2200	57.26\\
2200	80\\
2170	55.95\\
2200	56\\
2200	56\\
2200	50.06\\
2200	49.94\\
2200	44.19\\
2300	47.17\\
2300	42.49\\
2300	39.94\\
2300	35\\
2300	34.82\\
2300	37.08\\
2300	44.12\\
2200	56.33\\
2200	53\\
2200	55.5\\
2200	55.5\\
2200	55.5\\
2200	52.5\\
2200	51\\
2200	49.96\\
2200	50.06\\
2200	54.78\\
2200	72.32\\
2200	55.5\\
2200	50.06\\
2200	55\\
2200	50.06\\
2200	47.3\\
2200	44.94\\
2200	42.18\\
2200	37.01\\
2200	37\\
2200	32.84\\
2200	30.88\\
2200	30.88\\
2200	31.13\\
2200	38.25\\
2200	43.41\\
2200	46.8\\
2200	47\\
2200	48\\
2200	45\\
2200	44.71\\
2200	44.72\\
2200	44.76\\
2200	50.12\\
2400	76.4\\
2400	76.4\\
2400	52.44\\
2400	46.79\\
2400	45\\
2400	47.59\\
2400	47.49\\
2500	37.01\\
2500	34.51\\
2500	34.51\\
2500	34.24\\
2500	30.26\\
2500	30.41\\
2500	31.49\\
2500	32.14\\
};
\addplot [color=mycolor1,line width=1.0pt,mark size=0.3pt,only marks,mark=*,mark options={solid},forget plot]
  table[row sep=crcr]{%
2500	31.29\\
2500	40\\
2500	42.31\\
2500	42.41\\
2500	42.41\\
2500	42\\
2500	40.12\\
2500	34.94\\
2500	42.44\\
2500	65\\
2500	60.79\\
2500	54.06\\
2500	52\\
2500	42.62\\
2500	42.31\\
2500	32.24\\
2300	41.78\\
2300	34.77\\
2300	38\\
2300	33.74\\
2300	41.67\\
2300	39.16\\
2300	51.7\\
2293	53.43\\
2149	53\\
2176	54\\
2300	53\\
2300	56.36\\
2016	55.35\\
1821	57\\
2178	53.64\\
2300	53\\
2083	52.7\\
2151	60\\
1776	80.74\\
1338	59.71\\
1947	54.46\\
2300	54.4\\
2300	57\\
2300	57\\
2300	54.4\\
2300	51.4\\
2300	51.4\\
2300	51.4\\
2300	51.4\\
2300	51.4\\
2300	50\\
1883	57.76\\
1882	64.42\\
2002	64.83\\
2183	64.43\\
2049	63.36\\
1925	60.4\\
1832	59.07\\
1674	58.08\\
1796	56.96\\
1960	61.84\\
1555	77.52\\
965	68.19\\
1143	60.09\\
1580	54.87\\
1108	48.06\\
1261	47.52\\
1562	45.87\\
1828	49.05\\
1795	46.3\\
1764	42.36\\
2213	31.67\\
2300	31.47\\
1250	35.85\\
977	48.7\\
797	63\\
880	62.09\\
1000	64.9\\
1274	64.55\\
1456	69.8\\
1348	62.3\\
1737	64.62\\
1888	67.76\\
2300	65.66\\
2300	65.95\\
1281	82.87\\
891	72.72\\
739	69.91\\
780	58.96\\
1440	50.6\\
1487	58\\
1751	56.31\\
1809	46.04\\
1995	46.06\\
2023	49.5\\
1878	43.62\\
2197	39.57\\
1785	47.54\\
1605	51.43\\
1590	60.87\\
1662	64\\
1770	67.64\\
1739	67.7\\
2130	67.7\\
1861	65.99\\
2068	67.47\\
1816	67.87\\
2037	65.83\\
1673	72.23\\
1615	87.97\\
1403	73.59\\
1299	64.67\\
1494	57.5\\
1242	56.89\\
1860	54.96\\
2339	49.4\\
1251	57.62\\
1846	54\\
2246	47.99\\
2030	45.17\\
2015	43.36\\
2190	47.16\\
1969	50.91\\
1237	66.64\\
753	70.41\\
1070	68.99\\
1087	69.77\\
1163	66.39\\
1131	68.05\\
1183	64.39\\
976	58.67\\
1081	57.5\\
1167	59.98\\
700	70.61\\
417	64.9\\
468	63.4\\
38	58.64\\
845	48.04\\
1669	53.96\\
1959	52.44\\
1890	50.1\\
1279	46.49\\
1105	40\\
1291	39.14\\
1913	33.91\\
2049	33.99\\
1262	37.02\\
1419	47.46\\
1680	49.52\\
1905	50.43\\
2400	54.06\\
2400	59.99\\
2400	66.5\\
2336	50.03\\
1871	48.92\\
1903	48.91\\
2023	50.48\\
2400	62.17\\
2294	59.96\\
1940	53.97\\
1999	51.36\\
2395	48.69\\
2149	49.45\\
2167	48.66\\
1915	46.89\\
1459	42.43\\
1066	41.5\\
1787	30.07\\
2400	30.42\\
2264	31.33\\
2378	29.53\\
1891	33.89\\
1476	33.45\\
881	42.89\\
1256	49.36\\
1178	53.36\\
1549	57.13\\
1317	52.87\\
1584	45.5\\
1758	41.27\\
2400	45\\
2400	60\\
2400	64.5\\
2073	52.72\\
2314	53.57\\
1726	51.4\\
1560	54\\
1505	50\\
821	49.69\\
711	46.82\\
634	42\\
932	39.14\\
1006	36.16\\
822	40.78\\
1147	47.2\\
1137	57.53\\
1864	50.95\\
2207	50.82\\
2300	52\\
2300	54.06\\
2300	52.5\\
2300	51.5\\
2300	52\\
2300	51.5\\
2300	56.29\\
2300	73\\
1744	61.36\\
1709	56.25\\
1468	51.29\\
1386	45\\
1654	49.99\\
1950	49.02\\
1120	45.45\\
996	41.37\\
1642	39.53\\
1978	31.52\\
1931	32.23\\
1843	39.85\\
1003	46.77\\
782	62.52\\
1028	67.41\\
674	69.85\\
1029	63.71\\
748	61.28\\
849	56.84\\
1113	57.92\\
900	60.06\\
1055	59.74\\
1079	62.57\\
538	74.69\\
266	72.42\\
316	62.32\\
61	55.87\\
558	50.24\\
817	51\\
929	46.92\\
386	46\\
130	43.2\\
581	41.21\\
702	36.27\\
890	31.09\\
868	39.49\\
628	44.39\\
387	52.78\\
525	58.71\\
597	59.25\\
644	55.1\\
635	57.96\\
527	57.69\\
553	56.84\\
547	55.93\\
710	55.93\\
867	54.87\\
1141	59.72\\
653	61.25\\
126	59.68\\
257	53.05\\
-108	50.45\\
45	51.04\\
900	50.73\\
1030	49.5\\
574	45.34\\
911	41.06\\
1044	36.53\\
1554	33.2\\
1250	38.12\\
1183	45.17\\
817	55.16\\
1095	55.45\\
1263	52.02\\
1475	52.08\\
1504	53.01\\
1232	52.9\\
1472	52.59\\
1674	48.88\\
1856	46.96\\
2300	49.76\\
2300	59.99\\
1301	54.7\\
861	54.15\\
1144	49.45\\
908	45\\
1386	45.84\\
1415	46.62\\
1066	47.98\\
1168	41.66\\
1551	37.41\\
2209	31.22\\
2300	31\\
1648	37.93\\
1139	46.13\\
1132	54.96\\
1338	50.98\\
2002	52.46\\
2299	51.99\\
2300	59.94\\
1940	52.88\\
1910	50.22\\
1775	51.31\\
2113	48.85\\
2300	53.8\\
2300	64.94\\
2245	56.03\\
1789	53.98\\
2173	50.66\\
2235	46.06\\
2055	51.44\\
1995	48.9\\
2300	51.39\\
2300	49.04\\
2300	41.93\\
2300	40.46\\
2300	41.76\\
2300	43.5\\
2300	42.44\\
2234	47.17\\
2300	46.99\\
2300	60.32\\
2300	62.9\\
2300	62.9\\
2142	52.1\\
1928	50.96\\
1743	48.3\\
1620	45.67\\
2300	45.23\\
2300	62.9\\
2269	55.69\\
1764	51.08\\
2040	47.44\\
2201	42.5\\
1997	48.39\\
2235	49.03\\
2288	49.43\\
1962	44.37\\
1680	42.08\\
2288	38.76\\
2171	34.9\\
2188	35\\
2003	36.9\\
1517	39.99\\
1975	40.64\\
1950	44.99\\
1546	47.44\\
1317	50.44\\
1434	52.35\\
1000	50.97\\
823	52.03\\
1016	50.2\\
1309	50.42\\
2300	59.2\\
2107	57.69\\
1307	57.44\\
1205	53.39\\
1194	49.6\\
1105	47.79\\
1129	46.1\\
102	47.51\\
-75	38.65\\
133	34.38\\
668	30.42\\
674	30.34\\
-197	36.5\\
176	47.17\\
466	55.54\\
931	59.94\\
1035	61.45\\
1250	61.6\\
1310	61.66\\
1169	59.99\\
1417	58.78\\
1432	57.14\\
1607	54.13\\
1872	59.21\\
2027	74.27\\
1577	71.23\\
1299	68\\
985	57.59\\
1211	51.6\\
1453	51.93\\
1616	50.27\\
1792	47.44\\
1930	43.5\\
1986	41.37\\
2201	33.75\\
2300	33.32\\
2007	36\\
1906	42.07\\
1645	54.75\\
1478	54.87\\
1768	55.96\\
1775	55.96\\
1718	55.64\\
1452	54.09\\
1541	53.74\\
1797	52.44\\
2137	51.42\\
2293	52.62\\
2300	60.99\\
1910	59.66\\
1516	58.07\\
1987	55.3\\
2105	47.46\\
2300	55.3\\
2167	47.12\\
2300	42.72\\
2300	40.44\\
2300	35.83\\
2300	32.35\\
2300	28.51\\
2300	29.19\\
1983	39.06\\
1859	50.5\\
1657	49.5\\
2212	51.02\\
2300	49.94\\
2300	52.44\\
2279	53.91\\
2300	54.3\\
2300	54.3\\
2300	57.39\\
2300	57.13\\
2300	75\\
1911	57.61\\
1792	54.3\\
2239	48.54\\
2300	45\\
1705	51.36\\
2300	47.16\\
2300	41.07\\
2300	40.4\\
2300	33.72\\
2300	33.2\\
2300	29.36\\
2300	29.75\\
2300	39.96\\
2300	51.13\\
2281	50.65\\
2300	63.7\\
2300	61.21\\
2300	62.3\\
2152	46.46\\
2300	52.59\\
2300	42.44\\
2300	41.85\\
2300	42.5\\
2300	49.5\\
1763	50.61\\
1479	48.73\\
2300	41.9\\
2300	40\\
2300	40.23\\
2300	37.43\\
2300	32.82\\
2300	29.15\\
2300	30\\
2300	29.49\\
2300	27.43\\
2300	27.48\\
2300	33.69\\
2300	52.25\\
2300	47.39\\
2300	68\\
2300	69.29\\
2300	66.53\\
2300	64.94\\
2300	49.67\\
2300	44.97\\
2300	43.03\\
2300	47.39\\
2300	67.23\\
2300	59.96\\
2300	51.99\\
2300	52.44\\
2300	48.81\\
2300	48.52\\
2300	44.94\\
2300	32.64\\
2300	27.91\\
2300	28.99\\
2300	24.69\\
2300	19.96\\
2300	13.15\\
2300	19.68\\
2100	19.96\\
2100	38.59\\
2100	36.08\\
2100	40\\
2100	40\\
2100	40\\
2100	40\\
2100	39.42\\
2100	40.18\\
2300	53.38\\
2300	70.5\\
2300	70.5\\
2300	67.21\\
2300	48\\
2300	40.72\\
2300	42.44\\
2300	40.18\\
2300	39.39\\
2300	34.85\\
2300	36.08\\
2300	34.83\\
2300	33.99\\
2300	32.64\\
2300	30.74\\
2300	30.75\\
2300	30.16\\
2300	38.06\\
2300	43.86\\
2400	45.44\\
2400	63.18\\
2400	59.99\\
2400	40.34\\
2400	39.98\\
2400	39.49\\
2300	54.94\\
2300	63.18\\
2300	49.94\\
2300	47.44\\
2300	41.2\\
2300	39.73\\
2300	30\\
2139	16.57\\
2027	15.11\\
2247	12.99\\
2178	6.31\\
2183	3.46\\
2300	13.24\\
2300	19.22\\
2300	40.66\\
2208	44.39\\
2300	44.94\\
2300	45\\
2300	54.94\\
2300	47\\
2300	45\\
2300	54.43\\
2300	48\\
2300	55\\
2300	59.94\\
2208	45.81\\
1969	46.99\\
2300	49.94\\
2300	50\\
2300	40.28\\
2300	39.26\\
2300	28.49\\
2300	27.91\\
2300	27.28\\
2300	26.87\\
2300	27\\
2300	23.17\\
2300	32.95\\
2300	40\\
2300	45\\
2300	43.52\\
2300	47.39\\
2300	47.39\\
2300	43.16\\
2300	42.44\\
2300	41.81\\
2300	42.26\\
2300	47.39\\
2300	70\\
2300	47.39\\
2300	45\\
2300	47.39\\
2300	40.32\\
2300	44.94\\
2300	40\\
2300	29.63\\
2300	28.49\\
2300	28.49\\
2300	28.12\\
2300	16.36\\
2300	13.09\\
2300	34.6\\
2300	38.92\\
2300	44.5\\
2300	47.44\\
2300	47.44\\
2300	60\\
2300	59.99\\
2300	56.55\\
2300	49.34\\
2300	44.55\\
2300	47.39\\
2300	55\\
2300	47.39\\
2300	45\\
2300	44.94\\
2300	40.87\\
2300	44.34\\
2300	47.39\\
2300	37.18\\
2300	29\\
2300	28.49\\
2300	28.4\\
2300	28.49\\
2300	26.17\\
2300	18.85\\
2300	15\\
2300	26.03\\
2300	33.26\\
2300	37.83\\
2300	39.27\\
2300	39.69\\
2300	38.74\\
2300	38.45\\
2300	38.53\\
2300	39.99\\
2300	59.25\\
2300	50\\
2300	48.72\\
2300	50\\
2300	41.09\\
2300	42.44\\
2300	42.44\\
2300	39.27\\
2300	31.15\\
2300	29.51\\
2300	31.15\\
2300	29.54\\
2300	31.75\\
2300	31.75\\
2300	35.56\\
2300	40.29\\
2300	45\\
2300	49.4\\
2300	53.23\\
2300	49.99\\
2300	52\\
2300	45\\
2300	42.97\\
2300	44.99\\
2300	69.94\\
2111	52.71\\
1574	50.97\\
1787	44.41\\
1897	38.72\\
1466	41.76\\
1620	37.29\\
2300	32\\
2300	27.17\\
2300	22.64\\
2300	28.72\\
2300	20.14\\
2300	19.55\\
2400	29.3\\
2400	26\\
2400	34.22\\
2400	44.94\\
2400	47.97\\
2400	55.5\\
2400	58.2\\
2400	59.94\\
2400	42.44\\
2400	41.94\\
2400	44.94\\
2400	70.61\\
2400	69.94\\
2400	45.39\\
2400	44.96\\
2400	39.96\\
2400	39.48\\
2400	36.64\\
2365	29.99\\
2400	28.75\\
2384	20.73\\
2400	27.44\\
2400	29.54\\
2400	29.84\\
2400	29.87\\
2400	29.99\\
2400	27.51\\
2400	32.83\\
2400	34.09\\
2400	39.94\\
2400	40.11\\
2400	41.54\\
2400	42.23\\
2400	45.12\\
2400	49.94\\
2400	65\\
2400	66.69\\
2215	52.39\\
2400	60\\
2400	48.72\\
2337	51.85\\
2400	66\\
1819	49.95\\
1647	41.25\\
1496	42.24\\
1675	39.05\\
1602	35.05\\
1417	38.73\\
1345	40.74\\
1285	46.66\\
1733	47.99\\
1785	54.07\\
1805	53.73\\
1905	57.05\\
1803	57.37\\
2009	52.74\\
1983	50.6\\
2076	47.31\\
2212	51.88\\
2400	73.2\\
2200	60.12\\
1850	56.97\\
1932	53.8\\
1737	50.14\\
2080	50.12\\
2140	49.79\\
1198	43.36\\
1165	40.06\\
1253	35.27\\
1688	33.76\\
1758	33.71\\
1756	36.22\\
1288	40.25\\
1021	43.7\\
1324	49.04\\
1620	50.3\\
1558	51.3\\
1619	51\\
1622	50.84\\
1743	49.03\\
1190	47.06\\
1296	44.4\\
1780	47.53\\
2300	55.01\\
1517	58.15\\
1173	55.54\\
1545	51.63\\
1193	46.39\\
1559	49.73\\
1599	49.23\\
1225	48.76\\
821	43.43\\
473	41.19\\
1313	34.94\\
1821	32.32\\
1639	36.08\\
1689	41\\
1396	46.31\\
2270	46.85\\
2208	50.39\\
1492	52.35\\
1700	51.45\\
1323	51.77\\
1571	50.2\\
1476	45.75\\
1843	41.38\\
2031	41.77\\
1711	48.76\\
1098	52.92\\
585	52.16\\
726	48.01\\
608	45.39\\
923	48.43\\
1231	49.64\\
};
\addplot [color=mycolor2,solid,line width=2.0pt,forget plot]
  table[row sep=crcr]{%
-1368	51.6822958605293\\
-1268	51.3450274531889\\
-1168	51.0077590458484\\
-1068	50.6704906385079\\
-968	50.3332222311675\\
-868	49.995953823827\\
-768	49.6586854164866\\
-668	49.3214170091461\\
-568	48.9841486018057\\
-468	48.6468801944652\\
-368	48.3096117871247\\
-268	47.9723433797843\\
-168	47.6350749724438\\
-68	47.2978065651034\\
32	46.9605381577629\\
132	46.6232697504224\\
232	46.286001343082\\
332	45.9487329357415\\
432	45.6114645284011\\
532	45.2741961210606\\
632	44.9369277137201\\
732	44.5996593063797\\
832	44.2623908990392\\
932	43.9251224916988\\
1032	43.5878540843583\\
1132	43.2505856770178\\
1232	42.9133172696774\\
1332	42.5760488623369\\
1432	42.2387804549965\\
1532	41.901512047656\\
1632	41.5642436403155\\
1732	41.2269752329751\\
1832	40.8897068256346\\
1932	40.5524384182942\\
2032	40.2151700109537\\
2132	39.8779016036132\\
2232	39.5406331962728\\
2332	39.2033647889323\\
2432	38.8660963815919\\
2532	38.5288279742514\\
2632	38.1915595669109\\
2732	37.8542911595705\\
2832	37.51702275223\\
2932	37.1797543448896\\
};
\end{axis}
\end{tikzpicture}%
    \caption{Exchange/Price bewteen FR and BE}
    \label{fig:France}
\end{figure}
\end{minipage}
\begin{minipage}{0.495\textwidth} 
\begin{figure}[H]
    \centering
    \setlength\fheight{4cm}
    \setlength\fwidth{0.75\textwidth}
    % This file was created by matlab2tikz.
% Minimal pgfplots version: 1.3
%
%The latest updates can be retrieved from
%  http://www.mathworks.com/matlabcentral/fileexchange/22022-matlab2tikz
%where you can also make suggestions and rate matlab2tikz.
%
\definecolor{mycolor1}{rgb}{0.04314,0.51765,0.78039}%
\definecolor{mycolor2}{rgb}{0.84706,0.16078,0.00000}%
%
\begin{tikzpicture}

\begin{axis}[%
width=\fwidth,
height=\fheight,
at={(0\fwidth,0\fheight)},
scale only axis,
clip mode=individual,
separate axis lines,
every outer x axis line/.append style={black},
every x tick label/.append style={font=\color{black}},
xmin=-1600,
xmax=1600,
xlabel={Power Exchange},
xmajorgrids,
every outer y axis line/.append style={black},
every y tick label/.append style={font=\color{black}},
ymin=-50,
ymax=200,
ylabel={Market Price [\euro/MWh]},
xtick={-1000,0,1000},
ymajorgrids,
title style={font=\bfseries},
title={Netherlands}
]
\addplot [color=mycolor1,line width=1.0pt,mark size=0.3pt,only marks,mark=*,mark options={solid},forget plot]
  table[row sep=crcr]{%
-1401	15.15\\
-1401	12.96\\
-1401	12.09\\
-1401	11.7\\
-1401	11.66\\
-1401	11.35\\
-1401	9.85\\
-1401	9.54\\
-1401	9.49\\
-1401	11.64\\
-1401	11.94\\
-1401	13.15\\
-1401	15.24\\
-1401	13.69\\
-1401	12.43\\
-1401	9.92\\
-1401	12.12\\
-1401	15.24\\
-1401	15.73\\
-1401	17.73\\
-1401	15.63\\
-1401	13.93\\
-1401	15.1\\
-1401	12.95\\
-1401	9.62\\
-1401	7.64\\
-1283	4.96\\
-1401	0.06\\
-1401	1.05\\
-1401	7.08\\
-1401	12.5\\
-1401	21.31\\
-1401	30.44\\
-1401	35.48\\
-1401	33.06\\
-1401	33.78\\
-1401	37.97\\
-1401	37.42\\
-1401	36.24\\
-1401	32.18\\
-1401	33.56\\
-1401	52.94\\
-1292	66.7\\
-1147	53.53\\
-1401	39.54\\
-1401	35.9\\
-1016	35.47\\
-1308	30.64\\
-1351	27.4\\
-1401	25.23\\
-1401	15.63\\
-1401	8.74\\
-1401	11.28\\
-1401	12.34\\
-1401	26.02\\
-1401	31.66\\
-1401	31.96\\
-1401	31.94\\
-1401	30.96\\
-1401	31.45\\
-1401	39.05\\
-1401	30.99\\
-1401	30.46\\
-1401	30.43\\
-1401	31.12\\
-1401	36.98\\
-1401	34.97\\
-1401	42.18\\
-1401	30.39\\
-1401	26.32\\
-1105	31.82\\
-728	32.22\\
-1401	11.94\\
-1401	10.5\\
-1401	7.93\\
-1401	5.23\\
-1401	4.86\\
-1401	8.96\\
-1401	8.72\\
-1401	9.91\\
-1401	11.5\\
-1401	12.76\\
-1401	13.36\\
-1401	14.03\\
-1401	15.64\\
-1401	14.24\\
-1401	14.48\\
-1401	30\\
-1401	12.86\\
-1401	16.79\\
-1401	21.08\\
-1401	18.1\\
-1401	16.14\\
-1401	13.84\\
-1401	14.89\\
-1401	17.34\\
-1401	15.46\\
-1401	14.69\\
-1401	13.33\\
-1501	10.96\\
-1501	9.83\\
-1501	11.66\\
-1401	11\\
-1401	11.38\\
-1401	13.77\\
-1401	27.42\\
-1401	30.9\\
-1401	32.45\\
-1401	32.07\\
-1401	29.91\\
-1401	28.41\\
-1401	16.89\\
-1401	15.39\\
-1401	46.14\\
-1401	40.13\\
-1401	38.49\\
-1401	34.67\\
-1401	29.4\\
-1401	28.81\\
-1401	22.08\\
-1401	13.84\\
-1401	11.89\\
-1401	9.85\\
-1401	5.09\\
-1401	3.79\\
-1401	8.24\\
-1401	17.29\\
-1284	48.6\\
-1401	46.6\\
-1401	40\\
-1270	67.8\\
-1239	67.3\\
-1401	46.08\\
-1401	46.08\\
-1401	34.5\\
-1401	46.08\\
-1388	48.99\\
-872	84.71\\
-1401	54.57\\
-1401	50.7\\
-1401	41.75\\
-1401	28.34\\
-1401	38.52\\
-1348	33.69\\
-1401	9.26\\
-1401	9.98\\
-1401	3.88\\
-1401	0.07\\
-1401	2.05\\
-1401	6.52\\
-1401	26.07\\
-1401	34.99\\
-1401	49.42\\
-1401	52\\
-1401	52\\
-1401	54.36\\
-1334	52.78\\
-1401	49.99\\
-1401	45.93\\
-1401	43\\
-1401	46\\
-1401	54.44\\
-1285	61.93\\
-989	56.29\\
-1401	40.02\\
-1401	31.58\\
-1314	36.08\\
-1358	33.51\\
-1401	22.01\\
-1401	14.61\\
-1401	12.67\\
-1401	11.34\\
-1401	9.99\\
-1401	22.41\\
-1113	30.31\\
-259	46.42\\
-624	57.96\\
-616	57.82\\
-370	53.7\\
-574	54.51\\
-515	56.46\\
-866	54.3\\
-1024	51.11\\
-1358	47\\
-1235	52.06\\
-1167	70.01\\
-996	65\\
-1008	61.89\\
-1401	48.39\\
-1401	38.97\\
-1401	39.86\\
-1401	38.79\\
-1401	27.18\\
-1401	26.5\\
-1401	24.83\\
-1401	18.34\\
-1401	12.96\\
-1401	25.04\\
-601	34.65\\
-179	48\\
-1021	56.97\\
-1401	48.54\\
-1401	46.99\\
-1401	47.38\\
-1401	48.92\\
-1401	46.65\\
-1401	41.82\\
-1401	37.84\\
-1401	43.83\\
-1401	52.66\\
-1232	62.06\\
-897	59.78\\
-1295	47.89\\
-1401	40.66\\
-1088	45.35\\
-1093	44.29\\
-984	42.64\\
-654	38\\
-1158	32.2\\
-1401	16.27\\
-1401	15.49\\
-1093	33.77\\
-794	46.7\\
-311	62.87\\
-960	59.51\\
-1246	61.62\\
-1371	59.99\\
-1168	58.32\\
-1077	57.73\\
-1262	53.76\\
-1401	50.84\\
-1401	46.87\\
-1317	53.65\\
-1189	74.86\\
-913	66.02\\
-789	61.5\\
-1327	57.3\\
-1401	46.82\\
-306	50\\
-254	47.33\\
-166	40.36\\
-748	28.17\\
174	35.03\\
-605	28.26\\
-1268	26.76\\
-857	28.27\\
-40	34.12\\
-396	38.34\\
-994	43.65\\
-1157	45\\
-1401	46.36\\
-1401	46.34\\
-1077	49.72\\
-1401	41.89\\
-1401	37.94\\
-1401	38.34\\
-1401	40.28\\
-1401	51.79\\
-1401	59.12\\
-1401	58.65\\
-1401	50.94\\
-1401	44.8\\
-858	52.81\\
-691	48.69\\
-1129	43.47\\
-798	40\\
-1401	25.86\\
-1401	16.97\\
-1401	14.51\\
-1401	16.47\\
-1401	16.18\\
-1401	23.1\\
-1401	31.5\\
-1232	44.33\\
-1401	25.75\\
-1401	23.98\\
-1401	29.73\\
-1401	24.67\\
-1401	25.7\\
-1401	27.07\\
-1401	31.11\\
-1401	39.66\\
-1401	41.96\\
-1359	46.09\\
-1401	37.35\\
-1501	32.85\\
-799	37.62\\
-344	32.17\\
-981	26\\
-1424	27.86\\
-1501	27.6\\
-1501	13.45\\
-1501	14.24\\
-1215	26.71\\
-403	33.11\\
292	50.84\\
-464	51.45\\
-664	53.89\\
-573	64.36\\
-582	65.14\\
-709	55.89\\
-809	55\\
-849	54.94\\
-768	54.97\\
-374	57.93\\
-833	67.48\\
-475	64.41\\
-152	60.71\\
-255	51.62\\
-1228	44.67\\
-869	43.21\\
-523	43.21\\
-432	31.14\\
-865	29.65\\
-633	29.56\\
-1501	24.17\\
-1501	19.72\\
-796	30.34\\
-179	36.98\\
352	60.09\\
309	65.94\\
377	66.06\\
154	66.95\\
58	67.51\\
135	60.02\\
126	55.01\\
187	52.54\\
-34	51.86\\
-266	55.88\\
79	70.68\\
704	68\\
892	53.7\\
982	54.94\\
-713	45.14\\
-315	44.94\\
-275	39.99\\
-567	33.03\\
-617	30.66\\
-883	29.92\\
-1393	29.63\\
-1393	29.54\\
-799	30.65\\
119	39.94\\
-428	55.67\\
-645	62.65\\
-780	58.07\\
-736	55.47\\
-691	53.9\\
-451	54.16\\
-402	54.35\\
-505	50.75\\
-950	50.51\\
-1209	52.1\\
-1184	56.49\\
-221	62.96\\
314	61.29\\
-524	47\\
-1401	38.6\\
-967	39.46\\
-509	35.75\\
-985	30.06\\
-1320	28.92\\
-1401	25.6\\
-1401	21.42\\
-1401	18.42\\
-1111	28.66\\
-448	33.35\\
-106	50.09\\
-806	48.34\\
-1032	51.57\\
-1101	50.84\\
-1050	52.2\\
-1206	51.52\\
-1152	55\\
-1200	54.94\\
-1029	53.62\\
-963	55\\
-788	79.94\\
-1401	55.55\\
-1368	55\\
-1401	46.02\\
-1401	35.23\\
-1400	37.69\\
-776	38\\
-1194	29.53\\
-1401	28.17\\
-1401	19.58\\
-1401	11.23\\
-1401	11.18\\
-1401	20.74\\
-648	34.03\\
-917	48.97\\
-1297	49.96\\
-1501	47.71\\
-1501	47.91\\
-1501	47.96\\
-1501	48.68\\
-1501	46.17\\
-1501	42.32\\
-1488	47.44\\
-1335	48.87\\
-1339	68.99\\
-1441	52.75\\
-1005	49.43\\
-1501	43.74\\
-1501	33.33\\
-1243	34.45\\
-767	35.9\\
-967	30.5\\
-1310	29.79\\
-1501	28.58\\
-1501	26.79\\
-1501	18.63\\
-1501	22.25\\
-1501	22\\
-1469	30.25\\
-1496	33.34\\
-1501	39.96\\
-1501	35.17\\
-1501	32.44\\
-1401	31.28\\
-1401	30.89\\
-1401	30.68\\
-1401	30.88\\
-1401	33.17\\
-1401	54.53\\
-1401	45.05\\
-1401	39.72\\
-1401	30.21\\
-1401	28.63\\
-1401	30.4\\
-641	38.63\\
-575	36.1\\
-1018	30.14\\
-1401	26.15\\
-1401	9.48\\
-1401	12.8\\
-1401	16.41\\
-1401	12.18\\
-1401	15.76\\
-1401	29.99\\
-1156	37.57\\
-1401	44.38\\
-1401	46.47\\
-1379	49.96\\
-1401	44.12\\
-1401	35.64\\
-1401	27.42\\
-1401	29.67\\
-1313	60.71\\
-1401	60\\
-1401	46.54\\
-1401	42.51\\
-1401	40.02\\
-1401	41.86\\
-951	38\\
-549	34.75\\
-1185	29.46\\
-1259	29.21\\
-1401	14.99\\
-1367	29.33\\
-1401	25.5\\
-182	45.93\\
410	71.97\\
669	70.92\\
597	64.5\\
939	62.93\\
765	64.44\\
777	59.61\\
648	58.04\\
431	54.16\\
201	49.7\\
86	52.53\\
197	64.91\\
829	77.99\\
1024	63.97\\
346	62.7\\
-1427	50.4\\
-514	58.17\\
-749	49.38\\
-435	44.79\\
-480	34.35\\
-902	30.65\\
-991	30.3\\
-1041	30.65\\
-357	33.68\\
139	45.86\\
502	59.91\\
187	62.98\\
68	60.19\\
52	63.32\\
81	65.65\\
88	62.91\\
-53	61.46\\
203	57.41\\
-119	50.81\\
-212	54.09\\
-17	60.23\\
773	60.57\\
992	62.97\\
894	52.97\\
6	48.03\\
55	51.45\\
-297	47.44\\
-207	45.67\\
-10	43.09\\
-243	34.97\\
-514	31.02\\
-1014	31.14\\
-102	35.41\\
576	48.5\\
788	67.49\\
345	66.49\\
603	65.66\\
454	60.88\\
601	62.79\\
509	61.44\\
485	59.37\\
659	58.96\\
742	54.76\\
497	61.62\\
604	71.58\\
1081	80.06\\
1345	64.04\\
1501	62\\
1061	53.5\\
1501	61.58\\
1453	53.5\\
-248	41.81\\
114	41.5\\
-32	39.22\\
-911	31.32\\
-1050	31.17\\
-156	34.03\\
340	43.13\\
1258	60.59\\
1305	65.1\\
1320	65.05\\
1187	64.92\\
1087	62.17\\
951	60.92\\
869	60.91\\
617	54.77\\
264	49.69\\
75	48.08\\
155	55.17\\
810	63.96\\
1102	57.53\\
-122	53.78\\
-1401	45.45\\
-803	53.8\\
-729	46.91\\
-650	35.05\\
-669	32.29\\
-1363	30.34\\
-1401	29.99\\
-1401	30.53\\
-1075	32.17\\
110	43.42\\
1058	61.6\\
617	62.5\\
393	63.3\\
318	61.59\\
303	63.28\\
319	60.39\\
152	59.33\\
644	58.97\\
450	53.19\\
253	53.03\\
-62	64.41\\
504	64.94\\
574	63.71\\
207	54.64\\
-932	50.15\\
-1287	47.09\\
-938	49\\
-1433	37.68\\
-1047	32.05\\
-1497	31.1\\
-1501	30.32\\
-1501	29.83\\
-1501	29.04\\
-1401	30.9\\
-1401	31.58\\
-1401	39.27\\
-1401	42.04\\
-1401	43.63\\
-1401	44.11\\
-1401	45.82\\
-1401	37.68\\
-1401	32.69\\
-1401	32.69\\
-1401	33.57\\
-1401	38.13\\
-1401	48.04\\
-1401	43.14\\
-1401	32.57\\
-1401	30.05\\
-1401	32.86\\
-606	38.23\\
-1004	22.11\\
-1374	19.14\\
-1401	14.85\\
-1401	13.14\\
-1401	13.22\\
-1401	12.73\\
-1401	9.53\\
-1401	14.05\\
-1401	14.04\\
-1501	22.19\\
-1501	22.69\\
-1501	22.85\\
-1501	23.42\\
-1501	19.08\\
-1501	15.23\\
-1401	12.89\\
-1401	8.26\\
-1401	17.66\\
-1401	22.42\\
-1401	26.7\\
-1401	31.96\\
-1401	23.75\\
-1401	28.24\\
-1401	20.29\\
-1401	12.63\\
-1401	11.74\\
-1401	12.81\\
-1401	7.1\\
-1401	8.77\\
-1401	13.09\\
-1172	30.14\\
-649	47.43\\
-1069	51.07\\
-1178	52.47\\
-1075	53.47\\
-1110	52.73\\
-834	56.05\\
-1074	50.71\\
-1006	51.92\\
-913	51.92\\
-712	51.47\\
-1061	55.53\\
20	79.92\\
8	64.25\\
-671	47.25\\
-1100	38.71\\
-265	40.2\\
60	38.51\\
-452	36.41\\
-640	32.92\\
-666	30.28\\
-1046	29.57\\
-1190	30.16\\
-548	31.21\\
174	41.47\\
668	55.05\\
-1	56.58\\
-239	54.52\\
-292	52\\
-259	52.44\\
-162	49.94\\
-348	47.44\\
-356	46.5\\
-548	48\\
-631	49.77\\
-669	57.75\\
-93	70\\
-109	62.09\\
-226	56.94\\
-1301	47.4\\
-204	53.42\\
209	48.28\\
-313	38.65\\
-442	30.54\\
-667	29.63\\
-1193	28.98\\
-832	29.55\\
-345	31.08\\
754	37.1\\
1401	55.38\\
1375	52.95\\
1401	52.9\\
938	57.44\\
987	56.78\\
905	53.55\\
1258	54\\
1200	50.01\\
881	47.44\\
752	47.02\\
1164	61.52\\
1401	83.27\\
1401	72.99\\
1336	54.43\\
1117	49.5\\
1401	54.11\\
1401	52.05\\
840	37.75\\
613	34.94\\
625	34.29\\
-102	29.92\\
-318	30.32\\
429	34.43\\
790	45.2\\
979	53.07\\
1401	54.73\\
1299	64.78\\
1115	59.99\\
1179	64.16\\
1088	59.72\\
865	54.48\\
792	51.89\\
1018	51.5\\
817	51.56\\
754	55.12\\
1410	67.98\\
1451	59.67\\
1272	50.5\\
371	45.81\\
661	46.52\\
1342	47.1\\
770	42.96\\
932	42.5\\
827	36.54\\
663	33.06\\
481	31.95\\
852	40.03\\
508	46.9\\
1143	69.94\\
1081	71.95\\
448	72.19\\
284	65.88\\
215	60.93\\
176	52.93\\
158	48.14\\
-546	45\\
-565	44.1\\
-901	44.23\\
-470	52.92\\
683	61.76\\
691	56.35\\
-141	49.08\\
-478	40\\
323	41.99\\
739	39.17\\
-428	31.04\\
-1066	28.94\\
-1401	26.63\\
-1401	20.87\\
-1401	20.18\\
-1401	20.71\\
-1401	26.82\\
-1401	28.7\\
-1401	37.27\\
-1401	42.65\\
-1401	43.28\\
-1401	44.55\\
-1310	48.59\\
-1310	40.24\\
-1310	32.87\\
-1401	32.8\\
-1401	33.28\\
-1401	43.56\\
-1179	50.02\\
-1382	44.94\\
-1401	41.24\\
-1238	35.81\\
-844	35.01\\
-328	35.96\\
393	41.62\\
-32	31.96\\
-735	25.8\\
-1316	19.38\\
-1265	13.18\\
-1373	14.07\\
-1359	16.54\\
-1501	18.85\\
-1501	22.07\\
-1188	30.83\\
-1214	31.96\\
-1401	31.82\\
-1404	34.15\\
-1501	32.14\\
-1501	30.8\\
-1501	31.03\\
-1501	32.95\\
-1501	36.15\\
-1501	48.68\\
-1501	54.57\\
-1261	53.78\\
-1165	45.76\\
-488	49.81\\
290	48.84\\
-227	44.37\\
-180	37.8\\
-263	36\\
-681	31.56\\
-541	31.52\\
281	36.19\\
739	54.1\\
979	71.89\\
890	72.94\\
521	67.5\\
218	57.18\\
262	54.63\\
378	52.9\\
416	51.96\\
261	46.82\\
194	47.14\\
129	46.76\\
107	54.09\\
798	67.8\\
813	59.96\\
-227	57.98\\
-1341	48.72\\
-628	52.2\\
-59	52.27\\
-399	44.42\\
333	44.96\\
92	40.9\\
-514	30.63\\
-461	30\\
-254	32.38\\
479	47.13\\
209	60.3\\
-24	58.49\\
-356	53.88\\
-592	51.99\\
-478	52.83\\
-187	54.68\\
-411	51.68\\
-516	48.05\\
-544	47.59\\
-647	48.03\\
-668	53.05\\
-12	63.99\\
47	55.73\\
-356	48\\
-1501	41.26\\
-996	44.08\\
-541	42.46\\
-464	30.53\\
-614	29.98\\
-711	28.99\\
-1401	25.71\\
-1401	18.11\\
-936	28.08\\
-579	33.51\\
-1358	45.16\\
-1341	49.94\\
-1252	52.61\\
-1164	58.57\\
-1216	54.16\\
-1255	48.08\\
-1087	52.2\\
-1020	50.09\\
-1401	43.92\\
-1401	42.55\\
-1344	49.69\\
-725	67.38\\
-600	53.32\\
-1401	36.79\\
416	57.1\\
990	52\\
1036	45\\
1401	42.28\\
1314	39.94\\
808	34.79\\
-135	28.27\\
-115	28.67\\
896	32.2\\
1401	44\\
1401	55.87\\
1401	56.7\\
1401	58.84\\
1401	61.77\\
1401	60.43\\
1401	57.28\\
1401	55.91\\
1001	55.29\\
927	49.94\\
415	47.89\\
245	55.89\\
658	66.01\\
1401	60.02\\
492	48.05\\
-763	37.12\\
-475	33.71\\
-67	39.61\\
-1401	23.64\\
-1401	19.51\\
-1401	14.66\\
-1401	12.18\\
-1401	8.54\\
-1401	15.42\\
-1401	30.68\\
-1014	47.79\\
-1206	45.71\\
-1401	47.63\\
-1401	47.94\\
-1401	51.34\\
-1401	48.37\\
-1401	42.94\\
-1401	36.86\\
-1401	32.31\\
-1401	31.3\\
-1401	36.16\\
-1401	59.92\\
-1401	54.71\\
-1401	43.31\\
-1401	30.49\\
-1401	38.37\\
-1401	40.84\\
-1401	28\\
-1401	26.43\\
-1401	25.9\\
-1401	24.93\\
-1401	10.52\\
-1401	16.2\\
-1401	24.04\\
-1401	27.13\\
-1401	29.51\\
-1401	29.48\\
-1401	28.07\\
-1401	26.76\\
-1401	28.03\\
-1401	25.5\\
-1401	25.96\\
-1401	28.81\\
-1401	28.03\\
-1401	31.17\\
-1401	42.58\\
-1401	33.88\\
-1401	27.57\\
-1401	24.29\\
-1401	27.55\\
-1401	23.6\\
-1401	14.23\\
-1401	13.02\\
-1401	12.87\\
-1322	11.77\\
-1275	9.72\\
-1296	9.36\\
-1401	10.63\\
-1401	10.54\\
-1401	11.04\\
-1401	19.16\\
-1401	17.7\\
-1401	17.29\\
-1401	14.68\\
-1401	13.16\\
-1401	10.5\\
-1401	8.97\\
-1401	12.42\\
-1401	14.69\\
-1401	35.79\\
-1401	36.81\\
-1401	31.23\\
-1401	27.97\\
-1229	30.04\\
-1401	25.31\\
33	36.13\\
-297	33.87\\
-1135	27.22\\
-1401	23.86\\
-1154	23.08\\
-15	33.18\\
1139	45.43\\
1401	59.42\\
1401	60.99\\
1401	58.79\\
1393	60.02\\
1151	54.96\\
1255	56.12\\
1166	56.82\\
1260	55.63\\
1084	51.47\\
743	47.44\\
923	54.94\\
1401	79.49\\
1401	54.32\\
1401	50.06\\
359	46.57\\
664	47.59\\
245	42.83\\
-557	30.75\\
-218	32.07\\
-264	29.45\\
-904	28.27\\
-1256	28.05\\
-626	30.72\\
119	41.12\\
-303	53.8\\
-319	55.34\\
-556	56.62\\
-788	54.1\\
-585	51.74\\
-443	50.38\\
-415	49.25\\
-863	42.49\\
-1046	39.56\\
-775	41\\
-49	51.9\\
46	58.97\\
296	67.49\\
-39	50\\
-760	46.48\\
275	49.48\\
225	44.09\\
326	45\\
137	41.94\\
32	37.84\\
-264	28.43\\
-300	29.36\\
104	36.72\\
973	45.54\\
946	59.3\\
331	65.77\\
-29	68.42\\
-381	60.87\\
-184	55.9\\
-142	50.33\\
-254	49.42\\
-571	45.92\\
-781	43.34\\
-1046	40.93\\
-571	44.99\\
82	58.94\\
428	54.37\\
-375	52.03\\
-1374	42.78\\
-145	46.13\\
194	44.03\\
-83	42.97\\
-179	34.54\\
-586	32.23\\
-1401	18.33\\
-1401	12.47\\
-1401	26.67\\
-537	42.44\\
-182	55.88\\
-91	53.54\\
-141	57.79\\
-374	56.72\\
-550	57.98\\
-362	55\\
-529	52.89\\
-593	52.35\\
-960	45.25\\
-1376	42.91\\
-893	48.66\\
-34	68.92\\
-19	58.69\\
-1353	52.81\\
-1401	42\\
-1045	40.63\\
-332	38.04\\
-876	33\\
-798	32.72\\
-777	29.33\\
-1401	24.42\\
-1401	25.63\\
-1401	29.44\\
-94	46.4\\
-402	61.12\\
-391	64.8\\
-746	69.25\\
-1168	62.12\\
-1132	55\\
-1096	52.08\\
-1373	47.1\\
-1401	44.22\\
-1401	36.2\\
-1401	40\\
-1401	46.18\\
-961	53.48\\
-880	48.73\\
-1401	36.97\\
-1401	30.39\\
-1401	30.61\\
-1246	42.44\\
-1401	27.73\\
-1401	17.99\\
-1401	12.38\\
-1401	10.36\\
-1401	10.11\\
-1401	10.33\\
-1401	11.36\\
-1401	13.34\\
-1401	18.26\\
-1401	22.66\\
-1401	23.11\\
-1401	23.08\\
-1401	23.76\\
-1401	22.48\\
-1401	21.11\\
-1401	19.33\\
-1401	20.46\\
-1401	27.73\\
-1401	39.29\\
-1401	44.85\\
-1401	28.93\\
-1401	29.36\\
-1401	41.5\\
-1075	41.56\\
-1401	25.49\\
-1401	14.46\\
-1401	12.35\\
-1401	9.06\\
-1401	5.3\\
-1401	8.08\\
-1401	11.12\\
-1401	9.53\\
-1401	10.73\\
-1401	13.37\\
-1401	16.81\\
-1401	17.86\\
-1401	18.23\\
-1401	13.15\\
-1401	11.38\\
-1401	10.48\\
-1401	12.22\\
-1401	15.26\\
-1401	40.15\\
-1401	47.98\\
-1401	46.83\\
-1401	34.73\\
-1401	39.08\\
-1401	31.53\\
-1401	31.59\\
-1401	25.69\\
-1401	23.33\\
-1401	17.52\\
-1401	19.63\\
-1401	26.47\\
-689	43.79\\
-77	58.75\\
-596	53.31\\
-1158	54\\
-1227	49.32\\
-930	50\\
-1211	44.22\\
-1501	42.07\\
-1501	34.96\\
-1501	37.25\\
-1501	40.33\\
-1501	43.03\\
-298	82.04\\
-239	71.74\\
230	56.51\\
-660	47.19\\
98	49.94\\
-530	43.43\\
-823	31.95\\
-741	31.66\\
-736	30.09\\
-802	28.86\\
-859	29.2\\
-737	31.3\\
-92	42.03\\
294	56.19\\
514	54.9\\
558	53.79\\
676	51.9\\
755	50.74\\
847	51.42\\
654	48.48\\
16	43.27\\
-107	43.44\\
-183	45.72\\
-173	53.91\\
296	73.68\\
525	59.98\\
-181	45.94\\
-1501	39.68\\
-975	41.9\\
-781	36.47\\
-864	32.16\\
-658	29.65\\
-1002	28.46\\
-1402	27.67\\
-1501	28.51\\
-997	29.34\\
-256	38.19\\
-76	47.45\\
-488	49.44\\
-793	52.21\\
-716	53.93\\
-616	53.96\\
-483	50.76\\
-590	49.48\\
-583	49.37\\
-586	47.95\\
-737	44.71\\
-356	47.44\\
582	70\\
651	59.54\\
454	54.27\\
-847	47.02\\
-1144	44.18\\
-514	44.96\\
-574	31.28\\
-408	29.47\\
-1066	28.87\\
-1160	27.75\\
-1501	25.59\\
-1074	27.85\\
-222	34.51\\
-479	46.28\\
-239	48.34\\
-553	48.28\\
-1172	44.93\\
-864	46.95\\
-685	47.58\\
-1349	44.94\\
-1401	40.02\\
-1401	34.96\\
-1401	32.45\\
-1401	41.78\\
-1090	55\\
-1073	52.17\\
-1238	47.44\\
-1401	39.95\\
-1401	41.78\\
-1189	41.49\\
-1389	32.91\\
-1401	22.15\\
-1401	20.84\\
-1401	19.42\\
-1401	21.02\\
-1401	26.64\\
-1401	35.83\\
-1344	46.31\\
-1267	49.03\\
-1401	52.95\\
-1401	50.07\\
-1401	48.45\\
-1401	42.88\\
-1401	37.61\\
-1401	32.16\\
-1401	32.1\\
-1401	31.68\\
-1401	42.9\\
-803	49.44\\
-424	49.96\\
-1317	49.08\\
-1401	40.71\\
-837	40.93\\
-300	41\\
-71	35.11\\
-302	29.96\\
-835	26.09\\
-1401	21.02\\
-1401	18.67\\
-1401	21.15\\
-939	27.93\\
-1045	29.38\\
-1401	32.92\\
-1401	37.43\\
-1401	37.35\\
-1401	36.86\\
-1401	38.85\\
-1401	35.47\\
-1401	29.59\\
-1401	31.99\\
-1401	31.33\\
-1401	37.71\\
-1401	50.91\\
-1339	53.1\\
-1401	44.22\\
-1501	36.72\\
-1115	40.34\\
-653	36.96\\
-1401	26.11\\
-1401	21.63\\
-1401	22.07\\
-1401	19.48\\
-1401	16.28\\
-1401	19.86\\
-1304	15.73\\
-1401	15.73\\
-1401	17.1\\
-1401	16.63\\
-1401	17.46\\
-1401	16.77\\
-1401	14.68\\
-1401	12.98\\
-1401	11.26\\
-1401	7.45\\
-1401	10.84\\
-1401	11.28\\
-1401	24.31\\
-1401	35.49\\
-1401	27.81\\
-1401	23.11\\
-1401	27.1\\
-1241	25.71\\
-1401	14.5\\
-1401	15.09\\
-1401	17.22\\
-1401	13.68\\
-1401	14\\
-1401	18.12\\
-778	35.66\\
-920	46.95\\
-949	50\\
-1401	46.65\\
-1401	45.98\\
-1401	41.79\\
-1401	40.46\\
-1401	39.32\\
-1401	35.99\\
-1401	31.07\\
-1401	32.97\\
-1401	46.16\\
-1378	64.95\\
-1341	64.73\\
-1401	48.99\\
-1401	35.43\\
-1080	37.44\\
-646	33.87\\
-589	26.22\\
-1040	23.87\\
-1185	22.77\\
-1501	20.29\\
-1501	18.97\\
-719	25.94\\
-79	36.97\\
49	46.95\\
-691	46.66\\
-1501	46.12\\
-1501	44.59\\
-1501	44\\
-1421	44.94\\
-1501	40.49\\
-1501	36.09\\
-1501	35.3\\
-1501	33.71\\
-1501	42.61\\
-916	72.68\\
-1063	62.52\\
-1184	46.87\\
-1501	39.88\\
-733	42.89\\
-319	39.59\\
-610	36.05\\
-730	30\\
-904	27.29\\
-784	26.91\\
-731	27.86\\
-727	28.48\\
-413	39.22\\
-260	45.05\\
-777	49.24\\
-1135	51.3\\
-1452	50.42\\
-1269	50\\
-1265	46.91\\
-1242	46.42\\
-1320	44.96\\
-1164	43.35\\
-1401	41.81\\
-972	43.99\\
-250	71.19\\
195	72.95\\
44	49.85\\
-304	45.59\\
-410	42.44\\
-163	40.05\\
-348	33.99\\
-327	32.03\\
-356	30.28\\
-387	29.61\\
-511	29.48\\
-133	32.18\\
-615	41.54\\
-563	53.34\\
-869	50.1\\
-1469	47.1\\
-1501	48.14\\
-1148	46\\
-957	44.24\\
-920	44.81\\
-987	43.22\\
-1249	41.94\\
-1392	40.62\\
-770	51.76\\
-637	64.96\\
-1115	57.83\\
-1324	48.66\\
-1249	43.21\\
-961	41.45\\
-934	40.16\\
-759	36.25\\
-574	34.94\\
-727	30.19\\
-819	28.03\\
-718	28.54\\
-750	29.94\\
-832	39.6\\
-908	47.45\\
-1501	49.45\\
-1501	49.77\\
-1501	49.12\\
-1501	47.42\\
-1501	46.72\\
-1501	44.7\\
-1501	42.31\\
-1501	41.92\\
-1501	42.98\\
-1501	47.3\\
-1062	64.68\\
-828	59.92\\
-1305	47.28\\
-1401	39.75\\
-773	42.95\\
-423	42\\
-700	40.61\\
-585	35.72\\
-507	33.79\\
-661	30.22\\
-869	29.28\\
-873	29.33\\
-1148	28.98\\
-1347	33.71\\
-1480	39.51\\
-1315	47.4\\
-1420	50.34\\
-1410	48.18\\
-1230	47.97\\
-1437	43.25\\
-1501	38.63\\
-1501	36.46\\
-1501	33.77\\
-1347	41.27\\
-591	63.21\\
-300	74.76\\
-950	45.93\\
-822	42.66\\
190	48.38\\
25	44.97\\
244	44.94\\
-137	39.4\\
-237	32.28\\
-379	29.31\\
-494	27.57\\
-507	27.86\\
-774	27.33\\
-1149	27.04\\
-773	27.49\\
-1106	32.39\\
-1079	33.69\\
-1067	33.67\\
-1064	34.38\\
-1501	27.45\\
-1501	23.55\\
-1401	24.59\\
-1401	25.42\\
-1401	28.22\\
-1401	33.96\\
-1164	42.62\\
-1400	39.44\\
-1208	31.89\\
-424	36.16\\
184	32.59\\
-564	26.52\\
-636	25.11\\
-1014	24.08\\
-1384	22.14\\
-1355	24.03\\
-580	27.01\\
411	43.93\\
-23	49.94\\
-222	51.33\\
-509	53.68\\
-648	51.43\\
-498	57.13\\
-625	47.83\\
-584	49.41\\
-625	46.48\\
-1128	38.95\\
-891	40.98\\
-710	44.96\\
92	72.91\\
290	80.69\\
-36	47.81\\
-251	43.99\\
222	45.3\\
-35	43.34\\
765	45.03\\
454	39.19\\
165	33.29\\
-242	28.99\\
-279	29.68\\
47	33.28\\
880	47.18\\
434	53.9\\
664	57.9\\
666	57.32\\
735	51.02\\
815	51.98\\
1128	52.3\\
987	50\\
876	48.59\\
601	46.51\\
389	45.92\\
288	45.87\\
924	75.64\\
1023	79.38\\
684	54.77\\
119	50.68\\
872	49.01\\
1128	50.54\\
381	40.21\\
540	39.94\\
809	42.1\\
530	34.72\\
333	33.37\\
508	41.28\\
420	47.32\\
-176	62.16\\
-120	67.76\\
-685	68.08\\
-704	53.01\\
-845	44.96\\
-958	43.85\\
-739	44.19\\
-1501	40.21\\
-1501	38.02\\
-1501	38.08\\
-738	46.54\\
-394	58.03\\
-242	68.37\\
-821	49.04\\
-1335	42.58\\
-553	45.99\\
-305	44.1\\
-476	37.43\\
-528	36.72\\
-412	34.14\\
-654	28.81\\
-607	29.19\\
-402	33.98\\
-137	42.85\\
180	55.33\\
300	56.97\\
-395	56.22\\
-692	50\\
-652	48.58\\
-508	46.41\\
-807	44.92\\
-1447	40.51\\
-1501	34.11\\
-1501	37\\
-741	45\\
106	61.59\\
1084	69.75\\
366	53.9\\
-687	42.17\\
317	45.73\\
483	40.97\\
720	40.01\\
273	33.98\\
6	32.22\\
-256	28.9\\
-176	29.1\\
219	32.06\\
-52	40.9\\
-414	48.26\\
-664	53.76\\
-881	62\\
-1310	49.17\\
-1215	45.36\\
-1273	44.82\\
-1501	37.93\\
-1501	35.64\\
-1501	31.97\\
-1501	31.94\\
-1501	39.36\\
-1225	47.09\\
-1249	49.26\\
-1467	45.95\\
-1501	41.92\\
-825	44.28\\
-538	43.68\\
-842	39.97\\
-1016	33.31\\
-1091	30.44\\
-1043	25.71\\
-959	24.96\\
-941	25.82\\
-1056	26.92\\
-1225	29.6\\
-1401	31.96\\
-1401	31.51\\
-1401	28.19\\
-1401	26.17\\
-1401	24.42\\
-1401	20.02\\
-1401	15.91\\
-1401	12.03\\
-1401	16.02\\
-1401	22.55\\
-1401	35.58\\
-1401	42.77\\
-1401	30.03\\
-1401	25.14\\
-1016	25.47\\
-595	23.65\\
-994	15.67\\
-1401	13.75\\
-1401	12.1\\
-1401	11.4\\
-1401	10.35\\
-1401	11.4\\
-1401	11.49\\
-1401	11.4\\
-1401	11.55\\
-1401	12.05\\
-1401	11.54\\
-1401	12.11\\
-1401	12.26\\
-1401	10.29\\
-1401	6.2\\
-1401	1.75\\
-1401	1.87\\
-1401	9.83\\
-1401	22.41\\
-1401	41.73\\
-1401	34.86\\
-1401	28.07\\
-1367	30.09\\
-665	26.92\\
-871	24.26\\
-1293	22.62\\
-1350	22.73\\
-1401	22.76\\
-1401	22.11\\
-1076	25.57\\
-586	43.5\\
-993	48.28\\
-1312	48.12\\
-1401	44.76\\
-1401	40.9\\
-1401	37.42\\
-1401	36.46\\
-1401	35.85\\
-1401	33.8\\
-1401	30.13\\
-1401	30.19\\
-1401	36.36\\
-861	56.07\\
-620	75.46\\
-756	46.32\\
-1401	38.68\\
-1083	35.17\\
-716	30.93\\
-679	25.83\\
-598	25.9\\
-790	24.63\\
-1399	24.27\\
-1401	23.78\\
-1041	26.5\\
-740	35.08\\
-1346	42.55\\
-1401	44.58\\
-1401	40.03\\
-1401	40.2\\
-1401	36.07\\
-1401	31.88\\
-1401	30.65\\
-1401	28.18\\
-1401	28.72\\
-1401	30.91\\
-1401	36.26\\
-1360	49.94\\
-1401	54.1\\
-1401	42.56\\
-1401	34.04\\
-1024	40.03\\
-993	32.99\\
-1217	28.87\\
-1025	28.04\\
-1057	26.65\\
-1365	25.15\\
-1200	25.69\\
-1045	28.09\\
-1401	35.8\\
-1401	44.54\\
-1401	43.73\\
-1401	41.27\\
-1401	32.59\\
-1401	33.67\\
-1401	30\\
-1401	28.78\\
-1401	28.42\\
-1401	29.25\\
-1401	30.79\\
-1401	37.04\\
-1401	49\\
-1359	75.98\\
-1401	48.43\\
-1401	42.01\\
-1401	38.28\\
-1401	32.08\\
-1401	31.35\\
-1401	30.65\\
-1387	29.96\\
-1318	30\\
-1172	30.11\\
-1185	31.43\\
-1401	43.03\\
-988	53.46\\
-1401	49.84\\
-1401	41.27\\
-1401	34.74\\
-1401	29.77\\
-1401	30.58\\
-1401	28.42\\
-1401	28.78\\
-1401	30.95\\
-1401	35.54\\
-1401	30.84\\
-1263	52.76\\
-1024	75.55\\
-982	48.69\\
-1401	43.63\\
-1401	42.87\\
-1401	40.69\\
-1401	34.96\\
-1313	32.01\\
-1397	30.99\\
-1401	29.57\\
-1401	29.65\\
-1299	32.89\\
-1401	42.04\\
-805	49.94\\
-1401	47.93\\
-1401	42.94\\
-1401	41.26\\
-1401	38.83\\
-1401	37.01\\
-1401	34.91\\
-1401	30.96\\
-1401	29.53\\
-1401	30.91\\
-1401	37.91\\
-1151	45.07\\
-630	53.95\\
-1226	49.94\\
-1401	38.64\\
-1180	42.44\\
-1116	38.99\\
-874	28.64\\
-1216	18.86\\
-1401	21.1\\
-1401	11.3\\
-1401	7.58\\
-1401	11.81\\
-1401	13.36\\
-1401	14.02\\
-1401	22.83\\
-1401	20.01\\
-1401	16.52\\
-1401	15.8\\
-1401	21.57\\
-1401	15.2\\
-1401	12.55\\
-1401	11.78\\
-1401	11.26\\
-1401	18.81\\
-1399	44.94\\
-1401	48\\
-1401	25.54\\
-1401	23.29\\
-1401	20.33\\
-1401	19.29\\
-1401	15.89\\
-1401	14.23\\
-1401	13.29\\
-1401	11.77\\
-1401	11.11\\
-1401	10.12\\
-1401	11.23\\
-1401	10.91\\
-1401	11.44\\
-1401	10.77\\
-1401	12.34\\
-1401	13.02\\
-1401	13.95\\
-1401	11.07\\
-1401	9.33\\
-1401	3.07\\
-1401	5.41\\
-1401	9.43\\
-1401	22.78\\
-1389	52\\
-1401	49\\
-1401	21.31\\
-1401	20.51\\
-1401	28.62\\
-814	29.73\\
-783	26.33\\
-885	23.93\\
-1401	20.73\\
-1374	21.96\\
-918	26.41\\
-650	42\\
-848	50.81\\
-678	49\\
-1151	56.84\\
-772	59.16\\
-815	65.27\\
-1007	50\\
-824	50\\
-799	49.96\\
-706	49.08\\
-595	51.87\\
-459	67.78\\
-837	47.44\\
-669	69.94\\
-969	54.86\\
-1020	44\\
-976	45\\
-991	44.25\\
-580	31.57\\
-333	32.18\\
-648	29.99\\
-658	24.07\\
-332	24.89\\
-282	28.06\\
-201	40.44\\
-537	49.49\\
-978	50.12\\
-1018	63.77\\
-1125	55\\
-870	61.99\\
-697	58.13\\
-793	52.24\\
-895	45.13\\
-979	42\\
-1017	41.36\\
-822	42.44\\
-963	39.68\\
-227	46.28\\
-563	41.55\\
-1300	39\\
-973	32.73\\
-862	30.33\\
-1401	22.2\\
-1220	22.46\\
-1329	14.3\\
-1315	22.46\\
-1108	29.66\\
-1350	27.44\\
-711	42.36\\
-479	49.79\\
-1301	47.99\\
-1401	48\\
-1401	39.01\\
-1401	34.72\\
-1401	33.41\\
-1401	32.59\\
-1401	31.48\\
-1401	28.19\\
-1401	42.44\\
-1032	47.59\\
-1319	42.44\\
-1102	54.97\\
-1096	49\\
-1100	45\\
-939	44\\
-1056	41.93\\
-618	26.47\\
-489	26.72\\
-1126	24.54\\
-1401	17.18\\
-1331	18.05\\
-812	24.57\\
-96	37.54\\
-109	46.15\\
-1401	41.33\\
-1401	35.3\\
-1401	31.3\\
-1401	29.01\\
-1401	26.8\\
-1401	26.56\\
-1401	25.76\\
-1401	26.52\\
-1401	26.84\\
-1401	28.95\\
-1401	36.94\\
-1401	45.07\\
-1401	32\\
-1401	27.84\\
-1401	31.24\\
-983	30.47\\
-924	21.24\\
-1401	12.08\\
-1401	11.32\\
-1401	10.25\\
-1401	10.65\\
-1401	18.63\\
-1212	29.62\\
-1112	40\\
-1401	39.18\\
-1401	38.33\\
-1401	44.54\\
-880	63.62\\
-1169	57.66\\
-1227	55.89\\
-1383	45\\
-1361	39.96\\
-1348	39.94\\
-1401	38.93\\
-782	42.43\\
-658	55\\
-752	47.39\\
-465	43.08\\
-19	43.08\\
420	36.5\\
-245	33.94\\
-538	28.54\\
-513	24.08\\
-1501	19.07\\
-1501	13.66\\
-1501	17.92\\
-694	22.76\\
-844	24.31\\
-925	29.83\\
-746	32.37\\
-1016	32.33\\
-971	32.48\\
-1009	31.22\\
-1295	27.59\\
-1169	27.84\\
-1067	26.45\\
-1089	27.99\\
-1242	31.39\\
-1034	38.04\\
-226	45.01\\
-932	36.93\\
-863	31.7\\
-558	30.99\\
-354	24.69\\
-1082	22.91\\
-1073	20.54\\
-1019	20.14\\
-1288	20.59\\
-1501	17.65\\
-1501	19.82\\
-1501	17.05\\
-1501	16.07\\
-1501	21.36\\
-1401	24.42\\
-1401	26.9\\
-1401	30.58\\
-1283	29.26\\
-1401	24.04\\
-1401	23.46\\
-1501	18.55\\
-1501	17.34\\
-1501	19.29\\
-1314	40\\
-1450	45.12\\
-1501	42\\
-1501	37.1\\
-792	40.51\\
-591	35.55\\
-596	32.64\\
-451	30\\
-516	27.8\\
-496	25.08\\
-403	25.3\\
-324	28.82\\
-300	40.58\\
9	57.32\\
-654	57.52\\
-986	54.94\\
-1001	42.91\\
-764	42.6\\
-549	43.18\\
-429	44.64\\
-418	43.91\\
-382	41.46\\
-610	40.91\\
-636	40.93\\
402	47.23\\
807	96.69\\
-228	48.73\\
-1235	41.12\\
-607	44.62\\
-871	41.69\\
476	37.3\\
832	36.95\\
596	33.3\\
340	30.65\\
622	31\\
977	35.52\\
1183	42.44\\
573	45\\
1294	49.47\\
1286	49.99\\
1401	51.15\\
1401	45.85\\
1401	45.62\\
1401	42.97\\
1401	42.92\\
824	39.82\\
998	40.1\\
739	44.07\\
1266	49.08\\
1401	81.94\\
1001	48.12\\
1008	43.42\\
1273	42.8\\
1401	46.99\\
1355	39.53\\
1401	41.02\\
1401	40.42\\
856	32.22\\
677	31.13\\
907	34.68\\
1301	44.49\\
1130	48.93\\
506	52.06\\
257	52.07\\
358	54.01\\
304	48.32\\
433	47\\
208	43.52\\
129	41.95\\
69	41.09\\
-8	40.24\\
296	42.24\\
1057	48.68\\
1401	86.74\\
1153	46.5\\
321	42.37\\
1175	43.59\\
1324	42.44\\
1187	41.11\\
1401	42.44\\
1401	38.41\\
1324	35.02\\
1265	34.24\\
1401	38.08\\
1401	46.51\\
1401	51.57\\
1401	59.9\\
1401	52.44\\
1401	51.19\\
1401	50.58\\
1401	48.79\\
1401	49.4\\
1401	46.77\\
1401	43.27\\
1401	42.07\\
1401	44.45\\
1401	50\\
1401	81.51\\
1401	62.32\\
1401	46.92\\
1401	48.57\\
1401	45.4\\
1401	40.77\\
1401	41.97\\
1401	41.04\\
1401	37.01\\
1401	35.1\\
1401	40.21\\
1401	45.87\\
1401	51.76\\
1174	58.1\\
1093	57.39\\
647	47.4\\
616	45\\
31	42.06\\
-25	40.59\\
-3	40.81\\
221	39.99\\
334	40\\
607	43\\
1377	44.29\\
1401	70\\
1125	45\\
191	42.44\\
536	42.44\\
740	40.06\\
806	43.99\\
474	32.81\\
253	31.02\\
302	31.11\\
258	32\\
166	31.32\\
353	36.89\\
51	39.94\\
-2	43\\
-33	47.44\\
-123	45.62\\
-239	42.62\\
-409	40\\
-527	38\\
-586	34.99\\
-428	34.96\\
-128	39.96\\
225	43\\
357	43\\
40	46.17\\
567	54.96\\
506	46\\
587	44.94\\
555	40.56\\
616	44\\
501	34.94\\
297	29.49\\
230	29.49\\
292	30.7\\
519	35.04\\
214	35.04\\
-48	39.33\\
-49	40\\
-118	41\\
-358	38.83\\
-462	37.44\\
-597	36.99\\
-782	32.44\\
-769	30.81\\
-713	31.8\\
-297	39.96\\
-236	39.96\\
211	45\\
346	69.97\\
335	59.94\\
957	55\\
506	42.44\\
-359	30.91\\
113	32.46\\
127	31.28\\
364	32.2\\
558	35.77\\
241	28.14\\
996	40.69\\
1128	69.53\\
467	55.47\\
776	56.01\\
1049	48.33\\
853	46.08\\
538	46.7\\
265	43.73\\
179	41.79\\
103	40\\
294	42.22\\
520	43.22\\
648	40.47\\
112	48.47\\
756	54.61\\
1054	50\\
1073	44.94\\
686	44\\
1128	54.48\\
1124	43.64\\
1003	41.06\\
959	40.5\\
880	36.26\\
1128	44.94\\
1128	48\\
825	57.44\\
609	62.19\\
1087	57.89\\
957	47.29\\
1128	46.64\\
734	37.77\\
892	40\\
820	39.96\\
813	38.76\\
909	39.96\\
1035	42.11\\
1128	73.81\\
877	50.12\\
899	45\\
975	47\\
1128	120\\
1128	120\\
1128	65\\
965	37.69\\
1039	37.53\\
1056	37.53\\
1099	37.75\\
805	36.93\\
1128	44\\
687	49.22\\
918	62.17\\
1128	54.35\\
1013	48\\
947	45\\
678	45\\
565	44\\
562	42.12\\
614	41.19\\
698	44\\
898	47.44\\
1128	46\\
1078	52.92\\
430	58.31\\
407	53.12\\
548	44.94\\
938	40.98\\
1128	34.99\\
894	31.27\\
914	31.27\\
1022	32.81\\
984	28.3\\
899	29.55\\
1054	38.64\\
1128	55.5\\
745	52.09\\
858	49.94\\
990	50\\
1128	51\\
927	47.44\\
851	47\\
787	44.94\\
852	44.42\\
1052	45\\
1128	50\\
1128	69.97\\
1047	44.94\\
891	50\\
904	47.44\\
1128	63.41\\
1128	64.08\\
1115	37.23\\
697	34.37\\
803	34.24\\
676	30.51\\
765	30.51\\
1024	31.4\\
982	37.77\\
1128	46.07\\
1128	51.91\\
1112	56.15\\
1128	63.46\\
1128	120\\
1128	58.9\\
967	47.44\\
962	44.94\\
952	40\\
1009	40\\
998	40\\
995	39.94\\
664	40.99\\
667	41.76\\
694	38.72\\
1128	65.37\\
945	43\\
942	34.3\\
916	37.84\\
772	34\\
818	34\\
731	30.74\\
710	30.74\\
379	29.5\\
454	34.44\\
800	43.18\\
666	46\\
626	45\\
813	47.44\\
706	42.8\\
427	35.87\\
210	34.44\\
206	34.44\\
363	33.61\\
507	35\\
953	42.44\\
732	42.44\\
850	49.99\\
1014	47.44\\
1049	42\\
895	44.01\\
263	35.16\\
69	42.16\\
40	34.94\\
-72	34.21\\
-37	31.5\\
3	33.97\\
-570	24.25\\
-660	24.3\\
-653	25.31\\
-507	34.94\\
-489	38.99\\
-353	41.96\\
-107	47.44\\
-380	39.94\\
-483	35.15\\
-501	34.9\\
-432	34.94\\
-497	39.96\\
-181	42.17\\
78	44.94\\
253	59.94\\
207	54.96\\
196	48\\
191	35.99\\
400	29.88\\
395	27.73\\
387	28.85\\
237	29.85\\
375	29.75\\
744	29.99\\
813	41\\
515	49.94\\
687	53.05\\
652	51.06\\
844	47.45\\
829	44.61\\
542	38.94\\
803	37.09\\
818	36.16\\
835	36.26\\
915	36.31\\
1056	39.12\\
722	45.31\\
525	46.99\\
516	48.75\\
946	38.58\\
1128	34.86\\
968	30.07\\
157	28.01\\
244	28.01\\
118	21.92\\
101	21.44\\
130	22.69\\
305	24.02\\
693	31.84\\
543	38.99\\
876	43.96\\
981	45.8\\
858	48.41\\
1055	53.7\\
856	43.88\\
828	44.6\\
1128	45.32\\
617	40.42\\
588	39.94\\
833	44.46\\
821	39.94\\
892	39.96\\
1128	66.56\\
1128	63.54\\
1128	65.06\\
1128	55\\
1128	41.06\\
968	37.77\\
1078	35.97\\
976	35.74\\
1123	37.06\\
1128	73.11\\
1128	80.78\\
1128	50.12\\
1128	55.91\\
1128	59.81\\
888	52.44\\
800	47.12\\
548	44\\
578	44.8\\
530	41.22\\
731	42.98\\
868	42.99\\
1094	45.87\\
1128	55\\
1128	50.12\\
882	54.25\\
882	42.44\\
1002	42.44\\
1128	70\\
1128	81.93\\
1090	39.75\\
1128	60.01\\
1128	39.15\\
1128	38.88\\
952	38.44\\
1128	69.9\\
1128	59.94\\
1128	63.98\\
1128	60.89\\
923	54.91\\
1128	60\\
975	45\\
1014	44.94\\
1128	72.04\\
1128	71.86\\
1128	69.9\\
1128	75.33\\
1128	75.67\\
1128	50\\
976	50\\
1128	55\\
1128	69.9\\
1106	39.78\\
1128	50\\
1128	100\\
1128	55\\
1128	40.1\\
1128	59.94\\
1128	37.34\\
1128	42.91\\
1128	65\\
1128	70.93\\
1128	69.71\\
1128	69.54\\
1093	52.44\\
1128	68.76\\
1054	50\\
822	42.86\\
799	39.99\\
708	38\\
737	39.94\\
975	39.94\\
959	44.44\\
806	42.91\\
1128	42.5\\
1121	45\\
965	42.02\\
1128	79.9\\
1045	41.06\\
983	39.05\\
946	39\\
1117	38.27\\
1128	39.3\\
1076	39.54\\
970	42.44\\
945	46.24\\
1110	50.4\\
1112	60.52\\
896	56.09\\
653	45.12\\
366	40\\
174	38.97\\
160	37.84\\
272	37.66\\
455	39.94\\
704	42.44\\
780	42.44\\
562	44.94\\
831	53.12\\
950	50.31\\
1128	50\\
831	40.88\\
542	30\\
258	19.99\\
363	30\\
521	34.61\\
556	33.86\\
651	34.92\\
542	34.99\\
386	39.65\\
-9	40.86\\
-94	39.94\\
-106	39.96\\
-349	39.99\\
-552	38.38\\
-638	35\\
-681	32.96\\
-663	34.94\\
-613	38.37\\
-167	41.55\\
151	42.44\\
332	49.96\\
437	48\\
442	40\\
370	31.42\\
-49	20.12\\
-326	14.94\\
211	14.92\\
105	15.78\\
144	19.96\\
556	22.54\\
604	37.14\\
524	43.2\\
1128	44.35\\
1071	48.65\\
621	48.66\\
512	46.67\\
433	42.9\\
248	42.44\\
211	45\\
248	44.96\\
389	44.96\\
614	55.74\\
833	48\\
878	48\\
693	50\\
833	55\\
1038	49.04\\
955	42.44\\
560	38.11\\
752	36.22\\
541	34.31\\
394	30\\
429	29.27\\
381	27.44\\
847	38.07\\
1128	44.28\\
1128	50.97\\
1006	58.5\\
926	51\\
911	50.12\\
672	47.44\\
476	49.94\\
354	48.66\\
315	43\\
202	41.33\\
129	41.06\\
493	40\\
536	40\\
614	42.32\\
642	45\\
977	45\\
922	42.14\\
796	36.96\\
720	34.48\\
713	33.9\\
759	33.94\\
814	35.15\\
747	37.36\\
870	40.92\\
829	55.1\\
1033	58.87\\
1128	63.9\\
579	47\\
271	42.44\\
95	39.94\\
-282	39.47\\
-218	39.94\\
29	42\\
230	44\\
601	49.31\\
880	45\\
975	44.94\\
1090	47.44\\
822	46.73\\
1128	72.29\\
1054	42.39\\
706	37.73\\
748	35.49\\
678	34.99\\
614	33.04\\
705	29.25\\
762	31.76\\
803	38.36\\
728	46.11\\
836	51.52\\
389	50\\
249	49.94\\
506	55.02\\
135	44.94\\
-24	45\\
-143	42.44\\
-171	42.13\\
-87	42.44\\
74	42.44\\
124	39.99\\
205	47.44\\
500	47.44\\
591	54.25\\
805	45\\
854	41.45\\
527	36.83\\
368	31.73\\
285	31.23\\
210	29.95\\
328	28.68\\
408	29.95\\
702	36.83\\
361	38.03\\
454	42.48\\
353	46.72\\
475	48\\
453	51.9\\
325	43.45\\
173	42.12\\
-50	39.94\\
-93	38.92\\
-108	39.94\\
189	42.12\\
333	39.99\\
317	39.09\\
94	39.94\\
369	42.44\\
771	43.84\\
708	43.84\\
424	35.91\\
121	29.1\\
-66	27.72\\
-55	27.98\\
-21	27.59\\
-267	24.71\\
-136	27.72\\
-105	29.99\\
35	35.65\\
-14	39.94\\
-47	39.94\\
-165	38.12\\
-280	36\\
-614	34.94\\
-718	32.99\\
-563	35\\
-511	35\\
-560	35.65\\
-524	38.38\\
-355	37.85\\
-566	37.84\\
-622	38.19\\
-280	38.39\\
296	37.97\\
-640	22.76\\
-947	18.19\\
-768	19.79\\
-686	19.75\\
-684	19.78\\
-804	19.9\\
-1045	19.79\\
-1272	12.92\\
-1139	27.21\\
-906	28.74\\
-801	30.4\\
-758	31.3\\
-809	31.3\\
-1024	27.46\\
-1196	21.68\\
-1174	21.47\\
-1140	24.96\\
-779	32.16\\
-188	39.94\\
176	43\\
27	50\\
35	54.94\\
122	53.22\\
96	42\\
44	27.46\\
60	23\\
58	19.94\\
-120	19.88\\
-127	20.09\\
-67	28.74\\
-214	29.45\\
-827	20.61\\
-826	29.79\\
-770	32.44\\
-816	34.99\\
-859	35\\
-1007	33.86\\
-1279	30.65\\
-1497	29.79\\
-1501	14.99\\
-1439	27.99\\
-1060	34.94\\
-622	40\\
-432	40\\
-698	42.44\\
-714	47.44\\
-399	49.94\\
-248	40.41\\
179	36.1\\
430	35.94\\
222	31.81\\
41	31.15\\
43	29.7\\
196	31.4\\
549	40.67\\
1128	50.24\\
1128	54.16\\
819	55.12\\
888	51\\
726	48\\
709	46.72\\
638	46.72\\
582	44.26\\
562	43.94\\
618	44.16\\
880	47.44\\
1066	44.94\\
1096	45.15\\
1055	45.67\\
916	44.94\\
1128	56.96\\
1128	43.75\\
595	31.96\\
448	31.04\\
429	31.96\\
387	31.95\\
443	28.94\\
686	34.33\\
488	40.59\\
348	46.49\\
635	50.8\\
202	44\\
144	44.94\\
339	47.44\\
115	44.94\\
41	44.94\\
174	44.94\\
257	43.82\\
391	44.94\\
230	47.44\\
594	45\\
695	46.32\\
397	47.44\\
786	47.44\\
933	44\\
535	37\\
704	37.32\\
483	32.29\\
494	31.63\\
447	31.38\\
540	31.9\\
384	32.16\\
619	39.83\\
759	45.42\\
1081	46.73\\
819	42.56\\
609	45\\
748	48.28\\
753	47.44\\
571	44.94\\
521	41.33\\
560	40.36\\
583	40.87\\
530	39.94\\
757	39.97\\
633	39.95\\
431	39.45\\
101	42.44\\
193	43.38\\
244	35.98\\
-51	29.5\\
-83	30.92\\
-199	29.66\\
-280	29.99\\
-147	29.99\\
60	31.95\\
-49	35.82\\
30	44.4\\
-277	44.94\\
-164	57.24\\
-433	53.99\\
-718	44.94\\
-1059	37.44\\
-926	39.94\\
-803	40.01\\
-794	43.67\\
-607	43.64\\
-331	50.11\\
-179	49.33\\
-104	45\\
-74	44.99\\
25	48.9\\
-56	39.94\\
-142	42.39\\
82	40.16\\
13	33.49\\
-284	31.75\\
-595	28.02\\
-691	26.5\\
-671	28.08\\
-670	29.16\\
-685	30.35\\
-869	32.79\\
-904	35\\
-1100	34.81\\
-947	35.95\\
-1009	34.94\\
-1450	29.7\\
-1037	31.75\\
-1391	27.9\\
-1400	28.42\\
-1126	34.58\\
-897	40\\
-702	42.44\\
-703	44.94\\
-624	50.48\\
-577	54.94\\
-713	45\\
-678	35.28\\
-789	29.68\\
-736	30.94\\
-770	27.15\\
-795	25.23\\
-1144	19.19\\
-1178	17.49\\
-795	27.9\\
-1498	22.51\\
-1331	32.5\\
-1431	34.96\\
-1275	38.46\\
-1318	38\\
-1420	40\\
-1501	35.23\\
-1401	19.79\\
-1401	16.22\\
-1268	42\\
-1205	42.44\\
-987	41\\
-1146	38\\
-1056	44.96\\
-911	41\\
-641	39.94\\
-168	33.28\\
-251	29.68\\
-305	30.94\\
-260	29.68\\
-184	28.42\\
37	28.25\\
235	38.1\\
328	55.52\\
382	52.44\\
168	55.54\\
48	55.56\\
254	54.02\\
354	52.57\\
231	42.57\\
120	40.96\\
-55	39.27\\
-7	36.13\\
-276	38.23\\
7	43.38\\
526	53.12\\
229	44.95\\
361	50.07\\
391	47.44\\
592	44.94\\
244	35.84\\
289	36.03\\
381	34.99\\
322	33.53\\
410	33.53\\
597	33.59\\
179	38.97\\
371	47.92\\
336	50.38\\
411	50.79\\
432	47.44\\
392	49.32\\
492	46.74\\
751	50.58\\
877	48\\
854	44.34\\
862	43.37\\
974	48\\
1069	45\\
864	45.96\\
735	46.16\\
748	44.96\\
945	49.96\\
576	43.59\\
390	31.85\\
71	30.01\\
284	31.23\\
220	31.23\\
533	34.51\\
631	33.27\\
407	35.03\\
295	44.93\\
314	52.26\\
342	49.11\\
336	50\\
267	49.94\\
135	43.59\\
293	48.28\\
388	48\\
319	44\\
344	40.56\\
486	44.94\\
600	40.12\\
441	37.44\\
418	37.51\\
655	42.44\\
772	45\\
406	39.94\\
-40	34.99\\
-22	38.29\\
14	33.38\\
-30	31.84\\
-52	33.38\\
-693	30.76\\
-1055	29.27\\
-1341	31.05\\
-1501	34.78\\
-1435	40\\
-1489	42.44\\
-1493	45\\
-1347	48\\
-1501	43\\
-1501	25\\
-1501	35\\
-1392	42.44\\
-1354	42\\
-981	42.44\\
-778	42.44\\
-717	40\\
-636	47.44\\
-228	50\\
-252	37.99\\
-912	47.44\\
-730	36.49\\
-731	32.07\\
-752	30.78\\
-850	29.15\\
-733	31.5\\
-895	33.78\\
-1148	38\\
-1171	42\\
-1059	51.27\\
-1038	56.13\\
-1118	50.88\\
-1105	46\\
-1501	35\\
-1501	33.41\\
-1401	26.06\\
-1401	25.38\\
-1360	35\\
-995	33.38\\
-929	33.26\\
-833	33.78\\
-700	35\\
-357	36\\
-556	34.99\\
-814	34.96\\
-1042	26.73\\
-1310	23.06\\
-474	30.43\\
-437	31.72\\
-580	30.43\\
-957	28.29\\
-1161	31.77\\
-1038	38\\
-1037	42\\
-1179	41.8\\
-1260	39.96\\
-1464	35.61\\
-1501	25.47\\
-1501	23.28\\
-1501	22.2\\
-1383	32.44\\
-1118	37.44\\
-795	44.96\\
-855	42.44\\
-1026	40\\
-897	49.94\\
-774	53.76\\
-822	49.94\\
-378	50.76\\
-629	41.19\\
-746	32.44\\
-775	30.7\\
-259	35.92\\
-96	37.74\\
-95	33.16\\
-878	25.68\\
-1045	30.7\\
-1069	33.07\\
-1172	34\\
-1290	32.44\\
-1478	31.44\\
-1501	11.22\\
-1501	5.34\\
-1501	4.73\\
-1501	7.3\\
-1154	32.75\\
-658	40\\
-448	40.12\\
-445	42.44\\
-484	49.94\\
-237	54.96\\
-184	45.99\\
29	30.83\\
7	30.28\\
193	34.33\\
8	34.96\\
101	36.45\\
156	37.5\\
593	40.09\\
987	44.08\\
632	48.09\\
579	43.94\\
370	44.94\\
289	47\\
-32	39.99\\
146	47.44\\
291	42.44\\
303	44\\
442	43.96\\
659	47.63\\
931	52.44\\
969	48\\
978	46.99\\
896	53.94\\
1024	49.94\\
653	35.02\\
606	36.98\\
184	34.44\\
119	31.05\\
-117	24.89\\
78	30.65\\
-306	27.45\\
283	34.37\\
561	37.98\\
591	38.94\\
610	44.94\\
514	57.71\\
393	57.71\\
110	44.94\\
132	42.44\\
118	40\\
242	41.44\\
393	41.27\\
558	47\\
235	37.04\\
508	38.94\\
555	39.94\\
340	40.12\\
570	45\\
633	41.45\\
63	32.58\\
-53	30.22\\
-133	29.88\\
-163	28.89\\
-64	28.89\\
237	30.74\\
525	38.02\\
585	44.82\\
662	47.07\\
576	59.49\\
586	57.46\\
551	57.46\\
345	54.39\\
122	47\\
100	44.94\\
76	45\\
68	42.44\\
328	47.25\\
480	42.41\\
477	38.44\\
445	38.51\\
488	45\\
673	48.87\\
553	32\\
47	29.67\\
-60	26.87\\
-163	26.13\\
-228	26.07\\
-370	25.96\\
-2	27.03\\
315	35.78\\
614	38.09\\
864	38\\
1075	44.94\\
1104	47.44\\
1144	48.87\\
852	47.44\\
815	48.87\\
766	46.56\\
717	44.93\\
733	45.26\\
860	53.26\\
810	47\\
564	40\\
281	47.6\\
254	43.18\\
448	36.69\\
450	34.43\\
-168	36.43\\
-532	30.11\\
-851	22.82\\
-668	27.23\\
-568	23.15\\
-267	29.35\\
60	34.86\\
489	39.94\\
516	39.35\\
478	47\\
386	48.88\\
288	45\\
73	42.44\\
-97	37.44\\
-287	34.94\\
-446	34.29\\
-439	34.3\\
-350	34.31\\
-228	33.28\\
-294	34.74\\
-295	35.27\\
-235	35\\
-179	47.44\\
-275	40.83\\
-683	40\\
-849	30.52\\
-745	30.52\\
-733	30.06\\
-730	25.29\\
-734	27.99\\
-1154	23.53\\
-944	28.35\\
-877	34.94\\
-816	45\\
-742	55.62\\
-794	50\\
-823	45\\
-874	40\\
-1074	33.7\\
-1065	32.25\\
-1023	32.25\\
-1026	35\\
-851	33.7\\
-957	33.99\\
-996	32.31\\
-980	31.63\\
-763	32.06\\
-900	23.55\\
-1119	22.77\\
-1212	21.56\\
-1394	21.09\\
-1401	6.38\\
-1401	4.97\\
-1401	2.8\\
-1401	-0.01\\
-1401	-0.01\\
-1401	2.28\\
-1401	8.06\\
-1401	10.09\\
-1401	10.46\\
-1401	10.82\\
-1401	9.8\\
-1401	20.01\\
-1401	3.51\\
-1401	20.01\\
-1401	2.47\\
-1395	35\\
-1401	33.71\\
-1308	33.73\\
-979	33.63\\
-716	32.76\\
-779	24.48\\
-929	23.15\\
-495	23.01\\
-655	22.7\\
-740	22.7\\
-694	23.05\\
-66	31.8\\
-111	37.69\\
-99	39.94\\
36	42.44\\
68	48\\
433	57.71\\
332	58.48\\
63	44.99\\
187	52\\
286	53.71\\
404	48.4\\
551	48.87\\
619	64.96\\
536	42.44\\
726	42.44\\
796	42.92\\
794	48.87\\
896	57.44\\
635	49.01\\
1219	50\\
1082	35.16\\
998	32.75\\
938	31.6\\
990	31.91\\
443	29.26\\
945	37.17\\
601	50.3\\
1116	58.51\\
946	63.46\\
156	51.85\\
155	52.02\\
751	48.87\\
102	42.65\\
169	39.78\\
805	44.17\\
635	38\\
270	36.95\\
938	39.96\\
972	41.79\\
906	42.27\\
968	41.03\\
1219	42.44\\
1070	34.07\\
911	31.83\\
547	27.44\\
813	31.12\\
1178	32.56\\
1086	31.57\\
1219	58.74\\
1196	33.71\\
1219	59.35\\
1219	49.11\\
1067	44.94\\
1036	42.44\\
941	40.33\\
952	39.94\\
564	39.94\\
582	42.44\\
585	44\\
576	46.67\\
727	49.11\\
927	43\\
916	39.99\\
962	39\\
1102	40.66\\
1219	72.25\\
1219	47\\
763	31.77\\
595	30.89\\
793	31.89\\
851	30.95\\
886	30\\
787	30.55\\
728	38.64\\
836	47.42\\
634	55.64\\
771	51.69\\
649	47.39\\
561	46.74\\
427	40.37\\
419	39.99\\
527	37.98\\
319	35.2\\
340	35.18\\
714	37.44\\
1061	40\\
1109	40.57\\
1160	40\\
1219	43.2\\
1219	70.58\\
1151	40\\
684	32.34\\
967	33.91\\
915	33.72\\
927	32\\
896	30.65\\
574	29.1\\
841	34.09\\
1056	55.13\\
926	44.96\\
853	47\\
776	45\\
524	41\\
116	41\\
-24	40\\
-102	41\\
-108	38\\
69	39.94\\
296	42.44\\
567	44.16\\
617	40\\
667	39.47\\
782	39.94\\
741	39.94\\
939	40\\
318	31.5\\
-145	27.88\\
269	32.98\\
131	31.68\\
190	33.67\\
-48	31.68\\
-780	26.31\\
-776	30.39\\
-778	34.96\\
-654	42.44\\
-1080	37.44\\
-1182	34.83\\
-1114	34.27\\
-1248	31.68\\
-1374	30.39\\
-1314	29.99\\
-1179	31.98\\
-944	36.4\\
-579	44.94\\
-389	46.83\\
-318	47.12\\
-224	50\\
-468	49.94\\
-37	48.06\\
-324	52.35\\
-597	32\\
-326	35.13\\
-450	33.33\\
-395	33.19\\
-395	30.75\\
-1130	17.97\\
-1369	20.01\\
-1268	21.22\\
-1197	29\\
-1264	30.75\\
-1261	30.75\\
-1333	29.96\\
-1501	22\\
-1284	31.03\\
-1030	33.2\\
-986	33.2\\
-1348	31.77\\
-1079	37.44\\
-997	37.44\\
-786	39.94\\
-648	45\\
-606	49\\
-430	39.94\\
-199	30.9\\
-170	31.47\\
-169	30.57\\
-174	30.1\\
-187	28.12\\
-136	29.43\\
71	38.44\\
171	49.95\\
406	53.11\\
437	50.39\\
644	46.96\\
511	47.2\\
618	44\\
449	42.43\\
313	39.45\\
171	37.44\\
303	37.92\\
557	41.18\\
754	46.5\\
714	53.46\\
626	49.47\\
645	47.44\\
646	39.98\\
781	33.29\\
-25	31.59\\
-142	30.59\\
-141	29.02\\
-176	29.42\\
-136	29.62\\
-225	31.5\\
-60	42.77\\
138	53.94\\
297	52.39\\
433	46.82\\
619	42.7\\
558	41.22\\
366	37.99\\
371	38.06\\
284	36.08\\
320	36.97\\
441	41\\
700	57.21\\
900	48\\
963	48\\
732	53.11\\
881	46.04\\
789	44.94\\
709	48.01\\
865	45.85\\
813	35.72\\
729	33.52\\
633	32.84\\
712	32.7\\
847	31.95\\
709	41.19\\
502	48.83\\
608	52.91\\
792	52.99\\
682	44\\
1056	68.83\\
1028	39.94\\
912	44.94\\
827	42.3\\
810	39.94\\
782	44.94\\
772	53.11\\
704	43.12\\
460	47.27\\
388	43.46\\
489	37.97\\
624	44.94\\
172	44.18\\
-112	28.25\\
-171	24\\
-3	20.41\\
-244	15.13\\
-150	18.41\\
-174	22.91\\
10	29.25\\
351	36.55\\
505	39.66\\
815	38.58\\
952	40.5\\
743	42.29\\
766	41.26\\
468	39.94\\
574	42.29\\
645	42.29\\
647	48\\
529	48\\
1032	54.24\\
603	44.7\\
676	43.82\\
946	46.63\\
1056	55\\
1056	47.01\\
455	30.05\\
439	34.8\\
305	31.81\\
277	31.03\\
376	31.81\\
326	30.75\\
487	34.04\\
653	49.95\\
386	51.06\\
883	52.97\\
630	52.41\\
481	57.04\\
317	47.98\\
415	45.23\\
352	40.53\\
287	37.38\\
64	35\\
241	37.54\\
260	42\\
177	43.98\\
117	41.92\\
108	40.49\\
380	42.58\\
400	49.78\\
49	47.24\\
-199	34.57\\
-349	34.29\\
-339	33.19\\
-368	32.08\\
-420	30.58\\
-474	29.99\\
-550	32.44\\
-797	39.94\\
-799	40.77\\
-784	48\\
-847	47.44\\
-899	44.94\\
-1231	35\\
-1326	34.09\\
-1392	32.44\\
-1324	32\\
-1214	37.44\\
-1111	40\\
-1074	42.44\\
-969	45\\
-1027	40\\
-733	49.96\\
-886	44.96\\
-796	32.94\\
-755	24.96\\
-668	20.34\\
-693	16.18\\
-813	15.45\\
-709	15.79\\
-733	18.94\\
-809	27\\
-752	30.85\\
-1068	37.99\\
-1246	37.44\\
-1278	38.99\\
-1340	38.3\\
-1501	32.94\\
-1501	13.43\\
-1501	13.88\\
-1494	30\\
-1292	42.44\\
-1110	44.94\\
-956	50\\
-951	49.99\\
-874	50\\
-819	55\\
-777	44.94\\
-264	50\\
-360	32.75\\
-357	32.77\\
-353	29.99\\
-275	32.75\\
-147	29.99\\
225	37.12\\
800	46.58\\
991	42.44\\
1056	42.2\\
1000	42.4\\
1056	60\\
1045	52.08\\
1056	70.42\\
1056	70.83\\
1056	69.99\\
1056	49.04\\
1056	60\\
1056	55\\
1056	45.21\\
797	42.29\\
1056	44.94\\
1056	69.99\\
920	41.75\\
666	30.03\\
680	31.82\\
718	33.35\\
717	33.85\\
836	34.01\\
1056	50\\
987	33.63\\
891	43.97\\
552	47.44\\
696	50.01\\
882	49.93\\
833	50.01\\
443	44.71\\
407	44\\
543	45.42\\
865	41.89\\
858	41.2\\
1041	45\\
1056	50.12\\
977	40.94\\
1056	42.44\\
1056	41.67\\
841	39.99\\
982	37.58\\
1056	200\\
624	34.62\\
138	30.85\\
-18	29.63\\
174	31.63\\
458	31.72\\
939	33.85\\
1056	44.94\\
786	44.1\\
831	45.17\\
1056	54\\
962	50\\
1056	60\\
1020	50\\
978	45.06\\
912	43.01\\
850	46.4\\
880	54.31\\
943	44.94\\
984	41.2\\
746	38.13\\
824	41.96\\
1056	54\\
969	44.94\\
471	39.94\\
221	33.94\\
169	34.09\\
189	33.58\\
182	32\\
207	29.42\\
184	30.43\\
-112	29.96\\
-349	28.61\\
-145	32.44\\
-142	34.37\\
-63	37.44\\
-175	39.94\\
-401	34.94\\
-562	33.89\\
-559	33\\
-660	33.25\\
-730	34.33\\
-310	44.94\\
-121	45\\
-144	49.83\\
142	44.53\\
160	47.94\\
13	45\\
110	33.05\\
-252	30.06\\
-43	31.35\\
-42	31.35\\
-79	30.66\\
45	31.35\\
-327	32.37\\
-210	38.73\\
-392	42.42\\
-338	44.94\\
-445	42.44\\
-534	42.44\\
-519	40.95\\
-738	35.63\\
-626	37.17\\
-848	34.69\\
-809	33.95\\
-980	37.11\\
-682	47.44\\
-659	45.24\\
-601	45.84\\
-532	42.76\\
-406	42.32\\
-304	39.94\\
-33	36.47\\
-87	44.25\\
-72	35\\
-102	32.87\\
-28	32.51\\
-61	32.87\\
-192	32.51\\
-302	35\\
-630	35\\
-564	43.15\\
-1077	37.12\\
-1102	36\\
-1201	34.94\\
-836	38.75\\
-914	34.25\\
-914	34.99\\
-775	34.94\\
-599	38.75\\
-492	41\\
-395	42.44\\
-354	39.94\\
-281	39.94\\
-357	44.94\\
-113	44.94\\
-254	38.23\\
-409	34.99\\
-423	38.79\\
-478	34.29\\
-499	32.29\\
-605	31.35\\
-731	29.3\\
-833	29.96\\
-719	31.35\\
-986	30.9\\
-874	33.56\\
-822	36\\
-874	38\\
-1058	33.67\\
-1275	31.68\\
-951	35\\
-1158	30.37\\
-1153	31.68\\
-685	36\\
-714	39\\
-605	44\\
-456	44\\
-708	45\\
-61	46.9\\
-274	49.23\\
-288	41.08\\
-194	37.04\\
-106	34.04\\
-14	34.04\\
-90	30.58\\
387	38.33\\
579	42.04\\
236	43.01\\
155	43.5\\
-86	43.72\\
-205	44.99\\
-89	41.01\\
-12	50\\
30	47.7\\
110	43.58\\
248	42.44\\
447	54.85\\
148	40.76\\
182	42.89\\
166	42.89\\
347	41.91\\
735	48.93\\
533	45.32\\
-120	32.46\\
9	32.55\\
-59	31.47\\
-105	30.13\\
-108	30.79\\
-146	30.11\\
-173	40.94\\
215	48.29\\
-156	53.15\\
2	50.94\\
-50	47.86\\
8	51.25\\
-244	47.69\\
-457	44.99\\
-57	45.84\\
-109	44.37\\
29	44.88\\
-327	47.02\\
133	48.47\\
193	48.97\\
81	47.61\\
105	44.48\\
454	45.61\\
247	37.5\\
223	32.49\\
273	31.54\\
203	33.2\\
205	33.2\\
202	33.25\\
-49	31.54\\
58	37.66\\
-20	47.44\\
488	50.44\\
179	50.01\\
251	53.15\\
141	55.55\\
251	54.94\\
352	55.12\\
446	48.37\\
388	45.17\\
310	47\\
306	54.94\\
315	48\\
57	41.91\\
-36	39.67\\
64	37.24\\
529	45.51\\
414	39.94\\
-562	28.95\\
-370	28\\
-251	27.58\\
-216	27.58\\
-261	24.07\\
-534	25.03\\
-506	30.97\\
-354	37.67\\
-134	42\\
-215	46.44\\
-182	49\\
-192	47.96\\
-338	42\\
-502	44.94\\
-613	40.12\\
-588	37.44\\
-469	37.44\\
-226	43.56\\
-43	39.94\\
-13	40\\
137	37.2\\
159	36.83\\
588	45.41\\
364	41.99\\
51	31.38\\
-96	31.4\\
99	31.41\\
-46	31.37\\
59	31.39\\
112	32.86\\
270	40.01\\
387	41.71\\
311	44.94\\
-94	47.18\\
-322	45\\
-420	44.94\\
-496	40\\
-621	39.94\\
-564	39.94\\
-438	38\\
-373	38.49\\
-204	42.44\\
-158	44\\
-111	39.44\\
-17	37.68\\
-62	36.01\\
277	43.7\\
-266	32.3\\
-1085	28.34\\
-1274	26.07\\
-1251	25.06\\
-1229	24.73\\
-1284	24.99\\
-1400	24.09\\
-1501	10.35\\
-1501	15.98\\
-1476	29.19\\
-1293	31.86\\
-1253	34.94\\
-1219	32.14\\
-1334	31.21\\
-1401	20.01\\
-1401	17.38\\
-1427	29.58\\
-1281	30.43\\
-1100	33.86\\
-905	38.43\\
-805	40\\
-710	38\\
-688	33.87\\
-494	39.94\\
-580	30.57\\
-727	29.51\\
-896	27.3\\
-946	20.12\\
-796	25.04\\
-805	24.99\\
-817	16.32\\
-899	14.91\\
-946	15\\
-946	15.98\\
-946	18.13\\
-946	27.41\\
-946	27.46\\
-946	18.45\\
-946	17.1\\
-946	14.06\\
-946	13.7\\
-946	16.55\\
-946	20.08\\
-946	27.51\\
-799	37\\
-721	37\\
-583	34.93\\
-828	37.11\\
-635	37\\
-776	31.37\\
-963	28.36\\
-1114	25.04\\
-1056	24.99\\
-1095	24.99\\
-1171	22.28\\
-1158	24.33\\
-1132	27.99\\
-1401	26.09\\
-1401	26.93\\
-1401	26.46\\
-1401	28.89\\
-1401	29.87\\
-1401	26.71\\
-1401	25.25\\
-1401	21.61\\
-1375	31.09\\
-1385	36\\
-1214	37\\
-1060	39.94\\
-892	45\\
-895	39.94\\
-681	45.37\\
-596	35.99\\
-536	30.47\\
-501	31.37\\
-604	30.01\\
-706	29.25\\
-617	28.66\\
-457	21.77\\
-176	30.01\\
380	40.12\\
547	41.27\\
567	45.39\\
465	48\\
370	49.88\\
264	45\\
122	45.1\\
250	44.94\\
458	44.57\\
445	44.49\\
105	49.69\\
413	43.84\\
412	43.03\\
604	39.91\\
493	36.92\\
659	41.84\\
272	43.16\\
-215	31.07\\
-214	30.71\\
-200	30.68\\
-116	30.52\\
195	30.71\\
210	30.52\\
48	34.92\\
0	44.99\\
-140	49.54\\
-310	49.51\\
-667	49.88\\
-606	49.94\\
-431	49.4\\
-405	44.98\\
-284	43.77\\
-431	41.81\\
-534	39.57\\
-480	40.96\\
-114	47.96\\
192	44.94\\
324	39.6\\
251	36.24\\
378	43\\
162	44.01\\
51	30.95\\
-50	30.57\\
-91	29.39\\
-33	28.74\\
58	28.74\\
169	29.66\\
56	33.09\\
347	44.07\\
341	46.37\\
15	46.58\\
-37	46.12\\
-89	46.68\\
25	42.94\\
-144	41.05\\
-90	39.99\\
-75	38.35\\
-50	37.07\\
50	39.4\\
490	42\\
514	44.77\\
603	43.05\\
524	40.51\\
761	39.39\\
315	39\\
-46	32.37\\
-185	32.39\\
-220	31.39\\
-210	30.49\\
17	31.11\\
124	31.39\\
215	36\\
578	44.82\\
580	45.55\\
91	53.88\\
-148	51.42\\
-132	55\\
-414	45.6\\
-582	43\\
-333	47.75\\
-368	39.94\\
-304	38\\
-152	44\\
101	40\\
131	35.12\\
185	47\\
250	32.36\\
405	49.52\\
12	45\\
-77	44.99\\
-53	32\\
-178	30.34\\
-87	29.73\\
-149	27.34\\
-92	20\\
-75	22.33\\
-408	29.34\\
-95	32.03\\
-132	48.57\\
-273	50\\
-404	49.94\\
-520	42.44\\
-540	37.44\\
-655	33.32\\
-925	31.91\\
-971	31.87\\
-625	33.32\\
-485	36\\
-413	39.94\\
-354	36\\
-233	34.35\\
13	45.1\\
-302	48.33\\
-368	31.61\\
-697	31.3\\
-926	28.82\\
-829	28.62\\
-842	27.46\\
-697	23.12\\
-750	16.43\\
-795	22.89\\
-1014	27.44\\
-817	31.24\\
-622	32.44\\
-735	32.91\\
-945	33.11\\
-1060	31.6\\
-1315	30\\
-1401	22.89\\
-1401	22.89\\
-1163	30.65\\
-772	31.54\\
-562	33.11\\
-380	35.25\\
-425	31.59\\
-223	34.96\\
-458	31.63\\
-230	30.04\\
-36	27.8\\
-105	25.84\\
-259	21.49\\
145	27.46\\
239	24.89\\
453	33.93\\
796	41.17\\
837	41.93\\
775	40\\
754	42.44\\
826	47.44\\
956	45\\
882	44.94\\
762	37.49\\
773	34.99\\
639	33.09\\
};
\addplot [color=mycolor1,line width=1.0pt,mark size=0.3pt,only marks,mark=*,mark options={solid},forget plot]
  table[row sep=crcr]{%
849	33.61\\
923	35.11\\
708	36.82\\
686	35.99\\
596	34.53\\
883	36.05\\
565	30.9\\
460	30.84\\
522	30.87\\
618	30.87\\
675	31.36\\
695	30.87\\
666	28.57\\
308	36.96\\
544	44.71\\
557	51.99\\
703	52.44\\
175	50\\
190	49.99\\
191	42.98\\
200	41.83\\
802	41.45\\
906	39.94\\
830	37.44\\
981	38.34\\
1055	39.84\\
907	41.21\\
892	41.22\\
759	36.31\\
1068	40.48\\
1108	35.65\\
614	29.87\\
727	31.03\\
715	31.32\\
591	31.2\\
675	31.05\\
478	28.22\\
827	34.97\\
1219	48.51\\
1219	43.23\\
1219	48.51\\
1219	48.51\\
1219	48.51\\
1219	62.13\\
1144	39\\
984	40\\
910	37.44\\
953	37.44\\
1219	60.01\\
1219	65.5\\
1219	57.01\\
1219	60\\
1219	44.99\\
1219	69.47\\
1219	47.44\\
704	31.33\\
446	30.52\\
580	31.22\\
559	30.68\\
848	30.86\\
612	29.14\\
1013	30.52\\
1004	38\\
949	39.94\\
1029	43.73\\
986	49.47\\
944	55\\
817	40\\
915	42\\
1163	44.57\\
1107	39.94\\
1078	37.44\\
1212	39.94\\
1219	50.01\\
996	36.07\\
1019	45.95\\
1027	36.09\\
1219	60\\
1123	35.5\\
555	30.84\\
661	31.61\\
655	30.99\\
616	30.48\\
706	30.25\\
424	27.02\\
478	30.43\\
513	35\\
610	35.96\\
703	40\\
738	42.44\\
792	49.88\\
772	42.44\\
850	44.94\\
962	42.44\\
946	35\\
881	34.11\\
958	35\\
1004	36.01\\
951	35.42\\
817	35.52\\
808	33.51\\
1044	43\\
895	44.83\\
423	31.49\\
62	30.97\\
181	31.49\\
188	30.68\\
-7	26.71\\
10	18.63\\
139	27.4\\
-217	26.71\\
102	31.01\\
-152	31.44\\
-92	34.09\\
44	35\\
-86	34.31\\
-89	34.94\\
-245	33\\
-390	32.21\\
-354	31.41\\
-11	34.94\\
145	36\\
243	39.94\\
337	40\\
397	32.44\\
488	42\\
391	36\\
130	31.49\\
-262	30.13\\
208	31.01\\
128	17.58\\
114	15.91\\
64	14.59\\
29	13.55\\
24	14.24\\
-658	13.56\\
-490	25.04\\
-488	29\\
-509	30\\
-637	30.55\\
-863	30\\
-977	29.75\\
-998	29.75\\
-894	29.96\\
-627	32.85\\
-222	34.94\\
-25	44.94\\
55	50\\
123	40\\
310	50.5\\
176	45.01\\
-187	31.24\\
-171	30.33\\
289	30.98\\
255	27.4\\
269	20.18\\
-77	23.88\\
319	31.21\\
705	39.01\\
753	35.96\\
947	34.26\\
1107	33.99\\
1066	35.9\\
1219	37\\
1207	38.02\\
1219	39.47\\
1144	39.23\\
1069	38.36\\
1030	41.94\\
1137	47.91\\
1176	47.84\\
1037	43.98\\
706	38.48\\
1219	41.92\\
1219	36.31\\
1065	30.36\\
1219	32.08\\
1133	28.54\\
1064	28.01\\
914	27.87\\
642	29.03\\
892	36.12\\
597	44.9\\
513	48\\
596	49.68\\
755	50.61\\
1072	54.11\\
1050	47.85\\
1219	45.23\\
1219	41.98\\
1219	39.94\\
1219	38.73\\
1219	40.79\\
1219	41.99\\
1142	42.93\\
1103	41.46\\
955	38.86\\
1219	39.82\\
1219	36.02\\
765	30.98\\
297	28.71\\
261	28.23\\
216	26.9\\
223	26.81\\
270	28.87\\
330	35.06\\
222	44.4\\
667	45\\
524	44.81\\
424	43.67\\
349	45.98\\
489	41.2\\
712	40.85\\
913	39.47\\
980	38.91\\
1106	38.41\\
1054	40.45\\
1185	41.68\\
973	43.22\\
681	41.23\\
793	38.86\\
1219	40.63\\
1219	34.96\\
944	31.98\\
788	29.69\\
799	28.85\\
775	27.76\\
708	27.71\\
641	29.11\\
809	36.71\\
659	44.69\\
1016	47.55\\
1219	44.06\\
1219	43.07\\
1219	45.33\\
1219	42.85\\
1219	41.2\\
1219	44.98\\
1219	43.62\\
1219	41.9\\
1219	44.83\\
1219	48.59\\
1200	49.07\\
771	45.59\\
1017	41.96\\
1084	42.49\\
1171	36.74\\
863	32.48\\
438	29.67\\
579	28.99\\
649	28.14\\
539	28.27\\
617	29.08\\
530	35.39\\
620	41.97\\
961	45.41\\
1203	44.55\\
1219	43.79\\
1001	43.04\\
965	40.96\\
990	38.98\\
1216	36.07\\
1029	34.24\\
845	34.94\\
904	37.93\\
739	40.47\\
583	41.97\\
473	38.9\\
718	36\\
928	41.96\\
634	38.54\\
-63	46.71\\
-393	30.33\\
-549	28.98\\
-662	28.02\\
-482	27.15\\
-527	27.07\\
-755	27.99\\
-520	29.99\\
-99	31.78\\
49	45.12\\
105	56.38\\
199	57.97\\
107	54.25\\
-15	42.44\\
-80	38.79\\
-28	35.33\\
-7	36.39\\
260	40\\
342	50\\
-105	40\\
-184	36\\
94	39.96\\
342	48.46\\
107	46.09\\
-271	31.22\\
-587	29.91\\
-811	24.58\\
-769	22.95\\
-335	28.47\\
-526	22.51\\
-529	22.47\\
-617	22.78\\
-409	29\\
-789	30.35\\
-690	35\\
-704	39.04\\
-733	39.94\\
-897	33.97\\
-1098	30\\
-1210	29.73\\
-1246	29.94\\
-973	36\\
-652	35.2\\
-470	35.2\\
-895	33.96\\
-513	37\\
-842	33.98\\
-844	31.37\\
-649	28.08\\
-629	29\\
-626	25\\
-724	24.01\\
-647	24.4\\
-625	26.78\\
-330	34.08\\
-158	41.26\\
-61	41.93\\
-1	42.29\\
-102	43\\
-135	47\\
-170	42.55\\
-305	41.25\\
-336	39.94\\
-276	36\\
-235	33.87\\
19	35\\
512	37\\
533	40.37\\
247	40.27\\
289	36.23\\
621	37\\
248	37\\
131	31.75\\
73	30.54\\
147	30.1\\
66	29.32\\
25	25.97\\
48	26.9\\
210	33.59\\
357	41.06\\
281	40.99\\
53	42.44\\
-109	40\\
-335	39.94\\
-394	35.69\\
-422	36\\
-433	36\\
-328	35.17\\
-253	34.94\\
-21	35.18\\
-67	39\\
-198	41.92\\
-60	40.09\\
-163	38.87\\
432	39.38\\
209	36\\
-212	30.26\\
-219	29.29\\
-12	29.37\\
-137	25.15\\
43	25.81\\
-22	27.97\\
-47	31.63\\
-15	39.91\\
136	39.99\\
263	39.61\\
238	39.13\\
195	40.43\\
238	38.41\\
160	36.96\\
196	36\\
137	34.95\\
42	33.82\\
119	36.08\\
212	38.59\\
130	39.91\\
61	39.02\\
-28	36.9\\
710	38.88\\
701	32.59\\
122	30.03\\
-172	28.3\\
151	29.39\\
113	29.59\\
288	29.52\\
529	26.87\\
315	31.4\\
201	36.92\\
548	38.73\\
683	36.95\\
668	35.46\\
599	36.3\\
521	33.97\\
362	32.45\\
376	33.79\\
185	31.83\\
251	32.44\\
272	36\\
391	38.79\\
14	39.96\\
44	39.3\\
104	38.94\\
441	39.43\\
341	33.12\\
118	29.63\\
-387	26.87\\
-177	25.69\\
-214	25.06\\
-145	24.99\\
-280	26.04\\
142	30.28\\
450	37\\
502	38.5\\
739	37.44\\
684	39.1\\
648	41\\
622	37.44\\
504	37\\
461	38\\
491	37.44\\
505	37.61\\
622	44.98\\
753	36.94\\
610	37.96\\
558	36.93\\
495	32.9\\
785	33.22\\
352	30.15\\
-512	30.34\\
-932	34.8\\
-1052	30.23\\
-1099	29.3\\
-1074	28.75\\
-1148	24.99\\
-1166	24.43\\
-1429	27.6\\
-1501	30\\
-1382	34\\
-1501	30\\
-1501	30.18\\
-1501	30\\
-1467	30.26\\
-1501	20.03\\
-1501	20.31\\
-1501	19.35\\
-1501	28.32\\
-1501	28.33\\
-1501	28.32\\
-1167	33.2\\
-1165	31.6\\
-875	31.85\\
-1072	28.86\\
-1200	47\\
-1501	18\\
-1208	23.55\\
-1454	22.6\\
-1344	22.02\\
-1501	4.79\\
-1501	0.29\\
-1390	22.79\\
-1501	14.21\\
-1297	26.1\\
-1194	28.23\\
-1458	28.89\\
-1460	30.32\\
-1301	32.44\\
-1414	30\\
-1345	28.89\\
-1265	29.94\\
-1132	30.32\\
-862	43\\
-650	47.27\\
-718	46.22\\
-604	40.96\\
-670	42.95\\
-614	31.6\\
-1142	28.23\\
-978	28.25\\
-920	25.35\\
-838	25.35\\
-840	24.14\\
-770	26.02\\
-391	33.09\\
-219	36.93\\
-300	37.16\\
-270	36.94\\
-322	37.26\\
-230	38.13\\
-133	37.98\\
-182	36.97\\
-159	36.19\\
-10	35.29\\
25	36.15\\
-187	37.97\\
125	39.23\\
137	40\\
454	38.91\\
402	36.17\\
1074	36\\
958	29.64\\
239	28.07\\
265	27.55\\
18	25.97\\
231	25.07\\
192	25.08\\
124	25.93\\
33	30.87\\
-76	40.26\\
298	42.98\\
490	43.07\\
547	45.09\\
640	44.99\\
418	42.03\\
344	40.49\\
461	38.55\\
583	36.99\\
593	36.18\\
717	35\\
703	33.98\\
520	33.03\\
518	30.83\\
375	29.64\\
675	29.15\\
396	27.8\\
291	25.22\\
6	23.99\\
217	22.04\\
99	20.92\\
-9	21.7\\
-48	24.2\\
-465	27.61\\
-771	34.95\\
-773	39.08\\
-819	39.75\\
-807	40\\
-645	41.91\\
-591	38.64\\
-648	39.74\\
-424	36.94\\
-500	35.1\\
-466	32\\
-365	34.9\\
-388	35.61\\
-627	35.19\\
-485	33.41\\
-121	32.14\\
416	30.74\\
107	29.11\\
-592	25.07\\
-939	21.85\\
-595	21.04\\
-229	21.37\\
-436	22.95\\
-652	25.56\\
-594	29.4\\
-740	36.46\\
-593	40.18\\
-522	39.91\\
-723	39.68\\
-679	39\\
-446	34.58\\
-402	32.86\\
-392	31.88\\
-174	31.85\\
26	32.77\\
125	34.94\\
21	35.85\\
-168	36\\
-193	33.8\\
92	32.11\\
295	39.99\\
102	29.48\\
-489	25.95\\
-248	25\\
-136	23.84\\
-145	23.24\\
-531	23.59\\
-813	25.86\\
-746	30.09\\
-772	36.74\\
-1012	37.47\\
-1326	36.48\\
-1072	35.94\\
-406	36.27\\
-365	39.1\\
-459	35\\
-643	38\\
-742	35\\
-648	35\\
-544	35\\
-614	33.53\\
-1013	35\\
-976	35.97\\
-680	35.93\\
-333	35.98\\
-449	29.88\\
-2	30.13\\
-388	26.5\\
-534	25.1\\
-489	25.07\\
-730	24.88\\
-665	24.58\\
-643	25.1\\
-477	29.79\\
-593	32.93\\
-458	34.65\\
-283	34.96\\
-306	47.44\\
-319	40\\
-536	34\\
-630	29.94\\
-662	27.55\\
-597	29.94\\
-461	31.53\\
-407	32.86\\
-493	35.96\\
-544	35.98\\
-571	35.91\\
-394	35.97\\
-509	29.89\\
-857	27.72\\
-1097	25.27\\
-1074	24.02\\
-1269	22.69\\
-1124	21.04\\
-1214	19.61\\
-1280	18.09\\
-1205	19.99\\
-1294	23.66\\
-1096	25.62\\
-1089	28.08\\
-1020	34.94\\
-1052	39.94\\
-1274	34\\
-1334	29.89\\
-1421	28.04\\
-1341	28.03\\
-1268	34.94\\
-1073	35\\
-929	35\\
-888	35\\
-742	32.99\\
-808	42.44\\
-860	34\\
-1276	35\\
-1226	27.4\\
-1401	17.57\\
-1350	24.01\\
-1027	26.88\\
-843	26.71\\
-1117	34\\
-1168	41.46\\
-651	41.96\\
-755	40.78\\
-685	38.01\\
-562	39.94\\
-377	33\\
-547	32.44\\
-611	29.94\\
-684	29.7\\
-637	29.94\\
-527	33.97\\
-644	37.5\\
-961	38.87\\
-905	38.08\\
-942	35.93\\
-261	35.91\\
-163	30\\
-624	28.21\\
-915	26.1\\
-723	25.22\\
-798	25.07\\
-879	25.23\\
-630	27.05\\
-1300	29.62\\
-1401	28.24\\
-1180	36.1\\
-894	34.92\\
-401	35.23\\
-216	36.94\\
46	35.93\\
-58	34.92\\
-172	34\\
-317	34.94\\
-323	40\\
-244	42.44\\
-197	37.49\\
-667	39.7\\
-695	39.81\\
-625	38.98\\
-234	38.04\\
200	33.54\\
-165	29.35\\
-298	28.93\\
-225	28.94\\
-301	28.91\\
-164	26.79\\
-496	28.33\\
-49	33.74\\
-366	40.3\\
-205	41.8\\
-332	44.94\\
-419	46.01\\
-580	46.42\\
-844	40\\
-923	42.44\\
-1032	39.94\\
-1010	39\\
-1076	34.94\\
-1083	36.12\\
-611	38.47\\
-867	42.28\\
-834	42.06\\
-677	39.24\\
-290	39.81\\
-6	34.45\\
-376	31.36\\
-455	28.56\\
-694	28.01\\
-286	28.31\\
-327	28.21\\
-198	29.59\\
-604	35.85\\
-602	42.1\\
-583	43.05\\
-818	46.73\\
-929	46.54\\
-1040	41\\
-921	49.67\\
-1200	54.94\\
-1220	50.11\\
-1159	48.99\\
-1048	50\\
-746	50.11\\
-616	42.36\\
-876	42.64\\
-1154	40.65\\
-987	38.94\\
-538	38.89\\
-261	33.15\\
-844	31.09\\
-686	29.44\\
-790	28.4\\
-830	28.09\\
-881	27.69\\
-1246	29.44\\
-1152	35.44\\
-1368	41.75\\
-1282	42.44\\
-1087	40.72\\
-889	39.77\\
-732	38.1\\
-712	36.99\\
-1002	34.26\\
-1328	33.93\\
-1333	34.84\\
-1207	34.94\\
-991	37.93\\
-905	41.16\\
-991	43.03\\
-1087	40.73\\
-1023	36.93\\
-513	39.5\\
-422	37.92\\
-591	34.47\\
-828	31.42\\
-838	30.68\\
-854	30.68\\
-982	29.23\\
-1118	26.24\\
-1501	9.74\\
-1501	14.54\\
-1501	22.03\\
-1501	27.3\\
-1501	29.32\\
-1501	27.31\\
-1401	26.41\\
-1401	23.27\\
-1401	23.94\\
-1501	26\\
-1501	26.02\\
-1399	30.91\\
-986	35\\
-869	35.02\\
-741	37.44\\
-838	32.35\\
-683	46.7\\
-631	46.92\\
-801	47.76\\
-928	32.87\\
-818	30.24\\
-850	21.62\\
-835	18.42\\
-881	15.67\\
-1019	16.76\\
-1059	20.38\\
-1375	19.96\\
-1241	31.13\\
-1065	35\\
-1215	41.63\\
-1445	35\\
-1386	35\\
-1441	33\\
-1489	33.75\\
-1464	34.02\\
-1427	34.94\\
-1186	45\\
-1047	40\\
-961	42.44\\
-871	39.1\\
-883	37.99\\
-806	31.16\\
-935	34.94\\
-1023	23.86\\
-1055	22.81\\
-1094	22.99\\
-1213	22.69\\
-1310	15.17\\
-1310	30.6\\
-1310	24.77\\
-1310	31.35\\
-1310	37.19\\
-1310	42.86\\
-1310	44.29\\
-1310	43.51\\
-1310	41.78\\
-1310	38.44\\
-1310	35.96\\
-1310	34.14\\
-1310	33.95\\
-1310	35.52\\
-1310	35.92\\
-904	32.87\\
-881	30.33\\
-956	29.1\\
-902	29.04\\
-1141	28\\
-1212	26.17\\
-1310	21.92\\
-1219	25.09\\
-1310	11.67\\
-1310	19.67\\
-1310	29.73\\
-1224	37.92\\
-1084	42.69\\
-963	41.95\\
-988	42.92\\
-1068	42.97\\
-966	38.04\\
-1127	35.03\\
-1005	33.65\\
-1046	32.39\\
-983	32.18\\
-560	35\\
-391	35.97\\
-435	37.94\\
-349	35\\
-279	33.07\\
-162	35.95\\
-339	31.88\\
-834	29.98\\
-890	29.05\\
-703	25.81\\
-612	25.32\\
-677	25.94\\
-690	27.3\\
-858	30.9\\
-765	38.67\\
-772	38.07\\
-641	36.5\\
-701	35.94\\
-808	36.67\\
-796	34\\
-818	34.23\\
-889	32.23\\
-1052	32.44\\
-899	34.03\\
-663	44.94\\
-444	36.4\\
-534	39.08\\
-379	35.93\\
-269	32.88\\
-189	44.92\\
-404	34\\
-813	29.63\\
-949	26.64\\
-803	25.79\\
-643	25.05\\
-673	25.62\\
-808	27.32\\
-710	29.62\\
-652	36.49\\
-581	36.9\\
-400	35.45\\
-351	35.23\\
-539	36.41\\
-458	36.97\\
-554	36.55\\
-548	35\\
-490	34.99\\
-688	35.4\\
-572	39.94\\
-509	39.93\\
-659	41.36\\
-646	40.07\\
-449	39.95\\
-296	40.34\\
-118	40.5\\
-509	31.13\\
-810	30.14\\
-707	30.45\\
-577	30.14\\
-477	30.14\\
-556	28.85\\
-610	32.66\\
-736	40\\
-582	41.42\\
-626	46.03\\
-731	48.96\\
-785	50.01\\
-786	48.96\\
-748	49.99\\
-690	48.96\\
-485	46.46\\
-454	45.86\\
-397	39.99\\
-264	38.87\\
-373	38.16\\
-505	36.12\\
-492	35.01\\
-380	36.91\\
-588	35\\
-492	36.5\\
-799	32.09\\
-1056	30.11\\
-973	27.98\\
-804	27.99\\
-859	27.67\\
-991	29.31\\
-1126	30.1\\
-722	32.07\\
-646	36\\
-665	40\\
-751	46.56\\
-848	40\\
-919	36\\
-1018	32.46\\
-996	32.44\\
-962	31.91\\
-752	32.96\\
-755	35.91\\
-694	38.05\\
-601	38.9\\
-591	38.1\\
-524	47.03\\
-469	36\\
-739	31.04\\
-901	29.23\\
-952	28.56\\
-977	27.95\\
-1027	27.12\\
-1057	26.83\\
-1100	26.91\\
-1137	26.04\\
-1310	26.07\\
-1191	29.17\\
-1310	17.03\\
-1310	19.28\\
-1310	22.48\\
-1310	18.59\\
-1310	15.91\\
-1310	15.73\\
-1310	13.87\\
-1240	32.46\\
-952	34.94\\
-865	37.11\\
-635	36\\
-623	44.11\\
-764	48.96\\
-848	37.11\\
-838	36\\
-944	30.64\\
-771	29.08\\
-814	27.67\\
-760	27.84\\
-830	28.97\\
-674	36.98\\
-225	41.97\\
-218	43\\
-168	41.28\\
-120	39.99\\
-204	47\\
-229	42.44\\
-203	42.44\\
-102	40.23\\
-53	37.44\\
-11	37.94\\
-122	41.02\\
-153	41.58\\
-311	42.95\\
-439	39.93\\
-398	35.99\\
-215	36\\
-389	30.44\\
-778	29.73\\
-972	28.58\\
-1128	11.2\\
-1128	5.23\\
-1128	25.85\\
-792	29.31\\
-1069	32.48\\
-919	39.41\\
-911	42.5\\
-966	42.31\\
-891	42.5\\
-942	42.5\\
-802	42.5\\
-1024	42.5\\
-1128	36\\
-1104	34.97\\
-1028	34.94\\
-1077	36.48\\
-917	36.5\\
-876	35.78\\
-340	35.94\\
-331	36\\
-67	48.97\\
-147	34.1\\
-475	33.1\\
-518	33.7\\
-491	33.7\\
-532	31.97\\
-495	31.57\\
-560	31.54\\
-533	32.17\\
-573	35.98\\
-697	41.03\\
-809	40.59\\
-800	39.98\\
-804	39.99\\
-845	37.44\\
-907	38\\
-862	38\\
-792	38\\
-746	38\\
-585	48.96\\
-424	50\\
-423	44.94\\
-417	39.94\\
-387	34.1\\
-252	49.94\\
-512	44.58\\
-503	32.59\\
-925	32.68\\
-825	27.19\\
-684	31.41\\
-698	30\\
-517	31.71\\
-493	32.42\\
-281	36.93\\
-299	38\\
-287	34.94\\
-315	38\\
-527	34.94\\
-483	34.71\\
-555	35.81\\
-556	38\\
-597	39.94\\
-501	43.05\\
-560	44.94\\
-347	35\\
-446	38\\
-233	40\\
-74	35.97\\
122	41.15\\
-148	38\\
-216	31.93\\
-539	31.62\\
-387	31.58\\
-478	31.57\\
-414	31.61\\
-494	31.64\\
-449	37.78\\
-494	40.3\\
-435	39.96\\
-486	44.03\\
-655	48.97\\
-867	46.44\\
-593	39.99\\
-745	38.74\\
-779	37.44\\
-752	35.02\\
-816	42.97\\
-701	47.51\\
-531	46.44\\
-466	45.26\\
-274	48.9\\
40	48.77\\
172	57.44\\
-156	44.13\\
-362	32.03\\
-520	32.03\\
-748	34.77\\
-799	32.03\\
-742	31.67\\
-794	29.99\\
-893	27.27\\
-781	29.06\\
-776	29.94\\
-896	38.1\\
-1103	41\\
-1086	44.94\\
-1128	30.01\\
-1091	42.44\\
-1078	42.44\\
-966	37.73\\
-855	37.68\\
-769	41.45\\
-754	49\\
-649	39.78\\
-385	45.63\\
-277	44.58\\
-202	49.28\\
-376	39.78\\
-835	31.31\\
-1087	28.48\\
-1110	28.09\\
-1128	29.2\\
-1078	28.91\\
-1100	27.15\\
-1128	10.59\\
-1128	10.9\\
-1128	12.93\\
-1128	13.47\\
-1128	13.73\\
-1128	14.92\\
-1128	15.73\\
-1128	14.6\\
-1128	14.44\\
-1128	13.9\\
-1128	13.5\\
-1128	14.51\\
-1128	17.39\\
-1128	30.79\\
-1124	38\\
-1069	46.48\\
-1020	52.44\\
-1128	39.8\\
-1128	27.59\\
-1128	27.59\\
-1128	17.91\\
-1128	27.56\\
-1125	29.4\\
-1026	29.56\\
-770	32.6\\
-764	38.09\\
-532	37.4\\
-397	38.7\\
-274	39.5\\
-281	43.4\\
-409	39.41\\
-524	38\\
-585	35.4\\
-607	35\\
-586	35\\
-346	40\\
-419	36.94\\
-464	38.3\\
-214	38.09\\
-63	39.94\\
63	47.44\\
-77	32.73\\
-479	29.24\\
-880	29.14\\
-946	29.19\\
-852	29.93\\
-808	29.95\\
-572	29.95\\
-525	32.58\\
-214	38.6\\
-415	41.04\\
-464	45.08\\
-595	49.69\\
-1053	45.11\\
-995	47\\
-1074	45.19\\
-1057	44.94\\
-952	39.02\\
-845	42\\
-583	52.44\\
-296	39.49\\
-285	39.93\\
-249	38.99\\
-145	38.89\\
42	35\\
129	30.75\\
-480	28.95\\
-743	28.67\\
-822	28.22\\
-983	28.08\\
-986	27.78\\
-945	28.18\\
-888	30.76\\
-573	37.73\\
-241	40.19\\
-462	36.82\\
-540	37.44\\
-538	40\\
-729	36.82\\
-682	34.94\\
-798	32.65\\
-654	31.22\\
-647	32.44\\
-538	33.01\\
-509	37.9\\
-653	39.37\\
-708	38.76\\
-633	37.78\\
-410	32.75\\
-332	30.35\\
2	31.21\\
-152	28.64\\
2	28.41\\
5	29.06\\
93	29.04\\
24	28.5\\
259	30.61\\
279	35.62\\
2	37.95\\
291	41\\
149	47\\
-92	44.94\\
-67	39.94\\
-239	39.94\\
-254	39.5\\
-195	37\\
-120	42.44\\
-70	47.55\\
138	46.44\\
-94	39.22\\
265	38.63\\
362	38.3\\
496	47\\
262	39.99\\
445	33.71\\
592	31.2\\
550	30.44\\
451	30.45\\
482	30\\
486	29.94\\
599	39.97\\
350	44\\
122	42.03\\
146	44.44\\
-83	42.83\\
-137	44.99\\
-133	42.44\\
85	59.94\\
112	52.63\\
182	52.63\\
27	52.63\\
-253	52.27\\
-242	44.94\\
-120	41.07\\
-322	36.87\\
-234	34.53\\
-129	32.8\\
-90	38.16\\
178	47.52\\
-156	30\\
-190	27.44\\
-351	27.17\\
-415	25.3\\
-421	24.6\\
-469	24.1\\
-272	26.03\\
-160	29.98\\
-12	34\\
12	34.94\\
-18	35\\
-160	33.1\\
-401	31.9\\
-455	30.62\\
-455	30.6\\
-293	32.7\\
18	35\\
91	34.96\\
335	39.94\\
196	35.66\\
267	43.78\\
291	47.44\\
369	42.44\\
452	34.41\\
120	34\\
-137	34.41\\
-336	30.43\\
-465	29.68\\
-412	27.27\\
-416	26.18\\
-373	27.2\\
-158	28.03\\
-168	29.94\\
-181	32.44\\
-130	33.7\\
-236	33\\
-442	32.44\\
-524	30.43\\
-707	29.93\\
-698	29.94\\
-585	32.44\\
-526	32.7\\
-331	33\\
-187	32.7\\
81	43.26\\
67	47.36\\
-117	29.19\\
-705	26.17\\
-946	13.45\\
-946	8.99\\
-946	5.37\\
-912	21.95\\
-608	24.8\\
-96	29.4\\
350	31.97\\
538	34.96\\
343	32.9\\
263	32.96\\
198	34.94\\
210	35\\
114	35.37\\
148	33\\
240	33.7\\
391	35.7\\
347	37.87\\
549	34.94\\
512	37.27\\
571	36.91\\
648	38.32\\
999	35.7\\
551	29.94\\
-19	30.85\\
-27	29.87\\
-151	26\\
-133	25.1\\
145	27.48\\
-133	26.96\\
313	35.01\\
185	41.05\\
219	39.18\\
268	38.28\\
212	35.12\\
298	37.2\\
83	33.3\\
-19	34.94\\
35	34.94\\
84	34.94\\
182	37.2\\
305	43\\
514	41\\
169	41.1\\
282	42.44\\
397	39.86\\
634	40\\
570	37.2\\
449	30.86\\
241	30.12\\
506	30.54\\
516	30.45\\
345	30.05\\
612	30.56\\
553	33.95\\
530	38.19\\
751	39.98\\
627	41.78\\
456	44.8\\
334	48\\
113	42.11\\
34	42.98\\
-39	41.97\\
-31	41.26\\
1	40.1\\
63	45.51\\
-33	42.97\\
91	39.97\\
117	37.23\\
513	38.65\\
824	37\\
562	34\\
582	30.82\\
666	30.86\\
589	29.86\\
546	29.08\\
632	28.97\\
716	29.85\\
955	33.99\\
995	39.03\\
612	40.63\\
821	37.27\\
816	42.44\\
821	42.44\\
590	39.94\\
473	39.94\\
472	39.94\\
487	39.94\\
529	42.44\\
523	47.44\\
743	39.94\\
578	41\\
723	41.08\\
844	40.97\\
952	37.43\\
1006	35.99\\
966	31.15\\
689	31.21\\
506	30.3\\
424	29.78\\
378	29.79\\
122	29.92\\
-9	31.36\\
-21	36.18\\
112	35.91\\
97	37.97\\
149	40.07\\
198	41.05\\
139	37.97\\
-7	36.4\\
-270	35\\
-396	33.17\\
-358	32.54\\
-78	34.99\\
245	37.91\\
442	40\\
610	40\\
566	37.97\\
637	42.44\\
321	39.94\\
208	46.95\\
0	32.5\\
15	32.27\\
64	31.9\\
53	30.96\\
128	31.64\\
-127	29.35\\
-79	31.9\\
169	34.99\\
-34	45\\
-51	49.94\\
-104	49.94\\
-253	45.12\\
-427	37.44\\
-614	32.2\\
-729	31.9\\
-657	31.72\\
-384	32.5\\
-135	40\\
-138	44\\
-304	41.11\\
-368	47.19\\
-212	50.56\\
-297	32.5\\
-293	30.96\\
-622	29.64\\
-924	28.03\\
-912	27.08\\
-946	5.51\\
-946	8.06\\
-909	18.23\\
-920	24.33\\
-739	25\\
-897	31.31\\
-718	29.96\\
-652	30.11\\
-680	31.9\\
-929	30.97\\
-933	30.12\\
-946	17.44\\
-931	31.49\\
-873	32.92\\
-634	39.96\\
-539	47\\
-827	44.94\\
-700	40\\
-674	39.94\\
-468	32.45\\
-703	41.99\\
-656	28.51\\
-608	28.07\\
-656	28.09\\
-638	28.28\\
-647	30.03\\
-179	33.24\\
-517	35\\
-394	40\\
-490	43.78\\
-506	41.89\\
-679	45.85\\
-831	41.76\\
-791	48.53\\
-662	44.72\\
-508	42.22\\
-340	42.8\\
-76	47.44\\
-25	43.78\\
-119	43.78\\
-212	45\\
-194	48.97\\
-208	48.53\\
-106	48.53\\
611	38.38\\
540	36.11\\
629	34.28\\
714	33.45\\
681	33.45\\
737	33.59\\
751	33.96\\
746	35.6\\
977	40.64\\
721	52.44\\
577	53.89\\
564	55.69\\
578	52.44\\
241	40\\
504	44.96\\
512	39.99\\
593	39.65\\
579	35.03\\
1043	38\\
1056	39.94\\
1056	47.44\\
960	37.18\\
907	35.29\\
686	30.63\\
587	31.22\\
507	29.94\\
623	29.46\\
627	29.46\\
663	31.43\\
977	32.18\\
871	33.36\\
582	40\\
368	39.5\\
352	52.44\\
105	52.44\\
49	55.04\\
-365	39.94\\
-85	49.99\\
-331	39.94\\
-165	40.78\\
22	44.94\\
259	58.25\\
428	38.77\\
400	42.33\\
338	41.05\\
902	45\\
848	44.96\\
828	39.5\\
734	34.01\\
718	32.66\\
754	32.11\\
656	27.96\\
619	27.79\\
722	29.16\\
797	39.37\\
1056	66.87\\
1008	43.35\\
356	52.2\\
217	55\\
170	59.3\\
266	55\\
-11	40.17\\
652	48\\
675	43.78\\
713	44.31\\
540	45\\
983	38.71\\
741	41.02\\
704	41.01\\
553	39.98\\
570	37.09\\
614	32.94\\
301	29.92\\
240	28.99\\
262	28.47\\
319	28.05\\
313	28.09\\
346	29.37\\
223	32.82\\
153	38.91\\
796	41.43\\
463	38.2\\
436	35.7\\
460	37.44\\
286	34.96\\
529	39.37\\
395	37.75\\
513	38.3\\
605	39.94\\
531	44.94\\
483	43.78\\
514	40\\
688	40\\
766	42.5\\
1025	39.94\\
691	40\\
257	37.3\\
237	39.97\\
285	38.3\\
221	32.7\\
171	32.04\\
170	33.37\\
72	34.94\\
-152	33.37\\
36	34.99\\
-125	35\\
-137	40\\
-275	39.38\\
-352	34.94\\
-288	33.37\\
-365	34.99\\
-532	32.54\\
-422	32.72\\
-114	39.94\\
291	35.09\\
391	35\\
600	40\\
409	44.94\\
351	44.73\\
512	48.75\\
204	50.28\\
-180	39.45\\
-38	36.94\\
-129	32.7\\
-83	32.43\\
-96	32.42\\
-103	32.7\\
-271	31.74\\
-289	32.43\\
-372	33.59\\
-476	39.36\\
-437	42.44\\
-517	48\\
-733	40.5\\
-799	37.44\\
-813	35\\
-670	37.12\\
-388	44.94\\
-114	39.94\\
-244	34.94\\
-1	39.94\\
371	57.44\\
386	50.23\\
486	47.06\\
-177	39.38\\
-96	33.87\\
-56	32.83\\
39	29.3\\
-8	29.42\\
-45	30.13\\
-31	37.91\\
-244	42\\
-273	40.81\\
-19	43.24\\
197	49.45\\
-82	60\\
-147	64.44\\
-264	64.35\\
-24	48.96\\
138	43.52\\
106	43.29\\
-361	46.39\\
-429	54.79\\
-617	44.01\\
-535	44.52\\
-261	40.45\\
57	38.37\\
137	37.26\\
154	32.28\\
167	32.01\\
-32	31.69\\
-33	32.86\\
76	33.29\\
401	33.54\\
383	36.46\\
-15	43.9\\
-208	48.86\\
72	51.98\\
96	53.59\\
89	54.27\\
36	48.96\\
-135	47.17\\
-133	44.08\\
-55	42.94\\
-115	42.09\\
-29	42.28\\
70	43.74\\
368	42.59\\
524	41.97\\
304	45\\
591	45.67\\
654	35.99\\
474	32.49\\
439	32.6\\
420	32.5\\
534	32.37\\
547	32.64\\
568	32.96\\
420	35.99\\
85	41.99\\
185	45.03\\
79	49.5\\
15	48\\
-57	49.94\\
-100	47.44\\
-156	48.96\\
-192	48.75\\
-124	43.27\\
-81	42.13\\
5	50.95\\
237	47.44\\
206	43.6\\
-5	42.02\\
9	40.65\\
276	40\\
-29	37.33\\
377	38.91\\
150	32.55\\
-73	31.05\\
-46	29.47\\
-72	27.9\\
169	31.84\\
283	35.77\\
305	44\\
144	41.63\\
-86	44.96\\
-56	49.41\\
-54	51.42\\
-226	44\\
-295	46\\
-310	46\\
100	48.96\\
68	49.41\\
-326	49.41\\
-79	46\\
-201	41.99\\
-48	46\\
-82	51.42\\
143	39.99\\
72	44.59\\
-129	35.14\\
-56	32.31\\
155	32.71\\
74	31.14\\
154	31.5\\
88	29.96\\
360	37.46\\
504	44.5\\
690	47.81\\
219	54.44\\
116	55.84\\
62	54.94\\
-47	45\\
-241	39.94\\
-332	38.73\\
-568	37.46\\
-666	35.88\\
-567	37.44\\
-211	38\\
-57	37.71\\
-74	37.52\\
-382	36.72\\
607	40\\
580	35.14\\
522	36.32\\
238	33\\
-51	29.78\\
-134	28.67\\
-159	25.1\\
-130	27.7\\
-72	32.99\\
-17	34\\
187	34.96\\
379	44.96\\
473	47.44\\
540	49.5\\
511	45\\
398	38.49\\
308	36.25\\
345	34.96\\
353	35\\
361	39.94\\
500	42.44\\
592	44.96\\
371	51.99\\
350	54.94\\
472	49.94\\
332	42.48\\
419	35.03\\
266	34\\
477	33.9\\
358	31.41\\
314	31.31\\
326	31.27\\
372	31.38\\
240	31.74\\
245	32.15\\
429	34.06\\
318	34.9\\
345	35.05\\
179	35.02\\
59	34.9\\
-130	32.64\\
-236	32.44\\
-209	32.4\\
-5	34.9\\
229	36.3\\
647	45\\
489	52.44\\
246	52.44\\
495	50\\
604	42.68\\
383	34.89\\
372	33.64\\
312	34.22\\
476	34.47\\
597	34.88\\
525	34.89\\
954	41.03\\
757	41.03\\
858	43.42\\
738	44.94\\
648	43.76\\
606	47.44\\
658	50.44\\
697	47.44\\
745	44.94\\
784	44.94\\
845	48.96\\
808	44\\
946	40.21\\
1056	50\\
1019	44.94\\
1022	48.96\\
1056	50.44\\
1056	44.99\\
671	37.88\\
841	38.35\\
827	37.55\\
793	37.08\\
734	37.07\\
906	36.81\\
754	41.2\\
1056	74.36\\
851	55.39\\
452	59.94\\
17	62.29\\
-95	63.66\\
-103	59.96\\
-170	60.01\\
38	58.2\\
35	55.24\\
30	52.21\\
94	64.94\\
473	42.44\\
726	42.21\\
981	53.11\\
933	45.2\\
1056	52\\
929	47.44\\
1054	49.99\\
1056	41\\
949	38.96\\
974	38.55\\
981	38.62\\
903	37.81\\
1036	42.54\\
1045	57\\
856	57.86\\
861	60.5\\
572	58.9\\
444	55.78\\
512	50\\
384	50\\
313	47.82\\
319	52.07\\
407	55.78\\
494	69.9\\
691	41\\
790	39.96\\
1056	55.78\\
947	55.78\\
1056	60.39\\
1048	42.44\\
909	37.18\\
919	35.96\\
857	35.5\\
843	35.03\\
915	35.03\\
1056	35.94\\
1056	39.3\\
747	47.44\\
328	45.68\\
269	54.7\\
94	44.94\\
241	55.78\\
58	44.96\\
117	50\\
217	49.94\\
320	47.03\\
445	54.96\\
298	55\\
861	55.31\\
1056	50\\
977	52.44\\
872	44.12\\
1056	39.97\\
1056	49.99\\
890	39.5\\
1056	39.94\\
1006	35.28\\
1056	35.28\\
1056	37.5\\
1056	50\\
946	38.26\\
1056	55.78\\
928	49.48\\
883	60.75\\
757	63.99\\
664	61\\
503	55\\
538	60\\
582	54.94\\
557	49.05\\
593	47.44\\
650	55.83\\
760	54.94\\
730	42.36\\
813	47.44\\
917	45.86\\
1056	55\\
989	46.9\\
948	60.12\\
1028	49.99\\
1171	46.01\\
1028	42.44\\
1072	39.92\\
1054	39.46\\
727	41.49\\
627	43.07\\
831	45.71\\
840	49.91\\
907	60\\
867	58.69\\
832	52.44\\
778	39\\
925	37.81\\
833	36.57\\
882	36.57\\
1036	39.81\\
982	39.94\\
1014	39.94\\
1052	55\\
976	49.99\\
1055	47.44\\
985	40\\
766	37.13\\
638	36.29\\
976	36.29\\
1007	31.97\\
912	31.74\\
932	31.68\\
1020	32.09\\
416	31.83\\
476	32.56\\
761	36.91\\
632	39.94\\
732	39.99\\
613	40\\
396	40\\
489	37.07\\
446	36.91\\
454	36.53\\
541	42.44\\
643	49.78\\
1000	48.19\\
823	59.94\\
730	57\\
1050	46\\
1089	36.67\\
724	31.9\\
870	30.99\\
1034	30.83\\
1048	30.8\\
871	31.34\\
779	32.58\\
1056	54.99\\
1056	60\\
1056	53.11\\
1056	53.11\\
1056	53.11\\
1056	43.21\\
1056	43.53\\
1056	42.77\\
1056	45.13\\
1056	45.07\\
1056	45.54\\
1056	43.93\\
1056	54\\
1056	54\\
1056	65.15\\
1056	70\\
1056	57.11\\
1050	36.92\\
1056	35.45\\
1056	39\\
1056	41.42\\
1056	39\\
1056	35.47\\
1056	35.47\\
880	37.95\\
579	45.76\\
435	50.95\\
169	54\\
603	53.75\\
643	53.1\\
304	45\\
372	44.94\\
599	48\\
632	45.9\\
711	44.19\\
843	60\\
1048	45.38\\
1046	45.9\\
850	53.78\\
743	41.89\\
991	44.17\\
874	37.17\\
687	35.24\\
348	31.53\\
376	31.33\\
441	31.27\\
419	31.73\\
658	32.61\\
912	43.04\\
946	57.11\\
946	57.11\\
946	46.17\\
907	44.01\\
728	45\\
883	42.45\\
480	50\\
422	45\\
435	43.24\\
699	46.09\\
617	42.98\\
946	55.1\\
946	46.24\\
946	56.23\\
946	42.88\\
946	72.09\\
946	46\\
928	35.39\\
946	34.66\\
884	31.55\\
871	31.51\\
946	34.71\\
946	34.91\\
946	56\\
670	45.96\\
817	47.67\\
946	65.4\\
946	45.85\\
925	45.96\\
906	44.42\\
725	42.47\\
721	41.05\\
757	39.66\\
710	38.48\\
833	41.46\\
946	45.91\\
946	47.1\\
946	46.18\\
946	42.64\\
946	37.8\\
946	36.29\\
906	32.54\\
691	32.21\\
766	32.14\\
803	32.2\\
737	32.22\\
847	32.46\\
946	40.52\\
882	44.94\\
868	47.39\\
696	47.8\\
749	47.32\\
740	47.09\\
625	46.24\\
677	42.91\\
609	39.34\\
801	38.1\\
406	35.85\\
379	42\\
512	39.38\\
767	40.97\\
552	50\\
699	43.74\\
651	50\\
390	44\\
463	36.27\\
422	34.34\\
445	34.09\\
387	32.81\\
274	32.77\\
229	32.84\\
186	33.96\\
98	34.29\\
170	41.52\\
145	44.94\\
96	45\\
-25	40\\
-114	37.93\\
-34	37.85\\
-127	35.2\\
-177	34.94\\
-87	35.18\\
95	38.52\\
96	43\\
106	49.94\\
-40	59.94\\
-40	50\\
79	40.88\\
23	37.99\\
58	34.33\\
-36	32.7\\
-84	31.95\\
215	32.18\\
288	32.22\\
343	32.11\\
432	32.51\\
224	32.71\\
236	33.38\\
352	37.44\\
337	40\\
413	44.94\\
323	49.99\\
48	38.5\\
-71	36.87\\
-147	34.96\\
-67	35.87\\
88	39.94\\
303	38.5\\
561	49.11\\
464	59.94\\
461	49.94\\
558	48\\
524	46\\
486	47.46\\
302	36.87\\
237	36.65\\
176	35.41\\
208	35.23\\
396	35.23\\
463	43.27\\
946	45.01\\
946	47.92\\
946	48.37\\
946	49\\
875	49.2\\
921	47.75\\
786	48.95\\
622	48.56\\
632	46.74\\
542	59.58\\
833	69.94\\
946	53.11\\
946	49.95\\
946	48.41\\
946	41.48\\
946	41.93\\
946	38.43\\
609	34.99\\
781	33.44\\
685	33.24\\
584	33.24\\
676	33.57\\
736	34.05\\
946	45.01\\
946	74.87\\
946	57.11\\
946	57.11\\
946	57.11\\
946	50.55\\
946	47.23\\
827	47.44\\
822	46.91\\
433	57.11\\
618	49.94\\
815	57.05\\
946	57.11\\
946	57.11\\
946	74.95\\
946	60\\
946	57.11\\
946	40\\
615	37.56\\
701	33.14\\
679	32.5\\
733	32.53\\
735	33.44\\
897	34.01\\
946	58.13\\
946	55.78\\
946	69.21\\
946	71\\
946	55.78\\
939	54.94\\
884	52.5\\
531	57.04\\
446	57.05\\
415	57.04\\
389	57.04\\
534	57.57\\
932	47.74\\
946	55.78\\
946	55.78\\
923	53.12\\
913	46.46\\
780	45.65\\
764	37.61\\
946	34.21\\
946	34.21\\
946	34.21\\
946	37.61\\
946	37.61\\
946	42\\
946	49.71\\
946	55.78\\
946	55.78\\
832	57.04\\
946	55.78\\
946	48.7\\
946	47.95\\
946	47.93\\
946	47.31\\
946	45.74\\
946	55.78\\
946	55.78\\
946	55.78\\
946	55.78\\
946	55\\
946	55.12\\
946	42\\
946	70\\
946	37.95\\
946	37.95\\
946	37.95\\
946	37.95\\
946	42\\
946	60\\
946	51.78\\
946	51.78\\
946	51.78\\
769	58.95\\
779	62\\
683	48.51\\
688	49.94\\
830	57.04\\
806	46.38\\
890	50\\
874	57.04\\
946	51.78\\
946	51.78\\
946	60.01\\
919	46.1\\
819	52.44\\
901	59.94\\
946	53\\
946	37.85\\
946	39.01\\
946	33.59\\
922	33.37\\
946	36.57\\
946	39.99\\
946	38.84\\
946	46.44\\
945	57.05\\
676	53\\
653	53\\
504	48.24\\
457	42\\
618	40.24\\
550	38.16\\
529	38\\
564	46.44\\
711	52.44\\
701	57.05\\
729	69.96\\
667	52.44\\
915	44.94\\
946	43.54\\
700	42\\
792	36.63\\
924	35.18\\
888	32.31\\
828	32.05\\
814	31.92\\
946	34\\
479	32.33\\
546	31.7\\
688	35.28\\
732	34.17\\
735	34.99\\
701	37.08\\
463	33.27\\
37	32.82\\
-17	33\\
-5	32.69\\
426	35.18\\
646	42\\
859	51\\
764	60\\
847	51\\
912	42\\
858	35.47\\
367	36.63\\
562	36.63\\
552	33.73\\
529	35.18\\
633	35.59\\
812	33.73\\
946	44.99\\
946	76.99\\
946	54\\
946	54\\
946	49.31\\
946	50.47\\
946	51.24\\
946	49.14\\
946	48.76\\
946	47.95\\
708	47.34\\
793	46.66\\
946	54\\
875	54\\
669	70\\
921	54\\
946	44.58\\
810	45\\
946	59.99\\
946	45\\
946	38.4\\
946	38.4\\
946	38.4\\
946	44.99\\
946	72.2\\
946	74.92\\
946	60.27\\
937	52.44\\
920	50.44\\
847	47.49\\
926	44.06\\
760	42.56\\
815	43\\
660	45.94\\
776	45\\
946	60.22\\
946	72.2\\
946	69.3\\
946	64.94\\
946	47.33\\
946	65\\
946	76.66\\
946	40.43\\
946	37.1\\
797	33.3\\
753	33.3\\
802	33.14\\
899	33.51\\
946	42\\
946	52.43\\
946	54.01\\
946	57.78\\
946	55.49\\
946	57.84\\
946	51.56\\
850	53.65\\
946	50.26\\
946	48.26\\
946	47.05\\
946	46.91\\
946	59.17\\
946	59.92\\
946	52.1\\
946	48.36\\
946	48.43\\
946	44.38\\
946	47.7\\
946	37.87\\
780	32.61\\
748	32.37\\
879	32.9\\
927	35.76\\
946	51.7\\
820	55.5\\
946	61\\
913	69.32\\
751	60.21\\
946	57.33\\
946	49.95\\
946	50.07\\
946	48.5\\
946	47.78\\
946	46.05\\
946	44.89\\
946	51.78\\
946	53.93\\
946	74.57\\
748	49.96\\
684	49.94\\
946	45.03\\
467	44.16\\
716	37.87\\
844	37.87\\
736	35.6\\
806	34.29\\
762	34.33\\
872	40.06\\
946	52.48\\
946	52.37\\
831	57.04\\
740	59.94\\
738	56\\
721	52.13\\
764	53.44\\
839	48.5\\
946	48.5\\
946	48.5\\
866	52.06\\
946	50.44\\
946	62.13\\
825	57.44\\
577	55\\
808	50.44\\
946	48.5\\
467	59.46\\
365	45.91\\
393	40.36\\
556	38.87\\
607	38.87\\
556	38.87\\
287	40.11\\
364	46.93\\
583	50.84\\
783	55\\
723	59.94\\
684	57.05\\
603	50.44\\
464	42.44\\
265	39.99\\
249	39.55\\
381	40.69\\
688	48.76\\
768	55\\
816	61\\
901	61\\
946	50.44\\
780	43.95\\
510	41.43\\
412	43\\
465	35.3\\
429	35.24\\
390	33.95\\
346	32.67\\
404	32.35\\
429	33.32\\
154	32.35\\
274	37.67\\
259	43.4\\
233	47.44\\
156	52.44\\
28	51.03\\
-190	41.12\\
-197	37.83\\
-190	36.21\\
-6	36.53\\
303	44.94\\
641	53.95\\
718	59.92\\
858	75\\
816	59.94\\
855	50\\
611	42.44\\
99	32.74\\
208	32.67\\
306	32.74\\
400	32.23\\
348	32.61\\
464	33.47\\
882	44.95\\
946	53.97\\
946	57.54\\
807	55.88\\
723	50.12\\
638	51.06\\
687	51.67\\
480	51.78\\
427	48.23\\
661	47.5\\
544	48.34\\
835	60.01\\
946	60\\
874	63.24\\
740	57.05\\
787	47.5\\
946	47.5\\
862	40.7\\
696	36.51\\
779	35.54\\
946	36.57\\
946	37.6\\
946	37.6\\
888	34.15\\
912	44.49\\
946	54.76\\
946	58.41\\
946	56.6\\
946	54.56\\
946	55.93\\
946	52\\
946	50.6\\
946	48.59\\
946	46.31\\
946	48.67\\
946	50.97\\
946	53.11\\
946	66.93\\
946	64\\
946	52\\
946	52\\
946	43\\
946	46.44\\
946	38.09\\
935	32.01\\
937	31.58\\
934	32.13\\
862	35.71\\
946	48.11\\
946	58\\
946	60.88\\
946	65\\
946	65.08\\
946	65.19\\
946	63.65\\
946	64.81\\
946	60.9\\
946	58\\
946	52.7\\
946	54.74\\
946	56.75\\
946	69.45\\
946	70.98\\
946	50.59\\
946	54\\
946	48\\
0	38.29\\
0	40.81\\
0	38.29\\
0	38.29\\
0	39.88\\
0	40.46\\
0	54.3\\
0	59.98\\
0	59.64\\
0	55.15\\
0	54.3\\
0	54.3\\
0	55.98\\
0	54.37\\
0	56\\
0	56\\
0	54.3\\
0	70\\
0	54.3\\
0	63.38\\
0	61.04\\
0	54.3\\
0	54.3\\
0	54.3\\
946	50.98\\
946	45\\
946	41.47\\
946	41.34\\
946	41.52\\
946	49\\
946	50\\
946	56.03\\
946	56.42\\
946	59.53\\
946	59.88\\
946	60.43\\
946	60.01\\
946	56.48\\
946	54\\
946	50.27\\
946	49.47\\
946	51.88\\
946	54\\
946	74.6\\
946	79.03\\
946	54\\
946	56.96\\
946	55.94\\
946	38.65\\
923	37.59\\
946	38.69\\
946	33.2\\
905	31.35\\
764	30.6\\
291	30.83\\
505	34.02\\
250	38.19\\
248	44.42\\
212	42.44\\
120	40.95\\
252	41.86\\
346	40.6\\
136	38.41\\
-56	37.9\\
92	38.16\\
360	44.94\\
455	52.44\\
223	64.94\\
160	53.11\\
221	48.8\\
230	46.5\\
583	44.78\\
63	38.91\\
-70	36.45\\
-209	32.43\\
124	35.2\\
215	34.99\\
314	35.8\\
16	32.8\\
-46	34.48\\
-120	38.64\\
-129	43\\
-55	42.44\\
-121	45.2\\
-243	45.2\\
-332	43\\
-380	39.94\\
-356	39.8\\
-324	41.32\\
-182	48.72\\
119	46\\
-165	60\\
-164	59.94\\
159	53.11\\
343	45.97\\
481	44.95\\
1010	43.52\\
604	37.27\\
962	37.95\\
889	37.6\\
1086	37.18\\
851	32.43\\
1310	43.24\\
1310	52.78\\
1310	56.79\\
1310	55.89\\
1310	57.6\\
1310	58.94\\
1310	59.31\\
1310	62.6\\
1310	59\\
1310	54.61\\
1310	51.78\\
1310	50.27\\
1310	59.3\\
1310	84.15\\
1310	68.49\\
1205	48.66\\
1310	51.71\\
1310	49.15\\
1255	40.5\\
584	36.63\\
458	32.74\\
452	32.33\\
681	32.9\\
1003	38.9\\
1020	38.99\\
1310	56.79\\
1310	60.08\\
1310	61.66\\
1310	58.12\\
1310	57.85\\
1310	57.21\\
1310	54.92\\
1310	48.29\\
1310	46.47\\
1310	44.42\\
1310	44.39\\
1310	47.62\\
1310	65.01\\
1310	57.46\\
1310	49.99\\
1310	50\\
1310	48.69\\
1079	41.26\\
822	37.88\\
721	37.16\\
793	33.16\\
840	37.02\\
1138	37.98\\
1021	38.51\\
1310	51.96\\
1310	55.31\\
1300	54.84\\
1251	53.77\\
1310	52.87\\
1310	51.29\\
1238	46.07\\
1146	44.96\\
1158	42.26\\
942	40\\
824	40.53\\
1310	41.14\\
1310	54.95\\
1310	51\\
1103	44.94\\
1208	42\\
1304	38.47\\
351	32.53\\
431	34.94\\
383	33.12\\
358	32.56\\
455	33\\
703	37.23\\
760	37.92\\
1219	51.74\\
1219	53.11\\
1210	57.44\\
1219	51.17\\
1219	50.6\\
1219	52.94\\
1219	50.67\\
1219	48\\
1219	43.3\\
1219	41.08\\
1219	43.79\\
1219	51.83\\
1219	61.68\\
1219	53.11\\
1219	42.8\\
1219	46.48\\
1219	46.18\\
767	37.88\\
602	32.27\\
561	28.93\\
506	28.92\\
573	29.64\\
942	30.86\\
1310	37.41\\
1310	51.94\\
1310	54.15\\
1310	55.82\\
1310	53.23\\
1310	52.35\\
1310	52.73\\
1310	51.5\\
1310	49.55\\
1310	47.09\\
1310	45\\
1170	49.89\\
1192	52.22\\
1291	58.41\\
1310	52.06\\
1310	42.94\\
1195	42.1\\
1005	40.76\\
866	38.85\\
682	38.96\\
880	36.05\\
764	33.54\\
763	33.54\\
849	35.25\\
629	39.15\\
338	41\\
513	46.36\\
467	56.8\\
452	62.23\\
483	61.55\\
317	49.5\\
376	44.94\\
270	42\\
276	42\\
425	44.94\\
654	46.42\\
456	43\\
497	53.11\\
462	47\\
661	39.31\\
878	39.66\\
929	38.62\\
754	34.75\\
572	30.96\\
981	33.85\\
935	33.54\\
950	32.69\\
1081	33.54\\
766	31.16\\
522	31.13\\
579	34.94\\
631	38.7\\
636	40.12\\
577	40.25\\
379	40.63\\
286	39.94\\
286	38.8\\
336	37.64\\
163	38.99\\
155	44.94\\
108	47.44\\
112	54.99\\
171	49\\
344	40\\
832	39.25\\
1260	35.52\\
192	29.4\\
812	30.94\\
854	30.71\\
922	30.82\\
1005	32.16\\
1309	32.69\\
946	51.8\\
946	54.94\\
946	55.95\\
946	56.66\\
835	51.84\\
806	51.8\\
946	51.2\\
945	47.06\\
793	43.26\\
946	43.95\\
946	44.92\\
946	47.24\\
946	51.8\\
946	62.47\\
946	59.91\\
946	49.37\\
946	47.67\\
946	46.46\\
946	36.68\\
946	39.06\\
936	31.83\\
922	31.07\\
946	36.67\\
946	39.62\\
946	40.41\\
946	54.87\\
946	57.34\\
946	55.95\\
946	53.79\\
946	53.99\\
946	54.22\\
946	52.87\\
946	50.16\\
946	47.32\\
946	42.86\\
946	53.6\\
946	57.77\\
946	67.8\\
946	59.89\\
946	49.5\\
946	48.5\\
946	51.3\\
946	40.28\\
946	39.45\\
946	41.3\\
946	41.33\\
946	51.36\\
946	51.36\\
946	51.36\\
946	97.81\\
946	54.15\\
946	150\\
946	77.1\\
946	60.74\\
946	54.66\\
946	53.12\\
946	51.5\\
946	60.74\\
946	70\\
946	80\\
946	55\\
946	120\\
946	52.32\\
946	47.5\\
946	47.5\\
946	47.5\\
946	40.23\\
946	51.1\\
946	44.99\\
946	41\\
946	39.17\\
848	32.55\\
946	40.98\\
946	52.98\\
946	54.94\\
946	55.04\\
946	56.55\\
946	55.52\\
946	52.66\\
946	51.07\\
946	49\\
946	49\\
946	49\\
946	51.4\\
946	50\\
946	57.98\\
946	51.78\\
946	41\\
946	40.28\\
946	48.4\\
946	47.1\\
946	41.5\\
946	47.1\\
946	45.33\\
946	47.1\\
946	47.1\\
946	51.2\\
946	55.78\\
946	53.87\\
946	55.1\\
946	54.11\\
946	51.93\\
946	49.18\\
946	45.35\\
946	43.63\\
946	41.5\\
946	48\\
946	78.4\\
1162	45.94\\
1239	52.29\\
1389	45.15\\
1143	38.28\\
1220	45.5\\
1369	39.94\\
764	38.34\\
323	36.6\\
170	32.8\\
356	31\\
175	30.63\\
47	30.67\\
50	29.99\\
395	38.15\\
564	40.25\\
335	46.3\\
374	46.3\\
483	45\\
359	42\\
345	40.2\\
319	38.09\\
394	38.09\\
539	40.3\\
794	46.3\\
770	46.3\\
704	54.94\\
242	51.77\\
400	46.3\\
624	46.28\\
489	44.88\\
169	26.25\\
90	24.38\\
287	26.55\\
332	26.55\\
254	22.76\\
117	23.08\\
-15	26.31\\
-168	27.32\\
-102	28.89\\
-74	36.6\\
-74	36\\
-81	34.94\\
-261	30.92\\
-560	30.09\\
-646	29.87\\
-594	29.64\\
-451	29.83\\
-1	34.8\\
-160	42\\
-95	52.6\\
-37	49.8\\
293	43\\
277	37.99\\
204	32.02\\
276	20.59\\
532	18.1\\
957	16.37\\
560	10.05\\
370	11.52\\
480	15.38\\
1044	34.94\\
1310	60\\
1310	45.61\\
1310	48.65\\
1200	46.76\\
1175	49.72\\
1310	53.61\\
1148	54.81\\
1310	52.95\\
1310	49.67\\
1310	48.21\\
1310	60.6\\
1310	50\\
1310	57.98\\
1310	50\\
1310	39.99\\
1310	41.64\\
1101	39.96\\
982	35.14\\
1078	35.14\\
1123	34.68\\
1074	31.95\\
1114	30\\
1191	30.07\\
1108	37.44\\
970	46.44\\
684	48.93\\
560	46.19\\
548	44.94\\
748	50.12\\
651	46.54\\
451	49\\
551	47.01\\
363	44.62\\
335	42.44\\
452	47.71\\
766	42.12\\
1129	49.53\\
974	39.96\\
816	37.1\\
790	39.71\\
949	33.92\\
115	42.44\\
119	35.06\\
333	35.06\\
220	31.88\\
357	32.22\\
444	34.47\\
691	42.94\\
1055	50.5\\
1068	50.15\\
840	51.78\\
822	59\\
730	50\\
783	49.21\\
594	45.84\\
471	45.24\\
437	45\\
388	49.94\\
158	54.53\\
824	59\\
327	66.4\\
502	59\\
574	50\\
581	53.12\\
579	51.3\\
1310	48.41\\
1310	45.7\\
1310	41.49\\
1195	36.97\\
1265	36.06\\
1310	42.94\\
1310	52.58\\
1310	63.64\\
1310	70\\
1310	68.06\\
1310	61.05\\
1310	63.64\\
1310	57.02\\
1310	52.15\\
1310	52.9\\
1310	50.52\\
1310	48.91\\
1310	52.03\\
1310	61.05\\
1310	69\\
1310	69.36\\
1310	53.57\\
1310	47.75\\
1310	42.48\\
839	38.5\\
735	32.91\\
726	30.79\\
669	29.17\\
715	24.99\\
753	29.99\\
761	40\\
758	53.49\\
779	53.11\\
228	53.11\\
201	54.53\\
308	59.94\\
330	52.44\\
240	51\\
346	50.25\\
488	49.52\\
520	49.51\\
508	50.39\\
1040	49.93\\
1310	50.81\\
776	50.5\\
682	47.07\\
1272	49.99\\
1280	48.23\\
488	47.02\\
717	39.21\\
441	33.32\\
427	28.98\\
350	33.15\\
436	34.4\\
166	23.34\\
96	42\\
422	46.59\\
340	52.44\\
316	55.78\\
379	55.78\\
564	57.5\\
382	46.68\\
177	39.94\\
84	40\\
324	42.44\\
604	50.97\\
441	57.5\\
54	57.5\\
102	55.78\\
88	42.09\\
390	45\\
397	46.21\\
493	44.99\\
520	38.37\\
343	33.7\\
173	35.74\\
193	28.39\\
216	20\\
429	18.3\\
528	23.36\\
259	25.56\\
-145	25.43\\
-196	37.44\\
-160	38.4\\
-200	42\\
-210	44.94\\
-309	38.3\\
-488	34.94\\
-568	34.93\\
-658	38.41\\
-843	48.29\\
-804	59.37\\
-601	59.37\\
-333	46.44\\
-178	43.71\\
79	43.91\\
920	43.9\\
623	39.27\\
401	32.75\\
248	27.82\\
286	22.1\\
334	18.6\\
767	22.93\\
601	39.53\\
660	44.8\\
678	50.03\\
568	51.86\\
339	49.97\\
429	48.86\\
550	50\\
356	49.12\\
392	47.85\\
291	48.76\\
235	52.04\\
-272	60.51\\
361	62.07\\
860	56.95\\
1128	47.65\\
682	44.94\\
1116	45.75\\
913	40.5\\
1401	44.46\\
1401	38.6\\
1298	32.74\\
967	31.54\\
1152	25.48\\
1401	31.84\\
1310	41.58\\
1310	48.66\\
1310	52.6\\
1310	52.6\\
1310	51.69\\
1310	51.7\\
1310	47.15\\
1310	48.69\\
1310	49.24\\
1300	46.32\\
1011	49.42\\
669	58.4\\
1401	64.02\\
1401	62.19\\
1401	52.6\\
1250	45\\
1401	49.98\\
1401	49\\
1401	42.68\\
1401	38.17\\
1342	28.48\\
1156	24.8\\
1033	24.36\\
1289	28.42\\
1310	47.61\\
1310	50.38\\
1310	51.78\\
1310	52.87\\
1310	51.78\\
1310	52\\
1310	52.59\\
1310	52.79\\
1310	51.7\\
1211	52.25\\
1147	47.99\\
861	54.35\\
1401	69.96\\
1401	60.97\\
1401	52.44\\
1401	46.93\\
1115	46.59\\
1401	47.11\\
1401	42.28\\
1401	32.36\\
1401	32.73\\
1010	29.94\\
1074	30.19\\
1401	34.29\\
1310	47\\
1310	54.56\\
1310	58.32\\
1310	57.53\\
1310	56.97\\
1216	56.71\\
1310	54.18\\
1310	52.99\\
1154	53.45\\
1235	52.93\\
903	52\\
907	56.42\\
1214	64.1\\
1401	54.91\\
1401	49.62\\
1401	45.92\\
1401	49.97\\
1401	43.83\\
1401	37.59\\
767	30.64\\
716	30.22\\
730	28.83\\
779	28.83\\
742	30.64\\
740	39.96\\
630	50.03\\
847	50.03\\
1062	49.11\\
843	47.44\\
742	50\\
794	49.11\\
526	48.38\\
589	47.44\\
642	45\\
660	49.11\\
630	58.49\\
545	60\\
742	46.14\\
295	49.94\\
95	49.94\\
127	44.94\\
286	36.17\\
292	27.24\\
239	19.67\\
372	17.25\\
-109	15.34\\
-548	14.29\\
-359	13.52\\
-166	16.2\\
-498	17.61\\
-804	20.83\\
-878	30.97\\
-931	33.62\\
-940	33.6\\
-908	33.67\\
-824	33.56\\
-730	32.78\\
-555	34.68\\
-409	39.99\\
-273	50.22\\
-523	63.4\\
-489	51.37\\
-435	42.44\\
-231	37\\
53	37.78\\
118	32\\
691	16.11\\
780	14.91\\
750	13.03\\
402	10.78\\
449	11.65\\
639	11.88\\
967	12.98\\
625	13.55\\
-49	14.31\\
-180	15.1\\
-532	15.9\\
-566	16.9\\
-636	21\\
-637	19.96\\
-607	14.69\\
-493	21\\
-294	31.18\\
-235	42.44\\
-348	46.25\\
-405	40\\
-587	34.94\\
-181	32.76\\
-124	22.47\\
609	19.16\\
98	18.77\\
297	14.5\\
281	6.33\\
-158	0.12\\
-90	2.58\\
591	13.48\\
526	28.97\\
1105	39.8\\
1401	43.38\\
1375	46.64\\
1401	48.05\\
1401	48.09\\
1401	47.64\\
1401	46.54\\
1401	43.06\\
1116	43.12\\
749	40.77\\
402	49.23\\
1270	62.74\\
1401	58.03\\
1401	50.46\\
1401	54.69\\
1401	49.05\\
1401	46.8\\
1401	51.02\\
1401	42.57\\
1376	40.21\\
1362	34.18\\
1401	35.38\\
1401	39.44\\
1401	51.32\\
1401	62.7\\
1401	58.79\\
1401	58.9\\
1401	55.96\\
1401	54.2\\
1401	52\\
1401	49.52\\
1283	45.51\\
1301	46.41\\
1151	48.93\\
1048	57.47\\
1401	62.19\\
1401	63.95\\
1401	57.1\\
1401	44.52\\
1401	46.96\\
1401	43\\
1081	35.81\\
1317	34.85\\
1365	33.91\\
1362	31.65\\
1401	33\\
1401	30.71\\
1401	43.36\\
1401	53.75\\
1401	56.83\\
1401	56.28\\
1401	55.31\\
1401	55.87\\
1401	53.52\\
1401	53.5\\
1401	51.15\\
1395	51.46\\
1114	53.5\\
1121	61.3\\
1401	75.05\\
1401	67.79\\
1401	52.43\\
1401	46.78\\
1401	49.41\\
1401	49.77\\
1114	43.3\\
1401	40.63\\
1401	35.08\\
938	30.21\\
1111	31.02\\
1401	38.74\\
1401	46.31\\
1310	55.66\\
1285	60.56\\
1277	60.06\\
968	60.42\\
836	60.08\\
774	55.41\\
662	53.75\\
825	53.5\\
433	53.96\\
358	55.98\\
44	63.72\\
495	77.92\\
852	63.5\\
980	54.92\\
1108	45.94\\
1401	47.44\\
1401	43.3\\
526	29.88\\
-82	27.86\\
311	26.7\\
305	25.91\\
218	26.81\\
88	26.7\\
723	34.76\\
1401	49.04\\
1171	50.4\\
870	47.31\\
728	44.96\\
666	45.33\\
719	43.3\\
680	42.8\\
665	41.37\\
713	40.49\\
535	43.3\\
298	61.3\\
441	61.3\\
897	52.44\\
760	44.96\\
161	39.89\\
785	42.15\\
943	43.7\\
946	34.94\\
498	30.33\\
397	29.75\\
-82	23.15\\
-284	23.32\\
-267	23.43\\
386	28.75\\
1063	35\\
773	39.98\\
375	44.02\\
153	43.99\\
125	42.33\\
199	44.75\\
204	41.77\\
-83	37.7\\
13	37.21\\
-48	39.23\\
-155	52.79\\
142	53.6\\
516	51.18\\
302	44\\
440	38.79\\
991	41.43\\
1310	41.8\\
991	38.38\\
321	30.95\\
-428	20.78\\
-127	21.06\\
-126	21.11\\
-314	22.39\\
-357	22.7\\
-727	22.75\\
-666	23.34\\
-388	26\\
250	33\\
217	34.94\\
409	40.46\\
-40	37.44\\
-102	33.47\\
0	37.32\\
50	39.94\\
-239	54.94\\
1	64.99\\
147	55\\
232	48.06\\
87	41.76\\
553	44.22\\
1123	40.85\\
1263	38.43\\
484	28.71\\
386	25.14\\
235	29.1\\
242	24.08\\
602	29.98\\
1101	42.92\\
1048	50.13\\
1232	50.43\\
1174	50.06\\
1118	49.97\\
1199	49.73\\
1215	50.04\\
991	48.72\\
912	48.44\\
684	48.62\\
497	47.99\\
966	53.53\\
1501	60.33\\
1501	53.36\\
1217	44.96\\
408	37.89\\
1285	40.38\\
1194	34.26\\
773	34.22\\
292	28.81\\
290	27.25\\
-57	25.6\\
-181	25.44\\
-327	27.98\\
-910	31.29\\
-966	41.13\\
-1223	42.72\\
-1173	42.87\\
-1063	43.5\\
-858	44.94\\
-766	43.11\\
-1034	44.5\\
-986	43.5\\
-876	44\\
-797	57.44\\
-1190	60.99\\
-683	57.8\\
-154	51.91\\
-126	55\\
-242	42.36\\
645	43.01\\
815	35.38\\
-124	25.14\\
118	23.88\\
199	23.74\\
145	23.7\\
278	29.66\\
166	25.85\\
717	37.01\\
1060	46.68\\
1396	47.14\\
1401	49.77\\
1401	50.5\\
1401	50.58\\
1401	49.2\\
1401	46.98\\
1363	47.12\\
1000	49.15\\
670	49.94\\
1138	60.7\\
1401	68.13\\
1401	57.81\\
1401	49.51\\
905	50.8\\
1092	47.44\\
980	43.13\\
784	39.96\\
823	35.13\\
788	34.28\\
687	33.32\\
863	31.71\\
867	31.71\\
1220	39.97\\
1214	54.09\\
1401	53.9\\
1261	55\\
1150	54.27\\
1083	53.8\\
1003	51.62\\
903	49.91\\
946	47.42\\
470	47.01\\
679	48.05\\
1303	58.01\\
1298	58.93\\
1401	52.97\\
1401	47.18\\
1141	37.93\\
1005	36.13\\
684	36\\
-59	41.63\\
-42	40.22\\
224	39.71\\
123	32.44\\
253	28.14\\
315	29.2\\
577	37.07\\
911	47.3\\
1401	47.91\\
1374	46.21\\
1089	45.2\\
1117	44.03\\
521	44.77\\
599	47.44\\
647	56.41\\
746	58\\
792	56.43\\
813	71.2\\
899	70\\
1035	56.1\\
752	51.37\\
516	50\\
1107	52.44\\
1088	44\\
457	35.91\\
583	39.99\\
384	37.48\\
368	35.17\\
281	33.69\\
361	33.5\\
365	37.66\\
307	39.94\\
352	49.94\\
245	44.99\\
322	44.5\\
378	50.5\\
432	49.99\\
430	45.2\\
460	44.5\\
361	43.27\\
-67	45.35\\
-278	68.9\\
16	68.9\\
393	57\\
72	40.44\\
230	42.21\\
586	47.44\\
639	43.91\\
305	40.44\\
299	41.17\\
425	39.16\\
306	34.99\\
250	33.69\\
233	33.69\\
-132	32.17\\
-396	31.31\\
-170	39.16\\
81	43.75\\
170	49.9\\
294	53.49\\
327	52\\
307	44.94\\
340	40.71\\
470	41\\
545	40.66\\
729	59.96\\
986	68.9\\
1105	51.78\\
960	40.79\\
891	41.17\\
852	44.75\\
849	40.83\\
476	35.17\\
493	34.02\\
549	33.43\\
446	33.8\\
517	33.43\\
734	33.21\\
1057	44.94\\
1433	51.32\\
1365	54.29\\
1450	53.4\\
1274	52.43\\
1314	53.79\\
1154	51.98\\
968	54.15\\
699	56.25\\
723	54.97\\
316	57.41\\
542	70.11\\
1056	67.52\\
1070	54.99\\
1450	49.78\\
1450	42.07\\
1450	46.57\\
1358	46.2\\
1195	42.69\\
1450	44\\
1302	39.03\\
1188	36.55\\
1379	38.28\\
1131	37.17\\
1300	44.83\\
1450	53.5\\
1274	55.5\\
1376	54.99\\
967	54.76\\
752	54.96\\
815	52.88\\
856	53.03\\
859	54.5\\
783	54.75\\
482	57.02\\
748	78.61\\
1105	63.3\\
1450	58.3\\
1450	50.89\\
1301	50.8\\
1450	50.8\\
1450	49.99\\
1450	42.02\\
1333	38.43\\
1450	41.43\\
1450	40\\
1450	41.42\\
1450	41.47\\
1450	50.5\\
1450	55.19\\
1003	59.94\\
1033	60\\
711	58.75\\
598	57.78\\
668	54.41\\
725	54.32\\
682	52.98\\
934	50.5\\
617	54.96\\
720	75\\
774	66.97\\
1058	59.45\\
1450	52.26\\
1450	47.61\\
1450	50.5\\
1450	49.99\\
1161	42.44\\
1450	41.49\\
1450	40.6\\
1450	40.39\\
1427	40.6\\
1450	41.57\\
1432	44.75\\
926	58.69\\
484	58.13\\
428	56.37\\
375	54.06\\
394	56.15\\
511	51\\
501	50.45\\
464	49.97\\
428	50.57\\
426	56.48\\
440	75\\
623	61.47\\
780	58.34\\
1224	52.74\\
1200	46.85\\
1392	47.7\\
1450	46.4\\
910	49.99\\
985	41.74\\
986	41.23\\
900	38.63\\
1008	40.76\\
1113	41.45\\
1394	44.36\\
1251	54.94\\
867	55.32\\
864	56.37\\
877	60\\
859	64\\
818	53.94\\
926	56.36\\
1029	50.29\\
955	50.38\\
1062	53.5\\
756	65.12\\
844	55.37\\
1390	52.85\\
1447	50.61\\
1450	49.88\\
1450	53.5\\
1450	52.32\\
685	40.34\\
300	35.88\\
307	34.22\\
508	35.13\\
363	32.64\\
210	31.49\\
-151	31.56\\
-8	39.96\\
-152	44.7\\
-368	46.3\\
-659	52.38\\
-694	54.94\\
-748	55.12\\
-661	44.94\\
-706	50\\
-390	49.21\\
-457	48.94\\
-560	66.4\\
-284	59.94\\
6	47.39\\
182	49.99\\
155	45.23\\
359	41.92\\
417	40.8\\
130	31.94\\
-182	28.91\\
-39	30.34\\
120	31.87\\
32	28.12\\
-57	28.74\\
-132	30\\
-363	30\\
-458	31\\
-272	36.57\\
-265	41.64\\
-231	50.21\\
-399	57.39\\
-557	46.79\\
-618	42.11\\
-531	42.84\\
-446	52.62\\
-787	69\\
-648	64.94\\
-269	57.39\\
-14	57.39\\
68	44.5\\
133	40.46\\
402	37.93\\
-164	31.69\\
-97	31.71\\
65	34.2\\
70	34.2\\
143	34.62\\
-172	30.34\\
261	38.11\\
738	51.34\\
1178	51\\
877	49.44\\
866	48.97\\
1026	50.32\\
1116	48.5\\
688	49\\
689	49.95\\
554	50.54\\
487	51.51\\
472	60.16\\
1233	63.34\\
1368	57.47\\
1002	51\\
524	48.58\\
985	47.44\\
1040	44.88\\
782	46.7\\
723	41.79\\
1087	41.3\\
1080	37.7\\
1154	39.94\\
889	35.7\\
997	46.9\\
993	56.27\\
869	56.03\\
797	55.94\\
750	53.99\\
716	54.33\\
828	53.01\\
823	54\\
802	55\\
590	57.29\\
164	57.99\\
454	69.44\\
1393	68.03\\
1483	57.72\\
1500	52.22\\
1412	47\\
1500	50.7\\
1500	49.27\\
648	35.33\\
817	36.09\\
950	34.72\\
838	34.45\\
894	34.45\\
1124	34.72\\
1310	46.1\\
1310	59.99\\
1310	54.49\\
1310	52.5\\
1310	64.63\\
1310	78.9\\
1310	69.01\\
1310	69.06\\
1310	63\\
1310	69.06\\
1310	78.9\\
1169	84.63\\
1310	78.9\\
1310	54.35\\
1310	49.48\\
1310	78.9\\
1310	78.9\\
1310	55.72\\
1401	49.64\\
1304	43.01\\
1252	39.3\\
1208	35.7\\
1252	37.25\\
1259	42.94\\
1500	46.54\\
1500	60\\
1500	57.47\\
1500	58.93\\
1500	58\\
1500	58.1\\
1493	57.92\\
1454	55.5\\
1494	55.5\\
1500	55.5\\
1467	57.26\\
1391	80\\
1401	55.95\\
1401	56\\
1401	56\\
1173	50.06\\
1374	49.94\\
1335	44.19\\
769	47.17\\
876	42.49\\
935	39.94\\
796	35\\
861	34.82\\
966	37.08\\
1097	44.12\\
930	56.33\\
836	53\\
669	55.5\\
712	55.5\\
871	55.5\\
960	52.5\\
986	51\\
1126	49.96\\
1138	50.06\\
1005	54.78\\
499	72.32\\
650	55.5\\
1113	50.06\\
818	55\\
703	50.06\\
954	47.3\\
1000	44.94\\
472	42.18\\
552	37.01\\
586	37\\
386	32.84\\
336	30.88\\
482	30.88\\
211	31.13\\
180	38.25\\
8	43.41\\
307	46.8\\
272	47\\
277	48\\
310	45\\
246	44.71\\
252	44.72\\
423	44.76\\
334	50.12\\
205	76.4\\
461	76.4\\
310	52.44\\
254	46.79\\
252	45\\
454	47.59\\
579	47.49\\
547	37.01\\
630	34.51\\
737	34.51\\
673	34.24\\
495	30.26\\
444	30.41\\
474	31.49\\
199	32.14\\
};
\addplot [color=mycolor1,line width=1.0pt,mark size=0.3pt,only marks,mark=*,mark options={solid},forget plot]
  table[row sep=crcr]{%
-279	31.29\\
-131	40\\
-217	42.31\\
-136	42.41\\
-155	42.41\\
-254	42\\
-229	40.12\\
-87	34.94\\
-22	42.44\\
-60	65\\
11	60.79\\
18	54.06\\
149	52\\
-107	42.62\\
258	42.31\\
275	32.24\\
1201	41.78\\
1240	34.77\\
1401	38\\
1220	33.74\\
1401	41.67\\
1401	39.16\\
1401	51.7\\
1262	53.43\\
1401	53\\
1283	54\\
1251	53\\
1156	56.36\\
1401	55.35\\
1401	57\\
1401	53.64\\
1401	53\\
1363	52.7\\
1133	60\\
1401	80.74\\
1401	59.71\\
1401	54.46\\
1401	54.4\\
1401	57\\
1401	57\\
1401	54.4\\
1401	51.4\\
1401	51.4\\
1500	51.4\\
1500	51.4\\
1500	51.4\\
1500	50\\
1500	57.76\\
1071	64.42\\
1014	64.83\\
882	64.43\\
825	63.36\\
1140	60.4\\
731	59.07\\
1042	58.08\\
1022	56.96\\
814	61.84\\
1114	77.52\\
1500	68.19\\
1500	60.09\\
1500	54.87\\
1500	48.06\\
1500	47.52\\
1500	45.87\\
1500	49.05\\
1500	46.3\\
1500	42.36\\
1126	31.67\\
1093	31.47\\
1500	35.85\\
1500	48.7\\
1446	63\\
1268	62.09\\
1144	64.9\\
832	64.55\\
703	69.8\\
796	62.3\\
493	64.62\\
331	67.76\\
120	65.66\\
138	65.95\\
1166	82.87\\
1500	72.72\\
1500	69.91\\
1500	58.96\\
1500	50.6\\
1500	58\\
1500	56.31\\
1500	46.04\\
1500	46.06\\
1500	49.5\\
1500	43.62\\
1500	39.57\\
1500	47.54\\
1500	51.43\\
1100	60.87\\
1100	64\\
1100	67.64\\
1062	67.7\\
914	67.7\\
1100	65.99\\
955	67.47\\
1063	67.87\\
914	65.83\\
971	72.23\\
1362	87.97\\
1500	73.59\\
1500	64.67\\
1500	57.5\\
1500	56.89\\
1500	54.96\\
1500	49.4\\
1500	57.62\\
1500	54\\
1500	47.99\\
1500	45.17\\
1500	43.36\\
1500	47.16\\
1500	50.91\\
1500	66.64\\
1500	70.41\\
1500	68.99\\
1459	69.77\\
1394	66.39\\
1365	68.05\\
1343	64.39\\
1500	58.67\\
1385	57.5\\
1294	59.98\\
1500	70.61\\
1500	64.9\\
1500	63.4\\
1500	58.64\\
1500	48.04\\
1500	53.96\\
1500	52.44\\
1347	50.1\\
1500	46.49\\
1500	40\\
1500	39.14\\
1240	33.91\\
1167	33.99\\
1369	37.02\\
1179	47.46\\
1389	49.52\\
1500	50.43\\
1100	54.06\\
1100	59.99\\
1100	66.5\\
1100	50.03\\
1501	48.92\\
1498	48.91\\
1190	50.48\\
926	62.17\\
1081	59.96\\
1501	53.97\\
1412	51.36\\
1038	48.69\\
1501	49.45\\
1501	48.66\\
1500	46.89\\
1500	42.43\\
1500	41.5\\
877	30.07\\
705	30.42\\
843	31.33\\
860	29.53\\
1236	33.89\\
1022	33.45\\
1401	42.89\\
1230	49.36\\
1401	53.36\\
919	57.13\\
1172	52.87\\
765	45.5\\
630	41.27\\
-19	45\\
-190	60\\
-266	64.5\\
197	52.72\\
269	53.57\\
985	51.4\\
1298	54\\
1401	50\\
1401	49.69\\
1401	46.82\\
1401	42\\
1401	39.14\\
1401	36.16\\
1401	40.78\\
1401	47.2\\
1401	57.53\\
1401	50.95\\
1161	50.82\\
912	52\\
989	54.06\\
1035	52.5\\
1004	51.5\\
988	52\\
1033	51.5\\
918	56.29\\
664	73\\
1375	61.36\\
1401	56.25\\
1401	51.29\\
1372	45\\
1401	49.99\\
1401	49.02\\
1500	45.45\\
1500	41.37\\
1500	39.53\\
1232	31.52\\
1346	32.23\\
1500	39.85\\
1500	46.77\\
1500	62.52\\
1500	67.41\\
1500	69.85\\
1500	63.71\\
1500	61.28\\
1500	56.84\\
1276	57.92\\
1402	60.06\\
1226	59.74\\
1082	62.57\\
1355	74.69\\
1401	72.42\\
1401	62.32\\
1401	55.87\\
1401	50.24\\
1401	51\\
1401	46.92\\
1401	46\\
1401	43.2\\
1401	41.21\\
1401	36.27\\
1342	31.09\\
1401	39.49\\
1401	44.39\\
1401	52.78\\
1401	58.71\\
1401	59.25\\
1401	55.1\\
1401	57.96\\
1401	57.69\\
1401	56.84\\
1401	55.93\\
1401	55.93\\
1401	54.87\\
1157	59.72\\
1401	61.25\\
1401	59.68\\
1294	53.05\\
1311	50.45\\
1401	51.04\\
1401	50.73\\
1122	49.5\\
1401	45.34\\
1401	41.06\\
1401	36.53\\
890	33.2\\
1401	38.12\\
934	45.17\\
1358	55.16\\
1382	55.45\\
1274	52.02\\
1234	52.08\\
1205	53.01\\
1401	52.9\\
1152	52.59\\
1131	48.88\\
1052	46.96\\
677	49.76\\
505	59.99\\
1333	54.7\\
1401	54.15\\
1269	49.45\\
1222	45\\
958	45.84\\
1128	46.62\\
1276	47.98\\
1212	41.66\\
1185	37.41\\
452	31.22\\
423	31\\
1087	37.93\\
1401	46.13\\
1401	54.96\\
1251	50.98\\
849	52.46\\
704	51.99\\
714	59.94\\
854	52.88\\
899	50.22\\
1211	51.31\\
967	48.85\\
889	53.8\\
310	64.94\\
751	56.03\\
934	53.98\\
592	50.66\\
463	46.06\\
924	51.44\\
1106	48.9\\
836	51.39\\
722	49.04\\
754	41.93\\
806	40.46\\
898	41.76\\
877	43.5\\
685	42.44\\
567	47.17\\
870	46.99\\
895	60.32\\
746	62.9\\
673	62.9\\
953	52.1\\
1125	50.96\\
1190	48.3\\
1401	45.67\\
959	45.23\\
767	62.9\\
1031	55.69\\
1401	51.08\\
1322	47.44\\
1085	42.5\\
1401	48.39\\
1401	49.03\\
1399	49.43\\
1325	44.37\\
1500	42.08\\
1438	38.76\\
1500	34.9\\
1500	35\\
1500	36.9\\
1360	39.99\\
965	40.64\\
1114	44.99\\
1500	47.44\\
1500	50.44\\
1401	52.35\\
1401	50.97\\
1401	52.03\\
1401	50.2\\
1351	50.42\\
745	59.2\\
878	57.69\\
1288	57.44\\
1401	53.39\\
1323	49.6\\
1401	47.79\\
1401	46.1\\
1401	47.51\\
1401	38.65\\
1401	34.38\\
1071	30.42\\
1077	30.34\\
1401	36.5\\
1401	47.17\\
1312	55.54\\
1401	59.94\\
1401	61.45\\
1401	61.6\\
1401	61.66\\
1401	59.99\\
1401	58.78\\
1401	57.14\\
1401	54.13\\
1182	59.21\\
1009	74.27\\
1401	71.23\\
1401	68\\
1351	57.59\\
1401	51.6\\
1401	51.93\\
1401	50.27\\
1401	47.44\\
1401	43.5\\
1401	41.37\\
1252	33.75\\
1221	33.32\\
1401	36\\
1401	42.07\\
1401	54.75\\
1401	54.87\\
1401	55.96\\
1401	55.96\\
1401	55.64\\
1401	54.09\\
1370	53.74\\
1173	52.44\\
1035	51.42\\
892	52.62\\
1095	60.99\\
1401	59.66\\
1401	58.07\\
1401	55.3\\
1257	47.46\\
1401	55.3\\
1401	47.12\\
639	42.72\\
767	40.44\\
657	35.83\\
514	32.35\\
393	28.51\\
476	29.19\\
790	39.06\\
919	50.5\\
1208	49.5\\
729	51.02\\
534	49.94\\
594	52.44\\
537	53.91\\
407	54.3\\
412	54.3\\
440	57.39\\
604	57.13\\
421	75\\
761	57.61\\
1051	54.3\\
573	48.54\\
686	45\\
1142	51.36\\
978	47.16\\
494	41.07\\
670	40.4\\
490	33.72\\
204	33.2\\
205	29.36\\
422	29.75\\
609	39.96\\
763	51.13\\
1006	50.65\\
158	63.7\\
714	61.21\\
729	62.3\\
613	46.46\\
550	52.59\\
623	42.44\\
643	41.85\\
653	42.5\\
651	49.5\\
1400	50.61\\
1400	48.73\\
292	41.9\\
-54	40\\
225	40.23\\
487	37.43\\
-91	32.82\\
-429	29.15\\
12	30\\
-157	29.49\\
-34	27.43\\
407	27.48\\
270	33.69\\
135	52.25\\
500	47.39\\
556	68\\
548	69.29\\
766	66.53\\
1033	64.94\\
917	49.67\\
1051	44.97\\
1103	43.03\\
1211	47.39\\
1294	67.23\\
1258	59.96\\
1123	51.99\\
707	52.44\\
372	48.81\\
593	48.52\\
529	44.94\\
-568	32.64\\
-496	27.91\\
-529	28.99\\
-788	24.69\\
-904	19.96\\
-856	13.15\\
-1113	19.68\\
-1187	19.96\\
-849	38.59\\
-963	36.08\\
-982	40\\
-997	40\\
-1074	40\\
-832	40\\
-871	39.42\\
-568	40.18\\
-625	53.38\\
-873	70.5\\
-654	70.5\\
-460	67.21\\
-560	48\\
-419	40.72\\
-347	42.44\\
57	40.18\\
-458	39.39\\
-495	34.85\\
-184	36.08\\
-412	34.83\\
-493	33.99\\
-425	32.64\\
-572	30.74\\
-1202	30.75\\
-856	30.16\\
-650	38.06\\
-936	43.86\\
-1056	45.44\\
-1090	63.18\\
-1036	59.99\\
-1128	40.34\\
-1128	39.98\\
-1128	39.49\\
-1128	54.94\\
-1105	63.18\\
-968	49.94\\
-896	47.44\\
-1057	41.2\\
-625	39.73\\
-689	30\\
-1267	16.57\\
-1401	15.11\\
-1401	12.99\\
-1401	6.31\\
-1401	3.46\\
-1307	13.24\\
-1092	19.22\\
-836	40.66\\
-220	44.39\\
-119	44.94\\
-32	45\\
-21	54.94\\
-109	47\\
-131	45\\
-192	54.43\\
-221	48\\
-382	55\\
-575	59.94\\
-342	45.81\\
-310	46.99\\
-965	49.94\\
-1065	50\\
-812	40.28\\
-601	39.26\\
-1084	28.49\\
-1212	27.91\\
-1243	27.28\\
-1263	26.87\\
-1240	27\\
-968	23.17\\
-1026	32.95\\
-866	40\\
-200	45\\
59	43.52\\
52	47.39\\
-26	47.39\\
125	43.16\\
138	42.44\\
91	41.81\\
173	42.26\\
-13	47.39\\
134	70\\
240	47.39\\
234	45\\
-418	47.39\\
-794	40.32\\
-454	44.94\\
-468	40\\
-948	29.63\\
-1257	28.49\\
-1211	28.49\\
-1326	28.12\\
-1351	16.36\\
-1280	13.09\\
-1058	34.6\\
-1180	38.92\\
-748	44.5\\
-497	47.44\\
-354	47.44\\
-342	60\\
-294	59.99\\
110	56.55\\
147	49.34\\
270	44.55\\
223	47.39\\
-38	55\\
206	47.39\\
-14	45\\
-505	44.94\\
-695	40.87\\
-190	44.34\\
223	47.39\\
283	37.18\\
155	29\\
-280	28.49\\
-247	28.4\\
-166	28.49\\
-135	26.17\\
-439	18.85\\
-1001	15\\
-818	26.03\\
-559	33.26\\
-220	37.83\\
-36	39.27\\
-148	39.69\\
-51	38.74\\
-101	38.45\\
49	38.53\\
-28	39.99\\
-66	59.25\\
-46	50\\
212	48.72\\
151	50\\
340	41.09\\
673	42.44\\
799	42.44\\
730	39.27\\
242	31.15\\
241	29.51\\
457	31.15\\
400	29.54\\
560	31.75\\
543	31.75\\
773	35.56\\
1161	40.29\\
1289	45\\
1364	49.4\\
1374	53.23\\
1101	49.99\\
1231	52\\
1335	45\\
1388	42.97\\
1446	44.99\\
1030	69.94\\
1115	52.71\\
1254	50.97\\
981	44.41\\
652	38.72\\
1112	41.76\\
1149	37.29\\
157	32\\
-325	27.17\\
-514	22.64\\
-139	28.72\\
-120	20.14\\
47	19.55\\
-254	29.3\\
-538	26\\
-511	34.22\\
-179	44.94\\
127	47.97\\
211	55.5\\
20	58.2\\
242	59.94\\
80	42.44\\
97	41.94\\
-39	44.94\\
-113	70.61\\
-42	69.94\\
-6	45.39\\
-100	44.96\\
-146	39.96\\
160	39.48\\
446	36.64\\
233	29.99\\
-39	28.75\\
-133	20.73\\
50	27.44\\
68	29.54\\
116	29.84\\
24	29.87\\
-275	29.99\\
-696	27.51\\
-412	32.83\\
-163	34.09\\
-102	39.94\\
-163	40.11\\
-158	41.54\\
-38	42.23\\
243	45.12\\
606	49.94\\
813	65\\
938	66.69\\
1025	52.39\\
947	60\\
1100	48.72\\
1058	51.85\\
1100	66\\
1100	49.95\\
1100	41.25\\
1100	42.24\\
1100	39.05\\
1100	35.05\\
1100	38.73\\
1100	40.74\\
1100	46.66\\
1100	47.99\\
1100	54.07\\
1100	53.73\\
1100	57.05\\
1100	57.37\\
1100	52.74\\
1100	50.6\\
1100	47.31\\
645	51.88\\
382	73.2\\
742	60.12\\
1162	56.97\\
1282	53.8\\
1400	50.14\\
1400	50.12\\
1400	49.79\\
1206	43.36\\
1500	40.06\\
1494	35.27\\
1333	33.76\\
1226	33.71\\
1268	36.22\\
1169	40.25\\
1410	43.7\\
1500	49.04\\
1500	50.3\\
1500	51.3\\
1421	51\\
1500	50.84\\
1500	49.03\\
1500	47.06\\
1500	44.4\\
1249	47.53\\
888	55.01\\
1500	58.15\\
1500	55.54\\
1500	51.63\\
1485	46.39\\
1500	49.73\\
1500	49.23\\
1500	48.76\\
1500	43.43\\
1500	41.19\\
1420	34.94\\
1291	32.32\\
1500	36.08\\
1494	41\\
1081	46.31\\
438	46.85\\
745	50.39\\
1135	52.35\\
940	51.45\\
1154	51.77\\
1261	50.2\\
1387	45.75\\
1169	41.38\\
1132	41.77\\
1142	48.76\\
1500	52.92\\
1500	52.16\\
1500	48.01\\
1401	45.39\\
1401	48.43\\
1401	49.64\\
};
\addplot [color=mycolor2,solid,line width=2.0pt,forget plot]
  table[row sep=crcr]{%
-1511	29.8224904506903\\
-1411	30.5270395689944\\
-1311	31.2315886872986\\
-1211	31.9361378056027\\
-1111	32.6406869239068\\
-1011	33.3452360422109\\
-911	34.0497851605151\\
-811	34.7543342788192\\
-711	35.4588833971233\\
-611	36.1634325154274\\
-511	36.8679816337316\\
-411	37.5725307520357\\
-311	38.2770798703398\\
-211	38.9816289886439\\
-111	39.6861781069481\\
-11	40.3907272252522\\
89	41.0952763435563\\
189	41.7998254618604\\
289	42.5043745801646\\
389	43.2089236984687\\
489	43.9134728167728\\
589	44.6180219350769\\
689	45.3225710533811\\
789	46.0271201716852\\
889	46.7316692899893\\
989	47.4362184082934\\
1089	48.1407675265976\\
1189	48.8453166449017\\
1289	49.5498657632058\\
1389	50.2544148815099\\
1489	50.9589639998141\\
};
\end{axis}
\end{tikzpicture}%
    \caption{Exchange/Price bewteen NL and BE}
    \label{fig:Netherlands}
\end{figure}
\end{minipage} \\
\begin{figure}[H]
    \centering
    \setlength\fheight{4cm}
    \setlength\fwidth{0.8\textwidth}
    % This file was created by matlab2tikz.
% Minimal pgfplots version: 1.3
%
%The latest updates can be retrieved from
%  http://www.mathworks.com/matlabcentral/fileexchange/22022-matlab2tikz
%where you can also make suggestions and rate matlab2tikz.
%
\definecolor{mycolor1}{rgb}{0.04314,0.51765,0.78039}%
\definecolor{mycolor2}{rgb}{0.84706,0.16078,0.00000}%
%
\begin{tikzpicture}

\begin{axis}[%
width=\fwidth,
height=\fheight,
at={(0\fwidth,0\fheight)},
clip mode=individual,
scale only axis,
separate axis lines,
every outer x axis line/.append style={black},
every x tick label/.append style={font=\color{black}},
xmin=-2500,
xmax=4000,
xlabel={Power Exchange},
xmajorgrids,
xtick={-2000,-1500,-1000,-500,0,500,1000,1500,2000,2500,3000,3500},
every outer y axis line/.append style={black},
every y tick label/.append style={font=\color{black}},
ymin=-50,
ymax=200,
ylabel={Market Price [\euro/MWh]},
ymajorgrids,
title style={font=\bfseries},
title={France and Netherlands combined}
]
\addplot [color=mycolor1,line width=1.0pt,mark size=0.3pt,only marks,mark=*,mark options={solid},forget plot]
  table[row sep=crcr]{%
239	15.15\\
-183	12.96\\
-266	12.09\\
-69	11.7\\
-365	11.66\\
-483	11.35\\
-475	9.85\\
-1246	9.54\\
-2107	9.49\\
-1911	11.64\\
-1679	11.94\\
-1720	13.15\\
-1677	15.24\\
-1404	13.69\\
-1485	12.43\\
-1448	9.92\\
-1353	12.12\\
-1133	15.24\\
-1488	15.73\\
-1522	17.73\\
-1107	15.63\\
-903	13.93\\
-525	15.1\\
-94	12.95\\
-366	9.62\\
-585	7.64\\
-675	4.96\\
-768	0.06\\
-746	1.05\\
-706	7.08\\
-332	12.5\\
-314	21.31\\
34	30.44\\
165	35.48\\
333	33.06\\
462	33.78\\
397	37.97\\
429	37.42\\
411	36.24\\
528	32.18\\
757	33.56\\
369	52.94\\
353	66.7\\
410	53.53\\
287	39.54\\
56	35.9\\
295	35.47\\
236	30.64\\
20	27.4\\
24	25.23\\
-30	15.63\\
-236	8.74\\
-209	11.28\\
89	12.34\\
-223	26.02\\
165	31.66\\
415	31.96\\
367	31.94\\
628	30.96\\
695	31.45\\
508	39.05\\
438	30.99\\
303	30.46\\
319	30.43\\
406	31.12\\
780	36.98\\
200	34.97\\
330	42.18\\
66	30.39\\
-128	26.32\\
457	31.82\\
593	32.22\\
505	11.94\\
831	10.5\\
397	7.93\\
229	5.23\\
155	4.86\\
211	8.96\\
-559	8.72\\
-388	9.91\\
132	11.5\\
472	12.76\\
517	13.36\\
623	14.03\\
278	15.64\\
582	14.24\\
1087	14.48\\
1599	30\\
1341	12.86\\
1159	16.79\\
645	21.08\\
523	18.1\\
235	16.14\\
677	13.84\\
819	14.89\\
1012	17.34\\
1057	15.46\\
869	14.69\\
1071	13.33\\
854	10.96\\
787	9.83\\
806	11.66\\
-346	11\\
-354	11.38\\
-132	13.77\\
295	27.42\\
398	30.9\\
458	32.45\\
465	32.07\\
651	29.91\\
837	28.41\\
892	16.89\\
880	15.39\\
1599	46.14\\
1272	40.13\\
1147	38.49\\
848	34.67\\
465	29.4\\
583	28.81\\
706	22.08\\
335	13.84\\
-24	11.89\\
-182	9.85\\
-323	5.09\\
-240	3.79\\
-134	8.24\\
778	17.29\\
1361	48.6\\
1004	46.6\\
1222	40\\
1730	67.8\\
1761	67.3\\
1599	46.08\\
1599	46.08\\
1577	34.5\\
1599	46.08\\
1612	48.99\\
2128	84.71\\
1526	54.57\\
1055	50.7\\
1349	41.75\\
930	28.34\\
1299	38.52\\
1124	33.69\\
564	9.26\\
858	9.98\\
536	3.88\\
422	0.07\\
486	2.05\\
890	6.52\\
1126	26.07\\
1599	34.99\\
1399	49.42\\
753	52\\
596	52\\
359	54.36\\
182	52.78\\
689	49.99\\
933	45.93\\
1180	43\\
1173	46\\
1151	54.44\\
1075	61.93\\
657	56.29\\
998	40.02\\
1159	31.58\\
1304	36.08\\
1056	33.51\\
846	22.01\\
1068	14.61\\
956	12.67\\
826	11.34\\
950	9.99\\
1249	22.41\\
1217	30.31\\
915	46.42\\
1430	57.96\\
1123	57.82\\
980	53.7\\
844	54.51\\
1330	56.46\\
1436	54.3\\
1511	51.11\\
1642	47\\
1765	52.06\\
1833	70.01\\
1773	65\\
1225	61.89\\
1442	48.39\\
1413	38.97\\
1561	39.86\\
1236	38.79\\
787	27.18\\
991	26.5\\
660	24.83\\
440	18.34\\
473	12.96\\
820	25.04\\
920	34.65\\
1032	48\\
966	56.97\\
348	48.54\\
375	46.99\\
370	47.38\\
672	48.92\\
861	46.65\\
836	41.82\\
902	37.84\\
1066	43.83\\
731	52.66\\
764	62.06\\
270	59.78\\
44	47.89\\
922	40.66\\
1274	45.35\\
1123	44.29\\
939	42.64\\
1049	38\\
1248	32.2\\
1163	16.27\\
1260	15.49\\
1648	33.77\\
1631	46.7\\
1831	62.87\\
1889	59.51\\
1474	61.62\\
1236	59.99\\
1509	58.32\\
938	57.73\\
1290	53.76\\
1462	50.84\\
1552	46.87\\
1683	53.65\\
1811	74.86\\
1658	66.02\\
1094	61.5\\
552	57.3\\
688	46.82\\
882	50\\
949	47.33\\
1072	40.36\\
634	28.17\\
1346	35.03\\
1217	28.26\\
1109	26.76\\
1116	28.27\\
997	34.12\\
22	38.34\\
896	43.65\\
757	45\\
863	46.36\\
800	46.34\\
706	49.72\\
749	41.89\\
710	37.94\\
827	38.34\\
1112	40.28\\
1537	51.79\\
1230	59.12\\
903	58.65\\
429	50.94\\
901	44.8\\
1260	52.81\\
655	48.69\\
1204	43.47\\
969	40\\
715	25.86\\
611	16.97\\
1373	14.51\\
1256	16.47\\
1392	16.18\\
1404	23.1\\
1318	31.5\\
1768	44.33\\
787	25.75\\
624	23.98\\
522	29.73\\
405	24.67\\
1159	25.7\\
1223	27.07\\
834	31.11\\
1224	39.66\\
963	41.96\\
642	46.09\\
351	37.35\\
66	32.85\\
120	37.62\\
781	32.17\\
526	26\\
1040	27.86\\
790	27.6\\
633	13.45\\
738	14.24\\
1032	26.71\\
1367	33.11\\
1581	50.84\\
1446	51.45\\
1446	53.89\\
2027	64.36\\
2018	65.14\\
1775	55.89\\
1791	55\\
1751	54.94\\
1832	54.97\\
1906	57.93\\
1767	67.48\\
1828	64.41\\
1311	60.71\\
1236	51.62\\
1372	44.67\\
1731	43.21\\
2077	43.21\\
825	31.14\\
126	29.65\\
927	29.56\\
780	24.17\\
860	19.72\\
1207	30.34\\
883	36.98\\
1279	60.09\\
1073	65.94\\
1255	66.06\\
1285	66.95\\
1322	67.51\\
1140	60.02\\
1204	55.01\\
1541	52.54\\
1564	51.86\\
1648	55.88\\
1921	70.68\\
1528	68\\
1403	53.7\\
1573	54.94\\
1222	45.14\\
1470	44.94\\
1310	39.99\\
1194	33.03\\
992	30.66\\
876	29.92\\
744	29.63\\
600	29.54\\
983	30.65\\
1215	39.94\\
1217	55.67\\
756	62.65\\
731	58.07\\
940	55.47\\
943	53.9\\
794	54.16\\
823	54.35\\
866	50.75\\
945	50.51\\
1010	52.1\\
965	56.49\\
852	62.96\\
1019	61.29\\
1161	47\\
1104	38.6\\
861	39.46\\
825	35.75\\
688	30.06\\
983	28.92\\
1014	25.6\\
841	21.42\\
891	18.42\\
1151	28.66\\
961	33.35\\
1588	50.09\\
1193	48.34\\
967	51.57\\
831	50.84\\
977	52.2\\
1085	51.52\\
1648	55\\
1600	54.94\\
1771	53.62\\
1837	55\\
2012	79.94\\
1364	55.55\\
1174	55\\
1241	46.02\\
1032	35.23\\
1196	37.69\\
2024	38\\
897	29.53\\
1399	28.17\\
1073	19.58\\
912	11.23\\
984	11.18\\
917	20.74\\
1078	34.03\\
867	48.97\\
666	49.96\\
1091	47.71\\
1069	47.91\\
927	47.96\\
948	48.68\\
1109	46.17\\
1239	42.32\\
1312	47.44\\
1465	48.87\\
1461	68.99\\
1097	52.75\\
889	49.43\\
1021	43.74\\
1048	33.33\\
1433	34.45\\
1340	35.9\\
1146	30.5\\
680	29.79\\
1010	28.58\\
1218	26.79\\
1109	18.63\\
954	22.25\\
191	22\\
238	30.25\\
513	33.34\\
896	39.96\\
859	35.17\\
721	32.44\\
456	31.28\\
527	30.89\\
740	30.68\\
901	30.88\\
1362	33.17\\
1399	54.53\\
1399	45.05\\
1151	39.72\\
1178	30.21\\
925	28.63\\
1044	30.4\\
1327	38.63\\
887	36.1\\
389	30.14\\
1074	26.15\\
871	9.48\\
847	12.8\\
903	16.41\\
-17	12.18\\
77	15.76\\
441	29.99\\
809	37.57\\
1125	44.38\\
1228	46.47\\
1287	49.96\\
1360	44.12\\
1417	35.64\\
1516	27.42\\
1518	29.67\\
1687	60.71\\
1599	60\\
1524	46.54\\
1289	42.51\\
1379	40.02\\
1530	41.86\\
1660	38\\
1739	34.75\\
1815	29.46\\
1741	29.21\\
1599	14.99\\
1633	29.33\\
1320	25.5\\
1840	45.93\\
2148	71.97\\
1716	70.92\\
2033	64.5\\
2088	62.93\\
2102	64.44\\
1874	59.61\\
2344	58.04\\
2327	54.16\\
2390	49.7\\
2405	52.53\\
2516	64.91\\
2199	77.99\\
1809	63.97\\
1863	62.7\\
1467	50.4\\
1771	58.17\\
1720	49.38\\
1711	44.79\\
1763	34.35\\
2071	30.65\\
2009	30.3\\
1959	30.65\\
2263	33.68\\
1917	45.86\\
2409	59.91\\
1754	62.98\\
1251	60.19\\
1095	63.32\\
1312	65.65\\
1264	62.91\\
1530	61.46\\
1614	57.41\\
1707	50.81\\
1747	54.09\\
2187	60.23\\
1556	60.57\\
1527	62.97\\
1522	52.97\\
989	48.03\\
1135	51.45\\
1314	47.44\\
1311	45.67\\
1712	43.09\\
1391	34.97\\
1272	31.02\\
1300	31.14\\
964	35.41\\
1008	48.5\\
1111	67.49\\
-166	66.49\\
1	65.66\\
-202	60.88\\
-235	62.79\\
-441	61.44\\
-58	59.37\\
244	58.96\\
458	54.76\\
544	61.62\\
866	71.58\\
395	80.06\\
474	64.04\\
649	62\\
382	53.5\\
654	61.58\\
863	53.5\\
1074	41.81\\
1727	41.5\\
1748	39.22\\
1603	31.32\\
1661	31.17\\
1668	34.03\\
1775	43.13\\
2115	60.59\\
1219	65.1\\
1266	65.05\\
1264	64.92\\
1233	62.17\\
1208	60.92\\
1584	60.91\\
1594	54.77\\
1717	49.69\\
1704	48.08\\
1727	55.17\\
1197	63.96\\
862	57.53\\
775	53.78\\
650	45.45\\
875	53.8\\
1205	46.91\\
556	35.05\\
1162	32.29\\
1035	30.34\\
1039	29.99\\
1137	30.53\\
806	32.17\\
1372	43.42\\
1238	61.6\\
632	62.5\\
717	63.3\\
611	61.59\\
590	63.28\\
574	60.39\\
644	59.33\\
850	58.97\\
839	53.19\\
953	53.03\\
998	64.41\\
531	64.94\\
330	63.71\\
418	54.64\\
487	50.15\\
933	47.09\\
914	49\\
1264	37.68\\
804	32.05\\
671	31.1\\
1248	30.32\\
1170	29.83\\
1076	29.04\\
1300	30.9\\
568	31.58\\
905	39.27\\
1033	42.04\\
1162	43.63\\
1116	44.11\\
928	45.82\\
998	37.68\\
829	32.69\\
810	32.69\\
848	33.57\\
1119	38.13\\
922	48.04\\
916	43.14\\
526	32.57\\
188	30.05\\
413	32.86\\
662	38.23\\
126	22.11\\
-36	19.14\\
-42	14.85\\
4	13.14\\
-24	13.22\\
174	12.73\\
0	9.53\\
6	14.05\\
53	14.04\\
383	22.19\\
470	22.69\\
332	22.85\\
-45	23.42\\
205	19.08\\
-29	15.23\\
-153	12.89\\
-143	8.26\\
344	17.66\\
-47	22.42\\
-61	26.7\\
467	31.96\\
244	23.75\\
565	28.24\\
599	20.29\\
1	12.63\\
128	11.74\\
248	12.81\\
94	7.1\\
205	8.77\\
633	13.09\\
681	30.14\\
885	47.43\\
580	51.07\\
331	52.47\\
263	53.47\\
146	52.73\\
304	56.05\\
244	50.71\\
351	51.92\\
417	51.92\\
502	51.47\\
167	55.53\\
272	79.92\\
-106	64.25\\
282	47.25\\
12	38.71\\
236	40.2\\
149	38.51\\
367	36.41\\
550	32.92\\
669	30.28\\
601	29.57\\
603	30.16\\
920	31.21\\
657	41.47\\
846	55.05\\
456	56.58\\
504	54.52\\
508	52\\
468	52.44\\
353	49.94\\
733	47.44\\
876	46.5\\
949	48\\
1013	49.77\\
666	57.75\\
538	70\\
127	62.09\\
282	56.94\\
142	47.4\\
255	53.42\\
263	48.28\\
885	38.65\\
982	30.54\\
889	29.63\\
725	28.98\\
700	29.55\\
961	31.08\\
936	37.1\\
961	55.38\\
697	52.95\\
683	52.9\\
351	57.44\\
294	56.78\\
59	53.55\\
366	54\\
533	50.01\\
851	47.44\\
1005	47.02\\
698	61.52\\
772	83.27\\
495	72.99\\
261	54.43\\
350	49.5\\
786	54.11\\
888	52.05\\
806	37.75\\
1210	34.94\\
1075	34.29\\
1102	29.92\\
1183	30.32\\
729	34.43\\
1190	45.2\\
1385	53.07\\
476	54.73\\
952	64.78\\
832	59.99\\
967	64.16\\
743	59.72\\
980	54.48\\
1140	51.89\\
1266	51.5\\
1352	51.56\\
1487	55.12\\
1131	67.98\\
810	59.67\\
1066	50.5\\
721	45.81\\
1363	46.52\\
1400	47.1\\
1497	42.96\\
1431	42.5\\
1595	36.54\\
1463	33.06\\
1580	31.95\\
1817	40.03\\
1514	46.9\\
1425	69.94\\
577	71.95\\
463	72.19\\
609	65.88\\
339	60.93\\
7	52.93\\
463	48.14\\
328	45\\
268	44.1\\
337	44.23\\
403	52.92\\
260	61.76\\
-318	56.35\\
-172	49.08\\
-160	40\\
162	41.99\\
211	39.17\\
201	31.04\\
-254	28.94\\
-140	26.63\\
2	20.87\\
-115	20.18\\
-77	20.71\\
122	26.82\\
-680	28.7\\
-429	37.27\\
-90	42.65\\
17	43.28\\
16	44.55\\
-123	48.59\\
-159	40.24\\
-293	32.87\\
-247	32.8\\
-147	33.28\\
76	43.56\\
-48	50.02\\
-248	44.94\\
57	41.24\\
-268	35.81\\
53	35.01\\
235	35.96\\
-169	41.62\\
-512	31.96\\
-741	25.8\\
-93	19.38\\
-158	13.18\\
-142	14.07\\
-372	16.54\\
-961	18.85\\
-993	22.07\\
-719	30.83\\
-464	31.96\\
-496	31.82\\
-496	34.15\\
-460	32.14\\
-537	30.8\\
134	31.03\\
668	32.95\\
160	36.15\\
158	48.68\\
143	54.57\\
125	53.78\\
51	45.76\\
329	49.81\\
343	48.84\\
195	44.37\\
2	37.8\\
675	36\\
453	31.56\\
517	31.52\\
912	36.19\\
1037	54.1\\
1483	71.89\\
893	72.94\\
513	67.5\\
268	57.18\\
281	54.63\\
351	52.9\\
742	51.96\\
908	46.82\\
1100	47.14\\
1227	46.76\\
1297	54.09\\
1193	67.8\\
888	59.96\\
931	57.98\\
497	48.72\\
623	52.2\\
659	52.27\\
677	44.42\\
619	44.96\\
808	40.9\\
696	30.63\\
684	30\\
125	32.38\\
508	47.13\\
348	60.3\\
-32	58.49\\
48	53.88\\
356	51.99\\
529	52.83\\
434	54.68\\
582	51.68\\
833	48.05\\
903	47.59\\
967	48.03\\
1179	53.05\\
1015	63.99\\
554	55.73\\
631	48\\
554	41.26\\
693	44.08\\
772	42.46\\
298	30.53\\
-103	29.98\\
476	28.99\\
300	25.71\\
332	18.11\\
621	28.08\\
374	33.51\\
-346	45.16\\
-271	49.94\\
-211	52.61\\
149	58.57\\
90	54.16\\
-191	48.08\\
146	52.2\\
307	50.09\\
514	43.92\\
698	42.55\\
705	49.69\\
391	67.38\\
51	53.32\\
-78	36.79\\
-614	57.1\\
-368	52\\
-260	45\\
842	42.28\\
691	39.94\\
482	34.79\\
416	28.27\\
500	28.67\\
-7	32.2\\
486	44\\
171	55.87\\
204	56.7\\
100	58.84\\
279	61.77\\
256	60.43\\
126	57.28\\
665	55.91\\
983	55.29\\
1084	49.94\\
1085	47.89\\
884	55.89\\
586	66.01\\
443	60.02\\
96	48.05\\
230	37.12\\
588	33.71\\
828	39.61\\
256	23.64\\
633	19.51\\
342	14.66\\
159	12.18\\
226	8.54\\
187	15.42\\
505	30.68\\
546	47.79\\
485	45.71\\
555	47.63\\
800	47.94\\
1040	51.34\\
816	48.37\\
854	42.94\\
851	36.86\\
831	32.31\\
916	31.3\\
1205	36.16\\
884	59.92\\
568	54.71\\
676	43.31\\
377	30.49\\
864	38.37\\
909	40.84\\
325	28\\
-241	26.43\\
269	25.9\\
-18	24.93\\
-133	10.52\\
-139	16.2\\
73	24.04\\
-752	27.13\\
-512	29.51\\
-160	29.48\\
190	28.07\\
270	26.76\\
210	28.03\\
292	25.5\\
176	25.96\\
886	28.81\\
687	28.03\\
211	31.17\\
255	42.58\\
-22	33.88\\
-419	27.57\\
-324	24.29\\
-160	27.55\\
235	23.6\\
-239	14.23\\
-738	13.02\\
-710	12.87\\
-855	11.77\\
-997	9.72\\
-912	9.36\\
-785	10.63\\
-1285	10.54\\
-1026	11.04\\
-674	19.16\\
-399	17.7\\
-176	17.29\\
62	14.68\\
194	13.16\\
-25	10.5\\
-27	8.97\\
-82	12.42\\
15	14.69\\
404	35.79\\
140	36.81\\
192	31.23\\
-160	27.97\\
49	30.04\\
515	25.31\\
808	36.13\\
918	33.87\\
752	27.22\\
530	23.86\\
-153	23.08\\
1173	33.18\\
1164	45.43\\
1883	59.42\\
1826	60.99\\
1614	58.79\\
972	60.02\\
897	54.96\\
841	56.12\\
1570	56.82\\
1603	55.63\\
1685	51.47\\
1710	47.44\\
1847	54.94\\
1354	79.49\\
1044	54.32\\
1326	50.06\\
1180	46.57\\
1417	47.59\\
1594	42.83\\
1205	30.75\\
1487	32.07\\
1203	29.45\\
1018	28.27\\
970	28.05\\
1015	30.72\\
880	41.12\\
343	53.8\\
324	55.34\\
447	56.62\\
215	54.1\\
338	51.74\\
171	50.38\\
465	49.25\\
630	42.49\\
722	39.56\\
785	41\\
737	51.9\\
650	58.97\\
391	67.49\\
205	50\\
392	46.48\\
902	49.48\\
985	44.09\\
1090	45\\
886	41.94\\
698	37.84\\
549	28.43\\
580	29.36\\
128	36.72\\
509	45.54\\
797	59.3\\
249	65.77\\
-113	68.42\\
-278	60.87\\
-312	55.9\\
-289	50.33\\
251	49.42\\
277	45.92\\
384	43.34\\
675	40.93\\
478	44.99\\
336	58.94\\
-8	54.37\\
-186	52.03\\
55	42.78\\
338	46.13\\
631	44.03\\
610	42.97\\
730	34.54\\
485	32.23\\
437	18.33\\
498	12.47\\
-97	26.67\\
443	42.44\\
-60	55.88\\
153	53.54\\
465	57.79\\
642	56.72\\
909	57.98\\
922	55\\
1243	52.89\\
1178	52.35\\
1226	45.25\\
1427	42.91\\
1228	48.66\\
941	68.92\\
711	58.69\\
309	52.81\\
385	42\\
1004	40.63\\
940	38.04\\
1579	33\\
1549	32.72\\
1364	29.33\\
1237	24.42\\
1329	25.63\\
842	29.44\\
1380	46.4\\
1213	61.12\\
758	64.8\\
689	69.25\\
583	62.12\\
570	55\\
697	52.08\\
1024	47.1\\
1020	44.22\\
1079	36.2\\
1115	40\\
1019	46.18\\
682	53.48\\
781	48.73\\
955	36.97\\
694	30.39\\
1256	30.61\\
1702	42.44\\
253	27.73\\
-210	17.99\\
148	12.38\\
-99	10.36\\
-203	10.11\\
-157	10.33\\
143	11.36\\
-600	13.34\\
-375	18.26\\
-397	22.66\\
35	23.11\\
1	23.08\\
-256	23.76\\
-155	22.48\\
193	21.11\\
264	19.33\\
257	20.46\\
149	27.73\\
190	39.29\\
-124	44.85\\
-270	28.93\\
-328	29.36\\
-208	41.5\\
132	41.56\\
0	25.49\\
-300	14.46\\
-22	12.35\\
-173	9.06\\
-157	5.3\\
-82	8.08\\
-13	11.12\\
-475	9.53\\
-497	10.73\\
-301	13.37\\
-242	16.81\\
-248	17.86\\
-409	18.23\\
-414	13.15\\
-503	11.38\\
-301	10.48\\
-42	12.22\\
322	15.26\\
718	40.15\\
813	47.98\\
733	46.83\\
646	34.73\\
967	39.08\\
1086	31.53\\
756	31.59\\
1090	25.69\\
1003	23.33\\
889	17.52\\
920	19.63\\
1314	26.47\\
1125	43.79\\
1426	58.75\\
1141	53.31\\
796	54\\
791	49.32\\
1059	50\\
764	44.22\\
949	42.07\\
984	34.96\\
1315	37.25\\
556	40.33\\
579	43.03\\
488	82.04\\
404	71.74\\
618	56.51\\
351	47.19\\
772	49.94\\
803	43.43\\
707	31.95\\
1039	31.66\\
801	30.09\\
686	28.86\\
764	29.2\\
587	31.3\\
732	42.03\\
42	56.19\\
90	54.9\\
114	53.79\\
362	51.9\\
643	50.74\\
594	51.42\\
705	48.48\\
801	43.27\\
944	43.44\\
1013	45.72\\
1244	53.91\\
821	73.68\\
588	59.98\\
725	45.94\\
580	39.68\\
781	41.9\\
809	36.47\\
920	32.16\\
1182	29.65\\
960	28.46\\
864	27.67\\
971	28.51\\
541	29.34\\
1240	38.19\\
885	47.45\\
676	49.44\\
751	52.21\\
759	53.93\\
1053	53.96\\
878	50.76\\
1269	49.48\\
1198	49.37\\
1240	47.95\\
1284	44.71\\
1387	47.44\\
1086	70\\
869	59.54\\
978	54.27\\
894	47.02\\
1273	44.18\\
1213	44.96\\
1003	31.28\\
1508	29.47\\
1145	28.87\\
856	27.75\\
817	25.59\\
152	27.85\\
891	34.51\\
585	46.28\\
714	48.34\\
793	48.28\\
902	44.93\\
1329	46.95\\
1247	47.58\\
1318	44.94\\
1424	40.02\\
1505	34.96\\
1497	32.45\\
1599	41.78\\
1475	55\\
1169	52.17\\
1119	47.44\\
922	39.95\\
1266	41.78\\
1376	41.49\\
1338	32.91\\
1388	22.15\\
1151	20.84\\
1026	19.42\\
1108	21.02\\
534	26.64\\
1311	35.83\\
1162	46.31\\
1169	49.03\\
1107	52.95\\
1317	50.07\\
1473	48.45\\
1117	42.88\\
1264	37.61\\
1143	32.16\\
1155	32.1\\
1227	31.68\\
1193	42.9\\
1690	49.44\\
1013	49.96\\
681	49.08\\
447	40.71\\
903	40.93\\
994	41\\
581	35.11\\
220	29.96\\
117	26.09\\
447	21.02\\
362	18.67\\
356	21.15\\
587	27.93\\
-169	29.38\\
169	32.92\\
601	37.43\\
904	37.35\\
1045	36.86\\
971	38.85\\
797	35.47\\
771	29.59\\
620	31.99\\
586	31.33\\
846	37.71\\
723	50.91\\
618	53.1\\
177	44.22\\
90	36.72\\
209	40.34\\
740	36.96\\
436	26.11\\
-137	21.63\\
98	22.07\\
157	19.48\\
40	16.28\\
26	19.86\\
116	15.73\\
-938	15.73\\
-1409	17.1\\
-1092	16.63\\
-688	17.46\\
-558	16.77\\
-521	14.68\\
382	12.98\\
249	11.26\\
335	7.45\\
110	10.84\\
37	11.28\\
378	24.31\\
474	35.49\\
603	27.81\\
398	23.11\\
524	27.1\\
620	25.71\\
-40	14.5\\
415	15.09\\
197	17.22\\
98	13.68\\
167	14\\
-234	18.12\\
455	35.66\\
482	46.95\\
399	50\\
436	46.65\\
577	45.98\\
452	41.79\\
25	40.46\\
578	39.32\\
832	35.99\\
1056	31.07\\
1260	32.97\\
1299	46.16\\
1322	64.95\\
1103	64.73\\
822	48.99\\
586	35.43\\
791	37.44\\
789	33.87\\
153	26.22\\
153	23.87\\
287	22.77\\
137	20.29\\
234	18.97\\
609	25.94\\
778	36.97\\
994	46.95\\
363	46.66\\
102	46.12\\
-148	44.59\\
-116	44\\
153	44.94\\
522	40.49\\
700	36.09\\
976	35.3\\
1089	33.71\\
922	42.61\\
846	72.68\\
729	62.52\\
198	46.87\\
506	39.88\\
612	42.89\\
761	39.59\\
1122	36.05\\
1090	30\\
801	27.29\\
696	26.91\\
797	27.86\\
306	28.48\\
1058	39.22\\
649	45.05\\
614	49.24\\
1031	51.3\\
901	50.42\\
823	50\\
692	46.91\\
720	46.42\\
678	44.96\\
796	43.35\\
441	41.81\\
564	43.99\\
865	71.19\\
734	72.95\\
268	49.85\\
276	45.59\\
769	42.44\\
1099	40.05\\
971	33.99\\
1103	32.03\\
781	30.28\\
583	29.61\\
462	29.48\\
280	32.18\\
459	41.54\\
497	53.34\\
-151	50.1\\
-179	47.1\\
-216	48.14\\
-219	46\\
144	44.24\\
370	44.81\\
402	43.22\\
577	41.94\\
696	40.62\\
707	51.76\\
732	64.96\\
657	57.83\\
460	48.66\\
728	43.21\\
986	41.45\\
911	40.16\\
816	36.25\\
1229	34.94\\
1030	30.19\\
966	28.03\\
991	28.54\\
352	29.94\\
886	39.6\\
327	47.45\\
352	49.45\\
531	49.77\\
865	49.12\\
976	47.42\\
1134	46.72\\
1166	44.7\\
1136	42.31\\
1117	41.92\\
1078	42.98\\
986	47.3\\
343	64.68\\
123	59.92\\
256	47.28\\
224	39.75\\
473	42.95\\
574	42\\
491	40.61\\
605	35.72\\
679	33.79\\
510	30.22\\
462	29.28\\
478	29.33\\
-433	28.98\\
-320	33.71\\
-351	39.51\\
-87	47.4\\
90	50.34\\
114	48.18\\
507	47.97\\
350	43.25\\
171	38.63\\
128	36.46\\
26	33.77\\
-1	41.27\\
338	63.21\\
298	74.76\\
-113	45.93\\
83	42.66\\
357	48.38\\
485	44.97\\
189	44.94\\
-151	39.4\\
-208	32.28\\
5	29.31\\
-273	27.57\\
-341	27.86\\
-282	27.33\\
-808	27.04\\
-1079	27.49\\
-793	32.39\\
-743	33.69\\
-599	33.67\\
-665	34.38\\
-544	27.45\\
-711	23.55\\
-704	24.59\\
-688	25.42\\
-603	28.22\\
-723	33.96\\
-589	42.62\\
-127	39.44\\
-409	31.89\\
-380	36.16\\
-353	32.59\\
-729	26.52\\
-196	25.11\\
-405	24.08\\
-448	22.14\\
-251	24.03\\
-358	27.01\\
94	43.93\\
-249	49.94\\
-454	51.33\\
-386	53.68\\
-134	51.43\\
138	57.13\\
-166	47.83\\
14	49.41\\
110	46.48\\
498	38.95\\
590	40.98\\
724	44.96\\
257	72.91\\
26	80.69\\
63	47.81\\
24	43.99\\
480	45.3\\
463	43.34\\
1066	45.03\\
1088	39.19\\
1120	33.29\\
1051	28.99\\
1168	29.68\\
865	33.28\\
802	47.18\\
999	53.9\\
588	57.9\\
691	57.32\\
595	51.02\\
634	51.98\\
828	52.3\\
965	50\\
947	48.59\\
1040	46.51\\
1138	45.92\\
1244	45.87\\
961	75.64\\
1008	79.38\\
712	54.77\\
739	50.68\\
1039	49.01\\
948	50.54\\
724	40.21\\
375	39.94\\
821	42.1\\
842	34.72\\
961	33.37\\
1063	41.28\\
746	47.32\\
815	62.16\\
660	67.76\\
603	68.08\\
784	53.01\\
635	44.96\\
434	43.85\\
437	44.19\\
579	40.21\\
890	38.02\\
1015	38.08\\
1031	46.54\\
706	58.03\\
763	68.37\\
976	49.04\\
730	42.58\\
898	45.99\\
716	44.1\\
757	37.43\\
710	36.72\\
546	34.14\\
642	28.81\\
765	29.19\\
909	33.98\\
586	42.85\\
48	55.33\\
-29	56.97\\
-154	56.22\\
364	50\\
316	48.58\\
41	46.41\\
19	44.92\\
219	40.51\\
569	34.11\\
637	37\\
863	45\\
344	61.59\\
409	69.75\\
832	53.9\\
318	42.17\\
369	45.73\\
281	40.97\\
-85	40.01\\
403	33.98\\
-96	32.22\\
152	28.9\\
215	29.1\\
331	32.06\\
-293	40.9\\
-664	48.26\\
-521	53.76\\
-389	62\\
-130	49.17\\
-248	45.36\\
-374	44.82\\
-315	37.93\\
-203	35.64\\
160	31.97\\
337	31.94\\
452	39.36\\
171	47.09\\
219	49.26\\
378	45.95\\
24	41.92\\
242	44.28\\
523	43.68\\
177	39.97\\
-110	33.31\\
-554	30.44\\
-267	25.71\\
-383	24.96\\
-316	25.82\\
-390	26.92\\
-588	29.6\\
-142	31.96\\
-47	31.51\\
-197	28.19\\
-379	26.17\\
-462	24.42\\
-628	20.02\\
-404	15.91\\
-193	12.03\\
220	16.02\\
667	22.55\\
662	35.58\\
296	42.77\\
-108	30.03\\
-263	25.14\\
68	25.47\\
27	23.65\\
-480	15.67\\
-253	13.75\\
-575	12.1\\
-744	11.4\\
-1500	10.35\\
-1491	11.4\\
-994	11.49\\
-1562	11.4\\
-1473	11.55\\
-1372	12.05\\
-1440	11.54\\
-952	12.11\\
-810	12.26\\
-587	10.29\\
-561	6.2\\
-1082	1.75\\
-981	1.87\\
-579	9.83\\
58	22.41\\
354	41.73\\
257	34.86\\
-35	28.07\\
54	30.09\\
-22	26.92\\
328	24.26\\
287	22.62\\
124	22.73\\
144	22.76\\
407	22.11\\
90	25.57\\
1008	43.5\\
917	48.28\\
615	48.12\\
364	44.76\\
365	40.9\\
203	37.42\\
76	36.46\\
172	35.85\\
240	33.8\\
431	30.13\\
657	30.19\\
955	36.36\\
850	56.07\\
276	75.46\\
111	46.32\\
306	38.68\\
391	35.17\\
402	30.93\\
338	25.83\\
532	25.9\\
263	24.63\\
173	24.27\\
306	23.78\\
-105	26.5\\
547	35.08\\
458	42.55\\
487	44.58\\
443	40.03\\
1015	40.2\\
873	36.07\\
519	31.88\\
445	30.65\\
623	28.18\\
755	28.72\\
892	30.91\\
1183	36.26\\
1240	49.94\\
740	54.1\\
1058	42.56\\
709	34.04\\
802	40.03\\
771	32.99\\
273	28.87\\
672	28.04\\
460	26.65\\
428	25.15\\
564	25.69\\
688	28.09\\
866	35.8\\
474	44.54\\
316	43.73\\
805	41.27\\
552	32.59\\
329	33.67\\
190	30\\
260	28.78\\
462	28.42\\
691	29.25\\
876	30.79\\
1299	37.04\\
1014	49\\
945	75.98\\
853	48.43\\
816	42.01\\
1076	38.28\\
1047	32.08\\
1132	31.35\\
228	30.65\\
810	29.96\\
692	30\\
713	30.11\\
480	31.43\\
1348	43.03\\
1127	53.46\\
891	49.84\\
441	41.27\\
400	34.74\\
152	29.77\\
-73	30.58\\
-31	28.42\\
9	28.78\\
160	30.95\\
292	35.54\\
567	30.84\\
222	52.76\\
517	75.55\\
614	48.69\\
893	43.63\\
1261	42.87\\
1141	40.69\\
776	34.96\\
1011	32.01\\
1186	30.99\\
1097	29.57\\
1243	29.65\\
586	32.89\\
1184	42.04\\
729	49.94\\
757	47.93\\
835	42.94\\
1387	41.26\\
1191	38.83\\
912	37.01\\
888	34.91\\
781	30.96\\
878	29.53\\
973	30.91\\
1239	37.91\\
1294	45.07\\
1335	53.95\\
1574	49.94\\
1395	38.64\\
1620	42.44\\
1684	38.99\\
902	28.64\\
602	18.86\\
436	21.1\\
419	11.3\\
383	7.58\\
435	11.81\\
446	13.36\\
536	14.02\\
870	22.83\\
1353	20.01\\
1394	16.52\\
1423	15.8\\
1266	21.57\\
1157	15.2\\
1096	12.55\\
1107	11.78\\
1169	11.26\\
1466	18.81\\
1601	44.94\\
1599	48\\
1469	25.54\\
1222	23.29\\
1233	20.33\\
977	19.29\\
639	15.89\\
1054	14.23\\
748	13.29\\
571	11.77\\
649	11.11\\
-438	10.12\\
-325	11.23\\
-389	10.91\\
-162	11.44\\
61	10.77\\
149	12.34\\
167	13.02\\
80	13.95\\
546	11.07\\
465	9.33\\
509	3.07\\
738	5.41\\
1163	9.43\\
1122	22.78\\
1611	52\\
1599	49\\
1266	21.31\\
1317	20.51\\
1443	28.62\\
1498	29.73\\
1133	26.33\\
1172	23.93\\
1139	20.73\\
1226	21.96\\
919	26.41\\
1950	42\\
1619	50.81\\
1444	49\\
1449	56.84\\
1828	59.16\\
1785	65.27\\
1593	50\\
1776	50\\
1801	49.96\\
1894	49.08\\
2005	51.87\\
2141	67.78\\
1763	47.44\\
1814	69.94\\
1482	54.86\\
1580	44\\
1624	45\\
1609	44.25\\
1820	31.57\\
2067	32.18\\
1752	29.99\\
1682	24.07\\
1771	24.89\\
2090	28.06\\
1938	40.44\\
1863	49.49\\
1422	50.12\\
1382	63.77\\
1275	55\\
1530	61.99\\
1703	58.13\\
1607	52.24\\
1505	45.13\\
1421	42\\
1383	41.36\\
1578	42.44\\
1437	39.68\\
1178	46.28\\
1112	41.55\\
1100	39\\
1367	32.73\\
1451	30.33\\
1064	22.2\\
1380	22.46\\
1264	14.3\\
1285	22.46\\
1492	29.66\\
1250	27.44\\
1889	42.36\\
2121	49.79\\
1299	47.99\\
1199	48\\
1199	39.01\\
1189	34.72\\
964	33.41\\
912	32.59\\
883	31.48\\
1183	28.19\\
1199	42.44\\
1568	47.59\\
1281	42.44\\
1498	54.97\\
1504	49\\
1500	45\\
1661	44\\
1544	41.93\\
1250	26.47\\
1443	26.72\\
1209	24.54\\
1140	17.18\\
1241	18.05\\
1445	24.57\\
1524	37.54\\
1676	46.15\\
856	41.33\\
917	35.3\\
943	31.3\\
886	29.01\\
555	26.8\\
521	26.56\\
539	25.76\\
850	26.52\\
1024	26.84\\
1223	28.95\\
1064	36.94\\
1139	45.07\\
1399	32\\
1191	27.84\\
1389	31.24\\
1263	30.47\\
1341	21.24\\
995	12.08\\
835	11.32\\
866	10.25\\
1126	10.65\\
978	18.63\\
1588	29.62\\
1688	40\\
1399	39.18\\
1273	38.33\\
1399	44.54\\
1920	63.62\\
1631	57.66\\
1573	55.89\\
1417	45\\
1439	39.96\\
1452	39.94\\
1399	38.93\\
851	42.43\\
953	55\\
689	47.39\\
1084	43.08\\
1309	43.08\\
1192	36.5\\
1094	33.94\\
576	28.54\\
946	24.08\\
703	19.07\\
608	13.66\\
463	17.92\\
802	22.76\\
32	24.31\\
253	29.83\\
671	32.37\\
867	32.33\\
935	32.48\\
936	31.22\\
792	27.59\\
826	27.84\\
796	26.45\\
676	27.99\\
844	31.39\\
626	38.04\\
1030	45.01\\
710	36.93\\
930	31.7\\
1103	30.99\\
1195	24.69\\
706	22.91\\
532	20.54\\
605	20.14\\
502	20.59\\
647	17.65\\
577	19.82\\
313	17.05\\
-66	16.07\\
-83	21.36\\
176	24.42\\
290	26.9\\
369	30.58\\
454	29.26\\
771	24.04\\
826	23.46\\
959	18.55\\
1021	17.34\\
1094	19.29\\
1486	40\\
1350	45.12\\
1299	42\\
1135	37.1\\
1153	40.51\\
1520	35.55\\
1141	32.64\\
1610	30\\
1418	27.8\\
1499	25.08\\
1531	25.3\\
1658	28.82\\
1900	40.58\\
2209	57.32\\
1154	57.52\\
979	54.94\\
1114	42.91\\
1311	42.6\\
1092	43.18\\
1108	44.64\\
1230	43.91\\
1579	41.46\\
1843	40.91\\
2164	40.93\\
1562	47.23\\
1526	96.69\\
1629	48.73\\
1509	41.12\\
1781	44.62\\
1769	41.69\\
1227	37.3\\
669	36.95\\
925	33.3\\
923	30.65\\
970	31\\
1124	35.52\\
1327	42.44\\
258	45\\
968	49.47\\
901	49.99\\
1035	51.15\\
700	45.85\\
480	45.62\\
605	42.97\\
801	42.92\\
928	39.82\\
973	40.1\\
973	44.07\\
1208	49.08\\
749	81.94\\
1199	48.12\\
1252	43.42\\
989	42.8\\
1165	46.99\\
1118	39.53\\
1174	41.02\\
1396	40.42\\
1399	32.22\\
1516	31.13\\
1301	34.68\\
1746	44.49\\
1264	48.93\\
1139	52.06\\
1127	52.07\\
1443	54.01\\
1437	48.32\\
1241	47\\
1381	43.52\\
1379	41.95\\
1695	41.09\\
1823	40.24\\
1927	42.24\\
1645	48.68\\
1357	86.74\\
1467	46.5\\
1655	42.37\\
1780	43.59\\
1653	42.44\\
3268	41.11\\
3252	42.44\\
3584	38.41\\
3611	35.02\\
3707	34.24\\
2921	38.08\\
3445	46.51\\
3484	51.57\\
3368	59.9\\
3254	52.44\\
3556	51.19\\
3435	50.58\\
3050	48.79\\
3052	49.4\\
3089	46.77\\
3374	43.27\\
3552	42.07\\
3746	44.45\\
3179	50\\
2679	81.51\\
2681	62.32\\
2588	46.92\\
3187	48.57\\
3519	45.4\\
3230	40.77\\
2648	41.97\\
3341	41.04\\
3429	37.01\\
3595	35.1\\
3893	40.21\\
3473	45.87\\
3078	51.76\\
2894	58.1\\
2735	57.39\\
2389	47.4\\
2390	45\\
2461	42.06\\
2460	40.59\\
2598	40.81\\
3021	39.99\\
3134	40\\
3407	43\\
3582	44.29\\
2930	70\\
3002	45\\
2991	42.44\\
3166	42.44\\
3039	40.06\\
3106	43.99\\
2731	32.81\\
2548	31.02\\
2602	31.11\\
2558	32\\
2466	31.32\\
2653	36.89\\
2351	39.94\\
2298	43\\
2267	47.44\\
2177	45.62\\
2061	42.62\\
1891	40\\
1773	38\\
1714	34.99\\
1872	34.96\\
2172	39.96\\
2525	43\\
2657	43\\
2340	46.17\\
2867	54.96\\
2806	46\\
2887	44.94\\
2855	40.56\\
2466	44\\
2351	34.94\\
2147	29.49\\
2080	29.49\\
2142	30.7\\
2369	35.04\\
2064	35.04\\
1802	39.33\\
1801	40\\
1732	41\\
1492	38.83\\
1388	37.44\\
1253	36.99\\
1068	32.44\\
1081	30.81\\
1137	31.8\\
1553	39.96\\
1614	39.96\\
2061	45\\
2196	69.97\\
2185	59.94\\
2807	55\\
2356	42.44\\
821	30.91\\
1963	32.46\\
1977	31.28\\
2214	32.2\\
2408	35.77\\
2091	28.14\\
2846	40.69\\
2978	69.53\\
2317	55.47\\
2235	56.01\\
2410	48.33\\
2208	46.08\\
1829	46.7\\
1960	43.73\\
1826	41.79\\
1953	40\\
2144	42.22\\
2370	43.22\\
2498	40.47\\
1771	48.47\\
2606	54.61\\
2904	50\\
2923	44.94\\
2536	44\\
2978	54.48\\
2974	43.64\\
2853	41.06\\
2809	40.5\\
2730	36.26\\
2978	44.94\\
2978	48\\
2675	57.44\\
2271	62.19\\
2473	57.89\\
2600	47.29\\
2720	46.64\\
2525	37.77\\
2742	40\\
2670	39.96\\
2663	38.76\\
2759	39.96\\
2885	42.11\\
2978	73.81\\
2727	50.12\\
2749	45\\
2825	47\\
2978	120\\
2978	120\\
2978	65\\
2815	37.69\\
2889	37.53\\
2906	37.53\\
2949	37.75\\
2655	36.93\\
2978	44\\
2537	49.22\\
2298	62.17\\
2534	54.35\\
2863	48\\
2797	45\\
2528	45\\
2415	44\\
2412	42.12\\
2464	41.19\\
2548	44\\
2748	47.44\\
2978	46\\
2928	52.92\\
2280	58.31\\
2257	53.12\\
2398	44.94\\
2253	40.98\\
2978	34.99\\
2744	31.27\\
2764	31.27\\
2872	32.81\\
2834	28.3\\
2749	29.55\\
2904	38.64\\
2978	55.5\\
2595	52.09\\
2708	49.94\\
2840	50\\
2978	51\\
2777	47.44\\
2701	47\\
2637	44.94\\
2702	44.42\\
2902	45\\
2978	50\\
2978	69.97\\
2897	44.94\\
2741	50\\
2754	47.44\\
2978	63.41\\
2978	64.08\\
2552	37.23\\
2547	34.37\\
2653	34.24\\
2526	30.51\\
2615	30.51\\
2874	31.4\\
2832	37.77\\
2740	46.07\\
2924	51.91\\
2925	56.15\\
2978	63.46\\
2978	120\\
2978	58.9\\
2817	47.44\\
2812	44.94\\
2802	40\\
2859	40\\
2848	40\\
2845	39.94\\
2415	40.99\\
2374	41.76\\
2429	38.72\\
2978	65.37\\
2795	43\\
2792	34.3\\
2766	37.84\\
2622	34\\
2668	34\\
2581	30.74\\
2560	30.74\\
2229	29.5\\
2304	34.44\\
2650	43.18\\
2516	46\\
2476	45\\
2663	47.44\\
2556	42.8\\
2277	35.87\\
2060	34.44\\
2056	34.44\\
2213	33.61\\
2357	35\\
2803	42.44\\
2582	42.44\\
2700	49.99\\
2864	47.44\\
2899	42\\
2745	44.01\\
2113	35.16\\
1919	42.16\\
1890	34.94\\
1778	34.21\\
1813	31.5\\
1853	33.97\\
1280	24.25\\
1190	24.3\\
1197	25.31\\
1343	34.94\\
1361	38.99\\
1497	41.96\\
1743	47.44\\
1470	39.94\\
1367	35.15\\
1349	34.9\\
1418	34.94\\
1353	39.96\\
1669	42.17\\
1928	44.94\\
2103	59.94\\
2057	54.96\\
2046	48\\
2041	35.99\\
2250	29.88\\
2245	27.73\\
2237	28.85\\
2087	29.85\\
2225	29.75\\
2594	29.99\\
2663	41\\
2343	49.94\\
2468	53.05\\
2502	51.06\\
2694	47.45\\
2679	44.61\\
2392	38.94\\
2589	37.09\\
2668	36.16\\
2685	36.26\\
2765	36.31\\
2906	39.12\\
2572	45.31\\
2375	46.99\\
2366	48.75\\
2796	38.58\\
2978	34.86\\
2818	30.07\\
2007	28.01\\
2094	28.01\\
1968	21.92\\
1951	21.44\\
1980	22.69\\
2155	24.02\\
2442	31.84\\
2385	38.99\\
2559	43.96\\
2469	45.8\\
2438	48.41\\
2751	53.7\\
2349	43.88\\
2181	44.6\\
2134	45.32\\
2232	40.42\\
2438	39.94\\
2683	44.46\\
2651	39.94\\
2742	39.96\\
2978	66.56\\
2978	63.54\\
2978	65.06\\
2978	55\\
2804	41.06\\
2818	37.77\\
2928	35.97\\
2826	35.74\\
2973	37.06\\
2978	73.11\\
2978	80.78\\
2911	50.12\\
2720	55.91\\
2853	59.81\\
2738	52.44\\
2650	47.12\\
2398	44\\
2352	44.8\\
2098	41.22\\
2581	42.98\\
2718	42.99\\
2944	45.87\\
2978	55\\
2978	50.12\\
2732	54.25\\
2732	42.44\\
2852	42.44\\
2978	70\\
2978	81.93\\
2940	39.75\\
2978	60.01\\
2978	39.15\\
2978	38.88\\
2802	38.44\\
2978	69.9\\
2978	59.94\\
2912	63.98\\
2797	60.89\\
2773	54.91\\
2978	60\\
2825	45\\
2864	44.94\\
2978	72.04\\
2978	71.86\\
2978	69.9\\
2978	75.33\\
2978	75.67\\
2978	50\\
2826	50\\
2978	55\\
2978	69.9\\
2956	39.78\\
2978	50\\
2978	100\\
2978	55\\
2978	40.1\\
2978	59.94\\
2978	37.34\\
2978	42.91\\
2978	65\\
2978	70.93\\
2978	69.71\\
2978	69.54\\
2943	52.44\\
2978	68.76\\
2904	50\\
2672	42.86\\
2649	39.99\\
2558	38\\
2587	39.94\\
2825	39.94\\
2809	44.44\\
2656	42.91\\
2978	42.5\\
2971	45\\
2815	42.02\\
2978	79.9\\
2895	41.06\\
2833	39.05\\
2796	39\\
2967	38.27\\
2978	39.3\\
2926	39.54\\
2820	42.44\\
2795	46.24\\
2960	50.4\\
2962	60.52\\
2746	56.09\\
2503	45.12\\
2216	40\\
2024	38.97\\
2010	37.84\\
2122	37.66\\
2305	39.94\\
2554	42.44\\
2630	42.44\\
2412	44.94\\
2681	53.12\\
2800	50.31\\
2978	50\\
2681	40.88\\
2392	30\\
2108	19.99\\
2213	30\\
2371	34.61\\
2406	33.86\\
2501	34.92\\
2392	34.99\\
2236	39.65\\
1841	40.86\\
1756	39.94\\
1744	39.96\\
1501	39.99\\
1298	38.38\\
1212	35\\
1169	32.96\\
1187	34.94\\
1237	38.37\\
1683	41.55\\
2001	42.44\\
2182	49.96\\
2287	48\\
2292	40\\
2220	31.42\\
1801	20.12\\
1483	14.94\\
2061	14.92\\
1955	15.78\\
1994	19.96\\
2406	22.54\\
2454	37.14\\
2374	43.2\\
2325	44.35\\
2168	48.65\\
2104	48.66\\
2050	46.67\\
1909	42.9\\
2098	42.44\\
2061	45\\
2098	44.96\\
2239	44.96\\
2464	55.74\\
2683	48\\
2728	48\\
2543	50\\
2683	55\\
2888	49.04\\
2805	42.44\\
2410	38.11\\
2602	36.22\\
2391	34.31\\
2244	30\\
2279	29.27\\
2231	27.44\\
2667	38.07\\
2732	44.28\\
2886	50.97\\
2746	58.5\\
2776	51\\
2761	50.12\\
2522	47.44\\
2326	49.94\\
2204	48.66\\
2165	43\\
2052	41.33\\
1979	41.06\\
2343	40\\
2386	40\\
2464	42.32\\
2492	45\\
2827	45\\
2772	42.14\\
2575	36.96\\
2413	34.48\\
2563	33.9\\
2609	33.94\\
2664	35.15\\
2597	37.36\\
2720	40.92\\
2679	55.1\\
2417	58.87\\
2337	63.9\\
2374	47\\
2121	42.44\\
1942	39.94\\
1568	39.47\\
1632	39.94\\
1879	42\\
2080	44\\
2451	49.31\\
2730	45\\
2825	44.94\\
2940	47.44\\
2672	46.73\\
2978	72.29\\
2904	42.39\\
2556	37.73\\
2598	35.49\\
2528	34.99\\
2464	33.04\\
2555	29.25\\
2612	31.76\\
2653	38.36\\
2268	46.11\\
2274	51.52\\
2239	50\\
2099	49.94\\
2356	55.02\\
1985	44.94\\
1826	45\\
1707	42.44\\
1679	42.13\\
1763	42.44\\
1924	42.44\\
1974	39.99\\
2055	47.44\\
2350	47.44\\
2441	54.25\\
2655	45\\
2704	41.45\\
2377	36.83\\
2218	31.73\\
2135	31.23\\
2060	29.95\\
2178	28.68\\
2258	29.95\\
2552	36.83\\
2211	38.03\\
2304	42.48\\
2203	46.72\\
2325	48\\
2303	51.9\\
2175	43.45\\
2023	42.12\\
1800	39.94\\
1757	38.92\\
1742	39.94\\
2039	42.12\\
2183	39.99\\
2167	39.09\\
1944	39.94\\
2219	42.44\\
2621	43.84\\
2558	43.84\\
2227	35.91\\
1824	29.1\\
1834	27.72\\
1845	27.98\\
1879	27.59\\
1633	24.71\\
1764	27.72\\
1795	29.99\\
1935	35.65\\
1886	39.94\\
1853	39.94\\
1735	38.12\\
1620	36\\
1286	34.94\\
1182	32.99\\
1337	35\\
1389	35\\
1340	35.65\\
1376	38.38\\
1545	37.85\\
1334	37.84\\
1278	38.19\\
1278	38.39\\
1385	37.97\\
1260	22.76\\
849	18.19\\
1132	19.79\\
1214	19.75\\
1216	19.78\\
1096	19.9\\
855	19.79\\
465	12.92\\
761	27.21\\
994	28.74\\
1099	30.4\\
1142	31.3\\
1091	31.3\\
876	27.46\\
704	21.68\\
726	21.47\\
760	24.96\\
1121	32.16\\
1712	39.94\\
2076	43\\
1927	50\\
1935	54.94\\
2022	53.22\\
1996	42\\
1944	27.46\\
1960	23\\
1958	19.94\\
1780	19.88\\
1773	20.09\\
1833	28.74\\
1686	29.45\\
1073	20.61\\
1074	29.79\\
1130	32.44\\
1084	34.99\\
1041	35\\
893	33.86\\
621	30.65\\
403	29.79\\
398	14.99\\
461	27.99\\
840	34.94\\
1278	40\\
1468	40\\
1202	42.44\\
1186	47.44\\
1501	49.94\\
1652	40.41\\
2079	36.1\\
2330	35.94\\
2122	31.81\\
1941	31.15\\
1943	29.7\\
2096	31.4\\
2449	40.67\\
2677	50.24\\
2555	54.16\\
2523	55.12\\
2788	51\\
2626	48\\
2609	46.72\\
2538	46.72\\
2482	44.26\\
2462	43.94\\
2518	44.16\\
2780	47.44\\
2966	44.94\\
2996	45.15\\
2955	45.67\\
2816	44.94\\
3028	56.96\\
2973	43.75\\
2495	31.96\\
2178	31.04\\
2329	31.96\\
2287	31.95\\
2343	28.94\\
2586	34.33\\
2383	40.59\\
2184	46.49\\
2093	50.8\\
2102	44\\
2044	44.94\\
2239	47.44\\
2015	44.94\\
1941	44.94\\
2074	44.94\\
2157	43.82\\
2291	44.94\\
2130	47.44\\
2494	45\\
2595	46.32\\
2297	47.44\\
2686	47.44\\
2833	44\\
2379	37\\
2038	37.32\\
2383	32.29\\
2394	31.63\\
2347	31.38\\
2440	31.9\\
2284	32.16\\
2416	39.83\\
2559	45.42\\
2613	46.73\\
2673	42.56\\
2509	45\\
2648	48.28\\
2653	47.44\\
2471	44.94\\
2421	41.33\\
2460	40.36\\
2483	40.87\\
2430	39.94\\
2657	39.97\\
2533	39.95\\
2331	39.45\\
2001	42.44\\
2093	43.38\\
2144	35.98\\
1787	29.5\\
1817	30.92\\
1701	29.66\\
1620	29.99\\
1753	29.99\\
1960	31.95\\
1851	35.82\\
1930	44.4\\
1623	44.94\\
1736	57.24\\
1467	53.99\\
1182	44.94\\
841	37.44\\
974	39.94\\
1097	40.01\\
1106	43.67\\
1293	43.64\\
1569	50.11\\
1721	49.33\\
1796	45\\
1826	44.99\\
1925	48.9\\
1844	39.94\\
1758	42.39\\
1162	40.16\\
1416	33.49\\
1462	31.75\\
1407	28.02\\
1438	26.5\\
1568	28.08\\
1000	29.16\\
1033	30.35\\
1093	32.79\\
1014	35\\
986	34.81\\
837	35.95\\
662	34.94\\
633	29.7\\
528	31.75\\
655	27.9\\
892	28.42\\
1174	34.58\\
1403	40\\
1598	42.44\\
1597	44.94\\
1676	50.48\\
1723	54.94\\
1563	45\\
1622	35.28\\
1511	29.68\\
1564	30.94\\
1530	27.15\\
1505	25.23\\
1156	19.19\\
1122	17.49\\
1505	27.9\\
802	22.51\\
969	32.5\\
869	34.96\\
1025	38.46\\
982	38\\
880	40\\
799	35.23\\
799	19.79\\
854	16.22\\
1032	42\\
1095	42.44\\
1313	41\\
1154	38\\
1244	44.96\\
1389	41\\
1659	39.94\\
1732	33.28\\
1649	29.68\\
1595	30.94\\
1640	29.68\\
1716	28.42\\
1937	28.25\\
2135	38.1\\
1474	55.52\\
1441	52.44\\
1245	55.54\\
1171	55.56\\
1620	54.02\\
1824	52.57\\
1841	42.57\\
1827	40.96\\
1788	39.27\\
1893	36.13\\
1557	38.23\\
1641	43.38\\
1740	53.12\\
1765	44.95\\
1626	50.07\\
2291	47.44\\
1897	44.94\\
1823	35.84\\
2189	36.03\\
2281	34.99\\
2222	33.53\\
2310	33.53\\
2497	33.59\\
1810	38.97\\
1741	47.92\\
1862	50.38\\
2024	50.79\\
2034	47.44\\
2244	49.32\\
2157	46.74\\
2651	50.58\\
2777	48\\
2754	44.34\\
2762	43.37\\
2874	48\\
2969	45\\
2764	45.96\\
2635	46.16\\
2648	44.96\\
2845	49.96\\
2476	43.59\\
2290	31.85\\
1971	30.01\\
2184	31.23\\
2120	31.23\\
2433	34.51\\
2531	33.27\\
2307	35.03\\
2107	44.93\\
1936	52.26\\
2211	49.11\\
2236	50\\
2167	49.94\\
2035	43.59\\
2193	48.28\\
2288	48\\
2219	44\\
2244	40.56\\
2386	44.94\\
2500	40.12\\
2341	37.44\\
2318	37.51\\
2555	42.44\\
2672	45\\
2306	39.94\\
2260	34.99\\
2278	38.29\\
2314	33.38\\
2270	31.84\\
2248	33.38\\
1607	30.76\\
1245	29.27\\
959	31.05\\
799	34.78\\
865	40\\
811	42.44\\
807	45\\
953	48\\
799	43\\
799	25\\
799	35\\
908	42.44\\
946	42\\
1319	42.44\\
1522	42.44\\
1583	40\\
1664	47.44\\
2072	50\\
2048	37.99\\
1388	47.44\\
1570	36.49\\
1569	32.07\\
1548	30.78\\
1450	29.15\\
1567	31.5\\
1405	33.78\\
1152	38\\
1129	42\\
1241	51.27\\
1262	56.13\\
1182	50.88\\
1195	46\\
799	35\\
799	33.41\\
848	26.06\\
895	25.38\\
940	35\\
1305	33.38\\
1371	33.26\\
1467	33.78\\
1600	35\\
1943	36\\
1744	34.99\\
1486	34.96\\
1258	26.73\\
990	23.06\\
1826	30.43\\
1863	31.72\\
1720	30.43\\
1343	28.29\\
1139	31.77\\
1262	38\\
1263	42\\
1121	41.8\\
1040	39.96\\
836	35.61\\
739	25.47\\
575	23.28\\
662	22.2\\
917	32.44\\
1182	37.44\\
1505	44.96\\
1445	42.44\\
1274	40\\
1403	49.94\\
1526	53.76\\
1478	49.94\\
1922	50.76\\
1671	41.19\\
1554	32.44\\
1525	30.7\\
2041	35.92\\
2204	37.74\\
2205	33.16\\
1422	25.68\\
1255	30.7\\
1231	33.07\\
1128	34\\
1010	32.44\\
822	31.44\\
665	11.22\\
527	5.34\\
568	4.73\\
776	7.3\\
1146	32.75\\
1642	40\\
1852	40.12\\
1855	42.44\\
1816	49.94\\
2063	54.96\\
2116	45.99\\
1929	30.83\\
1907	30.28\\
2093	34.33\\
1908	34.96\\
2001	36.45\\
2056	37.5\\
2493	40.09\\
2887	44.08\\
2532	48.09\\
2479	43.94\\
2270	44.94\\
2189	47\\
1868	39.99\\
2046	47.44\\
2191	42.44\\
2203	44\\
2342	43.96\\
2559	47.63\\
2831	52.44\\
2869	48\\
2878	46.99\\
2796	53.94\\
2924	49.94\\
2553	35.02\\
2506	36.98\\
2084	34.44\\
2019	31.05\\
1783	24.89\\
1978	30.65\\
1594	27.45\\
2183	34.37\\
2461	37.98\\
2345	38.94\\
2510	44.94\\
2414	57.71\\
2293	57.71\\
2010	44.94\\
2032	42.44\\
2018	40\\
2142	41.44\\
2293	41.27\\
2458	47\\
2135	37.04\\
2408	38.94\\
2455	39.94\\
2240	40.12\\
2470	45\\
2533	41.45\\
1963	32.58\\
1847	30.22\\
1767	29.88\\
1737	28.89\\
1836	28.89\\
2137	30.74\\
2425	38.02\\
2485	44.82\\
2562	47.07\\
2476	59.49\\
2486	57.46\\
2451	57.46\\
2245	54.39\\
2022	47\\
2000	44.94\\
1976	45\\
1968	42.44\\
2228	47.25\\
2380	42.41\\
2377	38.44\\
2345	38.51\\
2388	45\\
2573	48.87\\
2453	32\\
1947	29.67\\
1840	26.87\\
1737	26.13\\
1672	26.07\\
1530	25.96\\
1898	27.03\\
2215	35.78\\
2514	38.09\\
2764	38\\
2975	44.94\\
3004	47.44\\
3044	48.87\\
2752	47.44\\
2715	48.87\\
2666	46.56\\
2617	44.93\\
2633	45.26\\
2760	53.26\\
2710	47\\
2464	40\\
2181	47.6\\
2154	43.18\\
2348	36.69\\
2350	34.43\\
1832	36.43\\
1468	30.11\\
1149	22.82\\
1332	27.23\\
1432	23.15\\
1733	29.35\\
2060	34.86\\
2489	39.94\\
2516	39.35\\
2478	47\\
2386	48.88\\
2288	45\\
2073	42.44\\
1903	37.44\\
1713	34.94\\
1554	34.29\\
1561	34.3\\
1650	34.31\\
1772	33.28\\
1706	34.74\\
1705	35.27\\
1765	35\\
1821	47.44\\
1725	40.83\\
1617	40\\
1451	30.52\\
1555	30.52\\
1567	30.06\\
1570	25.29\\
1566	27.99\\
1146	23.53\\
1356	28.35\\
1423	34.94\\
1484	45\\
1558	55.62\\
1506	50\\
1477	45\\
1426	40\\
1226	33.7\\
1235	32.25\\
1277	32.25\\
1274	35\\
1449	33.7\\
1343	33.99\\
1304	32.31\\
1320	31.63\\
1537	32.06\\
1400	23.55\\
1181	22.77\\
1088	21.56\\
906	21.09\\
668	6.38\\
680	4.97\\
524	2.8\\
899	-0.01\\
899	-0.01\\
592	2.28\\
556	8.06\\
707	10.09\\
841	10.46\\
753	10.82\\
651	9.8\\
899	20.01\\
848	3.51\\
899	20.01\\
726	2.47\\
905	35\\
899	33.71\\
992	33.73\\
1321	33.63\\
1584	32.76\\
1521	24.48\\
1071	23.15\\
1505	23.01\\
1345	22.7\\
1260	22.7\\
1306	23.05\\
1934	31.8\\
1889	37.69\\
1798	39.94\\
2036	42.44\\
2068	48\\
2433	57.71\\
2332	58.48\\
2063	44.99\\
2187	52\\
2286	53.71\\
2404	48.4\\
2551	48.87\\
2619	64.96\\
2536	42.44\\
2496	42.44\\
2796	42.92\\
2794	48.87\\
2896	57.44\\
2635	49.01\\
3219	50\\
3082	35.16\\
2998	32.75\\
2938	31.6\\
2990	31.91\\
2443	29.26\\
2945	37.17\\
2601	50.3\\
3116	58.51\\
2946	63.46\\
2156	51.85\\
2155	52.02\\
2751	48.87\\
2102	42.65\\
2169	39.78\\
2805	44.17\\
2635	38\\
2259	36.95\\
2938	39.96\\
2972	41.79\\
2906	42.27\\
2968	41.03\\
3219	42.44\\
2969	34.07\\
2911	31.83\\
2547	27.44\\
2813	31.12\\
3178	32.56\\
3086	31.57\\
3219	58.74\\
3196	33.71\\
3219	59.35\\
3219	49.11\\
3067	44.94\\
3036	42.44\\
2921	40.33\\
2837	39.94\\
2564	39.94\\
2582	42.44\\
2585	44\\
2576	46.67\\
2727	49.11\\
2927	43\\
2916	39.99\\
2962	39\\
3102	40.66\\
3219	72.25\\
3219	47\\
2473	31.77\\
2595	30.89\\
2793	31.89\\
2851	30.95\\
2886	30\\
2521	30.55\\
2727	38.64\\
2809	47.42\\
2421	55.64\\
2771	51.69\\
2548	47.39\\
2561	46.74\\
2258	40.37\\
2183	39.99\\
2095	37.98\\
2173	35.2\\
2340	35.18\\
2714	37.44\\
3061	40\\
3109	40.57\\
3160	40\\
3219	43.2\\
3219	70.58\\
3151	40\\
2684	32.34\\
2967	33.91\\
2915	33.72\\
2927	32\\
2896	30.65\\
2574	29.1\\
2841	34.09\\
3056	55.13\\
2926	44.96\\
2853	47\\
2776	45\\
2524	41\\
2116	41\\
1976	40\\
1898	41\\
1892	38\\
2069	39.94\\
2296	42.44\\
2567	44.16\\
2617	40\\
2667	39.47\\
2782	39.94\\
2741	39.94\\
2939	40\\
2618	31.5\\
2155	27.88\\
2569	32.98\\
2431	31.68\\
2490	33.67\\
2252	31.68\\
1520	26.31\\
1524	30.39\\
1522	34.96\\
1646	42.44\\
1220	37.44\\
1118	34.83\\
1186	34.27\\
1052	31.68\\
926	30.39\\
986	29.99\\
1121	31.98\\
1356	36.4\\
1721	44.94\\
1911	46.83\\
1982	47.12\\
2076	50\\
1832	49.94\\
2263	48.06\\
1976	52.35\\
1703	32\\
1974	35.13\\
1850	33.33\\
1905	33.19\\
1905	30.75\\
1170	17.97\\
931	20.01\\
1032	21.22\\
1103	29\\
1036	30.75\\
1039	30.75\\
967	29.96\\
758	22\\
1016	31.03\\
1270	33.2\\
1314	33.2\\
952	31.77\\
1221	37.44\\
1303	37.44\\
1514	39.94\\
1652	45\\
1694	49\\
1870	39.94\\
1801	30.9\\
1830	31.47\\
1831	30.57\\
1826	30.1\\
1813	28.12\\
1864	29.43\\
2071	38.44\\
2171	49.95\\
2406	53.11\\
2437	50.39\\
2644	46.96\\
2511	47.2\\
2123	44\\
2073	42.43\\
2066	39.45\\
2171	37.44\\
2303	37.92\\
2557	41.18\\
2754	46.5\\
2714	53.46\\
2626	49.47\\
2645	47.44\\
2646	39.98\\
2581	33.29\\
1975	31.59\\
1858	30.59\\
1859	29.02\\
1824	29.42\\
1864	29.62\\
1775	31.5\\
1940	42.77\\
2138	53.94\\
2297	52.39\\
2433	46.82\\
2619	42.7\\
2558	41.22\\
2224	37.99\\
2196	38.06\\
2215	36.08\\
2320	36.97\\
2441	41\\
2700	57.21\\
2900	48\\
2963	48\\
2732	53.11\\
2881	46.04\\
2789	44.94\\
2709	48.01\\
2865	45.85\\
2813	35.72\\
2729	33.52\\
2633	32.84\\
2712	32.7\\
2847	31.95\\
2709	41.19\\
2502	48.83\\
2608	52.91\\
2792	52.99\\
2682	44\\
3056	68.83\\
3028	39.94\\
2912	44.94\\
2827	42.3\\
2810	39.94\\
2782	44.94\\
2772	53.11\\
2704	43.12\\
2460	47.27\\
2388	43.46\\
2489	37.97\\
2624	44.94\\
2172	44.18\\
1632	28.25\\
1331	24\\
1588	20.41\\
1700	15.13\\
1850	18.41\\
1826	22.91\\
1939	29.25\\
2351	36.55\\
2379	39.66\\
2560	38.58\\
2836	40.5\\
2743	42.29\\
2481	41.26\\
2468	39.94\\
2574	42.29\\
2645	42.29\\
2647	48\\
2529	48\\
3032	54.24\\
2603	44.7\\
2676	43.82\\
2946	46.63\\
3056	55\\
2655	47.01\\
2194	30.05\\
2439	34.8\\
2305	31.81\\
2277	31.03\\
2376	31.81\\
2326	30.75\\
2487	34.04\\
2653	49.95\\
2386	51.06\\
2883	52.97\\
2630	52.41\\
2481	57.04\\
2317	47.98\\
2415	45.23\\
2352	40.53\\
2287	37.38\\
2064	35\\
2241	37.54\\
2260	42\\
2177	43.98\\
2117	41.92\\
2108	40.49\\
2380	42.58\\
2400	49.78\\
2349	47.24\\
2101	34.57\\
1951	34.29\\
1961	33.19\\
1932	32.08\\
1880	30.58\\
1826	29.99\\
1750	32.44\\
1503	39.94\\
1501	40.77\\
1516	48\\
1453	47.44\\
1401	44.94\\
1069	35\\
974	34.09\\
908	32.44\\
976	32\\
1086	37.44\\
1189	40\\
1226	42.44\\
1331	45\\
1273	40\\
1567	49.96\\
1414	44.96\\
1504	32.94\\
1545	24.96\\
1334	20.34\\
1437	16.18\\
1487	15.45\\
1591	15.79\\
1567	18.94\\
1491	27\\
1548	30.85\\
1232	37.99\\
1054	37.44\\
1022	38.99\\
960	38.3\\
799	32.94\\
719	13.43\\
724	13.88\\
806	30\\
1008	42.44\\
1190	44.94\\
1344	50\\
1349	49.99\\
1426	50\\
1481	55\\
1523	44.94\\
1636	50\\
1540	32.75\\
1543	32.77\\
1547	29.99\\
1625	32.75\\
1753	29.99\\
2125	37.12\\
2700	46.58\\
2891	42.44\\
2853	42.2\\
2900	42.4\\
2956	60\\
2945	52.08\\
2956	70.42\\
2956	70.83\\
2956	69.99\\
2956	49.04\\
2956	60\\
2956	55\\
2810	45.21\\
2666	42.29\\
2956	44.94\\
2956	69.99\\
2820	41.75\\
2280	30.03\\
2580	31.82\\
2618	33.35\\
2617	33.85\\
2736	34.01\\
2956	50\\
2887	33.63\\
2791	43.97\\
2452	47.44\\
2596	50.01\\
2782	49.93\\
2733	50.01\\
2158	44.71\\
2307	44\\
2443	45.42\\
2765	41.89\\
2758	41.2\\
2941	45\\
2956	50.12\\
2877	40.94\\
2956	42.44\\
2956	41.67\\
2741	39.99\\
2882	37.58\\
2956	200\\
2524	34.62\\
2038	30.85\\
1882	29.63\\
2074	31.63\\
2358	31.72\\
2839	33.85\\
2956	44.94\\
2686	44.1\\
2731	45.17\\
2956	54\\
2862	50\\
2956	60\\
2920	50\\
2878	45.06\\
2812	43.01\\
2750	46.4\\
2780	54.31\\
2843	44.94\\
2884	41.2\\
2646	38.13\\
2724	41.96\\
2956	54\\
2869	44.94\\
2371	39.94\\
2121	33.94\\
2069	34.09\\
2089	33.58\\
2082	32\\
2107	29.42\\
2084	30.43\\
1788	29.96\\
1551	28.61\\
1755	32.44\\
1758	34.37\\
1837	37.44\\
1725	39.94\\
1499	34.94\\
1338	33.89\\
1341	33\\
1240	33.25\\
1170	34.33\\
1590	44.94\\
1779	45\\
1756	49.83\\
2042	44.53\\
2060	47.94\\
1913	45\\
1366	33.05\\
1648	30.06\\
1857	31.35\\
1858	31.35\\
1821	30.66\\
1945	31.35\\
1573	32.37\\
1690	38.73\\
1508	42.42\\
1562	44.94\\
1455	42.44\\
1366	42.44\\
816	40.95\\
719	35.63\\
935	37.17\\
903	34.69\\
985	33.95\\
789	37.11\\
1218	47.44\\
1241	45.24\\
1299	45.84\\
1368	42.76\\
1390	42.32\\
1596	39.94\\
1867	36.47\\
1813	44.25\\
1828	35\\
1798	32.87\\
1872	32.51\\
1839	32.87\\
1708	32.51\\
1598	35\\
1270	35\\
1336	43.15\\
823	37.12\\
798	36\\
699	34.94\\
1064	38.75\\
986	34.25\\
986	34.99\\
1125	34.94\\
1301	38.75\\
1408	41\\
1505	42.44\\
1546	39.94\\
1619	39.94\\
1543	44.94\\
1787	44.94\\
1746	38.23\\
1591	34.99\\
1577	38.79\\
1522	34.29\\
1501	32.29\\
1395	31.35\\
1269	29.3\\
1167	29.96\\
1281	31.35\\
1014	30.9\\
1126	33.56\\
1178	36\\
1126	38\\
942	33.67\\
725	31.68\\
1049	35\\
842	30.37\\
847	31.68\\
1315	36\\
1286	39\\
1395	44\\
1544	44\\
1292	45\\
1939	46.9\\
1626	49.23\\
1612	41.08\\
1706	37.04\\
1794	34.04\\
1886	34.04\\
1810	30.58\\
2287	38.33\\
2479	42.04\\
2136	43.01\\
2055	43.5\\
1814	43.72\\
1695	44.99\\
1550	41.01\\
1888	50\\
1930	47.7\\
2010	43.58\\
2148	42.44\\
2347	54.85\\
2048	40.76\\
2082	42.89\\
2066	42.89\\
2247	41.91\\
2635	48.93\\
2433	45.32\\
1576	32.46\\
1909	32.55\\
1841	31.47\\
1795	30.13\\
1792	30.79\\
1754	30.11\\
1727	40.94\\
2115	48.29\\
1744	53.15\\
1902	50.94\\
1850	47.86\\
1908	51.25\\
1656	47.69\\
1356	44.99\\
1487	45.84\\
1619	44.37\\
1929	44.88\\
1573	47.02\\
2033	48.47\\
2093	48.97\\
1981	47.61\\
1876	44.48\\
2067	45.61\\
1886	37.5\\
2123	32.49\\
2173	31.54\\
2103	33.2\\
2105	33.2\\
2102	33.25\\
1851	31.54\\
1958	37.66\\
1830	47.44\\
2122	50.44\\
2029	50.01\\
2101	53.15\\
1991	55.55\\
2101	54.94\\
2202	55.12\\
2296	48.37\\
2238	45.17\\
2160	47\\
2206	54.94\\
2215	48\\
1957	41.91\\
1864	39.67\\
1964	37.24\\
2429	45.51\\
2314	39.94\\
1402	28.95\\
1630	28\\
1749	27.58\\
1784	27.58\\
1739	24.07\\
1466	25.03\\
1494	30.97\\
1646	37.67\\
1866	42\\
1785	46.44\\
1818	49\\
1808	47.96\\
1662	42\\
1498	44.94\\
1387	40.12\\
1412	37.44\\
1531	37.44\\
1774	43.56\\
1957	39.94\\
1987	40\\
2137	37.2\\
2159	36.83\\
2588	45.41\\
2364	41.99\\
2051	31.38\\
1904	31.4\\
2099	31.41\\
1954	31.37\\
2059	31.39\\
2112	32.86\\
2270	40.01\\
2387	41.71\\
2311	44.94\\
1906	47.18\\
1678	45\\
1580	44.94\\
1504	40\\
1379	39.94\\
1436	39.94\\
1562	38\\
1627	38.49\\
1796	42.44\\
1842	44\\
1889	39.44\\
1983	37.68\\
1938	36.01\\
2277	43.7\\
1734	32.3\\
1215	28.34\\
1026	26.07\\
1049	25.06\\
1071	24.73\\
1016	24.99\\
900	24.09\\
770	10.35\\
786	15.98\\
824	29.19\\
1007	31.86\\
1047	34.94\\
1081	32.14\\
966	31.21\\
895	20.01\\
883	17.38\\
873	29.58\\
1019	30.43\\
1200	33.86\\
1395	38.43\\
1495	40\\
1590	38\\
1612	33.87\\
1806	39.94\\
1651	30.57\\
1573	29.51\\
1404	27.3\\
1254	20.12\\
1504	25.04\\
1495	24.99\\
1483	16.32\\
1401	14.91\\
1195	15\\
1046	15.98\\
1270	18.13\\
1354	27.41\\
1354	27.46\\
1232	18.45\\
899	17.1\\
844	14.06\\
749	13.7\\
724	16.55\\
1062	20.08\\
1354	27.51\\
1501	37\\
1579	37\\
1717	34.93\\
1472	37.11\\
1665	37\\
1424	31.37\\
1237	28.36\\
1086	25.04\\
1144	24.99\\
1105	24.99\\
1029	22.28\\
1042	24.33\\
1068	27.99\\
727	26.09\\
753	26.93\\
613	26.46\\
625	28.89\\
681	29.87\\
544	26.71\\
604	25.25\\
640	21.61\\
825	31.09\\
815	36\\
986	37\\
1140	39.94\\
1308	45\\
1305	39.94\\
1519	45.37\\
1604	35.99\\
1464	30.47\\
1499	31.37\\
1396	30.01\\
1294	29.25\\
1383	28.66\\
1543	21.77\\
1824	30.01\\
2380	40.12\\
2547	41.27\\
2567	45.39\\
2465	48\\
2370	49.88\\
2264	45\\
2122	45.1\\
2250	44.94\\
2458	44.57\\
2445	44.49\\
2105	49.69\\
2413	43.84\\
2412	43.03\\
2604	39.91\\
2493	36.92\\
2555	41.84\\
2272	43.16\\
1785	31.07\\
1786	30.71\\
1800	30.68\\
1884	30.52\\
2195	30.71\\
2210	30.52\\
2048	34.92\\
2000	44.99\\
1860	49.54\\
1690	49.51\\
1333	49.88\\
1195	49.94\\
1219	49.4\\
1142	44.98\\
1183	43.77\\
1160	41.81\\
1284	39.57\\
1520	40.96\\
1886	47.96\\
2192	44.94\\
2324	39.6\\
2251	36.24\\
2378	43\\
2162	44.01\\
2051	30.95\\
1950	30.57\\
1909	29.39\\
1967	28.74\\
2058	28.74\\
2169	29.66\\
2056	33.09\\
2347	44.07\\
2341	46.37\\
2015	46.58\\
1963	46.12\\
1850	46.68\\
1562	42.94\\
1722	41.05\\
1910	39.99\\
1863	38.35\\
1889	37.07\\
2050	39.4\\
2340	42\\
2514	44.77\\
2603	43.05\\
2524	40.51\\
2761	39.39\\
2315	39\\
1954	32.37\\
1815	32.39\\
1780	31.39\\
1790	30.49\\
2017	31.11\\
2124	31.39\\
2215	36\\
2578	44.82\\
2580	45.55\\
2091	53.88\\
1852	51.42\\
1868	55\\
1586	45.6\\
1418	43\\
1667	47.75\\
1632	39.94\\
1696	38\\
1848	44\\
2101	40\\
2131	35.12\\
2185	47\\
2250	32.36\\
2405	49.52\\
2012	45\\
1923	44.99\\
1947	32\\
1822	30.34\\
1913	29.73\\
1851	27.34\\
1908	20\\
1925	22.33\\
1592	29.34\\
1905	32.03\\
1868	48.57\\
1727	50\\
1596	49.94\\
1480	42.44\\
1460	37.44\\
1345	33.32\\
1075	31.91\\
1029	31.87\\
1375	33.32\\
1515	36\\
1587	39.94\\
1646	36\\
1767	34.35\\
2013	45.1\\
1698	48.33\\
1632	31.61\\
1303	31.3\\
1074	28.82\\
1171	28.62\\
1158	27.46\\
1303	23.12\\
1250	16.43\\
1205	22.89\\
986	27.44\\
1183	31.24\\
1378	32.44\\
1265	32.91\\
1055	33.11\\
940	31.6\\
685	30\\
599	22.89\\
599	22.89\\
837	30.65\\
1228	31.54\\
1438	33.11\\
1620	35.25\\
1575	31.59\\
1777	34.96\\
1542	31.63\\
1770	30.04\\
1964	27.8\\
1895	25.84\\
1741	21.49\\
2145	27.46\\
2239	24.89\\
2453	33.93\\
2796	41.17\\
2837	41.93\\
2775	40\\
2754	42.44\\
2826	47.44\\
2956	45\\
2882	44.94\\
2762	37.49\\
2773	34.99\\
2639	33.09\\
};
\addplot [color=mycolor1,line width=1.0pt,mark size=0.3pt,only marks,mark=*,mark options={solid},forget plot]
  table[row sep=crcr]{%
2849	33.61\\
2923	35.11\\
2708	36.82\\
2686	35.99\\
2596	34.53\\
2845	36.05\\
2565	30.9\\
2460	30.84\\
2522	30.87\\
2618	30.87\\
2675	31.36\\
2695	30.87\\
2666	28.57\\
2308	36.96\\
2544	44.71\\
2557	51.99\\
2703	52.44\\
2175	50\\
2190	49.99\\
2075	42.98\\
2009	41.83\\
2802	41.45\\
2906	39.94\\
2830	37.44\\
2981	38.34\\
3055	39.84\\
2907	41.21\\
2892	41.22\\
2759	36.31\\
3068	40.48\\
2837	35.65\\
2614	29.87\\
2727	31.03\\
2715	31.32\\
2591	31.2\\
2675	31.05\\
2478	28.22\\
2827	34.97\\
3219	48.51\\
3219	43.23\\
3219	48.51\\
3219	48.51\\
3219	48.51\\
3219	62.13\\
3144	39\\
2984	40\\
2910	37.44\\
2953	37.44\\
3219	60.01\\
3219	65.5\\
3219	57.01\\
3219	60\\
3219	44.99\\
3219	69.47\\
3219	47.44\\
2704	31.33\\
2446	30.52\\
2580	31.22\\
2559	30.68\\
2848	30.86\\
2612	29.14\\
3013	30.52\\
3004	38\\
2949	39.94\\
3029	43.73\\
2986	49.47\\
2944	55\\
2817	40\\
2915	42\\
3163	44.57\\
3107	39.94\\
3078	37.44\\
3212	39.94\\
3219	50.01\\
2996	36.07\\
3019	45.95\\
3027	36.09\\
3219	60\\
3123	35.5\\
2555	30.84\\
2661	31.61\\
2655	30.99\\
2616	30.48\\
2706	30.25\\
2424	27.02\\
2478	30.43\\
2490	35\\
2610	35.96\\
2703	40\\
2738	42.44\\
2792	49.88\\
2772	42.44\\
2850	44.94\\
2962	42.44\\
2946	35\\
2881	34.11\\
2958	35\\
3004	36.01\\
2873	35.42\\
2817	35.52\\
2808	33.51\\
3044	43\\
2895	44.83\\
2423	31.49\\
2062	30.97\\
2181	31.49\\
2188	30.68\\
1993	26.71\\
2010	18.63\\
2139	27.4\\
1783	26.71\\
2102	31.01\\
1848	31.44\\
1908	34.09\\
2044	35\\
1914	34.31\\
1911	34.94\\
1755	33\\
1610	32.21\\
1646	31.41\\
1989	34.94\\
2145	36\\
2243	39.94\\
2337	40\\
2397	32.44\\
2488	42\\
2391	36\\
2130	31.49\\
1738	30.13\\
2208	31.01\\
2128	17.58\\
2114	15.91\\
2064	14.59\\
2029	13.55\\
2024	14.24\\
1342	13.56\\
1510	25.04\\
1512	29\\
1491	30\\
1363	30.55\\
1137	30\\
1023	29.75\\
1002	29.75\\
1106	29.96\\
1373	32.85\\
1778	34.94\\
1975	44.94\\
2055	50\\
2123	40\\
2310	50.5\\
2176	45.01\\
1813	31.24\\
1829	30.33\\
2289	30.98\\
2255	27.4\\
2269	20.18\\
1923	23.88\\
2319	31.21\\
2705	39.01\\
2753	35.96\\
2699	34.26\\
2515	33.99\\
2448	35.9\\
2237	37\\
2175	38.02\\
2156	39.47\\
2163	39.23\\
2236	38.36\\
2385	41.94\\
2644	47.91\\
2563	47.84\\
2524	43.98\\
2251	38.48\\
2643	41.92\\
2609	36.31\\
2656	30.36\\
3219	32.08\\
3133	28.54\\
3064	28.01\\
2914	27.87\\
2642	29.03\\
2321	36.12\\
1761	44.9\\
1360	48\\
1173	49.68\\
1400	50.61\\
1943	54.11\\
1531	47.85\\
1713	45.23\\
1873	41.98\\
1951	39.94\\
2173	38.73\\
2418	40.79\\
1974	41.99\\
1958	42.93\\
2380	41.46\\
2716	38.86\\
3053	39.82\\
2914	36.02\\
2440	30.98\\
2297	28.71\\
2261	28.23\\
2216	26.9\\
2223	26.81\\
2270	28.87\\
2176	35.06\\
1967	44.4\\
1608	45\\
1536	44.81\\
1475	43.67\\
1423	45.98\\
1359	41.2\\
1958	40.85\\
1906	39.47\\
1935	38.91\\
1961	38.41\\
2220	40.45\\
2302	41.68\\
2210	43.22\\
2186	41.23\\
2310	38.86\\
2627	40.63\\
2515	34.96\\
2497	31.98\\
2788	29.69\\
2799	28.85\\
2775	27.76\\
2708	27.71\\
2641	29.11\\
2264	36.71\\
1786	44.69\\
1570	47.55\\
1828	44.06\\
1925	43.07\\
1642	45.33\\
1746	42.85\\
1854	41.2\\
1700	44.98\\
1697	43.62\\
1780	41.9\\
2073	44.83\\
1922	48.59\\
1416	49.07\\
1458	45.59\\
1869	41.96\\
2225	42.49\\
2576	36.74\\
2349	32.48\\
2438	29.67\\
2579	28.99\\
2649	28.14\\
2539	28.27\\
2617	29.08\\
2530	35.39\\
2538	41.97\\
2316	45.41\\
2863	44.55\\
3044	43.79\\
2425	43.04\\
2600	40.96\\
2935	38.98\\
2954	36.07\\
2897	34.24\\
2845	34.94\\
2904	37.93\\
2650	40.47\\
2516	41.97\\
2473	38.9\\
2718	36\\
2886	41.96\\
2625	38.54\\
2037	46.71\\
1707	30.33\\
1551	28.98\\
1438	28.02\\
1618	27.15\\
1573	27.07\\
1345	27.99\\
1580	29.99\\
2001	31.78\\
2149	45.12\\
2205	56.38\\
2299	57.97\\
2207	54.25\\
2085	42.44\\
2020	38.79\\
2072	35.33\\
2093	36.39\\
2360	40\\
2442	50\\
1995	40\\
1916	36\\
2194	39.96\\
2442	48.46\\
2207	46.09\\
1829	31.22\\
1513	29.91\\
1289	24.58\\
1331	22.95\\
1765	28.47\\
1574	22.51\\
1571	22.47\\
1483	22.78\\
1691	29\\
1311	30.35\\
1410	35\\
1396	39.04\\
1367	39.94\\
1203	33.97\\
1002	30\\
890	29.73\\
854	29.94\\
1127	36\\
1448	35.2\\
1630	35.2\\
1205	33.96\\
1587	37\\
1258	33.98\\
1169	31.37\\
1228	28.08\\
1371	29\\
1374	25\\
1276	24.01\\
1353	24.4\\
1375	26.78\\
1670	34.08\\
1842	41.26\\
1752	41.93\\
1984	42.29\\
1898	43\\
1865	47\\
1778	42.55\\
1677	41.25\\
1489	39.94\\
1724	36\\
1765	33.87\\
2019	35\\
2176	37\\
2081	40.37\\
2247	40.27\\
2289	36.23\\
2621	37\\
2248	37\\
2131	31.75\\
2073	30.54\\
2147	30.1\\
2066	29.32\\
2025	25.97\\
2048	26.9\\
2210	33.59\\
2357	41.06\\
2281	40.99\\
2053	42.44\\
1891	40\\
1665	39.94\\
1505	35.69\\
1578	36\\
1567	36\\
1672	35.17\\
1747	34.94\\
1979	35.18\\
1933	39\\
1802	41.92\\
1940	40.09\\
1837	38.87\\
2432	39.38\\
2209	36\\
1788	30.26\\
1781	29.29\\
1988	29.37\\
1863	25.15\\
2043	25.81\\
1978	27.97\\
1953	31.63\\
1985	39.91\\
2136	39.99\\
2263	39.61\\
2124	39.13\\
2020	40.43\\
1689	38.41\\
1615	36.96\\
1559	36\\
1612	34.95\\
1804	33.82\\
1872	36.08\\
1917	38.59\\
1817	39.91\\
2061	39.02\\
1972	36.9\\
2465	38.88\\
2297	32.59\\
2122	30.03\\
1828	28.3\\
2151	29.39\\
2113	29.59\\
2288	29.52\\
2529	26.87\\
2315	31.4\\
2201	36.92\\
2548	38.73\\
2683	36.95\\
2597	35.46\\
2410	36.3\\
2180	33.97\\
2127	32.45\\
2171	33.79\\
2185	31.83\\
2251	32.44\\
2272	36\\
2391	38.79\\
2014	39.96\\
2044	39.3\\
2104	38.94\\
2441	39.43\\
2341	33.12\\
2118	29.63\\
1613	26.87\\
1823	25.69\\
1786	25.06\\
1855	24.99\\
1720	26.04\\
2142	30.28\\
2450	37\\
2502	38.5\\
2739	37.44\\
2684	39.1\\
2648	41\\
2622	37.44\\
2504	37\\
2461	38\\
2491	37.44\\
2505	37.61\\
2622	44.98\\
2753	36.94\\
2610	37.96\\
2558	36.93\\
2495	32.9\\
2785	33.22\\
2352	30.15\\
1788	30.34\\
1368	34.8\\
1248	30.23\\
1201	29.3\\
1226	28.75\\
1152	24.99\\
1134	24.43\\
871	27.6\\
799	30\\
918	34\\
799	30\\
799	30.18\\
799	30\\
833	30.26\\
758	20.03\\
733	20.31\\
792	19.35\\
799	28.32\\
799	28.33\\
799	28.32\\
1133	33.2\\
1135	31.6\\
1425	31.85\\
1228	28.86\\
1100	47\\
799	18\\
1092	23.55\\
846	22.6\\
956	22.02\\
499	4.79\\
579	0.29\\
910	22.79\\
716	14.21\\
1003	26.1\\
1106	28.23\\
842	28.89\\
840	30.32\\
999	32.44\\
886	30\\
955	28.89\\
1035	29.94\\
1168	30.32\\
1438	43\\
1650	47.27\\
1582	46.22\\
1696	40.96\\
1630	42.95\\
1686	31.6\\
858	28.23\\
1022	28.25\\
1080	25.35\\
1162	25.35\\
1160	24.14\\
1230	26.02\\
1609	33.09\\
1781	36.93\\
1700	37.16\\
1730	36.94\\
1678	37.26\\
1770	38.13\\
1680	37.98\\
1737	36.97\\
1841	36.19\\
1990	35.29\\
2025	36.15\\
1813	37.97\\
2125	39.23\\
2137	40\\
2454	38.91\\
2402	36.17\\
3074	36\\
2958	29.64\\
2239	28.07\\
2265	27.55\\
2018	25.97\\
2231	25.07\\
2192	25.08\\
2124	25.93\\
2033	30.87\\
1924	40.26\\
2298	42.98\\
2490	43.07\\
2547	45.09\\
2640	44.99\\
2418	42.03\\
2344	40.49\\
2461	38.55\\
2583	36.99\\
2593	36.18\\
2717	35\\
2703	33.98\\
2520	33.03\\
2518	30.83\\
2375	29.64\\
2675	29.15\\
2396	27.8\\
2391	25.22\\
2106	23.99\\
2317	22.04\\
2199	20.92\\
2091	21.7\\
2052	24.2\\
1635	27.61\\
1329	34.95\\
1327	39.08\\
1281	39.75\\
1293	40\\
1455	41.91\\
1509	38.64\\
1452	39.74\\
1676	36.94\\
1600	35.1\\
1634	32\\
1735	34.9\\
1712	35.61\\
1473	35.19\\
1615	33.41\\
1979	32.14\\
2516	30.74\\
2207	29.11\\
1508	25.07\\
1161	21.85\\
1505	21.04\\
1871	21.37\\
1664	22.95\\
1448	25.56\\
1506	29.4\\
1360	36.46\\
1507	40.18\\
1578	39.91\\
1377	39.68\\
1421	39\\
1654	34.58\\
1698	32.86\\
1708	31.88\\
1926	31.85\\
2126	32.77\\
2225	34.94\\
2121	35.85\\
1932	36\\
1907	33.8\\
2192	32.11\\
2395	39.99\\
2202	29.48\\
1811	25.95\\
2052	25\\
2164	23.84\\
2155	23.24\\
1769	23.59\\
1487	25.86\\
1554	30.09\\
1528	36.74\\
1288	37.47\\
974	36.48\\
1211	35.94\\
1894	36.27\\
1935	39.1\\
1841	35\\
1657	38\\
1558	35\\
1652	35\\
1756	35\\
1686	33.53\\
1287	35\\
1324	35.97\\
1620	35.93\\
1967	35.98\\
1851	29.88\\
1898	30.13\\
1512	26.5\\
1366	25.1\\
1411	25.07\\
1170	24.88\\
1235	24.58\\
1257	25.1\\
1423	29.79\\
1307	32.93\\
1442	34.65\\
1617	34.96\\
1594	47.44\\
1581	40\\
1364	34\\
1270	29.94\\
1238	27.55\\
1303	29.94\\
1439	31.53\\
1493	32.86\\
1407	35.96\\
1356	35.98\\
1329	35.91\\
1506	35.97\\
1391	29.89\\
1443	27.72\\
1203	25.27\\
1226	24.02\\
1031	22.69\\
1176	21.04\\
1086	19.61\\
1020	18.09\\
1095	19.99\\
1006	23.66\\
1204	25.62\\
1211	28.08\\
1280	34.94\\
1248	39.94\\
1026	34\\
966	29.89\\
879	28.04\\
959	28.03\\
1032	34.94\\
1227	35\\
1371	35\\
1412	35\\
1558	32.99\\
1492	42.44\\
1440	34\\
924	35\\
974	27.4\\
794	17.57\\
850	24.01\\
1173	26.88\\
1357	26.71\\
1083	34\\
1032	41.46\\
1549	41.96\\
1445	40.78\\
1515	38.01\\
1638	39.94\\
1823	33\\
1653	32.44\\
1589	29.94\\
1516	29.7\\
1563	29.94\\
1673	33.97\\
1556	37.5\\
1239	38.87\\
1295	38.08\\
1258	35.93\\
1939	35.91\\
2037	30\\
1576	28.21\\
1285	26.1\\
1477	25.22\\
1402	25.07\\
1321	25.23\\
1570	27.05\\
900	29.62\\
799	28.24\\
1020	36.1\\
1306	34.92\\
1799	35.23\\
1954	36.94\\
1951	35.93\\
1896	34.92\\
1984	34\\
1883	34.94\\
1877	40\\
1956	42.44\\
2003	37.49\\
1533	39.7\\
1505	39.81\\
1575	38.98\\
1810	38.04\\
1999	33.54\\
2035	29.35\\
1902	28.93\\
1975	28.94\\
1899	28.91\\
2036	26.79\\
1704	28.33\\
2151	33.74\\
1834	40.3\\
1995	41.8\\
1868	44.94\\
1781	46.01\\
1620	46.42\\
1356	40\\
1277	42.44\\
1168	39.94\\
1190	39\\
1093	34.94\\
1117	36.12\\
1589	38.47\\
1333	42.28\\
1366	42.06\\
1523	39.24\\
1910	39.81\\
1960	34.45\\
1924	31.36\\
1845	28.56\\
1606	28.01\\
2014	28.31\\
1973	28.21\\
2102	29.59\\
1696	35.85\\
1698	42.1\\
1717	43.05\\
1482	46.73\\
1371	46.54\\
1260	41\\
1379	49.67\\
1100	54.94\\
1080	50.11\\
1141	48.99\\
1252	50\\
1554	50.11\\
1684	42.36\\
1424	42.64\\
1146	40.65\\
1313	38.94\\
1731	38.89\\
1998	33.15\\
1456	31.09\\
1614	29.44\\
1510	28.4\\
1470	28.09\\
1419	27.69\\
1054	29.44\\
1148	35.44\\
932	41.75\\
1018	42.44\\
1160	40.72\\
1121	39.77\\
1347	38.1\\
957	36.99\\
630	34.26\\
825	33.93\\
967	34.84\\
1093	34.94\\
1297	37.93\\
1395	41.16\\
1309	43.03\\
1213	40.73\\
1277	36.93\\
1648	39.5\\
1275	37.92\\
1709	34.47\\
1472	31.42\\
1462	30.68\\
1446	30.68\\
1318	29.23\\
1182	26.24\\
617	9.74\\
489	14.54\\
510	22.03\\
674	27.3\\
777	29.32\\
225	27.31\\
-85	26.41\\
61	23.27\\
86	23.94\\
292	26\\
604	26.02\\
901	30.91\\
1314	35\\
1431	35.02\\
1559	37.44\\
1462	32.35\\
1617	46.7\\
1669	46.92\\
1499	47.76\\
1372	32.87\\
1482	30.24\\
1450	21.62\\
1465	18.42\\
1419	15.67\\
1281	16.76\\
1241	20.38\\
925	19.96\\
1059	31.13\\
1235	35\\
1085	41.63\\
855	35\\
914	35\\
859	33\\
811	33.75\\
836	34.02\\
873	34.94\\
1114	45\\
1253	40\\
1339	42.44\\
1429	39.1\\
1417	37.99\\
1494	31.16\\
1365	34.94\\
1277	23.86\\
1245	22.81\\
1206	22.99\\
1087	22.69\\
466	15.17\\
990	30.6\\
617	24.77\\
93	31.35\\
256	37.19\\
415	42.86\\
175	44.29\\
109	43.51\\
458	41.78\\
260	38.44\\
160	35.96\\
223	34.14\\
402	33.95\\
546	35.52\\
628	35.92\\
1279	32.87\\
1419	30.33\\
1344	29.1\\
1398	29.04\\
1059	28\\
988	26.17\\
786	21.92\\
981	25.09\\
838	11.67\\
877	19.67\\
890	29.73\\
976	37.92\\
1116	42.69\\
1237	41.95\\
1212	42.92\\
1053	42.97\\
1007	38.04\\
981	35.03\\
1160	33.65\\
1154	32.39\\
1208	32.18\\
1640	35\\
1809	35.97\\
1765	37.94\\
1851	35\\
1921	33.07\\
1872	35.95\\
1861	31.88\\
1366	29.98\\
1310	29.05\\
1497	25.81\\
1588	25.32\\
1523	25.94\\
1510	27.3\\
1342	30.9\\
1435	38.67\\
1428	38.07\\
1486	36.5\\
1457	35.94\\
1360	36.67\\
1103	34\\
1101	34.23\\
1091	32.23\\
1148	32.44\\
1301	34.03\\
1537	44.94\\
1756	36.4\\
1666	39.08\\
1821	35.93\\
1931	32.88\\
2011	44.92\\
1796	34\\
1387	29.63\\
1251	26.64\\
1397	25.79\\
1557	25.05\\
1527	25.62\\
1392	27.32\\
1490	29.62\\
1548	36.49\\
1549	36.9\\
1552	35.45\\
1381	35.23\\
1317	36.41\\
1400	36.97\\
1259	36.55\\
1222	35\\
1332	34.99\\
1387	35.4\\
1628	39.94\\
1691	39.93\\
1507	41.36\\
1554	40.07\\
1751	39.95\\
1904	40.34\\
2082	40.5\\
1691	31.13\\
1390	30.14\\
1493	30.45\\
1623	30.14\\
1723	30.14\\
1644	28.85\\
1590	32.66\\
1464	40\\
1618	41.42\\
1574	46.03\\
1469	48.96\\
1415	50.01\\
1414	48.96\\
1452	49.99\\
1510	48.96\\
1715	46.46\\
1746	45.86\\
1803	39.99\\
1936	38.87\\
1827	38.16\\
1695	36.12\\
1708	35.01\\
1820	36.91\\
1612	35\\
1708	36.5\\
1401	32.09\\
1144	30.11\\
1227	27.98\\
1396	27.99\\
1341	27.67\\
1209	29.31\\
1074	30.1\\
1478	32.07\\
1554	36\\
1535	40\\
1449	46.56\\
1352	40\\
1281	36\\
1182	32.46\\
1204	32.44\\
1238	31.91\\
1448	32.96\\
1445	35.91\\
1506	38.05\\
1599	38.9\\
1609	38.1\\
1676	47.03\\
1731	36\\
1461	31.04\\
1299	29.23\\
1248	28.56\\
1223	27.95\\
1173	27.12\\
1143	26.83\\
1100	26.91\\
1063	26.04\\
890	26.07\\
1009	29.17\\
848	17.03\\
878	19.28\\
756	22.48\\
702	18.59\\
829	15.91\\
867	15.73\\
833	13.87\\
960	32.46\\
1248	34.94\\
1335	37.11\\
1565	36\\
1577	44.11\\
1436	48.96\\
1352	37.11\\
1262	36\\
1156	30.64\\
1329	29.08\\
1286	27.67\\
1340	27.84\\
1270	28.97\\
1426	36.98\\
1875	41.97\\
1882	43\\
1932	41.28\\
1980	39.99\\
1896	47\\
1871	42.44\\
1897	42.44\\
1998	40.23\\
2047	37.44\\
2089	37.94\\
1978	41.02\\
1947	41.58\\
1789	42.95\\
1661	39.93\\
1702	35.99\\
1885	36\\
1711	30.44\\
1322	29.73\\
1128	28.58\\
961	11.2\\
967	5.23\\
972	25.85\\
1308	29.31\\
1031	32.48\\
1181	39.41\\
1189	42.5\\
1134	42.31\\
1209	42.5\\
1158	42.5\\
1298	42.5\\
1076	42.5\\
972	36\\
996	34.97\\
1072	34.94\\
1023	36.48\\
1183	36.5\\
1224	35.78\\
1760	35.94\\
1769	36\\
2033	48.97\\
1953	34.1\\
1625	33.1\\
1582	33.7\\
1609	33.7\\
1568	31.97\\
1605	31.57\\
1540	31.54\\
1567	32.17\\
1527	35.98\\
1403	41.03\\
1291	40.59\\
1300	39.98\\
1296	39.99\\
1255	37.44\\
1193	38\\
1238	38\\
1308	38\\
1354	38\\
1515	48.96\\
1676	50\\
1677	44.94\\
1683	39.94\\
1713	34.1\\
1848	49.94\\
1588	44.58\\
1597	32.59\\
1175	32.68\\
1275	27.19\\
1416	31.41\\
1402	30\\
1583	31.71\\
1607	32.42\\
1819	36.93\\
1801	38\\
1813	34.94\\
1785	38\\
1573	34.94\\
1617	34.71\\
1545	35.81\\
1544	38\\
1503	39.94\\
1599	43.05\\
1540	44.94\\
1753	35\\
1654	38\\
1867	40\\
2026	35.97\\
2222	41.15\\
1952	38\\
1884	31.93\\
1561	31.62\\
1713	31.58\\
1622	31.57\\
1686	31.61\\
1606	31.64\\
1651	37.78\\
1606	40.3\\
1665	39.96\\
1614	44.03\\
1445	48.97\\
1233	46.44\\
1507	39.99\\
1355	38.74\\
1321	37.44\\
1348	35.02\\
1284	42.97\\
1399	47.51\\
1569	46.44\\
1634	45.26\\
1826	48.9\\
2140	48.77\\
2272	57.44\\
1944	44.13\\
1738	32.03\\
1580	32.03\\
1352	34.77\\
1301	32.03\\
1358	31.67\\
1306	29.99\\
1207	27.27\\
1319	29.06\\
1324	29.94\\
1204	38.1\\
997	41\\
1014	44.94\\
972	30.01\\
1009	42.44\\
1022	42.44\\
1134	37.73\\
1245	37.68\\
1331	41.45\\
1346	49\\
1451	39.78\\
1715	45.63\\
1823	44.58\\
1898	49.28\\
1724	39.78\\
1265	31.31\\
1013	28.48\\
990	28.09\\
972	29.2\\
1022	28.91\\
1000	27.15\\
969	10.59\\
914	10.9\\
797	12.93\\
670	13.47\\
695	13.73\\
704	14.92\\
573	15.73\\
339	14.6\\
529	14.44\\
357	13.9\\
576	13.5\\
560	14.51\\
651	17.39\\
972	30.79\\
976	38\\
1031	46.48\\
1080	52.44\\
972	39.8\\
1072	27.59\\
1072	27.59\\
1029	17.91\\
1072	27.56\\
1075	29.4\\
1174	29.56\\
1430	32.6\\
1436	38.09\\
1668	37.4\\
1803	38.7\\
1926	39.5\\
1919	43.4\\
1791	39.41\\
1676	38\\
1615	35.4\\
1593	35\\
1614	35\\
1854	40\\
1781	36.94\\
1736	38.3\\
1986	38.09\\
2137	39.94\\
2263	47.44\\
2123	32.73\\
1721	29.24\\
1320	29.14\\
1254	29.19\\
1348	29.93\\
1392	29.95\\
1628	29.95\\
1675	32.58\\
1986	38.6\\
1785	41.04\\
1736	45.08\\
1605	49.69\\
1147	45.11\\
1205	47\\
1126	45.19\\
1143	44.94\\
1248	39.02\\
1355	42\\
1617	52.44\\
1904	39.49\\
1915	39.93\\
1951	38.99\\
2055	38.89\\
2242	35\\
2329	30.75\\
1720	28.95\\
1457	28.67\\
1378	28.22\\
1217	28.08\\
1214	27.78\\
1255	28.18\\
1312	30.76\\
1627	37.73\\
1959	40.19\\
1738	36.82\\
1660	37.44\\
1662	40\\
1471	36.82\\
1518	34.94\\
1402	32.65\\
1546	31.22\\
1553	32.44\\
1662	33.01\\
1691	37.9\\
1547	39.37\\
1492	38.76\\
1567	37.78\\
1790	32.75\\
1868	30.35\\
2002	31.21\\
1848	28.64\\
2002	28.41\\
2005	29.06\\
2093	29.04\\
2024	28.5\\
2259	30.61\\
2279	35.62\\
2002	37.95\\
2291	41\\
2149	47\\
1908	44.94\\
1933	39.94\\
1761	39.94\\
1746	39.5\\
1805	37\\
1880	42.44\\
1930	47.55\\
2138	46.44\\
1906	39.22\\
2265	38.63\\
2362	38.3\\
2496	47\\
2262	39.99\\
2645	33.71\\
2792	31.2\\
2750	30.44\\
2651	30.45\\
2682	30\\
2686	29.94\\
2799	39.97\\
2550	44\\
2322	42.03\\
2346	44.44\\
2117	42.83\\
2063	44.99\\
2067	42.44\\
2285	59.94\\
2312	52.63\\
2382	52.63\\
2227	52.63\\
1947	52.27\\
1958	44.94\\
2080	41.07\\
1878	36.87\\
1966	34.53\\
2071	32.8\\
2110	38.16\\
2278	47.52\\
1944	30\\
1910	27.44\\
1749	27.17\\
1685	25.3\\
1679	24.6\\
1631	24.1\\
1828	26.03\\
1940	29.98\\
2088	34\\
2112	34.94\\
2082	35\\
1940	33.1\\
1699	31.9\\
1645	30.62\\
1645	30.6\\
1807	32.7\\
2118	35\\
2191	34.96\\
2435	39.94\\
2296	35.66\\
2367	43.78\\
2391	47.44\\
2469	42.44\\
2552	34.41\\
2220	34\\
1963	34.41\\
1764	30.43\\
1635	29.68\\
1688	27.27\\
1684	26.18\\
1727	27.2\\
1942	28.03\\
1932	29.94\\
1919	32.44\\
1970	33.7\\
1864	33\\
1658	32.44\\
1576	30.43\\
1393	29.93\\
1402	29.94\\
1515	32.44\\
1574	32.7\\
1769	33\\
1913	32.7\\
2181	43.26\\
2167	47.36\\
1983	29.19\\
1495	26.17\\
1254	13.45\\
1191	8.99\\
1174	5.37\\
1288	21.95\\
1592	24.8\\
2104	29.4\\
2550	31.97\\
2738	34.96\\
2543	32.9\\
2463	32.96\\
2398	34.94\\
2410	35\\
2314	35.37\\
2348	33\\
2440	33.7\\
2591	35.7\\
2547	37.87\\
2749	34.94\\
2712	37.27\\
2771	36.91\\
2848	38.32\\
3199	35.7\\
2751	29.94\\
2181	30.85\\
2173	29.87\\
2049	26\\
2067	25.1\\
2345	27.48\\
2067	26.96\\
2513	35.01\\
2385	41.05\\
2419	39.18\\
2468	38.28\\
2412	35.12\\
2498	37.2\\
2283	33.3\\
2181	34.94\\
2235	34.94\\
2284	34.94\\
2382	37.2\\
2505	43\\
2714	41\\
2369	41.1\\
2482	42.44\\
2597	39.86\\
2834	40\\
2770	37.2\\
2649	30.86\\
2441	30.12\\
2706	30.54\\
2716	30.45\\
2545	30.05\\
2812	30.56\\
2753	33.95\\
2730	38.19\\
2951	39.98\\
2827	41.78\\
2656	44.8\\
2534	48\\
2313	42.11\\
2234	42.98\\
2161	41.97\\
2169	41.26\\
2201	40.1\\
2263	45.51\\
2167	42.97\\
2291	39.97\\
2317	37.23\\
2713	38.65\\
3024	37\\
2762	34\\
2782	30.82\\
2866	30.86\\
2789	29.86\\
2746	29.08\\
2832	28.97\\
2916	29.85\\
3155	33.99\\
3195	39.03\\
2812	40.63\\
3021	37.27\\
3016	42.44\\
3021	42.44\\
2790	39.94\\
2673	39.94\\
2672	39.94\\
2687	39.94\\
2729	42.44\\
2723	47.44\\
2943	39.94\\
2778	41\\
2923	41.08\\
3044	40.97\\
3152	37.43\\
3206	35.99\\
3166	31.15\\
2889	31.21\\
2706	30.3\\
2624	29.78\\
2578	29.79\\
2322	29.92\\
2191	31.36\\
2179	36.18\\
2312	35.91\\
2297	37.97\\
2349	40.07\\
2398	41.05\\
2339	37.97\\
2193	36.4\\
1930	35\\
1804	33.17\\
1842	32.54\\
2122	34.99\\
2445	37.91\\
2642	40\\
2810	40\\
2766	37.97\\
2837	42.44\\
2521	39.94\\
2408	46.95\\
2200	32.5\\
2215	32.27\\
2264	31.9\\
2253	30.96\\
2328	31.64\\
2073	29.35\\
2121	31.9\\
2369	34.99\\
2166	45\\
2149	49.94\\
2096	49.94\\
1947	45.12\\
1773	37.44\\
1586	32.2\\
1471	31.9\\
1543	31.72\\
1816	32.5\\
2065	40\\
2062	44\\
1896	41.11\\
1832	47.19\\
1988	50.56\\
1903	32.5\\
1907	30.96\\
1578	29.64\\
1276	28.03\\
1288	27.08\\
1231	5.51\\
1254	8.06\\
1291	18.23\\
1280	24.33\\
1461	25\\
1303	31.31\\
1482	29.96\\
1548	30.11\\
1520	31.9\\
1271	30.97\\
1267	30.12\\
1254	17.44\\
1269	31.49\\
1327	32.92\\
1566	39.96\\
1661	47\\
1373	44.94\\
1500	40\\
1526	39.94\\
1732	32.45\\
1497	41.99\\
1544	28.51\\
1592	28.07\\
1544	28.09\\
1562	28.28\\
1553	30.03\\
2021	33.24\\
1683	35\\
1806	40\\
1710	43.78\\
1694	41.89\\
1521	45.85\\
1369	41.76\\
1409	48.53\\
1538	44.72\\
1692	42.22\\
1860	42.8\\
2124	47.44\\
2175	43.78\\
2081	43.78\\
1988	45\\
2006	48.97\\
1992	48.53\\
2094	48.53\\
2811	38.38\\
2740	36.11\\
2829	34.28\\
2914	33.45\\
2881	33.45\\
2937	33.59\\
2951	33.96\\
2946	35.6\\
3177	40.64\\
2921	52.44\\
2777	53.89\\
2764	55.69\\
2778	52.44\\
2441	40\\
2704	44.96\\
2712	39.99\\
2793	39.65\\
2779	35.03\\
3243	38\\
3256	39.94\\
3256	47.44\\
3160	37.18\\
3107	35.29\\
2886	30.63\\
2787	31.22\\
2707	29.94\\
2823	29.46\\
2827	29.46\\
2863	31.43\\
3177	32.18\\
3071	33.36\\
2782	40\\
2568	39.5\\
2552	52.44\\
2305	52.44\\
2249	55.04\\
1835	39.94\\
2115	49.99\\
1869	39.94\\
2035	40.78\\
2222	44.94\\
2459	58.25\\
2628	38.77\\
2600	42.33\\
2538	41.05\\
3102	45\\
3048	44.96\\
3028	39.5\\
2934	34.01\\
2918	32.66\\
2954	32.11\\
2856	27.96\\
2819	27.79\\
2922	29.16\\
2997	39.37\\
3256	66.87\\
3208	43.35\\
2556	52.2\\
2417	55\\
2370	59.3\\
2466	55\\
2189	40.17\\
2852	48\\
2875	43.78\\
2913	44.31\\
2740	45\\
3183	38.71\\
2941	41.02\\
2904	41.01\\
2753	39.98\\
2770	37.09\\
2814	32.94\\
2501	29.92\\
2440	28.99\\
2462	28.47\\
2519	28.05\\
2513	28.09\\
2546	29.37\\
2423	32.82\\
2353	38.91\\
2996	41.43\\
2663	38.2\\
2636	35.7\\
2660	37.44\\
2486	34.96\\
2729	39.37\\
2595	37.75\\
2713	38.3\\
2805	39.94\\
2731	44.94\\
2683	43.78\\
2714	40\\
2888	40\\
2966	42.5\\
3225	39.94\\
2891	40\\
2457	37.3\\
2437	39.97\\
2485	38.3\\
2421	32.7\\
2371	32.04\\
2370	33.37\\
2272	34.94\\
2048	33.37\\
2236	34.99\\
2075	35\\
2063	40\\
1925	39.38\\
1848	34.94\\
1912	33.37\\
1835	34.99\\
1668	32.54\\
1778	32.72\\
2086	39.94\\
2491	35.09\\
2591	35\\
2800	40\\
2609	44.94\\
2551	44.73\\
2712	48.75\\
2404	50.28\\
2020	39.45\\
2162	36.94\\
2071	32.7\\
2117	32.43\\
2104	32.42\\
2097	32.7\\
1929	31.74\\
1911	32.43\\
1828	33.59\\
1724	39.36\\
1763	42.44\\
1683	48\\
1467	40.5\\
1401	37.44\\
1387	35\\
1530	37.12\\
1812	44.94\\
2086	39.94\\
1956	34.94\\
2199	39.94\\
2571	57.44\\
2586	50.23\\
2686	47.06\\
2023	39.38\\
2104	33.87\\
2144	32.83\\
2239	29.3\\
2192	29.42\\
2155	30.13\\
2169	37.91\\
1956	42\\
1927	40.81\\
2181	43.24\\
2397	49.45\\
2118	60\\
2053	64.44\\
1936	64.35\\
2176	48.96\\
2338	43.52\\
2306	43.29\\
1839	46.39\\
1771	54.79\\
1564	44.01\\
1515	44.52\\
1506	40.45\\
2105	38.37\\
2209	37.26\\
2354	32.28\\
2367	32.01\\
2168	31.69\\
2167	32.86\\
2276	33.29\\
2601	33.54\\
2583	36.46\\
2185	43.9\\
1992	48.86\\
2089	51.98\\
2149	53.59\\
2289	54.27\\
2089	48.96\\
2033	47.17\\
1889	44.08\\
2145	42.94\\
2085	42.09\\
2171	42.28\\
1848	43.74\\
2271	42.59\\
2370	41.97\\
2504	45\\
2791	45.67\\
2854	35.99\\
2774	32.49\\
2739	32.6\\
2720	32.5\\
2834	32.37\\
2847	32.64\\
2868	32.96\\
2720	35.99\\
2385	41.99\\
2485	45.03\\
2379	49.5\\
2315	48\\
2243	49.94\\
2200	47.44\\
2144	48.96\\
2108	48.75\\
2176	43.27\\
2219	42.13\\
2305	50.95\\
2537	47.44\\
2506	43.6\\
2043	42.02\\
2072	40.65\\
2576	40\\
2271	37.33\\
2677	38.91\\
2450	32.55\\
2227	31.05\\
2254	29.47\\
2228	27.9\\
2469	31.84\\
2583	35.77\\
2605	44\\
2444	41.63\\
2214	44.96\\
2244	49.41\\
2246	51.42\\
2074	44\\
2005	46\\
1990	46\\
2400	48.96\\
2368	49.41\\
1974	49.41\\
2221	46\\
2099	41.99\\
2252	46\\
2218	51.42\\
2443	39.99\\
2372	44.59\\
2171	35.14\\
2244	32.31\\
2455	32.71\\
2374	31.14\\
2454	31.5\\
2388	29.96\\
2660	37.46\\
2804	44.5\\
2990	47.81\\
2519	54.44\\
2416	55.84\\
2362	54.94\\
2253	45\\
2059	39.94\\
1968	38.73\\
1732	37.46\\
1634	35.88\\
1733	37.44\\
2089	38\\
2084	37.71\\
2129	37.52\\
1918	36.72\\
2907	40\\
2880	35.14\\
2822	36.32\\
2538	33\\
2249	29.78\\
2166	28.67\\
2141	25.1\\
2170	27.7\\
2228	32.99\\
2283	34\\
2487	34.96\\
2679	44.96\\
2773	47.44\\
2840	49.5\\
2811	45\\
2698	38.49\\
2608	36.25\\
2645	34.96\\
2653	35\\
2661	39.94\\
2800	42.44\\
2892	44.96\\
2671	51.99\\
2650	54.94\\
2772	49.94\\
2632	42.48\\
2719	35.03\\
2566	34\\
2777	33.9\\
2658	31.41\\
2614	31.31\\
2626	31.27\\
2672	31.38\\
2540	31.74\\
2545	32.15\\
2729	34.06\\
2618	34.9\\
2645	35.05\\
2479	35.02\\
2359	34.9\\
2170	32.64\\
2064	32.44\\
2091	32.4\\
2295	34.9\\
2529	36.3\\
2947	45\\
2789	52.44\\
2546	52.44\\
2795	50\\
2904	42.68\\
2683	34.89\\
2672	33.64\\
2612	34.22\\
2776	34.47\\
2897	34.88\\
2825	34.89\\
3254	41.03\\
3057	41.03\\
3158	43.42\\
3038	44.94\\
2934	43.76\\
2906	47.44\\
2958	50.44\\
2997	47.44\\
3045	44.94\\
3084	44.94\\
3145	48.96\\
3108	44\\
3246	40.21\\
3356	50\\
3319	44.94\\
3322	48.96\\
3356	50.44\\
3356	44.99\\
2671	37.88\\
2841	38.35\\
2827	37.55\\
2793	37.08\\
2734	37.07\\
2906	36.81\\
2754	41.2\\
3056	74.36\\
2851	55.39\\
2452	59.94\\
2017	62.29\\
1905	63.66\\
1897	59.96\\
1830	60.01\\
2038	58.2\\
2035	55.24\\
2030	52.21\\
2094	64.94\\
2473	42.44\\
2726	42.21\\
2981	53.11\\
2933	45.2\\
3056	52\\
2929	47.44\\
3054	49.99\\
3056	41\\
2949	38.96\\
2974	38.55\\
2981	38.62\\
2903	37.81\\
3036	42.54\\
3045	57\\
2856	57.86\\
2861	60.5\\
2572	58.9\\
2444	55.78\\
2512	50\\
2384	50\\
2313	47.82\\
2319	52.07\\
2407	55.78\\
2494	69.9\\
2691	41\\
2790	39.96\\
3056	55.78\\
2947	55.78\\
3056	60.39\\
3048	42.44\\
2909	37.18\\
2919	35.96\\
2857	35.5\\
2843	35.03\\
2915	35.03\\
3056	35.94\\
3056	39.3\\
2747	47.44\\
2328	45.68\\
2269	54.7\\
2094	44.94\\
2241	55.78\\
2058	44.96\\
2117	50\\
2217	49.94\\
2320	47.03\\
2445	54.96\\
2298	55\\
2861	55.31\\
3056	50\\
2977	52.44\\
2872	44.12\\
3056	39.97\\
3056	49.99\\
2890	39.5\\
3056	39.94\\
3006	35.28\\
3056	35.28\\
3056	37.5\\
3056	50\\
2946	38.26\\
3056	55.78\\
2928	49.48\\
2883	60.75\\
2757	63.99\\
2664	61\\
2503	55\\
2538	60\\
2582	54.94\\
2557	49.05\\
2593	47.44\\
2650	55.83\\
2760	54.94\\
2730	42.36\\
2813	47.44\\
2917	45.86\\
3056	55\\
2989	46.9\\
2948	60.12\\
3028	49.99\\
3171	46.01\\
3028	42.44\\
3072	39.92\\
3054	39.46\\
2727	41.49\\
2627	43.07\\
2831	45.71\\
2840	49.91\\
2907	60\\
2867	58.69\\
2832	52.44\\
2778	39\\
2925	37.81\\
2833	36.57\\
2882	36.57\\
3036	39.81\\
2982	39.94\\
3014	39.94\\
3052	55\\
2976	49.99\\
3055	47.44\\
2985	40\\
2766	37.13\\
2638	36.29\\
2976	36.29\\
3007	31.97\\
2912	31.74\\
2932	31.68\\
3020	32.09\\
2416	31.83\\
2476	32.56\\
2761	36.91\\
2632	39.94\\
2732	39.99\\
2613	40\\
2396	40\\
2489	37.07\\
2446	36.91\\
2454	36.53\\
2541	42.44\\
2643	49.78\\
3000	48.19\\
2823	59.94\\
2730	57\\
3050	46\\
3089	36.67\\
2924	31.9\\
3070	30.99\\
3234	30.83\\
3248	30.8\\
3071	31.34\\
2979	32.58\\
3256	54.99\\
3256	60\\
3256	53.11\\
3256	53.11\\
3256	53.11\\
3170	43.21\\
2794	43.53\\
2667	42.77\\
2678	45.13\\
2743	45.07\\
2901	45.54\\
3248	43.93\\
3256	54\\
3256	54\\
3256	65.15\\
3256	70\\
3256	57.11\\
3250	36.92\\
3256	35.45\\
3256	39\\
3256	41.42\\
3256	39\\
3256	35.47\\
3256	35.47\\
3080	37.95\\
2779	45.76\\
2635	50.95\\
2369	54\\
2803	53.75\\
2843	53.1\\
2504	45\\
2572	44.94\\
2799	48\\
2832	45.9\\
2911	44.19\\
3043	60\\
3248	45.38\\
2967	45.9\\
3050	53.78\\
2897	41.89\\
3191	44.17\\
3074	37.17\\
2887	35.24\\
2548	31.53\\
2576	31.33\\
2641	31.27\\
2619	31.73\\
2858	32.61\\
3112	43.04\\
3146	57.11\\
3146	57.11\\
2987	46.17\\
2947	44.01\\
2928	45\\
2854	42.45\\
2680	50\\
2622	45\\
2635	43.24\\
2899	46.09\\
2791	42.98\\
3146	55.1\\
2829	46.24\\
2854	56.23\\
3136	42.88\\
3146	72.09\\
3146	46\\
3128	35.39\\
3146	34.66\\
3084	31.55\\
3071	31.51\\
3146	34.71\\
3146	34.91\\
3146	56\\
2870	45.96\\
2980	47.67\\
3146	65.4\\
3077	45.85\\
3008	45.96\\
2796	44.42\\
2366	42.47\\
2646	41.05\\
2619	39.66\\
2654	38.48\\
2921	41.46\\
3021	45.91\\
2884	47.1\\
2796	46.18\\
2912	42.64\\
2915	37.8\\
2779	36.29\\
3106	32.54\\
2891	32.21\\
2966	32.14\\
3003	32.2\\
2937	32.22\\
3047	32.46\\
3146	40.52\\
2861	44.94\\
2840	47.39\\
2852	47.8\\
2609	47.32\\
2464	47.09\\
2277	46.24\\
2197	42.91\\
2260	39.34\\
2402	38.1\\
2591	35.85\\
2579	42\\
2712	39.38\\
2590	40.97\\
2752	50\\
2899	43.74\\
2851	50\\
2590	44\\
2663	36.27\\
2622	34.34\\
2645	34.09\\
2587	32.81\\
2474	32.77\\
2429	32.84\\
2386	33.96\\
2298	34.29\\
2370	41.52\\
2345	44.94\\
2296	45\\
2175	40\\
2086	37.93\\
2166	37.85\\
2073	35.2\\
2023	34.94\\
2113	35.18\\
2295	38.52\\
2296	43\\
2306	49.94\\
2160	59.94\\
2160	50\\
2279	40.88\\
2223	37.99\\
2258	34.33\\
2164	32.7\\
2116	31.95\\
2415	32.18\\
2488	32.22\\
2543	32.11\\
2632	32.51\\
2424	32.71\\
2436	33.38\\
2552	37.44\\
2537	40\\
2613	44.94\\
2523	49.99\\
2248	38.5\\
2129	36.87\\
2053	34.96\\
2133	35.87\\
2288	39.94\\
2503	38.5\\
2761	49.11\\
2664	59.94\\
2661	49.94\\
2758	48\\
2724	46\\
2686	47.46\\
2502	36.87\\
2437	36.65\\
2376	35.41\\
2408	35.23\\
2596	35.23\\
2663	43.27\\
2675	45.01\\
3028	47.92\\
2904	48.37\\
2755	49\\
2837	49.2\\
2779	47.75\\
2716	48.95\\
2576	48.56\\
2815	46.74\\
2742	59.58\\
3033	69.94\\
3146	53.11\\
3145	49.95\\
3044	48.41\\
2941	41.48\\
2709	41.93\\
2759	38.43\\
2809	34.99\\
2981	33.44\\
2885	33.24\\
2784	33.24\\
2876	33.57\\
2936	34.05\\
3146	45.01\\
3146	74.87\\
3146	57.11\\
3146	57.11\\
3146	57.11\\
3146	50.55\\
2738	47.23\\
2664	47.44\\
2679	46.91\\
2633	57.11\\
2818	49.94\\
3015	57.05\\
3146	57.11\\
3146	57.11\\
3146	74.95\\
3146	60\\
3146	57.11\\
3146	40\\
2815	37.56\\
2901	33.14\\
2879	32.5\\
2933	32.53\\
2935	33.44\\
3097	34.01\\
3146	58.13\\
3146	55.78\\
3146	69.21\\
3146	71\\
3146	55.78\\
3139	54.94\\
2765	52.5\\
2731	57.04\\
2646	57.05\\
2615	57.04\\
2589	57.04\\
2734	57.57\\
3132	47.74\\
3146	55.78\\
3146	55.78\\
3123	53.12\\
3113	46.46\\
2980	45.65\\
2964	37.61\\
3146	34.21\\
3146	34.21\\
3146	34.21\\
3146	37.61\\
3146	37.61\\
3146	42\\
3124	49.71\\
3146	55.78\\
3146	55.78\\
3032	57.04\\
3146	55.78\\
2856	48.7\\
2795	47.95\\
2824	47.93\\
2896	47.31\\
2996	45.74\\
3146	55.78\\
3146	55.78\\
3146	55.78\\
3146	55.78\\
3146	55\\
3146	55.12\\
3146	42\\
3146	70\\
3146	37.95\\
3146	37.95\\
3146	37.95\\
3146	37.95\\
3146	42\\
3146	60\\
3146	51.78\\
3146	51.78\\
3146	51.78\\
2969	58.95\\
2979	62\\
2874	48.51\\
2888	49.94\\
3030	57.04\\
3006	46.38\\
3090	50\\
3074	57.04\\
3146	51.78\\
3146	51.78\\
3146	60.01\\
3119	46.1\\
3019	52.44\\
3101	59.94\\
3146	53\\
3146	37.85\\
3146	39.01\\
3146	33.59\\
3122	33.37\\
3146	36.57\\
3146	39.99\\
3146	38.84\\
3146	46.44\\
3145	57.05\\
2876	53\\
2853	53\\
2704	48.24\\
2657	42\\
2584	40.24\\
2570	38.16\\
2729	38\\
2764	46.44\\
2911	52.44\\
2901	57.05\\
2929	69.96\\
2867	52.44\\
3115	44.94\\
3071	43.54\\
2900	42\\
2992	36.63\\
3124	35.18\\
3088	32.31\\
3028	32.05\\
3014	31.92\\
3146	34\\
2679	32.33\\
2746	31.7\\
2888	35.28\\
2932	34.17\\
2935	34.99\\
2901	37.08\\
2663	33.27\\
2237	32.82\\
2183	33\\
2195	32.69\\
2626	35.18\\
2846	42\\
3059	51\\
2964	60\\
3047	51\\
3112	42\\
3028	35.47\\
2567	36.63\\
2762	36.63\\
2752	33.73\\
2729	35.18\\
2833	35.59\\
3012	33.73\\
3146	44.99\\
3146	76.99\\
3146	54\\
3146	54\\
3085	49.31\\
3023	50.47\\
2826	51.24\\
2916	49.14\\
2900	48.76\\
2904	47.95\\
2813	47.34\\
2985	46.66\\
3146	54\\
3075	54\\
2869	70\\
3121	54\\
3039	44.58\\
3010	45\\
3146	59.99\\
3146	45\\
3146	38.4\\
3146	38.4\\
3146	38.4\\
3146	44.99\\
3146	72.2\\
3146	74.92\\
3146	60.27\\
3137	52.44\\
3120	50.44\\
3047	47.49\\
2818	44.06\\
2876	42.56\\
2863	43\\
2860	45.94\\
2976	45\\
3146	60.22\\
3146	72.2\\
3129	69.3\\
3130	64.94\\
3018	47.33\\
3146	65\\
3146	76.66\\
2914	40.43\\
3146	37.1\\
2997	33.3\\
2953	33.3\\
3002	33.14\\
3099	33.51\\
3146	42\\
2893	52.43\\
2805	54.01\\
2821	57.78\\
2652	55.49\\
2725	57.84\\
2545	51.56\\
2684	53.65\\
2739	50.26\\
2733	48.26\\
2800	47.05\\
2800	46.91\\
2697	59.17\\
2730	59.92\\
2778	52.1\\
2616	48.36\\
2878	48.43\\
2882	44.38\\
3146	47.7\\
3146	37.87\\
2980	32.61\\
2948	32.37\\
3079	32.9\\
3127	35.76\\
3146	51.7\\
3020	55.5\\
3146	61\\
3113	69.32\\
2951	60.21\\
2993	57.33\\
2766	49.95\\
2613	50.07\\
2560	48.5\\
2813	47.78\\
2961	46.05\\
3093	44.89\\
3146	51.78\\
3145	53.93\\
3146	74.57\\
2948	49.96\\
2884	49.94\\
3020	45.03\\
2667	44.16\\
2916	37.87\\
3044	37.87\\
2936	35.6\\
3006	34.29\\
2962	34.33\\
3063	40.06\\
3079	52.48\\
2912	52.37\\
3031	57.04\\
2940	59.94\\
2938	56\\
2921	52.13\\
2964	53.44\\
3039	48.5\\
3146	48.5\\
3146	48.5\\
3066	52.06\\
3146	50.44\\
3146	62.13\\
3025	57.44\\
2777	55\\
3008	50.44\\
3146	48.5\\
2667	59.46\\
2565	45.91\\
2593	40.36\\
2756	38.87\\
2807	38.87\\
2756	38.87\\
2487	40.11\\
2564	46.93\\
2783	50.84\\
2983	55\\
2923	59.94\\
2884	57.05\\
2803	50.44\\
2664	42.44\\
2465	39.99\\
2449	39.55\\
2581	40.69\\
2888	48.76\\
2968	55\\
3016	61\\
3101	61\\
3146	50.44\\
2980	43.95\\
2710	41.43\\
2612	43\\
2665	35.3\\
2629	35.24\\
2590	33.95\\
2546	32.67\\
2604	32.35\\
2629	33.32\\
2354	32.35\\
2474	37.67\\
2459	43.4\\
2433	47.44\\
2356	52.44\\
2228	51.03\\
2010	41.12\\
2003	37.83\\
2010	36.21\\
2194	36.53\\
2503	44.94\\
2841	53.95\\
2918	59.92\\
3058	75\\
3016	59.94\\
3055	50\\
2811	42.44\\
2299	32.74\\
2408	32.67\\
2506	32.74\\
2600	32.23\\
2548	32.61\\
2664	33.47\\
3082	44.95\\
2766	53.97\\
2877	57.54\\
2934	55.88\\
2805	50.12\\
2814	51.06\\
2745	51.67\\
2680	51.78\\
2492	48.23\\
2861	47.5\\
2744	48.34\\
3035	60.01\\
3146	60\\
3064	63.24\\
2940	57.05\\
2987	47.5\\
3146	47.5\\
3062	40.7\\
2896	36.51\\
2979	35.54\\
3146	36.57\\
3146	37.6\\
3146	37.6\\
3088	34.15\\
3112	44.49\\
2834	54.76\\
2959	58.41\\
3140	56.6\\
3058	54.56\\
2983	55.93\\
2874	52\\
2912	50.6\\
2874	48.59\\
2955	46.31\\
3041	48.67\\
3146	50.97\\
3146	53.11\\
3146	66.93\\
3146	64\\
3146	52\\
3146	52\\
3146	43\\
2300	46.44\\
3061	38.09\\
3135	32.01\\
3137	31.58\\
3134	32.13\\
3049	35.71\\
2719	48.11\\
2683	58\\
2389	60.88\\
2259	65\\
2014	65.08\\
2032	65.19\\
1846	63.65\\
1807	64.81\\
1845	60.9\\
2126	58\\
2377	52.7\\
2597	54.74\\
2684	56.75\\
2434	69.45\\
2216	70.98\\
2603	50.59\\
2394	54\\
2620	48\\
1710	38.29\\
2200	40.81\\
2200	38.29\\
2200	38.29\\
2200	39.88\\
2200	40.46\\
2200	54.3\\
1919	59.98\\
2009	59.64\\
1975	55.15\\
2200	54.3\\
2200	54.3\\
1970	55.98\\
1958	54.37\\
2200	56\\
2200	56\\
2200	54.3\\
2200	70\\
2200	54.3\\
1837	63.38\\
1969	61.04\\
2200	54.3\\
2200	54.3\\
2200	54.3\\
2769	50.98\\
3146	45\\
3146	41.47\\
3146	41.34\\
3146	41.52\\
3146	49\\
3146	50\\
3114	56.03\\
2927	56.42\\
2904	59.53\\
2654	59.88\\
2725	60.43\\
2394	60.01\\
2277	56.48\\
2390	54\\
2618	50.27\\
2851	49.47\\
3112	51.88\\
2720	54\\
2611	74.6\\
2526	79.03\\
2755	54\\
2420	56.96\\
2472	55.94\\
2805	38.65\\
3123	37.59\\
3146	38.69\\
3146	33.2\\
3105	31.35\\
2964	30.6\\
2491	30.83\\
2641	34.02\\
2450	38.19\\
2448	44.42\\
2412	42.44\\
2320	40.95\\
2111	41.86\\
1928	40.6\\
1909	38.41\\
1991	37.9\\
2292	38.16\\
2560	44.94\\
2655	52.44\\
2423	64.94\\
2360	53.11\\
2421	48.8\\
2430	46.5\\
2303	44.78\\
1622	38.91\\
1953	36.45\\
1991	32.43\\
2324	35.2\\
2415	34.99\\
2514	35.8\\
2216	32.8\\
2154	34.48\\
2080	38.64\\
2071	43\\
2145	42.44\\
2079	45.2\\
1957	45.2\\
1868	43\\
1820	39.94\\
1844	39.8\\
1876	41.32\\
2018	48.72\\
2319	46\\
2035	60\\
2036	59.94\\
2359	53.11\\
2543	45.97\\
2681	44.95\\
2401	43.52\\
2804	37.27\\
3162	37.95\\
3089	37.6\\
3286	37.18\\
3051	32.43\\
3294	43.24\\
3031	52.78\\
2853	56.79\\
2806	55.89\\
2653	57.6\\
2552	58.94\\
2217	59.31\\
2376	62.6\\
2376	59\\
2428	54.61\\
2684	51.78\\
3156	50.27\\
2659	59.3\\
2188	84.15\\
2103	68.49\\
2404	48.66\\
1882	51.71\\
2041	49.15\\
1903	40.5\\
2784	36.63\\
2658	32.74\\
2652	32.33\\
2881	32.9\\
3203	38.9\\
3167	38.99\\
2359	56.79\\
2061	60.08\\
1859	61.66\\
1619	58.12\\
1602	57.85\\
1048	57.21\\
1347	54.92\\
1633	48.29\\
1896	46.47\\
2061	44.42\\
2362	44.39\\
2203	47.62\\
1947	65.01\\
1844	57.46\\
2223	49.99\\
2157	50\\
2292	48.69\\
2294	41.26\\
3022	37.88\\
2921	37.16\\
2993	33.16\\
3040	37.02\\
3338	37.98\\
3183	38.51\\
2707	51.96\\
2431	55.31\\
2540	54.84\\
2359	53.77\\
2436	52.87\\
2414	51.29\\
2721	46.07\\
2761	44.96\\
2721	42.26\\
3085	40\\
3024	40.53\\
3177	41.14\\
2191	54.95\\
2398	51\\
2645	44.94\\
2646	42\\
2775	38.47\\
2640	32.53\\
2731	34.94\\
2683	33.12\\
2658	32.56\\
2755	33\\
3003	37.23\\
3060	37.92\\
2499	51.74\\
2501	53.11\\
2340	57.44\\
2449	51.17\\
2221	50.6\\
1692	52.94\\
1814	50.67\\
2106	48\\
2307	43.3\\
2619	41.08\\
2811	43.79\\
2298	51.83\\
2298	61.68\\
2138	53.11\\
2643	42.8\\
2272	46.48\\
2393	46.18\\
2382	37.88\\
2801	32.27\\
2740	28.93\\
2806	28.92\\
2873	29.64\\
3242	30.86\\
3155	37.41\\
2380	51.94\\
2254	54.15\\
2252	55.82\\
2126	53.23\\
2243	52.35\\
1714	52.73\\
1964	51.5\\
2026	49.55\\
2227	47.09\\
2458	45\\
2621	49.89\\
2377	52.22\\
2395	58.41\\
2416	52.06\\
2763	42.94\\
2834	42.1\\
2701	40.76\\
3066	38.85\\
2882	38.96\\
3080	36.05\\
2964	33.54\\
2963	33.54\\
3049	35.25\\
2829	39.15\\
2538	41\\
2713	46.36\\
2667	56.8\\
2652	62.23\\
2683	61.55\\
2517	49.5\\
2576	44.94\\
2470	42\\
2476	42\\
2625	44.94\\
2854	46.42\\
2656	43\\
2697	53.11\\
2662	47\\
2678	39.31\\
2440	39.66\\
2415	38.62\\
2845	34.75\\
2872	30.96\\
3281	33.85\\
3235	33.54\\
3250	32.69\\
3381	33.54\\
3066	31.16\\
2822	31.13\\
2879	34.94\\
2931	38.7\\
2936	40.12\\
2877	40.25\\
2679	40.63\\
2586	39.94\\
2586	38.8\\
2636	37.64\\
2463	38.99\\
2455	44.94\\
2408	47.44\\
2412	54.99\\
2279	49\\
2500	40\\
2550	39.25\\
2438	35.52\\
2392	29.4\\
3012	30.94\\
3054	30.71\\
3122	30.82\\
3205	32.16\\
3509	32.69\\
2946	51.8\\
2946	54.94\\
2767	55.95\\
2394	56.66\\
2465	51.84\\
2438	51.8\\
2289	51.2\\
2393	47.06\\
2482	43.26\\
2562	43.95\\
2715	44.92\\
2946	47.24\\
2657	51.8\\
2329	62.47\\
2308	59.91\\
2600	49.37\\
2599	47.67\\
2873	46.46\\
2946	36.68\\
2946	39.06\\
2936	31.83\\
2922	31.07\\
2946	36.67\\
2946	39.62\\
2946	40.41\\
2707	54.87\\
2314	57.34\\
2432	55.95\\
2307	53.79\\
2247	53.99\\
1986	54.22\\
2358	52.87\\
2434	50.16\\
2721	47.32\\
2925	42.86\\
2946	53.6\\
2757	57.77\\
2932	67.8\\
2790	59.89\\
2946	49.5\\
2946	48.5\\
2946	51.3\\
2946	40.28\\
2946	39.45\\
2946	41.3\\
2946	41.33\\
2946	51.36\\
2946	51.36\\
2946	51.36\\
2946	97.81\\
2803	54.15\\
2946	150\\
2946	77.1\\
2946	60.74\\
2759	54.66\\
2895	53.12\\
2946	51.5\\
2946	60.74\\
2946	70\\
2946	80\\
2946	55\\
2946	120\\
2913	52.32\\
2946	47.5\\
2946	47.5\\
2946	47.5\\
2946	40.23\\
2946	51.1\\
2946	44.99\\
2946	41\\
2946	39.17\\
2848	32.55\\
2946	40.98\\
2834	52.98\\
2686	54.94\\
2716	55.04\\
2714	56.55\\
2686	55.52\\
2683	52.66\\
2810	51.07\\
2946	49\\
2946	49\\
2946	49\\
2946	51.4\\
2946	50\\
2625	57.98\\
2604	51.78\\
2946	41\\
2946	40.28\\
2946	48.4\\
2946	47.1\\
2946	41.5\\
2946	47.1\\
2946	45.33\\
2946	47.1\\
2946	47.1\\
2946	51.2\\
2946	55.78\\
2850	53.87\\
2843	55.1\\
2790	54.11\\
2654	51.93\\
2672	49.18\\
2856	45.35\\
2873	43.63\\
2946	41.5\\
2946	48\\
2946	78.4\\
3362	45.94\\
2984	52.29\\
3065	45.15\\
3343	38.28\\
3420	45.5\\
3569	39.94\\
2964	38.34\\
2523	36.6\\
2370	32.8\\
2556	31\\
2375	30.63\\
2247	30.67\\
2250	29.99\\
2395	38.15\\
2564	40.25\\
2335	46.3\\
2374	46.3\\
2483	45\\
2359	42\\
2345	40.2\\
2319	38.09\\
2394	38.09\\
2539	40.3\\
2794	46.3\\
2770	46.3\\
2704	54.94\\
2242	51.77\\
2400	46.3\\
2624	46.28\\
2489	44.88\\
2369	26.25\\
2290	24.38\\
2487	26.55\\
2532	26.55\\
2454	22.76\\
2317	23.08\\
2185	26.31\\
2032	27.32\\
2098	28.89\\
2126	36.6\\
2126	36\\
2119	34.94\\
1939	30.92\\
1640	30.09\\
1554	29.87\\
1606	29.64\\
1749	29.83\\
2199	34.8\\
2040	42\\
2105	52.6\\
2163	49.8\\
2493	43\\
2477	37.99\\
2404	32.02\\
2094	20.59\\
1686	18.1\\
2375	16.37\\
2363	10.05\\
2369	11.52\\
2049	15.38\\
3044	34.94\\
3310	60\\
3252	45.61\\
3162	48.65\\
3113	46.76\\
3045	49.72\\
2723	53.61\\
2861	54.81\\
2986	52.95\\
3141	49.67\\
3239	48.21\\
3310	60.6\\
3310	50\\
3032	57.98\\
3310	50\\
3220	39.99\\
3100	41.64\\
2855	39.96\\
2812	35.14\\
2978	35.14\\
3023	34.68\\
2974	31.95\\
3014	30\\
3091	30.07\\
3008	37.44\\
2554	46.44\\
2446	48.93\\
2560	46.19\\
2548	44.94\\
2748	50.12\\
2651	46.54\\
2451	49\\
2551	47.01\\
2363	44.62\\
2335	42.44\\
2452	47.71\\
2766	42.12\\
2317	49.53\\
2241	39.96\\
2816	37.1\\
2769	39.71\\
2725	33.92\\
2115	42.44\\
2119	35.06\\
2333	35.06\\
2220	31.88\\
2357	32.22\\
2444	34.47\\
2691	42.94\\
3055	50.5\\
2896	50.15\\
2554	51.78\\
2822	59\\
2548	50\\
2572	49.21\\
2241	45.84\\
2444	45.24\\
2437	45\\
2388	49.94\\
2158	54.53\\
2824	59\\
1899	66.4\\
2502	59\\
2574	50\\
2454	53.12\\
2579	51.3\\
2169	48.41\\
2507	45.7\\
2754	41.49\\
2848	36.97\\
2838	36.06\\
2414	42.94\\
1350	52.58\\
1479	63.64\\
1338	70\\
1305	68.06\\
1300	61.05\\
1526	63.64\\
1678	57.02\\
1957	52.15\\
1683	52.9\\
1889	50.52\\
2135	48.91\\
2156	52.03\\
1407	61.05\\
1294	69\\
865	69.36\\
856	53.57\\
1653	47.75\\
1961	42.48\\
2508	38.5\\
2688	32.91\\
2694	30.79\\
2669	29.17\\
2715	24.99\\
2753	29.99\\
2761	40\\
2419	53.49\\
2155	53.11\\
2228	53.11\\
2201	54.53\\
2308	59.94\\
2330	52.44\\
2240	51\\
2346	50.25\\
2488	49.52\\
2520	49.51\\
2508	50.39\\
2933	49.93\\
2183	50.81\\
2345	50.5\\
2640	47.07\\
2724	49.99\\
2683	48.23\\
2322	47.02\\
2430	39.21\\
2660	33.32\\
2727	28.98\\
2650	33.15\\
2736	34.4\\
2369	23.34\\
2396	42\\
2722	46.59\\
2640	52.44\\
2616	55.78\\
2679	55.78\\
2864	57.5\\
2682	46.68\\
2477	39.94\\
2384	40\\
2624	42.44\\
2904	50.97\\
2741	57.5\\
2354	57.5\\
2402	55.78\\
2375	42.09\\
2681	45\\
2504	46.21\\
2474	44.99\\
2600	38.37\\
2643	33.7\\
2411	35.74\\
2493	28.39\\
2516	20\\
2521	18.3\\
2524	23.36\\
2530	25.56\\
1986	25.43\\
2104	37.44\\
2140	38.4\\
2100	42\\
2090	44.94\\
1991	38.3\\
1812	34.94\\
1732	34.93\\
1642	38.41\\
1457	48.29\\
1496	59.37\\
1699	59.37\\
1967	46.44\\
2122	43.71\\
2128	43.91\\
2484	43.9\\
1840	39.27\\
2011	32.75\\
2348	27.82\\
2386	22.1\\
2432	18.6\\
2854	22.93\\
2108	39.53\\
2151	44.8\\
2171	50.03\\
2224	51.86\\
2176	49.97\\
1668	48.86\\
1823	50\\
1929	49.12\\
2060	47.85\\
2235	48.76\\
2299	52.04\\
1828	60.51\\
1916	62.07\\
1779	56.95\\
2681	47.65\\
2782	44.94\\
2772	45.75\\
2572	40.5\\
2050	44.46\\
2337	38.6\\
2461	32.74\\
2716	31.54\\
2893	25.48\\
2908	31.84\\
2496	41.58\\
2206	48.66\\
1648	52.6\\
1658	52.6\\
1842	51.69\\
1786	51.7\\
1728	47.15\\
1747	48.69\\
1817	49.24\\
1993	46.32\\
2069	49.42\\
1525	58.4\\
1604	64.02\\
1468	62.19\\
1582	52.6\\
2072	45\\
2091	49.98\\
1893	49\\
2024	42.68\\
2336	38.17\\
2474	28.48\\
2746	24.8\\
2925	24.36\\
3127	28.42\\
2887	47.61\\
2581	50.38\\
2665	51.78\\
2631	52.87\\
2839	51.78\\
2944	52\\
2822	52.59\\
2860	52.79\\
2921	51.7\\
2782	52.25\\
2959	47.99\\
2469	54.35\\
2467	69.96\\
2094	60.97\\
2196	52.44\\
2615	46.93\\
2803	46.59\\
2883	47.11\\
2524	42.28\\
2612	32.36\\
2970	32.73\\
2991	29.94\\
2963	30.19\\
3266	34.29\\
2853	47\\
2395	54.56\\
2337	58.32\\
2543	57.53\\
2495	56.97\\
2519	56.71\\
2310	54.18\\
2443	52.99\\
2313	53.45\\
2474	52.93\\
2525	52\\
2528	56.42\\
2391	64.1\\
2136	54.91\\
2360	49.62\\
2487	45.92\\
2563	49.97\\
2530	43.83\\
2312	37.59\\
2459	30.64\\
2716	30.22\\
2730	28.83\\
2779	28.83\\
2742	30.64\\
2740	39.96\\
2630	50.03\\
2847	50.03\\
3062	49.11\\
2843	47.44\\
2742	50\\
2640	49.11\\
2495	48.38\\
2589	47.44\\
2642	45\\
2660	49.11\\
2630	58.49\\
2545	60\\
2569	46.14\\
2295	49.94\\
2095	49.94\\
2127	44.94\\
2156	36.17\\
1720	27.24\\
1569	19.67\\
1665	17.25\\
1677	15.34\\
1652	14.29\\
1841	13.52\\
1447	16.2\\
1318	17.61\\
1332	20.83\\
1322	30.97\\
1269	33.62\\
1260	33.6\\
1292	33.67\\
1376	33.56\\
1470	32.78\\
1645	34.68\\
1791	39.99\\
1927	50.22\\
1677	63.4\\
1711	51.37\\
1765	42.44\\
1969	37\\
2253	37.78\\
2318	32\\
1918	16.11\\
2098	14.91\\
1797	13.03\\
1634	10.78\\
1567	11.65\\
1557	11.88\\
1759	12.98\\
1905	13.55\\
1353	14.31\\
1459	15.1\\
1475	15.9\\
1566	16.9\\
1564	21\\
1563	19.96\\
1593	14.69\\
1707	21\\
1906	31.18\\
1965	42.44\\
1852	46.25\\
1795	40\\
1613	34.94\\
2019	32.76\\
2009	22.47\\
2013	19.16\\
1392	18.77\\
1377	14.5\\
1093	6.33\\
1451	0.12\\
1577	2.58\\
1884	13.48\\
1978	28.97\\
2499	39.8\\
2024	43.38\\
1766	46.64\\
1973	48.05\\
1869	48.09\\
1836	47.64\\
1946	46.54\\
1989	43.06\\
2117	43.12\\
2143	40.77\\
2230	49.23\\
1804	62.74\\
1712	58.03\\
1559	50.46\\
1157	54.69\\
1411	49.05\\
1826	46.8\\
1831	51.02\\
2110	42.57\\
2564	40.21\\
2635	34.18\\
2803	35.38\\
2853	39.44\\
3094	51.32\\
3073	62.7\\
2485	58.79\\
2626	58.9\\
2606	55.96\\
2623	54.2\\
2690	52\\
2919	49.52\\
3021	45.51\\
3113	46.41\\
3272	48.93\\
3136	57.47\\
2953	62.19\\
2558	63.95\\
3135	57.1\\
3200	44.52\\
3198	46.96\\
3477	43\\
3196	35.81\\
3517	34.85\\
3565	33.91\\
3562	31.65\\
3601	33\\
3590	30.71\\
3259	43.36\\
2573	53.75\\
2209	56.83\\
2122	56.28\\
2211	55.31\\
2000	55.87\\
2138	53.52\\
2385	53.5\\
2378	51.15\\
2493	51.46\\
2614	53.5\\
2368	61.3\\
2406	75.05\\
2044	67.79\\
2734	52.43\\
2712	46.78\\
2465	49.41\\
2165	49.77\\
2526	43.3\\
2513	40.63\\
2947	35.08\\
3068	30.21\\
3103	31.02\\
3193	38.74\\
2343	46.31\\
1658	55.66\\
1700	60.56\\
1765	60.06\\
1717	60.42\\
1497	60.08\\
1430	55.41\\
1406	53.75\\
1526	53.5\\
1489	53.96\\
1571	55.98\\
1483	63.72\\
1438	77.92\\
996	63.5\\
1191	54.92\\
2144	45.94\\
2619	47.44\\
2488	43.3\\
2337	29.88\\
2118	27.86\\
2511	26.7\\
2505	25.91\\
2418	26.81\\
2288	26.7\\
2606	34.76\\
2660	49.04\\
2685	50.4\\
2792	47.31\\
2765	44.96\\
2806	45.33\\
2842	43.3\\
2880	42.8\\
2865	41.37\\
2913	40.49\\
2735	43.3\\
2498	61.3\\
2641	61.3\\
2795	52.44\\
2545	44.96\\
2361	39.89\\
2637	42.15\\
2450	43.7\\
2279	34.94\\
1860	30.33\\
2316	29.75\\
2102	23.15\\
2016	23.32\\
2033	23.43\\
2069	28.75\\
1866	35\\
1866	39.98\\
1931	44.02\\
1661	43.99\\
1848	42.33\\
1832	44.75\\
1899	41.77\\
2217	37.7\\
2313	37.21\\
2252	39.23\\
2145	52.79\\
2294	53.6\\
1930	51.18\\
1812	44\\
2243	38.79\\
2133	41.43\\
2097	41.8\\
1996	38.38\\
1658	30.95\\
1711	20.78\\
2173	21.06\\
2174	21.11\\
1986	22.39\\
1943	22.7\\
1573	22.75\\
1634	23.34\\
1912	26\\
2087	33\\
2325	34.94\\
2326	40.46\\
2260	37.44\\
2198	33.47\\
2300	37.32\\
2350	39.94\\
2061	54.94\\
2301	64.99\\
2447	55\\
2144	48.06\\
2387	41.76\\
2357	44.22\\
2535	40.85\\
2427	38.43\\
2280	28.71\\
2629	25.14\\
2535	29.1\\
2542	24.08\\
2357	29.98\\
2226	42.92\\
2027	50.13\\
2037	50.43\\
1733	50.06\\
1801	49.97\\
1668	49.73\\
1436	50.04\\
1479	48.72\\
1628	48.44\\
1907	48.62\\
1979	47.99\\
2324	53.53\\
2422	60.33\\
2159	53.36\\
2527	44.96\\
2359	37.89\\
2711	40.38\\
2651	34.26\\
2130	34.22\\
2187	28.81\\
2378	27.25\\
2243	25.6\\
2119	25.44\\
1973	27.98\\
1160	31.29\\
781	41.13\\
597	42.72\\
797	42.87\\
1237	43.5\\
1442	44.94\\
1433	43.11\\
1266	44.5\\
1314	43.5\\
1424	44\\
1503	57.44\\
1110	60.99\\
1617	57.8\\
1834	51.91\\
2174	55\\
2058	42.36\\
2230	43.01\\
2275	35.38\\
2076	25.14\\
2318	23.88\\
2399	23.74\\
2345	23.7\\
2478	29.66\\
2366	25.85\\
2917	37.01\\
3209	46.68\\
2863	47.14\\
2897	49.77\\
3045	50.5\\
2990	50.58\\
2973	49.2\\
3096	46.98\\
3028	47.12\\
3023	49.15\\
2827	49.94\\
3338	60.7\\
3368	68.13\\
2920	57.81\\
3340	49.51\\
3105	50.8\\
3292	47.44\\
3180	43.13\\
2984	39.96\\
3023	35.13\\
2988	34.28\\
2887	33.32\\
3063	31.71\\
3067	31.71\\
3420	39.97\\
3381	54.09\\
3467	53.9\\
3445	55\\
3063	54.27\\
3075	53.8\\
3040	51.62\\
3103	49.91\\
3146	47.42\\
2670	47.01\\
2879	48.05\\
3503	58.01\\
3498	58.93\\
2966	52.97\\
3339	47.18\\
3025	37.93\\
3065	36.13\\
2884	36\\
2141	41.63\\
2158	40.22\\
2424	39.71\\
2323	32.44\\
2453	28.14\\
2515	29.2\\
2777	37.07\\
2895	47.3\\
2891	47.91\\
2920	46.21\\
2795	45.2\\
2925	44.03\\
2721	44.77\\
2799	47.44\\
2847	56.41\\
2946	58\\
2992	56.43\\
3013	71.2\\
3099	70\\
2749	56.1\\
2952	51.37\\
2716	50\\
3307	52.44\\
3288	44\\
2757	35.91\\
2883	39.99\\
2684	37.48\\
2668	35.17\\
2581	33.69\\
2661	33.5\\
2665	37.66\\
2607	39.94\\
2652	49.94\\
2545	44.99\\
2622	44.5\\
2678	50.5\\
2732	49.99\\
2730	45.2\\
2760	44.5\\
2661	43.27\\
2233	45.35\\
2022	68.9\\
2316	68.9\\
2693	57\\
2372	40.44\\
2530	42.21\\
2886	47.44\\
2939	43.91\\
2705	40.44\\
2699	41.17\\
2825	39.16\\
2706	34.99\\
2650	33.69\\
2633	33.69\\
2268	32.17\\
2004	31.31\\
2230	39.16\\
2481	43.75\\
2570	49.9\\
2694	53.49\\
2727	52\\
2707	44.94\\
2740	40.71\\
2870	41\\
2945	40.66\\
3129	59.96\\
3386	68.9\\
3505	51.78\\
3360	40.79\\
3291	41.17\\
3252	44.75\\
3249	40.83\\
2776	35.17\\
2793	34.02\\
2849	33.43\\
2746	33.8\\
2817	33.43\\
3034	33.21\\
3357	44.94\\
3534	51.32\\
3025	54.29\\
3536	53.4\\
3186	52.43\\
3263	53.79\\
3161	51.98\\
2957	54.15\\
2999	56.25\\
3023	54.97\\
2616	57.41\\
2842	70.11\\
2834	67.52\\
2396	54.99\\
3452	49.78\\
3490	42.07\\
3516	46.57\\
3558	46.2\\
3495	42.69\\
3750	44\\
3602	39.03\\
3488	36.55\\
3679	38.28\\
3431	37.17\\
3600	44.83\\
3533	53.5\\
2734	55.5\\
2782	54.99\\
2749	54.76\\
2795	54.96\\
3044	52.88\\
3156	53.03\\
3104	54.5\\
3066	54.75\\
2782	57.02\\
3048	78.61\\
3079	63.3\\
2909	58.3\\
3491	50.89\\
3601	50.8\\
3750	50.8\\
3750	49.99\\
3455	42.02\\
3633	38.43\\
3750	41.43\\
3750	40\\
3750	41.42\\
3750	41.47\\
3750	50.5\\
3217	55.19\\
2986	59.94\\
3021	60\\
2942	58.75\\
2898	57.78\\
2807	54.41\\
3025	54.32\\
2982	52.98\\
3234	50.5\\
2917	54.96\\
3020	75\\
2992	66.97\\
2654	59.45\\
2983	52.26\\
3559	47.61\\
3750	50.5\\
3750	49.99\\
3461	42.44\\
3750	41.49\\
3750	40.6\\
3750	40.39\\
3727	40.6\\
3750	41.57\\
3732	44.75\\
3226	58.69\\
2722	58.13\\
2728	56.37\\
2675	54.06\\
2694	56.15\\
2811	51\\
2801	50.45\\
2764	49.97\\
2728	50.57\\
2726	56.48\\
2740	75\\
2663	61.47\\
2349	58.34\\
2968	52.74\\
3461	46.85\\
3692	47.7\\
3750	46.4\\
3210	49.99\\
3285	41.74\\
3286	41.23\\
3200	38.63\\
3308	40.76\\
3413	41.45\\
3694	44.36\\
3551	54.94\\
3133	55.32\\
3164	56.37\\
3177	60\\
3159	64\\
3032	53.94\\
3226	56.36\\
3329	50.29\\
3255	50.38\\
3362	53.5\\
3056	65.12\\
3132	55.37\\
3370	52.85\\
3747	50.61\\
3750	49.88\\
3750	53.5\\
3750	52.32\\
2985	40.34\\
2600	35.88\\
2607	34.22\\
2808	35.13\\
2663	32.64\\
2510	31.49\\
2149	31.56\\
2292	39.96\\
2148	44.7\\
1932	46.3\\
1641	52.38\\
1606	54.94\\
1552	55.12\\
1639	44.94\\
1594	50\\
1910	49.21\\
1843	48.94\\
1740	66.4\\
2016	59.94\\
2306	47.39\\
2482	49.99\\
2455	45.23\\
2659	41.92\\
2717	40.8\\
2430	31.94\\
2118	28.91\\
2261	30.34\\
2420	31.87\\
2332	28.12\\
2243	28.74\\
2168	30\\
1937	30\\
1842	31\\
2028	36.57\\
2035	41.64\\
2069	50.21\\
1901	57.39\\
1743	46.79\\
1682	42.11\\
1769	42.84\\
1854	52.62\\
1513	69\\
1652	64.94\\
2031	57.39\\
2286	57.39\\
2368	44.5\\
2433	40.46\\
2682	37.93\\
2136	31.69\\
2203	31.71\\
2365	34.2\\
2370	34.2\\
2443	34.62\\
2128	30.34\\
2516	38.11\\
2521	51.34\\
2923	51\\
3030	49.44\\
2901	48.97\\
2716	50.32\\
2601	48.5\\
2201	49\\
2245	49.95\\
2323	50.54\\
2523	51.51\\
2630	60.16\\
2655	63.34\\
2517	57.47\\
2631	51\\
2724	48.58\\
2985	47.44\\
2925	44.88\\
2782	46.7\\
2723	41.79\\
3087	41.3\\
3080	37.7\\
3154	39.94\\
2889	35.7\\
3075	46.9\\
3123	56.27\\
2769	56.03\\
2478	55.94\\
2450	53.99\\
2364	54.33\\
2162	53.01\\
2326	54\\
2402	55\\
2401	57.29\\
2324	57.99\\
2654	69.44\\
2584	68.03\\
2579	57.72\\
2748	52.22\\
2982	47\\
3257	50.7\\
3482	49.27\\
2848	35.33\\
3017	36.09\\
3150	34.72\\
3038	34.45\\
3094	34.45\\
3324	34.72\\
3510	46.1\\
3510	59.99\\
3307	54.49\\
3510	52.5\\
3510	64.63\\
3510	78.9\\
3510	69.01\\
3510	69.06\\
3510	63\\
3510	69.06\\
3510	78.9\\
3369	84.63\\
3510	78.9\\
3352	54.35\\
3210	49.48\\
3510	78.9\\
3510	78.9\\
3510	55.72\\
3071	49.64\\
3317	43.01\\
3452	39.3\\
3408	35.7\\
3452	37.25\\
3459	42.94\\
3689	46.54\\
3700	60\\
3426	57.47\\
3518	58.93\\
3312	58\\
3562	58.1\\
3387	57.92\\
3654	55.5\\
3694	55.5\\
3700	55.5\\
3667	57.26\\
3591	80\\
3571	55.95\\
3601	56\\
3601	56\\
3373	50.06\\
3574	49.94\\
3535	44.19\\
3069	47.17\\
3176	42.49\\
3235	39.94\\
3096	35\\
3161	34.82\\
3266	37.08\\
3397	44.12\\
3130	56.33\\
3036	53\\
2869	55.5\\
2912	55.5\\
3071	55.5\\
3160	52.5\\
3186	51\\
3326	49.96\\
3338	50.06\\
3205	54.78\\
2699	72.32\\
2850	55.5\\
3313	50.06\\
3018	55\\
2903	50.06\\
3154	47.3\\
3200	44.94\\
2672	42.18\\
2752	37.01\\
2786	37\\
2586	32.84\\
2536	30.88\\
2682	30.88\\
2411	31.13\\
2380	38.25\\
2208	43.41\\
2507	46.8\\
2472	47\\
2477	48\\
2510	45\\
2446	44.71\\
2452	44.72\\
2623	44.76\\
2534	50.12\\
2605	76.4\\
2861	76.4\\
2710	52.44\\
2654	46.79\\
2652	45\\
2854	47.59\\
2979	47.49\\
3047	37.01\\
3130	34.51\\
3237	34.51\\
3173	34.24\\
2995	30.26\\
2944	30.41\\
2974	31.49\\
2699	32.14\\
};
\addplot [color=mycolor1,line width=1.0pt,mark size=0.3pt,only marks,mark=*,mark options={solid},forget plot]
  table[row sep=crcr]{%
2221	31.29\\
2369	40\\
2283	42.31\\
2364	42.41\\
2345	42.41\\
2246	42\\
2271	40.12\\
2413	34.94\\
2478	42.44\\
2440	65\\
2511	60.79\\
2518	54.06\\
2649	52\\
2393	42.62\\
2758	42.31\\
2775	32.24\\
3501	41.78\\
3540	34.77\\
3701	38\\
3520	33.74\\
3701	41.67\\
3701	39.16\\
3701	51.7\\
3555	53.43\\
3550	53\\
3459	54\\
3551	53\\
3456	56.36\\
3417	55.35\\
3222	57\\
3579	53.64\\
3701	53\\
3446	52.7\\
3284	60\\
3177	80.74\\
2739	59.71\\
3348	54.46\\
3701	54.4\\
3701	57\\
3701	57\\
3701	54.4\\
3701	51.4\\
3701	51.4\\
3800	51.4\\
3800	51.4\\
3800	51.4\\
3800	50\\
3383	57.76\\
2953	64.42\\
3016	64.83\\
3065	64.43\\
2874	63.36\\
3065	60.4\\
2563	59.07\\
2716	58.08\\
2818	56.96\\
2774	61.84\\
2669	77.52\\
2465	68.19\\
2643	60.09\\
3080	54.87\\
2608	48.06\\
2761	47.52\\
3062	45.87\\
3328	49.05\\
3295	46.3\\
3264	42.36\\
3339	31.67\\
3393	31.47\\
2750	35.85\\
2477	48.7\\
2243	63\\
2148	62.09\\
2144	64.9\\
2106	64.55\\
2159	69.8\\
2144	62.3\\
2230	64.62\\
2219	67.76\\
2420	65.66\\
2438	65.95\\
2447	82.87\\
2391	72.72\\
2239	69.91\\
2280	58.96\\
2940	50.6\\
2987	58\\
3251	56.31\\
3309	46.04\\
3495	46.06\\
3523	49.5\\
3378	43.62\\
3697	39.57\\
3285	47.54\\
3105	51.43\\
2690	60.87\\
2762	64\\
2870	67.64\\
2801	67.7\\
3044	67.7\\
2961	65.99\\
3023	67.47\\
2879	67.87\\
2951	65.83\\
2644	72.23\\
2977	87.97\\
2903	73.59\\
2799	64.67\\
2994	57.5\\
2742	56.89\\
3360	54.96\\
3839	49.4\\
2751	57.62\\
3346	54\\
3746	47.99\\
3530	45.17\\
3515	43.36\\
3690	47.16\\
3469	50.91\\
2737	66.64\\
2253	70.41\\
2570	68.99\\
2546	69.77\\
2557	66.39\\
2496	68.05\\
2526	64.39\\
2476	58.67\\
2466	57.5\\
2461	59.98\\
2200	70.61\\
1917	64.9\\
1968	63.4\\
1538	58.64\\
2345	48.04\\
3169	53.96\\
3459	52.44\\
3237	50.1\\
2779	46.49\\
2605	40\\
2791	39.14\\
3153	33.91\\
3216	33.99\\
2631	37.02\\
2598	47.46\\
3069	49.52\\
3405	50.43\\
3500	54.06\\
3500	59.99\\
3500	66.5\\
3436	50.03\\
3372	48.92\\
3401	48.91\\
3213	50.48\\
3326	62.17\\
3375	59.96\\
3441	53.97\\
3411	51.36\\
3433	48.69\\
3650	49.45\\
3668	48.66\\
3415	46.89\\
2959	42.43\\
2566	41.5\\
2664	30.07\\
3105	30.42\\
3107	31.33\\
3238	29.53\\
3127	33.89\\
2498	33.45\\
2282	42.89\\
2486	49.36\\
2579	53.36\\
2468	57.13\\
2489	52.87\\
2349	45.5\\
2388	41.27\\
2381	45\\
2210	60\\
2134	64.5\\
2270	52.72\\
2583	53.57\\
2711	51.4\\
2858	54\\
2906	50\\
2222	49.69\\
2112	46.82\\
2035	42\\
2333	39.14\\
2407	36.16\\
2223	40.78\\
2548	47.2\\
2538	57.53\\
3265	50.95\\
3368	50.82\\
3212	52\\
3289	54.06\\
3335	52.5\\
3304	51.5\\
3288	52\\
3333	51.5\\
3218	56.29\\
2964	73\\
3119	61.36\\
3110	56.25\\
2869	51.29\\
2758	45\\
3055	49.99\\
3351	49.02\\
2620	45.45\\
2496	41.37\\
3142	39.53\\
3210	31.52\\
3277	32.23\\
3343	39.85\\
2503	46.77\\
2282	62.52\\
2528	67.41\\
2174	69.85\\
2529	63.71\\
2248	61.28\\
2349	56.84\\
2389	57.92\\
2302	60.06\\
2281	59.74\\
2161	62.57\\
1893	74.69\\
1667	72.42\\
1717	62.32\\
1462	55.87\\
1959	50.24\\
2218	51\\
2330	46.92\\
1787	46\\
1531	43.2\\
1982	41.21\\
2103	36.27\\
2232	31.09\\
2269	39.49\\
2029	44.39\\
1788	52.78\\
1926	58.71\\
1998	59.25\\
2045	55.1\\
2036	57.96\\
1928	57.69\\
1954	56.84\\
1948	55.93\\
2111	55.93\\
2268	54.87\\
2298	59.72\\
2054	61.25\\
1527	59.68\\
1551	53.05\\
1203	50.45\\
1446	51.04\\
2301	50.73\\
2152	49.5\\
1975	45.34\\
2312	41.06\\
2445	36.53\\
2444	33.2\\
2651	38.12\\
2117	45.17\\
2175	55.16\\
2477	55.45\\
2537	52.02\\
2709	52.08\\
2709	53.01\\
2633	52.9\\
2624	52.59\\
2805	48.88\\
2908	46.96\\
2977	49.76\\
2805	59.99\\
2634	54.7\\
2262	54.15\\
2413	49.45\\
2130	45\\
2344	45.84\\
2543	46.62\\
2342	47.98\\
2380	41.66\\
2736	37.41\\
2661	31.22\\
2723	31\\
2735	37.93\\
2540	46.13\\
2533	54.96\\
2589	50.98\\
2851	52.46\\
3003	51.99\\
3014	59.94\\
2794	52.88\\
2809	50.22\\
2986	51.31\\
3080	48.85\\
3189	53.8\\
2610	64.94\\
2996	56.03\\
2723	53.98\\
2765	50.66\\
2698	46.06\\
2979	51.44\\
3101	48.9\\
3136	51.39\\
3022	49.04\\
3054	41.93\\
3106	40.46\\
3198	41.76\\
3177	43.5\\
2985	42.44\\
2801	47.17\\
3170	46.99\\
3195	60.32\\
3046	62.9\\
2973	62.9\\
3095	52.1\\
3053	50.96\\
2933	48.3\\
3021	45.67\\
3259	45.23\\
3067	62.9\\
3300	55.69\\
3165	51.08\\
3362	47.44\\
3286	42.5\\
3398	48.39\\
3636	49.03\\
3687	49.43\\
3287	44.37\\
3180	42.08\\
3726	38.76\\
3671	34.9\\
3688	35\\
3503	36.9\\
2877	39.99\\
2940	40.64\\
3064	44.99\\
3046	47.44\\
2817	50.44\\
2835	52.35\\
2401	50.97\\
2224	52.03\\
2417	50.2\\
2660	50.42\\
3045	59.2\\
2985	57.69\\
2595	57.44\\
2606	53.39\\
2517	49.6\\
2506	47.79\\
2530	46.1\\
1503	47.51\\
1326	38.65\\
1534	34.38\\
1739	30.42\\
1751	30.34\\
1204	36.5\\
1577	47.17\\
1778	55.54\\
2332	59.94\\
2436	61.45\\
2651	61.6\\
2711	61.66\\
2570	59.99\\
2818	58.78\\
2833	57.14\\
3008	54.13\\
3054	59.21\\
3036	74.27\\
2978	71.23\\
2700	68\\
2336	57.59\\
2612	51.6\\
2854	51.93\\
3017	50.27\\
3193	47.44\\
3331	43.5\\
3387	41.37\\
3453	33.75\\
3521	33.32\\
3408	36\\
3307	42.07\\
3046	54.75\\
2879	54.87\\
3169	55.96\\
3176	55.96\\
3119	55.64\\
2853	54.09\\
2911	53.74\\
2970	52.44\\
3172	51.42\\
3185	52.62\\
3395	60.99\\
3311	59.66\\
2917	58.07\\
3388	55.3\\
3362	47.46\\
3701	55.3\\
3568	47.12\\
2939	42.72\\
3067	40.44\\
2957	35.83\\
2814	32.35\\
2693	28.51\\
2776	29.19\\
2773	39.06\\
2778	50.5\\
2865	49.5\\
2941	51.02\\
2834	49.94\\
2894	52.44\\
2816	53.91\\
2707	54.3\\
2712	54.3\\
2740	57.39\\
2904	57.13\\
2721	75\\
2672	57.61\\
2843	54.3\\
2812	48.54\\
2986	45\\
2847	51.36\\
3278	47.16\\
2794	41.07\\
2970	40.4\\
2790	33.72\\
2504	33.2\\
2505	29.36\\
2722	29.75\\
2909	39.96\\
3063	51.13\\
3287	50.65\\
2458	63.7\\
3014	61.21\\
3029	62.3\\
2765	46.46\\
2850	52.59\\
2923	42.44\\
2943	41.85\\
2953	42.5\\
2951	49.5\\
3163	50.61\\
2879	48.73\\
2592	41.9\\
2246	40\\
2525	40.23\\
2787	37.43\\
2209	32.82\\
1871	29.15\\
2312	30\\
2143	29.49\\
2266	27.43\\
2707	27.48\\
2570	33.69\\
2435	52.25\\
2800	47.39\\
2856	68\\
2848	69.29\\
3066	66.53\\
3333	64.94\\
3217	49.67\\
3351	44.97\\
3403	43.03\\
3511	47.39\\
3594	67.23\\
3558	59.96\\
3423	51.99\\
3007	52.44\\
2672	48.81\\
2893	48.52\\
2829	44.94\\
1732	32.64\\
1804	27.91\\
1771	28.99\\
1512	24.69\\
1396	19.96\\
1444	13.15\\
1187	19.68\\
913	19.96\\
1251	38.59\\
1137	36.08\\
1118	40\\
1103	40\\
1026	40\\
1268	40\\
1229	39.42\\
1532	40.18\\
1675	53.38\\
1427	70.5\\
1646	70.5\\
1840	67.21\\
1740	48\\
1881	40.72\\
1953	42.44\\
2357	40.18\\
1842	39.39\\
1805	34.85\\
2116	36.08\\
1888	34.83\\
1807	33.99\\
1875	32.64\\
1728	30.74\\
1098	30.75\\
1444	30.16\\
1650	38.06\\
1364	43.86\\
1344	45.44\\
1310	63.18\\
1364	59.99\\
1272	40.34\\
1272	39.98\\
1272	39.49\\
1172	54.94\\
1195	63.18\\
1332	49.94\\
1404	47.44\\
1243	41.2\\
1675	39.73\\
1611	30\\
872	16.57\\
626	15.11\\
846	12.99\\
777	6.31\\
782	3.46\\
993	13.24\\
1208	19.22\\
1464	40.66\\
1988	44.39\\
2181	44.94\\
2268	45\\
2279	54.94\\
2191	47\\
2169	45\\
2108	54.43\\
2079	48\\
1918	55\\
1725	59.94\\
1866	45.81\\
1659	46.99\\
1335	49.94\\
1235	50\\
1488	40.28\\
1699	39.26\\
1216	28.49\\
1088	27.91\\
1057	27.28\\
1037	26.87\\
1060	27\\
1332	23.17\\
1274	32.95\\
1434	40\\
2100	45\\
2359	43.52\\
2352	47.39\\
2274	47.39\\
2425	43.16\\
2438	42.44\\
2391	41.81\\
2473	42.26\\
2287	47.39\\
2434	70\\
2540	47.39\\
2534	45\\
1882	47.39\\
1506	40.32\\
1846	44.94\\
1832	40\\
1352	29.63\\
1043	28.49\\
1089	28.49\\
974	28.12\\
949	16.36\\
1020	13.09\\
1242	34.6\\
1120	38.92\\
1552	44.5\\
1803	47.44\\
1946	47.44\\
1958	60\\
2006	59.99\\
2410	56.55\\
2447	49.34\\
2570	44.55\\
2523	47.39\\
2262	55\\
2506	47.39\\
2286	45\\
1795	44.94\\
1605	40.87\\
2110	44.34\\
2523	47.39\\
2583	37.18\\
2455	29\\
2020	28.49\\
2053	28.4\\
2134	28.49\\
2165	26.17\\
1861	18.85\\
1299	15\\
1482	26.03\\
1741	33.26\\
2080	37.83\\
2264	39.27\\
2152	39.69\\
2249	38.74\\
2199	38.45\\
2349	38.53\\
2272	39.99\\
2234	59.25\\
2254	50\\
2512	48.72\\
2451	50\\
2640	41.09\\
2973	42.44\\
3099	42.44\\
3030	39.27\\
2542	31.15\\
2541	29.51\\
2757	31.15\\
2700	29.54\\
2860	31.75\\
2843	31.75\\
3073	35.56\\
3461	40.29\\
3589	45\\
3664	49.4\\
3674	53.23\\
3401	49.99\\
3531	52\\
3635	45\\
3688	42.97\\
3746	44.99\\
3330	69.94\\
3226	52.71\\
2828	50.97\\
2768	44.41\\
2549	38.72\\
2578	41.76\\
2769	37.29\\
2457	32\\
1975	27.17\\
1786	22.64\\
2161	28.72\\
2180	20.14\\
2347	19.55\\
2146	29.3\\
1862	26\\
1889	34.22\\
2221	44.94\\
2527	47.97\\
2611	55.5\\
2420	58.2\\
2642	59.94\\
2480	42.44\\
2497	41.94\\
2361	44.94\\
2287	70.61\\
2358	69.94\\
2394	45.39\\
2300	44.96\\
2254	39.96\\
2560	39.48\\
2846	36.64\\
2598	29.99\\
2361	28.75\\
2251	20.73\\
2450	27.44\\
2468	29.54\\
2516	29.84\\
2424	29.87\\
2125	29.99\\
1704	27.51\\
1988	32.83\\
2237	34.09\\
2298	39.94\\
2237	40.11\\
2242	41.54\\
2362	42.23\\
2643	45.12\\
3006	49.94\\
3213	65\\
3338	66.69\\
3240	52.39\\
3347	60\\
3500	48.72\\
3395	51.85\\
3500	66\\
2919	49.95\\
2747	41.25\\
2596	42.24\\
2775	39.05\\
2702	35.05\\
2517	38.73\\
2445	40.74\\
2385	46.66\\
2833	47.99\\
2885	54.07\\
2905	53.73\\
3005	57.05\\
2903	57.37\\
3109	52.74\\
3083	50.6\\
3176	47.31\\
2857	51.88\\
2782	73.2\\
2942	60.12\\
3012	56.97\\
3214	53.8\\
3137	50.14\\
3480	50.12\\
3540	49.79\\
2404	43.36\\
2665	40.06\\
2747	35.27\\
3021	33.76\\
2984	33.71\\
3024	36.22\\
2457	40.25\\
2431	43.7\\
2824	49.04\\
3120	50.3\\
3058	51.3\\
3040	51\\
3122	50.84\\
3243	49.03\\
2690	47.06\\
2796	44.4\\
3029	47.53\\
3188	55.01\\
3017	58.15\\
2673	55.54\\
3045	51.63\\
2678	46.39\\
3059	49.73\\
3099	49.23\\
2725	48.76\\
2321	43.43\\
1973	41.19\\
2733	34.94\\
3112	32.32\\
3139	36.08\\
3183	41\\
2477	46.31\\
2708	46.85\\
2953	50.39\\
2627	52.35\\
2640	51.45\\
2477	51.77\\
2832	50.2\\
2863	45.75\\
3012	41.38\\
3163	41.77\\
2853	48.76\\
2598	52.92\\
2085	52.16\\
2226	48.01\\
2009	45.39\\
2324	48.43\\
2632	49.64\\
};
\addplot [color=mycolor2,solid,line width=2.0pt,forget plot]
  table[row sep=crcr]{%
-2117	20.646715190753\\
-2017	21.1472211952389\\
-1917	21.6477271997248\\
-1817	22.1482332042107\\
-1717	22.6487392086966\\
-1617	23.1492452131825\\
-1517	23.6497512176684\\
-1417	24.1502572221543\\
-1317	24.6507632266402\\
-1217	25.1512692311261\\
-1117	25.651775235612\\
-1017	26.1522812400979\\
-917	26.6527872445838\\
-817	27.1532932490698\\
-717	27.6537992535557\\
-617	28.1543052580416\\
-517	28.6548112625275\\
-417	29.1553172670134\\
-317	29.6558232714993\\
-217	30.1563292759852\\
-117	30.6568352804711\\
-17	31.157341284957\\
83	31.6578472894429\\
183	32.1583532939288\\
283	32.6588592984147\\
383	33.1593653029006\\
483	33.6598713073865\\
583	34.1603773118724\\
683	34.6608833163583\\
783	35.1613893208442\\
883	35.6618953253301\\
983	36.162401329816\\
1083	36.6629073343019\\
1183	37.1634133387879\\
1283	37.6639193432738\\
1383	38.1644253477597\\
1483	38.6649313522456\\
1583	39.1654373567315\\
1683	39.6659433612174\\
1783	40.1664493657033\\
1883	40.6669553701892\\
1983	41.1674613746751\\
2083	41.667967379161\\
2183	42.1684733836469\\
2283	42.6689793881328\\
2383	43.1694853926187\\
2483	43.6699913971046\\
2583	44.1704974015905\\
2683	44.6710034060764\\
2783	45.1715094105623\\
2883	45.6720154150482\\
2983	46.1725214195341\\
3083	46.6730274240201\\
3183	47.173533428506\\
3283	47.6740394329919\\
3383	48.1745454374778\\
3483	48.6750514419637\\
3583	49.1755574464496\\
3683	49.6760634509355\\
3783	50.1765694554214\\
3883	50.6770754599073\\
};
\end{axis}
\end{tikzpicture}%
    \caption{Exchange/Price bewteen [FR+NL] and BE}
    \label{France_Netherlands}
\end{figure}

The previous assumptions made for the aforementioned model still hold. However, this time, the quantity exchange between Belgium and neighboring markets is considered as a variable and as a function of the supply function previously defined. Moreover, even if the neighboring market is represented as a generator, it's not allowed to participate in the auction for reserve. The optimization model can be described as follows : \\

\begin{center}
\boxput*(0,1){\colorbox{white}{\textbf{ Import/Export Economic Dispatch [ImpExp] }}}{
\setlength{\fboxsep}{10pt}
\fbox{\begin{minipage}{0.7\textwidth} \vspace{0.2cm}
$$ \min\limits_{r_g, p_g \geq 0, p_{exch}} \quad  \sum_{g \in \G} \int_0^{p_g} MC_g(x) dx + 31.2424 \: \frac{p_{exch}^2}{2} + 0.005\: p_{exch}$$
$$
\begin{array}{llll}
(\lambda)				& \sum_{g \in \G} p_g + p_{exch} + p^{t}_{r} \geq \D^{t} & & (1) \\
(\mu)					& \sum_{g \in \G} r_g \geq \R & & (2) \\
							& r_g \leq 15\: R^{t}_g & & (3) \\
							& p_g + r_g \leq P^{t}_g & & (4) \\
							& -ATC_1^{t} \leq p_{exch} \leq ATC_2^{t} & & (5) \\
\end{array}
$$
\vspace{0.1cm}
\end{minipage}}}
\end{center}

where the ATC values are :
\begin{align*}
ATC_1^{t} &= NTC\_BE2FR^{t} + NTC\_BE2NL^{t} \\
ATC_2^{t} &= NTC\_FR2BE^{t} + NTC\_NL2BE^{t} 
\end{align*}
\subsection{Model 3}
% MODEL 3

One drawback of reserve requirement is that they may result in highly volatile energy prices when the system is strained. High prices indicate scarcity and the need for investment in capacity. They are therefore desirable, because they trigger investment. However, in systems with limited elasticity, energy price spikes result in enormous price volatility, which increases the risk of investment. Operating reserve demand curves have been proposed as an approach for achieving high energy prices in conditions of scarcity through prices spikes in that are more frequent but less elevated. This is achieved by expressing reserve requirement through a demand function, rather than a hard constraint that needs to be satisfied. Our third model implement the economic dispatch with an operating reserve demand curve and we will see if this model correctly fits the Belgium market.
But first and foremost, let's build the operating reserve demand curve. It's centered around the fixed reserve requirement of previous models and has a width which is equal to $20 \%$ of the national peak demand. The curve is depicted on figure [\ref{fig:ReserveCurve}]. We see that the breakpoint is around $16$, then the slope is about $-1.83$, to finally reach the x-axis at $2700$ MW. The figure [\ref{fig:ReserveUtility}] shows the utility cost of the reserve (integration of the marginal utility).

\begin{minipage}{0.495\textwidth} 
\begin{figure}[H]
    \centering
    \setlength\fheight{3cm}
    \setlength\fwidth{0.8\textwidth}
    % This file was created by matlab2tikz.
% Minimal pgfplots version: 1.3
%
%The latest updates can be retrieved from
%  http://www.mathworks.com/matlabcentral/fileexchange/22022-matlab2tikz
%where you can also make suggestions and rate matlab2tikz.
%
\definecolor{mycolor1}{rgb}{0.84706,0.16078,0.00000}%
\definecolor{mycolor2}{rgb}{0.87059,0.49020,0.00000}%
%
\begin{tikzpicture}

\begin{axis}[%
width=\fwidth,
height=\fheight,
at={(0\fwidth,0\fheight)},
scale only axis,
separate axis lines,
every outer x axis line/.append style={black},
every x tick label/.append style={font=\color{black}},
xmin=0,
xmax=100,
xtick={0,16,50,100},
xlabel={Reserve [MW]},
xmajorgrids,
every outer y axis line/.append style={black},
every y tick label/.append style={font=\color{black}},
ymin=4860,
ymax=5030,
ytick={5000},
ylabel={Marginal Utility [euro/MWh]},
ymajorgrids
]
\addplot [color=mycolor1,solid,line width=2.0pt,forget plot]
  table[row sep=crcr]{%
0	5000\\
16.0972929999998	5000\\
};
\addplot [color=mycolor2,dash pattern=on 1pt off 3pt on 3pt off 3pt,line width=2.0pt,forget plot]
  table[row sep=crcr]{%
16.0972929999998	4860.27302946636\\
16.0972929999998	5000\\
};
\addplot [color=mycolor1,solid,line width=2.0pt,forget plot]
  table[row sep=crcr]{%
16.0972929999998	5000\\
17.0972929999998	4998.13697372622\\
18.0972929999998	4996.27394745244\\
19.0972929999998	4994.41092117865\\
20.0972929999998	4992.54789490487\\
21.0972929999998	4990.68486863109\\
22.0972929999998	4988.82184235731\\
23.0972929999998	4986.9588160833\\
24.0972929999998	4985.09578980974\\
25.0972929999998	4983.23276353596\\
26.0972929999998	4981.36973726218\\
27.0972929999998	4979.5067109884\\
28.0972929999998	4977.64368471462\\
29.0972929999998	4975.78065844084\\
30.0972929999998	4973.91763216705\\
31.0972929999998	4972.05460589327\\
32.0972929999998	4970.19157961949\\
33.0972929999998	4968.32855334571\\
34.0972929999998	4966.46552707193\\
35.0972929999998	4964.60250079814\\
36.0972929999998	4962.73947452436\\
37.0972929999998	4960.87644825058\\
38.0972929999998	4959.0134219768\\
39.0972929999998	4957.15039570302\\
40.0972929999998	4955.28736942924\\
41.0972929999998	4953.42434315545\\
42.0972929999998	4951.56131688167\\
43.0972929999998	4949.69829060789\\
44.0972929999998	4947.83526433411\\
45.0972929999998	4945.97223806033\\
46.0972929999998	4944.10921178654\\
47.0972929999998	4942.24618551276\\
48.0972929999998	4940.38315923898\\
49.0972929999998	4938.5201329652\\
50.0972929999998	4936.65710669142\\
51.0972929999998	4934.79408041764\\
52.0972929999998	4932.93105414385\\
53.0972929999998	4931.06802787007\\
54.0972929999998	4929.20500159629\\
55.0972929999998	4927.34197532251\\
56.0972929999998	4925.47894904873\\
57.0972929999998	4923.61592277494\\
58.0972929999998	4921.75289650116\\
59.0972929999998	4919.88987022738\\
60.0972929999998	4918.0268439536\\
61.0972929999998	4916.16381767982\\
62.0972929999998	4914.30079140603\\
63.0972929999998	4912.43776513225\\
64.0972929999998	4910.57473885847\\
65.0972929999998	4908.71171258469\\
66.0972929999998	4906.84868631091\\
};
\addplot [color=mycolor1,dashed,line width=2.0pt,forget plot]
  table[row sep=crcr]{%
66.0972929999998	4906.84868631091\\
67.0972929999998	4904.98566003712\\
68.0972929999998	4903.12263376334\\
69.0972929999998	4901.25960748956\\
70.0972929999998	4899.39658121578\\
71.0972929999998	4897.533554942\\
72.0972929999998	4895.67052866822\\
73.0972929999998	4893.80750239443\\
74.0972929999998	4891.94447612065\\
75.0972929999998	4890.08144984687\\
76.0972929999998	4888.21842357309\\
77.0972929999998	4886.35539729931\\
78.0972929999998	4884.49237102552\\
79.0972929999998	4882.62934475174\\
80.0972929999998	4880.76631847796\\
81.0972929999998	4878.90329220418\\
82.0972929999998	4877.0402659304\\
83.0972929999998	4875.1772396662\\
84.0972929999998	4873.31421338283\\
85.0972929999998	4871.45118710905\\
86.0972929999998	4869.58816083527\\
87.0972929999998	4867.72513456149\\
88.0972929999998	4865.86210828771\\
89.0972929999998	4863.99908201392\\
90.0972929999998	4862.13605574014\\
91.0972929999998	4860.27302946636\\
};
\end{axis}
\end{tikzpicture}%
    \caption{Marginal utility of reserves}
    \label{fig:ReserveCurve}
\end{figure}
\end{minipage}
\begin{minipage}{0.495\textwidth} 
\begin{figure}[H]
    \centering
    \setlength\fheight{3cm}
    \setlength\fwidth{0.8\textwidth}
    % This file was created by matlab2tikz.
% Minimal pgfplots version: 1.3
%
%The latest updates can be retrieved from
%  http://www.mathworks.com/matlabcentral/fileexchange/22022-matlab2tikz
%where you can also make suggestions and rate matlab2tikz.
%
\definecolor{mycolor1}{rgb}{0.04314,0.51765,0.78039}%
\definecolor{mycolor2}{rgb}{0.84706,0.16078,0.00000}%
%
\begin{tikzpicture}

\begin{axis}[%
width=\fwidth,
height=\fheight,
at={(0\fwidth,0\fheight)},
scale only axis,
separate axis lines,
every outer x axis line/.append style={black},
every x tick label/.append style={font=\color{black}},
xmin=0,
xmax=3200,
xlabel={Reserve [MW]},
xtick={0,1000,2000,2700},
xmajorgrids,
every outer y axis line/.append style={black},
every y tick label/.append style={font=\color{black}},
ymin=0,
ymax=7000000,
ylabel={Utility [\euro]},
ymajorgrids
]
\addplot [color=mycolor2,solid,line width=2.0pt,forget plot]
  table[row sep=crcr]{%
0	0\\
16.0972929999998	80486.464999999\\
};
\addplot [color=mycolor2,solid,line width=2.0pt,forget plot]
  table[row sep=crcr]{%
16.0972929999998	80728.841331323\\
17.0972929999998	85727.9098181865\\
18.0972929999998	90725.1152787758\\
19.0972929999998	95720.4577130913\\
20.0972929999998	100713.937121133\\
21.0972929999998	105705.553502901\\
22.0972929999998	110695.306858395\\
23.0972929999998	115683.197187616\\
24.0972929999998	120669.224490562\\
25.0972929999998	125653.388767235\\
26.0972929999998	130635.690017634\\
27.0972929999998	135616.12824176\\
28.0972929999998	140594.703439611\\
29.0972929999998	145571.415611189\\
30.0972929999998	150546.264756493\\
31.0972929999998	155519.250875523\\
32.0972929999998	160490.373968279\\
33.0972929999998	165459.634034762\\
34.0972929999998	170427.031074971\\
35.0972929999998	175392.565088906\\
36.0972929999998	180356.236076567\\
37.0972929999998	185318.044037954\\
38.0972929999998	190277.988973068\\
39.0972929999998	195236.070881908\\
40.0972929999998	200192.289764474\\
41.0972929999998	205146.645620767\\
42.0972929999998	210099.138450785\\
43.0972929999998	215049.76825453\\
44.0972929999998	219998.53503001\\
45.0972929999998	224945.438783198\\
46.0972929999998	229890.479508121\\
47.0972929999998	234833.657206771\\
48.0972929999998	239774.971879147\\
49.0972929999998	244714.423525249\\
50.0972929999998	249652.012145077\\
51.0972929999998	254587.737738632\\
52.0972929999998	259521.600305913\\
53.0972929999998	264453.59984692\\
54.0972929999998	269383.736361653\\
55.0972929999998	274312.009850112\\
56.0972929999998	279238.420312298\\
57.0972929999998	284162.96774821\\
58.0972929999998	289085.652157848\\
59.0972929999998	294006.473541212\\
60.0972929999998	298925.431898303\\
61.0972929999998	303842.527229119\\
62.0972929999998	308757.759533662\\
63.0972929999998	313671.128811931\\
64.0972929999998	318582.635063927\\
65.0972929999998	323492.278289648\\
66.0972929999998	328400.058489096\\
67.0972929999998	333305.97566227\\
68.0972929999998	338210.02980917\\
69.0972929999998	343112.220929797\\
70.0972929999998	348012.549024149\\
71.0972929999998	352911.014092228\\
72.0972929999998	357807.616134033\\
73.0972929999998	362702.355149565\\
74.0972929999998	367595.231138822\\
75.0972929999998	372486.244101806\\
76.0972929999998	377375.394038516\\
77.0972929999998	382262.680948952\\
78.0972929999998	387148.104833115\\
79.0972929999998	392031.665691003\\
80.0972929999998	396913.363522618\\
81.0972929999998	401793.198327959\\
82.0972929999998	406671.170107026\\
83.0972929999998	411547.27885982\\
84.0972929999998	416421.52458634\\
85.0972929999998	421293.907286586\\
86.0972929999998	426164.426960558\\
87.0972929999998	431033.083608256\\
88.0972929999998	435899.877229681\\
89.0972929999998	440764.807824832\\
90.0972929999998	445627.875393709\\
91.0972929999998	450489.079936312\\
92.0972929999998	455348.421452641\\
93.0972929999998	460205.899942697\\
94.0972929999998	465061.515406479\\
95.0972929999998	469915.267843987\\
96.0972929999998	474767.157255221\\
97.0972929999998	479617.183640182\\
98.0972929999998	484465.346998869\\
99.0972929999998	489311.647331282\\
100.097293	494156.084637421\\
101.097293	498998.658917286\\
102.097293	503839.370170878\\
103.097293	508678.218398196\\
104.097293	513515.20359924\\
105.097293	518350.32577401\\
106.097293	523183.584922507\\
107.097293	528014.98104473\\
108.097293	532844.514140678\\
109.097293	537672.184210354\\
110.097293	542497.991253755\\
111.097293	547321.935270883\\
112.097293	552144.016261737\\
113.097293	556964.234226317\\
114.097293	561782.589164623\\
115.097293	566599.081076655\\
116.097293	571413.709962414\\
117.097293	576226.475821899\\
118.097293	581037.37865511\\
119.097293	585846.418462047\\
120.097293	590653.595242711\\
121.097293	595458.908997101\\
122.097293	600262.359725217\\
123.097293	605063.947427059\\
124.097293	609863.672102628\\
125.097293	614661.533751922\\
126.097293	619457.532374943\\
127.097293	624251.66797169\\
128.097293	629043.940542164\\
129.097293	633834.350086363\\
130.097293	638622.896604289\\
131.097293	643409.580095941\\
132.097293	648194.400561319\\
133.097293	652977.358000423\\
134.097293	657758.452413254\\
135.097293	662537.683799811\\
136.097293	667315.052160094\\
137.097293	672090.557494103\\
138.097293	676864.199801839\\
139.097293	681635.9790833\\
140.097293	686405.895338488\\
141.097293	691173.948567403\\
142.097293	695940.138770043\\
143.097293	700704.46594641\\
144.097293	705466.930096502\\
145.097293	710227.531220321\\
146.097293	714986.269317867\\
147.097293	719743.144389138\\
148.097293	724498.156434136\\
149.097293	729251.30545286\\
150.097293	734002.59144531\\
151.097293	738752.014411486\\
152.097293	743499.574351389\\
153.097293	748245.271265017\\
154.097293	752989.105152373\\
155.097293	757731.076013454\\
156.097293	762471.183848261\\
157.097293	767209.428656795\\
158.097293	771945.810439055\\
159.097293	776680.329195041\\
160.097293	781412.984924753\\
161.097293	786143.777628192\\
162.097293	790872.707305356\\
163.097293	795599.773956247\\
164.097293	800324.977580864\\
165.097293	805048.318179208\\
166.097293	809769.795751277\\
167.097293	814489.410297073\\
168.097293	819207.161816595\\
169.097293	823923.050309844\\
170.097293	828637.075776818\\
171.097293	833349.238217519\\
172.097293	838059.537631946\\
173.097293	842767.974020099\\
174.097293	847474.547381978\\
175.097293	852179.257717584\\
176.097293	856882.105026916\\
177.097293	861583.089309974\\
178.097293	866282.210566758\\
179.097293	870979.468797268\\
180.097293	875674.864001505\\
181.097293	880368.396179468\\
182.097293	885060.065331157\\
183.097293	889749.871456572\\
184.097293	894437.814555714\\
185.097293	899123.894628582\\
186.097293	903808.111675175\\
187.097293	908490.465695496\\
188.097293	913170.956689542\\
189.097293	917849.584657315\\
190.097293	922526.349598814\\
191.097293	927201.251514039\\
192.097293	931874.29040299\\
193.097293	936545.466265667\\
194.097293	941214.779102071\\
195.097293	945882.228912201\\
196.097293	950547.815696057\\
197.097293	955211.53945364\\
198.097293	959873.400184948\\
199.097293	964533.397889983\\
200.097293	969191.532568744\\
201.097293	973847.804221231\\
202.097293	978502.212847445\\
203.097293	983154.758447384\\
204.097293	987805.44102105\\
205.097293	992454.260568442\\
206.097293	997101.217089561\\
207.097293	1001746.31058441\\
208.097293	1006389.54105298\\
209.097293	1011030.90849527\\
210.097293	1015670.4129113\\
211.097293	1020308.05430105\\
212.097293	1024943.83266452\\
213.097293	1029577.74800172\\
214.097293	1034209.80031265\\
215.097293	1038839.98959731\\
216.097293	1043468.31585569\\
217.097293	1048094.77908779\\
218.097293	1052719.37929363\\
219.097293	1057342.11647318\\
220.097293	1061962.99062647\\
221.097293	1066582.00175348\\
222.097293	1071199.14985422\\
223.097293	1075814.43492868\\
224.097293	1080427.85697687\\
225.097293	1085039.41599879\\
226.097293	1089649.11199443\\
227.097293	1094256.9449638\\
228.097293	1098862.9149069\\
229.097293	1103467.02182372\\
230.097293	1108069.26571427\\
231.097293	1112669.64657854\\
232.097293	1117268.16441654\\
233.097293	1121864.81922827\\
234.097293	1126459.61101372\\
235.097293	1131052.5397729\\
236.097293	1135643.6055058\\
237.097293	1140232.80821243\\
238.097293	1144820.14789279\\
239.097293	1149405.62454687\\
240.097293	1153989.23817468\\
241.097293	1158570.98877622\\
242.097293	1163150.87635148\\
243.097293	1167728.90090047\\
244.097293	1172305.06242319\\
245.097293	1176879.36091963\\
246.097293	1181451.79638979\\
247.097293	1186022.36883369\\
248.097293	1190591.07825131\\
249.097293	1195157.92464265\\
250.097293	1199722.90800772\\
251.097293	1204286.02834652\\
252.097293	1208847.28565905\\
253.097293	1213406.6799453\\
254.097293	1217964.21120527\\
255.097293	1222519.87943898\\
256.097293	1227073.68464641\\
257.097293	1231625.62682756\\
258.097293	1236175.70598244\\
259.097293	1240723.92211105\\
260.097293	1245270.27521339\\
261.097293	1249814.76528945\\
262.097293	1254357.39233923\\
263.097293	1258898.15636274\\
264.097293	1263437.05735998\\
265.097293	1267974.09533095\\
266.097293	1272509.27027564\\
267.097293	1277042.58219406\\
268.097293	1281574.0310862\\
269.097293	1286103.61695207\\
270.097293	1290631.33979167\\
271.097293	1295157.19960499\\
272.097293	1299681.19639204\\
273.097293	1304203.33015281\\
274.097293	1308723.60088732\\
275.097293	1313242.00859554\\
276.097293	1317758.5532775\\
277.097293	1322273.23493318\\
278.097293	1326786.05356258\\
279.097293	1331297.00916571\\
280.097293	1335806.10174257\\
281.097293	1340313.33129316\\
282.097293	1344818.69781747\\
283.097293	1349322.20131551\\
284.097293	1353823.84178727\\
285.097293	1358323.61923276\\
286.097293	1362821.53365197\\
287.097293	1367317.58504492\\
288.097293	1371811.77341158\\
289.097293	1376304.09875198\\
290.097293	1380794.5610661\\
291.097293	1385283.16035395\\
292.097293	1389769.89661552\\
293.097293	1394254.76985082\\
294.097293	1398737.78005984\\
295.097293	1403218.9272426\\
296.097293	1407698.21139907\\
297.097293	1412175.63252928\\
298.097293	1416651.19063321\\
299.097293	1421124.88571087\\
300.097293	1425596.71776225\\
301.097293	1430066.68678736\\
302.097293	1434534.79278619\\
303.097293	1439001.03575875\\
304.097293	1443465.41570504\\
305.097293	1447927.93262506\\
306.097293	1452388.5865188\\
307.097293	1456847.37738626\\
308.097293	1461304.30522746\\
309.097293	1465759.37004237\\
310.097293	1470212.57183102\\
311.097293	1474663.91059339\\
312.097293	1479113.38632949\\
313.097293	1483560.99903931\\
314.097293	1488006.74872286\\
315.097293	1492450.63538014\\
316.097293	1496892.65901114\\
317.097293	1501332.81961587\\
318.097293	1505771.11719432\\
319.097293	1510207.5517465\\
320.097293	1514642.12327241\\
321.097293	1519074.83177204\\
322.097293	1523505.6772454\\
323.097293	1527934.65969249\\
324.097293	1532361.7791133\\
325.097293	1536787.03550784\\
326.097293	1541210.4288761\\
327.097293	1545631.9592181\\
328.097293	1550051.62653381\\
329.097293	1554469.43082326\\
330.097293	1558885.37208643\\
331.097293	1563299.45032332\\
332.097293	1567711.66553394\\
333.097293	1572122.01771829\\
334.097293	1576530.50687637\\
335.097293	1580937.13300817\\
336.097293	1585341.89611369\\
337.097293	1589744.79619294\\
338.097293	1594145.83324592\\
339.097293	1598545.00727263\\
340.097293	1602942.31827306\\
341.097293	1607337.76624722\\
342.097293	1611731.3511951\\
343.097293	1616123.07311671\\
344.097293	1620512.93201205\\
345.097293	1624900.92788111\\
346.097293	1629287.0607239\\
347.097293	1633671.33054042\\
348.097293	1638053.73733066\\
349.097293	1642434.28109463\\
350.097293	1646812.96183232\\
351.097293	1651189.77954374\\
352.097293	1655564.73422889\\
353.097293	1659937.82588776\\
354.097293	1664309.05452036\\
355.097293	1668678.42012668\\
356.097293	1673045.92270673\\
357.097293	1677411.56226051\\
358.097293	1681775.33878801\\
359.097293	1686137.25228924\\
360.097293	1690497.3027642\\
361.097293	1694855.49021288\\
362.097293	1699211.81463529\\
363.097293	1703566.27603142\\
364.097293	1707918.87440128\\
365.097293	1712269.60974487\\
366.097293	1716618.48206218\\
367.097293	1720965.49135322\\
368.097293	1725310.63761799\\
369.097293	1729653.92085648\\
370.097293	1733995.3410687\\
371.097293	1738334.89825464\\
372.097293	1742672.59241431\\
373.097293	1747008.42354771\\
374.097293	1751342.39165483\\
375.097293	1755674.49673568\\
376.097293	1760004.73879026\\
377.097293	1764333.11781856\\
378.097293	1768659.63382059\\
379.097293	1772984.28679634\\
380.097293	1777307.07674582\\
381.097293	1781628.00366903\\
382.097293	1785947.06756596\\
383.097293	1790264.26843662\\
384.097293	1794579.60628101\\
385.097293	1798893.08109912\\
386.097293	1803204.69289096\\
387.097293	1807514.44165652\\
388.097293	1811822.32739581\\
389.097293	1816128.35010883\\
390.097293	1820432.50979557\\
391.097293	1824734.80645604\\
392.097293	1829035.24009023\\
393.097293	1833333.81069815\\
394.097293	1837630.5182798\\
395.097293	1841925.36283517\\
396.097293	1846218.34436427\\
397.097293	1850509.4628671\\
398.097293	1854798.71834365\\
399.097293	1859086.11079393\\
400.097293	1863371.64021793\\
401.097293	1867655.30661567\\
402.097293	1871937.10998712\\
403.097293	1876217.05033231\\
404.097293	1880495.12765122\\
405.097293	1884771.34194385\\
406.097293	1889045.69321021\\
407.097293	1893318.1814503\\
408.097293	1897588.80666412\\
409.097293	1901857.56885166\\
410.097293	1906124.46801292\\
411.097293	1910389.50414792\\
412.097293	1914652.67725664\\
413.097293	1918913.98733908\\
414.097293	1923173.43439525\\
415.097293	1927431.01842515\\
416.097293	1931686.73942877\\
417.097293	1935940.59740613\\
418.097293	1940192.5923572\\
419.097293	1944442.724282\\
420.097293	1948690.99318053\\
421.097293	1952937.39905279\\
422.097293	1957181.94189877\\
423.097293	1961424.62171848\\
424.097293	1965665.43851191\\
425.097293	1969904.39227907\\
426.097293	1974141.48301996\\
427.097293	1978376.71073457\\
428.097293	1982610.07542291\\
429.097293	1986841.57708497\\
430.097293	1991071.21572077\\
431.097293	1995298.99133028\\
432.097293	1999524.90391353\\
433.097293	2003748.9534705\\
434.097293	2007971.14000119\\
435.097293	2012191.46350562\\
436.097293	2016409.92398376\\
437.097293	2020626.52143564\\
438.097293	2024841.25586124\\
439.097293	2029054.12726057\\
440.097293	2033265.13563362\\
441.097293	2037474.2809804\\
442.097293	2041681.56330091\\
443.097293	2045886.98259514\\
444.097293	2050090.5388631\\
445.097293	2054292.23210478\\
446.097293	2058492.06232019\\
447.097293	2062690.02950933\\
448.097293	2066886.13367219\\
449.097293	2071080.37480878\\
450.097293	2075272.7529191\\
451.097293	2079463.26800314\\
452.097293	2083651.9200609\\
453.097293	2087838.7090924\\
454.097293	2092023.63509762\\
455.097293	2096206.69807657\\
456.097293	2100387.89802924\\
457.097293	2104567.23495564\\
458.097293	2108744.70885576\\
459.097293	2112920.31972962\\
460.097293	2117094.06757719\\
461.097293	2121265.9523985\\
462.097293	2125435.97419353\\
463.097293	2129604.13296228\\
464.097293	2133770.42870477\\
465.097293	2137934.86142098\\
466.097293	2142097.43111091\\
467.097293	2146258.13777457\\
468.097293	2150416.98141196\\
469.097293	2154573.96202307\\
470.097293	2158729.07960791\\
471.097293	2162882.33416648\\
472.097293	2167033.72569877\\
473.097293	2171183.25420479\\
474.097293	2175330.91968453\\
475.097293	2179476.72213801\\
476.097293	2183620.6615652\\
477.097293	2187762.73796613\\
478.097293	2191902.95134078\\
479.097293	2196041.30168915\\
480.097293	2200177.78901125\\
481.097293	2204312.41330708\\
482.097293	2208445.17457664\\
483.097293	2212576.07281992\\
484.097293	2216705.10803692\\
485.097293	2220832.28022766\\
486.097293	2224957.58939212\\
487.097293	2229081.0355303\\
488.097293	2233202.61864221\\
489.097293	2237322.33872785\\
490.097293	2241440.19578722\\
491.097293	2245556.18982031\\
492.097293	2249670.32082712\\
493.097293	2253782.58880767\\
494.097293	2257892.99376194\\
495.097293	2262001.53568993\\
496.097293	2266108.21459165\\
497.097293	2270213.0304671\\
498.097293	2274315.98331628\\
499.097293	2278417.07313918\\
500.097293	2282516.2999358\\
501.097293	2286613.66370615\\
502.097293	2290709.16445023\\
503.097293	2294802.80216804\\
504.097293	2298894.57685957\\
505.097293	2302984.48852483\\
506.097293	2307072.53716381\\
507.097293	2311158.72277652\\
508.097293	2315243.04536296\\
509.097293	2319325.50492312\\
510.097293	2323406.10145701\\
511.097293	2327484.83496462\\
512.097293	2331561.70544596\\
513.097293	2335636.71290103\\
514.097293	2339709.85732982\\
515.097293	2343781.13873235\\
516.097293	2347850.55710859\\
517.097293	2351918.11245856\\
518.097293	2355983.80478226\\
519.097293	2360047.63407969\\
520.097293	2364109.60035084\\
521.097293	2368169.70359571\\
522.097293	2372227.94381432\\
523.097293	2376284.32100665\\
524.097293	2380338.8351727\\
525.097293	2384391.48631248\\
526.097293	2388442.27442599\\
527.097293	2392491.19951323\\
528.097293	2396538.26157419\\
529.097293	2400583.46060887\\
530.097293	2404626.79661729\\
531.097293	2408668.26959943\\
532.097293	2412707.87955529\\
533.097293	2416745.62648488\\
534.097293	2420781.5103882\\
535.097293	2424815.53126525\\
536.097293	2428847.68911602\\
537.097293	2432877.98394051\\
538.097293	2436906.41573874\\
539.097293	2440932.98451068\\
540.097293	2444957.69025636\\
541.097293	2448980.53297576\\
542.097293	2453001.51266889\\
543.097293	2457020.62933574\\
544.097293	2461037.88297632\\
545.097293	2465053.27359063\\
546.097293	2469066.80117866\\
547.097293	2473078.46574042\\
548.097293	2477088.26727591\\
549.097293	2481096.20578512\\
550.097293	2485102.28126805\\
551.097293	2489106.49372472\\
552.097293	2493108.84315511\\
553.097293	2497109.32955922\\
554.097293	2501107.95293707\\
555.097293	2505104.71328863\\
556.097293	2509099.61061393\\
557.097293	2513092.64491295\\
558.097293	2517083.8161857\\
559.097293	2521073.12443217\\
560.097293	2525060.56965237\\
561.097293	2529046.1518463\\
562.097293	2533029.87101395\\
563.097293	2537011.72715533\\
564.097293	2540991.72027043\\
565.097293	2544969.85035926\\
566.097293	2548946.11742182\\
567.097293	2552920.5214581\\
568.097293	2556893.06246811\\
569.097293	2560863.74045184\\
570.097293	2564832.55540931\\
571.097293	2568799.50734049\\
572.097293	2572764.59624541\\
573.097293	2576727.82212405\\
574.097293	2580689.18497642\\
575.097293	2584648.68480251\\
576.097293	2588606.32160233\\
577.097293	2592562.09537587\\
578.097293	2596516.00612314\\
579.097293	2600468.05384414\\
580.097293	2604418.23853887\\
581.097293	2608366.56020732\\
582.097293	2612313.01884949\\
583.097293	2616257.61446539\\
584.097293	2620200.34705502\\
585.097293	2624141.21661838\\
586.097293	2628080.22315546\\
587.097293	2632017.36666627\\
588.097293	2635952.6471508\\
589.097293	2639886.06460906\\
590.097293	2643817.61904105\\
591.097293	2647747.31044676\\
592.097293	2651675.1388262\\
593.097293	2655601.10417936\\
594.097293	2659525.20650625\\
595.097293	2663447.44580687\\
596.097293	2667367.82208121\\
597.097293	2671286.33532928\\
598.097293	2675202.98555108\\
599.097293	2679117.7727466\\
600.097293	2683030.69691585\\
601.097293	2686941.75805883\\
602.097293	2690850.95617553\\
603.097293	2694758.29126595\\
604.097293	2698663.76333011\\
605.097293	2702567.37236799\\
606.097293	2706469.11837959\\
607.097293	2710369.00136492\\
608.097293	2714267.02132398\\
609.097293	2718163.17825677\\
610.097293	2722057.47216328\\
611.097293	2725949.90304351\\
612.097293	2729840.47089747\\
613.097293	2733729.17572516\\
614.097293	2737616.01752658\\
615.097293	2741500.99630172\\
616.097293	2745384.11205059\\
617.097293	2749265.36477318\\
618.097293	2753144.7544695\\
619.097293	2757022.28113955\\
620.097293	2760897.94478332\\
621.097293	2764771.74540082\\
622.097293	2768643.68299205\\
623.097293	2772513.757557\\
624.097293	2776381.96909567\\
625.097293	2780248.31760808\\
626.097293	2784112.80309421\\
627.097293	2787975.42555406\\
628.097293	2791836.18498765\\
629.097293	2795695.08139496\\
630.097293	2799552.11477599\\
631.097293	2803407.28513075\\
632.097293	2807260.59245924\\
633.097293	2811112.03676145\\
634.097293	2814961.61803739\\
635.097293	2818809.33628706\\
636.097293	2822655.19151045\\
637.097293	2826499.18370757\\
638.097293	2830341.31287841\\
639.097293	2834181.57902298\\
640.097293	2838019.98214128\\
641.097293	2841856.5222333\\
642.097293	2845691.19929905\\
643.097293	2849524.01333853\\
644.097293	2853354.96435173\\
645.097293	2857184.05233866\\
646.097293	2861011.27729931\\
647.097293	2864836.63923369\\
648.097293	2868660.1381418\\
649.097293	2872481.77402363\\
650.097293	2876301.54687919\\
651.097293	2880119.45670848\\
652.097293	2883935.50351149\\
653.097293	2887749.68728823\\
654.097293	2891562.00803869\\
655.097293	2895372.46576288\\
656.097293	2899181.0604608\\
657.097293	2902987.79213244\\
658.097293	2906792.66077781\\
659.097293	2910595.66639691\\
660.097293	2914396.80898973\\
661.097293	2918196.08855627\\
662.097293	2921993.50509655\\
663.097293	2925789.05861055\\
664.097293	2929582.74909827\\
665.097293	2933374.57655973\\
666.097293	2937164.54099491\\
667.097293	2940952.64240381\\
668.097293	2944738.88078644\\
669.097293	2948523.2561428\\
670.097293	2952305.76847288\\
671.097293	2956086.41777669\\
672.097293	2959865.20405423\\
673.097293	2963642.12730549\\
674.097293	2967417.18753048\\
675.097293	2971190.38472919\\
676.097293	2974961.71890163\\
677.097293	2978731.1900478\\
678.097293	2982498.79816769\\
679.097293	2986264.54326131\\
680.097293	2990028.42532866\\
681.097293	2993790.44436973\\
682.097293	2997550.60038453\\
683.097293	3001308.89337305\\
684.097293	3005065.32333531\\
685.097293	3008819.89027128\\
686.097293	3012572.59418099\\
687.097293	3016323.43506441\\
688.097293	3020072.41292157\\
689.097293	3023819.52775245\\
690.097293	3027564.77955706\\
691.097293	3031308.16833539\\
692.097293	3035049.69408745\\
693.097293	3038789.35681324\\
694.097293	3042527.15651275\\
695.097293	3046263.09318599\\
696.097293	3049997.16683296\\
697.097293	3053729.37745365\\
698.097293	3057459.72504807\\
699.097293	3061188.20961621\\
700.097293	3064914.83115808\\
701.097293	3068639.58967368\\
702.097293	3072362.485163\\
703.097293	3076083.51762605\\
704.097293	3079802.68706282\\
705.097293	3083519.99347332\\
706.097293	3087235.43685755\\
707.097293	3090949.01721551\\
708.097293	3094660.73454719\\
709.097293	3098370.58885259\\
710.097293	3102078.58013172\\
711.097293	3105784.70838458\\
712.097293	3109488.97361117\\
713.097293	3113191.37581148\\
714.097293	3116891.91498551\\
715.097293	3120590.59113328\\
716.097293	3124287.40425477\\
717.097293	3127982.35434998\\
718.097293	3131675.44141893\\
719.097293	3135366.66546159\\
720.097293	3139056.02647799\\
721.097293	3142743.52446811\\
722.097293	3146429.15943196\\
723.097293	3150112.93136953\\
724.097293	3153794.84028083\\
725.097293	3157474.88616585\\
726.097293	3161153.06902461\\
727.097293	3164829.38885708\\
728.097293	3168503.84566329\\
729.097293	3172176.43944322\\
730.097293	3175847.17019687\\
731.097293	3179516.03792426\\
732.097293	3183183.04262537\\
733.097293	3186848.1843002\\
734.097293	3190511.46294876\\
735.097293	3194172.87857105\\
736.097293	3197832.43116707\\
737.097293	3201490.12073681\\
738.097293	3205145.94728027\\
739.097293	3208799.91079746\\
740.097293	3212452.01128838\\
741.097293	3216102.24875303\\
742.097293	3219750.6231914\\
743.097293	3223397.1346035\\
744.097293	3227041.78298932\\
745.097293	3230684.56834887\\
746.097293	3234325.49068215\\
747.097293	3237964.54998915\\
748.097293	3241601.74626988\\
749.097293	3245237.07952433\\
750.097293	3248870.54975251\\
751.097293	3252502.15695442\\
752.097293	3256131.90113005\\
753.097293	3259759.78227941\\
754.097293	3263385.8004025\\
755.097293	3267009.95549931\\
756.097293	3270632.24756985\\
757.097293	3274252.67661412\\
758.097293	3277871.24263211\\
759.097293	3281487.94562382\\
760.097293	3285102.78558927\\
761.097293	3288715.76252844\\
762.097293	3292326.87644133\\
763.097293	3295936.12732795\\
764.097293	3299543.5151883\\
765.097293	3303149.04002238\\
766.097293	3306752.70183018\\
767.097293	3310354.5006117\\
768.097293	3313954.43636695\\
769.097293	3317552.50909593\\
770.097293	3321148.71879864\\
771.097293	3324743.06547507\\
772.097293	3328335.54912523\\
773.097293	3331926.16974911\\
774.097293	3335514.92734672\\
775.097293	3339101.82191806\\
776.097293	3342686.85346312\\
777.097293	3346270.02198191\\
778.097293	3349851.32747443\\
779.097293	3353430.76994067\\
780.097293	3357008.34938064\\
781.097293	3360584.06579433\\
782.097293	3364157.91918175\\
783.097293	3367729.90954289\\
784.097293	3371300.03687777\\
785.097293	3374868.30118637\\
786.097293	3378434.70246869\\
787.097293	3381999.24072474\\
788.097293	3385561.91595452\\
789.097293	3389122.72815802\\
790.097293	3392681.67733525\\
791.097293	3396238.76348621\\
792.097293	3399793.98661089\\
793.097293	3403347.3467093\\
794.097293	3406898.84378143\\
795.097293	3410448.47782729\\
796.097293	3413996.24884688\\
797.097293	3417542.1568402\\
798.097293	3421086.20180723\\
799.097293	3424628.383748\\
800.097293	3428168.70266249\\
801.097293	3431707.15855071\\
802.097293	3435243.75141265\\
803.097293	3438778.48124833\\
804.097293	3442311.34805772\\
805.097293	3445842.35184085\\
806.097293	3449371.49259769\\
807.097293	3452898.77032827\\
808.097293	3456424.18503257\\
809.097293	3459947.7367106\\
810.097293	3463469.42536235\\
811.097293	3466989.25098783\\
812.097293	3470507.21358704\\
813.097293	3474023.31315997\\
814.097293	3477537.54970663\\
815.097293	3481049.92322702\\
816.097293	3484560.43372113\\
817.097293	3488069.08118897\\
818.097293	3491575.86563053\\
819.097293	3495080.78704582\\
820.097293	3498583.84543484\\
821.097293	3502085.04079758\\
822.097293	3505584.37313405\\
823.097293	3509081.84244424\\
824.097293	3512577.44872816\\
825.097293	3516071.19198581\\
826.097293	3519563.07221718\\
827.097293	3523053.08942228\\
828.097293	3526541.24360111\\
829.097293	3530027.53475366\\
830.097293	3533511.96287994\\
831.097293	3536994.52797995\\
832.097293	3540475.23005368\\
833.097293	3543954.06910113\\
834.097293	3547431.04512232\\
835.097293	3550906.15811723\\
836.097293	3554379.40808586\\
837.097293	3557850.79502822\\
838.097293	3561320.31894431\\
839.097293	3564787.97983413\\
840.097293	3568253.77769767\\
841.097293	3571717.71253493\\
842.097293	3575179.78434593\\
843.097293	3578639.99313065\\
844.097293	3582098.33888909\\
845.097293	3585554.82162126\\
846.097293	3589009.44132716\\
847.097293	3592462.19800679\\
848.097293	3595913.09166014\\
849.097293	3599362.12228721\\
850.097293	3602809.28988802\\
851.097293	3606254.59446254\\
852.097293	3609698.0360108\\
853.097293	3613139.61453278\\
854.097293	3616579.33002849\\
855.097293	3620017.18249792\\
856.097293	3623453.17194108\\
857.097293	3626887.29835797\\
858.097293	3630319.56174858\\
859.097293	3633749.96211292\\
860.097293	3637178.49945099\\
861.097293	3640605.17376278\\
862.097293	3644029.98504829\\
863.097293	3647452.93330754\\
864.097293	3650874.01854051\\
865.097293	3654293.2407472\\
866.097293	3657710.59992763\\
867.097293	3661126.09608177\\
868.097293	3664539.72920965\\
869.097293	3667951.49931125\\
870.097293	3671361.40638658\\
871.097293	3674769.45043563\\
872.097293	3678175.63145841\\
873.097293	3681579.94945492\\
874.097293	3684982.40442515\\
875.097293	3688382.99636911\\
876.097293	3691781.72528679\\
877.097293	3695178.5911782\\
878.097293	3698573.59404334\\
879.097293	3701966.7338822\\
880.097293	3705358.01069479\\
881.097293	3708747.42448111\\
882.097293	3712134.97524115\\
883.097293	3715520.66297492\\
884.097293	3718904.48768241\\
885.097293	3722286.44936363\\
886.097293	3725666.54801858\\
887.097293	3729044.78364725\\
888.097293	3732421.15624965\\
889.097293	3735795.66582578\\
890.097293	3739168.31237563\\
891.097293	3742539.09589921\\
892.097293	3745908.01639651\\
893.097293	3749275.07386754\\
894.097293	3752640.2683123\\
895.097293	3756003.59973078\\
896.097293	3759365.06812299\\
897.097293	3762724.67348892\\
898.097293	3766082.41582858\\
899.097293	3769438.29514197\\
900.097293	3772792.31142909\\
901.097293	3776144.46468992\\
902.097293	3779494.75492449\\
903.097293	3782843.18213278\\
904.097293	3786189.7463148\\
905.097293	3789534.44747055\\
906.097293	3792877.28560002\\
907.097293	3796218.26070322\\
908.097293	3799557.37278014\\
909.097293	3802894.62183079\\
910.097293	3806230.00785516\\
911.097293	3809563.53085327\\
912.097293	3812895.19082509\\
913.097293	3816224.98777065\\
914.097293	3819552.92168993\\
915.097293	3822878.99258294\\
916.097293	3826203.20044967\\
917.097293	3829525.54529013\\
918.097293	3832846.02710432\\
919.097293	3836164.64589223\\
920.097293	3839481.40165387\\
921.097293	3842796.29438923\\
922.097293	3846109.32409832\\
923.097293	3849420.49078114\\
924.097293	3852729.79443768\\
925.097293	3856037.23506795\\
926.097293	3859342.81267194\\
927.097293	3862646.52724967\\
928.097293	3865948.37880111\\
929.097293	3869248.36732629\\
930.097293	3872546.49282519\\
931.097293	3875842.75529782\\
932.097293	3879137.15474417\\
933.097293	3882429.69116425\\
934.097293	3885720.36455805\\
935.097293	3889009.17492558\\
936.097293	3892296.12226684\\
937.097293	3895581.20658182\\
938.097293	3898864.42787053\\
939.097293	3902145.78613297\\
940.097293	3905425.28136913\\
941.097293	3908702.91357902\\
942.097293	3911978.68276264\\
943.097293	3915252.58891998\\
944.097293	3918524.63205105\\
945.097293	3921794.81215584\\
946.097293	3925063.12923436\\
947.097293	3928329.5832866\\
948.097293	3931594.17431258\\
949.097293	3934856.90231228\\
950.097293	3938117.7672857\\
951.097293	3941376.76923285\\
952.097293	3944633.90815373\\
953.097293	3947889.18404833\\
954.097293	3951142.59691666\\
955.097293	3954394.14675872\\
956.097293	3957643.8335745\\
957.097293	3960891.65736401\\
958.097293	3964137.61812724\\
959.097293	3967381.7158642\\
960.097293	3970623.95057489\\
961.097293	3973864.3222593\\
962.097293	3977102.83091744\\
963.097293	3980339.47654931\\
964.097293	3983574.2591549\\
965.097293	3986807.17873421\\
966.097293	3990038.23528726\\
967.097293	3993267.42881403\\
968.097293	3996494.75931453\\
969.097293	3999720.22678875\\
970.097293	4002943.8312367\\
971.097293	4006165.57265837\\
972.097293	4009385.45105377\\
973.097293	4012603.4664229\\
974.097293	4015819.61876576\\
975.097293	4019033.90808234\\
976.097293	4022246.33437264\\
977.097293	4025456.89763668\\
978.097293	4028665.59787443\\
979.097293	4031872.43508592\\
980.097293	4035077.40927113\\
981.097293	4038280.52043007\\
982.097293	4041481.76856273\\
983.097293	4044681.15366912\\
984.097293	4047878.67574924\\
985.097293	4051074.33480308\\
986.097293	4054268.13083065\\
987.097293	4057460.06383194\\
988.097293	4060650.13380696\\
989.097293	4063838.34075571\\
990.097293	4067024.68467818\\
991.097293	4070209.16557438\\
992.097293	4073391.78344431\\
993.097293	4076572.53828796\\
994.097293	4079751.43010534\\
995.097293	4082928.45889644\\
996.097293	4086103.62466127\\
997.097293	4089276.92739983\\
998.097293	4092448.36711211\\
999.097293	4095617.94379812\\
1000.097293	4098785.65745786\\
1001.097293	4101951.50809132\\
1002.097293	4105115.49569851\\
1003.097293	4108277.62027942\\
1004.097293	4111437.88183406\\
1005.097293	4114596.28036243\\
1006.097293	4117752.81586452\\
1007.097293	4120907.48834034\\
1008.097293	4124060.29778989\\
1009.097293	4127211.24421316\\
1010.097293	4130360.32761016\\
1011.097293	4133507.54798088\\
1012.097293	4136652.90532533\\
1013.097293	4139796.39964351\\
1014.097293	4142938.03093541\\
1015.097293	4146077.79920104\\
1016.097293	4149215.70444039\\
1017.097293	4152351.74665347\\
1018.097293	4155485.92584028\\
1019.097293	4158618.24200082\\
1020.097293	4161748.69513508\\
1021.097293	4164877.28524306\\
1022.097293	4168004.01232477\\
1023.097293	4171128.87638021\\
1024.097293	4174251.87740938\\
1025.097293	4177373.01541227\\
1026.097293	4180492.29038889\\
1027.097293	4183609.70233923\\
1028.097293	4186725.2512633\\
1029.097293	4189838.93716109\\
1030.097293	4192950.76003262\\
1031.097293	4196060.71987787\\
1032.097293	4199168.81669684\\
1033.097293	4202275.05048954\\
1034.097293	4205379.42125597\\
1035.097293	4208481.92899612\\
1036.097293	4211582.57371\\
1037.097293	4214681.35539761\\
1038.097293	4217778.27405894\\
1039.097293	4220873.329694\\
1040.097293	4223966.52230278\\
1041.097293	4227057.85188529\\
1042.097293	4230147.31844153\\
1043.097293	4233234.92197149\\
1044.097293	4236320.66247518\\
1045.097293	4239404.53995259\\
1046.097293	4242486.55440374\\
1047.097293	4245566.7058286\\
1048.097293	4248644.9942272\\
1049.097293	4251721.41959952\\
1050.097293	4254795.98194556\\
1051.097293	4257868.68126534\\
1052.097293	4260939.51755884\\
1053.097293	4264008.49082606\\
1054.097293	4267075.60106701\\
1055.097293	4270140.84828169\\
1056.097293	4273204.23247009\\
1057.097293	4276265.75363222\\
1058.097293	4279325.41176808\\
1059.097293	4282383.20687766\\
1060.097293	4285439.13896097\\
1061.097293	4288493.20801801\\
1062.097293	4291545.41404877\\
1063.097293	4294595.75705325\\
1064.097293	4297644.23703147\\
1065.097293	4300690.85398341\\
1066.097293	4303735.60790907\\
1067.097293	4306778.49880847\\
1068.097293	4309819.52668158\\
1069.097293	4312858.69152843\\
1070.097293	4315895.993349\\
1071.097293	4318931.4321433\\
1072.097293	4321965.00791132\\
1073.097293	4324996.72065307\\
1074.097293	4328026.57036855\\
1075.097293	4331054.55705775\\
1076.097293	4334080.68072067\\
1077.097293	4337104.94135733\\
1078.097293	4340127.33896771\\
1079.097293	4343147.87355182\\
1080.097293	4346166.54510965\\
1081.097293	4349183.35364121\\
1082.097293	4352198.29914649\\
1083.097293	4355211.38162551\\
1084.097293	4358222.60107824\\
1085.097293	4361231.95750471\\
1086.097293	4364239.4509049\\
1087.097293	4367245.08127881\\
1088.097293	4370248.84862646\\
1089.097293	4373250.75294783\\
1090.097293	4376250.79424292\\
1091.097293	4379248.97251174\\
1092.097293	4382245.28775429\\
1093.097293	4385239.73997056\\
1094.097293	4388232.32916056\\
1095.097293	4391223.05532429\\
1096.097293	4394211.91846174\\
1097.097293	4397198.91857292\\
1098.097293	4400184.05565783\\
1099.097293	4403167.32971646\\
1100.097293	4406148.74074882\\
1101.097293	4409128.2887549\\
1102.097293	4412105.97373471\\
1103.097293	4415081.79568824\\
1104.097293	4418055.75461551\\
1105.097293	4421027.85051649\\
1106.097293	4423998.08339121\\
1107.097293	4426966.45323965\\
1108.097293	4429932.96006182\\
1109.097293	4432897.60385771\\
1110.097293	4435860.38462733\\
1111.097293	4438821.30237068\\
1112.097293	4441780.35708775\\
1113.097293	4444737.54877855\\
1114.097293	4447692.87744307\\
1115.097293	4450646.34308132\\
1116.097293	4453597.9456933\\
1117.097293	4456547.685279\\
1118.097293	4459495.56183843\\
1119.097293	4462441.57537159\\
1120.097293	4465385.72587847\\
1121.097293	4468328.01335908\\
1122.097293	4471268.43781341\\
1123.097293	4474206.99924147\\
1124.097293	4477143.69764326\\
1125.097293	4480078.53301877\\
1126.097293	4483011.50536801\\
1127.097293	4485942.61469097\\
1128.097293	4488871.86098767\\
1129.097293	4491799.24425808\\
1130.097293	4494724.76450223\\
1131.097293	4497648.4217201\\
1132.097293	4500570.21591169\\
1133.097293	4503490.14707702\\
1134.097293	4506408.21521607\\
1135.097293	4509324.42032884\\
1136.097293	4512238.76241534\\
1137.097293	4515151.24147557\\
1138.097293	4518061.85750952\\
1139.097293	4520970.6105172\\
1140.097293	4523877.50049861\\
1141.097293	4526782.52745374\\
1142.097293	4529685.6913826\\
1143.097293	4532586.99228518\\
1144.097293	4535486.43016149\\
1145.097293	4538384.00501153\\
1146.097293	4541279.7168353\\
1147.097293	4544173.56563278\\
1148.097293	4547065.551404\\
1149.097293	4549955.67414894\\
1150.097293	4552843.93386761\\
1151.097293	4555730.33056001\\
1152.097293	4558614.86422613\\
1153.097293	4561497.53486597\\
1154.097293	4564378.34247955\\
1155.097293	4567257.28706685\\
1156.097293	4570134.36862787\\
1157.097293	4573009.58716262\\
1158.097293	4575882.9426711\\
1159.097293	4578754.43515331\\
1160.097293	4581624.06460924\\
1161.097293	4584491.83103889\\
1162.097293	4587357.73444228\\
1163.097293	4590221.77481938\\
1164.097293	4593083.95217022\\
1165.097293	4595944.26649478\\
1166.097293	4598802.71779307\\
1167.097293	4601659.30606508\\
1168.097293	4604514.03131082\\
1169.097293	4607366.89353029\\
1170.097293	4610217.89272348\\
1171.097293	4613067.0288904\\
1172.097293	4615914.30203105\\
1173.097293	4618759.71214542\\
1174.097293	4621603.25923351\\
1175.097293	4624444.94329534\\
1176.097293	4627284.76433089\\
1177.097293	4630122.72234016\\
1178.097293	4632958.81732317\\
1179.097293	4635793.04927989\\
1180.097293	4638625.41821035\\
1181.097293	4641455.92411453\\
1182.097293	4644284.56699244\\
1183.097293	4647111.34684407\\
1184.097293	4649936.26366943\\
1185.097293	4652759.31746852\\
1186.097293	4655580.50824133\\
1187.097293	4658399.83598787\\
1188.097293	4661217.30070813\\
1189.097293	4664032.90240212\\
1190.097293	4666846.64106984\\
1191.097293	4669658.51671128\\
1192.097293	4672468.52932645\\
1193.097293	4675276.67891535\\
1194.097293	4678082.96547797\\
1195.097293	4680887.38901432\\
1196.097293	4683689.94952439\\
1197.097293	4686490.64700819\\
1198.097293	4689289.48146572\\
1199.097293	4692086.45289697\\
1200.097293	4694881.56130195\\
1201.097293	4697674.80668066\\
1202.097293	4700466.18903309\\
1203.097293	4703255.70835925\\
1204.097293	4706043.36465913\\
1205.097293	4708829.15793274\\
1206.097293	4711613.08818008\\
1207.097293	4714395.15540114\\
1208.097293	4717175.35959593\\
1209.097293	4719953.70076445\\
1210.097293	4722730.17890669\\
1211.097293	4725504.79402265\\
1212.097293	4728277.54611235\\
1213.097293	4731048.43517577\\
1214.097293	4733817.46121291\\
1215.097293	4736584.62422379\\
1216.097293	4739349.92420838\\
1217.097293	4742113.36116671\\
1218.097293	4744874.93509876\\
1219.097293	4747634.64600454\\
1220.097293	4750392.49388404\\
1221.097293	4753148.47873727\\
1222.097293	4755902.60056423\\
1223.097293	4758654.85936491\\
1224.097293	4761405.25513932\\
1225.097293	4764153.78788745\\
1226.097293	4766900.45760931\\
1227.097293	4769645.2643049\\
1228.097293	4772388.20797421\\
1229.097293	4775129.28861725\\
1230.097293	4777868.50623402\\
1231.097293	4780605.86082451\\
1232.097293	4783341.35238873\\
1233.097293	4786074.98092667\\
1234.097293	4788806.74643834\\
1235.097293	4791536.64892374\\
1236.097293	4794264.68838286\\
1237.097293	4796990.86481571\\
1238.097293	4799715.17822229\\
1239.097293	4802437.62860259\\
1240.097293	4805158.21595662\\
1241.097293	4807876.94028437\\
1242.097293	4810593.80158585\\
1243.097293	4813308.79986106\\
1244.097293	4816021.93510999\\
1245.097293	4818733.20733265\\
1246.097293	4821442.61652904\\
1247.097293	4824150.16269915\\
1248.097293	4826855.84584299\\
1249.097293	4829559.66596055\\
1250.097293	4832261.62305184\\
1251.097293	4834961.71711686\\
1252.097293	4837659.9481556\\
1253.097293	4840356.31616807\\
1254.097293	4843050.82115426\\
1255.097293	4845743.46311418\\
1256.097293	4848434.24204783\\
1257.097293	4851123.1579552\\
1258.097293	4853810.2108363\\
1259.097293	4856495.40069113\\
1260.097293	4859178.72751968\\
1261.097293	4861860.19132196\\
1262.097293	4864539.79209797\\
1263.097293	4867217.5298477\\
1264.097293	4869893.40457115\\
1265.097293	4872567.41626834\\
1266.097293	4875239.56493925\\
1267.097293	4877909.85058388\\
1268.097293	4880578.27320224\\
1269.097293	4883244.83279433\\
1270.097293	4885909.52936015\\
1271.097293	4888572.36289969\\
1272.097293	4891233.33341295\\
1273.097293	4893892.44089995\\
1274.097293	4896549.68536067\\
1275.097293	4899205.06679511\\
1276.097293	4901858.58520328\\
1277.097293	4904510.24058518\\
1278.097293	4907160.03294081\\
1279.097293	4909807.96227016\\
1280.097293	4912454.02857323\\
1281.097293	4915098.23185004\\
1282.097293	4917740.57210056\\
1283.097293	4920381.04932482\\
1284.097293	4923019.6635228\\
1285.097293	4925656.41469451\\
1286.097293	4928291.30283994\\
1287.097293	4930924.3279591\\
1288.097293	4933555.49005199\\
1289.097293	4936184.7891186\\
1290.097293	4938812.22515894\\
1291.097293	4941437.79817301\\
1292.097293	4944061.5081608\\
1293.097293	4946683.35512231\\
1294.097293	4949303.33905756\\
1295.097293	4951921.45996653\\
1296.097293	4954537.71784922\\
1297.097293	4957152.11270565\\
1298.097293	4959764.6445358\\
1299.097293	4962375.31333967\\
1300.097293	4964984.11911727\\
1301.097293	4967591.0618686\\
1302.097293	4970196.14159365\\
1303.097293	4972799.35829243\\
1304.097293	4975400.71196494\\
1305.097293	4978000.20261117\\
1306.097293	4980597.83023113\\
1307.097293	4983193.59482481\\
1308.097293	4985787.49639222\\
1309.097293	4988379.53493336\\
1310.097293	4990969.71044822\\
1311.097293	4993558.02293681\\
1312.097293	4996144.47239913\\
1313.097293	4998729.05883517\\
1314.097293	5001311.78224494\\
1315.097293	5003892.64262843\\
1316.097293	5006471.63998565\\
1317.097293	5009048.7743166\\
1318.097293	5011624.04562127\\
1319.097293	5014197.45389967\\
1320.097293	5016768.9991518\\
1321.097293	5019338.68137765\\
1322.097293	5021906.50057723\\
1323.097293	5024472.45675053\\
1324.097293	5027036.54989756\\
1325.097293	5029598.78001832\\
1326.097293	5032159.1471128\\
1327.097293	5034717.65118101\\
1328.097293	5037274.29222294\\
1329.097293	5039829.0702386\\
1330.097293	5042381.98522799\\
1331.097293	5044933.03719111\\
1332.097293	5047482.22612795\\
1333.097293	5050029.55203851\\
1334.097293	5052575.01492281\\
1335.097293	5055118.61478082\\
1336.097293	5057660.35161257\\
1337.097293	5060200.22541804\\
1338.097293	5062738.23619724\\
1339.097293	5065274.38395016\\
1340.097293	5067808.66867681\\
1341.097293	5070341.09037719\\
1342.097293	5072871.64905129\\
1343.097293	5075400.34469912\\
1344.097293	5077927.17732067\\
1345.097293	5080452.14691595\\
1346.097293	5082975.25348496\\
1347.097293	5085496.49702769\\
1348.097293	5088015.87754415\\
1349.097293	5090533.39503434\\
1350.097293	5093049.04949825\\
1351.097293	5095562.84093589\\
1352.097293	5098074.76934725\\
1353.097293	5100584.83473234\\
1354.097293	5103093.03709116\\
1355.097293	5105599.3764237\\
1356.097293	5108103.85272997\\
1357.097293	5110606.46600997\\
1358.097293	5113107.21626369\\
1359.097293	5115606.10349113\\
1360.097293	5118103.12769231\\
1361.097293	5120598.28886721\\
1362.097293	5123091.58701584\\
1363.097293	5125583.02213819\\
1364.097293	5128072.59423427\\
1365.097293	5130560.30330407\\
1366.097293	5133046.1493476\\
1367.097293	5135530.13236486\\
1368.097293	5138012.25235585\\
1369.097293	5140492.50932056\\
1370.097293	5142970.90325899\\
1371.097293	5145447.43417116\\
1372.097293	5147922.10205704\\
1373.097293	5150394.90691666\\
1374.097293	5152865.84875\\
1375.097293	5155334.92755707\\
1376.097293	5157802.14333786\\
1377.097293	5160267.49609238\\
1378.097293	5162730.98582063\\
1379.097293	5165192.6125226\\
1380.097293	5167652.3761983\\
1381.097293	5170110.27684772\\
1382.097293	5172566.31447087\\
1383.097293	5175020.48906775\\
1384.097293	5177472.80063835\\
1385.097293	5179923.24918268\\
1386.097293	5182371.83470074\\
1387.097293	5184818.55719252\\
1388.097293	5187263.41665803\\
1389.097293	5189706.41309726\\
1390.097293	5192147.54651022\\
1391.097293	5194586.81689691\\
1392.097293	5197024.22425732\\
1393.097293	5199459.76859146\\
1394.097293	5201893.44989933\\
1395.097293	5204325.26818092\\
1396.097293	5206755.22343624\\
1397.097293	5209183.31566528\\
1398.097293	5211609.54486805\\
1399.097293	5214033.91104455\\
1400.097293	5216456.41419477\\
1401.097293	5218877.05431872\\
1402.097293	5221295.83141639\\
1403.097293	5223712.7454878\\
1404.097293	5226127.79653292\\
1405.097293	5228540.98455178\\
1406.097293	5230952.30954436\\
1407.097293	5233361.77151067\\
1408.097293	5235769.3704507\\
1409.097293	5238175.10636446\\
1410.097293	5240578.97925194\\
1411.097293	5242980.98911315\\
1412.097293	5245381.13594809\\
1413.097293	5247779.41975675\\
1414.097293	5250175.84053914\\
1415.097293	5252570.39829526\\
1416.097293	5254963.0930251\\
1417.097293	5257353.92472867\\
1418.097293	5259742.89340596\\
1419.097293	5262129.99905699\\
1420.097293	5264515.24168173\\
1421.097293	5266898.62128021\\
1422.097293	5269280.13785241\\
1423.097293	5271659.79139833\\
1424.097293	5274037.58191798\\
1425.097293	5276413.50941136\\
1426.097293	5278787.57387847\\
1427.097293	5281159.7753193\\
1428.097293	5283530.11373385\\
1429.097293	5285898.58912214\\
1430.097293	5288265.20148415\\
1431.097293	5290629.95081988\\
1432.097293	5292992.83712934\\
1433.097293	5295353.86041253\\
1434.097293	5297713.02066945\\
1435.097293	5300070.31790009\\
1436.097293	5302425.75210445\\
1437.097293	5304779.32328255\\
1438.097293	5307131.03143436\\
1439.097293	5309480.87655991\\
1440.097293	5311828.85865918\\
1441.097293	5314174.97773218\\
1442.097293	5316519.2337789\\
1443.097293	5318861.62679935\\
1444.097293	5321202.15679353\\
1445.097293	5323540.82376143\\
1446.097293	5325877.62770306\\
1447.097293	5328212.56861842\\
1448.097293	5330545.6465075\\
1449.097293	5332876.86137031\\
1450.097293	5335206.21320684\\
1451.097293	5337533.7020171\\
1452.097293	5339859.32780109\\
1453.097293	5342183.0905588\\
1454.097293	5344504.99029024\\
1455.097293	5346825.0269954\\
1456.097293	5349143.20067429\\
1457.097293	5351459.51132691\\
1458.097293	5353773.95895325\\
1459.097293	5356086.54355332\\
1460.097293	5358397.26512712\\
1461.097293	5360706.12367464\\
1462.097293	5363013.11919589\\
1463.097293	5365318.25169086\\
1464.097293	5367621.52115956\\
1465.097293	5369922.92760199\\
1466.097293	5372222.47101814\\
1467.097293	5374520.15140802\\
1468.097293	5376815.96877163\\
1469.097293	5379109.92310896\\
1470.097293	5381402.01442002\\
1471.097293	5383692.2427048\\
1472.097293	5385980.60796331\\
1473.097293	5388267.11019555\\
1474.097293	5390551.74940151\\
1475.097293	5392834.5255812\\
1476.097293	5395115.43873462\\
1477.097293	5397394.48886176\\
1478.097293	5399671.67596263\\
1479.097293	5401947.00003722\\
1480.097293	5404220.46108554\\
1481.097293	5406492.05910759\\
1482.097293	5408761.79410336\\
1483.097293	5411029.66607286\\
1484.097293	5413295.67501608\\
1485.097293	5415559.82093304\\
1486.097293	5417822.10382371\\
1487.097293	5420082.52368812\\
1488.097293	5422341.08052625\\
1489.097293	5424597.7743381\\
1490.097293	5426852.60512369\\
1491.097293	5429105.57288299\\
1492.097293	5431356.67761603\\
1493.097293	5433605.91932279\\
1494.097293	5435853.29800328\\
1495.097293	5438098.81365749\\
1496.097293	5440342.46628543\\
1497.097293	5442584.2558871\\
1498.097293	5444824.18246249\\
1499.097293	5447062.24601161\\
1500.097293	5449298.44653445\\
1501.097293	5451532.78403102\\
1502.097293	5453765.25850132\\
1503.097293	5455995.86994534\\
1504.097293	5458224.61836309\\
1505.097293	5460451.50375457\\
1506.097293	5462676.52611977\\
1507.097293	5464899.6854587\\
1508.097293	5467120.98177135\\
1509.097293	5469340.41505773\\
1510.097293	5471557.98531784\\
1511.097293	5473773.69255167\\
1512.097293	5475987.53675923\\
1513.097293	5478199.51794052\\
1514.097293	5480409.63609553\\
1515.097293	5482617.89122427\\
1516.097293	5484824.28332673\\
1517.097293	5487028.81240292\\
1518.097293	5489231.47845284\\
1519.097293	5491432.28147648\\
1520.097293	5493631.22147385\\
1521.097293	5495828.29844495\\
1522.097293	5498023.51238977\\
1523.097293	5500216.86330831\\
1524.097293	5502408.35120059\\
1525.097293	5504597.97606659\\
1526.097293	5506785.73790631\\
1527.097293	5508971.63671977\\
1528.097293	5511155.67250695\\
1529.097293	5513337.84526785\\
1530.097293	5515518.15500248\\
1531.097293	5517696.60171084\\
1532.097293	5519873.18539292\\
1533.097293	5522047.90604873\\
1534.097293	5524220.76367827\\
1535.097293	5526391.75828153\\
1536.097293	5528560.88985852\\
1537.097293	5530728.15840923\\
1538.097293	5532893.56393368\\
1539.097293	5535057.10643184\\
1540.097293	5537218.78590374\\
1541.097293	5539378.60234936\\
1542.097293	5541536.5557687\\
1543.097293	5543692.64616177\\
1544.097293	5545846.87352857\\
1545.097293	5547999.2378691\\
1546.097293	5550149.73918335\\
1547.097293	5552298.37747132\\
1548.097293	5554445.15273303\\
1549.097293	5556590.06496846\\
1550.097293	5558733.11417761\\
1551.097293	5560874.30036049\\
1552.097293	5563013.6235171\\
1553.097293	5565151.08364743\\
1554.097293	5567286.6807515\\
1555.097293	5569420.41482928\\
1556.097293	5571552.28588079\\
1557.097293	5573682.29390603\\
1558.097293	5575810.438905\\
1559.097293	5577936.72087769\\
1560.097293	5580061.13982411\\
1561.097293	5582183.69574425\\
1562.097293	5584304.38863812\\
1563.097293	5586423.21850572\\
1564.097293	5588540.18534704\\
1565.097293	5590655.28916209\\
1566.097293	5592768.52995086\\
1567.097293	5594879.90771337\\
1568.097293	5596989.42244959\\
1569.097293	5599097.07415955\\
1570.097293	5601202.86284323\\
1571.097293	5603306.78850063\\
1572.097293	5605408.85113177\\
1573.097293	5607509.05073662\\
1574.097293	5609607.38731521\\
1575.097293	5611703.86086752\\
1576.097293	5613798.47139356\\
1577.097293	5615891.21889332\\
1578.097293	5617982.10336681\\
1579.097293	5620071.12481403\\
1580.097293	5622158.28323497\\
1581.097293	5624243.57862964\\
1582.097293	5626327.01099803\\
1583.097293	5628408.58034015\\
1584.097293	5630488.286656\\
1585.097293	5632566.12994557\\
1586.097293	5634642.11020887\\
1587.097293	5636716.2274459\\
1588.097293	5638788.48165665\\
1589.097293	5640858.87284113\\
1590.097293	5642927.40099933\\
1591.097293	5644994.06613126\\
1592.097293	5647058.86823692\\
1593.097293	5649121.8073163\\
1594.097293	5651182.88336941\\
1595.097293	5653242.09639625\\
1596.097293	5655299.44639681\\
1597.097293	5657354.93337109\\
1598.097293	5659408.55731911\\
1599.097293	5661460.31824085\\
1600.097293	5663510.21613632\\
1601.097293	5665558.25100551\\
1602.097293	5667604.42284843\\
1603.097293	5669648.73166507\\
1604.097293	5671691.17745544\\
1605.097293	5673731.76021954\\
1606.097293	5675770.47995736\\
1607.097293	5677807.33666891\\
1608.097293	5679842.33035419\\
1609.097293	5681875.46101319\\
1610.097293	5683906.72864592\\
1611.097293	5685936.13325238\\
1612.097293	5687963.67483256\\
1613.097293	5689989.35338646\\
1614.097293	5692013.1689141\\
1615.097293	5694035.12141546\\
1616.097293	5696055.21089054\\
1617.097293	5698073.43733936\\
1618.097293	5700089.8007619\\
1619.097293	5702104.30115816\\
1620.097293	5704116.93852815\\
1621.097293	5706127.71287187\\
1622.097293	5708136.62418931\\
1623.097293	5710143.67248048\\
1624.097293	5712148.85774538\\
1625.097293	5714152.179984\\
1626.097293	5716153.63919635\\
1627.097293	5718153.23538242\\
1628.097293	5720150.96854222\\
1629.097293	5722146.83867575\\
1630.097293	5724140.845783\\
1631.097293	5726132.98986398\\
1632.097293	5728123.27091868\\
1633.097293	5730111.68894712\\
1634.097293	5732098.24394927\\
1635.097293	5734082.93592516\\
1636.097293	5736065.76487477\\
1637.097293	5738046.7307981\\
1638.097293	5740025.83369517\\
1639.097293	5742003.07356596\\
1640.097293	5743978.45041047\\
1641.097293	5745951.96422871\\
1642.097293	5747923.61502068\\
1643.097293	5749893.40278637\\
1644.097293	5751861.32752579\\
1645.097293	5753827.38923894\\
1646.097293	5755791.58792581\\
1647.097293	5757753.92358641\\
1648.097293	5759714.39622074\\
1649.097293	5761673.00582879\\
1650.097293	5763629.75241056\\
1651.097293	5765584.63596607\\
1652.097293	5767537.6564953\\
1653.097293	5769488.81399825\\
1654.097293	5771438.10847494\\
1655.097293	5773385.53992534\\
1656.097293	5775331.10834948\\
1657.097293	5777274.81374734\\
1658.097293	5779216.65611893\\
1659.097293	5781156.63546424\\
1660.097293	5783094.75178328\\
1661.097293	5785031.00507605\\
1662.097293	5786965.39534254\\
1663.097293	5788897.92258276\\
1664.097293	5790828.5867967\\
1665.097293	5792757.38798437\\
1666.097293	5794684.32614577\\
1667.097293	5796609.40128089\\
1668.097293	5798532.61338974\\
1669.097293	5800453.96247232\\
1670.097293	5802373.44852862\\
1671.097293	5804291.07155864\\
1672.097293	5806206.8315624\\
1673.097293	5808120.72853988\\
1674.097293	5810032.76249109\\
1675.097293	5811942.93341602\\
1676.097293	5813851.24131468\\
1677.097293	5815757.68618706\\
1678.097293	5817662.26803317\\
1679.097293	5819564.98685301\\
1680.097293	5821465.84264658\\
1681.097293	5823364.83541387\\
1682.097293	5825261.96515488\\
1683.097293	5827157.23186962\\
1684.097293	5829050.63555809\\
1685.097293	5830942.17622029\\
1686.097293	5832831.85385621\\
1687.097293	5834719.66846586\\
1688.097293	5836605.62004923\\
1689.097293	5838489.70860633\\
1690.097293	5840371.93413716\\
1691.097293	5842252.29664171\\
1692.097293	5844130.79611999\\
1693.097293	5846007.43257199\\
1694.097293	5847882.20599772\\
1695.097293	5849755.11639718\\
1696.097293	5851626.16377036\\
1697.097293	5853495.34811727\\
1698.097293	5855362.66943791\\
1699.097293	5857228.12773227\\
1700.097293	5859091.72300036\\
1701.097293	5860953.45524217\\
1702.097293	5862813.32445772\\
1703.097293	5864671.33064698\\
1704.097293	5866527.47380998\\
1705.097293	5868381.75394669\\
1706.097293	5870234.17105714\\
1707.097293	5872084.72514131\\
1708.097293	5873933.41619921\\
1709.097293	5875780.24423083\\
1710.097293	5877625.20923618\\
1711.097293	5879468.31121526\\
1712.097293	5881309.55016806\\
1713.097293	5883148.92609459\\
1714.097293	5884986.43899485\\
1715.097293	5886822.08886883\\
1716.097293	5888655.87571654\\
1717.097293	5890487.79953797\\
1718.097293	5892317.86033313\\
1719.097293	5894146.05810202\\
1720.097293	5895972.39284463\\
1721.097293	5897796.86456097\\
1722.097293	5899619.47325103\\
1723.097293	5901440.21891483\\
1724.097293	5903259.10155234\\
1725.097293	5905076.12116359\\
1726.097293	5906891.27774856\\
1727.097293	5908704.57130725\\
1728.097293	5910516.00183968\\
1729.097293	5912325.56934582\\
1730.097293	5914133.2738257\\
1731.097293	5915939.1152793\\
1732.097293	5917743.09370663\\
1733.097293	5919545.20910768\\
1734.097293	5921345.46148246\\
1735.097293	5923143.85083096\\
1736.097293	5924940.3771532\\
1737.097293	5926735.04044916\\
1738.097293	5928527.84071884\\
1739.097293	5930318.77796225\\
1740.097293	5932107.85217939\\
1741.097293	5933895.06337025\\
1742.097293	5935680.41153484\\
1743.097293	5937463.89667316\\
1744.097293	5939245.5187852\\
1745.097293	5941025.27787097\\
1746.097293	5942803.17393046\\
1747.097293	5944579.20696368\\
1748.097293	5946353.37697063\\
1749.097293	5948125.6839513\\
1750.097293	5949896.1279057\\
1751.097293	5951664.70883382\\
1752.097293	5953431.42673568\\
1753.097293	5955196.28161125\\
1754.097293	5956959.27346056\\
1755.097293	5958720.40228359\\
1756.097293	5960479.66808034\\
1757.097293	5962237.07085083\\
1758.097293	5963992.61059504\\
1759.097293	5965746.28731297\\
1760.097293	5967498.10100463\\
1761.097293	5969248.05167002\\
1762.097293	5970996.13930913\\
1763.097293	5972742.36392198\\
1764.097293	5974486.72550854\\
1765.097293	5976229.22406883\\
1766.097293	5977969.85960285\\
1767.097293	5979708.6321106\\
1768.097293	5981445.54159207\\
1769.097293	5983180.58804727\\
1770.097293	5984913.77147619\\
1771.097293	5986645.09187884\\
1772.097293	5988374.54925521\\
1773.097293	5990102.14360532\\
1774.097293	5991827.87492914\\
1775.097293	5993551.7432267\\
1776.097293	5995273.74849798\\
1777.097293	5996993.89074299\\
1778.097293	5998712.16996172\\
1779.097293	6000428.58615418\\
1780.097293	6002143.13932037\\
1781.097293	6003855.82946028\\
1782.097293	6005566.65657392\\
1783.097293	6007275.62066128\\
1784.097293	6008982.72172237\\
1785.097293	6010687.95975719\\
1786.097293	6012391.33476573\\
1787.097293	6014092.846748\\
1788.097293	6015792.495704\\
1789.097293	6017490.28163372\\
1790.097293	6019186.20453716\\
1791.097293	6020880.26441434\\
1792.097293	6022572.46126524\\
1793.097293	6024262.79508987\\
1794.097293	6025951.26588822\\
1795.097293	6027637.8736603\\
1796.097293	6029322.6184061\\
1797.097293	6031005.50012563\\
1798.097293	6032686.51881889\\
1799.097293	6034365.67448588\\
1800.097293	6036042.96712659\\
1801.097293	6037718.39674102\\
1802.097293	6039391.96332918\\
1803.097293	6041063.66689107\\
1804.097293	6042733.50742669\\
1805.097293	6044401.48493603\\
1806.097293	6046067.5994191\\
1807.097293	6047731.85087589\\
1808.097293	6049394.23930641\\
1809.097293	6051054.76471065\\
1810.097293	6052713.42708863\\
1811.097293	6054370.22644033\\
1812.097293	6056025.16276575\\
1813.097293	6057678.2360649\\
1814.097293	6059329.44633778\\
1815.097293	6060978.79358438\\
1816.097293	6062626.27780471\\
1817.097293	6064271.89899877\\
1818.097293	6065915.65716655\\
1819.097293	6067557.55230806\\
1820.097293	6069197.58442329\\
1821.097293	6070835.75351225\\
1822.097293	6072472.05957494\\
1823.097293	6074106.50261135\\
1824.097293	6075739.08262149\\
1825.097293	6077369.79960536\\
1826.097293	6078998.65356295\\
1827.097293	6080625.64449427\\
1828.097293	6082250.77239931\\
1829.097293	6083874.03727808\\
1830.097293	6085495.43913058\\
1831.097293	6087114.9779568\\
1832.097293	6088732.65375675\\
1833.097293	6090348.46653042\\
1834.097293	6091962.41627783\\
1835.097293	6093574.50299895\\
1836.097293	6095184.72669381\\
1837.097293	6096793.08736239\\
1838.097293	6098399.58500469\\
1839.097293	6100004.21962073\\
1840.097293	6101606.99121049\\
1841.097293	6103207.89977397\\
1842.097293	6104806.94531118\\
1843.097293	6106404.12782212\\
1844.097293	6107999.44730678\\
1845.097293	6109592.90376517\\
1846.097293	6111184.49719729\\
1847.097293	6112774.22760313\\
1848.097293	6114362.0949827\\
1849.097293	6115948.09933599\\
1850.097293	6117532.24066302\\
1851.097293	6119114.51896376\\
1852.097293	6120694.93423824\\
1853.097293	6122273.48648644\\
1854.097293	6123850.17570836\\
1855.097293	6125425.00190401\\
1856.097293	6126997.96507339\\
1857.097293	6128569.0652165\\
1858.097293	6130138.30233333\\
1859.097293	6131705.67642388\\
1860.097293	6133271.18748817\\
1861.097293	6134834.83552618\\
1862.097293	6136396.62053791\\
1863.097293	6137956.54252337\\
1864.097293	6139514.60148256\\
1865.097293	6141070.79741548\\
1866.097293	6142625.13032212\\
1867.097293	6144177.60020248\\
1868.097293	6145728.20705658\\
1869.097293	6147276.9508844\\
1870.097293	6148823.83168594\\
1871.097293	6150368.84946121\\
1872.097293	6151912.00421021\\
1873.097293	6153453.29593293\\
1874.097293	6154992.72462939\\
1875.097293	6156530.29029956\\
1876.097293	6158065.99294346\\
1877.097293	6159599.83256109\\
1878.097293	6161131.80915245\\
1879.097293	6162661.92271753\\
1880.097293	6164190.17325634\\
1881.097293	6165716.56076887\\
1882.097293	6167241.08525513\\
1883.097293	6168763.74671512\\
1884.097293	6170284.54514883\\
1885.097293	6171803.48055627\\
1886.097293	6173320.55293743\\
1887.097293	6174835.76229232\\
1888.097293	6176349.10862094\\
1889.097293	6177860.59192328\\
1890.097293	6179370.21219935\\
1891.097293	6180877.96944915\\
1892.097293	6182383.86367267\\
1893.097293	6183887.89486992\\
1894.097293	6185390.06304089\\
1895.097293	6186890.3681856\\
1896.097293	6188388.81030402\\
1897.097293	6189885.38939618\\
1898.097293	6191380.10546205\\
1899.097293	6192872.95850166\\
1900.097293	6194363.94851499\\
1901.097293	6195853.07550205\\
1902.097293	6197340.33946284\\
1903.097293	6198825.74039735\\
1904.097293	6200309.27830558\\
1905.097293	6201790.95318754\\
1906.097293	6203270.76504323\\
1907.097293	6204748.71387265\\
1908.097293	6206224.79967579\\
1909.097293	6207699.02245266\\
1910.097293	6209171.38220325\\
1911.097293	6210641.87892757\\
1912.097293	6212110.51262562\\
1913.097293	6213577.28329739\\
1914.097293	6215042.19094289\\
1915.097293	6216505.23556212\\
1916.097293	6217966.41715507\\
1917.097293	6219425.73572175\\
1918.097293	6220883.19126215\\
1919.097293	6222338.78377628\\
1920.097293	6223792.51326413\\
1921.097293	6225244.37972572\\
1922.097293	6226694.38316103\\
1923.097293	6228142.52357006\\
1924.097293	6229588.80095282\\
1925.097293	6231033.21530931\\
1926.097293	6232475.76663952\\
1927.097293	6233916.45494346\\
1928.097293	6235355.28022113\\
1929.097293	6236792.24247252\\
1930.097293	6238227.34169764\\
1931.097293	6239660.57789648\\
1932.097293	6241091.95106906\\
1933.097293	6242521.46121535\\
1934.097293	6243949.10833538\\
1935.097293	6245374.89242912\\
1936.097293	6246798.8134966\\
1937.097293	6248220.8715378\\
1938.097293	6249641.06655273\\
1939.097293	6251059.39854139\\
1940.097293	6252475.86750377\\
1941.097293	6253890.47343987\\
1942.097293	6255303.21634971\\
1943.097293	6256714.09623327\\
1944.097293	6258123.11309055\\
1945.097293	6259530.26692156\\
1946.097293	6260935.5577263\\
1947.097293	6262338.98550476\\
1948.097293	6263740.55025696\\
1949.097293	6265140.25198287\\
1950.097293	6266538.09068251\\
1951.097293	6267934.06635588\\
1952.097293	6269328.17900298\\
1953.097293	6270720.4286238\\
1954.097293	6272110.81521835\\
1955.097293	6273499.33878662\\
1956.097293	6274885.99932862\\
1957.097293	6276270.79684435\\
1958.097293	6277653.7313338\\
1959.097293	6279034.80279698\\
1960.097293	6280414.01123388\\
1961.097293	6281791.35664452\\
1962.097293	6283166.83902887\\
1963.097293	6284540.45838696\\
1964.097293	6285912.21471877\\
1965.097293	6287282.1080243\\
1966.097293	6288650.13830356\\
1967.097293	6290016.30555655\\
1968.097293	6291380.60978327\\
1969.097293	6292743.05098371\\
1970.097293	6294103.62915787\\
1971.097293	6295462.34430577\\
1972.097293	6296819.19642739\\
1973.097293	6298174.18552273\\
1974.097293	6299527.31159181\\
1975.097293	6300878.5746346\\
1976.097293	6302227.97465113\\
1977.097293	6303575.51164138\\
1978.097293	6304921.18560536\\
1979.097293	6306264.99654306\\
1980.097293	6307606.94445449\\
1981.097293	6308947.02933964\\
1982.097293	6310285.25119853\\
1983.097293	6311621.61003113\\
1984.097293	6312956.10583747\\
1985.097293	6314288.73861753\\
1986.097293	6315619.50837131\\
1987.097293	6316948.41509883\\
1988.097293	6318275.45880007\\
1989.097293	6319600.63947503\\
1990.097293	6320923.95712372\\
1991.097293	6322245.41174614\\
1992.097293	6323565.00334228\\
1993.097293	6324882.73191216\\
1994.097293	6326198.59745575\\
1995.097293	6327512.59997308\\
1996.097293	6328824.73946412\\
1997.097293	6330135.0159289\\
1998.097293	6331443.4293674\\
1999.097293	6332749.97977963\\
2000.097293	6334054.66716558\\
2001.097293	6335357.49152526\\
2002.097293	6336658.45285867\\
2003.097293	6337957.5511658\\
2004.097293	6339254.78644666\\
2005.097293	6340550.15870124\\
2006.097293	6341843.66792955\\
2007.097293	6343135.31413159\\
2008.097293	6344425.09730735\\
2009.097293	6345713.01745684\\
2010.097293	6346999.07458006\\
2011.097293	6348283.268677\\
2012.097293	6349565.59974767\\
2013.097293	6350846.06779207\\
2014.097293	6352124.67281018\\
2015.097293	6353401.41480203\\
2016.097293	6354676.2937676\\
2017.097293	6355949.30970691\\
2018.097293	6357220.46261993\\
2019.097293	6358489.75250668\\
2020.097293	6359757.17936716\\
2021.097293	6361022.74320136\\
2022.097293	6362286.44400929\\
2023.097293	6363548.28179095\\
2024.097293	6364808.25654634\\
2025.097293	6366066.36827544\\
2026.097293	6367322.61697828\\
2027.097293	6368577.00265484\\
2028.097293	6369829.52530513\\
2029.097293	6371080.18492914\\
2030.097293	6372328.98152688\\
2031.097293	6373575.91509835\\
2032.097293	6374820.98564354\\
2033.097293	6376064.19316246\\
2034.097293	6377305.53765511\\
2035.097293	6378545.01912148\\
2036.097293	6379782.63756157\\
2037.097293	6381018.3929754\\
2038.097293	6382252.28536295\\
2039.097293	6383484.31472423\\
2040.097293	6384714.48105923\\
2041.097293	6385942.78436796\\
2042.097293	6387169.22465041\\
2043.097293	6388393.80190659\\
2044.097293	6389616.5161365\\
2045.097293	6390837.36734013\\
2046.097293	6392056.35551749\\
2047.097293	6393273.48066858\\
2048.097293	6394488.74279339\\
2049.097293	6395702.14189193\\
2050.097293	6396913.67796419\\
2051.097293	6398123.35101018\\
2052.097293	6399331.1610299\\
2053.097293	6400537.10802334\\
2054.097293	6401741.19199051\\
2055.097293	6402943.41293141\\
2056.097293	6404143.77084603\\
2057.097293	6405342.26573438\\
2058.097293	6406538.89759645\\
2059.097293	6407733.66643225\\
2060.097293	6408926.57224178\\
2061.097293	6410117.61502503\\
2062.097293	6411306.79478201\\
2063.097293	6412494.11151272\\
2064.097293	6413679.56521715\\
2065.097293	6414863.15589531\\
2066.097293	6416044.88354719\\
2067.097293	6417224.7481728\\
2068.097293	6418402.74977214\\
2069.097293	6419578.8883452\\
2070.097293	6420753.16389199\\
2071.097293	6421925.57641251\\
2072.097293	6423096.12590675\\
2073.097293	6424264.81237471\\
2074.097293	6425431.63581641\\
2075.097293	6426596.59623183\\
2076.097293	6427759.69362098\\
2077.097293	6428920.92798385\\
2078.097293	6430080.29932045\\
2079.097293	6431237.80763077\\
2080.097293	6432393.45291482\\
2081.097293	6433547.2351726\\
2082.097293	6434699.1544041\\
2083.097293	6435849.21060933\\
2084.097293	6436997.40378829\\
2085.097293	6438143.73394097\\
2086.097293	6439288.20106738\\
2087.097293	6440430.80516751\\
2088.097293	6441571.54624137\\
2089.097293	6442710.42428896\\
2090.097293	6443847.43931028\\
2091.097293	6444982.59130532\\
2092.097293	6446115.88027408\\
2093.097293	6447247.30621657\\
2094.097293	6448376.86913279\\
2095.097293	6449504.56902274\\
2096.097293	6450630.40588641\\
2097.097293	6451754.3797238\\
2098.097293	6452876.49053493\\
2099.097293	6453996.73831978\\
2100.097293	6455115.12307835\\
2101.097293	6456231.64481065\\
2102.097293	6457346.30351668\\
2103.097293	6458459.09919643\\
2104.097293	6459570.03184992\\
2105.097293	6460679.10147712\\
2106.097293	6461786.30807805\\
2107.097293	6462891.65165271\\
2108.097293	6463995.1322011\\
2109.097293	6465096.74972321\\
2110.097293	6466196.50421905\\
2111.097293	6467294.39568861\\
2112.097293	6468390.4241319\\
2113.097293	6469484.58954892\\
2114.097293	6470576.89193966\\
2115.097293	6471667.33130413\\
2116.097293	6472755.90764232\\
2117.097293	6473842.62095425\\
2118.097293	6474927.47123989\\
2119.097293	6476010.45849927\\
2120.097293	6477091.58273237\\
2121.097293	6478170.84393919\\
2122.097293	6479248.24211974\\
2123.097293	6480323.77727402\\
2124.097293	6481397.44940203\\
2125.097293	6482469.25850376\\
2126.097293	6483539.20457922\\
2127.097293	6484607.2876284\\
2128.097293	6485673.50765131\\
2129.097293	6486737.86464794\\
2130.097293	6487800.35861831\\
2131.097293	6488860.98956239\\
2132.097293	6489919.75748021\\
2133.097293	6490976.66237175\\
2134.097293	6492031.70423702\\
2135.097293	6493084.88307601\\
2136.097293	6494136.19888873\\
2137.097293	6495185.65167518\\
2138.097293	6496233.24143535\\
2139.097293	6497278.96816925\\
2140.097293	6498322.83187687\\
2141.097293	6499364.83255822\\
2142.097293	6500404.9702133\\
2143.097293	6501443.2448421\\
2144.097293	6502479.65644463\\
2145.097293	6503514.20502088\\
2146.097293	6504546.89057086\\
2147.097293	6505577.71309457\\
2148.097293	6506606.67259201\\
2149.097293	6507633.76906317\\
2150.097293	6508659.00250805\\
2151.097293	6509682.37292667\\
2152.097293	6510703.88031901\\
2153.097293	6511723.52468507\\
2154.097293	6512741.30602486\\
2155.097293	6513757.22433838\\
2156.097293	6514771.27962562\\
2157.097293	6515783.47188659\\
2158.097293	6516793.80112129\\
2159.097293	6517802.26732971\\
2160.097293	6518808.87051186\\
2161.097293	6519813.61066774\\
2162.097293	6520816.48779734\\
2163.097293	6521817.50190066\\
2164.097293	6522816.65297772\\
2165.097293	6523813.9410285\\
2166.097293	6524809.366053\\
2167.097293	6525802.92805123\\
2168.097293	6526794.62702319\\
2169.097293	6527784.46296888\\
2170.097293	6528772.43588829\\
2171.097293	6529758.54578143\\
2172.097293	6530742.79264829\\
2173.097293	6531725.17648888\\
2174.097293	6532705.69730319\\
2175.097293	6533684.35509123\\
2176.097293	6534661.149853\\
2177.097293	6535636.0815885\\
2178.097293	6536609.15029772\\
2179.097293	6537580.35598066\\
2180.097293	6538549.69863734\\
2181.097293	6539517.17826774\\
2182.097293	6540482.79487186\\
2183.097293	6541446.54844971\\
2184.097293	6542408.43900129\\
2185.097293	6543368.46652659\\
2186.097293	6544326.63102563\\
2187.097293	6545282.93249838\\
2188.097293	6546237.37094486\\
2189.097293	6547189.94636507\\
2190.097293	6548140.65875901\\
2191.097293	6549089.50812667\\
2192.097293	6550036.49446806\\
2193.097293	6550981.61778317\\
2194.097293	6551924.87807201\\
2195.097293	6552866.27533458\\
2196.097293	6553805.80957087\\
2197.097293	6554743.48078089\\
2198.097293	6555679.28896463\\
2199.097293	6556613.2341221\\
2200.097293	6557545.3162533\\
2201.097293	6558475.53535822\\
2202.097293	6559403.89143688\\
2203.097293	6560330.38448925\\
2204.097293	6561255.01451535\\
2205.097293	6562177.78151518\\
2206.097293	6563098.68548874\\
2207.097293	6564017.72643602\\
2208.097293	6564934.90435702\\
2209.097293	6565850.21925176\\
2210.097293	6566763.67112022\\
2211.097293	6567675.2599624\\
2212.097293	6568584.98577831\\
2213.097293	6569492.84856795\\
2214.097293	6570398.84833132\\
2215.097293	6571302.98506841\\
2216.097293	6572205.25877922\\
2217.097293	6573105.66946377\\
2218.097293	6574004.21712204\\
2219.097293	6574900.90175403\\
2220.097293	6575795.72335975\\
2221.097293	6576688.6819392\\
2222.097293	6577579.77749238\\
2223.097293	6578469.01001928\\
2224.097293	6579356.3795199\\
2225.097293	6580241.88599426\\
2226.097293	6581125.52944234\\
2227.097293	6582007.30986414\\
2228.097293	6582887.22725967\\
2229.097293	6583765.28162893\\
2230.097293	6584641.47297191\\
2231.097293	6585515.80128862\\
2232.097293	6586388.26657906\\
2233.097293	6587258.86884322\\
2234.097293	6588127.60808111\\
2235.097293	6588994.48429273\\
2236.097293	6589859.49747807\\
2237.097293	6590722.64763713\\
2238.097293	6591583.93476993\\
2239.097293	6592443.35887645\\
2240.097293	6593300.91995669\\
2241.097293	6594156.61801067\\
2242.097293	6595010.45303836\\
2243.097293	6595862.42503979\\
2244.097293	6596712.53401494\\
2245.097293	6597560.77996382\\
2246.097293	6598407.16288642\\
2247.097293	6599251.68278275\\
2248.097293	6600094.33965281\\
2249.097293	6600935.13349659\\
2250.097293	6601774.0643141\\
2251.097293	6602611.13210533\\
2252.097293	6603446.33687029\\
2253.097293	6604279.67860898\\
2254.097293	6605111.15732139\\
2255.097293	6605940.77300753\\
2256.097293	6606768.5256674\\
2257.097293	6607594.41530099\\
2258.097293	6608418.44190831\\
2259.097293	6609240.60548935\\
2260.097293	6610060.90604412\\
2261.097293	6610879.34357262\\
2262.097293	6611695.91807484\\
2263.097293	6612510.62955079\\
2264.097293	6613323.47800046\\
2265.097293	6614134.46342387\\
2266.097293	6614943.58582099\\
2267.097293	6615750.84519185\\
2268.097293	6616556.24153643\\
2269.097293	6617359.77485473\\
2270.097293	6618161.44514676\\
2271.097293	6618961.25241253\\
2272.097293	6619759.19665201\\
2273.097293	6620555.27786522\\
2274.097293	6621349.49605216\\
2275.097293	6622141.85121282\\
2276.097293	6622932.34334721\\
2277.097293	6623720.97245533\\
2278.097293	6624507.73853717\\
2279.097293	6625292.64159274\\
2280.097293	6626075.68162203\\
2281.097293	6626856.85862505\\
2282.097293	6627636.1726018\\
2283.097293	6628413.62355227\\
2284.097293	6629189.21147648\\
2285.097293	6629962.9363744\\
2286.097293	6630734.79824605\\
2287.097293	6631504.79709143\\
2288.097293	6632272.93291054\\
2289.097293	6633039.20570337\\
2290.097293	6633803.61546992\\
2291.097293	6634566.16221021\\
2292.097293	6635326.84592422\\
2293.097293	6636085.66661195\\
2294.097293	6636842.62427341\\
2295.097293	6637597.7189086\\
2296.097293	6638350.95051751\\
2297.097293	6639102.31910016\\
2298.097293	6639851.82465652\\
2299.097293	6640599.46718661\\
2300.097293	6641345.24669044\\
2301.097293	6642089.16316798\\
2302.097293	6642831.21661925\\
2303.097293	6643571.40704425\\
2304.097293	6644309.73444297\\
2305.097293	6645046.19881542\\
2306.097293	6645780.8001616\\
2307.097293	6646513.5384815\\
2308.097293	6647244.41377513\\
2309.097293	6647973.42604249\\
2310.097293	6648700.57528357\\
2311.097293	6649425.86149838\\
2312.097293	6650149.28468691\\
2313.097293	6650870.84484917\\
2314.097293	6651590.54198516\\
2315.097293	6652308.37609487\\
2316.097293	6653024.34717831\\
2317.097293	6653738.45523547\\
2318.097293	6654450.70026636\\
2319.097293	6655161.08227098\\
2320.097293	6655869.60124932\\
2321.097293	6656576.25720139\\
2322.097293	6657281.05012719\\
2323.097293	6657983.98002671\\
2324.097293	6658685.04689996\\
2325.097293	6659384.25074693\\
2326.097293	6660081.59156764\\
2327.097293	6660777.06936206\\
2328.097293	6661470.68413022\\
2329.097293	6662162.43587209\\
2330.097293	6662852.3245877\\
2331.097293	6663540.35027703\\
2332.097293	6664226.51294009\\
2333.097293	6664910.81257687\\
2334.097293	6665593.24918738\\
2335.097293	6666273.82277162\\
2336.097293	6666952.53332958\\
2337.097293	6667629.38086127\\
2338.097293	6668304.36536669\\
2339.097293	6668977.48684583\\
2340.097293	6669648.7452987\\
2341.097293	6670318.14072529\\
2342.097293	6670985.67312561\\
2343.097293	6671651.34249966\\
2344.097293	6672315.14884743\\
2345.097293	6672977.09216893\\
2346.097293	6673637.17246416\\
2347.097293	6674295.38973311\\
2348.097293	6674951.74397579\\
2349.097293	6675606.23519219\\
2350.097293	6676258.86338232\\
2351.097293	6676909.62854617\\
2352.097293	6677558.53068376\\
2353.097293	6678205.56979507\\
2354.097293	6678850.7458801\\
2355.097293	6679494.05893886\\
2356.097293	6680135.50897135\\
2357.097293	6680775.09597756\\
2358.097293	6681412.8199575\\
2359.097293	6682048.68091117\\
2360.097293	6682682.67883856\\
2361.097293	6683314.81373968\\
2362.097293	6683945.08561452\\
2363.097293	6684573.4944631\\
2364.097293	6685200.04028539\\
2365.097293	6685824.72308142\\
2366.097293	6686447.54285116\\
2367.097293	6687068.49959464\\
2368.097293	6687687.59331184\\
2369.097293	6688304.82400277\\
2370.097293	6688920.19166743\\
2371.097293	6689533.69630581\\
2372.097293	6690145.33791791\\
2373.097293	6690755.11650375\\
2374.097293	6691363.03206331\\
2375.097293	6691969.08459659\\
2376.097293	6692573.2741036\\
2377.097293	6693175.60058434\\
2378.097293	6693776.0640388\\
2379.097293	6694374.66446699\\
2380.097293	6694971.40186891\\
2381.097293	6695566.27624455\\
2382.097293	6696159.28759392\\
2383.097293	6696750.43591702\\
2384.097293	6697339.72121384\\
2385.097293	6697927.14348439\\
2386.097293	6698512.70272866\\
2387.097293	6699096.39894666\\
2388.097293	6699678.23213839\\
2389.097293	6700258.20230384\\
2390.097293	6700836.30944302\\
2391.097293	6701412.55355592\\
2392.097293	6701986.93464256\\
2393.097293	6702559.45270291\\
2394.097293	6703130.107737\\
2395.097293	6703698.89974481\\
2396.097293	6704265.82872634\\
2397.097293	6704830.8946816\\
2398.097293	6705394.09761059\\
2399.097293	6705955.43751331\\
2400.097293	6706514.91438975\\
2401.097293	6707072.52823992\\
2402.097293	6707628.27906381\\
2403.097293	6708182.16686143\\
2404.097293	6708734.19163278\\
2405.097293	6709284.35337785\\
2406.097293	6709832.65209664\\
2407.097293	6710379.08778917\\
2408.097293	6710923.66045542\\
2409.097293	6711466.3700954\\
2410.097293	6712007.2167091\\
2411.097293	6712546.20029653\\
2412.097293	6713083.32085769\\
2413.097293	6713618.57839257\\
2414.097293	6714151.97290117\\
2415.097293	6714683.50438351\\
2416.097293	6715213.17283957\\
2417.097293	6715740.97826936\\
2418.097293	6716266.92067287\\
2419.097293	6716791.00005011\\
2420.097293	6717313.21640107\\
2421.097293	6717833.56972576\\
2422.097293	6718352.06002418\\
2423.097293	6718868.68729633\\
2424.097293	6719383.4515422\\
2425.097293	6719896.35276179\\
2426.097293	6720407.39095512\\
2427.097293	6720916.56612216\\
2428.097293	6721423.87826294\\
2429.097293	6721929.32737744\\
2430.097293	6722432.91346567\\
2431.097293	6722934.63652762\\
2432.097293	6723434.4965633\\
2433.097293	6723932.49357271\\
2434.097293	6724428.62755584\\
2435.097293	6724922.8985127\\
2436.097293	6725415.30644328\\
2437.097293	6725905.8513476\\
2438.097293	6726394.53322563\\
2439.097293	6726881.3520774\\
2440.097293	6727366.30790288\\
2441.097293	6727849.4007021\\
2442.097293	6728330.63047504\\
2443.097293	6728809.99722171\\
2444.097293	6729287.50094211\\
2445.097293	6729763.14163623\\
2446.097293	6730236.91930407\\
2447.097293	6730708.83394565\\
2448.097293	6731178.88556095\\
2449.097293	6731647.07414997\\
2450.097293	6732113.39971272\\
2451.097293	6732577.8622492\\
2452.097293	6733040.46175941\\
2453.097293	6733501.19824334\\
2454.097293	6733960.07170099\\
2455.097293	6734417.08213238\\
2456.097293	6734872.22953749\\
2457.097293	6735325.51391632\\
2458.097293	6735776.93526888\\
2459.097293	6736226.49359517\\
2460.097293	6736674.18889518\\
2461.097293	6737120.02116892\\
2462.097293	6737563.99041639\\
2463.097293	6738006.09663758\\
2464.097293	6738446.3398325\\
2465.097293	6738884.72000115\\
2466.097293	6739321.23714352\\
2467.097293	6739755.89125962\\
2468.097293	6740188.68234944\\
2469.097293	6740619.61041299\\
2470.097293	6741048.67545027\\
2471.097293	6741475.87746127\\
2472.097293	6741901.216446\\
2473.097293	6742324.69240445\\
2474.097293	6742746.30533663\\
2475.097293	6743166.05524254\\
2476.097293	6743583.94212218\\
2477.097293	6743999.96597553\\
2478.097293	6744414.12680262\\
2479.097293	6744826.42460343\\
2480.097293	6745236.85937797\\
2481.097293	6745645.43112623\\
2482.097293	6746052.13984823\\
2483.097293	6746456.98554394\\
2484.097293	6746859.96821339\\
2485.097293	6747261.08785656\\
2486.097293	6747660.34447345\\
2487.097293	6748057.73806407\\
2488.097293	6748453.26862842\\
2489.097293	6748846.9361665\\
2490.097293	6749238.7406783\\
2491.097293	6749628.68216382\\
2492.097293	6750016.76062308\\
2493.097293	6750402.97605606\\
2494.097293	6750787.32846276\\
2495.097293	6751169.81784319\\
2496.097293	6751550.44419735\\
2497.097293	6751929.20752523\\
2498.097293	6752306.10782685\\
2499.097293	6752681.14510218\\
2500.097293	6753054.31935124\\
2501.097293	6753425.63057403\\
2502.097293	6753795.07877055\\
2503.097293	6754162.66394079\\
2504.097293	6754528.38608476\\
2505.097293	6754892.24520245\\
2506.097293	6755254.24129387\\
2507.097293	6755614.37435902\\
2508.097293	6755972.64439789\\
2509.097293	6756329.05141049\\
2510.097293	6756683.59539681\\
2511.097293	6757036.27635687\\
2512.097293	6757387.09429064\\
2513.097293	6757736.04919815\\
2514.097293	6758083.14107938\\
2515.097293	6758428.36993433\\
2516.097293	6758771.73576301\\
2517.097293	6759113.23856542\\
2518.097293	6759452.87834156\\
2519.097293	6759790.65509142\\
2520.097293	6760126.568815\\
2521.097293	6760460.61951232\\
2522.097293	6760792.80718336\\
2523.097293	6761123.13182812\\
2524.097293	6761451.59344662\\
2525.097293	6761778.19203883\\
2526.097293	6762102.92760478\\
2527.097293	6762425.80014445\\
2528.097293	6762746.80965785\\
2529.097293	6763065.95614497\\
2530.097293	6763383.23960582\\
2531.097293	6763698.66004039\\
2532.097293	6764012.2174487\\
2533.097293	6764323.91183072\\
2534.097293	6764633.74318648\\
2535.097293	6764941.71151596\\
2536.097293	6765247.81681916\\
2537.097293	6765552.0590961\\
2538.097293	6765854.43834676\\
2539.097293	6766154.95457114\\
2540.097293	6766453.60776925\\
2541.097293	6766750.39794109\\
2542.097293	6767045.32508665\\
2543.097293	6767338.38920594\\
2544.097293	6767629.59029896\\
2545.097293	6767918.9283657\\
2546.097293	6768206.40340617\\
2547.097293	6768492.01542037\\
2548.097293	6768775.76440829\\
2549.097293	6769057.65036994\\
2550.097293	6769337.67330531\\
2551.097293	6769615.83321441\\
2552.097293	6769892.13009724\\
2553.097293	6770166.56395379\\
2554.097293	6770439.13478407\\
2555.097293	6770709.84258807\\
2556.097293	6770978.6873658\\
2557.097293	6771245.66911726\\
2558.097293	6771510.78784244\\
2559.097293	6771774.04354135\\
2560.097293	6772035.43621399\\
2561.097293	6772294.96586035\\
2562.097293	6772552.63248044\\
2563.097293	6772808.43607426\\
2564.097293	6773062.37664179\\
2565.097293	6773314.45418306\\
2566.097293	6773564.66869805\\
2567.097293	6773813.02018677\\
2568.097293	6774059.50864922\\
2569.097293	6774304.13408539\\
2570.097293	6774546.89649529\\
2571.097293	6774787.79587891\\
2572.097293	6775026.83223626\\
2573.097293	6775264.00556734\\
2574.097293	6775499.31587214\\
2575.097293	6775732.76315067\\
2576.097293	6775964.34740293\\
2577.097293	6776194.06862891\\
2578.097293	6776421.92682862\\
2579.097293	6776647.92200205\\
2580.097293	6776872.05414921\\
2581.097293	6777094.3232701\\
2582.097293	6777314.72936471\\
2583.097293	6777533.27243305\\
2584.097293	6777749.95247511\\
2585.097293	6777964.76949091\\
2586.097293	6778177.72348042\\
2587.097293	6778388.81444367\\
2588.097293	6778598.04238064\\
2589.097293	6778805.40729133\\
2590.097293	6779010.90917575\\
2591.097293	6779214.5480339\\
2592.097293	6779416.32386578\\
2593.097293	6779616.23667138\\
2594.097293	6779814.28645071\\
2595.097293	6780010.47320376\\
2596.097293	6780204.79693054\\
2597.097293	6780397.25763105\\
2598.097293	6780587.85530528\\
2599.097293	6780776.58995324\\
2600.097293	6780963.46157492\\
2601.097293	6781148.47017033\\
2602.097293	6781331.61573947\\
2603.097293	6781512.89828233\\
2604.097293	6781692.31779892\\
2605.097293	6781869.87428924\\
2606.097293	6782045.56775328\\
2607.097293	6782219.39819105\\
2608.097293	6782391.36560254\\
2609.097293	6782561.46998776\\
2610.097293	6782729.71134671\\
2611.097293	6782896.08967938\\
2612.097293	6783060.60498578\\
2613.097293	6783223.25726591\\
2614.097293	6783384.04651976\\
2615.097293	6783542.97274734\\
2616.097293	6783700.03594864\\
2617.097293	6783855.23612367\\
2618.097293	6784008.57327243\\
2619.097293	6784160.04739491\\
2620.097293	6784309.65849112\\
2621.097293	6784457.40656105\\
2622.097293	6784603.29160471\\
2623.097293	6784747.3136221\\
2624.097293	6784889.47261322\\
2625.097293	6785029.76857806\\
2626.097293	6785168.20151662\\
2627.097293	6785304.77142891\\
2628.097293	6785439.47831493\\
2629.097293	6785572.32217468\\
2630.097293	6785703.30300815\\
2631.097293	6785832.42081535\\
2632.097293	6785959.67559627\\
2633.097293	6786085.06735092\\
2634.097293	6786208.5960793\\
2635.097293	6786330.2617814\\
2636.097293	6786450.06445723\\
2637.097293	6786568.00410678\\
2638.097293	6786684.08073006\\
2639.097293	6786798.29432707\\
2640.097293	6786910.6448978\\
2641.097293	6787021.13244226\\
2642.097293	6787129.75696045\\
2643.097293	6787236.51845236\\
2644.097293	6787341.416918\\
2645.097293	6787444.45235736\\
2646.097293	6787545.62477045\\
2647.097293	6787644.93415727\\
2648.097293	6787742.38051781\\
2649.097293	6787837.96385208\\
2650.097293	6787931.68416008\\
2651.097293	6788023.5414418\\
2652.097293	6788113.53569725\\
2653.097293	6788201.66692642\\
2654.097293	6788287.93512932\\
2655.097293	6788372.34030595\\
2656.097293	6788454.8824563\\
2657.097293	6788535.56158038\\
2658.097293	6788614.37767819\\
2659.097293	6788691.33074972\\
2660.097293	6788766.42079497\\
2661.097293	6788839.64781396\\
2662.097293	6788911.01180667\\
2663.097293	6788980.51277311\\
2664.097293	6789048.15071327\\
2665.097293	6789113.92562716\\
2666.097293	6789177.83751477\\
2667.097293	6789239.88637611\\
2668.097293	6789300.07221118\\
2669.097293	6789358.39501997\\
2670.097293	6789414.85480249\\
2671.097293	6789469.45155874\\
2672.097293	6789522.18528871\\
2673.097293	6789573.05599241\\
2674.097293	6789622.06366984\\
2675.097293	6789669.20832099\\
2676.097293	6789714.48994586\\
2677.097293	6789757.90854447\\
2678.097293	6789799.4641168\\
2679.097293	6789839.15666285\\
2680.097293	6789876.98618263\\
2681.097293	6789912.95267614\\
2682.097293	6789947.05614338\\
2683.097293	6789979.29658434\\
2684.097293	6790009.67399902\\
2685.097293	6790038.18838744\\
2686.097293	6790064.83974958\\
2687.097293	6790089.62808544\\
2688.097293	6790112.55339503\\
2689.097293	6790133.61567835\\
2690.097293	6790152.81493539\\
2691.097293	6790170.15116617\\
2692.097293	6790185.62437066\\
2693.097293	6790199.23454889\\
2694.097293	6790210.98170083\\
2695.097293	6790220.86582651\\
2696.097293	6790228.88692591\\
2697.097293	6790235.04499904\\
2698.097293	6790239.34004589\\
2699.097293	6790241.77206647\\
};
\addplot [color=mycolor2,solid,line width=2.0pt,forget plot]
  table[row sep=crcr]{%
2699.902707	6790242.37633132\\
3199.902707	6790242.37633132\\
};
\end{axis}
\end{tikzpicture}%
    \caption{Utility of reserves}
    \label{fig:ReserveUtility}
\end{figure}
\end{minipage}\\


Once again, the previous assumptions made still hold, but the hard constraint is replaced in favor of the operating reserve demand curve defined. The optimization model can be described as follows : \\

\begin{center}
\boxput*(0,1){\colorbox{white}{\textbf{ Operating Reserve Demand Curve [ORDC] }}}{
\setlength{\fboxsep}{10pt}
\fbox{\begin{minipage}{0.9\textwidth} \vspace{0.2cm}
\begin{align*}
 \max\limits_{r_g, p_g, r_1,r_2,r_3 \geq 0, p_{exch}} \quad  & r_1\: L + (a\: r_2 +2L)\: \frac{r_2}{2} - \sum_{g \in \G} \int_0^{p_g} MC_g(x) dx - 31.2424 \: \frac{p_{exch}^2}{2} - 0.005\: p_{exch} \\
\end{align*}
$$
\begin{array}{llll}
(\lambda)				& \sum_{g \in \G} p_g + p_{exch} + p^{t}_{r} \geq \D^{t} & & (1) \\
(\mu)					& r_1 + r_2 + r_3 = \sum_{g \in \G} r_g & & (2) \\
							& r_g \leq 15\: R^{t}_g & & (3) \\
							& p_g + r_g \leq P^{t}_g & & (4) \\
							& -ATC_1^{t} \leq p_{exch} \leq ATC_2^{t} & & (5) \\
							& r_1 \leq 16 & & (6) \\
							& r_2 \leq 2700 & & (7) \\
\end{array}
$$
\vspace{0.1cm}
\end{minipage}}}
\end{center}

We want to maximize the welfare due to the availability of reserves against the production cost. The parameter $L$ is the value of lost load, which is evaluated to $5000$ \euro /MW. And the variable $r_1$, $r_2$ and $r_3$ are the reserve quantities made in each part of the utility function (linear, quadratic and constant).

\newpage
\section{Questions}

In this section, we will analyze the three previously described models on data coming from the Belgian market (2014). But before answering all the questions about our models, let's take a look at the data. 

\subsection{The Belgian market}

In 2014, a combination of circumstances has led to the closure of a number of power stations. We saw that the country was facing a monumental challenge in terms of security of supply. A few months before winter started, Belgium unexpectedly and very suddenly lost a major portion of its generation capacity (figure [\ref{fig:capa}]). Nuclear reactors Doel 3 and Tihange 2 were shut down because hairline cracks were found in the reactor vessels, and these must be repaired. The Doel 4 reactor also had to be shut down due to a technical incident. The Doel 4 reactor was recommissioned just before Christmas. However, 20\% of Belgian generation capacity still remained offline\footnote{source : http://www.elia.be/en/about-elia/questions-about-the-risk-of-shortage-in-Belgium}. The figure [\ref{fig:ratio}] shows the ratio between the Belgian production capacity over the real demand level (demand level minus the power production coming from renewable energies). We see that the curve goes down in winter and even below one, meaning that the Belgian production wasn't able to satisfy its own demand level. Therefore, Belgium was structurally dependent on imports throughout the winter, which could lead to higher energy prices. It will be interesting to observe how the different models behave in those circumstances.

\begin{figure}[H]
    \centering
    \setlength\fheight{4cm}
    \setlength\fwidth{0.8\textwidth}
    % This file was created by matlab2tikz.
% Minimal pgfplots version: 1.3
%
%The latest updates can be retrieved from
%  http://www.mathworks.com/matlabcentral/fileexchange/22022-matlab2tikz
%where you can also make suggestions and rate matlab2tikz.
%
\definecolor{mycolor1}{rgb}{0.04314,0.51765,0.78039}%
%
\begin{tikzpicture}

\begin{axis}[%
width=\fwidth,
height=\fheight,
at={(0\fwidth,0\fheight)},
scale only axis,
separate axis lines,
every outer x axis line/.append style={black},
every x tick label/.append style={font=\color{black}},
xmin=0,
xmax=8760,
xlabel={time [hour]},
xmajorgrids,
every outer y axis line/.append style={black},
every y tick label/.append style={font=\color{black}},
ymin=9000,
ymax=15500,
ylabel={Maximal Production [MWh]},
ymajorgrids
]
\addplot [color=mycolor1,line width=1.3pt, solid,forget plot]
  table[row sep=crcr]{%
1	14987.1\\
2	14987.1\\
3	14987.1\\
4	14987.1\\
5	14987.1\\
6	14987.1\\
7	14987.1\\
8	14987.1\\
9	14987.1\\
10	14987.1\\
11	14987.1\\
12	14987.1\\
13	14987.1\\
14	14987.1\\
15	14987.1\\
16	14987.1\\
17	14987.1\\
18	14987.1\\
19	14987.1\\
20	14987.1\\
21	14987.1\\
22	14987.1\\
23	14987.1\\
24	14987.1\\
25	14987.1\\
26	14987.1\\
27	14987.1\\
28	14987.1\\
29	14987.1\\
30	14987.1\\
31	14987.1\\
32	14987.1\\
33	14987.1\\
34	14987.1\\
35	14987.1\\
36	14987.1\\
37	14987.1\\
38	14987.1\\
39	14987.1\\
40	14987.1\\
41	14987.1\\
42	14987.1\\
43	14987.1\\
44	14987.1\\
45	14987.1\\
46	14987.1\\
47	14987.1\\
48	14987.1\\
49	14987.1\\
50	14987.1\\
51	14987.1\\
52	14987.1\\
53	14987.1\\
54	14987.1\\
55	14987.1\\
56	14987.1\\
57	14987.1\\
58	14987.1\\
59	14987.1\\
60	14987.1\\
61	14987.1\\
62	14987.1\\
63	14987.1\\
64	14987.1\\
65	14987.1\\
66	14987.1\\
67	14987.1\\
68	14987.1\\
69	14987.1\\
70	14772.1\\
71	14772.1\\
72	14772.1\\
73	14772.1\\
74	14339.6\\
75	14339.6\\
76	14339.6\\
77	14339.6\\
78	14339.6\\
79	14339.6\\
80	14339.6\\
81	14339.6\\
82	14339.6\\
83	14339.6\\
84	14339.6\\
85	14339.6\\
86	14339.6\\
87	14339.6\\
88	14339.6\\
89	14339.6\\
90	14339.6\\
91	14339.6\\
92	14339.6\\
93	14339.6\\
94	14339.6\\
95	14339.6\\
96	14339.6\\
97	14339.6\\
98	14339.6\\
99	14339.6\\
100	14339.6\\
101	14339.6\\
102	14339.6\\
103	14339.6\\
104	14339.6\\
105	14339.6\\
106	14339.6\\
107	14339.6\\
108	14339.6\\
109	14339.6\\
110	14339.6\\
111	14339.6\\
112	14339.6\\
113	14339.6\\
114	14339.6\\
115	14339.6\\
116	14339.6\\
117	14339.6\\
118	14554.6\\
119	14554.6\\
120	14376.6\\
121	14376.6\\
122	14376.6\\
123	14376.6\\
124	14376.6\\
125	14376.6\\
126	14376.6\\
127	14376.6\\
128	14376.6\\
129	14376.6\\
130	14376.6\\
131	14376.6\\
132	14376.6\\
133	14376.6\\
134	14376.6\\
135	14376.6\\
136	14376.6\\
137	14376.6\\
138	14376.6\\
139	14376.6\\
140	14554.6\\
141	14554.6\\
142	14554.6\\
143	14554.6\\
144	14554.6\\
145	14554.6\\
146	14554.6\\
147	14554.6\\
148	14554.6\\
149	14554.6\\
150	14554.6\\
151	14554.6\\
152	14554.6\\
153	14144.6\\
154	14144.6\\
155	14149.6\\
156	14149.6\\
157	14149.6\\
158	14149.6\\
159	14149.6\\
160	13799.6\\
161	13799.6\\
162	13799.6\\
163	13799.6\\
164	13799.6\\
165	13799.6\\
166	14204.6\\
167	14204.6\\
168	13424.6\\
169	13424.6\\
170	13424.6\\
171	13799.6\\
172	13799.6\\
173	13799.6\\
174	13799.6\\
175	13799.6\\
176	13799.6\\
177	13799.6\\
178	13799.6\\
179	13799.6\\
180	13799.6\\
181	13799.6\\
182	12791.6\\
183	12791.6\\
184	12791.6\\
185	12791.6\\
186	12791.6\\
187	12791.6\\
188	12791.6\\
189	12641.6\\
190	12641.6\\
191	12641.6\\
192	12641.6\\
193	13649.6\\
194	13649.6\\
195	13649.6\\
196	13649.6\\
197	13649.6\\
198	13649.6\\
199	13649.6\\
200	13649.6\\
201	13649.6\\
202	13649.6\\
203	13649.6\\
204	13649.6\\
205	13649.6\\
206	13649.6\\
207	13649.6\\
208	13649.6\\
209	13799.6\\
210	13799.6\\
211	13799.6\\
212	13799.6\\
213	13799.6\\
214	13799.6\\
215	13799.6\\
216	13799.6\\
217	13799.6\\
218	13799.6\\
219	13799.6\\
220	13799.6\\
221	13799.6\\
222	13799.6\\
223	13799.6\\
224	13799.6\\
225	13799.6\\
226	13799.6\\
227	13799.6\\
228	13799.6\\
229	13799.6\\
230	13799.6\\
231	13799.6\\
232	13799.6\\
233	13799.6\\
234	13799.6\\
235	13799.6\\
236	13799.6\\
237	13799.6\\
238	13799.6\\
239	13799.6\\
240	13799.6\\
241	13799.6\\
242	13799.6\\
243	13799.6\\
244	13799.6\\
245	13799.6\\
246	13799.6\\
247	13799.6\\
248	13799.6\\
249	13799.6\\
250	13799.6\\
251	13799.6\\
252	13799.6\\
253	13799.6\\
254	13799.6\\
255	13799.6\\
256	13799.6\\
257	13799.6\\
258	13799.6\\
259	13799.6\\
260	13799.6\\
261	13799.6\\
262	13799.6\\
263	13799.6\\
264	13584.6\\
265	13584.6\\
266	13584.6\\
267	13584.6\\
268	13584.6\\
269	13584.6\\
270	13584.6\\
271	13584.6\\
272	13584.6\\
273	13584.6\\
274	13584.6\\
275	13584.6\\
276	13584.6\\
277	13584.6\\
278	13584.6\\
279	13934.6\\
280	13934.6\\
281	13934.6\\
282	13934.6\\
283	13694.6\\
284	13694.6\\
285	13694.6\\
286	13694.6\\
287	13694.6\\
288	13934.6\\
289	14339.6\\
290	14339.6\\
291	14339.6\\
292	14339.6\\
293	14339.6\\
294	14339.6\\
295	14339.6\\
296	14339.6\\
297	14554.6\\
298	14554.6\\
299	14554.6\\
300	14554.6\\
301	14554.6\\
302	14554.6\\
303	14554.6\\
304	14554.6\\
305	14554.6\\
306	14554.6\\
307	14554.6\\
308	14554.6\\
309	14554.6\\
310	14554.6\\
311	14554.6\\
312	14554.6\\
313	14554.6\\
314	14554.6\\
315	14554.6\\
316	14554.6\\
317	14554.6\\
318	14554.6\\
319	14554.6\\
320	14554.6\\
321	14554.6\\
322	14554.6\\
323	14554.6\\
324	14554.6\\
325	14554.6\\
326	14554.6\\
327	14554.6\\
328	14554.6\\
329	14554.6\\
330	14554.6\\
331	14554.6\\
332	14554.6\\
333	14554.6\\
334	14554.6\\
335	14554.6\\
336	14554.6\\
337	14554.6\\
338	14554.6\\
339	14554.6\\
340	14554.6\\
341	14554.6\\
342	14554.6\\
343	14554.6\\
344	14554.6\\
345	14554.6\\
346	14554.6\\
347	14554.6\\
348	14554.6\\
349	14554.6\\
350	14554.6\\
351	14554.6\\
352	14554.6\\
353	14554.6\\
354	14554.6\\
355	14554.6\\
356	14554.6\\
357	14554.6\\
358	14554.6\\
359	14554.6\\
360	14554.6\\
361	14554.6\\
362	14554.6\\
363	14554.6\\
364	14554.6\\
365	14554.6\\
366	14554.6\\
367	14554.6\\
368	14554.6\\
369	14554.6\\
370	14554.6\\
371	14554.6\\
372	14554.6\\
373	14554.6\\
374	14554.6\\
375	14554.6\\
376	14554.6\\
377	14554.6\\
378	14554.6\\
379	14554.6\\
380	14554.6\\
381	14554.6\\
382	14554.6\\
383	14554.6\\
384	14554.6\\
385	14554.6\\
386	14554.6\\
387	14554.6\\
388	14554.6\\
389	14554.6\\
390	14554.6\\
391	14554.6\\
392	14554.6\\
393	14554.6\\
394	14554.6\\
395	14554.6\\
396	14554.6\\
397	14554.6\\
398	14554.6\\
399	14554.6\\
400	14554.6\\
401	14554.6\\
402	14554.6\\
403	14554.6\\
404	14554.6\\
405	14554.6\\
406	14554.6\\
407	14554.6\\
408	14554.6\\
409	14554.6\\
410	14554.6\\
411	14554.6\\
412	14554.6\\
413	14554.6\\
414	14554.6\\
415	14554.6\\
416	14554.6\\
417	14554.6\\
418	14554.6\\
419	14554.6\\
420	14554.6\\
421	14554.6\\
422	14554.6\\
423	14554.6\\
424	14554.6\\
425	14554.6\\
426	14554.6\\
427	14554.6\\
428	14554.6\\
429	14554.6\\
430	14554.6\\
431	14554.6\\
432	14554.6\\
433	14554.6\\
434	14554.6\\
435	14554.6\\
436	14554.6\\
437	14554.6\\
438	14554.6\\
439	14554.6\\
440	14554.6\\
441	14554.6\\
442	14554.6\\
443	14554.6\\
444	14554.6\\
445	14554.6\\
446	14554.6\\
447	14554.6\\
448	14554.6\\
449	14554.6\\
450	14554.6\\
451	14554.6\\
452	14554.6\\
453	14554.6\\
454	14554.6\\
455	14554.6\\
456	14554.6\\
457	14554.6\\
458	14554.6\\
459	14554.6\\
460	14554.6\\
461	14554.6\\
462	14554.6\\
463	14554.6\\
464	14554.6\\
465	14554.6\\
466	14554.6\\
467	14554.6\\
468	14554.6\\
469	14554.6\\
470	14554.6\\
471	14554.6\\
472	14554.6\\
473	14554.6\\
474	14554.6\\
475	14554.6\\
476	14554.6\\
477	14554.6\\
478	14987.1\\
479	14987.1\\
480	14987.1\\
481	14987.1\\
482	14987.1\\
483	14987.1\\
484	14987.1\\
485	14987.1\\
486	14987.1\\
487	14987.1\\
488	14987.1\\
489	14987.1\\
490	14987.1\\
491	14987.1\\
492	14987.1\\
493	14987.1\\
494	14987.1\\
495	14987.1\\
496	14987.1\\
497	14987.1\\
498	14987.1\\
499	14987.1\\
500	14987.1\\
501	14987.1\\
502	14987.1\\
503	14987.1\\
504	14987.1\\
505	14987.1\\
506	14987.1\\
507	14987.1\\
508	14987.1\\
509	14987.1\\
510	14987.1\\
511	14987.1\\
512	14987.1\\
513	14987.1\\
514	14987.1\\
515	14987.1\\
516	14987.1\\
517	14987.1\\
518	14987.1\\
519	14987.1\\
520	14987.1\\
521	14987.1\\
522	14987.1\\
523	14987.1\\
524	14987.1\\
525	14987.1\\
526	14987.1\\
527	14987.1\\
528	14987.1\\
529	14987.1\\
530	14987.1\\
531	14987.1\\
532	14987.1\\
533	14987.1\\
534	14987.1\\
535	14987.1\\
536	14987.1\\
537	14987.1\\
538	14987.1\\
539	14987.1\\
540	14987.1\\
541	14987.1\\
542	14987.1\\
543	14987.1\\
544	14987.1\\
545	14987.1\\
546	14987.1\\
547	14987.1\\
548	14987.1\\
549	14987.1\\
550	14987.1\\
551	14987.1\\
552	14987.1\\
553	14987.1\\
554	14987.1\\
555	14987.1\\
556	14987.1\\
557	14987.1\\
558	14987.1\\
559	14987.1\\
560	14987.1\\
561	14987.1\\
562	14987.1\\
563	14987.1\\
564	14987.1\\
565	14987.1\\
566	14987.1\\
567	14987.1\\
568	14987.1\\
569	14987.1\\
570	14987.1\\
571	14987.1\\
572	14987.1\\
573	14987.1\\
574	14987.1\\
575	14987.1\\
576	14987.1\\
577	14987.1\\
578	14987.1\\
579	14987.1\\
580	14987.1\\
581	14987.1\\
582	14987.1\\
583	14987.1\\
584	14987.1\\
585	14987.1\\
586	14987.1\\
587	14987.1\\
588	14987.1\\
589	14987.1\\
590	14987.1\\
591	14987.1\\
592	14987.1\\
593	14987.1\\
594	14987.1\\
595	14987.1\\
596	14987.1\\
597	14987.1\\
598	14987.1\\
599	14987.1\\
600	14987.1\\
601	14987.1\\
602	14987.1\\
603	14987.1\\
604	14987.1\\
605	14987.1\\
606	14987.1\\
607	14987.1\\
608	14987.1\\
609	14987.1\\
610	14987.1\\
611	14987.1\\
612	14987.1\\
613	14987.1\\
614	14987.1\\
615	14987.1\\
616	14987.1\\
617	14987.1\\
618	14987.1\\
619	14987.1\\
620	14987.1\\
621	14987.1\\
622	14987.1\\
623	14987.1\\
624	14987.1\\
625	14987.1\\
626	14987.1\\
627	14987.1\\
628	14987.1\\
629	14987.1\\
630	14987.1\\
631	14987.1\\
632	14987.1\\
633	14987.1\\
634	14987.1\\
635	14987.1\\
636	14987.1\\
637	14987.1\\
638	14987.1\\
639	14987.1\\
640	14772.1\\
641	14772.1\\
642	14772.1\\
643	14772.1\\
644	14772.1\\
645	14987.1\\
646	14987.1\\
647	14987.1\\
648	14987.1\\
649	14987.1\\
650	14987.1\\
651	14987.1\\
652	14987.1\\
653	14987.1\\
654	14987.1\\
655	14987.1\\
656	14987.1\\
657	14987.1\\
658	14987.1\\
659	14987.1\\
660	14987.1\\
661	14987.1\\
662	14987.1\\
663	14987.1\\
664	14987.1\\
665	14987.1\\
666	14987.1\\
667	14987.1\\
668	14987.1\\
669	14987.1\\
670	14987.1\\
671	14527.1\\
672	14527.1\\
673	14527.1\\
674	14527.1\\
675	14527.1\\
676	14527.1\\
677	14527.1\\
678	14527.1\\
679	14987.1\\
680	14987.1\\
681	14987.1\\
682	14987.1\\
683	14987.1\\
684	14987.1\\
685	14987.1\\
686	14987.1\\
687	14987.1\\
688	14987.1\\
689	14987.1\\
690	14987.1\\
691	14987.1\\
692	14987.1\\
693	14987.1\\
694	14987.1\\
695	14987.1\\
696	14987.1\\
697	14987.1\\
698	14987.1\\
699	14987.1\\
700	14987.1\\
701	14987.1\\
702	14987.1\\
703	14987.1\\
704	14987.1\\
705	14987.1\\
706	14987.1\\
707	14987.1\\
708	14987.1\\
709	14987.1\\
710	14987.1\\
711	14987.1\\
712	14987.1\\
713	14987.1\\
714	14987.1\\
715	14987.1\\
716	14987.1\\
717	14987.1\\
718	14987.1\\
719	14987.1\\
720	14987.1\\
721	14987.1\\
722	14987.1\\
723	14987.1\\
724	14987.1\\
725	14987.1\\
726	14987.1\\
727	14987.1\\
728	14987.1\\
729	14987.1\\
730	14987.1\\
731	14987.1\\
732	14987.1\\
733	14987.1\\
734	14987.1\\
735	14987.1\\
736	14987.1\\
737	14987.1\\
738	14987.1\\
739	14987.1\\
740	14987.1\\
741	14987.1\\
742	14987.1\\
743	14987.1\\
744	14987.1\\
745	14987.1\\
746	14987.1\\
747	14987.1\\
748	14987.1\\
749	14987.1\\
750	14987.1\\
751	14987.1\\
752	14987.1\\
753	14987.1\\
754	14987.1\\
755	14987.1\\
756	14987.1\\
757	14987.1\\
758	14987.1\\
759	14987.1\\
760	14987.1\\
761	14987.1\\
762	14987.1\\
763	14987.1\\
764	14987.1\\
765	14987.1\\
766	14987.1\\
767	14987.1\\
768	14987.1\\
769	14987.1\\
770	14987.1\\
771	14987.1\\
772	14987.1\\
773	14987.1\\
774	14987.1\\
775	14987.1\\
776	14987.1\\
777	14987.1\\
778	14987.1\\
779	14987.1\\
780	14987.1\\
781	14987.1\\
782	14987.1\\
783	14987.1\\
784	14987.1\\
785	14987.1\\
786	14987.1\\
787	14987.1\\
788	14987.1\\
789	14987.1\\
790	14987.1\\
791	14987.1\\
792	14987.1\\
793	14987.1\\
794	14987.1\\
795	14987.1\\
796	14987.1\\
797	14987.1\\
798	14987.1\\
799	14987.1\\
800	14987.1\\
801	14987.1\\
802	14987.1\\
803	14987.1\\
804	14987.1\\
805	14987.1\\
806	14987.1\\
807	14987.1\\
808	14987.1\\
809	14987.1\\
810	14987.1\\
811	14987.1\\
812	14987.1\\
813	14987.1\\
814	14987.1\\
815	14987.1\\
816	14987.1\\
817	14987.1\\
818	14987.1\\
819	14987.1\\
820	14987.1\\
821	14987.1\\
822	14987.1\\
823	14987.1\\
824	14987.1\\
825	14987.1\\
826	14987.1\\
827	14987.1\\
828	14987.1\\
829	14987.1\\
830	14987.1\\
831	14987.1\\
832	14987.1\\
833	14987.1\\
834	14987.1\\
835	14987.1\\
836	14987.1\\
837	14987.1\\
838	14987.1\\
839	14987.1\\
840	14987.1\\
841	14987.1\\
842	14987.1\\
843	14987.1\\
844	14987.1\\
845	14987.1\\
846	14987.1\\
847	14987.1\\
848	14987.1\\
849	14682.1\\
850	14682.1\\
851	14682.1\\
852	14682.1\\
853	14682.1\\
854	14682.1\\
855	14682.1\\
856	14682.1\\
857	14682.1\\
858	14682.1\\
859	14682.1\\
860	14682.1\\
861	14682.1\\
862	14682.1\\
863	14682.1\\
864	14682.1\\
865	14682.1\\
866	14682.1\\
867	14682.1\\
868	14682.1\\
869	14682.1\\
870	14987.1\\
871	14987.1\\
872	14987.1\\
873	14987.1\\
874	14987.1\\
875	14987.1\\
876	14987.1\\
877	14987.1\\
878	14987.1\\
879	14987.1\\
880	14987.1\\
881	14987.1\\
882	14987.1\\
883	14987.1\\
884	14987.1\\
885	14987.1\\
886	14987.1\\
887	14987.1\\
888	14987.1\\
889	14987.1\\
890	14987.1\\
891	14987.1\\
892	14987.1\\
893	14987.1\\
894	14987.1\\
895	14987.1\\
896	14987.1\\
897	14987.1\\
898	14987.1\\
899	14987.1\\
900	14772.1\\
901	14772.1\\
902	14772.1\\
903	14772.1\\
904	14772.1\\
905	14987.1\\
906	14987.1\\
907	14987.1\\
908	14987.1\\
909	14987.1\\
910	14987.1\\
911	14987.1\\
912	14987.1\\
913	14987.1\\
914	14987.1\\
915	14987.1\\
916	14987.1\\
917	14987.1\\
918	14987.1\\
919	14987.1\\
920	14987.1\\
921	14987.1\\
922	14987.1\\
923	14987.1\\
924	14987.1\\
925	14987.1\\
926	14987.1\\
927	14987.1\\
928	14987.1\\
929	14987.1\\
930	14987.1\\
931	14987.1\\
932	14987.1\\
933	14987.1\\
934	14987.1\\
935	14987.1\\
936	14987.1\\
937	14987.1\\
938	14987.1\\
939	14987.1\\
940	14987.1\\
941	14987.1\\
942	14987.1\\
943	14987.1\\
944	14987.1\\
945	14987.1\\
946	14987.1\\
947	14987.1\\
948	14987.1\\
949	14987.1\\
950	14987.1\\
951	14987.1\\
952	14987.1\\
953	14987.1\\
954	14987.1\\
955	14987.1\\
956	14987.1\\
957	14987.1\\
958	14987.1\\
959	14987.1\\
960	14987.1\\
961	14987.1\\
962	14987.1\\
963	14987.1\\
964	14987.1\\
965	14987.1\\
966	14987.1\\
967	14987.1\\
968	14987.1\\
969	14987.1\\
970	14987.1\\
971	14987.1\\
972	14987.1\\
973	14987.1\\
974	14987.1\\
975	14987.1\\
976	14987.1\\
977	14987.1\\
978	14987.1\\
979	14987.1\\
980	14987.1\\
981	14987.1\\
982	14987.1\\
983	14987.1\\
984	14987.1\\
985	14987.1\\
986	14987.1\\
987	14987.1\\
988	14987.1\\
989	14987.1\\
990	14987.1\\
991	14987.1\\
992	14987.1\\
993	14987.1\\
994	14987.1\\
995	14987.1\\
996	14987.1\\
997	14987.1\\
998	14987.1\\
999	14987.1\\
1000	14987.1\\
1001	14987.1\\
1002	14987.1\\
1003	14987.1\\
1004	14987.1\\
1005	14987.1\\
1006	14987.1\\
1007	14987.1\\
1008	14987.1\\
1009	14987.1\\
1010	14987.1\\
1011	14987.1\\
1012	14987.1\\
1013	14987.1\\
1014	14987.1\\
1015	14987.1\\
1016	14682.1\\
1017	14682.1\\
1018	14682.1\\
1019	14987.1\\
1020	14987.1\\
1021	14987.1\\
1022	14987.1\\
1023	14987.1\\
1024	14987.1\\
1025	14987.1\\
1026	14987.1\\
1027	14987.1\\
1028	14987.1\\
1029	14987.1\\
1030	14987.1\\
1031	14987.1\\
1032	14987.1\\
1033	14987.1\\
1034	14987.1\\
1035	14987.1\\
1036	14987.1\\
1037	14987.1\\
1038	14987.1\\
1039	14987.1\\
1040	14987.1\\
1041	14987.1\\
1042	14554.6\\
1043	14554.6\\
1044	14554.6\\
1045	14554.6\\
1046	14554.6\\
1047	14554.6\\
1048	14554.6\\
1049	14554.6\\
1050	14554.6\\
1051	14554.6\\
1052	14554.6\\
1053	14554.6\\
1054	14554.6\\
1055	14554.6\\
1056	14554.6\\
1057	14554.6\\
1058	14554.6\\
1059	14554.6\\
1060	14554.6\\
1061	14554.6\\
1062	14554.6\\
1063	14554.6\\
1064	14987.1\\
1065	14987.1\\
1066	14987.1\\
1067	14987.1\\
1068	14987.1\\
1069	14987.1\\
1070	14987.1\\
1071	14987.1\\
1072	14987.1\\
1073	14987.1\\
1074	14987.1\\
1075	14987.1\\
1076	14987.1\\
1077	14987.1\\
1078	14987.1\\
1079	14987.1\\
1080	14987.1\\
1081	14987.1\\
1082	14987.1\\
1083	14987.1\\
1084	14987.1\\
1085	14987.1\\
1086	14987.1\\
1087	14987.1\\
1088	14987.1\\
1089	14987.1\\
1090	14987.1\\
1091	14987.1\\
1092	14987.1\\
1093	14987.1\\
1094	14987.1\\
1095	14987.1\\
1096	14987.1\\
1097	14987.1\\
1098	14987.1\\
1099	14987.1\\
1100	14987.1\\
1101	14987.1\\
1102	14987.1\\
1103	14987.1\\
1104	14987.1\\
1105	14987.1\\
1106	14987.1\\
1107	14987.1\\
1108	14987.1\\
1109	14987.1\\
1110	14987.1\\
1111	14987.1\\
1112	14987.1\\
1113	14987.1\\
1114	14987.1\\
1115	14987.1\\
1116	14987.1\\
1117	14987.1\\
1118	14987.1\\
1119	14987.1\\
1120	14987.1\\
1121	14987.1\\
1122	14987.1\\
1123	14987.1\\
1124	14987.1\\
1125	14987.1\\
1126	14987.1\\
1127	14987.1\\
1128	14987.1\\
1129	14987.1\\
1130	14987.1\\
1131	14987.1\\
1132	14987.1\\
1133	14987.1\\
1134	14987.1\\
1135	14987.1\\
1136	14987.1\\
1137	14987.1\\
1138	14987.1\\
1139	14987.1\\
1140	14987.1\\
1141	14987.1\\
1142	14987.1\\
1143	14987.1\\
1144	14987.1\\
1145	14987.1\\
1146	14987.1\\
1147	14987.1\\
1148	14987.1\\
1149	14987.1\\
1150	14772.1\\
1151	14772.1\\
1152	14772.1\\
1153	14772.1\\
1154	14987.1\\
1155	14772.1\\
1156	14772.1\\
1157	14772.1\\
1158	14987.1\\
1159	14987.1\\
1160	14987.1\\
1161	14987.1\\
1162	14987.1\\
1163	14987.1\\
1164	14987.1\\
1165	14987.1\\
1166	14987.1\\
1167	14987.1\\
1168	14987.1\\
1169	14987.1\\
1170	14987.1\\
1171	14987.1\\
1172	14987.1\\
1173	14987.1\\
1174	14987.1\\
1175	14987.1\\
1176	14987.1\\
1177	14987.1\\
1178	14987.1\\
1179	14987.1\\
1180	14987.1\\
1181	14987.1\\
1182	14987.1\\
1183	14987.1\\
1184	14987.1\\
1185	14987.1\\
1186	14987.1\\
1187	14987.1\\
1188	14987.1\\
1189	14987.1\\
1190	14987.1\\
1191	14987.1\\
1192	14987.1\\
1193	14987.1\\
1194	14987.1\\
1195	14987.1\\
1196	14987.1\\
1197	14987.1\\
1198	14987.1\\
1199	14987.1\\
1200	14987.1\\
1201	14987.1\\
1202	14987.1\\
1203	14987.1\\
1204	14987.1\\
1205	14987.1\\
1206	14987.1\\
1207	14987.1\\
1208	14987.1\\
1209	14987.1\\
1210	14987.1\\
1211	14987.1\\
1212	14987.1\\
1213	14987.1\\
1214	14987.1\\
1215	14987.1\\
1216	14987.1\\
1217	14987.1\\
1218	14987.1\\
1219	14987.1\\
1220	14987.1\\
1221	14987.1\\
1222	14987.1\\
1223	14987.1\\
1224	14987.1\\
1225	14987.1\\
1226	14987.1\\
1227	14987.1\\
1228	14987.1\\
1229	14987.1\\
1230	14987.1\\
1231	14987.1\\
1232	14987.1\\
1233	14987.1\\
1234	14987.1\\
1235	14987.1\\
1236	14987.1\\
1237	14987.1\\
1238	14987.1\\
1239	14987.1\\
1240	14987.1\\
1241	14987.1\\
1242	14987.1\\
1243	14987.1\\
1244	14987.1\\
1245	14987.1\\
1246	14987.1\\
1247	14987.1\\
1248	14987.1\\
1249	14987.1\\
1250	14987.1\\
1251	14987.1\\
1252	14987.1\\
1253	14987.1\\
1254	14987.1\\
1255	14987.1\\
1256	14987.1\\
1257	14987.1\\
1258	14987.1\\
1259	14987.1\\
1260	14987.1\\
1261	14987.1\\
1262	14987.1\\
1263	14987.1\\
1264	14987.1\\
1265	14987.1\\
1266	14987.1\\
1267	14987.1\\
1268	14987.1\\
1269	14987.1\\
1270	14987.1\\
1271	14987.1\\
1272	14987.1\\
1273	14987.1\\
1274	14987.1\\
1275	14987.1\\
1276	14987.1\\
1277	14987.1\\
1278	14987.1\\
1279	14987.1\\
1280	14987.1\\
1281	14987.1\\
1282	14987.1\\
1283	14987.1\\
1284	14987.1\\
1285	14987.1\\
1286	14987.1\\
1287	14987.1\\
1288	14987.1\\
1289	14987.1\\
1290	14987.1\\
1291	14987.1\\
1292	14987.1\\
1293	14987.1\\
1294	14987.1\\
1295	14987.1\\
1296	14987.1\\
1297	14987.1\\
1298	14987.1\\
1299	14987.1\\
1300	14987.1\\
1301	14987.1\\
1302	14987.1\\
1303	14987.1\\
1304	14987.1\\
1305	14987.1\\
1306	14987.1\\
1307	14987.1\\
1308	14987.1\\
1309	14987.1\\
1310	14987.1\\
1311	14987.1\\
1312	14987.1\\
1313	14987.1\\
1314	14987.1\\
1315	14987.1\\
1316	14987.1\\
1317	14987.1\\
1318	14987.1\\
1319	14987.1\\
1320	14987.1\\
1321	14987.1\\
1322	14987.1\\
1323	14987.1\\
1324	14987.1\\
1325	14987.1\\
1326	14987.1\\
1327	14987.1\\
1328	14987.1\\
1329	14987.1\\
1330	14987.1\\
1331	14987.1\\
1332	14987.1\\
1333	14987.1\\
1334	14987.1\\
1335	14987.1\\
1336	14987.1\\
1337	14987.1\\
1338	14987.1\\
1339	14987.1\\
1340	14987.1\\
1341	14987.1\\
1342	14987.1\\
1343	14987.1\\
1344	14987.1\\
1345	14987.1\\
1346	14987.1\\
1347	14987.1\\
1348	14987.1\\
1349	14987.1\\
1350	14987.1\\
1351	14987.1\\
1352	14987.1\\
1353	14987.1\\
1354	14987.1\\
1355	14987.1\\
1356	14987.1\\
1357	14987.1\\
1358	14987.1\\
1359	14987.1\\
1360	14987.1\\
1361	14987.1\\
1362	14987.1\\
1363	14987.1\\
1364	14987.1\\
1365	14987.1\\
1366	14987.1\\
1367	14987.1\\
1368	14987.1\\
1369	14987.1\\
1370	14987.1\\
1371	14987.1\\
1372	14987.1\\
1373	14987.1\\
1374	14987.1\\
1375	14987.1\\
1376	14987.1\\
1377	14987.1\\
1378	14987.1\\
1379	14987.1\\
1380	14987.1\\
1381	14987.1\\
1382	14987.1\\
1383	14987.1\\
1384	14987.1\\
1385	14987.1\\
1386	14987.1\\
1387	14987.1\\
1388	14987.1\\
1389	14582.1\\
1390	14582.1\\
1391	14582.1\\
1392	14582.1\\
1393	14987.1\\
1394	14987.1\\
1395	14987.1\\
1396	14987.1\\
1397	14987.1\\
1398	14577.1\\
1399	14577.1\\
1400	14577.1\\
1401	14577.1\\
1402	14577.1\\
1403	14577.1\\
1404	14577.1\\
1405	14577.1\\
1406	14577.1\\
1407	14577.1\\
1408	14577.1\\
1409	14577.1\\
1410	14577.1\\
1411	14577.1\\
1412	14577.1\\
1413	14577.1\\
1414	14577.1\\
1415	14577.1\\
1416	14577.1\\
1417	14987.1\\
1418	14987.1\\
1419	14987.1\\
1420	14987.1\\
1421	14987.1\\
1422	14987.1\\
1423	14987.1\\
1424	14987.1\\
1425	14987.1\\
1426	14987.1\\
1427	14987.1\\
1428	14987.1\\
1429	14987.1\\
1430	14987.1\\
1431	14987.1\\
1432	14987.1\\
1433	14987.1\\
1434	14987.1\\
1435	14987.1\\
1436	14987.1\\
1437	14987.1\\
1438	14987.1\\
1439	14987.1\\
1440	14987.1\\
1441	14987.1\\
1442	14987.1\\
1443	14987.1\\
1444	14987.1\\
1445	14987.1\\
1446	14987.1\\
1447	14987.1\\
1448	14987.1\\
1449	14987.1\\
1450	14987.1\\
1451	14987.1\\
1452	14987.1\\
1453	14987.1\\
1454	14987.1\\
1455	14987.1\\
1456	14987.1\\
1457	14987.1\\
1458	14987.1\\
1459	14987.1\\
1460	14987.1\\
1461	14987.1\\
1462	14987.1\\
1463	14987.1\\
1464	14987.1\\
1465	14987.1\\
1466	14987.1\\
1467	14987.1\\
1468	14987.1\\
1469	14987.1\\
1470	14987.1\\
1471	14987.1\\
1472	14987.1\\
1473	14987.1\\
1474	14987.1\\
1475	14987.1\\
1476	14987.1\\
1477	14987.1\\
1478	14987.1\\
1479	14987.1\\
1480	14987.1\\
1481	14987.1\\
1482	14987.1\\
1483	14987.1\\
1484	14987.1\\
1485	14987.1\\
1486	14987.1\\
1487	14987.1\\
1488	14987.1\\
1489	14987.1\\
1490	14987.1\\
1491	14987.1\\
1492	14987.1\\
1493	14987.1\\
1494	14987.1\\
1495	14987.1\\
1496	14987.1\\
1497	14987.1\\
1498	14987.1\\
1499	14837.1\\
1500	14837.1\\
1501	14837.1\\
1502	14837.1\\
1503	14837.1\\
1504	14837.1\\
1505	14837.1\\
1506	14571.1\\
1507	14550.1\\
1508	14550.1\\
1509	14550.1\\
1510	14550.1\\
1511	14550.1\\
1512	14550.1\\
1513	14550.1\\
1514	14550.1\\
1515	14550.1\\
1516	14550.1\\
1517	14550.1\\
1518	14550.1\\
1519	14550.1\\
1520	14550.1\\
1521	14550.1\\
1522	14550.1\\
1523	14550.1\\
1524	14550.1\\
1525	14550.1\\
1526	14550.1\\
1527	14550.1\\
1528	14550.1\\
1529	14550.1\\
1530	14550.1\\
1531	14676.71\\
1532	14676.71\\
1533	14676.71\\
1534	14676.71\\
1535	14676.71\\
1536	14676.71\\
1537	14676.71\\
1538	14676.71\\
1539	14987.1\\
1540	14987.1\\
1541	14987.1\\
1542	14987.1\\
1543	14987.1\\
1544	14987.1\\
1545	14987.1\\
1546	14987.1\\
1547	14987.1\\
1548	14987.1\\
1549	14987.1\\
1550	14987.1\\
1551	14987.1\\
1552	14987.1\\
1553	14987.1\\
1554	14987.1\\
1555	14987.1\\
1556	14987.1\\
1557	14987.1\\
1558	14987.1\\
1559	14987.1\\
1560	14987.1\\
1561	14987.1\\
1562	14987.1\\
1563	14987.1\\
1564	14987.1\\
1565	14987.1\\
1566	14987.1\\
1567	14987.1\\
1568	14987.1\\
1569	14987.1\\
1570	14987.1\\
1571	14987.1\\
1572	14987.1\\
1573	14987.1\\
1574	14987.1\\
1575	14987.1\\
1576	14987.1\\
1577	14987.1\\
1578	14987.1\\
1579	14987.1\\
1580	14987.1\\
1581	14987.1\\
1582	14987.1\\
1583	14987.1\\
1584	14987.1\\
1585	14987.1\\
1586	14987.1\\
1587	14987.1\\
1588	14987.1\\
1589	14987.1\\
1590	14987.1\\
1591	14987.1\\
1592	14987.1\\
1593	14987.1\\
1594	14987.1\\
1595	14987.1\\
1596	14987.1\\
1597	14772.1\\
1598	14772.1\\
1599	14772.1\\
1600	14772.1\\
1601	14772.1\\
1602	14772.1\\
1603	14772.1\\
1604	14772.1\\
1605	14772.1\\
1606	14772.1\\
1607	14772.1\\
1608	14772.1\\
1609	14772.1\\
1610	14772.1\\
1611	14772.1\\
1612	14772.1\\
1613	14772.1\\
1614	14772.1\\
1615	14772.1\\
1616	14772.1\\
1617	14772.1\\
1618	14772.1\\
1619	14772.1\\
1620	14772.1\\
1621	14772.1\\
1622	14772.1\\
1623	14772.1\\
1624	14772.1\\
1625	14772.1\\
1626	14772.1\\
1627	14772.1\\
1628	14772.1\\
1629	14772.1\\
1630	14772.1\\
1631	14772.1\\
1632	14422.1\\
1633	14422.1\\
1634	14422.1\\
1635	14422.1\\
1636	14422.1\\
1637	14422.1\\
1638	14422.1\\
1639	14422.1\\
1640	14422.1\\
1641	14422.1\\
1642	14422.1\\
1643	14422.1\\
1644	14422.1\\
1645	14422.1\\
1646	14422.1\\
1647	14422.1\\
1648	14422.1\\
1649	14422.1\\
1650	14422.1\\
1651	14422.1\\
1652	14422.1\\
1653	14422.1\\
1654	14422.1\\
1655	14422.1\\
1656	14422.1\\
1657	14772.1\\
1658	14772.1\\
1659	14772.1\\
1660	14772.1\\
1661	14772.1\\
1662	14772.1\\
1663	14772.1\\
1664	14772.1\\
1665	14772.1\\
1666	14772.1\\
1667	14772.1\\
1668	14772.1\\
1669	14772.1\\
1670	14772.1\\
1671	14772.1\\
1672	14772.1\\
1673	14772.1\\
1674	14772.1\\
1675	14772.1\\
1676	14772.1\\
1677	14772.1\\
1678	14772.1\\
1679	14772.1\\
1680	14772.1\\
1681	14772.1\\
1682	14772.1\\
1683	14772.1\\
1684	14772.1\\
1685	14772.1\\
1686	14772.1\\
1687	14772.1\\
1688	14772.1\\
1689	14772.1\\
1690	14772.1\\
1691	14772.1\\
1692	14772.1\\
1693	14772.1\\
1694	14772.1\\
1695	14772.1\\
1696	14772.1\\
1697	14772.1\\
1698	14772.1\\
1699	14772.1\\
1700	14772.1\\
1701	14772.1\\
1702	14772.1\\
1703	14772.1\\
1704	14772.1\\
1705	14772.1\\
1706	14772.1\\
1707	14772.1\\
1708	14772.1\\
1709	14772.1\\
1710	14772.1\\
1711	14772.1\\
1712	14772.1\\
1713	14772.1\\
1714	14772.1\\
1715	14335.1\\
1716	14335.1\\
1717	14335.1\\
1718	14335.1\\
1719	14335.1\\
1720	14772.1\\
1721	14772.1\\
1722	14772.1\\
1723	14772.1\\
1724	14772.1\\
1725	14772.1\\
1726	14772.1\\
1727	14772.1\\
1728	14772.1\\
1729	14772.1\\
1730	14772.1\\
1731	14772.1\\
1732	14772.1\\
1733	14772.1\\
1734	14772.1\\
1735	14772.1\\
1736	14772.1\\
1737	14772.1\\
1738	14772.1\\
1739	14772.1\\
1740	14772.1\\
1741	14772.1\\
1742	14772.1\\
1743	14772.1\\
1744	14772.1\\
1745	14772.1\\
1746	14772.1\\
1747	14772.1\\
1748	14772.1\\
1749	14772.1\\
1750	13734.1\\
1751	13734.1\\
1752	13734.1\\
1753	13734.1\\
1754	13734.1\\
1755	13949.1\\
1756	13949.1\\
1757	13949.1\\
1758	13949.1\\
1759	13949.1\\
1760	13949.1\\
1761	13949.1\\
1762	13949.1\\
1763	13949.1\\
1764	13949.1\\
1765	13949.1\\
1766	13949.1\\
1767	13949.1\\
1768	13949.1\\
1769	13949.1\\
1770	13949.1\\
1771	13949.1\\
1772	13949.1\\
1773	13949.1\\
1774	13949.1\\
1775	13949.1\\
1776	13949.1\\
1777	13949.1\\
1778	13949.1\\
1779	13949.1\\
1780	13949.1\\
1781	13949.1\\
1782	13949.1\\
1783	13949.1\\
1784	13949.1\\
1785	13949.1\\
1786	13949.1\\
1787	13949.1\\
1788	13949.1\\
1789	13949.1\\
1790	13949.1\\
1791	13949.1\\
1792	13949.1\\
1793	13949.1\\
1794	13949.1\\
1795	13949.1\\
1796	13949.1\\
1797	13949.1\\
1798	13949.1\\
1799	13949.1\\
1800	13949.1\\
1801	13949.1\\
1802	13949.1\\
1803	13949.1\\
1804	13949.1\\
1805	13949.1\\
1806	13714.1\\
1807	13714.1\\
1808	13714.1\\
1809	13714.1\\
1810	13714.1\\
1811	13714.1\\
1812	13714.1\\
1813	13714.1\\
1814	13714.1\\
1815	13714.1\\
1816	13714.1\\
1817	13714.1\\
1818	13714.1\\
1819	13714.1\\
1820	13714.1\\
1821	13714.1\\
1822	13714.1\\
1823	13714.1\\
1824	13714.1\\
1825	13714.1\\
1826	13714.1\\
1827	13714.1\\
1828	13714.1\\
1829	13714.1\\
1830	13714.1\\
1831	13691.1\\
1832	13691.1\\
1833	13691.1\\
1834	13691.1\\
1835	13691.1\\
1836	13691.1\\
1837	13714.1\\
1838	13714.1\\
1839	13714.1\\
1840	13714.1\\
1841	13714.1\\
1842	13714.1\\
1843	13714.1\\
1844	13714.1\\
1845	13714.1\\
1846	13714.1\\
1847	13714.1\\
1848	13714.1\\
1849	13714.1\\
1850	13714.1\\
1851	13714.1\\
1852	13714.1\\
1853	13714.1\\
1854	13714.1\\
1855	13714.1\\
1856	13714.1\\
1857	13714.1\\
1858	13714.1\\
1859	12752.1\\
1860	12752.1\\
1861	12752.1\\
1862	12752.1\\
1863	12752.1\\
1864	12752.1\\
1865	12752.1\\
1866	12752.1\\
1867	12987.1\\
1868	12987.1\\
1869	12987.1\\
1870	12987.1\\
1871	12987.1\\
1872	12987.1\\
1873	12859.1\\
1874	12859.1\\
1875	12859.1\\
1876	12859.1\\
1877	12859.1\\
1878	13340.1\\
1879	13821.1\\
1880	13821.1\\
1881	13821.1\\
1882	13821.1\\
1883	13821.1\\
1884	13821.1\\
1885	13821.1\\
1886	13821.1\\
1887	13821.1\\
1888	13821.1\\
1889	13821.1\\
1890	13821.1\\
1891	13821.1\\
1892	13821.1\\
1893	13821.1\\
1894	13821.1\\
1895	13821.1\\
1896	13821.1\\
1897	13821.1\\
1898	13821.1\\
1899	13821.1\\
1900	13821.1\\
1901	13821.1\\
1902	13821.1\\
1903	13821.1\\
1904	13821.1\\
1905	13821.1\\
1906	13821.1\\
1907	13821.1\\
1908	13821.1\\
1909	13821.1\\
1910	13821.1\\
1911	13821.1\\
1912	13821.1\\
1913	13821.1\\
1914	13821.1\\
1915	13821.1\\
1916	13821.1\\
1917	13821.1\\
1918	13821.1\\
1919	13821.1\\
1920	13821.1\\
1921	13821.1\\
1922	13821.1\\
1923	13821.1\\
1924	13821.1\\
1925	13821.1\\
1926	13821.1\\
1927	13821.1\\
1928	13821.1\\
1929	13821.1\\
1930	13821.1\\
1931	13821.1\\
1932	13821.1\\
1933	13821.1\\
1934	13821.1\\
1935	13821.1\\
1936	13821.1\\
1937	13821.1\\
1938	13821.1\\
1939	13821.1\\
1940	13821.1\\
1941	13821.1\\
1942	13821.1\\
1943	13821.1\\
1944	13821.1\\
1945	13821.1\\
1946	13821.1\\
1947	13821.1\\
1948	13821.1\\
1949	13821.1\\
1950	13821.1\\
1951	13821.1\\
1952	13821.1\\
1953	13821.1\\
1954	13821.1\\
1955	13821.1\\
1956	13821.1\\
1957	13821.1\\
1958	13821.1\\
1959	13821.1\\
1960	13821.1\\
1961	13821.1\\
1962	13821.1\\
1963	13821.1\\
1964	13821.1\\
1965	13821.1\\
1966	13821.1\\
1967	13821.1\\
1968	13821.1\\
1969	13821.1\\
1970	13821.1\\
1971	13821.1\\
1972	13821.1\\
1973	13821.1\\
1974	13821.1\\
1975	13821.1\\
1976	13821.1\\
1977	13821.1\\
1978	13821.1\\
1979	13821.1\\
1980	13821.1\\
1981	13821.1\\
1982	13821.1\\
1983	13821.1\\
1984	13821.1\\
1985	13821.1\\
1986	13821.1\\
1987	13821.1\\
1988	13821.1\\
1989	13821.1\\
1990	13821.1\\
1991	13821.1\\
1992	13821.1\\
1993	13821.1\\
1994	13821.1\\
1995	13821.1\\
1996	13821.1\\
1997	13821.1\\
1998	13821.1\\
1999	13821.1\\
2000	13821.1\\
2001	13821.1\\
2002	13821.1\\
2003	13821.1\\
2004	13821.1\\
2005	13821.1\\
2006	13821.1\\
2007	13821.1\\
2008	13821.1\\
2009	13821.1\\
2010	13821.1\\
2011	13821.1\\
2012	13821.1\\
2013	13821.1\\
2014	13821.1\\
2015	13821.1\\
2016	13821.1\\
2017	13821.1\\
2018	13821.1\\
2019	13821.1\\
2020	13821.1\\
2021	13821.1\\
2022	13821.1\\
2023	13821.1\\
2024	13821.1\\
2025	13821.1\\
2026	13821.1\\
2027	13821.1\\
2028	13821.1\\
2029	13821.1\\
2030	12815.1\\
2031	12815.1\\
2032	12815.1\\
2033	12815.1\\
2034	12815.1\\
2035	12815.1\\
2036	11807.1\\
2037	11807.1\\
2038	11807.1\\
2039	11807.1\\
2040	11657.1\\
2041	11657.1\\
2042	11657.1\\
2043	11657.1\\
2044	11657.1\\
2045	11807.1\\
2046	11807.1\\
2047	11807.1\\
2048	11807.1\\
2049	11807.1\\
2050	11807.1\\
2051	11807.1\\
2052	11807.1\\
2053	11807.1\\
2054	11667.1\\
2055	11667.1\\
2056	11667.1\\
2057	11667.1\\
2058	11667.1\\
2059	11517.1\\
2060	11517.1\\
2061	11517.1\\
2062	11667.1\\
2063	11667.1\\
2064	11667.1\\
2065	11667.1\\
2066	11667.1\\
2067	11667.1\\
2068	11667.1\\
2069	11667.1\\
2070	11667.1\\
2071	11667.1\\
2072	11667.1\\
2073	11667.1\\
2074	11667.1\\
2075	11667.1\\
2076	11452.1\\
2077	11452.1\\
2078	11275.1\\
2079	11490.1\\
2080	11490.1\\
2081	11630.1\\
2082	11630.1\\
2083	11630.1\\
2084	11630.1\\
2085	11630.1\\
2086	11630.1\\
2087	11630.1\\
2088	11630.1\\
2089	11630.1\\
2090	11630.1\\
2091	11630.1\\
2092	11630.1\\
2093	11630.1\\
2094	11630.1\\
2095	11630.1\\
2096	11630.1\\
2097	11630.1\\
2098	11630.1\\
2099	11630.1\\
2100	11630.1\\
2101	11630.1\\
2102	11630.1\\
2103	11630.1\\
2104	11630.1\\
2105	11630.1\\
2106	11630.1\\
2107	11630.1\\
2108	11630.1\\
2109	11630.1\\
2110	11630.1\\
2111	11630.1\\
2112	11630.1\\
2113	11630.1\\
2114	11630.1\\
2115	11630.1\\
2116	11630.1\\
2117	11630.1\\
2118	11630.1\\
2119	11630.1\\
2120	11630.1\\
2121	11630.1\\
2122	11630.1\\
2123	11630.1\\
2124	11630.1\\
2125	11630.1\\
2126	11630.1\\
2127	11630.1\\
2128	11630.1\\
2129	11630.1\\
2130	11630.1\\
2131	11630.1\\
2132	11630.1\\
2133	11630.1\\
2134	11630.1\\
2135	11630.1\\
2136	11630.1\\
2137	11630.1\\
2138	11630.1\\
2139	11630.1\\
2140	11630.1\\
2141	11630.1\\
2142	11630.1\\
2143	11630.1\\
2144	11630.1\\
2145	11630.1\\
2146	11630.1\\
2147	11630.1\\
2148	11630.1\\
2149	11630.1\\
2150	11630.1\\
2151	11630.1\\
2152	11630.1\\
2153	11630.1\\
2154	11630.1\\
2155	11630.1\\
2156	11630.1\\
2157	11630.1\\
2158	11630.1\\
2159	11630.1\\
2160	11255.1\\
2161	11255.1\\
2162	10905.1\\
2163	10905.1\\
2164	10905.1\\
2165	11255.1\\
2166	11255.1\\
2167	11516.1\\
2168	11516.1\\
2169	11516.1\\
2170	11255.1\\
2171	11255.1\\
2172	11255.1\\
2173	11255.1\\
2174	11255.1\\
2175	11255.1\\
2176	11255.1\\
2177	11255.1\\
2178	11255.1\\
2179	11255.1\\
2180	11255.1\\
2181	11255.1\\
2182	11255.1\\
2183	11255.1\\
2184	11255.1\\
2185	11255.1\\
2186	11255.1\\
2187	11255.1\\
2188	11255.1\\
2189	11255.1\\
2190	11255.1\\
2191	11255.1\\
2192	11255.1\\
2193	11255.1\\
2194	11255.1\\
2195	11255.1\\
2196	11255.1\\
2197	11255.1\\
2198	11255.1\\
2199	11255.1\\
2200	11255.1\\
2201	11255.1\\
2202	11255.1\\
2203	11255.1\\
2204	11255.1\\
2205	11255.1\\
2206	11255.1\\
2207	11255.1\\
2208	11255.1\\
2209	11255.1\\
2210	11255.1\\
2211	11255.1\\
2212	11255.1\\
2213	11255.1\\
2214	11255.1\\
2215	11255.1\\
2216	11255.1\\
2217	11255.1\\
2218	11255.1\\
2219	11255.1\\
2220	11255.1\\
2221	11255.1\\
2222	11255.1\\
2223	11255.1\\
2224	11255.1\\
2225	11255.1\\
2226	11255.1\\
2227	11255.1\\
2228	11255.1\\
2229	11255.1\\
2230	11255.1\\
2231	11255.1\\
2232	11255.1\\
2233	11255.1\\
2234	11255.1\\
2235	11255.1\\
2236	11255.1\\
2237	11255.1\\
2238	11255.1\\
2239	11255.1\\
2240	11255.1\\
2241	11255.1\\
2242	11255.1\\
2243	11255.1\\
2244	11255.1\\
2245	11255.1\\
2246	11255.1\\
2247	11255.1\\
2248	11255.1\\
2249	11255.1\\
2250	11255.1\\
2251	11255.1\\
2252	11255.1\\
2253	11255.1\\
2254	11255.1\\
2255	11255.1\\
2256	11255.1\\
2257	11255.1\\
2258	11255.1\\
2259	11255.1\\
2260	11255.1\\
2261	11255.1\\
2262	11255.1\\
2263	11255.1\\
2264	11255.1\\
2265	11255.1\\
2266	11255.1\\
2267	11255.1\\
2268	11255.1\\
2269	11255.1\\
2270	11255.1\\
2271	11255.1\\
2272	11255.1\\
2273	11255.1\\
2274	11255.1\\
2275	11255.1\\
2276	11255.1\\
2277	11255.1\\
2278	11255.1\\
2279	11255.1\\
2280	11255.1\\
2281	11255.1\\
2282	11255.1\\
2283	11255.1\\
2284	11255.1\\
2285	11255.1\\
2286	11255.1\\
2287	11255.1\\
2288	11255.1\\
2289	11255.1\\
2290	11255.1\\
2291	11255.1\\
2292	11255.1\\
2293	11255.1\\
2294	11255.1\\
2295	11255.1\\
2296	11255.1\\
2297	11255.1\\
2298	11255.1\\
2299	11255.1\\
2300	11255.1\\
2301	11255.1\\
2302	11255.1\\
2303	11255.1\\
2304	11255.1\\
2305	11255.1\\
2306	11255.1\\
2307	11255.1\\
2308	11255.1\\
2309	11255.1\\
2310	11255.1\\
2311	11255.1\\
2312	11255.1\\
2313	11255.1\\
2314	11255.1\\
2315	11255.1\\
2316	11255.1\\
2317	11255.1\\
2318	11255.1\\
2319	11255.1\\
2320	11255.1\\
2321	11255.1\\
2322	11255.1\\
2323	11255.1\\
2324	11255.1\\
2325	10818.1\\
2326	10818.1\\
2327	10818.1\\
2328	10818.1\\
2329	10818.1\\
2330	10818.1\\
2331	10818.1\\
2332	10818.1\\
2333	10818.1\\
2334	10818.1\\
2335	10818.1\\
2336	10818.1\\
2337	10818.1\\
2338	10818.1\\
2339	10818.1\\
2340	10818.1\\
2341	10818.1\\
2342	10818.1\\
2343	10818.1\\
2344	10818.1\\
2345	10818.1\\
2346	11255.1\\
2347	11255.1\\
2348	11020.1\\
2349	11020.1\\
2350	11020.1\\
2351	11020.1\\
2352	11020.1\\
2353	11020.1\\
2354	11020.1\\
2355	11020.1\\
2356	11020.1\\
2357	11020.1\\
2358	11020.1\\
2359	11020.1\\
2360	11020.1\\
2361	11020.1\\
2362	11020.1\\
2363	11020.1\\
2364	11020.1\\
2365	11020.1\\
2366	11020.1\\
2367	11020.1\\
2368	11020.1\\
2369	11020.1\\
2370	11020.1\\
2371	11020.1\\
2372	11020.1\\
2373	11020.1\\
2374	11020.1\\
2375	11020.1\\
2376	11020.1\\
2377	11020.1\\
2378	11020.1\\
2379	11020.1\\
2380	11020.1\\
2381	11020.1\\
2382	11020.1\\
2383	11020.1\\
2384	11020.1\\
2385	11020.1\\
2386	11020.1\\
2387	11020.1\\
2388	11020.1\\
2389	11020.1\\
2390	11020.1\\
2391	11020.1\\
2392	11020.1\\
2393	11020.1\\
2394	11020.1\\
2395	11020.1\\
2396	11020.1\\
2397	11020.1\\
2398	11020.1\\
2399	11020.1\\
2400	11255.1\\
2401	11255.1\\
2402	11255.1\\
2403	11255.1\\
2404	11255.1\\
2405	11255.1\\
2406	11255.1\\
2407	11255.1\\
2408	11255.1\\
2409	11255.1\\
2410	11255.1\\
2411	11255.1\\
2412	11255.1\\
2413	11255.1\\
2414	11255.1\\
2415	11255.1\\
2416	11255.1\\
2417	11255.1\\
2418	11255.1\\
2419	11255.1\\
2420	11255.1\\
2421	11255.1\\
2422	11255.1\\
2423	11255.1\\
2424	11255.1\\
2425	11255.1\\
2426	11255.1\\
2427	11255.1\\
2428	11255.1\\
2429	11255.1\\
2430	11255.1\\
2431	11255.1\\
2432	11255.1\\
2433	11255.1\\
2434	11255.1\\
2435	11255.1\\
2436	11255.1\\
2437	11255.1\\
2438	11255.1\\
2439	11255.1\\
2440	11255.1\\
2441	11255.1\\
2442	11255.1\\
2443	11255.1\\
2444	11255.1\\
2445	11255.1\\
2446	11255.1\\
2447	11255.1\\
2448	12293.1\\
2449	12293.1\\
2450	12293.1\\
2451	12293.1\\
2452	12293.1\\
2453	12293.1\\
2454	12293.1\\
2455	12293.1\\
2456	12293.1\\
2457	12293.1\\
2458	12293.1\\
2459	12293.1\\
2460	12293.1\\
2461	12293.1\\
2462	12293.1\\
2463	12293.1\\
2464	12293.1\\
2465	12293.1\\
2466	12293.1\\
2467	12293.1\\
2468	12293.1\\
2469	12293.1\\
2470	12293.1\\
2471	12293.1\\
2472	12293.1\\
2473	12293.1\\
2474	12293.1\\
2475	12293.1\\
2476	12293.1\\
2477	12293.1\\
2478	12293.1\\
2479	12293.1\\
2480	12293.1\\
2481	12293.1\\
2482	12293.1\\
2483	12293.1\\
2484	12293.1\\
2485	12293.1\\
2486	12293.1\\
2487	12293.1\\
2488	12293.1\\
2489	12293.1\\
2490	12293.1\\
2491	12293.1\\
2492	12293.1\\
2493	12293.1\\
2494	12293.1\\
2495	12293.1\\
2496	12293.1\\
2497	12293.1\\
2498	12293.1\\
2499	12293.1\\
2500	12293.1\\
2501	12293.1\\
2502	12293.1\\
2503	12293.1\\
2504	12293.1\\
2505	12293.1\\
2506	12293.1\\
2507	12293.1\\
2508	12293.1\\
2509	12293.1\\
2510	12293.1\\
2511	12293.1\\
2512	12293.1\\
2513	12293.1\\
2514	12293.1\\
2515	12293.1\\
2516	12293.1\\
2517	12293.1\\
2518	12293.1\\
2519	12293.1\\
2520	12293.1\\
2521	12293.1\\
2522	12293.1\\
2523	12293.1\\
2524	12293.1\\
2525	12293.1\\
2526	12293.1\\
2527	12293.1\\
2528	12293.1\\
2529	12293.1\\
2530	12293.1\\
2531	12293.1\\
2532	12293.1\\
2533	12293.1\\
2534	12293.1\\
2535	12293.1\\
2536	12293.1\\
2537	12293.1\\
2538	12293.1\\
2539	12293.1\\
2540	12293.1\\
2541	12293.1\\
2542	12293.1\\
2543	12293.1\\
2544	12293.1\\
2545	12293.1\\
2546	12293.1\\
2547	12293.1\\
2548	12293.1\\
2549	12293.1\\
2550	12293.1\\
2551	12293.1\\
2552	12293.1\\
2553	12293.1\\
2554	12293.1\\
2555	12293.1\\
2556	12293.1\\
2557	12293.1\\
2558	12293.1\\
2559	12293.1\\
2560	12293.1\\
2561	12293.1\\
2562	12293.1\\
2563	12293.1\\
2564	12293.1\\
2565	12293.1\\
2566	12293.1\\
2567	12293.1\\
2568	12293.1\\
2569	12293.1\\
2570	12293.1\\
2571	12293.1\\
2572	12293.1\\
2573	12293.1\\
2574	12293.1\\
2575	12293.1\\
2576	12293.1\\
2577	12293.1\\
2578	12293.1\\
2579	12293.1\\
2580	12293.1\\
2581	12293.1\\
2582	12293.1\\
2583	12293.1\\
2584	12293.1\\
2585	12293.1\\
2586	12293.1\\
2587	12293.1\\
2588	12293.1\\
2589	12293.1\\
2590	12293.1\\
2591	12293.1\\
2592	12293.1\\
2593	12293.1\\
2594	12293.1\\
2595	12293.1\\
2596	12293.1\\
2597	12293.1\\
2598	12293.1\\
2599	12293.1\\
2600	12293.1\\
2601	12293.1\\
2602	12293.1\\
2603	12293.1\\
2604	12293.1\\
2605	12293.1\\
2606	12293.1\\
2607	12293.1\\
2608	12293.1\\
2609	12293.1\\
2610	12293.1\\
2611	12293.1\\
2612	12293.1\\
2613	12293.1\\
2614	12293.1\\
2615	12293.1\\
2616	12293.1\\
2617	12293.1\\
2618	12293.1\\
2619	12293.1\\
2620	12293.1\\
2621	12293.1\\
2622	12293.1\\
2623	12293.1\\
2624	12293.1\\
2625	12293.1\\
2626	12293.1\\
2627	12293.1\\
2628	12293.1\\
2629	12293.1\\
2630	12293.1\\
2631	12293.1\\
2632	12293.1\\
2633	12293.1\\
2634	12293.1\\
2635	12293.1\\
2636	12293.1\\
2637	12293.1\\
2638	12293.1\\
2639	12293.1\\
2640	12293.1\\
2641	12293.1\\
2642	12293.1\\
2643	12293.1\\
2644	12293.1\\
2645	12293.1\\
2646	12293.1\\
2647	12293.1\\
2648	12293.1\\
2649	12293.1\\
2650	12293.1\\
2651	12293.1\\
2652	12293.1\\
2653	12293.1\\
2654	12293.1\\
2655	12293.1\\
2656	12293.1\\
2657	12293.1\\
2658	12293.1\\
2659	12293.1\\
2660	12293.1\\
2661	12293.1\\
2662	12293.1\\
2663	12293.1\\
2664	12293.1\\
2665	12293.1\\
2666	12293.1\\
2667	12293.1\\
2668	12293.1\\
2669	12293.1\\
2670	12293.1\\
2671	12293.1\\
2672	12293.1\\
2673	12293.1\\
2674	12293.1\\
2675	12293.1\\
2676	12293.1\\
2677	12293.1\\
2678	12293.1\\
2679	12293.1\\
2680	12293.1\\
2681	12293.1\\
2682	12293.1\\
2683	12293.1\\
2684	12293.1\\
2685	12293.1\\
2686	12293.1\\
2687	12293.1\\
2688	12293.1\\
2689	12293.1\\
2690	12293.1\\
2691	12293.1\\
2692	12293.1\\
2693	12293.1\\
2694	12293.1\\
2695	12293.1\\
2696	12293.1\\
2697	12293.1\\
2698	12293.1\\
2699	12293.1\\
2700	12293.1\\
2701	12293.1\\
2702	12293.1\\
2703	12293.1\\
2704	12293.1\\
2705	12293.1\\
2706	12293.1\\
2707	12293.1\\
2708	12293.1\\
2709	12293.1\\
2710	12293.1\\
2711	12293.1\\
2712	12293.1\\
2713	12293.1\\
2714	12293.1\\
2715	12293.1\\
2716	12293.1\\
2717	12293.1\\
2718	12293.1\\
2719	12293.1\\
2720	12293.1\\
2721	12293.1\\
2722	12293.1\\
2723	12293.1\\
2724	12293.1\\
2725	12293.1\\
2726	12293.1\\
2727	12293.1\\
2728	12293.1\\
2729	12293.1\\
2730	12293.1\\
2731	12293.1\\
2732	12293.1\\
2733	12293.1\\
2734	12598.1\\
2735	12598.1\\
2736	12598.1\\
2737	12598.1\\
2738	12598.1\\
2739	12598.1\\
2740	12598.1\\
2741	12598.1\\
2742	12598.1\\
2743	12598.1\\
2744	12598.1\\
2745	12598.1\\
2746	12598.1\\
2747	12483.1\\
2748	12483.1\\
2749	12483.1\\
2750	12483.1\\
2751	12483.1\\
2752	12483.1\\
2753	12483.1\\
2754	12293.1\\
2755	12293.1\\
2756	12293.1\\
2757	12293.1\\
2758	12293.1\\
2759	12293.1\\
2760	12293.1\\
2761	12293.1\\
2762	12293.1\\
2763	12293.1\\
2764	12293.1\\
2765	12293.1\\
2766	12293.1\\
2767	12293.1\\
2768	12293.1\\
2769	12293.1\\
2770	12293.1\\
2771	12293.1\\
2772	12293.1\\
2773	12293.1\\
2774	12293.1\\
2775	12293.1\\
2776	12293.1\\
2777	12293.1\\
2778	12293.1\\
2779	12293.1\\
2780	12293.1\\
2781	12293.1\\
2782	12598.1\\
2783	12598.1\\
2784	12598.1\\
2785	12598.1\\
2786	12598.1\\
2787	12598.1\\
2788	12598.1\\
2789	12598.1\\
2790	12598.1\\
2791	12598.1\\
2792	12598.1\\
2793	12598.1\\
2794	12598.1\\
2795	12598.1\\
2796	12598.1\\
2797	12598.1\\
2798	12598.1\\
2799	12598.1\\
2800	12598.1\\
2801	12598.1\\
2802	12428.1\\
2803	12428.1\\
2804	12428.1\\
2805	12428.1\\
2806	12428.1\\
2807	12428.1\\
2808	12428.1\\
2809	12428.1\\
2810	12428.1\\
2811	12428.1\\
2812	12428.1\\
2813	12428.1\\
2814	12428.1\\
2815	12428.1\\
2816	12428.1\\
2817	12428.1\\
2818	12248.1\\
2819	12248.1\\
2820	12248.1\\
2821	12248.1\\
2822	12248.1\\
2823	12248.1\\
2824	12248.1\\
2825	12248.1\\
2826	12248.1\\
2827	12248.1\\
2828	12248.1\\
2829	12248.1\\
2830	12248.1\\
2831	12248.1\\
2832	12248.1\\
2833	12248.1\\
2834	12248.1\\
2835	12248.1\\
2836	12248.1\\
2837	12248.1\\
2838	12248.1\\
2839	12248.1\\
2840	12248.1\\
2841	12248.1\\
2842	12248.1\\
2843	12248.1\\
2844	12248.1\\
2845	12248.1\\
2846	12248.1\\
2847	12248.1\\
2848	12248.1\\
2849	12598.1\\
2850	12598.1\\
2851	12598.1\\
2852	12598.1\\
2853	12598.1\\
2854	12598.1\\
2855	12598.1\\
2856	12598.1\\
2857	12598.1\\
2858	12598.1\\
2859	12598.1\\
2860	12598.1\\
2861	12598.1\\
2862	12598.1\\
2863	12598.1\\
2864	12598.1\\
2865	12598.1\\
2866	12598.1\\
2867	12598.1\\
2868	12598.1\\
2869	12598.1\\
2870	12598.1\\
2871	12598.1\\
2872	12598.1\\
2873	12598.1\\
2874	12598.1\\
2875	12598.1\\
2876	12598.1\\
2877	12598.1\\
2878	12598.1\\
2879	12598.1\\
2880	12598.1\\
2881	12598.1\\
2882	12598.1\\
2883	12598.1\\
2884	12598.1\\
2885	12598.1\\
2886	12598.1\\
2887	12598.1\\
2888	12598.1\\
2889	12598.1\\
2890	12598.1\\
2891	12598.1\\
2892	12598.1\\
2893	12598.1\\
2894	12293.1\\
2895	12293.1\\
2896	12293.1\\
2897	12293.1\\
2898	12293.1\\
2899	12293.1\\
2900	12293.1\\
2901	12293.1\\
2902	12293.1\\
2903	12293.1\\
2904	12293.1\\
2905	12293.1\\
2906	12293.1\\
2907	12293.1\\
2908	12293.1\\
2909	12293.1\\
2910	12293.1\\
2911	12598.1\\
2912	12598.1\\
2913	12598.1\\
2914	12598.1\\
2915	12598.1\\
2916	12598.1\\
2917	12598.1\\
2918	12598.1\\
2919	12598.1\\
2920	12598.1\\
2921	12598.1\\
2922	12598.1\\
2923	12598.1\\
2924	12598.1\\
2925	12598.1\\
2926	12598.1\\
2927	12598.1\\
2928	12598.1\\
2929	12598.1\\
2930	12598.1\\
2931	12598.1\\
2932	12598.1\\
2933	12598.1\\
2934	12598.1\\
2935	12598.1\\
2936	12598.1\\
2937	12598.1\\
2938	12598.1\\
2939	12598.1\\
2940	12598.1\\
2941	12598.1\\
2942	12598.1\\
2943	12598.1\\
2944	12598.1\\
2945	12598.1\\
2946	12598.1\\
2947	12598.1\\
2948	12598.1\\
2949	12598.1\\
2950	12598.1\\
2951	12598.1\\
2952	12598.1\\
2953	12598.1\\
2954	12598.1\\
2955	12598.1\\
2956	12598.1\\
2957	12598.1\\
2958	12598.1\\
2959	12598.1\\
2960	12598.1\\
2961	12598.1\\
2962	12598.1\\
2963	12598.1\\
2964	12598.1\\
2965	12598.1\\
2966	12598.1\\
2967	12598.1\\
2968	12598.1\\
2969	12598.1\\
2970	12598.1\\
2971	12598.1\\
2972	12598.1\\
2973	12598.1\\
2974	12598.1\\
2975	12598.1\\
2976	12598.1\\
2977	12598.1\\
2978	12598.1\\
2979	12598.1\\
2980	12598.1\\
2981	12433.1\\
2982	12433.1\\
2983	12433.1\\
2984	12433.1\\
2985	12218.1\\
2986	12218.1\\
2987	12433.1\\
2988	12433.1\\
2989	12433.1\\
2990	12433.1\\
2991	12433.1\\
2992	12433.1\\
2993	12433.1\\
2994	12433.1\\
2995	12433.1\\
2996	12433.1\\
2997	12433.1\\
2998	12433.1\\
2999	12433.1\\
3000	12433.1\\
3001	12433.1\\
3002	12433.1\\
3003	12433.1\\
3004	12433.1\\
3005	12433.1\\
3006	12433.1\\
3007	12433.1\\
3008	12433.1\\
3009	12433.1\\
3010	12433.1\\
3011	12433.1\\
3012	12433.1\\
3013	12433.1\\
3014	12433.1\\
3015	12433.1\\
3016	12433.1\\
3017	12433.1\\
3018	12433.1\\
3019	12433.1\\
3020	12433.1\\
3021	11996.1\\
3022	11996.1\\
3023	11996.1\\
3024	11996.1\\
3025	11996.1\\
3026	11996.1\\
3027	11996.1\\
3028	11996.1\\
3029	11996.1\\
3030	11996.1\\
3031	11996.1\\
3032	11996.1\\
3033	11996.1\\
3034	11996.1\\
3035	11996.1\\
3036	11996.1\\
3037	11996.1\\
3038	11996.1\\
3039	11996.1\\
3040	11996.1\\
3041	11996.1\\
3042	11996.1\\
3043	11996.1\\
3044	11996.1\\
3045	11996.1\\
3046	11996.1\\
3047	11996.1\\
3048	11996.1\\
3049	11996.1\\
3050	11996.1\\
3051	11996.1\\
3052	11996.1\\
3053	11996.1\\
3054	12161.1\\
3055	12161.1\\
3056	12161.1\\
3057	12161.1\\
3058	12161.1\\
3059	12161.1\\
3060	12161.1\\
3061	12161.1\\
3062	12161.1\\
3063	12161.1\\
3064	12161.1\\
3065	12161.1\\
3066	12161.1\\
3067	12161.1\\
3068	12161.1\\
3069	12161.1\\
3070	12161.1\\
3071	12161.1\\
3072	12161.1\\
3073	12161.1\\
3074	12161.1\\
3075	12161.1\\
3076	12161.1\\
3077	12161.1\\
3078	12161.1\\
3079	12161.1\\
3080	12161.1\\
3081	12161.1\\
3082	12161.1\\
3083	12161.1\\
3084	12161.1\\
3085	12161.1\\
3086	12161.1\\
3087	12161.1\\
3088	12161.1\\
3089	12161.1\\
3090	12161.1\\
3091	12161.1\\
3092	12598.1\\
3093	12598.1\\
3094	12598.1\\
3095	12598.1\\
3096	12161.1\\
3097	12161.1\\
3098	12161.1\\
3099	12161.1\\
3100	12161.1\\
3101	12161.1\\
3102	12161.1\\
3103	12161.1\\
3104	12161.1\\
3105	12161.1\\
3106	12161.1\\
3107	12287.71\\
3108	12287.71\\
3109	12287.71\\
3110	12287.71\\
3111	12287.71\\
3112	12598.1\\
3113	12598.1\\
3114	12598.1\\
3115	12598.1\\
3116	12598.1\\
3117	12598.1\\
3118	12598.1\\
3119	12598.1\\
3120	12598.1\\
3121	12598.1\\
3122	12598.1\\
3123	12598.1\\
3124	12598.1\\
3125	12598.1\\
3126	12598.1\\
3127	12598.1\\
3128	12598.1\\
3129	12598.1\\
3130	12598.1\\
3131	12598.1\\
3132	12598.1\\
3133	12598.1\\
3134	12598.1\\
3135	12598.1\\
3136	12598.1\\
3137	12598.1\\
3138	12363.1\\
3139	12363.1\\
3140	12363.1\\
3141	12363.1\\
3142	12363.1\\
3143	12363.1\\
3144	12363.1\\
3145	12363.1\\
3146	12363.1\\
3147	12363.1\\
3148	12363.1\\
3149	12363.1\\
3150	12363.1\\
3151	12363.1\\
3152	12363.1\\
3153	12363.1\\
3154	12363.1\\
3155	12363.1\\
3156	12363.1\\
3157	12363.1\\
3158	12363.1\\
3159	12363.1\\
3160	12363.1\\
3161	12363.1\\
3162	12363.1\\
3163	12363.1\\
3164	12363.1\\
3165	12363.1\\
3166	12363.1\\
3167	12363.1\\
3168	12363.1\\
3169	12598.1\\
3170	12598.1\\
3171	12598.1\\
3172	12598.1\\
3173	12598.1\\
3174	12598.1\\
3175	12598.1\\
3176	12598.1\\
3177	12598.1\\
3178	12598.1\\
3179	12598.1\\
3180	12598.1\\
3181	12598.1\\
3182	12598.1\\
3183	12598.1\\
3184	12598.1\\
3185	12598.1\\
3186	12598.1\\
3187	12598.1\\
3188	12598.1\\
3189	12598.1\\
3190	12598.1\\
3191	12598.1\\
3192	12598.1\\
3193	12598.1\\
3194	12598.1\\
3195	12598.1\\
3196	12598.1\\
3197	12598.1\\
3198	12598.1\\
3199	12598.1\\
3200	12598.1\\
3201	12598.1\\
3202	12598.1\\
3203	12598.1\\
3204	12598.1\\
3205	12598.1\\
3206	12598.1\\
3207	12598.1\\
3208	12598.1\\
3209	12598.1\\
3210	12598.1\\
3211	12598.1\\
3212	12598.1\\
3213	12598.1\\
3214	12598.1\\
3215	12598.1\\
3216	12598.1\\
3217	12598.1\\
3218	12598.1\\
3219	12598.1\\
3220	12598.1\\
3221	12598.1\\
3222	12598.1\\
3223	12598.1\\
3224	12383.1\\
3225	12383.1\\
3226	12383.1\\
3227	12383.1\\
3228	12383.1\\
3229	12598.1\\
3230	12598.1\\
3231	12598.1\\
3232	12598.1\\
3233	12598.1\\
3234	12598.1\\
3235	12598.1\\
3236	12598.1\\
3237	12598.1\\
3238	12598.1\\
3239	12598.1\\
3240	12598.1\\
3241	12598.1\\
3242	12598.1\\
3243	12598.1\\
3244	12598.1\\
3245	12598.1\\
3246	12598.1\\
3247	12598.1\\
3248	12598.1\\
3249	12598.1\\
3250	12598.1\\
3251	12598.1\\
3252	12598.1\\
3253	12598.1\\
3254	12598.1\\
3255	12598.1\\
3256	12598.1\\
3257	12598.1\\
3258	12598.1\\
3259	12598.1\\
3260	12598.1\\
3261	12598.1\\
3262	12598.1\\
3263	12598.1\\
3264	12193.1\\
3265	12193.1\\
3266	12193.1\\
3267	12193.1\\
3268	12193.1\\
3269	12193.1\\
3270	12193.1\\
3271	12193.1\\
3272	12193.1\\
3273	12193.1\\
3274	12193.1\\
3275	12193.1\\
3276	12193.1\\
3277	12193.1\\
3278	12193.1\\
3279	12193.1\\
3280	12193.1\\
3281	12193.1\\
3282	12193.1\\
3283	12193.1\\
3284	12193.1\\
3285	12193.1\\
3286	12193.1\\
3287	12193.1\\
3288	12193.1\\
3289	12598.1\\
3290	12598.1\\
3291	12598.1\\
3292	12598.1\\
3293	12598.1\\
3294	12598.1\\
3295	12598.1\\
3296	12598.1\\
3297	12598.1\\
3298	12598.1\\
3299	12598.1\\
3300	12598.1\\
3301	12598.1\\
3302	12598.1\\
3303	12598.1\\
3304	12598.1\\
3305	12598.1\\
3306	12598.1\\
3307	12598.1\\
3308	12598.1\\
3309	12598.1\\
3310	12598.1\\
3311	12598.1\\
3312	12598.1\\
3313	12598.1\\
3314	12598.1\\
3315	12598.1\\
3316	12598.1\\
3317	12598.1\\
3318	12598.1\\
3319	12598.1\\
3320	12598.1\\
3321	12598.1\\
3322	12598.1\\
3323	12598.1\\
3324	12598.1\\
3325	12598.1\\
3326	12598.1\\
3327	12598.1\\
3328	12598.1\\
3329	12598.1\\
3330	12598.1\\
3331	12598.1\\
3332	12598.1\\
3333	12598.1\\
3334	12598.1\\
3335	12598.1\\
3336	12193.1\\
3337	12193.1\\
3338	12193.1\\
3339	12193.1\\
3340	12193.1\\
3341	12193.1\\
3342	12193.1\\
3343	12193.1\\
3344	12193.1\\
3345	12193.1\\
3346	12193.1\\
3347	12193.1\\
3348	12193.1\\
3349	12193.1\\
3350	12193.1\\
3351	12193.1\\
3352	12193.1\\
3353	12193.1\\
3354	12193.1\\
3355	12193.1\\
3356	12193.1\\
3357	12193.1\\
3358	12193.1\\
3359	12193.1\\
3360	12188.1\\
3361	12193.1\\
3362	12193.1\\
3363	12193.1\\
3364	12193.1\\
3365	12193.1\\
3366	12193.1\\
3367	12193.1\\
3368	12193.1\\
3369	12193.1\\
3370	12193.1\\
3371	12193.1\\
3372	12193.1\\
3373	12193.1\\
3374	12193.1\\
3375	12193.1\\
3376	12193.1\\
3377	12193.1\\
3378	12193.1\\
3379	12193.1\\
3380	12193.1\\
3381	12193.1\\
3382	12193.1\\
3383	12193.1\\
3384	12193.1\\
3385	12598.1\\
3386	12598.1\\
3387	12598.1\\
3388	12598.1\\
3389	12598.1\\
3390	12598.1\\
3391	12598.1\\
3392	12598.1\\
3393	12598.1\\
3394	12598.1\\
3395	12598.1\\
3396	12598.1\\
3397	12598.1\\
3398	12598.1\\
3399	12598.1\\
3400	12598.1\\
3401	12598.1\\
3402	12598.1\\
3403	12598.1\\
3404	12598.1\\
3405	12598.1\\
3406	12598.1\\
3407	12598.1\\
3408	12598.1\\
3409	12598.1\\
3410	12598.1\\
3411	12598.1\\
3412	12598.1\\
3413	12598.1\\
3414	12598.1\\
3415	12598.1\\
3416	12598.1\\
3417	12598.1\\
3418	12598.1\\
3419	12598.1\\
3420	12598.1\\
3421	12598.1\\
3422	12598.1\\
3423	12598.1\\
3424	12598.1\\
3425	12598.1\\
3426	12598.1\\
3427	12598.1\\
3428	12598.1\\
3429	12598.1\\
3430	12598.1\\
3431	12598.1\\
3432	12598.1\\
3433	12598.1\\
3434	12598.1\\
3435	12598.1\\
3436	12598.1\\
3437	12598.1\\
3438	12598.1\\
3439	12598.1\\
3440	12598.1\\
3441	12598.1\\
3442	12598.1\\
3443	12598.1\\
3444	12598.1\\
3445	12598.1\\
3446	12598.1\\
3447	12598.1\\
3448	12598.1\\
3449	12598.1\\
3450	12598.1\\
3451	12598.1\\
3452	12598.1\\
3453	12598.1\\
3454	12598.1\\
3455	12598.1\\
3456	12598.1\\
3457	12598.1\\
3458	12598.1\\
3459	12598.1\\
3460	12598.1\\
3461	12598.1\\
3462	12598.1\\
3463	12598.1\\
3464	12598.1\\
3465	12598.1\\
3466	12598.1\\
3467	12598.1\\
3468	12598.1\\
3469	12598.1\\
3470	12598.1\\
3471	12598.1\\
3472	12598.1\\
3473	12598.1\\
3474	12598.1\\
3475	12598.1\\
3476	12598.1\\
3477	12598.1\\
3478	12598.1\\
3479	12598.1\\
3480	12598.1\\
3481	12598.1\\
3482	12598.1\\
3483	12598.1\\
3484	12598.1\\
3485	12598.1\\
3486	12598.1\\
3487	12383.1\\
3488	12383.1\\
3489	12383.1\\
3490	12383.1\\
3491	12383.1\\
3492	12383.1\\
3493	12598.1\\
3494	12598.1\\
3495	12598.1\\
3496	12598.1\\
3497	12598.1\\
3498	12598.1\\
3499	12598.1\\
3500	12598.1\\
3501	12598.1\\
3502	12598.1\\
3503	12598.1\\
3504	12598.1\\
3505	12598.1\\
3506	12598.1\\
3507	12598.1\\
3508	12598.1\\
3509	12598.1\\
3510	12598.1\\
3511	12598.1\\
3512	12598.1\\
3513	12598.1\\
3514	12598.1\\
3515	12598.1\\
3516	12598.1\\
3517	12598.1\\
3518	12598.1\\
3519	12598.1\\
3520	12598.1\\
3521	12598.1\\
3522	12598.1\\
3523	12598.1\\
3524	12598.1\\
3525	12598.1\\
3526	12598.1\\
3527	12598.1\\
3528	12598.1\\
3529	12598.1\\
3530	12598.1\\
3531	12598.1\\
3532	12598.1\\
3533	12598.1\\
3534	12598.1\\
3535	12598.1\\
3536	12598.1\\
3537	12598.1\\
3538	12598.1\\
3539	12598.1\\
3540	12598.1\\
3541	12598.1\\
3542	12598.1\\
3543	12598.1\\
3544	12598.1\\
3545	12598.1\\
3546	12598.1\\
3547	12598.1\\
3548	12598.1\\
3549	12598.1\\
3550	12598.1\\
3551	12598.1\\
3552	12598.1\\
3553	12598.1\\
3554	12598.1\\
3555	12598.1\\
3556	12598.1\\
3557	12598.1\\
3558	12598.1\\
3559	12598.1\\
3560	12598.1\\
3561	12598.1\\
3562	12598.1\\
3563	12598.1\\
3564	12598.1\\
3565	12598.1\\
3566	12598.1\\
3567	12598.1\\
3568	12598.1\\
3569	12598.1\\
3570	12598.1\\
3571	12598.1\\
3572	12598.1\\
3573	12598.1\\
3574	12598.1\\
3575	12598.1\\
3576	12598.1\\
3577	12598.1\\
3578	12598.1\\
3579	12598.1\\
3580	12598.1\\
3581	12598.1\\
3582	12598.1\\
3583	12598.1\\
3584	12598.1\\
3585	12598.1\\
3586	12598.1\\
3587	12598.1\\
3588	12598.1\\
3589	12598.1\\
3590	12598.1\\
3591	12598.1\\
3592	12598.1\\
3593	12598.1\\
3594	12598.1\\
3595	12598.1\\
3596	12598.1\\
3597	12598.1\\
3598	12598.1\\
3599	12598.1\\
3600	12598.1\\
3601	12598.1\\
3602	12598.1\\
3603	12483.1\\
3604	12483.1\\
3605	12483.1\\
3606	12483.1\\
3607	12483.1\\
3608	12483.1\\
3609	12483.1\\
3610	12483.1\\
3611	12483.1\\
3612	12483.1\\
3613	12483.1\\
3614	12483.1\\
3615	12483.1\\
3616	12483.1\\
3617	12483.1\\
3618	12483.1\\
3619	12483.1\\
3620	12483.1\\
3621	12483.1\\
3622	12483.1\\
3623	12483.1\\
3624	12483.1\\
3625	12483.1\\
3626	12483.1\\
3627	12133.1\\
3628	12133.1\\
3629	12133.1\\
3630	12133.1\\
3631	12133.1\\
3632	12248.1\\
3633	12248.1\\
3634	12248.1\\
3635	12248.1\\
3636	12248.1\\
3637	12248.1\\
3638	12248.1\\
3639	12248.1\\
3640	12248.1\\
3641	12248.1\\
3642	12248.1\\
3643	12248.1\\
3644	12248.1\\
3645	12248.1\\
3646	12248.1\\
3647	12248.1\\
3648	12248.1\\
3649	12248.1\\
3650	12248.1\\
3651	12248.1\\
3652	12248.1\\
3653	12248.1\\
3654	12248.1\\
3655	12248.1\\
3656	12248.1\\
3657	12248.1\\
3658	12248.1\\
3659	12248.1\\
3660	12248.1\\
3661	12248.1\\
3662	12248.1\\
3663	12248.1\\
3664	12248.1\\
3665	12248.1\\
3666	12248.1\\
3667	12248.1\\
3668	12248.1\\
3669	12248.1\\
3670	12248.1\\
3671	12248.1\\
3672	12248.1\\
3673	12248.1\\
3674	12248.1\\
3675	12248.1\\
3676	12248.1\\
3677	12248.1\\
3678	12248.1\\
3679	12248.1\\
3680	12248.1\\
3681	12248.1\\
3682	12248.1\\
3683	12248.1\\
3684	12248.1\\
3685	12248.1\\
3686	12248.1\\
3687	12248.1\\
3688	12248.1\\
3689	12248.1\\
3690	12248.1\\
3691	12248.1\\
3692	12248.1\\
3693	12248.1\\
3694	12248.1\\
3695	12248.1\\
3696	12248.1\\
3697	11811.1\\
3698	11811.1\\
3699	11811.1\\
3700	11811.1\\
3701	11811.1\\
3702	11811.1\\
3703	11811.1\\
3704	11811.1\\
3705	11811.1\\
3706	11811.1\\
3707	11811.1\\
3708	11811.1\\
3709	11811.1\\
3710	12248.1\\
3711	12248.1\\
3712	12248.1\\
3713	12248.1\\
3714	11811.1\\
3715	11811.1\\
3716	11811.1\\
3717	11811.1\\
3718	12248.1\\
3719	12248.1\\
3720	12248.1\\
3721	12248.1\\
3722	12248.1\\
3723	12248.1\\
3724	12248.1\\
3725	12248.1\\
3726	12248.1\\
3727	12248.1\\
3728	12248.1\\
3729	12248.1\\
3730	12248.1\\
3731	12248.1\\
3732	12248.1\\
3733	12248.1\\
3734	12248.1\\
3735	12248.1\\
3736	12248.1\\
3737	12248.1\\
3738	12248.1\\
3739	12248.1\\
3740	12248.1\\
3741	12248.1\\
3742	12248.1\\
3743	12248.1\\
3744	12248.1\\
3745	12248.1\\
3746	12248.1\\
3747	12248.1\\
3748	12248.1\\
3749	12248.1\\
3750	12248.1\\
3751	12248.1\\
3752	12248.1\\
3753	12248.1\\
3754	12248.1\\
3755	12248.1\\
3756	12248.1\\
3757	12248.1\\
3758	12248.1\\
3759	12248.1\\
3760	12248.1\\
3761	12248.1\\
3762	12248.1\\
3763	12248.1\\
3764	12248.1\\
3765	12248.1\\
3766	12248.1\\
3767	12248.1\\
3768	12248.1\\
3769	12248.1\\
3770	12248.1\\
3771	12248.1\\
3772	12248.1\\
3773	12248.1\\
3774	12248.1\\
3775	12248.1\\
3776	12248.1\\
3777	12248.1\\
3778	12248.1\\
3779	12248.1\\
3780	12248.1\\
3781	12248.1\\
3782	12248.1\\
3783	12248.1\\
3784	12248.1\\
3785	12248.1\\
3786	12248.1\\
3787	12248.1\\
3788	12248.1\\
3789	12248.1\\
3790	12248.1\\
3791	12248.1\\
3792	12248.1\\
3793	12248.1\\
3794	12248.1\\
3795	12248.1\\
3796	12248.1\\
3797	12248.1\\
3798	12248.1\\
3799	12248.1\\
3800	12248.1\\
3801	12248.1\\
3802	12248.1\\
3803	12248.1\\
3804	12248.1\\
3805	12248.1\\
3806	12248.1\\
3807	12248.1\\
3808	12248.1\\
3809	12248.1\\
3810	12248.1\\
3811	12248.1\\
3812	12248.1\\
3813	12248.1\\
3814	12248.1\\
3815	12248.1\\
3816	12248.1\\
3817	12248.1\\
3818	12248.1\\
3819	12248.1\\
3820	12248.1\\
3821	12248.1\\
3822	12248.1\\
3823	12248.1\\
3824	12248.1\\
3825	12248.1\\
3826	12248.1\\
3827	12248.1\\
3828	12248.1\\
3829	12248.1\\
3830	12248.1\\
3831	12248.1\\
3832	12248.1\\
3833	12248.1\\
3834	12248.1\\
3835	12248.1\\
3836	12248.1\\
3837	12248.1\\
3838	12248.1\\
3839	12248.1\\
3840	11783.1\\
3841	11783.1\\
3842	11783.1\\
3843	11783.1\\
3844	11783.1\\
3845	11783.1\\
3846	11783.1\\
3847	11783.1\\
3848	11783.1\\
3849	11783.1\\
3850	11783.1\\
3851	11783.1\\
3852	11783.1\\
3853	11783.1\\
3854	11783.1\\
3855	11783.1\\
3856	11783.1\\
3857	11783.1\\
3858	11783.1\\
3859	11783.1\\
3860	11783.1\\
3861	11783.1\\
3862	11783.1\\
3863	11783.1\\
3864	11783.1\\
3865	11613.1\\
3866	11613.1\\
3867	11613.1\\
3868	11613.1\\
3869	11613.1\\
3870	11455.6\\
3871	11455.6\\
3872	11455.6\\
3873	11455.6\\
3874	11455.6\\
3875	11455.6\\
3876	11455.6\\
3877	11455.6\\
3878	11455.6\\
3879	11455.6\\
3880	11455.6\\
3881	11455.6\\
3882	11783.1\\
3883	11783.1\\
3884	11783.1\\
3885	11783.1\\
3886	11783.1\\
3887	11783.1\\
3888	11783.1\\
3889	11783.1\\
3890	11783.1\\
3891	11783.1\\
3892	11783.1\\
3893	11783.1\\
3894	11783.1\\
3895	11783.1\\
3896	11783.1\\
3897	11783.1\\
3898	11783.1\\
3899	11783.1\\
3900	11783.1\\
3901	11783.1\\
3902	11783.1\\
3903	11783.1\\
3904	11783.1\\
3905	11783.1\\
3906	11783.1\\
3907	11783.1\\
3908	11783.1\\
3909	11783.1\\
3910	11783.1\\
3911	11783.1\\
3912	11433.1\\
3913	11433.1\\
3914	11433.1\\
3915	11433.1\\
3916	11783.1\\
3917	11783.1\\
3918	11625.6\\
3919	11625.6\\
3920	11625.6\\
3921	11625.6\\
3922	11625.6\\
3923	11625.6\\
3924	11625.6\\
3925	11625.6\\
3926	11625.6\\
3927	11625.6\\
3928	11625.6\\
3929	11625.6\\
3930	11625.6\\
3931	11625.6\\
3932	11625.6\\
3933	11625.6\\
3934	11625.6\\
3935	11625.6\\
3936	11625.6\\
3937	11625.6\\
3938	11193.1\\
3939	11193.1\\
3940	11193.1\\
3941	11193.1\\
3942	11193.1\\
3943	11193.1\\
3944	11193.1\\
3945	11193.1\\
3946	11193.1\\
3947	11193.1\\
3948	11193.1\\
3949	11193.1\\
3950	11193.1\\
3951	11193.1\\
3952	11193.1\\
3953	11193.1\\
3954	11193.1\\
3955	11193.1\\
3956	11193.1\\
3957	11193.1\\
3958	11193.1\\
3959	11193.1\\
3960	11193.1\\
3961	11350.6\\
3962	11350.6\\
3963	11350.6\\
3964	11350.6\\
3965	11350.6\\
3966	11350.6\\
3967	11350.6\\
3968	11350.6\\
3969	11350.6\\
3970	11350.6\\
3971	11350.6\\
3972	11350.6\\
3973	11350.6\\
3974	11350.6\\
3975	11350.6\\
3976	11350.6\\
3977	11350.6\\
3978	11350.6\\
3979	11350.6\\
3980	11350.6\\
3981	11350.6\\
3982	11350.6\\
3983	11350.6\\
3984	11135.6\\
3985	11600.6\\
3986	11600.6\\
3987	11600.6\\
3988	11600.6\\
3989	11600.6\\
3990	11600.6\\
3991	11600.6\\
3992	11600.6\\
3993	11600.6\\
3994	11600.6\\
3995	11600.6\\
3996	11600.6\\
3997	11600.6\\
3998	11600.6\\
3999	11600.6\\
4000	11600.6\\
4001	11600.6\\
};
\addplot [color=mycolor1,line width=1.3pt,solid,forget plot]
  table[row sep=crcr]{%
4001	11600.6\\
4002	11600.6\\
4003	11600.6\\
4004	11600.6\\
4005	11600.6\\
4006	11600.6\\
4007	11600.6\\
4008	11600.6\\
4009	11365.6\\
4010	11365.6\\
4011	11210.6\\
4012	11210.6\\
4013	11210.6\\
4014	11365.6\\
4015	11365.6\\
4016	11365.6\\
4017	11365.6\\
4018	11626.6\\
4019	11626.6\\
4020	11976.6\\
4021	11976.6\\
4022	11976.6\\
4023	11976.6\\
4024	11976.6\\
4025	11976.6\\
4026	11976.6\\
4027	11715.6\\
4028	11715.6\\
4029	11715.6\\
4030	11715.6\\
4031	11715.6\\
4032	11715.6\\
4033	11950.6\\
4034	11950.6\\
4035	11950.6\\
4036	11950.6\\
4037	11950.6\\
4038	11950.6\\
4039	11950.6\\
4040	11950.6\\
4041	11950.6\\
4042	11950.6\\
4043	11950.6\\
4044	11950.6\\
4045	12325.6\\
4046	12325.6\\
4047	12325.6\\
4048	12325.6\\
4049	12325.6\\
4050	12325.6\\
4051	12325.6\\
4052	12325.6\\
4053	12325.6\\
4054	12325.6\\
4055	12325.6\\
4056	12325.6\\
4057	12325.6\\
4058	12325.6\\
4059	12325.6\\
4060	12325.6\\
4061	12325.6\\
4062	12325.6\\
4063	12325.6\\
4064	12325.6\\
4065	12325.6\\
4066	12325.6\\
4067	12325.6\\
4068	12325.6\\
4069	12325.6\\
4070	12325.6\\
4071	12325.6\\
4072	12325.6\\
4073	12325.6\\
4074	12325.6\\
4075	12325.6\\
4076	12325.6\\
4077	12325.6\\
4078	12325.6\\
4079	12325.6\\
4080	12325.6\\
4081	12325.6\\
4082	12325.6\\
4083	12325.6\\
4084	12325.6\\
4085	12325.6\\
4086	12325.6\\
4087	12325.6\\
4088	12325.6\\
4089	12325.6\\
4090	12325.6\\
4091	12325.6\\
4092	12325.6\\
4093	12325.6\\
4094	11893.1\\
4095	11893.1\\
4096	11893.1\\
4097	11893.1\\
4098	11893.1\\
4099	11893.1\\
4100	11893.1\\
4101	11893.1\\
4102	11893.1\\
4103	11893.1\\
4104	11893.1\\
4105	11893.1\\
4106	11893.1\\
4107	11893.1\\
4108	11893.1\\
4109	11893.1\\
4110	11893.1\\
4111	11893.1\\
4112	11399.1\\
4113	11399.1\\
4114	11399.1\\
4115	11749.1\\
4116	11749.1\\
4117	11749.1\\
4118	11749.1\\
4119	11749.1\\
4120	11749.1\\
4121	11749.1\\
4122	11749.1\\
4123	11749.1\\
4124	11749.1\\
4125	11749.1\\
4126	11749.1\\
4127	11749.1\\
4128	11749.1\\
4129	11749.1\\
4130	11749.1\\
4131	11749.1\\
4132	11749.1\\
4133	11749.1\\
4134	11749.1\\
4135	11749.1\\
4136	11749.1\\
4137	11749.1\\
4138	11749.1\\
4139	11749.1\\
4140	11749.1\\
4141	11749.1\\
4142	11749.1\\
4143	11749.1\\
4144	11749.1\\
4145	11749.1\\
4146	11749.1\\
4147	11749.1\\
4148	11749.1\\
4149	11749.1\\
4150	11749.1\\
4151	11749.1\\
4152	11749.1\\
4153	11749.1\\
4154	11749.1\\
4155	11749.1\\
4156	11749.1\\
4157	11749.1\\
4158	11749.1\\
4159	11749.1\\
4160	11749.1\\
4161	11749.1\\
4162	11749.1\\
4163	11749.1\\
4164	11749.1\\
4165	11749.1\\
4166	11749.1\\
4167	11749.1\\
4168	11749.1\\
4169	11749.1\\
4170	11749.1\\
4171	11749.1\\
4172	11749.1\\
4173	11749.1\\
4174	11749.1\\
4175	11749.1\\
4176	11749.1\\
4177	11749.1\\
4178	11749.1\\
4179	11749.1\\
4180	11749.1\\
4181	11749.1\\
4182	11749.1\\
4183	11749.1\\
4184	11749.1\\
4185	11518.1\\
4186	11518.1\\
4187	11518.1\\
4188	11518.1\\
4189	11518.1\\
4190	11518.1\\
4191	11518.1\\
4192	11518.1\\
4193	11518.1\\
4194	11518.1\\
4195	11518.1\\
4196	11518.1\\
4197	11518.1\\
4198	11518.1\\
4199	11518.1\\
4200	11518.1\\
4201	11518.1\\
4202	11950.6\\
4203	11950.6\\
4204	11950.6\\
4205	11950.6\\
4206	11950.6\\
4207	11950.6\\
4208	11950.6\\
4209	11950.6\\
4210	11950.6\\
4211	11950.6\\
4212	11950.6\\
4213	11950.6\\
4214	11950.6\\
4215	11950.6\\
4216	11950.6\\
4217	11950.6\\
4218	11950.6\\
4219	11950.6\\
4220	11950.6\\
4221	11950.6\\
4222	11950.6\\
4223	11950.6\\
4224	11950.6\\
4225	11950.6\\
4226	11950.6\\
4227	11950.6\\
4228	11835.6\\
4229	11835.6\\
4230	11835.6\\
4231	11835.6\\
4232	11835.6\\
4233	11835.6\\
4234	11835.6\\
4235	11835.6\\
4236	11835.6\\
4237	11835.6\\
4238	11835.6\\
4239	11835.6\\
4240	11835.6\\
4241	11835.6\\
4242	11835.6\\
4243	11835.6\\
4244	11835.6\\
4245	11835.6\\
4246	11835.6\\
4247	11835.6\\
4248	11835.6\\
4249	11835.6\\
4250	11835.6\\
4251	11835.6\\
4252	11835.6\\
4253	11835.6\\
4254	11835.6\\
4255	11835.6\\
4256	11835.6\\
4257	11835.6\\
4258	11485.6\\
4259	11485.6\\
4260	11485.6\\
4261	11485.6\\
4262	11835.6\\
4263	12210.6\\
4264	12210.6\\
4265	12210.6\\
4266	12210.6\\
4267	12210.6\\
4268	12210.6\\
4269	12210.6\\
4270	12210.6\\
4271	12210.6\\
4272	12210.6\\
4273	12210.6\\
4274	12210.6\\
4275	12210.6\\
4276	12210.6\\
4277	12210.6\\
4278	12210.6\\
4279	12210.6\\
4280	12210.6\\
4281	12210.6\\
4282	12210.6\\
4283	12210.6\\
4284	12210.6\\
4285	12325.6\\
4286	12325.6\\
4287	12325.6\\
4288	12325.6\\
4289	12325.6\\
4290	12325.6\\
4291	12325.6\\
4292	12325.6\\
4293	12325.6\\
4294	12325.6\\
4295	12325.6\\
4296	12325.6\\
4297	12325.6\\
4298	12325.6\\
4299	12325.6\\
4300	12325.6\\
4301	11950.6\\
4302	11950.6\\
4303	11950.6\\
4304	12325.6\\
4305	12325.6\\
4306	12325.6\\
4307	12325.6\\
4308	11950.6\\
4309	11950.6\\
4310	11950.6\\
4311	11950.6\\
4312	11950.6\\
4313	11950.6\\
4314	11950.6\\
4315	11950.6\\
4316	11950.6\\
4317	11950.6\\
4318	11950.6\\
4319	11950.6\\
4320	11950.6\\
4321	11950.6\\
4322	11950.6\\
4323	11950.6\\
4324	11950.6\\
4325	11950.6\\
4326	11950.6\\
4327	11950.6\\
4328	11950.6\\
4329	11950.6\\
4330	11950.6\\
4331	11950.6\\
4332	11950.6\\
4333	11950.6\\
4334	11950.6\\
4335	11950.6\\
4336	11950.6\\
4337	11950.6\\
4338	11950.6\\
4339	12165.6\\
4340	12165.6\\
4341	12165.6\\
4342	12165.6\\
4343	12165.6\\
4344	12165.6\\
4345	12165.6\\
4346	12165.6\\
4347	12165.6\\
4348	12165.6\\
4349	12165.6\\
4350	12165.6\\
4351	12165.6\\
4352	12165.6\\
4353	12165.6\\
4354	12165.6\\
4355	12165.6\\
4356	12165.6\\
4357	12165.6\\
4358	12165.6\\
4359	12165.6\\
4360	12165.6\\
4361	12165.6\\
4362	12165.6\\
4363	12540.6\\
4364	12540.6\\
4365	12973.1\\
4366	12973.1\\
4367	12973.1\\
4368	12973.1\\
4369	12973.1\\
4370	12973.1\\
4371	12973.1\\
4372	12973.1\\
4373	12973.1\\
4374	12973.1\\
4375	12973.1\\
4376	12973.1\\
4377	12973.1\\
4378	12973.1\\
4379	12973.1\\
4380	12973.1\\
4381	12973.1\\
4382	12973.1\\
4383	12973.1\\
4384	12973.1\\
4385	12973.1\\
4386	12973.1\\
4387	12973.1\\
4388	12973.1\\
4389	12973.1\\
4390	12973.1\\
4391	12973.1\\
4392	12973.1\\
4393	12973.1\\
4394	12973.1\\
4395	12973.1\\
4396	12973.1\\
4397	12973.1\\
4398	12973.1\\
4399	12973.1\\
4400	12973.1\\
4401	12973.1\\
4402	12973.1\\
4403	12973.1\\
4404	12973.1\\
4405	12973.1\\
4406	12973.1\\
4407	12973.1\\
4408	12973.1\\
4409	12973.1\\
4410	12973.1\\
4411	12973.1\\
4412	12973.1\\
4413	12973.1\\
4414	12973.1\\
4415	12973.1\\
4416	12973.1\\
4417	12973.1\\
4418	12973.1\\
4419	12973.1\\
4420	12973.1\\
4421	12973.1\\
4422	12973.1\\
4423	12973.1\\
4424	12973.1\\
4425	12973.1\\
4426	12973.1\\
4427	12973.1\\
4428	12973.1\\
4429	12973.1\\
4430	12973.1\\
4431	12973.1\\
4432	12973.1\\
4433	12973.1\\
4434	12973.1\\
4435	12973.1\\
4436	12973.1\\
4437	12973.1\\
4438	12973.1\\
4439	12973.1\\
4440	12973.1\\
4441	12973.1\\
4442	12973.1\\
4443	12973.1\\
4444	12973.1\\
4445	12973.1\\
4446	12973.1\\
4447	12973.1\\
4448	12973.1\\
4449	12973.1\\
4450	12973.1\\
4451	12973.1\\
4452	12973.1\\
4453	12973.1\\
4454	12973.1\\
4455	12973.1\\
4456	12973.1\\
4457	12973.1\\
4458	12973.1\\
4459	12973.1\\
4460	12973.1\\
4461	12973.1\\
4462	12973.1\\
4463	12973.1\\
4464	12973.1\\
4465	12973.1\\
4466	12973.1\\
4467	12973.1\\
4468	12973.1\\
4469	12973.1\\
4470	12973.1\\
4471	12973.1\\
4472	12973.1\\
4473	12973.1\\
4474	12973.1\\
4475	12973.1\\
4476	12973.1\\
4477	12973.1\\
4478	12973.1\\
4479	12973.1\\
4480	12973.1\\
4481	12973.1\\
4482	12973.1\\
4483	12973.1\\
4484	12973.1\\
4485	12973.1\\
4486	12973.1\\
4487	12973.1\\
4488	12973.1\\
4489	12973.1\\
4490	12973.1\\
4491	12973.1\\
4492	12973.1\\
4493	12973.1\\
4494	12973.1\\
4495	12973.1\\
4496	12973.1\\
4497	12973.1\\
4498	12973.1\\
4499	12973.1\\
4500	12973.1\\
4501	12973.1\\
4502	12973.1\\
4503	12973.1\\
4504	12973.1\\
4505	12973.1\\
4506	12973.1\\
4507	12973.1\\
4508	12973.1\\
4509	12973.1\\
4510	12973.1\\
4511	12973.1\\
4512	12973.1\\
4513	12973.1\\
4514	12973.1\\
4515	12973.1\\
4516	12973.1\\
4517	12973.1\\
4518	12973.1\\
4519	12973.1\\
4520	12973.1\\
4521	12973.1\\
4522	12973.1\\
4523	12973.1\\
4524	12973.1\\
4525	12973.1\\
4526	12973.1\\
4527	12973.1\\
4528	12973.1\\
4529	12973.1\\
4530	12973.1\\
4531	12973.1\\
4532	12973.1\\
4533	12973.1\\
4534	12973.1\\
4535	12973.1\\
4536	12973.1\\
4537	12973.1\\
4538	12973.1\\
4539	12973.1\\
4540	12973.1\\
4541	12973.1\\
4542	12973.1\\
4543	12973.1\\
4544	12973.1\\
4545	12973.1\\
4546	12973.1\\
4547	12973.1\\
4548	12973.1\\
4549	12973.1\\
4550	12973.1\\
4551	12973.1\\
4552	12973.1\\
4553	12973.1\\
4554	12973.1\\
4555	12973.1\\
4556	12973.1\\
4557	12973.1\\
4558	12973.1\\
4559	12973.1\\
4560	12973.1\\
4561	12973.1\\
4562	12973.1\\
4563	12973.1\\
4564	12973.1\\
4565	12973.1\\
4566	12973.1\\
4567	12973.1\\
4568	12973.1\\
4569	12973.1\\
4570	12973.1\\
4571	12973.1\\
4572	12872.1\\
4573	12872.1\\
4574	12872.1\\
4575	12872.1\\
4576	12973.1\\
4577	12973.1\\
4578	12973.1\\
4579	12973.1\\
4580	12973.1\\
4581	12973.1\\
4582	12973.1\\
4583	12973.1\\
4584	12973.1\\
4585	12973.1\\
4586	12973.1\\
4587	12973.1\\
4588	12973.1\\
4589	12973.1\\
4590	12973.1\\
4591	12973.1\\
4592	12973.1\\
4593	12973.1\\
4594	12973.1\\
4595	12973.1\\
4596	12973.1\\
4597	12973.1\\
4598	12973.1\\
4599	12973.1\\
4600	12973.1\\
4601	12973.1\\
4602	12973.1\\
4603	12973.1\\
4604	12973.1\\
4605	12973.1\\
4606	12973.1\\
4607	12536.1\\
4608	12536.1\\
4609	12536.1\\
4610	12536.1\\
4611	12536.1\\
4612	12536.1\\
4613	12536.1\\
4614	12536.1\\
4615	12536.1\\
4616	12536.1\\
4617	12536.1\\
4618	12536.1\\
4619	12536.1\\
4620	12536.1\\
4621	12536.1\\
4622	12536.1\\
4623	12536.1\\
4624	12536.1\\
4625	12536.1\\
4626	12536.1\\
4627	12536.1\\
4628	12536.1\\
4629	12536.1\\
4630	12536.1\\
4631	12536.1\\
4632	12536.1\\
4633	12536.1\\
4634	12536.1\\
4635	12536.1\\
4636	12536.1\\
4637	12536.1\\
4638	12536.1\\
4639	12536.1\\
4640	12536.1\\
4641	12536.1\\
4642	12536.1\\
4643	12536.1\\
4644	12536.1\\
4645	12536.1\\
4646	12536.1\\
4647	12536.1\\
4648	12536.1\\
4649	12536.1\\
4650	12536.1\\
4651	12536.1\\
4652	12536.1\\
4653	12536.1\\
4654	12536.1\\
4655	12536.1\\
4656	12536.1\\
4657	12973.1\\
4658	12973.1\\
4659	12973.1\\
4660	12973.1\\
4661	12973.1\\
4662	12973.1\\
4663	12973.1\\
4664	12973.1\\
4665	12973.1\\
4666	12598.1\\
4667	12598.1\\
4668	12598.1\\
4669	12598.1\\
4670	12598.1\\
4671	12598.1\\
4672	12598.1\\
4673	12598.1\\
4674	12598.1\\
4675	12598.1\\
4676	12598.1\\
4677	12598.1\\
4678	12598.1\\
4679	12598.1\\
4680	12598.1\\
4681	12598.1\\
4682	12598.1\\
4683	12598.1\\
4684	12598.1\\
4685	12598.1\\
4686	12598.1\\
4687	12598.1\\
4688	12598.1\\
4689	12598.1\\
4690	12598.1\\
4691	12598.1\\
4692	12598.1\\
4693	12598.1\\
4694	12598.1\\
4695	12598.1\\
4696	12598.1\\
4697	12598.1\\
4698	12598.1\\
4699	12598.1\\
4700	12598.1\\
4701	12598.1\\
4702	12598.1\\
4703	12138.1\\
4704	12138.1\\
4705	12138.1\\
4706	12138.1\\
4707	12138.1\\
4708	12138.1\\
4709	12138.1\\
4710	12138.1\\
4711	12138.1\\
4712	12138.1\\
4713	12138.1\\
4714	12138.1\\
4715	12138.1\\
4716	12138.1\\
4717	12138.1\\
4718	12138.1\\
4719	12138.1\\
4720	12138.1\\
4721	12138.1\\
4722	12138.1\\
4723	12138.1\\
4724	12513.1\\
4725	12513.1\\
4726	12513.1\\
4727	12513.1\\
4728	12513.1\\
4729	12513.1\\
4730	12513.1\\
4731	12513.1\\
4732	12513.1\\
4733	12513.1\\
4734	12513.1\\
4735	12513.1\\
4736	12513.1\\
4737	12513.1\\
4738	12513.1\\
4739	12513.1\\
4740	12513.1\\
4741	12513.1\\
4742	12513.1\\
4743	12513.1\\
4744	12513.1\\
4745	12513.1\\
4746	12513.1\\
4747	12513.1\\
4748	12513.1\\
4749	12513.1\\
4750	12513.1\\
4751	12513.1\\
4752	12513.1\\
4753	12513.1\\
4754	12513.1\\
4755	12513.1\\
4756	12513.1\\
4757	12513.1\\
4758	12513.1\\
4759	12513.1\\
4760	12513.1\\
4761	12513.1\\
4762	12513.1\\
4763	12513.1\\
4764	12513.1\\
4765	12513.1\\
4766	12513.1\\
4767	12513.1\\
4768	12513.1\\
4769	12513.1\\
4770	12513.1\\
4771	12513.1\\
4772	12513.1\\
4773	12513.1\\
4774	12513.1\\
4775	12513.1\\
4776	12513.1\\
4777	12973.1\\
4778	12973.1\\
4779	12973.1\\
4780	12973.1\\
4781	12973.1\\
4782	12973.1\\
4783	12973.1\\
4784	12973.1\\
4785	12973.1\\
4786	12973.1\\
4787	12973.1\\
4788	12973.1\\
4789	12973.1\\
4790	12973.1\\
4791	12973.1\\
4792	12973.1\\
4793	12973.1\\
4794	12973.1\\
4795	12973.1\\
4796	12973.1\\
4797	12973.1\\
4798	12973.1\\
4799	12973.1\\
4800	12973.1\\
4801	12973.1\\
4802	12973.1\\
4803	12973.1\\
4804	12973.1\\
4805	12973.1\\
4806	12973.1\\
4807	12973.1\\
4808	12973.1\\
4809	12973.1\\
4810	12973.1\\
4811	12973.1\\
4812	12973.1\\
4813	12973.1\\
4814	12973.1\\
4815	12973.1\\
4816	12973.1\\
4817	12973.1\\
4818	12973.1\\
4819	12973.1\\
4820	12973.1\\
4821	12973.1\\
4822	12973.1\\
4823	12973.1\\
4824	12973.1\\
4825	12973.1\\
4826	12973.1\\
4827	12973.1\\
4828	12973.1\\
4829	12973.1\\
4830	12973.1\\
4831	12973.1\\
4832	12973.1\\
4833	12973.1\\
4834	12973.1\\
4835	12973.1\\
4836	12973.1\\
4837	12973.1\\
4838	12973.1\\
4839	12973.1\\
4840	12973.1\\
4841	12973.1\\
4842	12973.1\\
4843	12973.1\\
4844	12973.1\\
4845	12973.1\\
4846	12973.1\\
4847	12973.1\\
4848	12973.1\\
4849	12973.1\\
4850	12973.1\\
4851	12973.1\\
4852	12973.1\\
4853	12973.1\\
4854	12973.1\\
4855	12973.1\\
4856	12973.1\\
4857	12973.1\\
4858	12973.1\\
4859	12973.1\\
4860	12973.1\\
4861	12973.1\\
4862	12973.1\\
4863	12973.1\\
4864	12973.1\\
4865	12973.1\\
4866	12973.1\\
4867	12973.1\\
4868	12973.1\\
4869	12973.1\\
4870	12973.1\\
4871	12973.1\\
4872	12973.1\\
4873	12973.1\\
4874	12973.1\\
4875	12973.1\\
4876	12973.1\\
4877	12973.1\\
4878	12973.1\\
4879	12973.1\\
4880	12973.1\\
4881	12973.1\\
4882	12973.1\\
4883	12973.1\\
4884	12973.1\\
4885	12973.1\\
4886	12973.1\\
4887	12973.1\\
4888	12973.1\\
4889	12973.1\\
4890	12973.1\\
4891	12973.1\\
4892	12973.1\\
4893	12973.1\\
4894	12973.1\\
4895	12973.1\\
4896	12973.1\\
4897	12973.1\\
4898	12973.1\\
4899	12973.1\\
4900	12973.1\\
4901	12973.1\\
4902	12973.1\\
4903	12973.1\\
4904	12973.1\\
4905	12973.1\\
4906	12973.1\\
4907	12973.1\\
4908	12973.1\\
4909	12973.1\\
4910	12973.1\\
4911	12973.1\\
4912	12973.1\\
4913	12973.1\\
4914	12973.1\\
4915	12973.1\\
4916	12973.1\\
4917	12973.1\\
4918	12973.1\\
4919	12973.1\\
4920	12973.1\\
4921	12973.1\\
4922	12973.1\\
4923	12973.1\\
4924	12973.1\\
4925	12973.1\\
4926	12973.1\\
4927	12973.1\\
4928	12973.1\\
4929	12973.1\\
4930	12973.1\\
4931	12973.1\\
4932	12973.1\\
4933	12973.1\\
4934	12973.1\\
4935	12973.1\\
4936	12973.1\\
4937	12973.1\\
4938	12973.1\\
4939	12973.1\\
4940	12973.1\\
4941	12973.1\\
4942	12973.1\\
4943	12973.1\\
4944	12973.1\\
4945	12973.1\\
4946	12973.1\\
4947	12973.1\\
4948	12973.1\\
4949	12973.1\\
4950	12973.1\\
4951	12973.1\\
4952	12973.1\\
4953	12858.1\\
4954	12858.1\\
4955	12858.1\\
4956	12858.1\\
4957	12858.1\\
4958	12858.1\\
4959	12858.1\\
4960	12858.1\\
4961	12858.1\\
4962	12858.1\\
4963	12858.1\\
4964	12858.1\\
4965	12858.1\\
4966	12858.1\\
4967	12973.1\\
4968	12973.1\\
4969	12973.1\\
4970	12973.1\\
4971	12973.1\\
4972	12973.1\\
4973	12973.1\\
4974	12973.1\\
4975	12973.1\\
4976	12973.1\\
4977	12973.1\\
4978	12973.1\\
4979	12973.1\\
4980	12973.1\\
4981	12973.1\\
4982	12973.1\\
4983	12973.1\\
4984	12973.1\\
4985	12973.1\\
4986	12973.1\\
4987	12973.1\\
4988	12973.1\\
4989	12973.1\\
4990	12973.1\\
4991	12973.1\\
4992	12973.1\\
4993	12973.1\\
4994	12973.1\\
4995	12973.1\\
4996	12973.1\\
4997	12973.1\\
4998	12973.1\\
4999	12973.1\\
5000	12973.1\\
5001	12973.1\\
5002	12973.1\\
5003	12973.1\\
5004	12973.1\\
5005	12973.1\\
5006	12973.1\\
5007	12973.1\\
5008	12973.1\\
5009	12973.1\\
5010	12973.1\\
5011	12973.1\\
5012	12973.1\\
5013	12973.1\\
5014	12973.1\\
5015	12973.1\\
5016	12973.1\\
5017	12973.1\\
5018	12973.1\\
5019	12973.1\\
5020	12973.1\\
5021	12973.1\\
5022	12973.1\\
5023	12973.1\\
5024	12973.1\\
5025	12973.1\\
5026	12973.1\\
5027	12973.1\\
5028	12973.1\\
5029	12973.1\\
5030	12973.1\\
5031	12973.1\\
5032	12973.1\\
5033	12973.1\\
5034	12973.1\\
5035	12973.1\\
5036	12973.1\\
5037	12973.1\\
5038	12973.1\\
5039	12973.1\\
5040	12973.1\\
5041	12973.1\\
5042	12973.1\\
5043	12973.1\\
5044	12973.1\\
5045	12973.1\\
5046	12973.1\\
5047	12973.1\\
5048	12973.1\\
5049	12973.1\\
5050	12973.1\\
5051	12973.1\\
5052	12973.1\\
5053	12973.1\\
5054	12973.1\\
5055	12973.1\\
5056	12973.1\\
5057	12973.1\\
5058	12973.1\\
5059	12973.1\\
5060	12973.1\\
5061	12973.1\\
5062	12973.1\\
5063	12973.1\\
5064	12973.1\\
5065	12973.1\\
5066	12973.1\\
5067	12973.1\\
5068	12973.1\\
5069	12973.1\\
5070	12973.1\\
5071	12973.1\\
5072	12973.1\\
5073	12973.1\\
5074	12973.1\\
5075	12973.1\\
5076	12973.1\\
5077	12973.1\\
5078	12973.1\\
5079	12973.1\\
5080	12973.1\\
5081	12973.1\\
5082	12973.1\\
5083	12973.1\\
5084	12973.1\\
5085	12973.1\\
5086	12973.1\\
5087	12973.1\\
5088	12973.1\\
5089	12973.1\\
5090	12973.1\\
5091	12973.1\\
5092	12973.1\\
5093	12973.1\\
5094	12973.1\\
5095	12973.1\\
5096	12973.1\\
5097	12973.1\\
5098	12973.1\\
5099	12973.1\\
5100	12973.1\\
5101	12973.1\\
5102	12973.1\\
5103	12973.1\\
5104	12973.1\\
5105	12973.1\\
5106	12973.1\\
5107	12973.1\\
5108	12973.1\\
5109	12973.1\\
5110	12973.1\\
5111	12973.1\\
5112	12973.1\\
5113	12973.1\\
5114	12973.1\\
5115	12973.1\\
5116	12973.1\\
5117	12973.1\\
5118	12973.1\\
5119	12973.1\\
5120	12973.1\\
5121	12973.1\\
5122	12973.1\\
5123	12973.1\\
5124	12677.3\\
5125	12677.3\\
5126	12677.3\\
5127	12677.3\\
5128	12677.3\\
5129	12677.3\\
5130	12677.3\\
5131	12677.3\\
5132	12677.3\\
5133	12677.3\\
5134	12677.3\\
5135	12677.3\\
5136	12677.3\\
5137	12677.3\\
5138	12677.3\\
5139	12677.3\\
5140	12677.3\\
5141	12677.3\\
5142	12677.3\\
5143	12677.3\\
5144	12677.3\\
5145	12973.1\\
5146	12973.1\\
5147	12973.1\\
5148	12973.1\\
5149	12973.1\\
5150	12973.1\\
5151	12973.1\\
5152	12973.1\\
5153	12973.1\\
5154	12973.1\\
5155	12973.1\\
5156	12973.1\\
5157	12973.1\\
5158	12973.1\\
5159	12973.1\\
5160	12973.1\\
5161	12973.1\\
5162	12973.1\\
5163	12973.1\\
5164	12973.1\\
5165	12973.1\\
5166	12623.1\\
5167	12623.1\\
5168	12623.1\\
5169	12623.1\\
5170	12623.1\\
5171	12973.1\\
5172	12973.1\\
5173	12973.1\\
5174	12973.1\\
5175	12973.1\\
5176	12973.1\\
5177	12973.1\\
5178	12973.1\\
5179	12973.1\\
5180	12973.1\\
5181	12973.1\\
5182	12973.1\\
5183	12973.1\\
5184	12973.1\\
5185	12973.1\\
5186	12973.1\\
5187	12973.1\\
5188	12973.1\\
5189	12973.1\\
5190	12973.1\\
5191	12973.1\\
5192	12973.1\\
5193	12973.1\\
5194	12973.1\\
5195	12973.1\\
5196	11935.1\\
5197	11935.1\\
5198	11935.1\\
5199	11935.1\\
5200	11935.1\\
5201	11935.1\\
5202	11935.1\\
5203	11935.1\\
5204	11935.1\\
5205	11935.1\\
5206	11935.1\\
5207	11935.1\\
5208	11935.1\\
5209	11935.1\\
5210	11935.1\\
5211	11935.1\\
5212	11935.1\\
5213	11935.1\\
5214	11935.1\\
5215	11935.1\\
5216	11935.1\\
5217	11935.1\\
5218	11935.1\\
5219	11935.1\\
5220	11935.1\\
5221	11935.1\\
5222	11935.1\\
5223	11935.1\\
5224	11935.1\\
5225	11935.1\\
5226	11935.1\\
5227	11935.1\\
5228	11935.1\\
5229	11935.1\\
5230	11935.1\\
5231	11935.1\\
5232	11935.1\\
5233	11935.1\\
5234	11935.1\\
5235	11935.1\\
5236	11935.1\\
5237	11935.1\\
5238	11935.1\\
5239	11935.1\\
5240	11935.1\\
5241	11935.1\\
5242	11935.1\\
5243	11935.1\\
5244	11935.1\\
5245	11935.1\\
5246	11935.1\\
5247	11935.1\\
5248	11935.1\\
5249	11935.1\\
5250	11935.1\\
5251	11935.1\\
5252	11935.1\\
5253	11935.1\\
5254	11935.1\\
5255	11935.1\\
5256	11935.1\\
5257	11935.1\\
5258	11935.1\\
5259	11935.1\\
5260	11935.1\\
5261	11935.1\\
5262	11935.1\\
5263	11935.1\\
5264	11935.1\\
5265	11935.1\\
5266	11935.1\\
5267	11935.1\\
5268	11935.1\\
5269	11935.1\\
5270	11935.1\\
5271	11935.1\\
5272	11935.1\\
5273	11935.1\\
5274	11935.1\\
5275	11935.1\\
5276	11935.1\\
5277	11935.1\\
5278	11935.1\\
5279	11935.1\\
5280	11935.1\\
5281	11935.1\\
5282	11935.1\\
5283	11935.1\\
5284	11935.1\\
5285	11935.1\\
5286	11935.1\\
5287	11935.1\\
5288	11935.1\\
5289	11935.1\\
5290	11935.1\\
5291	11935.1\\
5292	11935.1\\
5293	11935.1\\
5294	11935.1\\
5295	11935.1\\
5296	11935.1\\
5297	11935.1\\
5298	11935.1\\
5299	11935.1\\
5300	11935.1\\
5301	11935.1\\
5302	11935.1\\
5303	11935.1\\
5304	11935.1\\
5305	11935.1\\
5306	11935.1\\
5307	11935.1\\
5308	11935.1\\
5309	11935.1\\
5310	11935.1\\
5311	11935.1\\
5312	11935.1\\
5313	11935.1\\
5314	11935.1\\
5315	11935.1\\
5316	11935.1\\
5317	11935.1\\
5318	11935.1\\
5319	11935.1\\
5320	11935.1\\
5321	11935.1\\
5322	11935.1\\
5323	11935.1\\
5324	11935.1\\
5325	11935.1\\
5326	11935.1\\
5327	11935.1\\
5328	11935.1\\
5329	11935.1\\
5330	11935.1\\
5331	11935.1\\
5332	11935.1\\
5333	11935.1\\
5334	11935.1\\
5335	11935.1\\
5336	11935.1\\
5337	11935.1\\
5338	11935.1\\
5339	11935.1\\
5340	11935.1\\
5341	11935.1\\
5342	11935.1\\
5343	11935.1\\
5344	11935.1\\
5345	11935.1\\
5346	11935.1\\
5347	11935.1\\
5348	11935.1\\
5349	11935.1\\
5350	11935.1\\
5351	11935.1\\
5352	11935.1\\
5353	11935.1\\
5354	11935.1\\
5355	11935.1\\
5356	11935.1\\
5357	11935.1\\
5358	11935.1\\
5359	11935.1\\
5360	11935.1\\
5361	11935.1\\
5362	11935.1\\
5363	11935.1\\
5364	11935.1\\
5365	11935.1\\
5366	11935.1\\
5367	11935.1\\
5368	11935.1\\
5369	11935.1\\
5370	11935.1\\
5371	11935.1\\
5372	11935.1\\
5373	11935.1\\
5374	11935.1\\
5375	11935.1\\
5376	11935.1\\
5377	11935.1\\
5378	11935.1\\
5379	11935.1\\
5380	11935.1\\
5381	11935.1\\
5382	11935.1\\
5383	11935.1\\
5384	11935.1\\
5385	11935.1\\
5386	11935.1\\
5387	11935.1\\
5388	11935.1\\
5389	11935.1\\
5390	11935.1\\
5391	11935.1\\
5392	11935.1\\
5393	11935.1\\
5394	11935.1\\
5395	11935.1\\
5396	11935.1\\
5397	11935.1\\
5398	11935.1\\
5399	11935.1\\
5400	11935.1\\
5401	11935.1\\
5402	11935.1\\
5403	11935.1\\
5404	11935.1\\
5405	11935.1\\
5406	11935.1\\
5407	11935.1\\
5408	11935.1\\
5409	11935.1\\
5410	11720.1\\
5411	11720.1\\
5412	11720.1\\
5413	11720.1\\
5414	11720.1\\
5415	11720.1\\
5416	11720.1\\
5417	11720.1\\
5418	11720.1\\
5419	11720.1\\
5420	11720.1\\
5421	11720.1\\
5422	11720.1\\
5423	11720.1\\
5424	11720.1\\
5425	11720.1\\
5426	11720.1\\
5427	11720.1\\
5428	11720.1\\
5429	11720.1\\
5430	11720.1\\
5431	11720.1\\
5432	11720.1\\
5433	11720.1\\
5434	11720.1\\
5435	11720.1\\
5436	11720.1\\
5437	11720.1\\
5438	11720.1\\
5439	11720.1\\
5440	11720.1\\
5441	11720.1\\
5442	11720.1\\
5443	11720.1\\
5444	11720.1\\
5445	11720.1\\
5446	11720.1\\
5447	11720.1\\
5448	11720.1\\
5449	11720.1\\
5450	11720.1\\
5451	11720.1\\
5452	11720.1\\
5453	11720.1\\
5454	11720.1\\
5455	11720.1\\
5456	11720.1\\
5457	11720.1\\
5458	11720.1\\
5459	11720.1\\
5460	11720.1\\
5461	11720.1\\
5462	11720.1\\
5463	11720.1\\
5464	11720.1\\
5465	11720.1\\
5466	11720.1\\
5467	11720.1\\
5468	11720.1\\
5469	11720.1\\
5470	11720.1\\
5471	11720.1\\
5472	11720.1\\
5473	11720.1\\
5474	11720.1\\
5475	11720.1\\
5476	11720.1\\
5477	11720.1\\
5478	11720.1\\
5479	11720.1\\
5480	11720.1\\
5481	11720.1\\
5482	11720.1\\
5483	11720.1\\
5484	11720.1\\
5485	11720.1\\
5486	11720.1\\
5487	11720.1\\
5488	11720.1\\
5489	11720.1\\
5490	11720.1\\
5491	11720.1\\
5492	11720.1\\
5493	11720.1\\
5494	11720.1\\
5495	11720.1\\
5496	11720.1\\
5497	11720.1\\
5498	11720.1\\
5499	11720.1\\
5500	11720.1\\
5501	11720.1\\
5502	11720.1\\
5503	11720.1\\
5504	11720.1\\
5505	11720.1\\
5506	11720.1\\
5507	11720.1\\
5508	11720.1\\
5509	11720.1\\
5510	11720.1\\
5511	11720.1\\
5512	11720.1\\
5513	11720.1\\
5514	11720.1\\
5515	11720.1\\
5516	11720.1\\
5517	11720.1\\
5518	11720.1\\
5519	11720.1\\
5520	11720.1\\
5521	11720.1\\
5522	11720.1\\
5523	11720.1\\
5524	11720.1\\
5525	11720.1\\
5526	11720.1\\
5527	11720.1\\
5528	11720.1\\
5529	11720.1\\
5530	11720.1\\
5531	11720.1\\
5532	11720.1\\
5533	11720.1\\
5534	11720.1\\
5535	11720.1\\
5536	11720.1\\
5537	11720.1\\
5538	11720.1\\
5539	11720.1\\
5540	11720.1\\
5541	11720.1\\
5542	11720.1\\
5543	11720.1\\
5544	11720.1\\
5545	11720.1\\
5546	11720.1\\
5547	11720.1\\
5548	11720.1\\
5549	11720.1\\
5550	11720.1\\
5551	11720.1\\
5552	11720.1\\
5553	11720.1\\
5554	11720.1\\
5555	11720.1\\
5556	11720.1\\
5557	11720.1\\
5558	11720.1\\
5559	11720.1\\
5560	11720.1\\
5561	11720.1\\
5562	11720.1\\
5563	11720.1\\
5564	11720.1\\
5565	11720.1\\
5566	11720.1\\
5567	11720.1\\
5568	11720.1\\
5569	11720.1\\
5570	11720.1\\
5571	11720.1\\
5572	11720.1\\
5573	11720.1\\
5574	11720.1\\
5575	11720.1\\
5576	11720.1\\
5577	11720.1\\
5578	11720.1\\
5579	11720.1\\
5580	11720.1\\
5581	11720.1\\
5582	11720.1\\
5583	11720.1\\
5584	11720.1\\
5585	11720.1\\
5586	11720.1\\
5587	11720.1\\
5588	11720.1\\
5589	11720.1\\
5590	11720.1\\
5591	11720.1\\
5592	11720.1\\
5593	11720.1\\
5594	11720.1\\
5595	11720.1\\
5596	11720.1\\
5597	11720.1\\
5598	11720.1\\
5599	11720.1\\
5600	11720.1\\
5601	11720.1\\
5602	11720.1\\
5603	11720.1\\
5604	11720.1\\
5605	11720.1\\
5606	11720.1\\
5607	11720.1\\
5608	11720.1\\
5609	11720.1\\
5610	11720.1\\
5611	11720.1\\
5612	11720.1\\
5613	11720.1\\
5614	11720.1\\
5615	11720.1\\
5616	11720.1\\
5617	11720.1\\
5618	10874.3\\
5619	10874.3\\
5620	10874.3\\
5621	10874.3\\
5622	10874.3\\
5623	10874.3\\
5624	10874.3\\
5625	10874.3\\
5626	10874.3\\
5627	10874.3\\
5628	10874.3\\
5629	10874.3\\
5630	10874.3\\
5631	10874.3\\
5632	10874.3\\
5633	10874.3\\
5634	10874.3\\
5635	10874.3\\
5636	10874.3\\
5637	10874.3\\
5638	10874.3\\
5639	10874.3\\
5640	10874.3\\
5641	11720.1\\
5642	11720.1\\
5643	11720.1\\
5644	11720.1\\
5645	11720.1\\
5646	11720.1\\
5647	11720.1\\
5648	11720.1\\
5649	11720.1\\
5650	11720.1\\
5651	11720.1\\
5652	11720.1\\
5653	11720.1\\
5654	11720.1\\
5655	11720.1\\
5656	11720.1\\
5657	11720.1\\
5658	11720.1\\
5659	11720.1\\
5660	11720.1\\
5661	11720.1\\
5662	11720.1\\
5663	11720.1\\
5664	11720.1\\
5665	11720.1\\
5666	11720.1\\
5667	11720.1\\
5668	11720.1\\
5669	11720.1\\
5670	11720.1\\
5671	11720.1\\
5672	11799.1\\
5673	11799.1\\
5674	11799.1\\
5675	11799.1\\
5676	11564.1\\
5677	11564.1\\
5678	11564.1\\
5679	11564.1\\
5680	11564.1\\
5681	11564.1\\
5682	11564.1\\
5683	11564.1\\
5684	11564.1\\
5685	11564.1\\
5686	11564.1\\
5687	11564.1\\
5688	11564.1\\
5689	11564.1\\
5690	11564.1\\
5691	11564.1\\
5692	11564.1\\
5693	11564.1\\
5694	11564.1\\
5695	11564.1\\
5696	11564.1\\
5697	11564.1\\
5698	11564.1\\
5699	11700.1\\
5700	11700.1\\
5701	11700.1\\
5702	11700.1\\
5703	11700.1\\
5704	11700.1\\
5705	11700.1\\
5706	11700.1\\
5707	11700.1\\
5708	11700.1\\
5709	11700.1\\
5710	11700.1\\
5711	11700.1\\
5712	11700.1\\
5713	11700.1\\
5714	11700.1\\
5715	11700.1\\
5716	11700.1\\
5717	11700.1\\
5718	11700.1\\
5719	11700.1\\
5720	11700.1\\
5721	11700.1\\
5722	11700.1\\
5723	11700.1\\
5724	11700.1\\
5725	11700.1\\
5726	11700.1\\
5727	11700.1\\
5728	11700.1\\
5729	11700.1\\
5730	11700.1\\
5731	11700.1\\
5732	11700.1\\
5733	11700.1\\
5734	11700.1\\
5735	11700.1\\
5736	11700.1\\
5737	11700.1\\
5738	11700.1\\
5739	11700.1\\
5740	11700.1\\
5741	11700.1\\
5742	11700.1\\
5743	11700.1\\
5744	11700.1\\
5745	11700.1\\
5746	11700.1\\
5747	11700.1\\
5748	11700.1\\
5749	11700.1\\
5750	11700.1\\
5751	11700.1\\
5752	11700.1\\
5753	11700.1\\
5754	11935.1\\
5755	11935.1\\
5756	11935.1\\
5757	11935.1\\
5758	11935.1\\
5759	11935.1\\
5760	11935.1\\
5761	11935.1\\
5762	11935.1\\
5763	11935.1\\
5764	11935.1\\
5765	11935.1\\
5766	11935.1\\
5767	11935.1\\
5768	11935.1\\
5769	11935.1\\
5770	11935.1\\
5771	11935.1\\
5772	11935.1\\
5773	11935.1\\
5774	11935.1\\
5775	11935.1\\
5776	11935.1\\
5777	11935.1\\
5778	11935.1\\
5779	11935.1\\
5780	11935.1\\
5781	11935.1\\
5782	11935.1\\
5783	11935.1\\
5784	11935.1\\
5785	11935.1\\
5786	10973.1\\
5787	10973.1\\
5788	10973.1\\
5789	10973.1\\
5790	10973.1\\
5791	10973.1\\
5792	10973.1\\
5793	10973.1\\
5794	10973.1\\
5795	10973.1\\
5796	10973.1\\
5797	10973.1\\
5798	10973.1\\
5799	10973.1\\
5800	10973.1\\
5801	10973.1\\
5802	10973.1\\
5803	10973.1\\
5804	10973.1\\
5805	10973.1\\
5806	10973.1\\
5807	10973.1\\
5808	10973.1\\
5809	10973.1\\
5810	10973.1\\
5811	10973.1\\
5812	10973.1\\
5813	10973.1\\
5814	10973.1\\
5815	10973.1\\
5816	10973.1\\
5817	10973.1\\
5818	10973.1\\
5819	10973.1\\
5820	10973.1\\
5821	10973.1\\
5822	10973.1\\
5823	10973.1\\
5824	10973.1\\
5825	10973.1\\
5826	10973.1\\
5827	10973.1\\
5828	10973.1\\
5829	10973.1\\
5830	10973.1\\
5831	10973.1\\
5832	10973.1\\
5833	10973.1\\
5834	10973.1\\
5835	10973.1\\
5836	10973.1\\
5837	10973.1\\
5838	10973.1\\
5839	10973.1\\
5840	10973.1\\
5841	10973.1\\
5842	10973.1\\
5843	10973.1\\
5844	10973.1\\
5845	10973.1\\
5846	10973.1\\
5847	10973.1\\
5848	10973.1\\
5849	10973.1\\
5850	10973.1\\
5851	10973.1\\
5852	10973.1\\
5853	10973.1\\
5854	10973.1\\
5855	10973.1\\
5856	10973.1\\
5857	10973.1\\
5858	10973.1\\
5859	10973.1\\
5860	10973.1\\
5861	10973.1\\
5862	10973.1\\
5863	10973.1\\
5864	10973.1\\
5865	10973.1\\
5866	10973.1\\
5867	10973.1\\
5868	10973.1\\
5869	10973.1\\
5870	10973.1\\
5871	10973.1\\
5872	10973.1\\
5873	10973.1\\
5874	10973.1\\
5875	10973.1\\
5876	10973.1\\
5877	10973.1\\
5878	10973.1\\
5879	10973.1\\
5880	10973.1\\
5881	10973.1\\
5882	10973.1\\
5883	10973.1\\
5884	10973.1\\
5885	10973.1\\
5886	10973.1\\
5887	10973.1\\
5888	10973.1\\
5889	10973.1\\
5890	10973.1\\
5891	10973.1\\
5892	10973.1\\
5893	10973.1\\
5894	10973.1\\
5895	10973.1\\
5896	10973.1\\
5897	10973.1\\
5898	10973.1\\
5899	10973.1\\
5900	10973.1\\
5901	10973.1\\
5902	10973.1\\
5903	10973.1\\
5904	10973.1\\
5905	10973.1\\
5906	10808.1\\
5907	10808.1\\
5908	10808.1\\
5909	10808.1\\
5910	10808.1\\
5911	10808.1\\
5912	10808.1\\
5913	10808.1\\
5914	10808.1\\
5915	10808.1\\
5916	10808.1\\
5917	10808.1\\
5918	10808.1\\
5919	10808.1\\
5920	10808.1\\
5921	10808.1\\
5922	10808.1\\
5923	10808.1\\
5924	10808.1\\
5925	10808.1\\
5926	10808.1\\
5927	10808.1\\
5928	10808.1\\
5929	10808.1\\
5930	10808.1\\
5931	10808.1\\
5932	10808.1\\
5933	10973.1\\
5934	10973.1\\
5935	10973.1\\
5936	10973.1\\
5937	10973.1\\
5938	10973.1\\
5939	10973.1\\
5940	10973.1\\
5941	10973.1\\
5942	10973.1\\
5943	10973.1\\
5944	10973.1\\
5945	10973.1\\
5946	10973.1\\
5947	10973.1\\
5948	10973.1\\
5949	10973.1\\
5950	10973.1\\
5951	10623.1\\
5952	10163.1\\
5953	10163.1\\
5954	10163.1\\
5955	10163.1\\
5956	10163.1\\
5957	10163.1\\
5958	10163.1\\
5959	10163.1\\
5960	10163.1\\
5961	10163.1\\
5962	10163.1\\
5963	10163.1\\
5964	10163.1\\
5965	10163.1\\
5966	10163.1\\
5967	10163.1\\
5968	10163.1\\
5969	10163.1\\
5970	10163.1\\
5971	10163.1\\
5972	10163.1\\
5973	10163.1\\
5974	10163.1\\
5975	10163.1\\
5976	10163.1\\
5977	10163.1\\
5978	9928.1\\
5979	9928.1\\
5980	9928.1\\
5981	9928.1\\
5982	9928.1\\
5983	9928.1\\
5984	9928.1\\
5985	9928.1\\
5986	9928.1\\
5987	9928.1\\
5988	9928.1\\
5989	9928.1\\
5990	9928.1\\
5991	9928.1\\
5992	9928.1\\
5993	10278.1\\
5994	10278.1\\
5995	10278.1\\
5996	10278.1\\
5997	10278.1\\
5998	10278.1\\
5999	10278.1\\
6000	10278.1\\
6001	10278.1\\
6002	10278.1\\
6003	10278.1\\
6004	10278.1\\
6005	10278.1\\
6006	10278.1\\
6007	10278.1\\
6008	10278.1\\
6009	10278.1\\
6010	10278.1\\
6011	10278.1\\
6012	10278.1\\
6013	10278.1\\
6014	10278.1\\
6015	10278.1\\
6016	10278.1\\
6017	10278.1\\
6018	10278.1\\
6019	10278.1\\
6020	10513.1\\
6021	10513.1\\
6022	10513.1\\
6023	10513.1\\
6024	10513.1\\
6025	10513.1\\
6026	10513.1\\
6027	10513.1\\
6028	10513.1\\
6029	10513.1\\
6030	10513.1\\
6031	10513.1\\
6032	10513.1\\
6033	10513.1\\
6034	10513.1\\
6035	10513.1\\
6036	10513.1\\
6037	10513.1\\
6038	10513.1\\
6039	10513.1\\
6040	10513.1\\
6041	10513.1\\
6042	10513.1\\
6043	10513.1\\
6044	10513.1\\
6045	10513.1\\
6046	10513.1\\
6047	10513.1\\
6048	10513.1\\
6049	10513.1\\
6050	10513.1\\
6051	10513.1\\
6052	10513.1\\
6053	10513.1\\
6054	10513.1\\
6055	10513.1\\
6056	10513.1\\
6057	10513.1\\
6058	10513.1\\
6059	10513.1\\
6060	10513.1\\
6061	10513.1\\
6062	10513.1\\
6063	10513.1\\
6064	10513.1\\
6065	10513.1\\
6066	10513.1\\
6067	10513.1\\
6068	10513.1\\
6069	10513.1\\
6070	10513.1\\
6071	10513.1\\
6072	10513.1\\
6073	10513.1\\
6074	10513.1\\
6075	10513.1\\
6076	10513.1\\
6077	10513.1\\
6078	10513.1\\
6079	10513.1\\
6080	10513.1\\
6081	10513.1\\
6082	10513.1\\
6083	10513.1\\
6084	10513.1\\
6085	10513.1\\
6086	10513.1\\
6087	10513.1\\
6088	10513.1\\
6089	10513.1\\
6090	10513.1\\
6091	10513.1\\
6092	10513.1\\
6093	10513.1\\
6094	10513.1\\
6095	10513.1\\
6096	10513.1\\
6097	10513.1\\
6098	10513.1\\
6099	10513.1\\
6100	10513.1\\
6101	10513.1\\
6102	10513.1\\
6103	10513.1\\
6104	10363.1\\
6105	10363.1\\
6106	10363.1\\
6107	10513.1\\
6108	10513.1\\
6109	10513.1\\
6110	10513.1\\
6111	10513.1\\
6112	10513.1\\
6113	10513.1\\
6114	10513.1\\
6115	10513.1\\
6116	10513.1\\
6117	10513.1\\
6118	10163.1\\
6119	10163.1\\
6120	10163.1\\
6121	10163.1\\
6122	10163.1\\
6123	10163.1\\
6124	10163.1\\
6125	10163.1\\
6126	10163.1\\
6127	10163.1\\
6128	10163.1\\
6129	10163.1\\
6130	10163.1\\
6131	10163.1\\
6132	10163.1\\
6133	10163.1\\
6134	10163.1\\
6135	10163.1\\
6136	10163.1\\
6137	10163.1\\
6138	10163.1\\
6139	10163.1\\
6140	10163.1\\
6141	10163.1\\
6142	10163.1\\
6143	10163.1\\
6144	10163.1\\
6145	10163.1\\
6146	10163.1\\
6147	10163.1\\
6148	10163.1\\
6149	10163.1\\
6150	10163.1\\
6151	10163.1\\
6152	10163.1\\
6153	10163.1\\
6154	10163.1\\
6155	10163.1\\
6156	10473.1\\
6157	10473.1\\
6158	10473.1\\
6159	10473.1\\
6160	10473.1\\
6161	10473.1\\
6162	10473.1\\
6163	10473.1\\
6164	10473.1\\
6165	10473.1\\
6166	10473.1\\
6167	10473.1\\
6168	10473.1\\
6169	10473.1\\
6170	10473.1\\
6171	10473.1\\
6172	10473.1\\
6173	10473.1\\
6174	10473.1\\
6175	10473.1\\
6176	10473.1\\
6177	10473.1\\
6178	10473.1\\
6179	10473.1\\
6180	10473.1\\
6181	10473.1\\
6182	10473.1\\
6183	10473.1\\
6184	10473.1\\
6185	10473.1\\
6186	10473.1\\
6187	10473.1\\
6188	10473.1\\
6189	10473.1\\
6190	10473.1\\
6191	10473.1\\
6192	10473.1\\
6193	10473.1\\
6194	10473.1\\
6195	10473.1\\
6196	10473.1\\
6197	10473.1\\
6198	10473.1\\
6199	10473.1\\
6200	10473.1\\
6201	10473.1\\
6202	10473.1\\
6203	10473.1\\
6204	10473.1\\
6205	10473.1\\
6206	10473.1\\
6207	10473.1\\
6208	10473.1\\
6209	10473.1\\
6210	10473.1\\
6211	10473.1\\
6212	10473.1\\
6213	10473.1\\
6214	10473.1\\
6215	10473.1\\
6216	10473.1\\
6217	10473.1\\
6218	10473.1\\
6219	10473.1\\
6220	10473.1\\
6221	10473.1\\
6222	10473.1\\
6223	10473.1\\
6224	10473.1\\
6225	10473.1\\
6226	10473.1\\
6227	10473.1\\
6228	10473.1\\
6229	10473.1\\
6230	10473.1\\
6231	10473.1\\
6232	10473.1\\
6233	10473.1\\
6234	10473.1\\
6235	10473.1\\
6236	10473.1\\
6237	10473.1\\
6238	10473.1\\
6239	10473.1\\
6240	10473.1\\
6241	10623.1\\
6242	10623.1\\
6243	10623.1\\
6244	10623.1\\
6245	10623.1\\
6246	10623.1\\
6247	10623.1\\
6248	10623.1\\
6249	10623.1\\
6250	10623.1\\
6251	10623.1\\
6252	10623.1\\
6253	10623.1\\
6254	10495.1\\
6255	10495.1\\
6256	10495.1\\
6257	10495.1\\
6258	10495.1\\
6259	10495.1\\
6260	10495.1\\
6261	10623.1\\
6262	10623.1\\
6263	10623.1\\
6264	10623.1\\
6265	10623.1\\
6266	10623.1\\
6267	10623.1\\
6268	10623.1\\
6269	10623.1\\
6270	10623.1\\
6271	10623.1\\
6272	10623.1\\
6273	10623.1\\
6274	10623.1\\
6275	10623.1\\
6276	10623.1\\
6277	10505.1\\
6278	10505.1\\
6279	10505.1\\
6280	10505.1\\
6281	10505.1\\
6282	10505.1\\
6283	10505.1\\
6284	10505.1\\
6285	10505.1\\
6286	10505.1\\
6287	10505.1\\
6288	10505.1\\
6289	10505.1\\
6290	10505.1\\
6291	10505.1\\
6292	10505.1\\
6293	10505.1\\
6294	10505.1\\
6295	10505.1\\
6296	10505.1\\
6297	10505.1\\
6298	10505.1\\
6299	10505.1\\
6300	10505.1\\
6301	10505.1\\
6302	10505.1\\
6303	10505.1\\
6304	10505.1\\
6305	10505.1\\
6306	10505.1\\
6307	10505.1\\
6308	10505.1\\
6309	10505.1\\
6310	10505.1\\
6311	10505.1\\
6312	10505.1\\
6313	10505.1\\
6314	10505.1\\
6315	10505.1\\
6316	10505.1\\
6317	10505.1\\
6318	10505.1\\
6319	10505.1\\
6320	10505.1\\
6321	10505.1\\
6322	10505.1\\
6323	10505.1\\
6324	10505.1\\
6325	10505.1\\
6326	10505.1\\
6327	10505.1\\
6328	10505.1\\
6329	10505.1\\
6330	10505.1\\
6331	10505.1\\
6332	10505.1\\
6333	10505.1\\
6334	10505.1\\
6335	10505.1\\
6336	10155.1\\
6337	10155.1\\
6338	10155.1\\
6339	10155.1\\
6340	10155.1\\
6341	10155.1\\
6342	10155.1\\
6343	10155.1\\
6344	10155.1\\
6345	10155.1\\
6346	10155.1\\
6347	10155.1\\
6348	10155.1\\
6349	10155.1\\
6350	10155.1\\
6351	10155.1\\
6352	10155.1\\
6353	10505.1\\
6354	10505.1\\
6355	10505.1\\
6356	10505.1\\
6357	10505.1\\
6358	10623.1\\
6359	10623.1\\
6360	10623.1\\
6361	10623.1\\
6362	10623.1\\
6363	10623.1\\
6364	10623.1\\
6365	10623.1\\
6366	10623.1\\
6367	10623.1\\
6368	10623.1\\
6369	10623.1\\
6370	10623.1\\
6371	10623.1\\
6372	10623.1\\
6373	10623.1\\
6374	10623.1\\
6375	10623.1\\
6376	10623.1\\
6377	10623.1\\
6378	10623.1\\
6379	10623.1\\
6380	10623.1\\
6381	10623.1\\
6382	10623.1\\
6383	10623.1\\
6384	10623.1\\
6385	10623.1\\
6386	10623.1\\
6387	10623.1\\
6388	10623.1\\
6389	10623.1\\
6390	10623.1\\
6391	10623.1\\
6392	10623.1\\
6393	10623.1\\
6394	10623.1\\
6395	10623.1\\
6396	10623.1\\
6397	10623.1\\
6398	10623.1\\
6399	10623.1\\
6400	10623.1\\
6401	10623.1\\
6402	10623.1\\
6403	10623.1\\
6404	10623.1\\
6405	10623.1\\
6406	10623.1\\
6407	10623.1\\
6408	10623.1\\
6409	10623.1\\
6410	10623.1\\
6411	10623.1\\
6412	10623.1\\
6413	10623.1\\
6414	10623.1\\
6415	10623.1\\
6416	10623.1\\
6417	10623.1\\
6418	10623.1\\
6419	10623.1\\
6420	10623.1\\
6421	10623.1\\
6422	10623.1\\
6423	10623.1\\
6424	10623.1\\
6425	10623.1\\
6426	10623.1\\
6427	10623.1\\
6428	10623.1\\
6429	10623.1\\
6430	10623.1\\
6431	10623.1\\
6432	10186.1\\
6433	10186.1\\
6434	10186.1\\
6435	10186.1\\
6436	10186.1\\
6437	10186.1\\
6438	10186.1\\
6439	10186.1\\
6440	10186.1\\
6441	10186.1\\
6442	10186.1\\
6443	10186.1\\
6444	10186.1\\
6445	10186.1\\
6446	10186.1\\
6447	10186.1\\
6448	10186.1\\
6449	10186.1\\
6450	10186.1\\
6451	10186.1\\
6452	10186.1\\
6453	10186.1\\
6454	10186.1\\
6455	10186.1\\
6456	10186.1\\
6457	9701.1\\
6458	9701.1\\
6459	9701.1\\
6460	9701.1\\
6461	9701.1\\
6462	9701.1\\
6463	9701.1\\
6464	9701.1\\
6465	9701.1\\
6466	9701.1\\
6467	9701.1\\
6468	9701.1\\
6469	9701.1\\
6470	9701.1\\
6471	9701.1\\
6472	9701.1\\
6473	9701.1\\
6474	9701.1\\
6475	9701.1\\
6476	9701.1\\
6477	9701.1\\
6478	9701.1\\
6479	9701.1\\
6480	9701.1\\
6481	10138.1\\
6482	10138.1\\
6483	10138.1\\
6484	10138.1\\
6485	10138.1\\
6486	10138.1\\
6487	10138.1\\
6488	10138.1\\
6489	10138.1\\
6490	10138.1\\
6491	10138.1\\
6492	10138.1\\
6493	10138.1\\
6494	10138.1\\
6495	10138.1\\
6496	10138.1\\
6497	10138.1\\
6498	10138.1\\
6499	9701.1\\
6500	9701.1\\
6501	10138.1\\
6502	10138.1\\
6503	10138.1\\
6504	10138.1\\
6505	10138.1\\
6506	10138.1\\
6507	10138.1\\
6508	10138.1\\
6509	10138.1\\
6510	10138.1\\
6511	10138.1\\
6512	10138.1\\
6513	10138.1\\
6514	10138.1\\
6515	10138.1\\
6516	10138.1\\
6517	10138.1\\
6518	10138.1\\
6519	10138.1\\
6520	10138.1\\
6521	10138.1\\
6522	10138.1\\
6523	10138.1\\
6524	10138.1\\
6525	10138.1\\
6526	10138.1\\
6527	10138.1\\
6528	10138.1\\
6529	10138.1\\
6530	10138.1\\
6531	10138.1\\
6532	10138.1\\
6533	10138.1\\
6534	9678.1\\
6535	9678.1\\
6536	9678.1\\
6537	9678.1\\
6538	9678.1\\
6539	9678.1\\
6540	9678.1\\
6541	9678.1\\
6542	10138.1\\
6543	10138.1\\
6544	10138.1\\
6545	10138.1\\
6546	10138.1\\
6547	10138.1\\
6548	10138.1\\
6549	10138.1\\
6550	10138.1\\
6551	10138.1\\
6552	10138.1\\
6553	10138.1\\
6554	10138.1\\
6555	10138.1\\
6556	10138.1\\
6557	10138.1\\
6558	10138.1\\
6559	10138.1\\
6560	10138.1\\
6561	10138.1\\
6562	10138.1\\
6563	10138.1\\
6564	10138.1\\
6565	10138.1\\
6566	10138.1\\
6567	10138.1\\
6568	10138.1\\
6569	10138.1\\
6570	10138.1\\
6571	10138.1\\
6572	10138.1\\
6573	10138.1\\
6574	10138.1\\
6575	10138.1\\
6576	10138.1\\
6577	10138.1\\
6578	10138.1\\
6579	10138.1\\
6580	10138.1\\
6581	10138.1\\
6582	10138.1\\
6583	10138.1\\
6584	10138.1\\
6585	10138.1\\
6586	10138.1\\
6587	10138.1\\
6588	10138.1\\
6589	10138.1\\
6590	10138.1\\
6591	10138.1\\
6592	10138.1\\
6593	10138.1\\
6594	10138.1\\
6595	10138.1\\
6596	10138.1\\
6597	10138.1\\
6598	10138.1\\
6599	10138.1\\
6600	10138.1\\
6601	10138.1\\
6602	10138.1\\
6603	10138.1\\
6604	10138.1\\
6605	10138.1\\
6606	10138.1\\
6607	10138.1\\
6608	10138.1\\
6609	10138.1\\
6610	10138.1\\
6611	10138.1\\
6612	10138.1\\
6613	10138.1\\
6614	10138.1\\
6615	10138.1\\
6616	10138.1\\
6617	10138.1\\
6618	10138.1\\
6619	10138.1\\
6620	10138.1\\
6621	10138.1\\
6622	10138.1\\
6623	10051.1\\
6624	10051.1\\
6625	10051.1\\
6626	10051.1\\
6627	10051.1\\
6628	10051.1\\
6629	10051.1\\
6630	10051.1\\
6631	10051.1\\
6632	10051.1\\
6633	10051.1\\
6634	10051.1\\
6635	10051.1\\
6636	10051.1\\
6637	10051.1\\
6638	10051.1\\
6639	10051.1\\
6640	10051.1\\
6641	10051.1\\
6642	10051.1\\
6643	10051.1\\
6644	10051.1\\
6645	10051.1\\
6646	10051.1\\
6647	10051.1\\
6648	10051.1\\
6649	10051.1\\
6650	10051.1\\
6651	10051.1\\
6652	10051.1\\
6653	10051.1\\
6654	10051.1\\
6655	10051.1\\
6656	10051.1\\
6657	10051.1\\
6658	10051.1\\
6659	10051.1\\
6660	10051.1\\
6661	10051.1\\
6662	10051.1\\
6663	10051.1\\
6664	10051.1\\
6665	10051.1\\
6666	10051.1\\
6667	10051.1\\
6668	10051.1\\
6669	10051.1\\
6670	10051.1\\
6671	10051.1\\
6672	10051.1\\
6673	10051.1\\
6674	10051.1\\
6675	10051.1\\
6676	10051.1\\
6677	10051.1\\
6678	10051.1\\
6679	10051.1\\
6680	10051.1\\
6681	9899.1\\
6682	9899.1\\
6683	9899.1\\
6684	10051.1\\
6685	10051.1\\
6686	10051.1\\
6687	10378.6\\
6688	10378.6\\
6689	10378.6\\
6690	10378.6\\
6691	10378.6\\
6692	10378.6\\
6693	10378.6\\
6694	10378.6\\
6695	10378.6\\
6696	10378.6\\
6697	10536.1\\
6698	10536.1\\
6699	10536.1\\
6700	10536.1\\
6701	10536.1\\
6702	10536.1\\
6703	10536.1\\
6704	10536.1\\
6705	10536.1\\
6706	10536.1\\
6707	10536.1\\
6708	10536.1\\
6709	10536.1\\
6710	10536.1\\
6711	10536.1\\
6712	10536.1\\
6713	10536.1\\
6714	10536.1\\
6715	10536.1\\
6716	10536.1\\
6717	10536.1\\
6718	10536.1\\
6719	10536.1\\
6720	10536.1\\
6721	10536.1\\
6722	10536.1\\
6723	10536.1\\
6724	10536.1\\
6725	10536.1\\
6726	10536.1\\
6727	10536.1\\
6728	10536.1\\
6729	10536.1\\
6730	10536.1\\
6731	10536.1\\
6732	10536.1\\
6733	10536.1\\
6734	10536.1\\
6735	10536.1\\
6736	10536.1\\
6737	10536.1\\
6738	10536.1\\
6739	10536.1\\
6740	10536.1\\
6741	10536.1\\
6742	10536.1\\
6743	10536.1\\
6744	10536.1\\
6745	10536.1\\
6746	10536.1\\
6747	10536.1\\
6748	10536.1\\
6749	10536.1\\
6750	10536.1\\
6751	10536.1\\
6752	10536.1\\
6753	10536.1\\
6754	10536.1\\
6755	10536.1\\
6756	10536.1\\
6757	10536.1\\
6758	10536.1\\
6759	10536.1\\
6760	10536.1\\
6761	10536.1\\
6762	10536.1\\
6763	10536.1\\
6764	10536.1\\
6765	10536.1\\
6766	10536.1\\
6767	10536.1\\
6768	10536.1\\
6769	10536.1\\
6770	10536.1\\
6771	10536.1\\
6772	10536.1\\
6773	10536.1\\
6774	10536.1\\
6775	10536.1\\
6776	10536.1\\
6777	10536.1\\
6778	10536.1\\
6779	10536.1\\
6780	10536.1\\
6781	10536.1\\
6782	10536.1\\
6783	10536.1\\
6784	10536.1\\
6785	10536.1\\
6786	10536.1\\
6787	10536.1\\
6788	10536.1\\
6789	10536.1\\
6790	10536.1\\
6791	10536.1\\
6792	10536.1\\
6793	10536.1\\
6794	10536.1\\
6795	10536.1\\
6796	10536.1\\
6797	10536.1\\
6798	10536.1\\
6799	10536.1\\
6800	10536.1\\
6801	10536.1\\
6802	10536.1\\
6803	10536.1\\
6804	10536.1\\
6805	10536.1\\
6806	10536.1\\
6807	10536.1\\
6808	10536.1\\
6809	10536.1\\
6810	10536.1\\
6811	10536.1\\
6812	10536.1\\
6813	10536.1\\
6814	10536.1\\
6815	10536.1\\
6816	10536.1\\
6817	10536.1\\
6818	10536.1\\
6819	10536.1\\
6820	10536.1\\
6821	10536.1\\
6822	10536.1\\
6823	10536.1\\
6824	10536.1\\
6825	10536.1\\
6826	10536.1\\
6827	10536.1\\
6828	10536.1\\
6829	10536.1\\
6830	10536.1\\
6831	10536.1\\
6832	10536.1\\
6833	10536.1\\
6834	10536.1\\
6835	10536.1\\
6836	10536.1\\
6837	10536.1\\
6838	10536.1\\
6839	10536.1\\
6840	10386.1\\
6841	10386.1\\
6842	10386.1\\
6843	10386.1\\
6844	10386.1\\
6845	10386.1\\
6846	10386.1\\
6847	10386.1\\
6848	10386.1\\
6849	10386.1\\
6850	10386.1\\
6851	10386.1\\
6852	10386.1\\
6853	10386.1\\
6854	10386.1\\
6855	10386.1\\
6856	10386.1\\
6857	10386.1\\
6858	10386.1\\
6859	10386.1\\
6860	10386.1\\
6861	10386.1\\
6862	10386.1\\
6863	10536.1\\
6864	10536.1\\
6865	10536.1\\
6866	10536.1\\
6867	10536.1\\
6868	10536.1\\
6869	10536.1\\
6870	10536.1\\
6871	10536.1\\
6872	10536.1\\
6873	10536.1\\
6874	10536.1\\
6875	10536.1\\
6876	10536.1\\
6877	10536.1\\
6878	10536.1\\
6879	10536.1\\
6880	10536.1\\
6881	10536.1\\
6882	10536.1\\
6883	10536.1\\
6884	10536.1\\
6885	10536.1\\
6886	10536.1\\
6887	10536.1\\
6888	10536.1\\
6889	10051.1\\
6890	10051.1\\
6891	10051.1\\
6892	10051.1\\
6893	10051.1\\
6894	10051.1\\
6895	10051.1\\
6896	10536.1\\
6897	10536.1\\
6898	10536.1\\
6899	10536.1\\
6900	10536.1\\
6901	10536.1\\
6902	10536.1\\
6903	10536.1\\
6904	10536.1\\
6905	10536.1\\
6906	10536.1\\
6907	10536.1\\
6908	10536.1\\
6909	10536.1\\
6910	10536.1\\
6911	10536.1\\
6912	10536.1\\
6913	10536.1\\
6914	10536.1\\
6915	10536.1\\
6916	10536.1\\
6917	10536.1\\
6918	10536.1\\
6919	10536.1\\
6920	10536.1\\
6921	10536.1\\
6922	10536.1\\
6923	10536.1\\
6924	10536.1\\
6925	10536.1\\
6926	10536.1\\
6927	10536.1\\
6928	10536.1\\
6929	10536.1\\
6930	10536.1\\
6931	10536.1\\
6932	10536.1\\
6933	10536.1\\
6934	10536.1\\
6935	10536.1\\
6936	10536.1\\
6937	10536.1\\
6938	10536.1\\
6939	10536.1\\
6940	10536.1\\
6941	10536.1\\
6942	10536.1\\
6943	10536.1\\
6944	10536.1\\
6945	10536.1\\
6946	10536.1\\
6947	10536.1\\
6948	10536.1\\
6949	10536.1\\
6950	10536.1\\
6951	10536.1\\
6952	10536.1\\
6953	10536.1\\
6954	10536.1\\
6955	10536.1\\
6956	10536.1\\
6957	10536.1\\
6958	10536.1\\
6959	10536.1\\
6960	10536.1\\
6961	10378.6\\
6962	10378.6\\
6963	10378.6\\
6964	10378.6\\
6965	10378.6\\
6966	10378.6\\
6967	10378.6\\
6968	10378.6\\
6969	10378.6\\
6970	10003.6\\
6971	10003.6\\
6972	10003.6\\
6973	10003.6\\
6974	10003.6\\
6975	10003.6\\
6976	10003.6\\
6977	10003.6\\
6978	10003.6\\
6979	10378.6\\
6980	10378.6\\
6981	10378.6\\
6982	10378.6\\
6983	10378.6\\
6984	10378.6\\
6985	10378.6\\
6986	10378.6\\
6987	10378.6\\
6988	10378.6\\
6989	10378.6\\
6990	10378.6\\
6991	10378.6\\
6992	10378.6\\
6993	10378.6\\
6994	10378.6\\
6995	10378.6\\
6996	10378.6\\
6997	10536.1\\
6998	10536.1\\
6999	10536.1\\
7000	10536.1\\
7001	10536.1\\
7002	10536.1\\
7003	10536.1\\
7004	10536.1\\
7005	10536.1\\
7006	10536.1\\
7007	10536.1\\
7008	10536.1\\
7009	10536.1\\
7010	10536.1\\
7011	10536.1\\
7012	10536.1\\
7013	10536.1\\
7014	10536.1\\
7015	10536.1\\
7016	10536.1\\
7017	10536.1\\
7018	10536.1\\
7019	10536.1\\
7020	10536.1\\
7021	10536.1\\
7022	10536.1\\
7023	10536.1\\
7024	10536.1\\
7025	10536.1\\
7026	10973.1\\
7027	11454.1\\
7028	11454.1\\
7029	11454.1\\
7030	11017.1\\
7031	11017.1\\
7032	11017.1\\
7033	11017.1\\
7034	11017.1\\
7035	11017.1\\
7036	11017.1\\
7037	11017.1\\
7038	11017.1\\
7039	11017.1\\
7040	11017.1\\
7041	11017.1\\
7042	11017.1\\
7043	11017.1\\
7044	11017.1\\
7045	11017.1\\
7046	11017.1\\
7047	11017.1\\
7048	11017.1\\
7049	11017.1\\
7050	11454.1\\
7051	11454.1\\
7052	11454.1\\
7053	11454.1\\
7054	11454.1\\
7055	11454.1\\
7056	11104.1\\
7057	11104.1\\
7058	11104.1\\
7059	11104.1\\
7060	11104.1\\
7061	11104.1\\
7062	11104.1\\
7063	11104.1\\
7064	11454.1\\
7065	11454.1\\
7066	11454.1\\
7067	11454.1\\
7068	11454.1\\
7069	11454.1\\
7070	11454.1\\
7071	11454.1\\
7072	11454.1\\
7073	11454.1\\
7074	11454.1\\
7075	11454.1\\
7076	11454.1\\
7077	11454.1\\
7078	11454.1\\
7079	11935.1\\
7080	11935.1\\
7081	11935.1\\
7082	11935.1\\
7083	11935.1\\
7084	11935.1\\
7085	11935.1\\
7086	11935.1\\
7087	11935.1\\
7088	11935.1\\
7089	11935.1\\
7090	11935.1\\
7091	11935.1\\
7092	11935.1\\
7093	11935.1\\
7094	11935.1\\
7095	11935.1\\
7096	11935.1\\
7097	11935.1\\
7098	11935.1\\
7099	11935.1\\
7100	11935.1\\
7101	11935.1\\
7102	11935.1\\
7103	11935.1\\
7104	11935.1\\
7105	11935.1\\
7106	11935.1\\
7107	11935.1\\
7108	11935.1\\
7109	11935.1\\
7110	11935.1\\
7111	11935.1\\
7112	11935.1\\
7113	11935.1\\
7114	11935.1\\
7115	11935.1\\
7116	11498.1\\
7117	11498.1\\
7118	11498.1\\
7119	11498.1\\
7120	11498.1\\
7121	11498.1\\
7122	11498.1\\
7123	11498.1\\
7124	11498.1\\
7125	11498.1\\
7126	11498.1\\
7127	11065.6\\
7128	11065.6\\
7129	11065.6\\
7130	11065.6\\
7131	11065.6\\
7132	11065.6\\
7133	11065.6\\
7134	11065.6\\
7135	11065.6\\
7136	11065.6\\
7137	11065.6\\
7138	11065.6\\
7139	11065.6\\
7140	11065.6\\
7141	11502.6\\
7142	11502.6\\
7143	11502.6\\
7144	11502.6\\
7145	11502.6\\
7146	11502.6\\
7147	11502.6\\
7148	11502.6\\
7149	11502.6\\
7150	11502.6\\
7151	11502.6\\
7152	11502.6\\
7153	11502.6\\
7154	11502.6\\
7155	11502.6\\
7156	11502.6\\
7157	11502.6\\
7158	11502.6\\
7159	11502.6\\
7160	11502.6\\
7161	11502.6\\
7162	11502.6\\
7163	11502.6\\
7164	11502.6\\
7165	11502.6\\
7166	11502.6\\
7167	11502.6\\
7168	11152.6\\
7169	11585.1\\
7170	11585.1\\
7171	11585.1\\
7172	11585.1\\
7173	11935.1\\
7174	11935.1\\
7175	11935.1\\
7176	11935.1\\
7177	11935.1\\
7178	11935.1\\
7179	11935.1\\
7180	11935.1\\
7181	11935.1\\
7182	11935.1\\
7183	11935.1\\
7184	11935.1\\
7185	11935.1\\
7186	11935.1\\
7187	11935.1\\
7188	11935.1\\
7189	11935.1\\
7190	11935.1\\
7191	11935.1\\
7192	11935.1\\
7193	11935.1\\
7194	11935.1\\
7195	11935.1\\
7196	11935.1\\
7197	11935.1\\
7198	11935.1\\
7199	11935.1\\
7200	11935.1\\
7201	11935.1\\
7202	11935.1\\
7203	11935.1\\
7204	11935.1\\
7205	11935.1\\
7206	11935.1\\
7207	11935.1\\
7208	11935.1\\
7209	11935.1\\
7210	11935.1\\
7211	11935.1\\
7212	11935.1\\
7213	11935.1\\
7214	11935.1\\
7215	11935.1\\
7216	11935.1\\
7217	11935.1\\
7218	11935.1\\
7219	11935.1\\
7220	11935.1\\
7221	11935.1\\
7222	11935.1\\
7223	11935.1\\
7224	11935.1\\
7225	11935.1\\
7226	11935.1\\
7227	11935.1\\
7228	11935.1\\
7229	11935.1\\
7230	11935.1\\
7231	11935.1\\
7232	11935.1\\
7233	11935.1\\
7234	11935.1\\
7235	11935.1\\
7236	11935.1\\
7237	11935.1\\
7238	11935.1\\
7239	11935.1\\
7240	11935.1\\
7241	11935.1\\
7242	11935.1\\
7243	11935.1\\
7244	11935.1\\
7245	11935.1\\
7246	11935.1\\
7247	11935.1\\
7248	11935.1\\
7249	11935.1\\
7250	11935.1\\
7251	11935.1\\
7252	11935.1\\
7253	11935.1\\
7254	11935.1\\
7255	11935.1\\
7256	11935.1\\
7257	11935.1\\
7258	11935.1\\
7259	11935.1\\
7260	11935.1\\
7261	11935.1\\
7262	11935.1\\
7263	11935.1\\
7264	11935.1\\
7265	11935.1\\
7266	11935.1\\
7267	11498.1\\
7268	11498.1\\
7269	11935.1\\
7270	11785.1\\
7271	11785.1\\
7272	11785.1\\
7273	11785.1\\
7274	11785.1\\
7275	11785.1\\
7276	11785.1\\
7277	11785.1\\
7278	11785.1\\
7279	11785.1\\
7280	11785.1\\
7281	11785.1\\
7282	11785.1\\
7283	11785.1\\
7284	11785.1\\
7285	11785.1\\
7286	11785.1\\
7287	11785.1\\
7288	11785.1\\
7289	11935.1\\
7290	11935.1\\
7291	11935.1\\
7292	11935.1\\
7293	11935.1\\
7294	11935.1\\
7295	11935.1\\
7296	11560.1\\
7297	11560.1\\
7298	11560.1\\
7299	11560.1\\
7300	11560.1\\
7301	11560.1\\
7302	11560.1\\
7303	11560.1\\
7304	11560.1\\
7305	11560.1\\
7306	11560.1\\
7307	11560.1\\
7308	11560.1\\
7309	11560.1\\
7310	11560.1\\
7311	11560.1\\
7312	11560.1\\
7313	11560.1\\
7314	11560.1\\
7315	11560.1\\
7316	11560.1\\
7317	11560.1\\
7318	11560.1\\
7319	11560.1\\
7320	11560.1\\
7321	11560.1\\
7322	11560.1\\
7323	11560.1\\
7324	11560.1\\
7325	11560.1\\
7326	11560.1\\
7327	11560.1\\
7328	11560.1\\
7329	11560.1\\
7330	11560.1\\
7331	11560.1\\
7332	11560.1\\
7333	11560.1\\
7334	11560.1\\
7335	11560.1\\
7336	11560.1\\
7337	11560.1\\
7338	11560.1\\
7339	11560.1\\
7340	11560.1\\
7341	11560.1\\
7342	11560.1\\
7343	11560.1\\
7344	11560.1\\
7345	11560.1\\
7346	11560.1\\
7347	11560.1\\
7348	11560.1\\
7349	11560.1\\
7350	11560.1\\
7351	11560.1\\
7352	11560.1\\
7353	11560.1\\
7354	11560.1\\
7355	11560.1\\
7356	11560.1\\
7357	11560.1\\
7358	11560.1\\
7359	11560.1\\
7360	11560.1\\
7361	11560.1\\
7362	11560.1\\
7363	11560.1\\
7364	11560.1\\
7365	11560.1\\
7366	11560.1\\
7367	11560.1\\
7368	11560.1\\
7369	11560.1\\
7370	11560.1\\
7371	11560.1\\
7372	11560.1\\
7373	11560.1\\
7374	11560.1\\
7375	11560.1\\
7376	11560.1\\
7377	11560.1\\
7378	11560.1\\
7379	11210.1\\
7380	11210.1\\
7381	11210.1\\
7382	11210.1\\
7383	11210.1\\
7384	11210.1\\
7385	11560.1\\
7386	11560.1\\
7387	11560.1\\
7388	11560.1\\
7389	11560.1\\
7390	11560.1\\
7391	11560.1\\
7392	11560.1\\
7393	11560.1\\
7394	11560.1\\
7395	11560.1\\
7396	11560.1\\
7397	11560.1\\
7398	11560.1\\
7399	11560.1\\
7400	11560.1\\
7401	11560.1\\
7402	11560.1\\
7403	11560.1\\
7404	11560.1\\
7405	11560.1\\
7406	11560.1\\
7407	11560.1\\
7408	11560.1\\
7409	11560.1\\
7410	11560.1\\
7411	11560.1\\
7412	11560.1\\
7413	11560.1\\
7414	11560.1\\
7415	11560.1\\
7416	11560.1\\
7417	11560.1\\
7418	11560.1\\
7419	11560.1\\
7420	11560.1\\
7421	11560.1\\
7422	11560.1\\
7423	11560.1\\
7424	11560.1\\
7425	11935.1\\
7426	11935.1\\
7427	11935.1\\
7428	11935.1\\
7429	11935.1\\
7430	11935.1\\
7431	11935.1\\
7432	11935.1\\
7433	11935.1\\
7434	11935.1\\
7435	11935.1\\
7436	11935.1\\
7437	11935.1\\
7438	11935.1\\
7439	11935.1\\
7440	11935.1\\
7441	11935.1\\
7442	11935.1\\
7443	11935.1\\
7444	11935.1\\
7445	11935.1\\
7446	11935.1\\
7447	11935.1\\
7448	11935.1\\
7449	11935.1\\
7450	11935.1\\
7451	11935.1\\
7452	11935.1\\
7453	11935.1\\
7454	11935.1\\
7455	11935.1\\
7456	11935.1\\
7457	11935.1\\
7458	11935.1\\
7459	11935.1\\
7460	11935.1\\
7461	11935.1\\
7462	11935.1\\
7463	11935.1\\
7464	11525.1\\
7465	11525.1\\
7466	11525.1\\
7467	11525.1\\
7468	11525.1\\
7469	11525.1\\
7470	11375.1\\
7471	11375.1\\
7472	11375.1\\
7473	11375.1\\
7474	11375.1\\
7475	11375.1\\
7476	11375.1\\
7477	11375.1\\
7478	11375.1\\
7479	11375.1\\
7480	11525.1\\
7481	11525.1\\
7482	11525.1\\
7483	11525.1\\
7484	11525.1\\
7485	11525.1\\
7486	11525.1\\
7487	11525.1\\
7488	11525.1\\
7489	11525.1\\
7490	11525.1\\
7491	11525.1\\
7492	11525.1\\
7493	11525.1\\
7494	11525.1\\
7495	11525.1\\
7496	11525.1\\
7497	11525.1\\
7498	11525.1\\
7499	11525.1\\
7500	11525.1\\
7501	11525.1\\
7502	11525.1\\
7503	11525.1\\
7504	11525.1\\
7505	11525.1\\
7506	11525.1\\
7507	11525.1\\
7508	11525.1\\
7509	11525.1\\
7510	11525.1\\
7511	11525.1\\
7512	11525.1\\
7513	11525.1\\
7514	11525.1\\
7515	11525.1\\
7516	11525.1\\
7517	11525.1\\
7518	11525.1\\
7519	11525.1\\
7520	11525.1\\
7521	11525.1\\
7522	11525.1\\
7523	11525.1\\
7524	11525.1\\
7525	11525.1\\
7526	11525.1\\
7527	11525.1\\
7528	11525.1\\
7529	11525.1\\
7530	11525.1\\
7531	11525.1\\
7532	11525.1\\
7533	11525.1\\
7534	11525.1\\
7535	11525.1\\
7536	11525.1\\
7537	11525.1\\
7538	11525.1\\
7539	11525.1\\
7540	11525.1\\
7541	11525.1\\
7542	11525.1\\
7543	11525.1\\
7544	11525.1\\
7545	11525.1\\
7546	11525.1\\
7547	11525.1\\
7548	11525.1\\
7549	11525.1\\
7550	11525.1\\
7551	11525.1\\
7552	11525.1\\
7553	11525.1\\
7554	11525.1\\
7555	11525.1\\
7556	11525.1\\
7557	11525.1\\
7558	11525.1\\
7559	11525.1\\
7560	11525.1\\
7561	11525.1\\
7562	11525.1\\
7563	11525.1\\
7564	11525.1\\
7565	11525.1\\
7566	11525.1\\
7567	11525.1\\
7568	11525.1\\
7569	11525.1\\
7570	11525.1\\
7571	11525.1\\
7572	11525.1\\
7573	11525.1\\
7574	11525.1\\
7575	11525.1\\
7576	11525.1\\
7577	11525.1\\
7578	11525.1\\
7579	11525.1\\
7580	11525.1\\
7581	11525.1\\
7582	11525.1\\
7583	11525.1\\
7584	11525.1\\
7585	11525.1\\
7586	11525.1\\
7587	11525.1\\
7588	11525.1\\
7589	11525.1\\
7590	11525.1\\
7591	11525.1\\
7592	11525.1\\
7593	11525.1\\
7594	11525.1\\
7595	11525.1\\
7596	11525.1\\
7597	11525.1\\
7598	11525.1\\
7599	11525.1\\
7600	11525.1\\
7601	11525.1\\
7602	11525.1\\
7603	11525.1\\
7604	11525.1\\
7605	11525.1\\
7606	11525.1\\
7607	11525.1\\
7608	11525.1\\
7609	11525.1\\
7610	11525.1\\
7611	11525.1\\
7612	11525.1\\
7613	11525.1\\
7614	11525.1\\
7615	11525.1\\
7616	11525.1\\
7617	11525.1\\
7618	11525.1\\
7619	11525.1\\
7620	11525.1\\
7621	11525.1\\
7622	11525.1\\
7623	11525.1\\
7624	11525.1\\
7625	11525.1\\
7626	11525.1\\
7627	11525.1\\
7628	11525.1\\
7629	11525.1\\
7630	11525.1\\
7631	11525.1\\
7632	11935.1\\
7633	11935.1\\
7634	11935.1\\
7635	11935.1\\
7636	11935.1\\
7637	11935.1\\
7638	11935.1\\
7639	11935.1\\
7640	11935.1\\
7641	11935.1\\
7642	11935.1\\
7643	11935.1\\
7644	11935.1\\
7645	11935.1\\
7646	11935.1\\
7647	11935.1\\
7648	11935.1\\
7649	11935.1\\
7650	11935.1\\
7651	11935.1\\
7652	11935.1\\
7653	11935.1\\
7654	11935.1\\
7655	11935.1\\
7656	11935.1\\
7657	11935.1\\
7658	11935.1\\
7659	11935.1\\
7660	11935.1\\
7661	11935.1\\
7662	11935.1\\
7663	11935.1\\
7664	11935.1\\
7665	11935.1\\
7666	11935.1\\
7667	11935.1\\
7668	11935.1\\
7669	11935.1\\
7670	11935.1\\
7671	11935.1\\
7672	11935.1\\
7673	11935.1\\
7674	11935.1\\
7675	11935.1\\
7676	11935.1\\
7677	11935.1\\
7678	11935.1\\
7679	11935.1\\
7680	11935.1\\
7681	11935.1\\
7682	11935.1\\
7683	11935.1\\
7684	11935.1\\
7685	11935.1\\
7686	11935.1\\
7687	11935.1\\
7688	11935.1\\
7689	11935.1\\
7690	11935.1\\
7691	11935.1\\
7692	11935.1\\
7693	11935.1\\
7694	11935.1\\
7695	11770.1\\
7696	11770.1\\
7697	11770.1\\
7698	11360.1\\
7699	11360.1\\
7700	11525.1\\
7701	11935.1\\
7702	11935.1\\
7703	11935.1\\
7704	11935.1\\
7705	11935.1\\
7706	11935.1\\
7707	11935.1\\
7708	11935.1\\
7709	11935.1\\
7710	11935.1\\
7711	11935.1\\
7712	11935.1\\
7713	11935.1\\
7714	11935.1\\
7715	11935.1\\
7716	11935.1\\
7717	11935.1\\
7718	11935.1\\
7719	11935.1\\
7720	11935.1\\
7721	11935.1\\
7722	11935.1\\
7723	11935.1\\
7724	11935.1\\
7725	11935.1\\
7726	11935.1\\
7727	11935.1\\
7728	11935.1\\
7729	11935.1\\
7730	11935.1\\
7731	11935.1\\
7732	11935.1\\
7733	11935.1\\
7734	11935.1\\
7735	11585.1\\
7736	11585.1\\
7737	11935.1\\
7738	11935.1\\
7739	11935.1\\
7740	11935.1\\
7741	11935.1\\
7742	11935.1\\
7743	11935.1\\
7744	11935.1\\
7745	11935.1\\
7746	11935.1\\
7747	11935.1\\
7748	11935.1\\
7749	11935.1\\
7750	11935.1\\
7751	11935.1\\
7752	11935.1\\
7753	11935.1\\
7754	11935.1\\
7755	11935.1\\
7756	11935.1\\
7757	11935.1\\
7758	11935.1\\
7759	11935.1\\
7760	11935.1\\
7761	11935.1\\
7762	11935.1\\
7763	11935.1\\
7764	11935.1\\
7765	11935.1\\
7766	11935.1\\
7767	11935.1\\
7768	11935.1\\
7769	11935.1\\
7770	11935.1\\
7771	11935.1\\
7772	11935.1\\
7773	11935.1\\
7774	11935.1\\
7775	11935.1\\
7776	11935.1\\
7777	11935.1\\
7778	11935.1\\
7779	11935.1\\
7780	11935.1\\
7781	11935.1\\
7782	11935.1\\
7783	11935.1\\
7784	11935.1\\
7785	11935.1\\
7786	11935.1\\
7787	11935.1\\
7788	11935.1\\
7789	11935.1\\
7790	11750.1\\
7791	11750.1\\
7792	11750.1\\
7793	11750.1\\
7794	11750.1\\
7795	11750.1\\
7796	11935.1\\
7797	11935.1\\
7798	11935.1\\
7799	11935.1\\
7800	11935.1\\
7801	11935.1\\
7802	11935.1\\
7803	11935.1\\
7804	11935.1\\
7805	11935.1\\
7806	11935.1\\
7807	11935.1\\
7808	11585.1\\
7809	11585.1\\
7810	11585.1\\
7811	11935.1\\
7812	11935.1\\
7813	11935.1\\
7814	11935.1\\
7815	11935.1\\
7816	11935.1\\
7817	11935.1\\
7818	11935.1\\
7819	11935.1\\
7820	11935.1\\
7821	11935.1\\
7822	11935.1\\
7823	11935.1\\
7824	11935.1\\
7825	11935.1\\
7826	11935.1\\
7827	11935.1\\
7828	11935.1\\
7829	11935.1\\
7830	11935.1\\
7831	11935.1\\
7832	11935.1\\
7833	11935.1\\
7834	11935.1\\
7835	11935.1\\
7836	11935.1\\
7837	11935.1\\
7838	11935.1\\
7839	11935.1\\
7840	11935.1\\
7841	11935.1\\
7842	11935.1\\
7843	11935.1\\
7844	11935.1\\
7845	11935.1\\
7846	11935.1\\
7847	11935.1\\
7848	11935.1\\
7849	11935.1\\
7850	11935.1\\
7851	11935.1\\
7852	11935.1\\
7853	11935.1\\
7854	11935.1\\
7855	11935.1\\
7856	11935.1\\
7857	11935.1\\
7858	11935.1\\
7859	11935.1\\
7860	11935.1\\
7861	11935.1\\
7862	11935.1\\
7863	11935.1\\
7864	11935.1\\
7865	11935.1\\
7866	11935.1\\
7867	11935.1\\
7868	11935.1\\
7869	11935.1\\
7870	11935.1\\
7871	11935.1\\
7872	11935.1\\
7873	11935.1\\
7874	11935.1\\
7875	11935.1\\
7876	11935.1\\
7877	11935.1\\
7878	11935.1\\
7879	11935.1\\
7880	11935.1\\
7881	11935.1\\
7882	11935.1\\
7883	11935.1\\
7884	11935.1\\
7885	11935.1\\
7886	11935.1\\
7887	11935.1\\
7888	11935.1\\
7889	11935.1\\
7890	11935.1\\
7891	11935.1\\
7892	11935.1\\
7893	11935.1\\
7894	11935.1\\
7895	11935.1\\
7896	11935.1\\
7897	11935.1\\
7898	11935.1\\
7899	11935.1\\
7900	11935.1\\
7901	11935.1\\
7902	11935.1\\
7903	11935.1\\
7904	11935.1\\
7905	11935.1\\
7906	11935.1\\
7907	11935.1\\
7908	11935.1\\
7909	11935.1\\
7910	11935.1\\
7911	11935.1\\
7912	11935.1\\
7913	11935.1\\
7914	11935.1\\
7915	11935.1\\
7916	11935.1\\
7917	11935.1\\
7918	11935.1\\
7919	11935.1\\
7920	11935.1\\
7921	11935.1\\
7922	11935.1\\
7923	11935.1\\
7924	11585.1\\
7925	11585.1\\
7926	11585.1\\
7927	11585.1\\
7928	11585.1\\
7929	11585.1\\
7930	11935.1\\
7931	11935.1\\
7932	11935.1\\
7933	11935.1\\
7934	11935.1\\
7935	11935.1\\
7936	11935.1\\
7937	11935.1\\
7938	11935.1\\
7939	11935.1\\
7940	11935.1\\
7941	11935.1\\
7942	11935.1\\
7943	11935.1\\
7944	11935.1\\
7945	11935.1\\
7946	11935.1\\
7947	11935.1\\
7948	11935.1\\
7949	11935.1\\
7950	11935.1\\
7951	11935.1\\
7952	11935.1\\
7953	11935.1\\
7954	11935.1\\
7955	11935.1\\
7956	11935.1\\
7957	11769.6\\
7958	11769.6\\
7959	11769.6\\
7960	11769.6\\
7961	11769.6\\
7962	11769.6\\
7963	11935.1\\
7964	11935.1\\
7965	11935.1\\
7966	11935.1\\
7967	11935.1\\
7968	11935.1\\
7969	11935.1\\
7970	11935.1\\
7971	11935.1\\
7972	11935.1\\
7973	11935.1\\
7974	11935.1\\
7975	11935.1\\
7976	11935.1\\
7977	11935.1\\
7978	11935.1\\
7979	11935.1\\
7980	11935.1\\
7981	11935.1\\
7982	11935.1\\
7983	11935.1\\
7984	11935.1\\
7985	11935.1\\
7986	11935.1\\
7987	11935.1\\
7988	11935.1\\
7989	11935.1\\
7990	11935.1\\
7991	11935.1\\
7992	11935.1\\
7993	11935.1\\
7994	11935.1\\
7995	11935.1\\
7996	11935.1\\
7997	11935.1\\
7998	11935.1\\
7999	11935.1\\
8000	11935.1\\
8001	11935.1\\
};
\addplot [color=mycolor1,line width=1.3pt,solid,forget plot]
  table[row sep=crcr]{%
8001	11935.1\\
8002	11935.1\\
8003	10889.3\\
8004	10889.3\\
8005	10889.3\\
8006	10889.3\\
8007	10889.3\\
8008	10889.3\\
8009	10739.3\\
8010	10739.3\\
8011	10739.3\\
8012	10739.3\\
8013	10739.3\\
8014	10739.3\\
8015	10739.3\\
8016	10739.3\\
8017	10739.3\\
8018	10739.3\\
8019	10739.3\\
8020	10739.3\\
8021	10739.3\\
8022	10739.3\\
8023	10739.3\\
8024	10739.3\\
8025	10739.3\\
8026	10739.3\\
8027	10739.3\\
8028	10739.3\\
8029	10739.3\\
8030	10739.3\\
8031	10739.3\\
8032	10739.3\\
8033	10739.3\\
8034	10739.3\\
8035	10739.3\\
8036	10739.3\\
8037	10739.3\\
8038	10889.3\\
8039	10889.3\\
8040	10889.3\\
8041	10889.3\\
8042	10889.3\\
8043	10889.3\\
8044	10889.3\\
8045	11935.1\\
8046	11935.1\\
8047	11935.1\\
8048	11935.1\\
8049	11935.1\\
8050	11935.1\\
8051	11935.1\\
8052	11935.1\\
8053	11935.1\\
8054	11935.1\\
8055	11935.1\\
8056	11935.1\\
8057	11935.1\\
8058	11935.1\\
8059	11935.1\\
8060	11935.1\\
8061	11935.1\\
8062	11935.1\\
8063	11935.1\\
8064	11935.1\\
8065	11935.1\\
8066	11935.1\\
8067	11935.1\\
8068	11935.1\\
8069	11935.1\\
8070	11935.1\\
8071	11935.1\\
8072	11935.1\\
8073	11935.1\\
8074	11935.1\\
8075	11935.1\\
8076	11935.1\\
8077	11935.1\\
8078	11935.1\\
8079	11935.1\\
8080	11935.1\\
8081	11935.1\\
8082	11935.1\\
8083	11935.1\\
8084	11935.1\\
8085	11935.1\\
8086	11935.1\\
8087	11935.1\\
8088	11935.1\\
8089	11935.1\\
8090	11935.1\\
8091	11935.1\\
8092	11935.1\\
8093	11935.1\\
8094	11935.1\\
8095	11935.1\\
8096	11935.1\\
8097	11935.1\\
8098	11935.1\\
8099	11935.1\\
8100	11935.1\\
8101	11935.1\\
8102	11935.1\\
8103	11935.1\\
8104	11935.1\\
8105	11935.1\\
8106	11935.1\\
8107	11935.1\\
8108	11935.1\\
8109	11935.1\\
8110	11935.1\\
8111	11935.1\\
8112	11935.1\\
8113	11935.1\\
8114	11935.1\\
8115	11935.1\\
8116	11935.1\\
8117	11935.1\\
8118	11935.1\\
8119	11935.1\\
8120	11935.1\\
8121	11935.1\\
8122	11935.1\\
8123	11935.1\\
8124	11935.1\\
8125	11935.1\\
8126	11935.1\\
8127	11935.1\\
8128	11935.1\\
8129	11935.1\\
8130	11935.1\\
8131	11935.1\\
8132	11935.1\\
8133	11935.1\\
8134	11498.1\\
8135	11498.1\\
8136	11498.1\\
8137	11498.1\\
8138	11498.1\\
8139	11498.1\\
8140	11498.1\\
8141	11498.1\\
8142	11498.1\\
8143	11498.1\\
8144	11498.1\\
8145	11498.1\\
8146	11498.1\\
8147	11498.1\\
8148	11498.1\\
8149	11498.1\\
8150	11498.1\\
8151	11498.1\\
8152	11498.1\\
8153	11498.1\\
8154	11498.1\\
8155	11498.1\\
8156	11498.1\\
8157	11498.1\\
8158	11498.1\\
8159	11498.1\\
8160	11038.1\\
8161	11038.1\\
8162	11038.1\\
8163	11038.1\\
8164	11038.1\\
8165	11038.1\\
8166	11038.1\\
8167	11038.1\\
8168	11038.1\\
8169	11038.1\\
8170	11038.1\\
8171	11038.1\\
8172	11038.1\\
8173	11038.1\\
8174	11498.1\\
8175	11498.1\\
8176	11498.1\\
8177	11498.1\\
8178	11498.1\\
8179	11498.1\\
8180	11498.1\\
8181	11498.1\\
8182	11498.1\\
8183	11498.1\\
8184	11935.1\\
8185	11935.1\\
8186	11935.1\\
8187	11935.1\\
8188	11935.1\\
8189	11935.1\\
8190	11935.1\\
8191	11935.1\\
8192	11935.1\\
8193	11935.1\\
8194	11935.1\\
8195	11935.1\\
8196	11935.1\\
8197	11935.1\\
8198	11935.1\\
8199	11935.1\\
8200	11935.1\\
8201	11935.1\\
8202	11935.1\\
8203	11935.1\\
8204	11935.1\\
8205	11935.1\\
8206	11935.1\\
8207	11935.1\\
8208	11935.1\\
8209	11935.1\\
8210	11935.1\\
8211	11935.1\\
8212	11935.1\\
8213	11935.1\\
8214	11935.1\\
8215	11935.1\\
8216	11935.1\\
8217	11935.1\\
8218	11935.1\\
8219	11935.1\\
8220	11935.1\\
8221	11935.1\\
8222	11935.1\\
8223	11935.1\\
8224	11935.1\\
8225	11935.1\\
8226	11935.1\\
8227	11935.1\\
8228	11935.1\\
8229	11935.1\\
8230	11935.1\\
8231	11935.1\\
8232	11935.1\\
8233	11935.1\\
8234	11935.1\\
8235	11935.1\\
8236	11935.1\\
8237	11935.1\\
8238	11935.1\\
8239	11935.1\\
8240	11935.1\\
8241	11935.1\\
8242	11935.1\\
8243	11935.1\\
8244	11935.1\\
8245	11935.1\\
8246	11465.1\\
8247	11465.1\\
8248	11465.1\\
8249	11465.1\\
8250	11465.1\\
8251	11465.1\\
8252	11465.1\\
8253	11465.1\\
8254	11465.1\\
8255	11465.1\\
8256	11465.1\\
8257	11465.1\\
8258	11465.1\\
8259	11465.1\\
8260	11465.1\\
8261	11465.1\\
8262	11465.1\\
8263	11465.1\\
8264	11465.1\\
8265	11465.1\\
8266	11465.1\\
8267	11465.1\\
8268	11465.1\\
8269	11465.1\\
8270	11465.1\\
8271	11465.1\\
8272	11465.1\\
8273	11465.1\\
8274	11465.1\\
8275	11465.1\\
8276	11465.1\\
8277	11465.1\\
8278	11465.1\\
8279	11465.1\\
8280	11465.1\\
8281	11465.1\\
8282	11465.1\\
8283	11465.1\\
8284	11465.1\\
8285	11465.1\\
8286	11465.1\\
8287	11465.1\\
8288	11465.1\\
8289	11465.1\\
8290	11465.1\\
8291	11465.1\\
8292	11465.1\\
8293	11465.1\\
8294	11465.1\\
8295	11465.1\\
8296	11465.1\\
8297	11465.1\\
8298	11465.1\\
8299	11465.1\\
8300	11465.1\\
8301	11465.1\\
8302	11465.1\\
8303	11465.1\\
8304	11465.1\\
8305	11465.1\\
8306	11465.1\\
8307	11465.1\\
8308	11465.1\\
8309	11465.1\\
8310	11465.1\\
8311	11465.1\\
8312	11465.1\\
8313	11465.1\\
8314	11465.1\\
8315	11465.1\\
8316	11465.1\\
8317	11465.1\\
8318	11465.1\\
8319	11465.1\\
8320	11465.1\\
8321	11465.1\\
8322	11465.1\\
8323	11465.1\\
8324	11465.1\\
8325	11465.1\\
8326	11465.1\\
8327	11465.1\\
8328	11465.1\\
8329	11465.1\\
8330	11465.1\\
8331	11465.1\\
8332	11465.1\\
8333	11465.1\\
8334	11465.1\\
8335	11465.1\\
8336	11465.1\\
8337	11465.1\\
8338	11465.1\\
8339	11465.1\\
8340	11465.1\\
8341	11465.1\\
8342	11465.1\\
8343	11465.1\\
8344	11465.1\\
8345	11465.1\\
8346	11465.1\\
8347	11465.1\\
8348	11465.1\\
8349	11465.1\\
8350	11465.1\\
8351	11465.1\\
8352	11465.1\\
8353	11465.1\\
8354	11465.1\\
8355	11465.1\\
8356	11465.1\\
8357	11465.1\\
8358	11465.1\\
8359	11465.1\\
8360	11465.1\\
8361	11465.1\\
8362	11465.1\\
8363	11465.1\\
8364	11465.1\\
8365	11465.1\\
8366	11465.1\\
8367	11465.1\\
8368	11465.1\\
8369	11465.1\\
8370	11465.1\\
8371	11465.1\\
8372	11465.1\\
8373	11465.1\\
8374	11465.1\\
8375	11465.1\\
8376	11465.1\\
8377	11465.1\\
8378	11465.1\\
8379	11465.1\\
8380	11465.1\\
8381	11465.1\\
8382	11465.1\\
8383	11315.1\\
8384	11315.1\\
8385	11315.1\\
8386	11315.1\\
8387	11465.1\\
8388	11465.1\\
8389	11465.1\\
8390	11465.1\\
8391	11465.1\\
8392	11465.1\\
8393	11465.1\\
8394	11465.1\\
8395	11465.1\\
8396	11465.1\\
8397	11465.1\\
8398	11465.1\\
8399	11465.1\\
8400	11465.1\\
8401	11465.1\\
8402	11465.1\\
8403	11465.1\\
8404	11465.1\\
8405	11465.1\\
8406	11465.1\\
8407	11465.1\\
8408	11465.1\\
8409	11465.1\\
8410	11465.1\\
8411	11465.1\\
8412	11465.1\\
8413	11465.1\\
8414	11465.1\\
8415	11465.1\\
8416	11465.1\\
8417	11465.1\\
8418	11465.1\\
8419	11465.1\\
8420	11465.1\\
8421	11465.1\\
8422	11465.1\\
8423	11465.1\\
8424	11465.1\\
8425	11465.1\\
8426	11465.1\\
8427	11154.71\\
8428	11154.71\\
8429	10804.71\\
8430	10804.71\\
8431	10804.71\\
8432	11154.71\\
8433	11154.71\\
8434	11465.1\\
8435	11465.1\\
8436	11465.1\\
8437	11465.1\\
8438	11465.1\\
8439	11465.1\\
8440	11465.1\\
8441	11465.1\\
8442	11465.1\\
8443	11465.1\\
8444	11465.1\\
8445	11465.1\\
8446	11465.1\\
8447	11465.1\\
8448	11465.1\\
8449	11465.1\\
8450	11465.1\\
8451	11465.1\\
8452	11465.1\\
8453	11465.1\\
8454	11465.1\\
8455	11465.1\\
8456	11465.1\\
8457	11465.1\\
8458	11465.1\\
8459	11465.1\\
8460	11465.1\\
8461	11459.1\\
8462	11765.1\\
8463	11765.1\\
8464	11765.1\\
8465	11765.1\\
8466	11765.1\\
8467	11765.1\\
8468	11765.1\\
8469	11765.1\\
8470	11765.1\\
8471	11765.1\\
8472	11765.1\\
8473	12503.1\\
8474	12503.1\\
8475	12503.1\\
8476	12503.1\\
8477	12503.1\\
8478	12503.1\\
8479	12503.1\\
8480	12503.1\\
8481	12503.1\\
8482	12503.1\\
8483	12503.1\\
8484	12503.1\\
8485	12503.1\\
8486	12503.1\\
8487	12503.1\\
8488	12503.1\\
8489	12503.1\\
8490	12503.1\\
8491	12503.1\\
8492	12503.1\\
8493	12503.1\\
8494	12503.1\\
8495	12503.1\\
8496	12503.1\\
8497	12503.1\\
8498	12503.1\\
8499	12503.1\\
8500	12503.1\\
8501	12503.1\\
8502	12503.1\\
8503	12503.1\\
8504	12503.1\\
8505	12503.1\\
8506	12503.1\\
8507	12503.1\\
8508	12503.1\\
8509	12503.1\\
8510	12503.1\\
8511	12503.1\\
8512	12503.1\\
8513	12503.1\\
8514	12503.1\\
8515	12503.1\\
8516	12503.1\\
8517	12503.1\\
8518	12503.1\\
8519	12503.1\\
8520	12503.1\\
8521	12503.1\\
8522	12503.1\\
8523	12503.1\\
8524	12503.1\\
8525	12503.1\\
8526	12503.1\\
8527	12503.1\\
8528	12503.1\\
8529	12503.1\\
8530	12503.1\\
8531	12503.1\\
8532	12503.1\\
8533	12973.1\\
8534	12973.1\\
8535	12973.1\\
8536	12973.1\\
8537	12973.1\\
8538	12973.1\\
8539	12973.1\\
8540	12973.1\\
8541	12973.1\\
8542	12973.1\\
8543	12973.1\\
8544	12973.1\\
8545	12973.1\\
8546	12973.1\\
8547	12973.1\\
8548	12973.1\\
8549	12973.1\\
8550	12973.1\\
8551	12513.1\\
8552	12513.1\\
8553	12513.1\\
8554	12513.1\\
8555	12513.1\\
8556	12513.1\\
8557	12513.1\\
8558	12513.1\\
8559	12513.1\\
8560	12513.1\\
8561	12513.1\\
8562	12513.1\\
8563	12973.1\\
8564	12973.1\\
8565	12973.1\\
8566	12973.1\\
8567	12973.1\\
8568	12973.1\\
8569	12973.1\\
8570	12973.1\\
8571	12973.1\\
8572	12973.1\\
8573	12973.1\\
8574	12973.1\\
8575	12973.1\\
8576	12973.1\\
8577	12973.1\\
8578	12973.1\\
8579	12808.1\\
8580	12808.1\\
8581	12808.1\\
8582	12808.1\\
8583	12808.1\\
8584	12808.1\\
8585	12808.1\\
8586	12758.1\\
8587	12758.1\\
8588	12758.1\\
8589	12758.1\\
8590	12758.1\\
8591	12758.1\\
8592	11712.3\\
8593	11712.3\\
8594	11712.3\\
8595	11712.3\\
8596	11712.3\\
8597	11712.3\\
8598	11712.3\\
8599	11712.3\\
8600	11712.3\\
8601	11712.3\\
8602	11712.3\\
8603	11712.3\\
8604	11712.3\\
8605	11712.3\\
8606	11712.3\\
8607	11712.3\\
8608	11712.3\\
8609	11712.3\\
8610	11712.3\\
8611	11712.3\\
8612	11712.3\\
8613	11712.3\\
8614	11712.3\\
8615	11712.3\\
8616	11712.3\\
8617	11712.3\\
8618	11712.3\\
8619	11712.3\\
8620	11712.3\\
8621	11712.3\\
8622	11712.3\\
8623	11712.3\\
8624	11712.3\\
8625	11712.3\\
8626	11712.3\\
8627	11712.3\\
8628	11712.3\\
8629	11712.3\\
8630	11712.3\\
8631	11712.3\\
8632	11712.3\\
8633	11712.3\\
8634	11712.3\\
8635	11712.3\\
8636	11712.3\\
8637	11712.3\\
8638	11712.3\\
8639	11712.3\\
8640	11712.3\\
8641	11712.3\\
8642	11712.3\\
8643	11712.3\\
8644	11712.3\\
8645	11712.3\\
8646	11712.3\\
8647	11712.3\\
8648	11712.3\\
8649	11712.3\\
8650	11712.3\\
8651	11712.3\\
8652	11712.3\\
8653	11712.3\\
8654	11712.3\\
8655	11712.3\\
8656	11712.3\\
8657	11712.3\\
8658	11712.3\\
8659	11712.3\\
8660	11712.3\\
8661	11712.3\\
8662	11712.3\\
8663	11712.3\\
8664	11712.3\\
8665	11712.3\\
8666	11712.3\\
8667	11712.3\\
8668	11712.3\\
8669	11712.3\\
8670	11712.3\\
8671	11712.3\\
8672	11712.3\\
8673	11712.3\\
8674	11712.3\\
8675	11712.3\\
8676	11712.3\\
8677	11712.3\\
8678	11712.3\\
8679	11712.3\\
8680	11712.3\\
8681	11712.3\\
8682	11712.3\\
8683	11712.3\\
8684	11712.3\\
8685	11712.3\\
8686	11712.3\\
8687	11712.3\\
8688	11337.3\\
8689	11337.3\\
8690	11337.3\\
8691	11337.3\\
8692	11337.3\\
8693	11337.3\\
8694	11337.3\\
8695	11337.3\\
8696	11337.3\\
8697	11337.3\\
8698	11712.3\\
8699	11712.3\\
8700	11712.3\\
8701	11712.3\\
8702	11712.3\\
8703	11712.3\\
8704	11712.3\\
8705	11712.3\\
8706	11712.3\\
8707	11712.3\\
8708	11712.3\\
8709	11712.3\\
8710	11712.3\\
8711	11712.3\\
8712	11712.3\\
8713	11712.3\\
8714	11712.3\\
8715	11712.3\\
8716	11712.3\\
8717	11712.3\\
8718	11712.3\\
8719	11712.3\\
8720	11712.3\\
8721	11712.3\\
8722	11712.3\\
8723	11712.3\\
8724	11712.3\\
8725	11712.3\\
8726	11712.3\\
8727	11712.3\\
8728	11712.3\\
8729	11712.3\\
8730	11712.3\\
8731	11712.3\\
8732	11712.3\\
8733	11712.3\\
8734	11712.3\\
8735	11712.3\\
8736	11712.3\\
8737	11712.3\\
8738	11712.3\\
8739	11712.3\\
8740	11712.3\\
8741	11712.3\\
8742	11712.3\\
8743	11712.3\\
8744	11712.3\\
8745	11712.3\\
8746	11712.3\\
8747	11712.3\\
8748	11712.3\\
8749	11712.3\\
8750	11712.3\\
8751	11712.3\\
8752	11712.3\\
8753	11712.3\\
8754	11712.3\\
8755	11712.3\\
8756	11712.3\\
8757	11712.3\\
8758	11712.3\\
8759	11712.3\\
8760	11712.3\\
};
\end{axis}
\end{tikzpicture}%
    \caption{Maximal production rate of Belgian generators}
    \label{fig:capa}
\end{figure}

\begin{figure}[H]
    \centering
    \setlength\fheight{4cm}
    \setlength\fwidth{0.8\textwidth}
    % This file was created by matlab2tikz.
% Minimal pgfplots version: 1.3
%
%The latest updates can be retrieved from
%  http://www.mathworks.com/matlabcentral/fileexchange/22022-matlab2tikz
%where you can also make suggestions and rate matlab2tikz.
%
\definecolor{mycolor1}{rgb}{0.84706,0.16078,0.00000}%
%
\begin{tikzpicture}

\begin{axis}[%
width=\fwidth,
height=\fheight,
at={(0\fwidth,0\fheight)},
scale only axis,
separate axis lines,
every outer x axis line/.append style={black},
every x tick label/.append style={font=\color{black}},
xmin=0,
xmax=8760,
xlabel={time [hour]},
xmajorgrids,
every outer y axis line/.append style={black},
every y tick label/.append style={font=\color{black}},
ymin=0.5,
ymax=2.5,
ylabel={ratio level},
ymajorgrids
]
\addplot [color=mycolor1,line width=1.3pt,solid,forget plot]
  table[row sep=crcr]{%
1	1.76277299026698\\
2	1.9362109758773\\
3	2.01383811417918\\
4	2.14556960277483\\
5	2.21691525772622\\
6	2.25450999839374\\
7	2.26315544633488\\
8	2.35434350380979\\
9	2.43391137968296\\
10	2.41028010959221\\
11	2.32649951900646\\
12	2.31734256915795\\
13	2.27152828955165\\
14	2.17701182593871\\
15	2.18950187093352\\
16	2.16738441375517\\
17	2.10329602620842\\
18	1.97969107222188\\
19	2.0982931513597\\
20	2.12403009322197\\
21	2.06913982707518\\
22	2.07783314631548\\
23	1.97292246943499\\
24	1.93649148120353\\
25	2.13109399445936\\
26	2.28501729833992\\
27	2.40536610933119\\
28	2.48094107909005\\
29	2.44346375633962\\
30	2.27445515666959\\
31	2.03308065689467\\
32	1.88155208674845\\
33	1.73216866221003\\
34	1.64000652518427\\
35	1.60057681297788\\
36	1.58151613686274\\
37	1.58058487502011\\
38	1.56606783677838\\
39	1.58544314319611\\
40	1.55974041513492\\
41	1.52603227566969\\
42	1.53144300145539\\
43	1.525416005635\\
44	1.52953071537887\\
45	1.59593745591488\\
46	1.67644591024023\\
47	1.63861553513276\\
48	1.65121653230106\\
49	1.77613915853804\\
50	1.94773556379149\\
51	2.03533445565709\\
52	2.11733591042659\\
53	2.11885435849448\\
54	2.0276289081425\\
55	1.85174931598301\\
56	1.73599673433243\\
57	1.66927759045363\\
58	1.66690536535875\\
59	1.61999094793536\\
60	1.59120753866144\\
61	1.58786097022828\\
62	1.62368626516258\\
63	1.66078900659206\\
64	1.64645781444153\\
65	1.59058698874881\\
66	1.51986726206597\\
67	1.61304799908995\\
68	1.57010443545918\\
69	1.62070690520079\\
70	1.66772940795751\\
71	1.62282496964554\\
72	1.61791168836035\\
73	1.75267714540524\\
74	1.83123061439868\\
75	1.92988005157148\\
76	1.98364437652562\\
77	1.99543556846451\\
78	1.98695024960769\\
79	1.90051604156909\\
80	1.85169720657404\\
81	1.71267194465335\\
82	1.64541711806509\\
83	1.64862209604406\\
84	1.60332965175916\\
85	1.67281694661606\\
86	1.59948840821261\\
87	1.61107300828397\\
88	1.60686449608187\\
89	1.56867037987429\\
90	1.51548309597428\\
91	1.57540234481252\\
92	1.62941053054905\\
93	1.70433420248443\\
94	1.63735890708814\\
95	1.63629059667911\\
96	1.60060428716605\\
97	1.60333361776825\\
98	1.708673988582\\
99	1.80413887459463\\
100	1.84530409485739\\
101	1.85130687020377\\
102	1.83900078997816\\
103	1.84939800153386\\
104	1.83193932063438\\
105	1.77380157351207\\
106	1.7142700447578\\
107	1.69152794751149\\
108	1.66794040072872\\
109	1.65017016272179\\
110	1.6452922441227\\
111	1.68088411609648\\
112	1.64843811576025\\
113	1.59744676877715\\
114	1.47500979093346\\
115	1.50530933054361\\
116	1.55305918071202\\
117	1.62607089550037\\
118	1.77848352311283\\
119	1.81048840080216\\
120	1.77720217708164\\
121	1.92955755097854\\
122	2.0586650212972\\
123	2.1598769632186\\
124	2.24455928924168\\
125	2.21228853250866\\
126	2.03376000182603\\
127	1.75874791088956\\
128	1.54699661737417\\
129	1.56648080193148\\
130	1.55673122189553\\
131	1.45903852996141\\
132	1.45690692494384\\
133	1.49567169832483\\
134	1.48083543336548\\
135	1.4779962361954\\
136	1.47477542990651\\
137	1.44906650377054\\
138	1.3604406435506\\
139	1.41912403058445\\
140	1.50388994692369\\
141	1.45971957118769\\
142	1.54310988359145\\
143	1.53823670087808\\
144	1.5957179230879\\
145	1.71352765438218\\
146	1.83434967945918\\
147	1.9298084065673\\
148	1.97640821697432\\
149	1.93860794318979\\
150	1.8216968497538\\
151	1.60093419425915\\
152	1.44256959766703\\
153	1.36260426079029\\
154	1.41204413651093\\
155	1.43865777808738\\
156	1.43382352713922\\
157	1.47426399251346\\
158	1.45456417218019\\
159	1.43658453492889\\
160	1.36700176783911\\
161	1.36056436315477\\
162	1.28154614249717\\
163	1.27501489230533\\
164	1.35115937186708\\
165	1.40542391889273\\
166	1.42776817561145\\
167	1.41974512425605\\
168	1.36596777402959\\
169	1.49388492672713\\
170	1.61341759269041\\
171	1.76656294926166\\
172	1.80540856979635\\
173	1.74095449895641\\
174	1.62969123554603\\
175	1.44203464831519\\
176	1.35588876972628\\
177	1.26549609812125\\
178	1.31240747128363\\
179	1.35282654998826\\
180	1.37722093617662\\
181	1.32003445397444\\
182	1.20120839897165\\
183	1.17864165503985\\
184	1.14837152696739\\
185	1.1324719261063\\
186	1.09369475619244\\
187	1.08296583518112\\
188	1.13393964490555\\
189	1.1534939896481\\
190	1.21425820839644\\
191	1.2072493286163\\
192	1.24985209442554\\
193	1.47786225976837\\
194	1.60703148308646\\
195	1.68771620741077\\
196	1.71999475243835\\
197	1.71036030092028\\
198	1.63790889272756\\
199	1.47141800969484\\
200	1.33413454908006\\
201	1.28769053648502\\
202	1.34515586211072\\
203	1.34616681130741\\
204	1.33537541599863\\
205	1.2984571870324\\
206	1.29047608089168\\
207	1.30252820413333\\
208	1.29644384757354\\
209	1.30748395845728\\
210	1.29851816762616\\
211	1.28592995826083\\
212	1.33319453925039\\
213	1.40518548656661\\
214	1.40175012865366\\
215	1.37204876271802\\
216	1.37931051578157\\
217	1.44203407418592\\
218	1.5329550020917\\
219	1.58922488996188\\
220	1.60005729224535\\
221	1.56657308444984\\
222	1.50507658910994\\
223	1.344867776067\\
224	1.21421912255514\\
225	1.18686565116026\\
226	1.241115355355\\
227	1.26697313609213\\
228	1.24434454821355\\
229	1.30892654343811\\
230	1.28631726467607\\
231	1.27072022446387\\
232	1.25486438864895\\
233	1.23053138918279\\
234	1.19175647135829\\
235	1.21057509119057\\
236	1.27222433362268\\
237	1.31865765608143\\
238	1.34677117714842\\
239	1.32770463768483\\
240	1.32075888993452\\
241	1.46066681046678\\
242	1.547044957561\\
243	1.58624427321258\\
244	1.64651415851665\\
245	1.67388840923648\\
246	1.65220191938573\\
247	1.59427635094351\\
248	1.59892002270501\\
249	1.45262103531662\\
250	1.40824512293663\\
251	1.39409060694883\\
252	1.39127721090953\\
253	1.3862368597072\\
254	1.38264197700962\\
255	1.41432073847967\\
256	1.40449281104224\\
257	1.38033142510119\\
258	1.30227087688349\\
259	1.32648268347874\\
260	1.36001145131298\\
261	1.43664635278679\\
262	1.41519867049676\\
263	1.36716440482475\\
264	1.44093747579285\\
265	1.42708775687099\\
266	1.45579779437206\\
267	1.5018144498523\\
268	1.54531878932517\\
269	1.5607828473973\\
270	1.57020401355159\\
271	1.55422790806552\\
272	1.53007104958006\\
273	1.47780263037668\\
274	1.44850251576378\\
275	1.44763465213742\\
276	1.48436113014352\\
277	1.48922489452031\\
278	1.52269035611753\\
279	1.58907165256958\\
280	1.55955281541541\\
281	1.52052486096639\\
282	1.39567062084007\\
283	1.3939770892222\\
284	1.42990942972091\\
285	1.50906604656547\\
286	1.58245128202283\\
287	1.60741687258773\\
288	1.5207598905816\\
289	1.67452532482118\\
290	1.77546489765398\\
291	1.85957918589554\\
292	1.91402148458577\\
293	1.89650070517672\\
294	1.74696708840451\\
295	1.50501748465052\\
296	1.3798084036594\\
297	1.38313294903386\\
298	1.3611042249003\\
299	1.31881420134862\\
300	1.3090446410692\\
301	1.33623091364315\\
302	1.33123361366905\\
303	1.34069078970888\\
304	1.32718363395993\\
305	1.30857466833095\\
306	1.28471155305273\\
307	1.25691034598817\\
308	1.33335631294786\\
309	1.36409933389232\\
310	1.42725057367219\\
311	1.4094913890812\\
312	1.43161837728613\\
313	1.574892929793\\
314	1.66336113250588\\
315	1.71557384708476\\
316	1.74978059408488\\
317	1.72650176668612\\
318	1.6179268743252\\
319	1.43004668188753\\
320	1.29662255454809\\
321	1.28774957483874\\
322	1.26190178514174\\
323	1.25859273979119\\
324	1.24772169123354\\
325	1.26472785981096\\
326	1.27342532226228\\
327	1.25336353826995\\
328	1.24188587235901\\
329	1.2256149716125\\
330	1.18724701068376\\
331	1.21491913485517\\
332	1.24042163532429\\
333	1.26041684345534\\
334	1.33549368655742\\
335	1.31910208885942\\
336	1.34178377902336\\
337	1.4414563995484\\
338	1.51649026335007\\
339	1.58874500977674\\
340	1.64401231065145\\
341	1.64333927896388\\
342	1.58487919328203\\
343	1.44765681706216\\
344	1.32349924033057\\
345	1.32040389513651\\
346	1.32683655341044\\
347	1.30646473725458\\
348	1.30183153724242\\
349	1.31938225968908\\
350	1.31653810883307\\
351	1.33728867550053\\
352	1.33449907545004\\
353	1.32684582732477\\
354	1.29624514138962\\
355	1.30342826413165\\
356	1.31598558017386\\
357	1.37149665391699\\
358	1.44411083117731\\
359	1.42273409673338\\
360	1.43288212789452\\
361	1.56156681111608\\
362	1.65805066814055\\
363	1.73383723304027\\
364	1.80027569463498\\
365	1.79076921017304\\
366	1.70237334242708\\
367	1.50627034577754\\
368	1.38280450851465\\
369	1.40363232494979\\
370	1.38039305901844\\
371	1.38561609737659\\
372	1.35886004449793\\
373	1.33061081263968\\
374	1.2762345619689\\
375	1.28868787611307\\
376	1.30370296482384\\
377	1.31588247829049\\
378	1.26232366833322\\
379	1.29904372010335\\
380	1.32493243669322\\
381	1.36408016984428\\
382	1.45092710060791\\
383	1.44301840641233\\
384	1.46803816272251\\
385	1.59014635166389\\
386	1.69124572927249\\
387	1.77323684872064\\
388	1.79692178576831\\
389	1.77942376903925\\
390	1.70842889842567\\
391	1.52400821779961\\
392	1.43431186190209\\
393	1.41364201785654\\
394	1.37109301865032\\
395	1.38852269101093\\
396	1.41491375462163\\
397	1.45016793839661\\
398	1.43598470585223\\
399	1.43376523935839\\
400	1.39392912081024\\
401	1.34787697522538\\
402	1.29405727168018\\
403	1.33263931757675\\
404	1.36772197434555\\
405	1.39394917679652\\
406	1.45785006353957\\
407	1.43687687000337\\
408	1.43756091893541\\
409	1.54513831675944\\
410	1.64113865265369\\
411	1.71862852481592\\
412	1.76524859711787\\
413	1.78858206471907\\
414	1.78215758801291\\
415	1.72590314407838\\
416	1.70769707835731\\
417	1.65374752305444\\
418	1.58853190547876\\
419	1.60792899698593\\
420	1.62537315846665\\
421	1.67945378457126\\
422	1.63589096353438\\
423	1.63413072563917\\
424	1.61398532914327\\
425	1.5598769260042\\
426	1.46610498216442\\
427	1.45798021940989\\
428	1.51970504444844\\
429	1.50500070260286\\
430	1.55883645125382\\
431	1.5642349073119\\
432	1.51760651649515\\
433	1.62134594600579\\
434	1.7299694453504\\
435	1.82609237324793\\
436	1.88109326894015\\
437	1.89297688525953\\
438	1.86954481734432\\
439	1.81858381661576\\
440	1.77531511191274\\
441	1.6962042059528\\
442	1.61739867472062\\
443	1.54522851459845\\
444	1.51219411784021\\
445	1.49357737919357\\
446	1.50762819614386\\
447	1.53846278673096\\
448	1.5369003657501\\
449	1.52938868049174\\
450	1.41576693939692\\
451	1.41954990165677\\
452	1.42689013367824\\
453	1.46615712933264\\
454	1.46560038353042\\
455	1.45953229505914\\
456	1.45174464719133\\
457	1.54359611563728\\
458	1.61802584355109\\
459	1.68459873285971\\
460	1.72303378197357\\
461	1.69861027943118\\
462	1.58686706626664\\
463	1.39976813220679\\
464	1.27520468506693\\
465	1.2847985204491\\
466	1.24220220381518\\
467	1.22855651510009\\
468	1.22016794031514\\
469	1.2450583394897\\
470	1.21590632775143\\
471	1.2219561900253\\
472	1.2158114460006\\
473	1.21235429323892\\
474	1.1916377178075\\
475	1.21335468320486\\
476	1.27196216754927\\
477	1.26932299767581\\
478	1.3755625747909\\
479	1.35404410395878\\
480	1.40172457699525\\
481	1.45344584899125\\
482	1.53708673344626\\
483	1.60293159732125\\
484	1.62303194170461\\
485	1.60835691862427\\
486	1.53757066792223\\
487	1.37327195455553\\
488	1.25152693074666\\
489	1.27313193920395\\
490	1.31139085201476\\
491	1.31294834155599\\
492	1.2786600410626\\
493	1.29458152892861\\
494	1.26897251758148\\
495	1.26593196571905\\
496	1.25878626657506\\
497	1.24713192049216\\
498	1.19395580505083\\
499	1.23576849880719\\
500	1.24268104965154\\
501	1.28854048682447\\
502	1.37442629548452\\
503	1.35954938540214\\
504	1.40446036298426\\
505	1.50441005927358\\
506	1.5996712178095\\
507	1.65111979905229\\
508	1.70210305802094\\
509	1.69733858565748\\
510	1.6188021154037\\
511	1.44519807008918\\
512	1.3313803862551\\
513	1.39051310205313\\
514	1.36482779727784\\
515	1.40483236387018\\
516	1.40081751171981\\
517	1.41690477471042\\
518	1.36939238212201\\
519	1.34687223843381\\
520	1.31890100056159\\
521	1.29432370040765\\
522	1.23448013905997\\
523	1.25388636083406\\
524	1.26339458873812\\
525	1.30136029501626\\
526	1.36069355959173\\
527	1.334616455149\\
528	1.3479062319039\\
529	1.45481212646417\\
530	1.52166570116622\\
531	1.56446933306228\\
532	1.61141915937192\\
533	1.60603872291545\\
534	1.53835877765572\\
535	1.38940337070939\\
536	1.27584549408996\\
537	1.29285316959904\\
538	1.28897411425697\\
539	1.29391135078365\\
540	1.30131126627011\\
541	1.31787921464603\\
542	1.29506739435805\\
543	1.30635435807575\\
544	1.30187104975722\\
545	1.3027247225854\\
546	1.25822590655936\\
547	1.29906035509768\\
548	1.33939292922362\\
549	1.38578264052682\\
550	1.45717525502922\\
551	1.43312031928977\\
552	1.44900088710226\\
553	1.55937696839204\\
554	1.63501686817876\\
555	1.69825399513108\\
556	1.71673484929485\\
557	1.68000184351256\\
558	1.59031421552231\\
559	1.39909073282687\\
560	1.27103151627006\\
561	1.29705670329196\\
562	1.31298671718102\\
563	1.33931474977036\\
564	1.34971069486066\\
565	1.36991431763676\\
566	1.37300173925463\\
567	1.3676158718002\\
568	1.3590842523661\\
569	1.34043924350929\\
570	1.28751488318873\\
571	1.31159591883922\\
572	1.34074268343609\\
573	1.38428162882628\\
574	1.46728353059486\\
575	1.42766947528636\\
576	1.43077365205217\\
577	1.46965592667972\\
578	1.55409949860337\\
579	1.63112223478094\\
580	1.68753651701994\\
581	1.72385709531491\\
582	1.7486000407942\\
583	1.71485279364862\\
584	1.65341377580699\\
585	1.55181332510421\\
586	1.50246746378642\\
587	1.47591604266792\\
588	1.47817806382629\\
589	1.47648757596111\\
590	1.49668606436226\\
591	1.53429645667311\\
592	1.54657927475067\\
593	1.54819179671975\\
594	1.50107870369551\\
595	1.50691695757208\\
596	1.55004630281238\\
597	1.63408831096859\\
598	1.70978822012148\\
599	1.6944591711967\\
600	1.66074523024259\\
601	1.84524928648012\\
602	1.97427958129578\\
603	1.9660597947546\\
604	1.99934685014634\\
605	1.9965626800371\\
606	1.95879041668584\\
607	1.88856356217999\\
608	1.84397626118771\\
609	1.74380632244057\\
610	1.67052369384502\\
611	1.62711929326756\\
612	1.62432160287537\\
613	1.68293577762177\\
614	1.64192684402621\\
615	1.69827504430018\\
616	1.71281082957701\\
617	1.71714891336484\\
618	1.64253052822686\\
619	1.70810125833876\\
620	1.71641323052057\\
621	1.610939613585\\
622	1.68466596648788\\
623	1.66924164434493\\
624	1.68926615694046\\
625	1.79173315575833\\
626	1.8895456475136\\
627	1.95497158077937\\
628	1.9632395147859\\
629	1.93272569793698\\
630	1.8068722183789\\
631	1.56907480502023\\
632	1.42438792911335\\
633	1.41629012897633\\
634	1.42021381795637\\
635	1.41449757964959\\
636	1.42114761720527\\
637	1.41624134453463\\
638	1.4105352894036\\
639	1.40796312305481\\
640	1.38083416960464\\
641	1.35758903685862\\
642	1.37651597731509\\
643	1.34087984697575\\
644	1.3943740736046\\
645	1.40900782611426\\
646	1.48405206524128\\
647	1.46675979578271\\
648	1.50042115530657\\
649	1.57704352113774\\
650	1.68550491404402\\
651	1.75785485568288\\
652	1.80529597597514\\
653	1.80472682539044\\
654	1.72292484599695\\
655	1.52731434800408\\
656	1.39117531321251\\
657	1.43240077204004\\
658	1.43476001300265\\
659	1.4282522784179\\
660	1.41731447177099\\
661	1.43466689998475\\
662	1.39747231627165\\
663	1.38870433466752\\
664	1.38557623323142\\
665	1.37930929628494\\
666	1.36208197108929\\
667	1.35616450006906\\
668	1.41736774690988\\
669	1.40405316011646\\
670	1.47956539000948\\
671	1.41024674014638\\
672	1.42344418122247\\
673	1.52325803237152\\
674	1.59533141440017\\
675	1.6505564709669\\
676	1.69652727946484\\
677	1.68846195401973\\
678	1.62230223076158\\
679	1.48692803428247\\
680	1.36224369716512\\
681	1.3989263607249\\
682	1.41634710975467\\
683	1.45541238110692\\
684	1.46336381528691\\
685	1.50759412554392\\
686	1.46231899419032\\
687	1.46864593217488\\
688	1.41624323824673\\
689	1.38063134541111\\
690	1.35477776778578\\
691	1.3124969732399\\
692	1.35691058341962\\
693	1.39050102140604\\
694	1.45048734994841\\
695	1.41514120107166\\
696	1.41968223960922\\
697	1.50658799381731\\
698	1.56716720961811\\
699	1.61672363131233\\
700	1.64387382392022\\
701	1.64758434846582\\
702	1.57114311960646\\
703	1.39538546285337\\
704	1.27453165043858\\
705	1.32637338316898\\
706	1.28784333473782\\
707	1.32630073733953\\
708	1.31373001990013\\
709	1.35353770978764\\
710	1.3389170180987\\
711	1.33448210637232\\
712	1.31638979835614\\
713	1.27744058876725\\
714	1.23086904047905\\
715	1.24328449590893\\
716	1.28443134705923\\
717	1.28831422952243\\
718	1.34981910891928\\
719	1.31751590455564\\
720	1.3250984687833\\
721	1.41029381786496\\
722	1.47271568453849\\
723	1.52511327109967\\
724	1.55530556132148\\
725	1.55298156179936\\
726	1.49084713950088\\
727	1.34552300752499\\
728	1.24355906130386\\
729	1.32191630991776\\
730	1.36580484448721\\
731	1.38818902663135\\
732	1.42453607201295\\
733	1.50568883311062\\
734	1.48015718000949\\
735	1.50877943097414\\
736	1.51444399329173\\
737	1.46671678543549\\
738	1.41540005388023\\
739	1.39675921516997\\
740	1.47420084196088\\
741	1.51309090010367\\
742	1.60314345034369\\
743	1.55891283964275\\
744	1.5663111153011\\
745	1.69312795693061\\
746	1.81661551150348\\
747	1.90260740418457\\
748	2.01874083549618\\
749	2.03977513107709\\
750	2.06088406889545\\
751	1.98006914670823\\
752	1.88272667411938\\
753	1.74453497973873\\
754	1.62420122970536\\
755	1.57111266755559\\
756	1.55641415600064\\
757	1.56620462673971\\
758	1.61877454000787\\
759	1.67990588330399\\
760	1.68256976758404\\
761	1.64874445800944\\
762	1.59728238502265\\
763	1.57546667661705\\
764	1.62058435974641\\
765	1.61118186076976\\
766	1.69074968040058\\
767	1.66389200712825\\
768	1.63749096504245\\
769	1.74521546561771\\
770	1.82523954905672\\
771	1.88369980800598\\
772	1.94603156291641\\
773	1.9931706110492\\
774	1.97706548395676\\
775	1.9657631653116\\
776	1.93629872464328\\
777	1.92762780438199\\
778	1.85127747329721\\
779	1.80614912803815\\
780	1.84183323828688\\
781	1.82971978275727\\
782	1.80673575813586\\
783	1.83209099976229\\
784	1.80146496904007\\
785	1.74112018505493\\
786	1.63620329096007\\
787	1.56239758790357\\
788	1.56133371021367\\
789	1.56526549258313\\
790	1.60291230934364\\
791	1.58096706003214\\
792	1.57160079126442\\
793	1.65024041418316\\
794	1.69549572236993\\
795	1.76453354348502\\
796	1.77771965203256\\
797	1.76434613352481\\
798	1.66792294655795\\
799	1.4726994966479\\
800	1.35123674826025\\
801	1.3879343116216\\
802	1.42620771940288\\
803	1.49182620814653\\
804	1.52114691085046\\
805	1.53887394517554\\
806	1.5011412453587\\
807	1.47808953502875\\
808	1.43111661761871\\
809	1.38545348625869\\
810	1.31397426159115\\
811	1.27892673805515\\
812	1.32452516571302\\
813	1.36838401324018\\
814	1.44268179624416\\
815	1.4221425234759\\
816	1.44205624768149\\
817	1.54633180058829\\
818	1.634870377183\\
819	1.70714180752753\\
820	1.76368526898442\\
821	1.76324107776226\\
822	1.68098099772347\\
823	1.48712609375201\\
824	1.36581602185607\\
825	1.41633156847929\\
826	1.41757913236043\\
827	1.42518289338379\\
828	1.41870378206515\\
829	1.48406492380573\\
830	1.4818217618441\\
831	1.49099733914413\\
832	1.45618971466303\\
833	1.42669944015305\\
834	1.37498702569358\\
835	1.33216594053293\\
836	1.39644531816491\\
837	1.41938768124388\\
838	1.52003720578367\\
839	1.49704043314214\\
840	1.521003682554\\
841	1.64521534923316\\
842	1.74865687424519\\
843	1.844020267108\\
844	1.89804313849664\\
845	1.88111088606937\\
846	1.80505130214543\\
847	1.58980536237336\\
848	1.48629598035734\\
849	1.42788900608666\\
850	1.44257561170201\\
851	1.42186849609995\\
852	1.42859525501049\\
853	1.4633932732358\\
854	1.4658599392089\\
855	1.50307760055265\\
856	1.47631532693688\\
857	1.44151765967084\\
858	1.36654744516495\\
859	1.37347738727311\\
860	1.42753918173368\\
861	1.48569428473421\\
862	1.54035622844191\\
863	1.52261597863579\\
864	1.53469167066894\\
865	1.65119713434204\\
866	1.75721840912235\\
867	1.82476338763658\\
868	1.86716994115569\\
869	1.84120614219687\\
870	1.76403278965096\\
871	1.57350134328106\\
872	1.49106334555056\\
873	1.49366230267771\\
874	1.47881111906073\\
875	1.47012370188843\\
876	1.45904716839532\\
877	1.45729861984901\\
878	1.42416624889929\\
879	1.42113976209236\\
880	1.42193104176669\\
881	1.40334629441289\\
882	1.35390958758701\\
883	1.37994750221938\\
884	1.41930795610026\\
885	1.50889179359612\\
886	1.55461034613988\\
887	1.5214481089317\\
888	1.54144564960204\\
889	1.66247581609949\\
890	1.74837899723901\\
891	1.8313750521402\\
892	1.87991195674547\\
893	1.87487045198552\\
894	1.80906090903083\\
895	1.61538029761186\\
896	1.50656396502617\\
897	1.50562551414465\\
898	1.47992271558814\\
899	1.42984968458926\\
900	1.36647573655411\\
901	1.399265528488\\
902	1.39152636079161\\
903	1.40714185134752\\
904	1.43237364118373\\
905	1.4796397109341\\
906	1.45499891954891\\
907	1.42255533792565\\
908	1.4637792016313\\
909	1.47571063943682\\
910	1.54560279274649\\
911	1.49637063699369\\
912	1.50644402335816\\
913	1.64095134846136\\
914	1.76655506853929\\
915	1.85169610411146\\
916	1.9550732967038\\
917	1.98811424541056\\
918	2.0295579324086\\
919	1.99456912037404\\
920	1.90831344503783\\
921	1.88240140879383\\
922	1.72506653982239\\
923	1.64743988972254\\
924	1.62896914972356\\
925	1.63508577264848\\
926	1.62693591095608\\
927	1.68858655290345\\
928	1.70165051868589\\
929	1.71211723253997\\
930	1.69522605053988\\
931	1.64213539082683\\
932	1.69361384529426\\
933	1.81790138662435\\
934	1.80487043961167\\
935	1.78939048649405\\
936	1.70467365940436\\
937	1.8437246357153\\
938	1.96291987526031\\
939	2.06159363965088\\
940	2.11625509932\\
941	2.14305138463847\\
942	2.14450642632571\\
943	2.07689007586728\\
944	2.03632435745472\\
945	1.98279245571543\\
946	1.88143228691825\\
947	1.85378804321048\\
948	1.82342656930013\\
949	1.81059847081694\\
950	1.86246835786853\\
951	1.93080628877364\\
952	1.90292890755801\\
953	1.85667907649698\\
954	1.74221015370484\\
955	1.6089250089362\\
956	1.6560649401284\\
957	1.64607088083511\\
958	1.71418491384621\\
959	1.68344246835954\\
960	1.60882677268093\\
961	1.67132819091707\\
962	1.72555516899171\\
963	1.77395723700183\\
964	1.79974412687181\\
965	1.75722522627825\\
966	1.66263727634158\\
967	1.46491959477324\\
968	1.31671221874543\\
969	1.31000900144211\\
970	1.31105534121874\\
971	1.34150114675608\\
972	1.32049163449593\\
973	1.35208248471259\\
974	1.30694859248855\\
975	1.3056796611363\\
976	1.28985572848652\\
977	1.28163973646872\\
978	1.23708156357472\\
979	1.23263819568989\\
980	1.27655062245345\\
981	1.3312759378392\\
982	1.38596304679113\\
983	1.37763506164926\\
984	1.39538360242385\\
985	1.50894801764543\\
986	1.61664389923648\\
987	1.67362610521443\\
988	1.72611573487201\\
989	1.72263354742266\\
990	1.63730140148844\\
991	1.49044782470049\\
992	1.42038857785016\\
993	1.44938679205965\\
994	1.45322839661172\\
995	1.49337573802593\\
996	1.47235105704674\\
997	1.51741748295604\\
998	1.49343186233545\\
999	1.47857161361539\\
1000	1.45848335473745\\
1001	1.42836115621628\\
1002	1.36301259836134\\
1003	1.34882501794498\\
1004	1.37386120893641\\
1005	1.4153343116423\\
1006	1.45354418012722\\
1007	1.43332551826903\\
1008	1.4427330067618\\
1009	1.57424484634047\\
1010	1.67224650123079\\
1011	1.74454373587716\\
1012	1.7673704406926\\
1013	1.75048302581888\\
1014	1.67351969563248\\
1015	1.49510195567119\\
1016	1.33013726410298\\
1017	1.39709039000402\\
1018	1.45587032411577\\
1019	1.55232287633046\\
1020	1.5776603903491\\
1021	1.58374480790873\\
1022	1.54456317195354\\
1023	1.53128507799663\\
1024	1.50049415357034\\
1025	1.46935885884903\\
1026	1.39705523322851\\
1027	1.3860071014324\\
1028	1.44062515351358\\
1029	1.48361467760865\\
1030	1.53571763920509\\
1031	1.52097841994156\\
1032	1.55695828662321\\
1033	1.68568835994277\\
1034	1.78707439912066\\
1035	1.84713248447074\\
1036	1.8790455424182\\
1037	1.85105993934949\\
1038	1.73016525699352\\
1039	1.54066388131943\\
1040	1.46423862487333\\
1041	1.44409807798525\\
1042	1.37072846645109\\
1043	1.35937500557471\\
1044	1.30765262988547\\
1045	1.31455165436822\\
1046	1.30069896496736\\
1047	1.3132226273768\\
1048	1.31433358389226\\
1049	1.30290503098499\\
1050	1.28851264875118\\
1051	1.29992203540171\\
1052	1.31439882460924\\
1053	1.38449667324796\\
1054	1.43353217012013\\
1055	1.41875802244783\\
1056	1.44082788992884\\
1057	1.51876833470765\\
1058	1.59651308186906\\
1059	1.66506821178456\\
1060	1.67614379268271\\
1061	1.65868065520219\\
1062	1.57788380915965\\
1063	1.3999346145719\\
1064	1.36923496882436\\
1065	1.4094496309803\\
1066	1.42822400878646\\
1067	1.45115474338077\\
1068	1.44200252952516\\
1069	1.43340494507921\\
1070	1.40251185098932\\
1071	1.39892990072047\\
1072	1.40960483957274\\
1073	1.41448926120727\\
1074	1.37757632871137\\
1075	1.43444644083881\\
1076	1.44988064418772\\
1077	1.51389512199336\\
1078	1.61649070187475\\
1079	1.57783043164336\\
1080	1.57948373189773\\
1081	1.73418352616715\\
1082	1.84317561747041\\
1083	1.94674236292082\\
1084	2.00828716588167\\
1085	2.01860126077626\\
1086	1.99586568962558\\
1087	1.91253637441223\\
1088	1.84217069234444\\
1089	1.81418068537273\\
1090	1.81752531641432\\
1091	1.71914294164772\\
1092	1.74266396939076\\
1093	1.80468110878044\\
1094	1.79802574778965\\
1095	1.83398481761457\\
1096	1.80525511400656\\
1097	1.77457408845398\\
1098	1.7097823020377\\
1099	1.66346135515253\\
1100	1.70830133148827\\
1101	1.77548399670531\\
1102	1.82616298123159\\
1103	1.74282534486713\\
1104	1.69227453217884\\
1105	1.78154237439685\\
1106	1.87606952171848\\
1107	1.94776430679222\\
1108	2.00055212707833\\
1109	2.0317298550167\\
1110	2.02677108784173\\
1111	1.9612154012034\\
1112	1.93058484874829\\
1113	1.87299832091689\\
1114	1.84890387234109\\
1115	1.84378229039902\\
1116	1.82978613907045\\
1117	1.85617918758327\\
1118	1.89172542611858\\
1119	1.94576232372104\\
1120	1.891292049772\\
1121	1.83464558254731\\
1122	1.7361635451403\\
1123	1.59472095999948\\
1124	1.53626895064466\\
1125	1.53266254771631\\
1126	1.56936328762474\\
1127	1.53237270227949\\
1128	1.52780875711811\\
1129	1.62914697623003\\
1130	1.70976532785867\\
1131	1.77444675760729\\
1132	1.80208436312904\\
1133	1.80256284269429\\
1134	1.69381241785247\\
1135	1.48624529764477\\
1136	1.35558423369473\\
1137	1.37382352078158\\
1138	1.41603960956189\\
1139	1.43085764994221\\
1140	1.4157662484423\\
1141	1.46799235969682\\
1142	1.47586247120746\\
1143	1.47914936054792\\
1144	1.46064706104408\\
1145	1.49755421461314\\
1146	1.43841391533394\\
1147	1.33490754047782\\
1148	1.34741107276169\\
1149	1.36916578503131\\
1150	1.43938449404708\\
1151	1.44415666785907\\
1152	1.46413645182811\\
1153	1.56676835124536\\
1154	1.68134294892787\\
1155	1.72749169674902\\
1156	1.77113549880987\\
1157	1.74370605836107\\
1158	1.6622781808038\\
1159	1.46883382597432\\
1160	1.40505027198735\\
1161	1.42505117805576\\
1162	1.4031398690306\\
1163	1.37754224128762\\
1164	1.34516865499778\\
1165	1.35212186955506\\
1166	1.33989112400067\\
1167	1.34192471492456\\
1168	1.3381780203424\\
1169	1.33352972431269\\
1170	1.29173057333978\\
1171	1.28172427725849\\
1172	1.32270899815576\\
1173	1.35658770252284\\
1174	1.44256060065255\\
1175	1.41899999129688\\
1176	1.44455728308238\\
1177	1.54716654607358\\
1178	1.63895035997102\\
1179	1.69528826144815\\
1180	1.71753325155394\\
1181	1.70604927609225\\
1182	1.61311983379675\\
1183	1.43011351138937\\
1184	1.34096832613863\\
1185	1.36494041505172\\
1186	1.35692566990408\\
1187	1.32413428124812\\
1188	1.29832548065922\\
1189	1.34451838854499\\
1190	1.33642490005658\\
1191	1.34528306547003\\
1192	1.33719362356672\\
1193	1.33143253026307\\
1194	1.30717042483666\\
1195	1.28699753265233\\
1196	1.30071540197641\\
1197	1.31977526311931\\
1198	1.38211224027533\\
1199	1.37213276154943\\
1200	1.40544091565439\\
1201	1.54600757081468\\
1202	1.65571047132507\\
1203	1.72908688272841\\
1204	1.81375617109401\\
1205	1.82549156106991\\
1206	1.72296749314694\\
1207	1.54964971186277\\
1208	1.50512413012915\\
1209	1.52318854159098\\
1210	1.51934242244397\\
1211	1.4904264290003\\
1212	1.42015123159394\\
1213	1.43765546374575\\
1214	1.42709584723878\\
1215	1.43125046260825\\
1216	1.4294696397711\\
1217	1.41446075282807\\
1218	1.38750100881769\\
1219	1.35966737410752\\
1220	1.39556448354121\\
1221	1.43928024060472\\
1222	1.49117563825254\\
1223	1.46093066891589\\
1224	1.47431628851414\\
1225	1.5702892436199\\
1226	1.65342391651488\\
1227	1.73958356940577\\
1228	1.77581647152714\\
1229	1.75252308001367\\
1230	1.65004500457537\\
1231	1.46956108396099\\
1232	1.41541568032266\\
1233	1.4404840930802\\
1234	1.44425556851776\\
1235	1.43109241740852\\
1236	1.43335937218678\\
1237	1.50783226912512\\
1238	1.51647939277927\\
1239	1.53089432825976\\
1240	1.52569688970089\\
1241	1.50764107425346\\
1242	1.46310086532589\\
1243	1.37171466336432\\
1244	1.41565065859734\\
1245	1.48133044168859\\
1246	1.56079048037675\\
1247	1.51530596275611\\
1248	1.52590567359812\\
1249	1.65830738857164\\
1250	1.77284659278082\\
1251	1.84627269789947\\
1252	1.92916958691043\\
1253	1.9514692096126\\
1254	1.91936108516281\\
1255	1.85311838917957\\
1256	1.78960761691917\\
1257	1.6984197898694\\
1258	1.60234131220283\\
1259	1.56517960449799\\
1260	1.5601282652347\\
1261	1.59418739433646\\
1262	1.66455205067402\\
1263	1.69594281764828\\
1264	1.68524320143251\\
1265	1.6637412476646\\
1266	1.59013997143677\\
1267	1.53330745590325\\
1268	1.5305837609147\\
1269	1.62017425838932\\
1270	1.69456206914855\\
1271	1.67524295653315\\
1272	1.61997175275389\\
1273	1.7128122291888\\
1274	1.83892109633137\\
1275	1.92821007330604\\
1276	1.99709661667188\\
1277	2.02470207018891\\
1278	2.0167708069749\\
1279	1.99346655253803\\
1280	2.0020826797628\\
1281	2.13494179147894\\
1282	2.08165802325483\\
1283	2.02736485298379\\
1284	2.05772736749435\\
1285	2.11739969432957\\
1286	2.08410972896309\\
1287	2.12454525944562\\
1288	2.06328396489125\\
1289	1.96729191393079\\
1290	1.83334565737404\\
1291	1.70053965614033\\
1292	1.65554380700852\\
1293	1.6306205619437\\
1294	1.6848442640391\\
1295	1.6936906605813\\
1296	1.68703692432894\\
1297	1.81779042153918\\
1298	1.91734803120658\\
1299	1.99753635343761\\
1300	2.04069685995945\\
1301	2.0041323861599\\
1302	1.86057238095737\\
1303	1.62660400581672\\
1304	1.53441390584699\\
1305	1.55285993887301\\
1306	1.5771227598962\\
1307	1.60086900917826\\
1308	1.65509076792046\\
1309	1.82705438066427\\
1310	1.69636652960844\\
1311	1.68122719697985\\
1312	1.61561404023547\\
1313	1.56220533760339\\
1314	1.48000616457412\\
1315	1.43375860464433\\
1316	1.42809870762589\\
1317	1.46380640575576\\
1318	1.54736593938664\\
1319	1.53630099156625\\
1320	1.5772069827285\\
1321	1.72666829983797\\
1322	1.83308546624763\\
1323	1.92860021277753\\
1324	1.96987612553902\\
1325	1.94407367345058\\
1326	1.82148681139616\\
1327	1.58634071762336\\
1328	1.4516716026449\\
1329	1.48433081978454\\
1330	1.52435060148051\\
1331	1.56482778914137\\
1332	1.55894405155434\\
1333	1.55291902719336\\
1334	1.52759096941009\\
1335	1.52278695974096\\
1336	1.49419614550913\\
1337	1.47481670109982\\
1338	1.42302706677315\\
1339	1.3852507082783\\
1340	1.39280115762667\\
1341	1.47150274418752\\
1342	1.49513474110871\\
1343	1.50841716221653\\
1344	1.525557710434\\
1345	1.59360655822387\\
1346	1.68621000822805\\
1347	1.7324332833404\\
1348	1.75403959686966\\
1349	1.72959305903211\\
1350	1.64206705515076\\
1351	1.46058117212\\
1352	1.40086565834898\\
1353	1.40356664142018\\
1354	1.37432197351209\\
1355	1.39485641607943\\
1356	1.4127757958776\\
1357	1.45468730151603\\
1358	1.45838102350342\\
1359	1.46411677847241\\
1360	1.44145826443804\\
1361	1.46346584666619\\
1362	1.43813661068\\
1363	1.34657057071765\\
1364	1.33715903934482\\
1365	1.40124385755666\\
1366	1.44231491980583\\
1367	1.42938421110351\\
1368	1.45351774799229\\
1369	1.58138740191801\\
1370	1.67672615186208\\
1371	1.75003893501209\\
1372	1.7999463368322\\
1373	1.78177422217253\\
1374	1.7168898775182\\
1375	1.53226876994737\\
1376	1.46047049375954\\
1377	1.55016230516151\\
1378	1.53381096483716\\
1379	1.51570071552217\\
1380	1.49514090130997\\
1381	1.46458343098298\\
1382	1.43410343242429\\
1383	1.4254082577596\\
1384	1.42084665961113\\
1385	1.41223462718777\\
1386	1.38083500718421\\
1387	1.3524611704329\\
1388	1.34219379276061\\
1389	1.37150598513743\\
1390	1.39924721265692\\
1391	1.38068118253955\\
1392	1.40196038423815\\
1393	1.53443022052717\\
1394	1.62900197407407\\
1395	1.68789366093929\\
1396	1.72018749582986\\
1397	1.69598503345778\\
1398	1.58750606444366\\
1399	1.40888645852836\\
1400	1.38150107632478\\
1401	1.40453323949878\\
1402	1.37938233679319\\
1403	1.34109685051927\\
1404	1.31970945985732\\
1405	1.30821381759089\\
1406	1.30131621996911\\
1407	1.31146954924539\\
1408	1.31744114181831\\
1409	1.33283286992557\\
1410	1.32136484552733\\
1411	1.32673567568304\\
1412	1.33547790600582\\
1413	1.35427568205696\\
1414	1.41551005276263\\
1415	1.36807806449716\\
1416	1.37530124004129\\
1417	1.50168182683362\\
1418	1.59919907420464\\
1419	1.67558168973102\\
1420	1.71702768699532\\
1421	1.72778884535952\\
1422	1.72628401587013\\
1423	1.66516483714779\\
1424	1.62587546375015\\
1425	1.61584059243753\\
1426	1.58094368368277\\
1427	1.54991786658728\\
1428	1.53441076548761\\
1429	1.48038882631661\\
1430	1.5178678587332\\
1431	1.55305750163302\\
1432	1.54358904734156\\
1433	1.55809171406616\\
1434	1.54138039413382\\
1435	1.46920816632231\\
1436	1.46867907159527\\
1437	1.54470153094711\\
1438	1.56091513706473\\
1439	1.53123459730367\\
1440	1.5202089901251\\
1441	1.59507679777851\\
1442	1.69391979925913\\
1443	1.77004241593774\\
1444	1.82562484134006\\
1445	1.85078851097035\\
1446	1.85846245218051\\
1447	1.83123107282825\\
1448	1.85596066594132\\
1449	1.94614240362187\\
1450	1.95439267279642\\
1451	2.0162418271528\\
1452	2.0495945739848\\
1453	2.11247447490105\\
1454	2.20363672188879\\
1455	2.24523750606185\\
1456	2.15487445730973\\
1457	2.02541537529442\\
1458	1.87220184323161\\
1459	1.8099706619807\\
1460	1.75205688404772\\
1461	1.68059939721125\\
1462	1.75945674734015\\
1463	1.74431498905925\\
1464	1.74905871448169\\
1465	1.86794069700398\\
1466	1.99117587025208\\
1467	2.07166175494405\\
1468	2.10029676957498\\
1469	2.0574036182004\\
1470	1.87453530116228\\
1471	1.63672415725931\\
1472	1.58548761528739\\
1473	1.54535297946052\\
1474	1.54907060445952\\
1475	1.55219104674505\\
1476	1.55324063961403\\
1477	1.60710110950456\\
1478	1.58358759005959\\
1479	1.55924947305533\\
1480	1.54052679571469\\
1481	1.516130193406\\
1482	1.44536937598167\\
1483	1.43815502854824\\
1484	1.42050435975442\\
1485	1.46217656232679\\
1486	1.50563157807399\\
1487	1.45940346329604\\
1488	1.45625588716567\\
1489	1.51921781955863\\
1490	1.59530272691583\\
1491	1.65566292720738\\
1492	1.66716210392124\\
1493	1.63366374585093\\
1494	1.54931815976876\\
1495	1.41246795665539\\
1496	1.3608661082163\\
1497	1.37959752525329\\
1498	1.36994551408288\\
1499	1.37178802752807\\
1500	1.37133145141073\\
1501	1.3932283232757\\
1502	1.39840905176265\\
1503	1.38499570788644\\
1504	1.36463223234458\\
1505	1.35778212231777\\
1506	1.30438025078657\\
1507	1.30175930997948\\
1508	1.28240855959225\\
1509	1.33135790458699\\
1510	1.35951327912801\\
1511	1.33088752828115\\
1512	1.342391139714\\
1513	1.43289680289457\\
1514	1.50498245090555\\
1515	1.56782241172476\\
1516	1.60466047494602\\
1517	1.58679205182118\\
1518	1.51876461029249\\
1519	1.37924235170322\\
1520	1.33039402821076\\
1521	1.3467614928955\\
1522	1.3780063479666\\
1523	1.38433063631076\\
1524	1.40269719637695\\
1525	1.44315335939091\\
1526	1.45084889138621\\
1527	1.45034427293121\\
1528	1.41940821161614\\
1529	1.40702302933709\\
1530	1.35243138606194\\
1531	1.3750792919581\\
1532	1.31808484248431\\
1533	1.3071664277574\\
1534	1.35908917568277\\
1535	1.33758717364877\\
1536	1.36848470516466\\
1537	1.44748587889809\\
1538	1.53734398499614\\
1539	1.64070652290812\\
1540	1.66447559468948\\
1541	1.65754605684342\\
1542	1.56999054904156\\
1543	1.43844868792774\\
1544	1.46128237330689\\
1545	1.47183338338961\\
1546	1.52462883813733\\
1547	1.48498313811263\\
1548	1.52963159133752\\
1549	1.58118044932827\\
1550	1.60220806681203\\
1551	1.61560514922772\\
1552	1.56428594459024\\
1553	1.52454022558502\\
1554	1.4306329873745\\
1555	1.41750683392254\\
1556	1.37367168967247\\
1557	1.36594801328238\\
1558	1.45631697971245\\
1559	1.44573644002731\\
1560	1.47489564176708\\
1561	1.62429619946677\\
1562	1.75563355881954\\
1563	1.83242635719538\\
1564	1.85746755308446\\
1565	1.81079859515086\\
1566	1.72942121047204\\
1567	1.5514024310081\\
1568	1.57275683891228\\
1569	1.57451989411418\\
1570	1.57044797064252\\
1571	1.54331330397133\\
1572	1.56869765284999\\
1573	1.61762568556321\\
1574	1.61594345964667\\
1575	1.58556763611253\\
1576	1.55275106936093\\
1577	1.52228056295366\\
1578	1.45817308710399\\
1579	1.45906048934699\\
1580	1.42562645991018\\
1581	1.40727641957975\\
1582	1.48243487965044\\
1583	1.45595926221496\\
1584	1.45482759021898\\
1585	1.55796116279909\\
1586	1.67982594510891\\
1587	1.79965601897131\\
1588	1.85857412869265\\
1589	1.87814112660555\\
1590	1.85735614877135\\
1591	1.8137446257556\\
1592	1.81673758816237\\
1593	1.76946495946296\\
1594	1.77145999668182\\
1595	1.84168713422355\\
1596	1.90583768608713\\
1597	1.92096984840545\\
1598	1.98665755001086\\
1599	2.00126782872708\\
1600	1.93035282183811\\
1601	1.83997915583286\\
1602	1.71090617574316\\
1603	1.59501272086269\\
1604	1.63727209576367\\
1605	1.73746729081917\\
1606	1.80149809904322\\
1607	1.73936625913349\\
1608	1.75815772123314\\
1609	1.889125721908\\
1610	1.99648928035087\\
1611	2.09178559434901\\
1612	2.10224433036551\\
1613	2.12153766438367\\
1614	2.14307574426379\\
1615	2.03818677033229\\
1616	2.06879852842859\\
1617	2.07258354101384\\
1618	2.05514477123435\\
1619	2.10928573639396\\
1620	2.11430343264358\\
1621	2.12668447545066\\
1622	2.2223374708055\\
1623	2.27960964757037\\
1624	2.17608798243745\\
1625	2.09466618423347\\
1626	1.92733995660852\\
1627	1.76677144041151\\
1628	1.66081583639286\\
1629	1.68424303642453\\
1630	1.72863444219807\\
1631	1.72047354995512\\
1632	1.69798580893511\\
1633	1.7573074431079\\
1634	1.83454350168685\\
1635	1.8759460895251\\
1636	1.89081150032069\\
1637	1.84736403073938\\
1638	1.69439280796366\\
1639	1.46090413248887\\
1640	1.44259810938236\\
1641	1.46252539770256\\
1642	1.49768144745225\\
1643	1.49464022591201\\
1644	1.50765745379114\\
1645	1.57525491826685\\
1646	1.59259902917661\\
1647	1.57555918127084\\
1648	1.54754161563465\\
1649	1.50573011635047\\
1650	1.41748685900654\\
1651	1.35382705508541\\
1652	1.38603411627146\\
1653	1.42827621005414\\
1654	1.47936141172255\\
1655	1.47484564850755\\
1656	1.50490501093331\\
1657	1.67978292973052\\
1658	1.7981577212744\\
1659	1.84470118513926\\
1660	1.84672085351022\\
1661	1.82153524846446\\
1662	1.70754224882545\\
1663	1.51158281735357\\
1664	1.50251572595272\\
1665	1.4722029846499\\
1666	1.4606812537432\\
1667	1.38784862406108\\
1668	1.41389564012361\\
1669	1.49860619408095\\
1670	1.5636632955075\\
1671	1.61789955546484\\
1672	1.58629574662685\\
1673	1.54833784102172\\
1674	1.46721619480739\\
1675	1.4008898023453\\
1676	1.4107517345248\\
1677	1.41050519916173\\
1678	1.49491242404481\\
1679	1.49032547220823\\
1680	1.49709869112513\\
1681	1.62459376572139\\
1682	1.71348889133602\\
1683	1.74860267977442\\
1684	1.76739178464789\\
1685	1.72817095292387\\
1686	1.61784194857867\\
1687	1.43723725282999\\
1688	1.44762286715377\\
1689	1.4742994115643\\
1690	1.4193922585723\\
1691	1.49056537934615\\
1692	1.53343044564681\\
1693	1.57964579307061\\
1694	1.58458509917406\\
1695	1.59603181550982\\
1696	1.55532110478149\\
1697	1.50231737394992\\
1698	1.41871205569991\\
1699	1.39938652586993\\
1700	1.34189301702986\\
1701	1.38007774878638\\
1702	1.40720814113439\\
1703	1.39591370423512\\
1704	1.45267068634472\\
1705	1.55608494773627\\
1706	1.63550364731917\\
1707	1.70027730810307\\
1708	1.71612924848952\\
1709	1.67414420269927\\
1710	1.58879508325501\\
1711	1.39725264495193\\
1712	1.34934101068217\\
1713	1.36806039323772\\
1714	1.41053448082051\\
1715	1.44337589280634\\
1716	1.48571185434296\\
1717	1.53365149349081\\
1718	1.54148051801797\\
1719	1.54027966659052\\
1720	1.53708589607791\\
1721	1.49447594209228\\
1722	1.43938879701447\\
1723	1.42703874132109\\
1724	1.35914291202739\\
1725	1.38285705858426\\
1726	1.42253724482949\\
1727	1.4131204135611\\
1728	1.4560139904295\\
1729	1.55890674253538\\
1730	1.63968905791443\\
1731	1.69579117670868\\
1732	1.71194400437713\\
1733	1.69625089320463\\
1734	1.60662952037256\\
1735	1.43447022362948\\
1736	1.4289854354799\\
1737	1.4488415761731\\
1738	1.49761255845512\\
1739	1.45615943305603\\
1740	1.48639204715042\\
1741	1.54486715584096\\
1742	1.58661335982603\\
1743	1.61318336238898\\
1744	1.58403227864481\\
1745	1.52612286033176\\
1746	1.45793657418767\\
1747	1.41480591744411\\
1748	1.39575256811726\\
1749	1.43059351379548\\
1750	1.41988541097448\\
1751	1.4161615610348\\
1752	1.45388974709468\\
1753	1.63695102681476\\
1754	1.73040313815786\\
1755	1.80297173784277\\
1756	1.85460222516377\\
1757	1.86971670899039\\
1758	1.87577281153016\\
1759	1.79371222171891\\
1760	1.75440137238114\\
1761	1.65395889676411\\
1762	1.56832947627\\
1763	1.56956781480838\\
1764	1.56037696387873\\
1765	1.57407886400857\\
1766	1.62979009579208\\
1767	1.69591165327552\\
1768	1.70376595381493\\
1769	1.67967940519809\\
1770	1.60789575288788\\
1771	1.54692549774427\\
1772	1.50221644691561\\
1773	1.54286511280289\\
1774	1.6119384417947\\
1775	1.62466493238114\\
1776	1.68238935792879\\
1777	1.7716261324079\\
1778	1.89617713467987\\
1779	2.00149403133383\\
1780	1.99146210053675\\
1781	2.00035429406344\\
1782	2.0151633650197\\
1783	1.94455241046621\\
1784	1.9756084817401\\
1785	1.89417172990412\\
1786	1.81842054568902\\
1787	1.77107591621315\\
1788	1.77099075145364\\
1789	1.79406632340052\\
1790	1.88447332974286\\
1791	1.98023216388053\\
1792	1.9749635871867\\
1793	1.93109558246915\\
1794	1.77785467034383\\
1795	1.65937214111547\\
1796	1.51796872338027\\
1797	1.51907720118241\\
1798	1.6061512061118\\
1799	1.61089500340504\\
1800	1.61359189767765\\
1801	1.74477292406613\\
1802	1.81596505981108\\
1803	1.87428337693944\\
1804	1.88650880602629\\
1805	1.82925161869589\\
1806	1.69887468305512\\
1807	1.46081176752934\\
1808	1.40905385442238\\
1809	1.41698212611693\\
1810	1.38726289260196\\
1811	1.32953623371697\\
1812	1.31625068922819\\
1813	1.37406028610578\\
1814	1.38674306413162\\
1815	1.38915323755735\\
1816	1.37112887059202\\
1817	1.34990954387465\\
1818	1.28264094207133\\
1819	1.2948907869418\\
1820	1.26619633130976\\
1821	1.31417235963518\\
1822	1.35149569338848\\
1823	1.36351366194984\\
1824	1.40257153346212\\
1825	1.513730380931\\
1826	1.61184346581024\\
1827	1.69042144381051\\
1828	1.70696220373394\\
1829	1.68995510552026\\
1830	1.60962686805314\\
1831	1.41206296728\\
1832	1.33076296333963\\
1833	1.35415638685079\\
1834	1.33511776681012\\
1835	1.36880960651899\\
1836	1.33416931831329\\
1837	1.35087188155977\\
1838	1.35529756790403\\
1839	1.38873404053547\\
1840	1.40624836752587\\
1841	1.40640025527753\\
1842	1.37249567976758\\
1843	1.36457179369257\\
1844	1.38659718123026\\
1845	1.43022767252236\\
1846	1.48702558171287\\
1847	1.47069785490974\\
1848	1.51691200035896\\
1849	1.65776043623007\\
1850	1.77343192701197\\
1851	1.82962263700183\\
1852	1.81513112737509\\
1853	1.76217233279003\\
1854	1.63928394254754\\
1855	1.44541225264629\\
1856	1.34912929828007\\
1857	1.45550972294685\\
1858	1.46969093720202\\
1859	1.38295157817602\\
1860	1.39391058278457\\
1861	1.43253194617518\\
1862	1.46235390781029\\
1863	1.48778842532059\\
1864	1.47081604193697\\
1865	1.44189731502055\\
1866	1.35027618622874\\
1867	1.37984530549828\\
1868	1.31642659283041\\
1869	1.33257362343575\\
1870	1.33077043969421\\
1871	1.40860751743824\\
1872	1.45810393898916\\
1873	1.55666609183253\\
1874	1.67740201521435\\
1875	1.76195674399085\\
1876	1.76303645233266\\
1877	1.74590758165143\\
1878	1.71498270705072\\
1879	1.5642467433521\\
1880	1.48086815545516\\
1881	1.63186999945816\\
1882	1.64552008873262\\
1883	1.66812570587663\\
1884	1.65545381804819\\
1885	1.73524766941985\\
1886	1.85360323686218\\
1887	1.87011244689236\\
1888	1.86027920708547\\
1889	1.70774128933448\\
1890	1.60462052134116\\
1891	1.59016896340545\\
1892	1.52660433873764\\
1893	1.47526706351551\\
1894	1.52803785843894\\
1895	1.54016786165953\\
1896	1.58028571960507\\
1897	1.73722323833139\\
1898	1.85400486933866\\
1899	1.914012621466\\
1900	1.89844945688985\\
1901	1.83326394187926\\
1902	1.70799935722249\\
1903	1.49678027829448\\
1904	1.3866625615625\\
1905	1.37372337893363\\
1906	1.35298081420226\\
1907	1.33920660534979\\
1908	1.29387807222542\\
1909	1.32512290018497\\
1910	1.36154073707651\\
1911	1.41851612703381\\
1912	1.44218756656511\\
1913	1.44384577123015\\
1914	1.42282456896212\\
1915	1.47825970224873\\
1916	1.42451614475714\\
1917	1.455640042439\\
1918	1.45528664248027\\
1919	1.44711405815108\\
1920	1.47201243838848\\
1921	1.62427523638119\\
1922	1.77139730908032\\
1923	1.87642804588696\\
1924	1.93309496789292\\
1925	1.96368739176182\\
1926	1.97087770099373\\
1927	1.94297786447001\\
1928	1.92995087174754\\
1929	1.87838890952762\\
1930	1.7792711013168\\
1931	1.70495593015082\\
1932	1.69857004074966\\
1933	1.68734060669256\\
1934	1.73386443571451\\
1935	1.77158333940517\\
1936	1.78734876680459\\
1937	1.74923622113497\\
1938	1.70176367764267\\
1939	1.73137680924973\\
1940	1.61287202835278\\
1941	1.67513699926892\\
1942	1.62295451078069\\
1943	1.62451691787511\\
1944	1.61397209209273\\
1945	1.68504432476538\\
1946	1.79372126494185\\
1947	1.87793708267298\\
1948	1.92803420941212\\
1949	1.93971387575075\\
1950	1.88929353978743\\
1951	1.86741046636232\\
1952	1.88782585664846\\
1953	1.86843253941805\\
1954	1.78787893189809\\
1955	1.76739587526286\\
1956	1.72022086450511\\
1957	1.68343823817002\\
1958	1.77387747410252\\
1959	1.84814932559063\\
1960	1.81449959543903\\
1961	1.73988279226465\\
1962	1.63993718597877\\
1963	1.54325771283558\\
1964	1.54941545471038\\
1965	1.54685819726808\\
1966	1.59180220585576\\
1967	1.61389201025443\\
1968	1.53186214711856\\
1969	1.6562602399322\\
1970	1.73973087426389\\
1971	1.77469881772308\\
1972	1.80224551631018\\
1973	1.74198639501277\\
1974	1.61683462525867\\
1975	1.42518622353491\\
1976	1.33854902225415\\
1977	1.41771661631\\
1978	1.43249233844722\\
1979	1.43883217883858\\
1980	1.4197472322135\\
1981	1.44705159016528\\
1982	1.44803986314577\\
1983	1.45744461705446\\
1984	1.41842554060967\\
1985	1.39739711782646\\
1986	1.32792677382235\\
1987	1.3465854806629\\
1988	1.28873378215774\\
1989	1.31291499090452\\
1990	1.36914038505421\\
1991	1.33957901983481\\
1992	1.38285870321052\\
1993	1.50137579804359\\
1994	1.58880382475931\\
1995	1.64586414667484\\
1996	1.66845089465294\\
1997	1.64475240938137\\
1998	1.54337933980264\\
1999	1.39240325192606\\
2000	1.41530942843544\\
2001	1.36886325024338\\
2002	1.3991903815184\\
2003	1.4461309881888\\
2004	1.46731355563547\\
2005	1.49913292312822\\
2006	1.49346677469573\\
2007	1.4818674447843\\
2008	1.44227426169692\\
2009	1.41599871637536\\
2010	1.37803380494369\\
2011	1.33580948086746\\
2012	1.32973233531299\\
2013	1.34299186609224\\
2014	1.40885021948812\\
2015	1.43363026919636\\
2016	1.44849030411333\\
2017	1.53298500377149\\
2018	1.6213728883071\\
2019	1.6829265434353\\
2020	1.71262939401528\\
2021	1.6750984709377\\
2022	1.55257374226178\\
2023	1.38995897042979\\
2024	1.37186886176007\\
2025	1.39993810395868\\
2026	1.40729234965379\\
2027	1.39500260148793\\
2028	1.39210137607535\\
2029	1.41152915559117\\
2030	1.29528327554444\\
2031	1.28415362028209\\
2032	1.25524114061316\\
2033	1.23094160661644\\
2034	1.18621696033999\\
2035	1.18240410194899\\
2036	1.08190854109592\\
2037	1.0959119559347\\
2038	1.1110907052666\\
2039	1.10803684505969\\
2040	1.11870695774836\\
2041	1.20388727638715\\
2042	1.28345400580662\\
2043	1.33579816205302\\
2044	1.35753965449947\\
2045	1.34846482013577\\
2046	1.27394573378156\\
2047	1.14935677581808\\
2048	1.11481004433171\\
2049	1.11957739316284\\
2050	1.12774296246977\\
2051	1.12200784757781\\
2052	1.14358709638932\\
2053	1.20029010197787\\
2054	1.192008986553\\
2055	1.18905711322673\\
2056	1.15910108325716\\
2057	1.13247250461958\\
2058	1.06861594711045\\
2059	1.02435859472313\\
2060	1.04605749810883\\
2061	1.06953696383025\\
2062	1.11499742408368\\
2063	1.10356003400873\\
2064	1.12421519946254\\
2065	1.20550297220782\\
2066	1.27156801943399\\
2067	1.32379843934259\\
2068	1.34476083523002\\
2069	1.33067384790344\\
2070	1.25564244415345\\
2071	1.13540613306394\\
2072	1.12897895652548\\
2073	1.14262546619346\\
2074	1.14315007029485\\
2075	1.18076292552699\\
2076	1.16168889149222\\
2077	1.17463786893323\\
2078	1.16826641675761\\
2079	1.19778965302969\\
2080	1.19973839331625\\
2081	1.19193636341114\\
2082	1.15125961691329\\
2083	1.11459134180412\\
2084	1.14897550548535\\
2085	1.13939546387112\\
2086	1.16865299055247\\
2087	1.16855593726436\\
2088	1.20441153324046\\
2089	1.31019504851979\\
2090	1.43017588308937\\
2091	1.49651417678919\\
2092	1.52075077256715\\
2093	1.515327431826\\
2094	1.49482567213735\\
2095	1.46602043876633\\
2096	1.43641650361988\\
2097	1.40867863395091\\
2098	1.40984645130894\\
2099	1.45200760216002\\
2100	1.49058486185025\\
2101	1.52617497497367\\
2102	1.57058733074416\\
2103	1.60690511993328\\
2104	1.5511024605411\\
2105	1.49981354137923\\
2106	1.40400515963621\\
2107	1.33345929458362\\
2108	1.35794117531352\\
2109	1.28065221816755\\
2110	1.35855488120386\\
2111	1.36621796861262\\
2112	1.37540271279591\\
2113	1.48131222245934\\
2114	1.53520784111136\\
2115	1.58602173467951\\
2116	1.61765212465098\\
2117	1.5973120276463\\
2118	1.56232629722312\\
2119	1.55303591365258\\
2120	1.52180530071842\\
2121	1.49047009408056\\
2122	1.50646127778187\\
2123	1.53953291035341\\
2124	1.57175875396113\\
2125	1.66050574526957\\
2126	1.71470279432474\\
2127	1.70445121349626\\
2128	1.66397360886721\\
2129	1.5476028960032\\
2130	1.49356049276508\\
2131	1.38743658983168\\
2132	1.31340688851364\\
2133	1.36124882512449\\
2134	1.32843669503136\\
2135	1.40977084844161\\
2136	1.45729178838005\\
2137	1.49614580710222\\
2138	1.53040319507457\\
2139	1.53251453491788\\
2140	1.49887061218682\\
2141	1.3946010972877\\
2142	1.21411778591768\\
2143	1.116401845099\\
2144	1.15250902237845\\
2145	1.17246909136943\\
2146	1.17265142155019\\
2147	1.21818821656209\\
2148	1.29386290766152\\
2149	1.3053255586896\\
2150	1.30080585202713\\
2151	1.2917248429784\\
2152	1.26136541747359\\
2153	1.20162832028489\\
2154	1.15252295848605\\
2155	1.18468043273594\\
2156	1.0998735026785\\
2157	1.14642716719058\\
2158	1.16865406271068\\
2159	1.23024645139652\\
2160	1.25362231827604\\
2161	1.33211962576126\\
2162	1.34698380147525\\
2163	1.36215022747716\\
2164	1.33600244187693\\
2165	1.29916447238862\\
2166	1.15194948808992\\
2167	1.11001375524118\\
2168	1.14488673052728\\
2169	1.15313012647366\\
2170	1.16079152357812\\
2171	1.16449684309263\\
2172	1.21238486835621\\
2173	1.21192453892436\\
2174	1.21235838656915\\
2175	1.20110650988779\\
2176	1.17383139258003\\
2177	1.1299454653307\\
2178	1.10273476349998\\
2179	1.13038853649776\\
2180	1.1099338367366\\
2181	1.10920691776337\\
2182	1.11015718993717\\
2183	1.15144718553193\\
2184	1.25921382441375\\
2185	1.34552152785436\\
2186	1.41173597875105\\
2187	1.42964089933858\\
2188	1.42990686604993\\
2189	1.35415286007748\\
2190	1.18344299113468\\
2191	1.1310611122114\\
2192	1.14664927399403\\
2193	1.15261377398376\\
2194	1.15189778975754\\
2195	1.16844673333761\\
2196	1.1856195638371\\
2197	1.1941319571991\\
2198	1.20821120391057\\
2199	1.2215083603636\\
2200	1.20930524558915\\
2201	1.17409309155889\\
2202	1.14103215067002\\
2203	1.13545361352693\\
2204	1.13161443864014\\
2205	1.12579861085768\\
2206	1.14445226780974\\
2207	1.20138378661713\\
2208	1.30723863397228\\
2209	1.39188712462851\\
2210	1.45251254156875\\
2211	1.45986604251502\\
2212	1.47227118092665\\
2213	1.39934378319654\\
2214	1.23844028773193\\
2215	1.15869458417374\\
2216	1.16949786960425\\
2217	1.17979775915639\\
2218	1.19823130494182\\
2219	1.18753840368005\\
2220	1.2250886064538\\
2221	1.22382026913614\\
2222	1.22585497227867\\
2223	1.23222683499767\\
2224	1.20314067917292\\
2225	1.14961444781581\\
2226	1.10736311791139\\
2227	1.08797243137346\\
2228	1.10002363382955\\
2229	1.11953994464978\\
2230	1.14611233540154\\
2231	1.17381456324535\\
2232	1.2977478906328\\
2233	1.36016548490334\\
2234	1.42509786365864\\
2235	1.42572573009011\\
2236	1.40517902888599\\
2237	1.32597586857305\\
2238	1.18271299065312\\
2239	1.10802290768129\\
2240	1.08731028979321\\
2241	1.08401313491717\\
2242	1.08067338329244\\
2243	1.0680488126064\\
2244	1.1131930311788\\
2245	1.1393891337524\\
2246	1.15816461818544\\
2247	1.18205908237356\\
2248	1.17642841463357\\
2249	1.140908849017\\
2250	1.12550155873831\\
2251	1.15847280999027\\
2252	1.14592203534015\\
2253	1.1755330712329\\
2254	1.1464245299415\\
2255	1.16997837449482\\
2256	1.29697677213853\\
2257	1.37378687541612\\
2258	1.43806431148202\\
2259	1.46098595064787\\
2260	1.46717699397583\\
2261	1.43343890518131\\
2262	1.37333557445879\\
2263	1.34218151070145\\
2264	1.27716739116472\\
2265	1.27732974659422\\
2266	1.27764520201606\\
2267	1.27095159630006\\
2268	1.30034698258039\\
2269	1.36041604747814\\
2270	1.40725192248557\\
2271	1.41701527624907\\
2272	1.41124911975999\\
2273	1.37816287363181\\
2274	1.30485957530233\\
2275	1.32077692006871\\
2276	1.29343242075199\\
2277	1.25842041112445\\
2278	1.28668748016478\\
2279	1.30099890388597\\
2280	1.39556208281814\\
2281	1.5087290453171\\
2282	1.59114428559744\\
2283	1.6301729105654\\
2284	1.66097522562648\\
2285	1.64443779745848\\
2286	1.59584227386098\\
2287	1.61554922888718\\
2288	1.56978848943736\\
2289	1.54249362260821\\
2290	1.52615539719417\\
2291	1.48310846470841\\
2292	1.42417189743338\\
2293	1.47952096023\\
2294	1.50532579406946\\
2295	1.52022621244826\\
2296	1.51645721091484\\
2297	1.45170418964588\\
2298	1.37416633057681\\
2299	1.35027105963606\\
2300	1.33165496668819\\
2301	1.3283142329384\\
2302	1.3444170015292\\
2303	1.37678338153215\\
2304	1.46473610788984\\
2305	1.54421205665638\\
2306	1.59932854886974\\
2307	1.60793414190795\\
2308	1.57665388746556\\
2309	1.47174786592852\\
2310	1.29400885220314\\
2311	1.25523808863086\\
2312	1.18286109221035\\
2313	1.1681242605485\\
2314	1.15480082543752\\
2315	1.16004771549767\\
2316	1.211633796579\\
2317	1.20768841413582\\
2318	1.21984005454083\\
2319	1.21419095067177\\
2320	1.21761609855799\\
2321	1.19308111030553\\
2322	1.21677834888146\\
2323	1.22214340779357\\
2324	1.20742662800161\\
2325	1.18180087565348\\
2326	1.19656697335478\\
2327	1.26069390956358\\
2328	1.39497790571682\\
2329	1.50023165710484\\
2330	1.56544020403265\\
2331	1.56901095728355\\
2332	1.53259375725895\\
2333	1.42787275750208\\
2334	1.27413302687551\\
2335	1.23941383329599\\
2336	1.19917492384489\\
2337	1.21103994948329\\
2338	1.2255178326586\\
2339	1.20933794822475\\
2340	1.26732308358186\\
2341	1.29611964690236\\
2342	1.30383912423664\\
2343	1.28165410849572\\
2344	1.2533760913301\\
2345	1.2015410388857\\
2346	1.23596112182939\\
2347	1.20691228716618\\
2348	1.11549810896798\\
2349	1.11125423281417\\
2350	1.10636252310033\\
2351	1.14376279668373\\
2352	1.22712499281646\\
2353	1.29491079146687\\
2354	1.33190399029072\\
2355	1.34773059361839\\
2356	1.32938663110779\\
2357	1.24264483112728\\
2358	1.13293049534791\\
2359	1.11267699175638\\
2360	1.10057644635096\\
2361	1.09978611589633\\
2362	1.12602394011848\\
2363	1.13414669067561\\
2364	1.19785898931502\\
2365	1.20835748683161\\
2366	1.24089396469019\\
2367	1.23911736496541\\
2368	1.21392546699658\\
2369	1.1713382388073\\
2370	1.1328321329232\\
2371	1.13512934890657\\
2372	1.14831144470472\\
2373	1.11885250293989\\
2374	1.12834542423324\\
2375	1.13471019761957\\
2376	1.2139496538016\\
2377	1.27680322990487\\
2378	1.31740345964107\\
2379	1.33742271042549\\
2380	1.3220500545829\\
2381	1.25009340761784\\
2382	1.12937306303011\\
2383	1.08872555154845\\
2384	1.0859101409649\\
2385	1.09983895749211\\
2386	1.11292667325356\\
2387	1.10472141941404\\
2388	1.16518225274952\\
2389	1.16092851526386\\
2390	1.16820609957146\\
2391	1.16529179925058\\
2392	1.15591962091945\\
2393	1.11789864764514\\
2394	1.09138081307153\\
2395	1.12954848815279\\
2396	1.11476865433246\\
2397	1.10266350896588\\
2398	1.11299973377227\\
2399	1.13232597274871\\
2400	1.23596810017628\\
2401	1.30312620252703\\
2402	1.35557894683369\\
2403	1.35933114394581\\
2404	1.35076964330021\\
2405	1.2754276065423\\
2406	1.17084074956832\\
2407	1.12778048488893\\
2408	1.10816412634679\\
2409	1.11599554095574\\
2410	1.1227061693306\\
2411	1.13464506371662\\
2412	1.16594656474901\\
2413	1.19708970858241\\
2414	1.21722249427645\\
2415	1.20561389513338\\
2416	1.20463798048718\\
2417	1.18566024069001\\
2418	1.16225154205194\\
2419	1.17610162869819\\
2420	1.1683460842755\\
2421	1.15260895809329\\
2422	1.15072053200747\\
2423	1.19311220132861\\
2424	1.27377917554457\\
2425	1.32528548355039\\
2426	1.38071594944198\\
2427	1.39911935794532\\
2428	1.39425816247011\\
2429	1.38433294778029\\
2430	1.3538383965631\\
2431	1.33107747473093\\
2432	1.2774136360115\\
2433	1.24581401493381\\
2434	1.262437634681\\
2435	1.31408767459794\\
2436	1.37146152088835\\
2437	1.44419537863653\\
2438	1.47896079596764\\
2439	1.46626633746875\\
2440	1.42568565183467\\
2441	1.36225250767165\\
2442	1.30713130162413\\
2443	1.28644698761417\\
2444	1.32456780771198\\
2445	1.28844063274588\\
2446	1.29210304011601\\
2447	1.3168093075537\\
2448	1.54569738496007\\
2449	1.62736138347947\\
2450	1.7045923436738\\
2451	1.72674686779652\\
2452	1.7340870591381\\
2453	1.70518596964295\\
2454	1.65752649389264\\
2455	1.64511110629238\\
2456	1.57916352005726\\
2457	1.58175052099711\\
2458	1.60050300250743\\
2459	1.5820747930671\\
2460	1.61063580045866\\
2461	1.67252834397789\\
2462	1.77702014454312\\
2463	1.79703931306778\\
2464	1.76590888932577\\
2465	1.66667138274024\\
2466	1.5449015580938\\
2467	1.46074089378204\\
2468	1.43803710081372\\
2469	1.38838094894339\\
2470	1.40372923548479\\
2471	1.45584617975335\\
2472	1.61719600001167\\
2473	1.70775233333337\\
2474	1.75254506910591\\
2475	1.78607547524171\\
2476	1.76725290535933\\
2477	1.64183909110945\\
2478	1.45301256859489\\
2479	1.37636833656826\\
2480	1.33346978661795\\
2481	1.34392089353061\\
2482	1.3434472233046\\
2483	1.34745134594234\\
2484	1.37970314352013\\
2485	1.38316479514204\\
2486	1.41236118578886\\
2487	1.41388764191191\\
2488	1.41335883718781\\
2489	1.35896353754277\\
2490	1.31844429544859\\
2491	1.29305720559928\\
2492	1.30037472447523\\
2493	1.26697672779742\\
2494	1.26727038353837\\
2495	1.28824194297648\\
2496	1.38623373761712\\
2497	1.43934534523005\\
2498	1.48340133081076\\
2499	1.50217478062299\\
2500	1.46297310781972\\
2501	1.39116748016938\\
2502	1.26329772666588\\
2503	1.21895637108433\\
2504	1.1924497042498\\
2505	1.20370512168887\\
2506	1.23334766699142\\
2507	1.24999000590635\\
2508	1.28734932154567\\
2509	1.30151438558564\\
2510	1.31839038732055\\
2511	1.32686232028049\\
2512	1.332024832477\\
2513	1.30247769249356\\
2514	1.25415712774159\\
2515	1.23369692757872\\
2516	1.23312244960995\\
2517	1.22336680697958\\
2518	1.20166523346958\\
2519	1.22887194156125\\
2520	1.31450536298898\\
2521	1.38009457525389\\
2522	1.4359783829818\\
2523	1.44451415574379\\
2524	1.41915123222873\\
2525	1.35278771784686\\
2526	1.23761028834652\\
2527	1.21705014516083\\
2528	1.2466677047671\\
2529	1.26551694863836\\
2530	1.27246142292551\\
2531	1.31104831961575\\
2532	1.33884714159865\\
2533	1.39074561949833\\
2534	1.38985965783256\\
2535	1.37864150280566\\
2536	1.35118644883546\\
2537	1.30043760379654\\
2538	1.23791873695638\\
2539	1.22335178988923\\
2540	1.2150831203055\\
2541	1.22740174990761\\
2542	1.2041427860058\\
2543	1.22626414502963\\
2544	1.3257494308657\\
2545	1.41532957472805\\
2546	1.46862943375303\\
2547	1.49311595766811\\
2548	1.50071439546335\\
2549	1.44818591056557\\
2550	1.32255930881883\\
2551	1.31509650543724\\
2552	1.28811764301072\\
2553	1.29578260450617\\
2554	1.31393769432316\\
2555	1.30654375495691\\
2556	1.37693620950614\\
2557	1.40562418581113\\
2558	1.43701424903589\\
2559	1.45008136909763\\
2560	1.41032666258952\\
2561	1.35243965499214\\
2562	1.31210344809688\\
2563	1.27924474953415\\
2564	1.26669281962999\\
2565	1.26364900979838\\
2566	1.25422801384745\\
2567	1.26574055607666\\
2568	1.37190347980952\\
2569	1.46488785693239\\
2570	1.5218000207516\\
2571	1.55038106770243\\
2572	1.54502494835417\\
2573	1.48823078155585\\
2574	1.35395768790798\\
2575	1.3308065528321\\
2576	1.30948607967116\\
2577	1.32211777319854\\
2578	1.33263929982467\\
2579	1.34597165877552\\
2580	1.36396720694012\\
2581	1.37916767424388\\
2582	1.40989921998949\\
2583	1.44075327336176\\
2584	1.42214830077937\\
2585	1.40468679552339\\
2586	1.36040222312283\\
2587	1.3549085945791\\
2588	1.38480887868092\\
2589	1.33352683146\\
2590	1.29724101831336\\
2591	1.30261583551133\\
2592	1.40713128225155\\
2593	1.50871457425981\\
2594	1.57341550225123\\
2595	1.5973228399871\\
2596	1.57370639269312\\
2597	1.56121004509647\\
2598	1.55418459100415\\
2599	1.52751603309624\\
2600	1.46403494318029\\
2601	1.44210635919353\\
2602	1.44630151224562\\
2603	1.42585724321132\\
2604	1.47737977346363\\
2605	1.57612154921073\\
2606	1.66100078748186\\
2607	1.71211465293857\\
2608	1.67929253862999\\
2609	1.59927165455158\\
2610	1.52396571082729\\
2611	1.48713710546762\\
2612	1.52094339575594\\
2613	1.52911614714488\\
2614	1.54768260481383\\
2615	1.5416127649904\\
2616	1.61856833304569\\
2617	1.74106241600103\\
2618	1.83185623282691\\
2619	1.86828133254571\\
2620	1.89703668379607\\
2621	1.86807388539285\\
2622	1.81196320830395\\
2623	1.80843274479394\\
2624	1.71069176403488\\
2625	1.6479618801787\\
2626	1.64995776787393\\
2627	1.66726745470556\\
2628	1.66794981749934\\
2629	1.70408763318624\\
2630	1.75590622185129\\
2631	1.79656337300872\\
2632	1.79147744249027\\
2633	1.67742110609325\\
2634	1.57899791437189\\
2635	1.49631156078159\\
2636	1.50874887081565\\
2637	1.47504863725732\\
2638	1.46047923775862\\
2639	1.48338298520586\\
2640	1.52516311773139\\
2641	1.60570141587666\\
2642	1.67184350094325\\
2643	1.72181440921284\\
2644	1.72082342892872\\
2645	1.68697727278875\\
2646	1.65485873693513\\
2647	1.67827766747208\\
2648	1.65453975028823\\
2649	1.64363758853879\\
2650	1.63577564464782\\
2651	1.69424943735699\\
2652	1.70098717301376\\
2653	1.78295520099112\\
2654	1.76240185606654\\
2655	1.75210621846416\\
2656	1.68799302566996\\
2657	1.55672892586897\\
2658	1.4538390391255\\
2659	1.42124163222008\\
2660	1.44536597511798\\
2661	1.45961274024156\\
2662	1.41020908243657\\
2663	1.4048154155646\\
2664	1.4804967012563\\
2665	1.57313067912719\\
2666	1.6274398129024\\
2667	1.63627378609345\\
2668	1.60910554437361\\
2669	1.50282722502828\\
2670	1.32299779316529\\
2671	1.22669761984199\\
2672	1.22954978065047\\
2673	1.24780555541982\\
2674	1.25465639791746\\
2675	1.28042838891942\\
2676	1.31514594187815\\
2677	1.31761052018671\\
2678	1.31432434008686\\
2679	1.31722110046228\\
2680	1.29392284056994\\
2681	1.24717700919381\\
2682	1.21800507493495\\
2683	1.20879285070284\\
2684	1.23578407752829\\
2685	1.22996456377151\\
2686	1.21329049008808\\
2687	1.25293507399839\\
2688	1.34865456141543\\
2689	1.43130831560934\\
2690	1.49884483603819\\
2691	1.5024228229863\\
2692	1.4874919077175\\
2693	1.39517921630652\\
2694	1.26401747435107\\
2695	1.2353476057694\\
2696	1.24713172814904\\
2697	1.25815683882602\\
2698	1.26741469348328\\
2699	1.26154593379812\\
2700	1.29229590674641\\
2701	1.28764042680623\\
2702	1.2721902130548\\
2703	1.26205589542907\\
2704	1.24752587886291\\
2705	1.23890993771766\\
2706	1.2031434691523\\
2707	1.20218488132074\\
2708	1.23313693442622\\
2709	1.18176480239868\\
2710	1.19314769892681\\
2711	1.25258946910222\\
2712	1.36125299596497\\
2713	1.43402336888919\\
2714	1.48380376034148\\
2715	1.51524114552537\\
2716	1.49818136240814\\
2717	1.42163242990463\\
2718	1.27702806006731\\
2719	1.20178935662885\\
2720	1.178797053884\\
2721	1.1652975620232\\
2722	1.18551497147898\\
2723	1.16695964491392\\
2724	1.19578442930858\\
2725	1.20275222536312\\
2726	1.22174214673781\\
2727	1.21266160170127\\
2728	1.21030129589833\\
2729	1.19911763530019\\
2730	1.19022154734175\\
2731	1.20828219369449\\
2732	1.23070343073045\\
2733	1.24742136063187\\
2734	1.2824652174485\\
2735	1.33433815408127\\
2736	1.46301780501111\\
2737	1.55630565801715\\
2738	1.6057028590418\\
2739	1.62933942023015\\
2740	1.6141781325833\\
2741	1.53775276113273\\
2742	1.39323826000312\\
2743	1.30925036937679\\
2744	1.33483412649255\\
2745	1.31224352036548\\
2746	1.36642636398326\\
2747	1.39444728635259\\
2748	1.4780944632717\\
2749	1.4666870932021\\
2750	1.42378118244147\\
2751	1.42043143865979\\
2752	1.38587628785267\\
2753	1.34384151492922\\
2754	1.29459083551554\\
2755	1.29432903753025\\
2756	1.30822385187185\\
2757	1.3016902080609\\
2758	1.32136015835214\\
2759	1.35848239855843\\
2760	1.48119278531623\\
2761	1.58008463165729\\
2762	1.66767657818132\\
2763	1.67967815102297\\
2764	1.67730297004474\\
2765	1.67521556707195\\
2766	1.67268258171578\\
2767	1.63718700122009\\
2768	1.57800394495903\\
2769	1.56130106475819\\
2770	1.58562013348031\\
2771	1.61149536855122\\
2772	1.65821998136198\\
2773	1.68843575910541\\
2774	1.70417603672251\\
2775	1.67226143571594\\
2776	1.60621061603248\\
2777	1.52167646694611\\
2778	1.49131732340811\\
2779	1.43301030487338\\
2780	1.44817201370191\\
2781	1.44237868590036\\
2782	1.47691444354499\\
2783	1.49009983499636\\
2784	1.60427812771019\\
2785	1.70244621603674\\
2786	1.78007231815471\\
2787	1.80544104673792\\
2788	1.81311487548213\\
2789	1.80425101087162\\
2790	1.80808164568491\\
2791	1.79133119362592\\
2792	1.73739985306386\\
2793	1.68931564172938\\
2794	1.68320999129273\\
2795	1.66633265017385\\
2796	1.65071332945278\\
2797	1.72126732447105\\
2798	1.73434652428938\\
2799	1.73492416114588\\
2800	1.71420446289097\\
2801	1.6274714455784\\
2802	1.54627627379111\\
2803	1.4931559960267\\
2804	1.49774013818914\\
2805	1.44695407204197\\
2806	1.45828420864959\\
2807	1.47895861899845\\
2808	1.57787051033176\\
2809	1.64429877050676\\
2810	1.69582001933958\\
2811	1.71235534657726\\
2812	1.6707279994851\\
2813	1.54013008368874\\
2814	1.37200233921205\\
2815	1.29868981497081\\
2816	1.28382853347628\\
2817	1.31894654825127\\
2818	1.32458761099919\\
2819	1.26584671240289\\
2820	1.26554290570379\\
2821	1.27102103940332\\
2822	1.27993106051647\\
2823	1.26932686719795\\
2824	1.2543207748956\\
2825	1.24447648554614\\
2826	1.22474699411563\\
2827	1.21185486311818\\
2828	1.23045663192858\\
2829	1.25117558902969\\
2830	1.23885718857785\\
2831	1.29554839643957\\
2832	1.39332762852082\\
2833	1.46314333848957\\
2834	1.51566237110565\\
2835	1.5251588171127\\
2836	1.51705806008034\\
2837	1.44480533272068\\
2838	1.33719038076799\\
2839	1.26532296518722\\
2840	1.21677202227245\\
2841	1.19666266911192\\
2842	1.20940272807806\\
2843	1.18362737453917\\
2844	1.20438516914433\\
2845	1.18370504336814\\
2846	1.18110698296097\\
2847	1.19523799640249\\
2848	1.20085974816424\\
2849	1.21549219083678\\
2850	1.21124063265605\\
2851	1.24462962584249\\
2852	1.26546566979005\\
2853	1.25971113912951\\
2854	1.24892347626245\\
2855	1.32417431005052\\
2856	1.40261835661346\\
2857	1.47859156172959\\
2858	1.52578652882179\\
2859	1.53707213685171\\
2860	1.52280349600958\\
2861	1.44534021338878\\
2862	1.32389088326081\\
2863	1.2465290901564\\
2864	1.23412603175284\\
2865	1.21576758756779\\
2866	1.21904881118622\\
2867	1.2360235560359\\
2868	1.27114026938582\\
2869	1.27871632207873\\
2870	1.30627613491297\\
2871	1.32512585350739\\
2872	1.32816831982406\\
2873	1.30160400881182\\
2874	1.28942760009927\\
2875	1.29449212758005\\
2876	1.30693192822099\\
2877	1.27406527698735\\
2878	1.25841353095422\\
2879	1.30963771720873\\
2880	1.43010610189892\\
2881	1.52477329599021\\
2882	1.58874374482988\\
2883	1.60661241263019\\
2884	1.629757570769\\
2885	1.62853745639297\\
2886	1.65310480793013\\
2887	1.68766117742958\\
2888	1.6631226097365\\
2889	1.64186001712883\\
2890	1.6252510900762\\
2891	1.59061216967792\\
2892	1.55140129452212\\
2893	1.61738490604192\\
2894	1.61419472971525\\
2895	1.61495585383827\\
2896	1.57397704418468\\
2897	1.51098625450491\\
2898	1.43741773661076\\
2899	1.41809226156888\\
2900	1.428802803546\\
2901	1.40151283847672\\
2902	1.35279757587901\\
2903	1.39134562035848\\
2904	1.55456881074965\\
2905	1.63400570030979\\
2906	1.70249452680334\\
2907	1.73018143975509\\
2908	1.73168025654978\\
2909	1.6771247358467\\
2910	1.58843407935424\\
2911	1.57713899770444\\
2912	1.54596078333973\\
2913	1.52267116201845\\
2914	1.55617464973651\\
2915	1.59050857665978\\
2916	1.61239952680814\\
2917	1.70964493965081\\
2918	1.74440909301212\\
2919	1.76507481200469\\
2920	1.71473539823691\\
2921	1.62706576329385\\
2922	1.55550033549874\\
2923	1.52915932865571\\
2924	1.51395064158707\\
2925	1.50451546374351\\
2926	1.46820489367934\\
2927	1.49459043555808\\
2928	1.61781575794469\\
2929	1.72366304261695\\
2930	1.78333966311024\\
2931	1.81346255026161\\
2932	1.7987125301121\\
2933	1.76529966573959\\
2934	1.74601003137112\\
2935	1.68253730217259\\
2936	1.61694097676544\\
2937	1.5957957896605\\
2938	1.61851495369323\\
2939	1.66785373892567\\
2940	1.69630664634076\\
2941	1.75829569578537\\
2942	1.77759551935295\\
2943	1.77757332137522\\
2944	1.72582342555817\\
2945	1.62656786340615\\
2946	1.54097084354168\\
2947	1.52427231962874\\
2948	1.57560345589223\\
2949	1.55240481514119\\
2950	1.53471817787526\\
2951	1.5509424582395\\
2952	1.52351696296232\\
2953	1.61379966329205\\
2954	1.66273556517011\\
2955	1.67793909044954\\
2956	1.67848842832389\\
2957	1.66136948920028\\
2958	1.67549071026775\\
2959	1.65052711144532\\
2960	1.64931262430877\\
2961	1.66064439311481\\
2962	1.68715179093269\\
2963	1.75487471771076\\
2964	1.79954582857799\\
2965	1.86665912446352\\
2966	1.95414966060664\\
2967	1.94143281038642\\
2968	1.82860362202457\\
2969	1.6875560073716\\
2970	1.5384119903671\\
2971	1.46691585024388\\
2972	1.45954491751169\\
2973	1.46860504090133\\
2974	1.44197355214318\\
2975	1.48112909777243\\
2976	1.58900991947209\\
2977	1.6492061331502\\
2978	1.71388445829987\\
2979	1.72470307103565\\
2980	1.68447170322625\\
2981	1.55606142056463\\
2982	1.41841206424193\\
2983	1.29695004162293\\
2984	1.338195700182\\
2985	1.3297561639832\\
2986	1.37611483226351\\
2987	1.43497606678726\\
2988	1.49534379869371\\
2989	1.481761823544\\
2990	1.47382190330478\\
2991	1.47541053270482\\
2992	1.43336632820087\\
2993	1.36241554998217\\
2994	1.31087485207229\\
2995	1.29113743083812\\
2996	1.31682387785546\\
2997	1.32750182583552\\
2998	1.33048173685631\\
2999	1.40861353083552\\
3000	1.56076167455419\\
3001	1.67599651653322\\
3002	1.7294154584597\\
3003	1.74009422597288\\
3004	1.71774984621465\\
3005	1.61692322990636\\
3006	1.47850401254996\\
3007	1.35059384100487\\
3008	1.30459516062067\\
3009	1.26770404482015\\
3010	1.2762124884976\\
3011	1.28134582306326\\
3012	1.34426333785239\\
3013	1.36172003810569\\
3014	1.3887744028308\\
3015	1.38545924027411\\
3016	1.37086953195588\\
3017	1.32714928937876\\
3018	1.34245623679965\\
3019	1.29938150042557\\
3020	1.32426603645872\\
3021	1.35583972344898\\
3022	1.33596022024731\\
3023	1.3859898084714\\
3024	1.50200658174123\\
3025	1.58608762425269\\
3026	1.65562854565878\\
3027	1.66423576880101\\
3028	1.64481730942736\\
3029	1.54808780127294\\
3030	1.42007644774353\\
3031	1.30878371148241\\
3032	1.28166751973344\\
3033	1.29234008311136\\
3034	1.31002587401842\\
3035	1.30427608034439\\
3036	1.31755521035858\\
3037	1.32522874408011\\
3038	1.35854755281493\\
3039	1.38178429816443\\
3040	1.39097311358233\\
3041	1.36434427908645\\
3042	1.33124061005872\\
3043	1.31572314804926\\
3044	1.3297374337493\\
3045	1.32947915426585\\
3046	1.29635457056906\\
3047	1.35059207001644\\
3048	1.50032302888691\\
3049	1.60498624213676\\
3050	1.68662725343702\\
3051	1.70696386801105\\
3052	1.66736813734637\\
3053	1.61344195176821\\
3054	1.40603709271621\\
3055	1.34528064733439\\
3056	1.3221421471203\\
3057	1.32139106253662\\
3058	1.27997927227389\\
3059	1.25407539053781\\
3060	1.28372343744995\\
3061	1.26598023397519\\
3062	1.26862492807164\\
3063	1.26270468159325\\
3064	1.26218266888468\\
3065	1.25059564642472\\
3066	1.25021022269834\\
3067	1.3022864974074\\
3068	1.35508599186767\\
3069	1.3782350388799\\
3070	1.3820788579868\\
3071	1.43163050132078\\
3072	1.58988690472834\\
3073	1.67028533827484\\
3074	1.74015861573471\\
3075	1.73151310362504\\
3076	1.71638138196528\\
3077	1.63018572320245\\
3078	1.4530828198303\\
3079	1.31993906522812\\
3080	1.31742931785501\\
3081	1.32135672946574\\
3082	1.353382866227\\
3083	1.34634730731225\\
3084	1.42737527321303\\
3085	1.48612329758788\\
3086	1.51946319262438\\
3087	1.54306102752882\\
3088	1.53981889928724\\
3089	1.50032024807032\\
3090	1.47062430579373\\
3091	1.46360793638203\\
3092	1.51393992545702\\
3093	1.4852142636451\\
3094	1.39154309617949\\
3095	1.41843282478392\\
3096	1.49342054911632\\
3097	1.60355968662351\\
3098	1.67384770423102\\
3099	1.71152111854656\\
3100	1.73423616609154\\
3101	1.72163099094091\\
3102	1.71159207144974\\
3103	1.64152514705456\\
3104	1.51765348824139\\
3105	1.4716710304345\\
3106	1.44252826235223\\
3107	1.47004942929362\\
3108	1.5083871206395\\
3109	1.54624016048612\\
3110	1.5787221569081\\
3111	1.59213760118659\\
3112	1.6501451685875\\
3113	1.63955618459683\\
3114	1.61104604917153\\
3115	1.61745313446041\\
3116	1.67819384598397\\
3117	1.67334323146827\\
3118	1.63722299352282\\
3119	1.66283991279328\\
3120	1.77691201942842\\
3121	1.93607386954957\\
3122	2.06972624774105\\
3123	2.1043370835617\\
3124	2.12993053724072\\
3125	2.13577251994424\\
3126	2.19042992453352\\
3127	2.15147441283882\\
3128	2.05320727765085\\
3129	2.01139375427059\\
3130	2.01867439192536\\
3131	1.98362000411839\\
3132	2.02836476388767\\
3133	2.08034576463229\\
3134	2.17165314406387\\
3135	2.21223088022835\\
3136	2.14689964238336\\
3137	1.9835123396369\\
3138	1.78882966133314\\
3139	1.75795179311635\\
3140	1.74273403878859\\
3141	1.7152032102684\\
3142	1.66637260408444\\
3143	1.70496535135049\\
3144	1.83336886883493\\
3145	1.88258737399275\\
3146	1.95646249763428\\
3147	1.96413238889372\\
3148	1.88073177761805\\
3149	1.75069824773805\\
3150	1.54580987554458\\
3151	1.46514814535492\\
3152	1.40413632318222\\
3153	1.40051430053381\\
3154	1.37558542282945\\
3155	1.37418369419289\\
3156	1.41724063215675\\
3157	1.37128415405376\\
3158	1.38246983586313\\
3159	1.37225886363751\\
3160	1.3329702583026\\
3161	1.30363658520668\\
3162	1.29416241439231\\
3163	1.27440547797985\\
3164	1.24445558903586\\
3165	1.25528550322493\\
3166	1.23799437553153\\
3167	1.27905789398656\\
3168	1.39224546044384\\
3169	1.50986882628868\\
3170	1.55887661334803\\
3171	1.57809572222256\\
3172	1.5586766377644\\
3173	1.49435135921643\\
3174	1.37123558807571\\
3175	1.34305585762286\\
3176	1.24255569762498\\
3177	1.23537644425507\\
3178	1.29837813822632\\
3179	1.28337902347451\\
3180	1.24083145192207\\
3181	1.29444198365764\\
3182	1.32153674810697\\
3183	1.29774817817401\\
3184	1.30196184405084\\
3185	1.32814280381696\\
3186	1.26683662589449\\
3187	1.27569286217303\\
3188	1.29070143650631\\
3189	1.29198690585418\\
3190	1.24834314431301\\
3191	1.28950288584908\\
3192	1.38272853047242\\
3193	1.46599908062339\\
3194	1.52034396325737\\
3195	1.52995941965078\\
3196	1.50347131047847\\
3197	1.42955984741482\\
3198	1.32049766529779\\
3199	1.25054157981764\\
3200	1.26082302086434\\
3201	1.29941380018177\\
3202	1.28592943344824\\
3203	1.29045615936632\\
3204	1.30169471950475\\
3205	1.33496054951618\\
3206	1.3544019218714\\
3207	1.3639260846577\\
3208	1.37584623837483\\
3209	1.35214859245916\\
3210	1.33242692074987\\
3211	1.32746522494677\\
3212	1.33731681799662\\
3213	1.3263898735602\\
3214	1.26936111348618\\
3215	1.30445716949973\\
3216	1.44246011537526\\
3217	1.49851716302101\\
3218	1.55083629323821\\
3219	1.55366668843999\\
3220	1.53244213027414\\
3221	1.45920226390856\\
3222	1.35022700545099\\
3223	1.28515074242048\\
3224	1.2708480252462\\
3225	1.2329522749346\\
3226	1.27798153475977\\
3227	1.30246231641602\\
3228	1.38889942110729\\
3229	1.43364028199064\\
3230	1.44495080544145\\
3231	1.44600731876367\\
3232	1.40910499713108\\
3233	1.34890346315035\\
3234	1.30676732477788\\
3235	1.29386389211209\\
3236	1.30856256469901\\
3237	1.31006625289483\\
3238	1.26800212298789\\
3239	1.30865292632285\\
3240	1.42724570366435\\
3241	1.50679937714101\\
3242	1.55907504962773\\
3243	1.56906439538011\\
3244	1.54191781281377\\
3245	1.47319600560829\\
3246	1.37062417179471\\
3247	1.27718499194161\\
3248	1.28212003873568\\
3249	1.30493013743004\\
3250	1.32547037469605\\
3251	1.36550828585216\\
3252	1.43192875736553\\
3253	1.4619326759724\\
3254	1.47386320317288\\
3255	1.50279779829202\\
3256	1.46519943905474\\
3257	1.41323725301835\\
3258	1.36612455317384\\
3259	1.34119200998597\\
3260	1.36033342631902\\
3261	1.37056829218819\\
3262	1.35748584674218\\
3263	1.35648293354937\\
3264	1.41080157154587\\
3265	1.50125597383828\\
3266	1.56078635123696\\
3267	1.58762764953674\\
3268	1.61783774733454\\
3269	1.61701021316971\\
3270	1.60827152048674\\
3271	1.54497607805231\\
3272	1.48652498910574\\
3273	1.46826172757546\\
3274	1.53616065520112\\
3275	1.58586872548615\\
3276	1.60254407817294\\
3277	1.68784516402285\\
3278	1.74584272920093\\
3279	1.73444523898619\\
3280	1.65245852744748\\
3281	1.57340246974063\\
3282	1.48832378213865\\
3283	1.43984835709963\\
3284	1.43645397406676\\
3285	1.42232314629383\\
3286	1.44393386121314\\
3287	1.42786570356729\\
3288	1.49746540357466\\
3289	1.63496492738467\\
3290	1.69324669635898\\
3291	1.72066030650189\\
3292	1.74713149984665\\
3293	1.76529860950729\\
3294	1.77427419508714\\
3295	1.73549026407388\\
3296	1.71547203534736\\
3297	1.72431697247761\\
3298	1.73853371122601\\
3299	1.77535380920386\\
3300	1.820779951402\\
3301	1.89047697451506\\
3302	1.92386840185119\\
3303	1.93568387580577\\
3304	1.85495539768721\\
3305	1.7691041369076\\
3306	1.63628723145542\\
3307	1.59375246503559\\
3308	1.57118432292419\\
3309	1.54559225738105\\
3310	1.51385771298304\\
3311	1.54286481145688\\
3312	1.66803299437027\\
3313	1.74501142046612\\
3314	1.80178353154346\\
3315	1.82022141355844\\
3316	1.762166920183\\
3317	1.66618826096858\\
3318	1.52012671003818\\
3319	1.41327606614251\\
3320	1.36495471377756\\
3321	1.3640878188046\\
3322	1.33921797211575\\
3323	1.34136625115527\\
3324	1.40612392870974\\
3325	1.40946783764641\\
3326	1.40431330862732\\
3327	1.39879108613499\\
3328	1.39042418118953\\
3329	1.3476821251704\\
3330	1.32245442822313\\
3331	1.32356323368919\\
3332	1.36678957723635\\
3333	1.39687840039249\\
3334	1.36427347009581\\
3335	1.42005436657419\\
3336	1.51526753618535\\
3337	1.60788820461274\\
3338	1.69213000193478\\
3339	1.71212827575141\\
3340	1.67691086800042\\
3341	1.62648124545162\\
3342	1.47843370047879\\
3343	1.37848362399023\\
3344	1.34469831308842\\
3345	1.33398414041637\\
3346	1.32849985764695\\
3347	1.34741368236285\\
3348	1.39773196240349\\
3349	1.39399599906194\\
3350	1.38551068242474\\
3351	1.35717826909054\\
3352	1.33437154095154\\
3353	1.29041304807137\\
3354	1.26542765416139\\
3355	1.25039313661155\\
3356	1.28509087171595\\
3357	1.26343660807086\\
3358	1.28864420916392\\
3359	1.32428747942227\\
3360	1.39525390942568\\
3361	1.46630120959867\\
3362	1.5172129512798\\
3363	1.54764232693269\\
3364	1.54034598996163\\
3365	1.49173601371162\\
3366	1.38316075251922\\
3367	1.34367526566712\\
3368	1.28910195809662\\
3369	1.25789369144479\\
3370	1.2888901291317\\
3371	1.23852090080989\\
3372	1.25332475765947\\
3373	1.27201195479046\\
3374	1.26673365668897\\
3375	1.25153391965774\\
3376	1.25934842024774\\
3377	1.26268975711321\\
3378	1.26722588911801\\
3379	1.28095761601508\\
3380	1.31767504707716\\
3381	1.31407404516357\\
3382	1.29539281255859\\
3383	1.35987790300152\\
3384	1.61594707365686\\
3385	1.79132064893098\\
3386	1.88158807311251\\
3387	1.88498566944586\\
3388	1.83470898003691\\
3389	1.75172180260946\\
3390	1.61043732266182\\
3391	1.46174167686307\\
3392	1.41419331358836\\
3393	1.40621282217625\\
3394	1.41142860182528\\
3395	1.44241639462054\\
3396	1.47023232122599\\
3397	1.45786604259448\\
3398	1.44533032896692\\
3399	1.43725284207534\\
3400	1.39740530670913\\
3401	1.37551035472875\\
3402	1.31920626336905\\
3403	1.36069091555962\\
3404	1.38980705388244\\
3405	1.36027274468698\\
3406	1.32902099523441\\
3407	1.39468891860774\\
3408	1.53317494916422\\
3409	1.62609327187703\\
3410	1.69954087617078\\
3411	1.70771432402119\\
3412	1.67132046542995\\
3413	1.61681720885396\\
3414	1.48012681342022\\
3415	1.3568684019681\\
3416	1.36188409987195\\
3417	1.3117467075406\\
3418	1.33385717357527\\
3419	1.33656111021969\\
3420	1.3675393183701\\
3421	1.36149987419627\\
3422	1.38584414019356\\
3423	1.3966567294372\\
3424	1.4030710212163\\
3425	1.35784220568983\\
3426	1.36756383204136\\
3427	1.37976397616426\\
3428	1.41919030573347\\
3429	1.46161474451579\\
3430	1.41432579392631\\
3431	1.43317467021934\\
3432	1.47281290949161\\
3433	1.56262111655571\\
3434	1.65824144324321\\
3435	1.71897535347258\\
3436	1.73110599194389\\
3437	1.76780695619659\\
3438	1.77095005917909\\
3439	1.71205712485509\\
3440	1.61649351778959\\
3441	1.61658726191139\\
3442	1.62908229567601\\
3443	1.66303976607236\\
3444	1.67561359593678\\
3445	1.77665004896764\\
3446	1.87334909791026\\
3447	1.8752947251489\\
3448	1.77542451937641\\
3449	1.66532414885776\\
3450	1.63191509002869\\
3451	1.58629056808435\\
3452	1.58261894229468\\
3453	1.63859011570852\\
3454	1.59260664817838\\
3455	1.66645475579075\\
3456	1.66612758869444\\
3457	1.73972144184186\\
3458	1.81971841162414\\
3459	1.83114987830095\\
3460	1.82510456271422\\
3461	1.84343365726735\\
3462	1.83157298540778\\
3463	1.76263196798128\\
3464	1.71961242101704\\
3465	1.70314618114934\\
3466	1.74095215346314\\
3467	1.78390212485685\\
3468	1.83014626247074\\
3469	1.89367040396159\\
3470	1.93140397903628\\
3471	1.89052395293083\\
3472	1.78628779462794\\
3473	1.66958798776458\\
3474	1.58703982480947\\
3475	1.54581442190018\\
3476	1.5769731172769\\
3477	1.56518434112045\\
3478	1.53831612931946\\
3479	1.57039814742227\\
3480	1.6521681058744\\
3481	1.74690180514905\\
3482	1.80655700125968\\
3483	1.81393693080482\\
3484	1.76954660165159\\
3485	1.69021367543069\\
3486	1.5310755320018\\
3487	1.34391415225513\\
3488	1.27868654411925\\
3489	1.24700879416168\\
3490	1.23776786633892\\
3491	1.23157299031289\\
3492	1.26797862801856\\
3493	1.2972487467166\\
3494	1.3048590693374\\
3495	1.30339370910691\\
3496	1.29783702054218\\
3497	1.28531041336101\\
3498	1.29053900488729\\
3499	1.33223357432914\\
3500	1.37441568037042\\
3501	1.40784757122763\\
3502	1.38343535839022\\
3503	1.42552532145859\\
3504	1.53789818600151\\
3505	1.58998504082942\\
3506	1.62810408197475\\
3507	1.6384141514744\\
3508	1.61469155202715\\
3509	1.56937500998751\\
3510	1.44870688543115\\
3511	1.34463938739476\\
3512	1.28059928334195\\
3513	1.24417346306498\\
3514	1.2214141630151\\
3515	1.23325792239952\\
3516	1.30998976853912\\
3517	1.28920939929713\\
3518	1.29006933877187\\
3519	1.25517423987638\\
3520	1.25404998788724\\
3521	1.25404682091675\\
3522	1.25346041950707\\
3523	1.29198659276015\\
3524	1.32273717481753\\
3525	1.3563345266784\\
3526	1.37617768795751\\
3527	1.40351327182402\\
3528	1.47528592363244\\
3529	1.54977655640936\\
3530	1.60034656336183\\
3531	1.60954591059092\\
3532	1.60616978223731\\
3533	1.55074400062311\\
3534	1.42347663104524\\
3535	1.30492611218036\\
3536	1.27542846589857\\
3537	1.24733913451658\\
3538	1.22524108598256\\
3539	1.22925654942286\\
3540	1.2224559181606\\
3541	1.23483608598032\\
3542	1.25898616100278\\
3543	1.27368045401772\\
3544	1.28592727030291\\
3545	1.26844210878786\\
3546	1.26869286857035\\
3547	1.29235150789516\\
3548	1.33120280021136\\
3549	1.34974275682199\\
3550	1.2971371181664\\
3551	1.33792135016413\\
3552	1.44924036273717\\
3553	1.5322292704943\\
3554	1.59541124545483\\
3555	1.63692208253824\\
3556	1.67068094196564\\
3557	1.70789486634917\\
3558	1.73523597918368\\
3559	1.69363641761737\\
3560	1.66118589227309\\
3561	1.61452761340408\\
3562	1.5768185379352\\
3563	1.56290645534421\\
3564	1.56942468634392\\
3565	1.63121988180104\\
3566	1.69733271691534\\
3567	1.71987436393198\\
3568	1.72557267772143\\
3569	1.69485947596681\\
3570	1.61069916956613\\
3571	1.54356529479618\\
3572	1.54515463718691\\
3573	1.51367739373205\\
3574	1.48851466791849\\
3575	1.55164483091248\\
3576	1.65441397155151\\
3577	1.73799538310475\\
3578	1.79117709459355\\
3579	1.81466938479262\\
3580	1.80218643834785\\
3581	1.76458771335886\\
3582	1.67702915228013\\
3583	1.59336596965713\\
3584	1.57537732329565\\
3585	1.57210313093325\\
3586	1.58404664371216\\
3587	1.55727568008797\\
3588	1.64919111447361\\
3589	1.69701652389237\\
3590	1.69149932758696\\
3591	1.72908952688909\\
3592	1.69174026650414\\
3593	1.68209387234293\\
3594	1.56634409964128\\
3595	1.51817310532781\\
3596	1.53716074891189\\
3597	1.55808618572393\\
3598	1.52417399160467\\
3599	1.53782941496754\\
3600	1.58858663847764\\
3601	1.67199244323598\\
3602	1.73394240018804\\
3603	1.7479021388503\\
3604	1.73969212078291\\
3605	1.75539473342113\\
3606	1.73430482651318\\
3607	1.66235908240646\\
3608	1.6234061318884\\
3609	1.57803804549519\\
3610	1.65983879327477\\
3611	1.66175692227782\\
3612	1.66393352235026\\
3613	1.69016979981357\\
3614	1.72678397910067\\
3615	1.72777192426236\\
3616	1.67422774028148\\
3617	1.59633273805267\\
3618	1.55661358997438\\
3619	1.52934884427107\\
3620	1.53210255384682\\
3621	1.54015754546434\\
3622	1.51095469502889\\
3623	1.50711765102413\\
3624	1.61253361673398\\
3625	1.69744073510795\\
3626	1.75096968820139\\
3627	1.73850099942952\\
3628	1.74609267288591\\
3629	1.77611295202285\\
3630	1.79176103724345\\
3631	1.74673999066845\\
3632	1.69757602657492\\
3633	1.65146137170387\\
3634	1.63942105286224\\
3635	1.64507404734072\\
3636	1.68035714813859\\
3637	1.73625795890791\\
3638	1.78730805691229\\
3639	1.80976327351318\\
3640	1.77145973174921\\
3641	1.66016125248812\\
3642	1.53404974995119\\
3643	1.50320855497487\\
3644	1.48680998465372\\
3645	1.46942726060924\\
3646	1.48781914445541\\
3647	1.45103288749957\\
3648	1.56929971760489\\
3649	1.64793459294226\\
3650	1.6946471472596\\
3651	1.70286727209379\\
3652	1.66667934061087\\
3653	1.59714174342091\\
3654	1.44073901465412\\
3655	1.32601138932868\\
3656	1.30301323734043\\
3657	1.31067097561998\\
3658	1.3336125897619\\
3659	1.34137039082495\\
3660	1.38190246380524\\
3661	1.37328253615264\\
3662	1.36781987980151\\
3663	1.36881231923175\\
3664	1.3483000075915\\
3665	1.30116201115787\\
3666	1.32899703918848\\
3667	1.31806256180328\\
3668	1.34906816405225\\
3669	1.34621159344954\\
3670	1.2898806268131\\
3671	1.35759877201769\\
3672	1.46399048280467\\
3673	1.52135835656344\\
3674	1.57021480511099\\
3675	1.58797552576248\\
3676	1.59526621297103\\
3677	1.56297554627002\\
3678	1.43779593390847\\
3679	1.30479307475122\\
3680	1.27612548527471\\
3681	1.26843508623987\\
3682	1.29362971128189\\
3683	1.29868774040491\\
3684	1.36976050892584\\
3685	1.40934185923732\\
3686	1.38940747508346\\
3687	1.35794283354143\\
3688	1.3221085620893\\
3689	1.32261712363637\\
3690	1.27710695910869\\
3691	1.27773223158342\\
3692	1.31265085461586\\
3693	1.34551436082535\\
3694	1.29861435260488\\
3695	1.36251358002126\\
3696	1.4577450528693\\
3697	1.48727946773778\\
3698	1.5388317344878\\
3699	1.55452120281891\\
3700	1.52154061254977\\
3701	1.49885676147523\\
3702	1.37376504855579\\
3703	1.2702724617331\\
3704	1.19023874194263\\
3705	1.18413930661907\\
3706	1.1630157864384\\
3707	1.17537368861946\\
3708	1.16339919020576\\
3709	1.16581014547604\\
3710	1.21562816837501\\
3711	1.22277445958487\\
3712	1.2291607139089\\
3713	1.2214768169612\\
3714	1.19868898979943\\
3715	1.23700370048132\\
3716	1.29578889936315\\
3717	1.34048347163383\\
3718	1.37802974481406\\
3719	1.45822808239194\\
3720	1.62542618084639\\
3721	1.68730299142826\\
3722	1.74743473806237\\
3723	1.74205322279998\\
3724	1.72774932969227\\
3725	1.69494664100439\\
3726	1.56468557746798\\
3727	1.42094100815621\\
3728	1.37101936290912\\
3729	1.38033453822624\\
3730	1.42279287325755\\
3731	1.44723406471445\\
3732	1.49583214348397\\
3733	1.51296153745204\\
3734	1.54105483778736\\
3735	1.56471908591135\\
3736	1.51332333354539\\
3737	1.44311707519592\\
3738	1.39298210201162\\
3739	1.37038238834379\\
3740	1.39239817397019\\
3741	1.40831787750714\\
3742	1.34286800616276\\
3743	1.3790829393274\\
3744	1.48194479275096\\
3745	1.53725267940936\\
3746	1.58306753027281\\
3747	1.58803974076524\\
3748	1.57221098167797\\
3749	1.52507385934158\\
3750	1.40956711889265\\
3751	1.28344811093828\\
3752	1.26751691644711\\
3753	1.30322577240288\\
3754	1.33817366140415\\
3755	1.36196216099695\\
3756	1.37990997560199\\
3757	1.40149768005111\\
3758	1.44106140393456\\
3759	1.44733111485408\\
3760	1.42319366424452\\
3761	1.37965334897607\\
3762	1.36175150018695\\
3763	1.3550096747808\\
3764	1.39599200495646\\
3765	1.44448988154368\\
3766	1.38321296646044\\
3767	1.45470505099715\\
3768	1.54610360837696\\
3769	1.63285676431404\\
3770	1.70283788060351\\
3771	1.75752370073179\\
3772	1.77242811655421\\
3773	1.845139379414\\
3774	1.82172366476321\\
3775	1.72109187834806\\
3776	1.63617585851796\\
3777	1.57305554230484\\
3778	1.54900100824105\\
3779	1.55181755128365\\
3780	1.58192125794566\\
3781	1.6238039297677\\
3782	1.67430992750607\\
3783	1.66113008807146\\
3784	1.60872641528051\\
3785	1.56266072923787\\
3786	1.51852139790263\\
3787	1.4842384558708\\
3788	1.4962097154783\\
3789	1.48272336021206\\
3790	1.44702916047302\\
3791	1.47717380881303\\
3792	1.55382520480484\\
3793	1.64134450067479\\
3794	1.70012093148227\\
3795	1.72123594661883\\
3796	1.72419116668653\\
3797	1.77709232047157\\
3798	1.79434058597665\\
3799	1.72700128606725\\
3800	1.6533575496581\\
3801	1.60440696604998\\
3802	1.5920694477183\\
3803	1.60750928296351\\
3804	1.65012075701476\\
3805	1.76370012586292\\
3806	1.82167989339568\\
3807	1.84721942346852\\
3808	1.79380198615085\\
3809	1.70970455462316\\
3810	1.62113977619217\\
3811	1.59440463601407\\
3812	1.57441629961808\\
3813	1.55062003804769\\
3814	1.5692028266816\\
3815	1.56578860682774\\
3816	1.64695455299042\\
3817	1.73985531875917\\
3818	1.82364266918675\\
3819	1.83129097486363\\
3820	1.85859748173624\\
3821	1.89877486272914\\
3822	1.88054116138669\\
3823	1.80386395160727\\
3824	1.74011410639167\\
3825	1.68755906541981\\
3826	1.69116369818957\\
3827	1.67020393189327\\
3828	1.68406761660961\\
3829	1.6885789829241\\
3830	1.73537129877557\\
3831	1.73729078131874\\
3832	1.68708028136734\\
3833	1.62949588014114\\
3834	1.5572716274429\\
3835	1.51407816913073\\
3836	1.48910655890562\\
3837	1.4980440130649\\
3838	1.47082422529746\\
3839	1.49433991807984\\
3840	1.54052871657566\\
3841	1.60428631668962\\
3842	1.63308234734281\\
3843	1.6969558133662\\
3844	1.69496751641167\\
3845	1.62260350459049\\
3846	1.45243584365977\\
3847	1.3228292356722\\
3848	1.24263593780518\\
3849	1.22138283008873\\
3850	1.23827362048467\\
3851	1.22652485901527\\
3852	1.25400584046108\\
3853	1.26094667228788\\
3854	1.25876985005813\\
3855	1.27917477701699\\
3856	1.28073583119562\\
3857	1.25744547095255\\
3858	1.22171156985212\\
3859	1.22891116308928\\
3860	1.26689679245714\\
3861	1.29490158118341\\
3862	1.25986617616603\\
3863	1.30108763867455\\
3864	1.40407772180231\\
3865	1.43696861080253\\
3866	1.48318372415007\\
3867	1.49023083137904\\
3868	1.46064690797695\\
3869	1.44201113404697\\
3870	1.31289587789456\\
3871	1.23586127612009\\
3872	1.19120594942631\\
3873	1.1972638113535\\
3874	1.2398721899103\\
3875	1.25057422285247\\
3876	1.26354581703321\\
3877	1.27044968208399\\
3878	1.29451832747296\\
3879	1.29598308630525\\
3880	1.26422731235269\\
3881	1.2287300933331\\
3882	1.24917263820293\\
3883	1.21545397069644\\
3884	1.21868048553675\\
3885	1.25825698980613\\
3886	1.22510620536524\\
3887	1.26122920319733\\
3888	1.36195714581954\\
3889	1.4227788760954\\
3890	1.47472041636434\\
3891	1.48990667545931\\
3892	1.476084215177\\
3893	1.44091426659924\\
3894	1.33466955241681\\
3895	1.21733333402759\\
3896	1.19258667451659\\
3897	1.23896370481109\\
3898	1.24160896712122\\
3899	1.26536832115598\\
3900	1.31756116606996\\
3901	1.30802354090699\\
3902	1.32336025552507\\
3903	1.31606495486159\\
3904	1.27615938561948\\
3905	1.25592999667836\\
3906	1.21196267173485\\
3907	1.21370914781147\\
3908	1.22968850754999\\
3909	1.24736051735245\\
3910	1.21217929774862\\
3911	1.26447318691481\\
3912	1.3117676751058\\
3913	1.37717033794416\\
3914	1.42006009476563\\
3915	1.43749764543925\\
3916	1.48179465658843\\
3917	1.44029768075546\\
3918	1.32510423346029\\
3919	1.20718088208272\\
3920	1.1938366114696\\
3921	1.21247297405552\\
3922	1.22776046245526\\
3923	1.23688781852953\\
3924	1.26364601638577\\
3925	1.27125675304712\\
3926	1.2970995147342\\
3927	1.32115271201425\\
3928	1.29236568355441\\
3929	1.2808615329221\\
3930	1.26359392223323\\
3931	1.2599150213718\\
3932	1.27375197498322\\
3933	1.29237982149662\\
3934	1.25319915935664\\
3935	1.29640593669678\\
3936	1.35762389833102\\
3937	1.43199603588574\\
3938	1.4363765672514\\
3939	1.46439119466006\\
3940	1.49501738750928\\
3941	1.52625790685329\\
3942	1.5442944915498\\
3943	1.47455622136698\\
3944	1.4241270723373\\
3945	1.39859170343386\\
3946	1.42986701893472\\
3947	1.43529114793996\\
3948	1.48743775745037\\
3949	1.53862980548707\\
3950	1.61879637728744\\
3951	1.65263377390224\\
3952	1.62132448938724\\
3953	1.5367700767582\\
3954	1.50135255902522\\
3955	1.43043805642336\\
3956	1.44554491491109\\
3957	1.45880427597704\\
3958	1.38959317308818\\
3959	1.42658075524807\\
3960	1.4828942256248\\
3961	1.55767679106148\\
3962	1.61945715869091\\
3963	1.65400475023648\\
3964	1.65529879628369\\
3965	1.70097957904286\\
3966	1.68577766784164\\
3967	1.63784243485142\\
3968	1.60233028135369\\
3969	1.60036305899578\\
3970	1.59258204495623\\
3971	1.59913302706937\\
3972	1.66177063349327\\
3973	1.71252910337078\\
3974	1.79692445855988\\
3975	1.82149129977999\\
3976	1.80320919268491\\
3977	1.69772670913118\\
3978	1.57152389225968\\
3979	1.49236695004415\\
3980	1.4623126337071\\
3981	1.47111193680151\\
3982	1.4299062830571\\
3983	1.46931162556171\\
3984	1.52019967917529\\
3985	1.63273460985966\\
3986	1.67102491117197\\
3987	1.66487172877752\\
3988	1.62559085268999\\
3989	1.55969390873533\\
3990	1.42438571926197\\
3991	1.29459521637564\\
3992	1.26179069912829\\
3993	1.28198386945227\\
3994	1.28976391795728\\
3995	1.28250838589964\\
3996	1.28666614901737\\
3997	1.30609644898647\\
3998	1.32957544298294\\
3999	1.33215841507071\\
4000	1.32964287270559\\
4001	1.31618941572949\\
};
\addplot [color=mycolor1,line width=1.3pt,solid,forget plot]
  table[row sep=crcr]{%
4001	1.31618941572949\\
4002	1.29559453608764\\
4003	1.31608266758528\\
4004	1.33720879124765\\
4005	1.35632342689396\\
4006	1.32519426188851\\
4007	1.37028832738409\\
4008	1.4977389816997\\
4009	1.52743305227643\\
4010	1.55597518957749\\
4011	1.53741919955852\\
4012	1.52273775635947\\
4013	1.46338672809951\\
4014	1.38049536430936\\
4015	1.29299617862611\\
4016	1.22901716047218\\
4017	1.19161613294501\\
4018	1.26613794170925\\
4019	1.22752355109554\\
4020	1.32703310646884\\
4021	1.39272714364408\\
4022	1.38956252880666\\
4023	1.38002175167595\\
4024	1.37688430919111\\
4025	1.36598053189071\\
4026	1.39530392793119\\
4027	1.38888498612239\\
4028	1.37708044681498\\
4029	1.37742622847031\\
4030	1.32503918265647\\
4031	1.3589989478727\\
4032	1.46296190842913\\
4033	1.5489825040329\\
4034	1.57890942962955\\
4035	1.58848172196584\\
4036	1.56434640038085\\
4037	1.52575360967673\\
4038	1.39913292452421\\
4039	1.29331335102123\\
4040	1.28752944825253\\
4041	1.2788540262545\\
4042	1.30497155265534\\
4043	1.30607507532779\\
4044	1.30419532142317\\
4045	1.34798516447704\\
4046	1.36962170502019\\
4047	1.37979812705748\\
4048	1.36015576330416\\
4049	1.32533379648762\\
4050	1.30916242532408\\
4051	1.31116368606774\\
4052	1.32910174823948\\
4053	1.3550526159864\\
4054	1.3094079319432\\
4055	1.33401645056993\\
4056	1.41082974207069\\
4057	1.48754059550494\\
4058	1.5309922071067\\
4059	1.54488315117306\\
4060	1.51825912317412\\
4061	1.47730403143664\\
4062	1.36202861622388\\
4063	1.28094636819851\\
4064	1.23951616471639\\
4065	1.24932626328895\\
4066	1.25655299482283\\
4067	1.28800106024549\\
4068	1.33371615525912\\
4069	1.35012796744352\\
4070	1.32569268443033\\
4071	1.3389173000853\\
4072	1.30711015174073\\
4073	1.26613376257791\\
4074	1.24000279063002\\
4075	1.27633836640107\\
4076	1.32565494149845\\
4077	1.35646304604422\\
4078	1.31605339367937\\
4079	1.36412268946916\\
4080	1.43432212815354\\
4081	1.4941993713795\\
4082	1.54922884134155\\
4083	1.5543428566154\\
4084	1.5544027752952\\
4085	1.52800711819146\\
4086	1.41858508721773\\
4087	1.34264594284767\\
4088	1.30000514987923\\
4089	1.2943399879378\\
4090	1.28493495620388\\
4091	1.29432312435724\\
4092	1.33368370772\\
4093	1.35827401510497\\
4094	1.33482638389293\\
4095	1.34932563517064\\
4096	1.32686895532556\\
4097	1.30851215315966\\
4098	1.30846913604266\\
4099	1.30062398604511\\
4100	1.306734996058\\
4101	1.32419673563923\\
4102	1.28197832097944\\
4103	1.30794238415821\\
4104	1.39463851459985\\
4105	1.46871050131794\\
4106	1.52509604660416\\
4107	1.55920180150872\\
4108	1.59059373299992\\
4109	1.62114511776913\\
4110	1.60898600693006\\
4111	1.53307665113458\\
4112	1.3965255885567\\
4113	1.39976472205997\\
4114	1.38352885513267\\
4115	1.41543659306338\\
4116	1.45901980087057\\
4117	1.48301428051985\\
4118	1.51606729874238\\
4119	1.53200780221918\\
4120	1.50760054625678\\
4121	1.48277494561907\\
4122	1.43654859047325\\
4123	1.40512665027457\\
4124	1.40178010498662\\
4125	1.40579542166428\\
4126	1.38121023434916\\
4127	1.38277274932911\\
4128	1.48593139271085\\
4129	1.56677604790267\\
4130	1.61588910652981\\
4131	1.65770656662642\\
4132	1.64818003488658\\
4133	1.69122635443168\\
4134	1.70687931213232\\
4135	1.64176620241366\\
4136	1.6096159211088\\
4137	1.57633759815222\\
4138	1.57807174480689\\
4139	1.58478422790385\\
4140	1.62589802095437\\
4141	1.6836661485127\\
4142	1.7284461776359\\
4143	1.72569223282136\\
4144	1.66631834281753\\
4145	1.58010044680156\\
4146	1.51981623098922\\
4147	1.48057370980043\\
4148	1.43551043421745\\
4149	1.41322940987785\\
4150	1.36455491342459\\
4151	1.39947015660366\\
4152	1.50055314312815\\
4153	1.56776093409121\\
4154	1.62701992615919\\
4155	1.63960017321494\\
4156	1.6272269599899\\
4157	1.56153425268857\\
4158	1.39918335456132\\
4159	1.27818261784583\\
4160	1.25744054867085\\
4161	1.24922171512694\\
4162	1.28245111978724\\
4163	1.29013877996645\\
4164	1.31061269710729\\
4165	1.31554713075102\\
4166	1.3357969529238\\
4167	1.32733589432155\\
4168	1.30187054689532\\
4169	1.27456663012423\\
4170	1.23976968286901\\
4171	1.2517705133191\\
4172	1.27925702398332\\
4173	1.33134761763261\\
4174	1.27308223218856\\
4175	1.27153217639943\\
4176	1.37033879151016\\
4177	1.44889311637665\\
4178	1.50113441758113\\
4179	1.51444214493756\\
4180	1.50294851761252\\
4181	1.47894263827432\\
4182	1.38115129611963\\
4183	1.29453127780985\\
4184	1.27677338202901\\
4185	1.27940665477827\\
4186	1.2504679660399\\
4187	1.21070151197521\\
4188	1.24669560883242\\
4189	1.22152868883905\\
4190	1.24662258234907\\
4191	1.27121023677755\\
4192	1.27800735373782\\
4193	1.25807399495768\\
4194	1.32342616543423\\
4195	1.30588897356459\\
4196	1.27173147058311\\
4197	1.30617960339352\\
4198	1.24784257551314\\
4199	1.29168720544665\\
4200	1.36930254930289\\
4201	1.43860399795486\\
4202	1.54115706589538\\
4203	1.55251368551415\\
4204	1.5318000913884\\
4205	1.51940348360105\\
4206	1.40357034748643\\
4207	1.30020666305435\\
4208	1.30449963782686\\
4209	1.30616872966461\\
4210	1.3205758628378\\
4211	1.33279259540913\\
4212	1.36111402758336\\
4213	1.32482021991956\\
4214	1.337075747628\\
4215	1.33923316122851\\
4216	1.32707581540292\\
4217	1.30093783383335\\
4218	1.29503857335015\\
4219	1.29023073893879\\
4220	1.30142816785006\\
4221	1.32317031619275\\
4222	1.27423209953808\\
4223	1.3050185353304\\
4224	1.39381099869254\\
4225	1.47259620424479\\
4226	1.5099449957334\\
4227	1.52247529444703\\
4228	1.49103816559809\\
4229	1.45999060519064\\
4230	1.35686387895285\\
4231	1.31204839104068\\
4232	1.29633566490037\\
4233	1.2672602861788\\
4234	1.25653432272031\\
4235	1.28336056932423\\
4236	1.29869825623951\\
4237	1.31149363486506\\
4238	1.31865120458664\\
4239	1.31816688386565\\
4240	1.29365060998065\\
4241	1.26707548168283\\
4242	1.26294105358684\\
4243	1.31685809554586\\
4244	1.32349105841922\\
4245	1.32945601747519\\
4246	1.275263031251\\
4247	1.28937517252696\\
4248	1.37516975771258\\
4249	1.4507748715812\\
4250	1.50428529116204\\
4251	1.53615720441293\\
4252	1.53014174788755\\
4253	1.49786195506492\\
4254	1.3897027345737\\
4255	1.31703111984698\\
4256	1.30014302122484\\
4257	1.25306773624843\\
4258	1.21205868549745\\
4259	1.2507790566151\\
4260	1.24386710789626\\
4261	1.22504138173063\\
4262	1.27566476678454\\
4263	1.34992393602151\\
4264	1.33608599772793\\
4265	1.32150293864922\\
4266	1.34047699202803\\
4267	1.35594931500238\\
4268	1.36632389642256\\
4269	1.36836322203444\\
4270	1.35212261250182\\
4271	1.38069026526006\\
4272	1.51156179774314\\
4273	1.59642943063689\\
4274	1.6670570419327\\
4275	1.70023517190724\\
4276	1.71780830959404\\
4277	1.75055956912029\\
4278	1.70717881839584\\
4279	1.6357603237253\\
4280	1.55633829594116\\
4281	1.50071451922941\\
4282	1.46417798799434\\
4283	1.43906343395589\\
4284	1.4458943989111\\
4285	1.46657920835261\\
4286	1.48498592031279\\
4287	1.46871914608118\\
4288	1.45223764355814\\
4289	1.42466349117417\\
4290	1.37780666905603\\
4291	1.43576404493913\\
4292	1.47440429002814\\
4293	1.45412490436344\\
4294	1.41485761799737\\
4295	1.45086280797502\\
4296	1.54660668211826\\
4297	1.62009706720831\\
4298	1.67512918834857\\
4299	1.69817175230901\\
4300	1.71907775282803\\
4301	1.70966078713112\\
4302	1.70552380828391\\
4303	1.67176822278353\\
4304	1.69602012441316\\
4305	1.6557943748279\\
4306	1.63650795491492\\
4307	1.59000037935756\\
4308	1.53565679833737\\
4309	1.58370596013948\\
4310	1.62444714963466\\
4311	1.66236512823489\\
4312	1.6758571222268\\
4313	1.61938441273584\\
4314	1.55152733731941\\
4315	1.50427356818753\\
4316	1.54600304732303\\
4317	1.48005370451145\\
4318	1.52546271982928\\
4319	1.54500236718032\\
4320	1.587055480313\\
4321	1.63986582772198\\
4322	1.70339907407017\\
4323	1.72830732682964\\
4324	1.69774362438139\\
4325	1.63559591193005\\
4326	1.49868134539045\\
4327	1.38747759799134\\
4328	1.36542381918368\\
4329	1.34076686573725\\
4330	1.3475781680163\\
4331	1.32212203584888\\
4332	1.33947405285563\\
4333	1.36646827606846\\
4334	1.39118570783722\\
4335	1.39569368131295\\
4336	1.37213237382825\\
4337	1.34862122664019\\
4338	1.29610999902684\\
4339	1.31094281482464\\
4340	1.30068895924441\\
4341	1.33194401231678\\
4342	1.29866674637721\\
4343	1.33277409611082\\
4344	1.43308884336158\\
4345	1.48896079374823\\
4346	1.52982045067167\\
4347	1.54969139079214\\
4348	1.531677182099\\
4349	1.51350008767921\\
4350	1.40842986192962\\
4351	1.29780928154691\\
4352	1.2828514689639\\
4353	1.30218647487576\\
4354	1.34082400338246\\
4355	1.36965423479867\\
4356	1.40076715329339\\
4357	1.40450760353909\\
4358	1.43486190642438\\
4359	1.43109248568233\\
4360	1.4176997666995\\
4361	1.37036234046939\\
4362	1.33721781023052\\
4363	1.367751105602\\
4364	1.35365030427683\\
4365	1.45178023021802\\
4366	1.41961971994557\\
4367	1.45305889390258\\
4368	1.49734977561513\\
4369	1.56429917586771\\
4370	1.63773994047444\\
4371	1.66566453615465\\
4372	1.6631546211206\\
4373	1.61894937399649\\
4374	1.52040563907253\\
4375	1.41169724829369\\
4376	1.38452894205561\\
4377	1.38246120632814\\
4378	1.39874972921873\\
4379	1.41556952017938\\
4380	1.47433117163176\\
4381	1.47888991377449\\
4382	1.49539917989589\\
4383	1.48617607997818\\
4384	1.46343701365672\\
4385	1.41836549821394\\
4386	1.3814034767776\\
4387	1.38023389184722\\
4388	1.34897251739531\\
4389	1.3700680038937\\
4390	1.34944104354227\\
4391	1.41146659636917\\
4392	1.49375223043106\\
4393	1.56339877036334\\
4394	1.62130907283976\\
4395	1.65428426986773\\
4396	1.64991234127051\\
4397	1.62685934860187\\
4398	1.52303540110398\\
4399	1.44665786131594\\
4400	1.3806424174047\\
4401	1.38622069551985\\
4402	1.41906802928409\\
4403	1.43917135714619\\
4404	1.489123813568\\
4405	1.49428537407353\\
4406	1.52841632212816\\
4407	1.52110293208369\\
4408	1.50318279847849\\
4409	1.4525013863931\\
4410	1.42695151458626\\
4411	1.43628270996561\\
4412	1.44921013147167\\
4413	1.45153943630807\\
4414	1.43264545473599\\
4415	1.46572374638287\\
4416	1.56117729606036\\
4417	1.63494625125762\\
4418	1.66730014758874\\
4419	1.67817866278197\\
4420	1.65854488156546\\
4421	1.62582398741912\\
4422	1.51918364787478\\
4423	1.40919358901159\\
4424	1.37456566965475\\
4425	1.36938750307759\\
4426	1.36851598028361\\
4427	1.35721676062842\\
4428	1.37102426727749\\
4429	1.37814190348712\\
4430	1.38182402966115\\
4431	1.39357266810357\\
4432	1.41966709337951\\
4433	1.39570683499555\\
4434	1.37664814182431\\
4435	1.39789643702387\\
4436	1.41702305726996\\
4437	1.4529054790648\\
4438	1.43357288746509\\
4439	1.50340537907493\\
4440	1.58627882237049\\
4441	1.67139264916676\\
4442	1.74721595307551\\
4443	1.79126752408154\\
4444	1.81345383280937\\
4445	1.86712780340955\\
4446	1.89911390117386\\
4447	1.89159565700153\\
4448	1.81344855658169\\
4449	1.72682297899704\\
4450	1.72481604771903\\
4451	1.71935614221376\\
4452	1.70919850584017\\
4453	1.71116454181024\\
4454	1.75262711598611\\
4455	1.74152683904705\\
4456	1.75140940000694\\
4457	1.7664738419082\\
4458	1.80644016836022\\
4459	1.82056468696452\\
4460	1.73554379043239\\
4461	1.70088990979367\\
4462	1.6391046229217\\
4463	1.68387426488656\\
4464	1.74015729387358\\
4465	1.82618560642051\\
4466	1.89788463572942\\
4467	1.96911901879963\\
4468	2.06242203825983\\
4469	2.10460953356121\\
4470	2.07989674249205\\
4471	2.01015472071837\\
4472	1.91898202071455\\
4473	1.85366613666493\\
4474	1.79303880888077\\
4475	1.8115170474606\\
4476	1.77723305597322\\
4477	1.77420648400289\\
4478	1.83198703343779\\
4479	1.83157906652971\\
4480	1.77111175630129\\
4481	1.71601188032177\\
4482	1.63568537282106\\
4483	1.60292933405205\\
4484	1.63823393067935\\
4485	1.63616405576641\\
4486	1.64214708472491\\
4487	1.67968903212644\\
4488	1.85086342053808\\
4489	1.89213865348715\\
4490	1.93079290992442\\
4491	1.94097741474605\\
4492	1.90153297139307\\
4493	1.80962426593558\\
4494	1.66034015945357\\
4495	1.48322529934938\\
4496	1.45676969024054\\
4497	1.44872628459886\\
4498	1.45742112137186\\
4499	1.44927901326722\\
4500	1.48253062990321\\
4501	1.46266001076143\\
4502	1.44232624143255\\
4503	1.43095134873069\\
4504	1.39703991266497\\
4505	1.41541745428682\\
4506	1.38118832530763\\
4507	1.3898779969399\\
4508	1.3626614495872\\
4509	1.3711925161759\\
4510	1.346548670668\\
4511	1.40229422129295\\
4512	1.51908662170931\\
4513	1.56949102626469\\
4514	1.62673430339605\\
4515	1.65522774760241\\
4516	1.65453580538204\\
4517	1.60999520846384\\
4518	1.50902906260639\\
4519	1.39323995463441\\
4520	1.30075678956672\\
4521	1.25914719882269\\
4522	1.24204335334177\\
4523	1.22705456276675\\
4524	1.25434556855222\\
4525	1.25624565363706\\
4526	1.25845731507884\\
4527	1.27481015404865\\
4528	1.28916955303821\\
4529	1.27474672671309\\
4530	1.28876295501267\\
4531	1.31620092247046\\
4532	1.33875450324422\\
4533	1.36370401235384\\
4534	1.33986278526885\\
4535	1.38578034642895\\
4536	1.51576137808137\\
4537	1.59773572525181\\
4538	1.65084962291435\\
4539	1.69944063584339\\
4540	1.73141648004091\\
4541	1.70811144672443\\
4542	1.5989073885938\\
4543	1.47890520288579\\
4544	1.40163689680588\\
4545	1.38218241204369\\
4546	1.34398461819648\\
4547	1.30551437190779\\
4548	1.3189031496298\\
4549	1.32542099030167\\
4550	1.33210588862896\\
4551	1.34559537679521\\
4552	1.35219570478913\\
4553	1.35782837680865\\
4554	1.37188077595834\\
4555	1.43414803589739\\
4556	1.48657104940747\\
4557	1.49166120453397\\
4558	1.47427716095113\\
4559	1.53326563493965\\
4560	1.66026418908893\\
4561	1.74973464018032\\
4562	1.81410121529767\\
4563	1.83728822992848\\
4564	1.80951872470068\\
4565	1.74328870369438\\
4566	1.60808549317835\\
4567	1.49231831445394\\
4568	1.39357565458146\\
4569	1.3733343406562\\
4570	1.39607238352986\\
4571	1.39414024193057\\
4572	1.35756491810609\\
4573	1.37540724405425\\
4574	1.39858145675128\\
4575	1.41149710352945\\
4576	1.42380046459854\\
4577	1.39748082007369\\
4578	1.39476678655035\\
4579	1.40678593059262\\
4580	1.46203329074518\\
4581	1.43009229546106\\
4582	1.40716270325324\\
4583	1.45982886349861\\
4584	1.57076282773402\\
4585	1.64901028246195\\
4586	1.67704434967407\\
4587	1.68945641370108\\
4588	1.69143367106243\\
4589	1.64202550255197\\
4590	1.53348693424498\\
4591	1.43442636352814\\
4592	1.43777076953143\\
4593	1.45609350827763\\
4594	1.43578341500415\\
4595	1.34356191608332\\
4596	1.3517702214149\\
4597	1.38796859423923\\
4598	1.41721169923852\\
4599	1.43955606133263\\
4600	1.44328519536852\\
4601	1.41248836347422\\
4602	1.41099903478099\\
4603	1.44706819654924\\
4604	1.42452859406177\\
4605	1.43363840147953\\
4606	1.41099990155249\\
4607	1.4164633150049\\
4608	1.48959269632043\\
4609	1.56482829603809\\
4610	1.63321459409466\\
4611	1.65867567202244\\
4612	1.68885727694936\\
4613	1.71107093472958\\
4614	1.71447633192934\\
4615	1.64426748601285\\
4616	1.55598306130484\\
4617	1.49471001728977\\
4618	1.46140184814382\\
4619	1.46756423983987\\
4620	1.4825638511596\\
4621	1.53004746963212\\
4622	1.57171399952105\\
4623	1.61232305970308\\
4624	1.5937741016283\\
4625	1.54667022151225\\
4626	1.51328147473042\\
4627	1.4789564989673\\
4628	1.49766778947659\\
4629	1.49874646914091\\
4630	1.46692690083988\\
4631	1.51917442172625\\
4632	1.59643110658413\\
4633	1.68205665066274\\
4634	1.76562988138068\\
4635	1.82139779965719\\
4636	1.85680281667358\\
4637	1.88634762553177\\
4638	1.90810067247642\\
4639	1.85109102391815\\
4640	1.78710531015798\\
4641	1.70537669035585\\
4642	1.64310502112223\\
4643	1.60752547456017\\
4644	1.60950587106563\\
4645	1.69854894657558\\
4646	1.75441749964696\\
4647	1.7927196285187\\
4648	1.78622703592402\\
4649	1.72021499991657\\
4650	1.63249995086164\\
4651	1.57525568025951\\
4652	1.55943295739602\\
4653	1.55590215686009\\
4654	1.5337712576679\\
4655	1.54670991201498\\
4656	1.62824688657946\\
4657	1.80015002456493\\
4658	1.86067167656401\\
4659	1.86658151007515\\
4660	1.83526076366312\\
4661	1.758554196914\\
4662	1.63388134749322\\
4663	1.49312248466233\\
4664	1.4159921309297\\
4665	1.43717093063491\\
4666	1.40105205718623\\
4667	1.38121887229992\\
4668	1.379420586745\\
4669	1.41736641358679\\
4670	1.44220858110202\\
4671	1.47427231216578\\
4672	1.4463027784979\\
4673	1.41411635440644\\
4674	1.39281215045093\\
4675	1.42045892611969\\
4676	1.43197511380951\\
4677	1.45198940349357\\
4678	1.41051716632336\\
4679	1.44570208590064\\
4680	1.56441372331271\\
4681	1.64520619050764\\
4682	1.71998946591133\\
4683	1.73928363703857\\
4684	1.73862636926446\\
4685	1.71448323193354\\
4686	1.60099615075696\\
4687	1.52544375290271\\
4688	1.47758088760188\\
4689	1.45281651911276\\
4690	1.39621914128058\\
4691	1.36065591674509\\
4692	1.35367553326594\\
4693	1.36683856624594\\
4694	1.37709943143013\\
4695	1.3701975618758\\
4696	1.37400036009692\\
4697	1.34351682421745\\
4698	1.33653357345793\\
4699	1.38730797561503\\
4700	1.39614313272518\\
4701	1.38567934719607\\
4702	1.3537124141453\\
4703	1.32735883451147\\
4704	1.41632498789465\\
4705	1.47579846449675\\
4706	1.53419652670055\\
4707	1.55052427503633\\
4708	1.54491365307537\\
4709	1.49543251762708\\
4710	1.41997404107156\\
4711	1.31657633738457\\
4712	1.26389757590045\\
4713	1.25966400963326\\
4714	1.27244325770386\\
4715	1.29949523023552\\
4716	1.35503269895928\\
4717	1.37840477525719\\
4718	1.40393192892372\\
4719	1.37865827204192\\
4720	1.38819032009804\\
4721	1.36177115447024\\
4722	1.31702585062606\\
4723	1.33288011375792\\
4724	1.38944016745084\\
4725	1.38917997143364\\
4726	1.33370630894813\\
4727	1.37348820032172\\
4728	1.45637121478727\\
4729	1.54325492971439\\
4730	1.5860351347298\\
4731	1.60344269108749\\
4732	1.60241388249113\\
4733	1.55757492769731\\
4734	1.47908198576506\\
4735	1.39840907533089\\
4736	1.33405975485323\\
4737	1.35872118709408\\
4738	1.37368454933541\\
4739	1.38699198157114\\
4740	1.39437850851295\\
4741	1.42694219815011\\
4742	1.4358988985985\\
4743	1.41664217601781\\
4744	1.40526567403826\\
4745	1.36460645055754\\
4746	1.34707465386839\\
4747	1.3662306970077\\
4748	1.41009324785018\\
4749	1.40746891907867\\
4750	1.37287335755327\\
4751	1.44316172079412\\
4752	1.5367839624099\\
4753	1.6454164712439\\
4754	1.70175677134148\\
4755	1.71687733837094\\
4756	1.71489152994698\\
4757	1.66131631364338\\
4758	1.57673992238283\\
4759	1.48638285845249\\
4760	1.41529170963782\\
4761	1.42928132490828\\
4762	1.436169521318\\
4763	1.4079416405844\\
4764	1.49510271864436\\
4765	1.56972322470649\\
4766	1.57365237071526\\
4767	1.54420650195928\\
4768	1.51722272960743\\
4769	1.46358228578607\\
4770	1.44037034043947\\
4771	1.41449613677074\\
4772	1.4220252268564\\
4773	1.43262990789637\\
4774	1.39613239688236\\
4775	1.46160725966154\\
4776	1.55076214797843\\
4777	1.71113143705093\\
4778	1.80692979979074\\
4779	1.88286163290274\\
4780	1.89838451304531\\
4781	1.95803987751944\\
4782	1.98463940441448\\
4783	1.9316290137851\\
4784	1.88483927324364\\
4785	1.8828981461269\\
4786	1.84872705952539\\
4787	1.96785469738677\\
4788	2.00959217780517\\
4789	1.99696822998982\\
4790	2.01802723008759\\
4791	1.97163969996087\\
4792	1.86814419746603\\
4793	1.76161080696037\\
4794	1.64209334311521\\
4795	1.5830768789506\\
4796	1.56383980879565\\
4797	1.60328359816925\\
4798	1.61911437800292\\
4799	1.61268820009824\\
4800	1.6835510361411\\
4801	1.7612807848468\\
4802	1.82229998084255\\
4803	1.8409248704038\\
4804	1.86778733933902\\
4805	1.87055365313631\\
4806	1.90265920113458\\
4807	1.86220440035996\\
4808	1.84146216677447\\
4809	1.76736351882109\\
4810	1.73358058090014\\
4811	1.74994800436423\\
4812	1.83399759788768\\
4813	1.83020386314253\\
4814	1.86015127329546\\
4815	1.90747294279015\\
4816	1.89961049238829\\
4817	1.88669371902829\\
4818	1.77882847063897\\
4819	1.70620640241822\\
4820	1.67787494940387\\
4821	1.66948884160969\\
4822	1.64974644937316\\
4823	1.68528977251559\\
4824	1.76547050375868\\
4825	1.85371676425487\\
4826	1.91164660786981\\
4827	1.96169788708705\\
4828	1.98443216780835\\
4829	2.01042628229444\\
4830	2.0291803130122\\
4831	1.96268752508897\\
4832	1.95555077393018\\
4833	1.87744129479487\\
4834	1.82268453042181\\
4835	1.85740629918543\\
4836	1.86417469450505\\
4837	1.81862010501414\\
4838	1.86030961873876\\
4839	1.89485246827313\\
4840	1.90107418324606\\
4841	1.86574660751424\\
4842	1.80909577564819\\
4843	1.79614900925671\\
4844	1.78708631690126\\
4845	1.77774472343373\\
4846	1.74443404188909\\
4847	1.79909011978581\\
4848	1.905945297287\\
4849	2.00254192002816\\
4850	2.07444399903042\\
4851	2.11371501523038\\
4852	2.10876263914645\\
4853	2.01190351177934\\
4854	1.86652443172401\\
4855	1.70962054720054\\
4856	1.59960197208602\\
4857	1.5533109549666\\
4858	1.55666290725773\\
4859	1.56466343196595\\
4860	1.62347457918826\\
4861	1.66336974400224\\
4862	1.70007192927527\\
4863	1.72744409033513\\
4864	1.73463252445493\\
4865	1.63876537071008\\
4866	1.61406687849377\\
4867	1.59339160555086\\
4868	1.57454075451458\\
4869	1.5977752908448\\
4870	1.57013228994685\\
4871	1.59494133446264\\
4872	1.72744498833086\\
4873	1.81880330211071\\
4874	1.88289572349143\\
4875	1.88381917203693\\
4876	1.88413072412988\\
4877	1.82725561358592\\
4878	1.76391936942108\\
4879	1.62500287078914\\
4880	1.56160487694797\\
4881	1.56853141219304\\
4882	1.59004603994113\\
4883	1.61699074762484\\
4884	1.69618214621117\\
4885	1.70980129877867\\
4886	1.72364875130709\\
4887	1.70196836667236\\
4888	1.67992038484474\\
4889	1.64887251180275\\
4890	1.62341148980498\\
4891	1.61331082972964\\
4892	1.60901041745858\\
4893	1.6167888230751\\
4894	1.55846212306039\\
4895	1.60484565626097\\
4896	1.71848761219632\\
4897	1.81964169044996\\
4898	1.86259764681552\\
4899	1.87348395449316\\
4900	1.8486400472033\\
4901	1.80918121993009\\
4902	1.7159125307324\\
4903	1.58933785249964\\
4904	1.57539940516794\\
4905	1.6008950267197\\
4906	1.64448830657587\\
4907	1.66553026417196\\
4908	1.66102770693597\\
4909	1.69252862786778\\
4910	1.71373967047149\\
4911	1.71218786438991\\
4912	1.68208041113866\\
4913	1.63055307356014\\
4914	1.60345581232375\\
4915	1.60251020065602\\
4916	1.58834678928129\\
4917	1.57791460829857\\
4918	1.53825648636354\\
4919	1.57322272022105\\
4920	1.67313177336046\\
4921	1.7476778886419\\
4922	1.80357778950498\\
4923	1.8106174509066\\
4924	1.79506213672045\\
4925	1.75795560184864\\
4926	1.68637131227605\\
4927	1.62447745270393\\
4928	1.53127677518992\\
4929	1.53925109264216\\
4930	1.56063915451773\\
4931	1.57454007572243\\
4932	1.59184305316604\\
4933	1.59948669322999\\
4934	1.59971185235114\\
4935	1.61113527883714\\
4936	1.62112318995445\\
4937	1.61703895491777\\
4938	1.59367069216149\\
4939	1.58035644530255\\
4940	1.59262165839242\\
4941	1.58312168194018\\
4942	1.53903304680397\\
4943	1.60274772170575\\
4944	1.64911441524974\\
4945	1.7294289797867\\
4946	1.79796213842512\\
4947	1.81985052470703\\
4948	1.83681941723779\\
4949	1.84434999235046\\
4950	1.83840910137275\\
4951	1.77451277725081\\
4952	1.71896618126226\\
4953	1.66561937010653\\
4954	1.65600660459341\\
4955	1.65837546966231\\
4956	1.68535008702192\\
4957	1.72959621294691\\
4958	1.78384264688621\\
4959	1.7898996791458\\
4960	1.75857664710909\\
4961	1.69549324553552\\
4962	1.67630774439399\\
4963	1.63080625522925\\
4964	1.61860175928722\\
4965	1.61812386986629\\
4966	1.59020053971504\\
4967	1.62273209933302\\
4968	1.73172778380117\\
4969	1.79250251454019\\
4970	1.84765893362585\\
4971	1.88667572673274\\
4972	1.90966192632447\\
4973	1.89946962366138\\
4974	1.93171208021666\\
4975	1.88451749837483\\
4976	1.84799288624546\\
4977	1.82676415740296\\
4978	1.88176883690491\\
4979	1.9114559227512\\
4980	1.96439988250826\\
4981	2.0134450101113\\
4982	2.03685617223114\\
4983	2.02117428741974\\
4984	1.96159526698572\\
4985	1.85282989629305\\
4986	1.75235608781471\\
4987	1.66547847190056\\
4988	1.62968739978369\\
4989	1.60979722249612\\
4990	1.61731847002928\\
4991	1.65104421276981\\
4992	1.7166140394037\\
4993	1.77290814718587\\
4994	1.83086023545586\\
4995	1.85256267079421\\
4996	1.83476126083532\\
4997	1.7708657720389\\
4998	1.68846573948948\\
4999	1.58268012990512\\
5000	1.51692483792136\\
5001	1.4867689134258\\
5002	1.48082339913355\\
5003	1.48491734050944\\
5004	1.48656367523982\\
5005	1.5165270779851\\
5006	1.53479128015968\\
5007	1.55405516917485\\
5008	1.54591603977543\\
5009	1.51626416234671\\
5010	1.51873866376434\\
5011	1.56820193470871\\
5012	1.60033548418052\\
5013	1.61153750555251\\
5014	1.55051441309273\\
5015	1.63284546290318\\
5016	1.73161353122053\\
5017	1.8115714429989\\
5018	1.88663941310593\\
5019	1.90138271338261\\
5020	1.89523654531382\\
5021	1.83886058890445\\
5022	1.75810156451926\\
5023	1.65386977402118\\
5024	1.58363824714227\\
5025	1.61054422338444\\
5026	1.61820325095725\\
5027	1.61809156963678\\
5028	1.59822428142478\\
5029	1.62602384566583\\
5030	1.65797821105665\\
5031	1.67409499720823\\
5032	1.65487478905342\\
5033	1.60424140757313\\
5034	1.58385799538476\\
5035	1.59566172634704\\
5036	1.58107770666115\\
5037	1.58315851178155\\
5038	1.50773612723059\\
5039	1.57338895937081\\
5040	1.66757852717405\\
5041	1.73605181008837\\
5042	1.78926302476676\\
5043	1.81385147630244\\
5044	1.81714886431123\\
5045	1.77389946741239\\
5046	1.68559943745563\\
5047	1.57775611467238\\
5048	1.53408596122801\\
5049	1.55419071098099\\
5050	1.57374051980781\\
5051	1.59108689728736\\
5052	1.63258477886812\\
5053	1.65343020470828\\
5054	1.63720708520389\\
5055	1.62731791701827\\
5056	1.59598030319354\\
5057	1.54797866157849\\
5058	1.52501544343363\\
5059	1.52810935058777\\
5060	1.54040161133377\\
5061	1.55433938742125\\
5062	1.48162796157517\\
5063	1.57710233351075\\
5064	1.65633597826658\\
5065	1.74967760847099\\
5066	1.81280157818883\\
5067	1.84479626036456\\
5068	1.83051299987061\\
5069	1.78151888702967\\
5070	1.70081070096294\\
5071	1.58915840467819\\
5072	1.55371143179058\\
5073	1.55720093785001\\
5074	1.54160679574318\\
5075	1.5798640425226\\
5076	1.57797769900633\\
5077	1.60155006491717\\
5078	1.60781128641772\\
5079	1.60918592605525\\
5080	1.60020241066137\\
5081	1.60950851383479\\
5082	1.58003059675887\\
5083	1.58244441940473\\
5084	1.54937033730299\\
5085	1.54080672416639\\
5086	1.47777251373869\\
5087	1.56573471745639\\
5088	1.67376730318411\\
5089	1.74734826794991\\
5090	1.80316304759763\\
5091	1.81075107951983\\
5092	1.81920290373721\\
5093	1.76704775578893\\
5094	1.6872228453427\\
5095	1.57078783391539\\
5096	1.52792117185262\\
5097	1.55356740505943\\
5098	1.58575156778369\\
5099	1.63122357907021\\
5100	1.61838305709686\\
5101	1.63282691505675\\
5102	1.64717377608998\\
5103	1.63528336405328\\
5104	1.62734605726679\\
5105	1.58223498990548\\
5106	1.54462242122231\\
5107	1.52383532445451\\
5108	1.50714520842046\\
5109	1.51708310458728\\
5110	1.46890113334712\\
5111	1.53192625856298\\
5112	1.63014375188912\\
5113	1.70782700355481\\
5114	1.78572825605775\\
5115	1.84820045665323\\
5116	1.87760196749425\\
5117	1.88900060069232\\
5118	1.90579329997623\\
5119	1.8414552400793\\
5120	1.7602877270298\\
5121	1.71969329404051\\
5122	1.69483311578048\\
5123	1.68425881146882\\
5124	1.73071569784826\\
5125	1.74355719873472\\
5126	1.77856662515523\\
5127	1.74978479329568\\
5128	1.69096680071591\\
5129	1.62202069139607\\
5130	1.59635912547937\\
5131	1.59553773707085\\
5132	1.57030763268662\\
5133	1.56497115459285\\
5134	1.55809580496493\\
5135	1.61074479066161\\
5136	1.73734458827019\\
5137	1.82502577365379\\
5138	1.90396230228593\\
5139	1.93556170413346\\
5140	1.93900888593752\\
5141	1.93485378900737\\
5142	1.93870669227527\\
5143	1.89789308580299\\
5144	1.85210376287785\\
5145	1.88741557551948\\
5146	1.89308855224634\\
5147	1.90130613593707\\
5148	1.94403826309424\\
5149	2.02408397691402\\
5150	2.08274590293557\\
5151	2.08997143586383\\
5152	2.02954623931267\\
5153	1.92223915227928\\
5154	1.86717796399558\\
5155	1.76341568771223\\
5156	1.73276232162147\\
5157	1.74623382952743\\
5158	1.71870825328518\\
5159	1.77540407154056\\
5160	1.77514182658803\\
5161	1.84160606562223\\
5162	1.90191231805711\\
5163	1.91458906251578\\
5164	1.89212618992821\\
5165	1.80851442392841\\
5166	1.63040586827657\\
5167	1.53170629277662\\
5168	1.44267265534897\\
5169	1.41583933595077\\
5170	1.41751349899803\\
5171	1.45773185081765\\
5172	1.47580528765776\\
5173	1.52193617103804\\
5174	1.55810706303978\\
5175	1.56043556540382\\
5176	1.55725465195629\\
5177	1.5072182374253\\
5178	1.52077556169586\\
5179	1.50515625634747\\
5180	1.47830295546505\\
5181	1.47459773941874\\
5182	1.42633000886513\\
5183	1.50894137781055\\
5184	1.62274076310759\\
5185	1.69801185877577\\
5186	1.74880952266386\\
5187	1.77398741817662\\
5188	1.77601374373296\\
5189	1.71592068516447\\
5190	1.61416614154227\\
5191	1.50302543294371\\
5192	1.47718305484837\\
5193	1.4982703868072\\
5194	1.51744582661318\\
5195	1.55902534663892\\
5196	1.45401256465141\\
5197	1.46088106454156\\
5198	1.48345262545796\\
5199	1.47868671455192\\
5200	1.4452289492623\\
5201	1.39130027189605\\
5202	1.35498253973797\\
5203	1.33563073129264\\
5204	1.33843077741259\\
5205	1.32990050309456\\
5206	1.29594162689493\\
5207	1.37998028538411\\
5208	1.50931882402675\\
5209	1.58680840816009\\
5210	1.65598476044806\\
5211	1.67629046453695\\
5212	1.70248968516289\\
5213	1.64742319156116\\
5214	1.57341055106209\\
5215	1.47813585156727\\
5216	1.41288527457515\\
5217	1.40911653878314\\
5218	1.41218156137593\\
5219	1.41100737218043\\
5220	1.44147498107671\\
5221	1.45788734030126\\
5222	1.47189193635809\\
5223	1.44062433100851\\
5224	1.40938804083369\\
5225	1.38094763711316\\
5226	1.37807359820861\\
5227	1.4088191845344\\
5228	1.42594450663982\\
5229	1.42303374286449\\
5230	1.39577929791604\\
5231	1.47416928194715\\
5232	1.54607456866111\\
5233	1.61853287878565\\
5234	1.6597083530823\\
5235	1.66398410347946\\
5236	1.64110545788209\\
5237	1.57235570498506\\
5238	1.48756502800201\\
5239	1.38894035402197\\
5240	1.34817971626027\\
5241	1.27784859919047\\
5242	1.29449558505161\\
5243	1.31513916046334\\
5244	1.31519956871601\\
5245	1.32989959811272\\
5246	1.34304371648652\\
5247	1.35606821483396\\
5248	1.36840808778791\\
5249	1.36026057765293\\
5250	1.34531113652791\\
5251	1.37067186544196\\
5252	1.33394425717413\\
5253	1.32412324279651\\
5254	1.31609459044885\\
5255	1.38152678354068\\
5256	1.45710224135249\\
5257	1.5372800707245\\
5258	1.58846422888717\\
5259	1.61401269979424\\
5260	1.61227001863672\\
5261	1.55394062218637\\
5262	1.47137860616508\\
5263	1.35616052354667\\
5264	1.30441055706259\\
5265	1.28624112563431\\
5266	1.30870565903884\\
5267	1.30912789195062\\
5268	1.32655439890589\\
5269	1.32156378749173\\
5270	1.33743493343918\\
5271	1.31385326192454\\
5272	1.30910422411239\\
5273	1.3310897619535\\
5274	1.33886382177027\\
5275	1.34099752573996\\
5276	1.37947404027159\\
5277	1.41050511575643\\
5278	1.41737739296711\\
5279	1.48415213543186\\
5280	1.60159451357936\\
5281	1.69594425879808\\
5282	1.76807540132066\\
5283	1.83403835894509\\
5284	1.85823354886763\\
5285	1.89578095518565\\
5286	1.87909598015072\\
5287	1.80924484423009\\
5288	1.71893845482719\\
5289	1.67369711565072\\
5290	1.66643139323896\\
5291	1.67139625902726\\
5292	1.75214174896658\\
5293	1.8366759049439\\
5294	1.83611447246856\\
5295	1.83304899897076\\
5296	1.78551166138504\\
5297	1.7021761780697\\
5298	1.62347881545896\\
5299	1.55206499068132\\
5300	1.58311649278145\\
5301	1.54909513929099\\
5302	1.51462197643426\\
5303	1.51848816054046\\
5304	1.61108557304273\\
5305	1.67485941268947\\
5306	1.74688930847379\\
5307	1.80713574042204\\
5308	1.81725705781646\\
5309	1.81353199867003\\
5310	1.86318941114022\\
5311	1.8632856567444\\
5312	1.81294109976004\\
5313	1.75042213253978\\
5314	1.71030391194284\\
5315	1.66407519997308\\
5316	1.6832142130594\\
5317	1.818208098987\\
5318	1.88189849297426\\
5319	1.94131938085684\\
5320	1.91007458231898\\
5321	1.85254388037823\\
5322	1.79366052196478\\
5323	1.77941849353858\\
5324	1.78026288096084\\
5325	1.7292115625666\\
5326	1.69676847016518\\
5327	1.76486022132411\\
5328	1.95634149695228\\
5329	2.03692559561564\\
5330	2.11693667105648\\
5331	2.11710379948344\\
5332	2.06115219033984\\
5333	1.92594419867381\\
5334	1.78099130667032\\
5335	1.62646104925123\\
5336	1.53013562995122\\
5337	1.56810068217786\\
5338	1.60095733737027\\
5339	1.60538389946524\\
5340	1.61893464266262\\
5341	1.62766999330259\\
5342	1.6358129796159\\
5343	1.63065145225729\\
5344	1.641455943792\\
5345	1.62123280486859\\
5346	1.57084192021961\\
5347	1.53956801892375\\
5348	1.53658878147442\\
5349	1.55277474148025\\
5350	1.50659371302244\\
5351	1.58471345906682\\
5352	1.70167542495962\\
5353	1.79191357455511\\
5354	1.83644959050952\\
5355	1.85421281616043\\
5356	1.83217155420689\\
5357	1.74451211216487\\
5358	1.62358719994193\\
5359	1.51868096832261\\
5360	1.48814251124298\\
5361	1.50002527210344\\
5362	1.51471096795269\\
5363	1.4950917900688\\
5364	1.54378742563379\\
5365	1.5435824246366\\
5366	1.57118218572204\\
5367	1.55295935427411\\
5368	1.51834499558395\\
5369	1.47272744480672\\
5370	1.42406740199846\\
5371	1.448154765647\\
5372	1.44602714915264\\
5373	1.43174062849104\\
5374	1.4034338730743\\
5375	1.49903291678204\\
5376	1.59534023401427\\
5377	1.6644642841554\\
5378	1.72031217376165\\
5379	1.73916179486891\\
5380	1.71252589842983\\
5381	1.63332238379239\\
5382	1.52610772464493\\
5383	1.43963827649943\\
5384	1.35418647296709\\
5385	1.35053504745947\\
5386	1.3425303174889\\
5387	1.33048171111976\\
5388	1.36659515084737\\
5389	1.38706680115272\\
5390	1.40272747641675\\
5391	1.41256725429627\\
5392	1.4154640095923\\
5393	1.40744563683002\\
5394	1.42411605783239\\
5395	1.38339276404149\\
5396	1.39176403141773\\
5397	1.38364230892908\\
5398	1.3637329599118\\
5399	1.4525950794569\\
5400	1.53465429083845\\
5401	1.60691282498894\\
5402	1.63385598665713\\
5403	1.65241461495679\\
5404	1.63599171935179\\
5405	1.56187399504891\\
5406	1.45679454600768\\
5407	1.34574494252237\\
5408	1.28710181256362\\
5409	1.26521994767664\\
5410	1.22338231673277\\
5411	1.22249590992553\\
5412	1.25735839657224\\
5413	1.27444494492442\\
5414	1.29331828632878\\
5415	1.29473945187724\\
5416	1.2901334071982\\
5417	1.28962633886417\\
5418	1.26046160086775\\
5419	1.28576299015438\\
5420	1.30156549689527\\
5421	1.29427749513685\\
5422	1.30887320185752\\
5423	1.35016341010438\\
5424	1.46701707210354\\
5425	1.54338391829028\\
5426	1.59222482495475\\
5427	1.61039715570794\\
5428	1.63406339534216\\
5429	1.63303567945935\\
5430	1.65047474883555\\
5431	1.63233513503368\\
5432	1.60151940501311\\
5433	1.5723367716938\\
5434	1.55246457136426\\
5435	1.5712670201461\\
5436	1.60357181727302\\
5437	1.6563657720104\\
5438	1.71807550126526\\
5439	1.74880129061556\\
5440	1.76208201506812\\
5441	1.70082676582338\\
5442	1.63205192582561\\
5443	1.57817281266107\\
5444	1.53421070422567\\
5445	1.48389533148456\\
5446	1.44063424859629\\
5447	1.50749066291435\\
5448	1.6060610934891\\
5449	1.68114943678217\\
5450	1.7598631561066\\
5451	1.78507259088368\\
5452	1.77504040396323\\
5453	1.73319460741223\\
5454	1.70486587290205\\
5455	1.63330124887663\\
5456	1.55137835780258\\
5457	1.52906235274967\\
5458	1.52086478730684\\
5459	1.53587181314355\\
5460	1.56823956965305\\
5461	1.62380505048316\\
5462	1.67558406565666\\
5463	1.68965017226418\\
5464	1.66229588603983\\
5465	1.60881942893149\\
5466	1.53112434107205\\
5467	1.51043805895321\\
5468	1.5698492112396\\
5469	1.56864764387915\\
5470	1.54261893939668\\
5471	1.59834994022549\\
5472	1.73936954069078\\
5473	1.84381862162982\\
5474	1.9230401475829\\
5475	1.98490142772909\\
5476	2.0207098743116\\
5477	1.97574382470606\\
5478	2.00037331938698\\
5479	1.97304086330386\\
5480	1.92610628910919\\
5481	1.84615843393231\\
5482	1.83181542035584\\
5483	1.8263522298329\\
5484	1.82657875690754\\
5485	1.8873455631738\\
5486	1.92261359116176\\
5487	1.92863984331248\\
5488	1.89612435575911\\
5489	1.80872329126979\\
5490	1.72831106567526\\
5491	1.64314957112438\\
5492	1.6392542606851\\
5493	1.63015755757443\\
5494	1.64203660446223\\
5495	1.66339736132379\\
5496	1.8089730536639\\
5497	1.88690833254568\\
5498	1.929683763976\\
5499	1.9475791810379\\
5500	1.9270507122255\\
5501	1.79389649811207\\
5502	1.59881153854887\\
5503	1.47420432205644\\
5504	1.41173133137598\\
5505	1.4474236323701\\
5506	1.44228319495421\\
5507	1.43328493734761\\
5508	1.44439265379685\\
5509	1.44127383534997\\
5510	1.43935390793754\\
5511	1.40993935582571\\
5512	1.38713995579126\\
5513	1.32821012306132\\
5514	1.29955847955\\
5515	1.26806385462351\\
5516	1.26088106274925\\
5517	1.26858422294256\\
5518	1.26385353440064\\
5519	1.2980479566834\\
5520	1.39787354837957\\
5521	1.47625874299585\\
5522	1.52820040443716\\
5523	1.55524581819548\\
5524	1.55334182054214\\
5525	1.50031619220396\\
5526	1.38522649285232\\
5527	1.32629965749529\\
5528	1.2617783623159\\
5529	1.25078305474792\\
5530	1.2657068082433\\
5531	1.28207578567975\\
5532	1.31392872827748\\
5533	1.37707578332813\\
5534	1.35208132529964\\
5535	1.36415605853591\\
5536	1.35123203830619\\
5537	1.33983933643455\\
5538	1.26604563445237\\
5539	1.25547986011411\\
5540	1.23265588858431\\
5541	1.24827742220998\\
5542	1.24568815367616\\
5543	1.30241909461925\\
5544	1.36699408418879\\
5545	1.43292201656171\\
5546	1.48468065253771\\
5547	1.49748764087071\\
5548	1.493385466564\\
5549	1.41570133148861\\
5550	1.30405988640266\\
5551	1.22353343181764\\
5552	1.21859330773147\\
5553	1.21397605226448\\
5554	1.23706142686088\\
5555	1.24160134701325\\
5556	1.28637985789731\\
5557	1.28883063038037\\
5558	1.31048969653655\\
5559	1.28411528000547\\
5560	1.25989638148022\\
5561	1.21097944321349\\
5562	1.21921393275501\\
5563	1.21531651580521\\
5564	1.21357382965277\\
5565	1.18016151959555\\
5566	1.18756528926452\\
5567	1.25452761629584\\
5568	1.36015967609146\\
5569	1.42706333086089\\
5570	1.45831613942052\\
5571	1.48401446226914\\
5572	1.47145838993584\\
5573	1.42022403900495\\
5574	1.30662743064457\\
5575	1.22454813323655\\
5576	1.20045939304318\\
5577	1.25763646414477\\
5578	1.28241092387083\\
5579	1.28502988156895\\
5580	1.26523675758345\\
5581	1.29438533410845\\
5582	1.26653911991705\\
5583	1.27604981480897\\
5584	1.2751015275153\\
5585	1.28070628434447\\
5586	1.2282199219755\\
5587	1.25995562996603\\
5588	1.26609833675301\\
5589	1.29395654703754\\
5590	1.29534675930638\\
5591	1.33784159044315\\
5592	1.47212816508155\\
5593	1.54226359899258\\
5594	1.58749114269617\\
5595	1.62995292674642\\
5596	1.62095490684671\\
5597	1.53351055331166\\
5598	1.41372736192254\\
5599	1.35533900212133\\
5600	1.26603055245063\\
5601	1.30755719920688\\
5602	1.30603672996378\\
5603	1.29118846559368\\
5604	1.3167957260591\\
5605	1.28543304761577\\
5606	1.30418705045692\\
5607	1.3026572527409\\
5608	1.307465315595\\
5609	1.29866740907259\\
5610	1.29695946526879\\
5611	1.27567837986484\\
5612	1.24435184214656\\
5613	1.22886152215657\\
5614	1.22624536992362\\
5615	1.2840601358095\\
5616	1.39361771579369\\
5617	1.45982176849056\\
5618	1.39801177216908\\
5619	1.43449938925321\\
5620	1.46108772208603\\
5621	1.46265335738634\\
5622	1.44448252385334\\
5623	1.41296669666264\\
5624	1.38073775340474\\
5625	1.42095400934462\\
5626	1.42193549484815\\
5627	1.4514484720637\\
5628	1.47474605235213\\
5629	1.48453048761654\\
5630	1.55044464676665\\
5631	1.55640259780404\\
5632	1.50931312794429\\
5633	1.43630996667747\\
5634	1.37101325279547\\
5635	1.32010863107557\\
5636	1.27821193122439\\
5637	1.28696306696023\\
5638	1.31037389486524\\
5639	1.32563339576269\\
5640	1.40841548778639\\
5641	1.60504500409052\\
5642	1.71125782775489\\
5643	1.76004685608421\\
5644	1.76979131235766\\
5645	1.76470723138057\\
5646	1.72020998641363\\
5647	1.63358727834434\\
5648	1.61686176534449\\
5649	1.61289088773149\\
5650	1.62723555575142\\
5651	1.60721160973948\\
5652	1.62167374480168\\
5653	1.73456167743034\\
5654	1.81593543900601\\
5655	1.75317913874717\\
5656	1.72985374340393\\
5657	1.62205625260664\\
5658	1.49384782840842\\
5659	1.41770001662979\\
5660	1.37052119193058\\
5661	1.31681906463247\\
5662	1.34208518594155\\
5663	1.38182363289655\\
5664	1.48922218376806\\
5665	1.55407636243386\\
5666	1.59997772850005\\
5667	1.61082639187422\\
5668	1.59195972053319\\
5669	1.49558273931934\\
5670	1.35299787618642\\
5671	1.28770368716674\\
5672	1.2250055854912\\
5673	1.19620369386882\\
5674	1.18200286254244\\
5675	1.17113251289197\\
5676	1.15499172560057\\
5677	1.14842180188832\\
5678	1.14949266808446\\
5679	1.1372142454434\\
5680	1.12853806322252\\
5681	1.14061574324205\\
5682	1.14748550378589\\
5683	1.1823414532674\\
5684	1.20412657272146\\
5685	1.25465900420794\\
5686	1.26483047570847\\
5687	1.32516838618609\\
5688	1.41409906265847\\
5689	1.46775210072691\\
5690	1.51320459437839\\
5691	1.51054636219352\\
5692	1.48981826823772\\
5693	1.42817155634371\\
5694	1.29295243275999\\
5695	1.20641740825486\\
5696	1.16106747005629\\
5697	1.11434017174618\\
5698	1.10194914536474\\
5699	1.09736532930531\\
5700	1.12059801924251\\
5701	1.13034856795127\\
5702	1.14534234052611\\
5703	1.12469465654186\\
5704	1.1277398899236\\
5705	1.12483976674299\\
5706	1.15059214103803\\
5707	1.15933093324485\\
5708	1.18415526274676\\
5709	1.20216708595653\\
5710	1.22266215333283\\
5711	1.27557832292514\\
5712	1.39142884025304\\
5713	1.46998819146466\\
5714	1.51361466233164\\
5715	1.51879850114344\\
5716	1.49969661563402\\
5717	1.42502552974751\\
5718	1.29773198358998\\
5719	1.23467294420552\\
5720	1.17026404208852\\
5721	1.17698745942853\\
5722	1.20478908754291\\
5723	1.22449524960657\\
5724	1.24432587597307\\
5725	1.23982624892832\\
5726	1.24722414516958\\
5727	1.23753503882601\\
5728	1.22971316439363\\
5729	1.21919639367707\\
5730	1.17942736661767\\
5731	1.18415767407641\\
5732	1.20431160166579\\
5733	1.19604590154725\\
5734	1.19733361527127\\
5735	1.27759178524405\\
5736	1.39126004907849\\
5737	1.45686212090595\\
5738	1.51874624980784\\
5739	1.55252742222899\\
5740	1.54204021844437\\
5741	1.47441862799314\\
5742	1.3401970419681\\
5743	1.24826486651964\\
5744	1.19779157417811\\
5745	1.18454691491549\\
5746	1.17023765218167\\
5747	1.17110727012785\\
5748	1.19873527910845\\
5749	1.20296739703471\\
5750	1.21116334161797\\
5751	1.19375052617038\\
5752	1.20802076600923\\
5753	1.22297319114427\\
5754	1.20191039211255\\
5755	1.22139480617933\\
5756	1.23180192903231\\
5757	1.24622534193359\\
5758	1.27486434801798\\
5759	1.3376843091237\\
5760	1.44842229558077\\
5761	1.52453791509157\\
5762	1.5809775329307\\
5763	1.58897283172159\\
5764	1.57419868572254\\
5765	1.50264083035475\\
5766	1.36313985408534\\
5767	1.28041084101333\\
5768	1.23543293154148\\
5769	1.25706239198116\\
5770	1.28892914072334\\
5771	1.30699567796741\\
5772	1.33331000907527\\
5773	1.37344121539491\\
5774	1.38556095517996\\
5775	1.42977511474274\\
5776	1.42985517086575\\
5777	1.39787320229039\\
5778	1.35498583078723\\
5779	1.35588477449822\\
5780	1.35041598156146\\
5781	1.37608262429397\\
5782	1.37732730046673\\
5783	1.43087523554639\\
5784	1.57899557756904\\
5785	1.6517181921652\\
5786	1.59092565801811\\
5787	1.62774765936544\\
5788	1.64703012260402\\
5789	1.65405676786047\\
5790	1.60130870579257\\
5791	1.54582361190645\\
5792	1.46423660166304\\
5793	1.3781005850287\\
5794	1.32277298260219\\
5795	1.2926565201257\\
5796	1.30551433909178\\
5797	1.32308602621241\\
5798	1.33375316059592\\
5799	1.33424151896849\\
5800	1.32568880327332\\
5801	1.2820414833445\\
5802	1.25393490297106\\
5803	1.23974602084725\\
5804	1.26198972976862\\
5805	1.27866069066774\\
5806	1.31232566045382\\
5807	1.35040058887513\\
5808	1.42971626700813\\
5809	1.50579917981056\\
5810	1.55894276301035\\
5811	1.58576888105651\\
5812	1.5872827042239\\
5813	1.58438574239399\\
5814	1.56343109047493\\
5815	1.54560684339355\\
5816	1.46324900974792\\
5817	1.39234482821837\\
5818	1.35485851920503\\
5819	1.3287628335157\\
5820	1.35499376035937\\
5821	1.41532770671408\\
5822	1.50210036562348\\
5823	1.55670528481722\\
5824	1.54541576085679\\
5825	1.45307367049636\\
5826	1.36208460898171\\
5827	1.26443908826209\\
5828	1.267337748578\\
5829	1.26588035095846\\
5830	1.25296212696964\\
5831	1.29627469613704\\
5832	1.37954066203569\\
5833	1.43523422848389\\
5834	1.47195077473596\\
5835	1.47391719490485\\
5836	1.45448786215656\\
5837	1.36247730197042\\
5838	1.19656569126235\\
5839	1.11631788056405\\
5840	1.11136244123613\\
5841	1.13140099695223\\
5842	1.14470740689901\\
5843	1.15660102697161\\
5844	1.16539911701212\\
5845	1.1651216024041\\
5846	1.15654056786033\\
5847	1.14821710317504\\
5848	1.13468228610589\\
5849	1.11787106510119\\
5850	1.10391495610845\\
5851	1.08120069244967\\
5852	1.09948168633151\\
5853	1.11260845752486\\
5854	1.12036684355616\\
5855	1.18746772939355\\
5856	1.25693787343719\\
5857	1.31656775047032\\
5858	1.35822007813894\\
5859	1.37194552509346\\
5860	1.36056761068959\\
5861	1.29889106701967\\
5862	1.18203120086223\\
5863	1.08414306941227\\
5864	1.07700927025048\\
5865	1.09155187176332\\
5866	1.12648669596175\\
5867	1.13583658068826\\
5868	1.15758430339695\\
5869	1.16354850685126\\
5870	1.1596230287892\\
5871	1.16442807559557\\
5872	1.14733291422169\\
5873	1.12650849873158\\
5874	1.10960632103904\\
5875	1.11057383766464\\
5876	1.11044628641042\\
5877	1.10811693366603\\
5878	1.12187328085963\\
5879	1.18177823579072\\
5880	1.26391221428105\\
5881	1.31630228879579\\
5882	1.34893799697681\\
5883	1.3681620642827\\
5884	1.3458414812721\\
5885	1.28837921619505\\
5886	1.17473335316091\\
5887	1.08026358750502\\
5888	1.07559771877615\\
5889	1.09313862128104\\
5890	1.1191026130816\\
5891	1.14973327520784\\
5892	1.19156268073116\\
5893	1.21425800291035\\
5894	1.22991687507756\\
5895	1.22589865464954\\
5896	1.18745309985811\\
5897	1.14614640788419\\
5898	1.1348680929746\\
5899	1.13773304291437\\
5900	1.10131118578349\\
5901	1.10056741762111\\
5902	1.1384744124771\\
5903	1.24253504700937\\
5904	1.3058739313817\\
5905	1.35917322252037\\
5906	1.37024330973227\\
5907	1.37981721843537\\
5908	1.37247255899177\\
5909	1.30617275144677\\
5910	1.17930684913725\\
5911	1.07651353758528\\
5912	1.07456476284438\\
5913	1.08293092780262\\
5914	1.10162108974457\\
5915	1.11468061953916\\
5916	1.15836678216867\\
5917	1.16879001881938\\
5918	1.17871087902505\\
5919	1.16630708813726\\
5920	1.12636404805949\\
5921	1.10424215884954\\
5922	1.05847068200151\\
5923	1.05862476807818\\
5924	1.0435211169413\\
5925	1.06583050263021\\
5926	1.08775551475772\\
5927	1.16458799768259\\
5928	1.23547070162894\\
5929	1.28061690546641\\
5930	1.31796478025354\\
5931	1.32900004074685\\
5932	1.31582752334871\\
5933	1.27543839109679\\
5934	1.16273857711909\\
5935	1.06218355581919\\
5936	1.03022387452698\\
5937	1.02253738616966\\
5938	1.03970365329546\\
5939	1.05601974619787\\
5940	1.10180546878053\\
5941	1.1132680619926\\
5942	1.12398585528369\\
5943	1.13641874303321\\
5944	1.12207561363152\\
5945	1.10102923270691\\
5946	1.09174807632419\\
5947	1.11146812437987\\
5948	1.09617047926679\\
5949	1.10798960600717\\
5950	1.14577312650666\\
5951	1.15282352972269\\
5952	1.14937252999667\\
5953	1.22998859307845\\
5954	1.28636649569764\\
5955	1.30705328552898\\
5956	1.33260309517695\\
5957	1.32010918232318\\
5958	1.27044449245754\\
5959	1.24109621964143\\
5960	1.18015506863458\\
5961	1.14369354232299\\
5962	1.08634688933081\\
5963	1.06575715456617\\
5964	1.07898969503636\\
5965	1.11065768839045\\
5966	1.12391784592702\\
5967	1.137606959095\\
5968	1.12950818346217\\
5969	1.08163977827933\\
5970	1.09158872546733\\
5971	1.09060222864698\\
5972	1.06662591528974\\
5973	1.08727439935209\\
5974	1.09054434965721\\
5975	1.11805329530521\\
5976	1.15962480165253\\
5977	1.2006740370044\\
5978	1.21320624076844\\
5979	1.22438771886312\\
5980	1.2338737459194\\
5981	1.22629635890309\\
5982	1.19076560866288\\
5983	1.19855154216158\\
5984	1.16919392967301\\
5985	1.15702277685768\\
5986	1.14089629979018\\
5987	1.13640510453971\\
5988	1.16066070538113\\
5989	1.20587213577556\\
5990	1.23338370143521\\
5991	1.22357037280189\\
5992	1.26978755757116\\
5993	1.28673879213862\\
5994	1.21388307776201\\
5995	1.17066682182889\\
5996	1.1394785501454\\
5997	1.13889333366554\\
5998	1.15745863390384\\
5999	1.22926841923568\\
6000	1.32479479079691\\
6001	1.36478858247178\\
6002	1.37903759264649\\
6003	1.38747876493051\\
6004	1.34714189412515\\
6005	1.2717794789387\\
6006	1.12166962045505\\
6007	1.02835782266586\\
6008	1.01626684684203\\
6009	1.02271592357479\\
6010	1.02712649148635\\
6011	1.02616922826202\\
6012	1.06900017267024\\
6013	1.10174748485759\\
6014	1.11716098100827\\
6015	1.12415304688421\\
6016	1.09141509721763\\
6017	1.05000368320096\\
6018	1.0374995527732\\
6019	1.04046010329966\\
6020	1.06420219878623\\
6021	1.07885148383994\\
6022	1.1043960428336\\
6023	1.14735528107585\\
6024	1.22691682975854\\
6025	1.27956120900612\\
6026	1.31020875241832\\
6027	1.31661986529784\\
6028	1.31348174538791\\
6029	1.25032748965298\\
6030	1.12231456277701\\
6031	1.0573693651132\\
6032	1.05777487037285\\
6033	1.10030652177927\\
6034	1.07967636410607\\
6035	1.06651108704169\\
6036	1.11012940970004\\
6037	1.12126844307729\\
6038	1.11352020116095\\
6039	1.10954756534984\\
6040	1.08768542674534\\
6041	1.0524387552992\\
6042	1.03632052009409\\
6043	1.07342849763167\\
6044	1.06288498650172\\
6045	1.08745289994388\\
6046	1.09640671600792\\
6047	1.17134316839144\\
6048	1.24897424955856\\
6049	1.30015402267315\\
6050	1.33311624613131\\
6051	1.34558700865597\\
6052	1.336144079424\\
6053	1.27053655685253\\
6054	1.14257266731081\\
6055	1.04891996453425\\
6056	1.03364959432722\\
6057	1.06904519574322\\
6058	1.09213732937494\\
6059	1.10532546019099\\
6060	1.12869217742457\\
6061	1.14741905463951\\
6062	1.15889831361385\\
6063	1.14122796413349\\
6064	1.10993835110732\\
6065	1.11265603690942\\
6066	1.06254685589474\\
6067	1.08957638347858\\
6068	1.07864332670375\\
6069	1.06886800722926\\
6070	1.12384541352968\\
6071	1.17660654743931\\
6072	1.26057443816591\\
6073	1.31081636878636\\
6074	1.34745123278675\\
6075	1.35337381043442\\
6076	1.33039775506741\\
6077	1.26565224377484\\
6078	1.12236037875668\\
6079	1.06157945531746\\
6080	1.05790408010644\\
6081	1.04370457175235\\
6082	1.05654786350356\\
6083	1.04802636466725\\
6084	1.08616259191234\\
6085	1.12649112795149\\
6086	1.10957472410668\\
6087	1.12442563525481\\
6088	1.11733153015403\\
6089	1.07727398244711\\
6090	1.06963522818834\\
6091	1.09058190568092\\
6092	1.08735627929065\\
6093	1.08935939551936\\
6094	1.12812102392293\\
6095	1.18040528723248\\
6096	1.27665801441634\\
6097	1.33211691920725\\
6098	1.359730816202\\
6099	1.36545276805536\\
6100	1.35060214287512\\
6101	1.28682629096096\\
6102	1.14428221440036\\
6103	1.08637831144836\\
6104	1.06030982881213\\
6105	1.08772209753902\\
6106	1.11890876596549\\
6107	1.17206645635426\\
6108	1.20269346410765\\
6109	1.21802988990734\\
6110	1.22325818316481\\
6111	1.23861236771138\\
6112	1.20624846318948\\
6113	1.15878884757743\\
6114	1.12771533594868\\
6115	1.11505857643594\\
6116	1.08355480754121\\
6117	1.12216105028957\\
6118	1.13708045492727\\
6119	1.19129918860632\\
6120	1.28973000782803\\
6121	1.3608403105672\\
6122	1.39348387301045\\
6123	1.42825207811461\\
6124	1.45036297876256\\
6125	1.431637893475\\
6126	1.3730390242025\\
6127	1.3234361627931\\
6128	1.24492193264778\\
6129	1.22523807973216\\
6130	1.24271831778083\\
6131	1.26923015280247\\
6132	1.31857164737088\\
6133	1.36064065534514\\
6134	1.43042639634228\\
6135	1.44842752066435\\
6136	1.40980035645572\\
6137	1.33660313983074\\
6138	1.28756867835219\\
6139	1.26706012551224\\
6140	1.2630487370346\\
6141	1.28556742210903\\
6142	1.28667642124576\\
6143	1.35105852601948\\
6144	1.39920819426311\\
6145	1.43568656069422\\
6146	1.48350071541414\\
6147	1.5259872280348\\
6148	1.50820972395439\\
6149	1.49538219625831\\
6150	1.44014480100143\\
6151	1.40728367893047\\
6152	1.33824340026268\\
6153	1.26811371140451\\
6154	1.24427660659886\\
6155	1.23011889955567\\
6156	1.25714074859751\\
6157	1.32528508441351\\
6158	1.36704146857922\\
6159	1.40224131353645\\
6160	1.39237867810958\\
6161	1.33985175859981\\
6162	1.31494513930004\\
6163	1.26293439400911\\
6164	1.22185704867832\\
6165	1.24917837666145\\
6166	1.25336444089468\\
6167	1.31105004801059\\
6168	1.39272021035613\\
6169	1.44233253878802\\
6170	1.46347373189142\\
6171	1.47283428509874\\
6172	1.44929268097609\\
6173	1.35067698585689\\
6174	1.18066787221276\\
6175	1.08178084191956\\
6176	1.06098054869838\\
6177	1.05636294274117\\
6178	1.0589506559257\\
6179	1.07013294906244\\
6180	1.11663067339669\\
6181	1.12752656381411\\
6182	1.12450394608213\\
6183	1.11853091226587\\
6184	1.10041441623096\\
6185	1.07661140038863\\
6186	1.04879083077439\\
6187	1.0403406776939\\
6188	1.04296457362708\\
6189	1.07610584627451\\
6190	1.11059202419838\\
6191	1.17783596480805\\
6192	1.27601713496547\\
6193	1.31607136031382\\
6194	1.35341744603008\\
6195	1.35138055424374\\
6196	1.33716289316041\\
6197	1.26584526671437\\
6198	1.11999379658759\\
6199	1.04115272999355\\
6200	1.03111949267814\\
6201	1.03526393389279\\
6202	1.07010684022263\\
6203	1.07702100184578\\
6204	1.12602337834447\\
6205	1.12680613811287\\
6206	1.13943831036479\\
6207	1.13085972592705\\
6208	1.08869784246234\\
6209	1.05478517286334\\
6210	1.03445610089754\\
6211	1.04156224706206\\
6212	1.0267209985147\\
6213	1.0569340706636\\
6214	1.09015673144429\\
6215	1.15369247150126\\
6216	1.26406558668682\\
6217	1.30830238327794\\
6218	1.3574709687194\\
6219	1.3545359207062\\
6220	1.34529857530903\\
6221	1.28452809977606\\
6222	1.13505862095601\\
6223	1.06018121914627\\
6224	1.06079363438171\\
6225	1.08662937534408\\
6226	1.09069267867392\\
6227	1.09380175025575\\
6228	1.14698781157204\\
6229	1.16457028762464\\
6230	1.16752522613824\\
6231	1.15140659166339\\
6232	1.14360928618166\\
6233	1.11090849013535\\
6234	1.06752203085328\\
6235	1.0660478212801\\
6236	1.05177412379669\\
6237	1.08836229888804\\
6238	1.13117857282727\\
6239	1.19743101334089\\
6240	1.27233062058796\\
6241	1.3405762519598\\
6242	1.37287331299431\\
6243	1.37333943463659\\
6244	1.35523204941726\\
6245	1.28679679179803\\
6246	1.15210127989248\\
6247	1.05610422175811\\
6248	1.02328918272038\\
6249	1.0056835200973\\
6250	1.00059279973531\\
6251	0.97507528725633\\
6252	1.01539917642716\\
6253	1.02382828167901\\
6254	1.0204997673555\\
6255	1.02874798914277\\
6256	1.02842637889958\\
6257	1.003221943468\\
6258	1.00906794426239\\
6259	1.02481714065144\\
6260	1.01466166234438\\
6261	1.06512463445152\\
6262	1.10900063712369\\
6263	1.16571909896689\\
6264	1.23516422725939\\
6265	1.28348310828458\\
6266	1.33421539911658\\
6267	1.34086978635431\\
6268	1.32742756609383\\
6269	1.26240126683649\\
6270	1.12474082129364\\
6271	1.03164483894063\\
6272	1.01982461541432\\
6273	1.00652577682091\\
6274	1.01720748497982\\
6275	1.02243355366693\\
6276	1.04608035205603\\
6277	1.05603602531326\\
6278	1.04791627831746\\
6279	1.05401417447043\\
6280	1.04920638100652\\
6281	1.03987776459649\\
6282	1.04248504705885\\
6283	1.0687470613805\\
6284	1.04620568071051\\
6285	1.08482727601794\\
6286	1.09811387572476\\
6287	1.14341263381382\\
6288	1.2350023808362\\
6289	1.2981220981621\\
6290	1.35632714322208\\
6291	1.38652180526813\\
6292	1.40009152398466\\
6293	1.39508588806609\\
6294	1.35641050551636\\
6295	1.28204189592731\\
6296	1.23315566424165\\
6297	1.17475253730895\\
6298	1.19433609390827\\
6299	1.20553811036044\\
6300	1.22153658277807\\
6301	1.22526043163222\\
6302	1.2483741798257\\
6303	1.25022299978474\\
6304	1.24119786007172\\
6305	1.21581292871173\\
6306	1.18469865011385\\
6307	1.18963553267854\\
6308	1.15385926292798\\
6309	1.20946490421989\\
6310	1.22044079720181\\
6311	1.28311960131124\\
6312	1.35242260801567\\
6313	1.4152817383348\\
6314	1.47582541743518\\
6315	1.54356540675367\\
6316	1.58938054373749\\
6317	1.60679665978099\\
6318	1.58196283678221\\
6319	1.51826973181335\\
6320	1.45009900539192\\
6321	1.3595650200756\\
6322	1.32644423703973\\
6323	1.33172374843957\\
6324	1.4026596764631\\
6325	1.51880094354691\\
6326	1.5924808246851\\
6327	1.63002213589022\\
6328	1.5962092652692\\
6329	1.48964086800758\\
6330	1.35386540900063\\
6331	1.29824960134502\\
6332	1.2697692956939\\
6333	1.28024253432463\\
6334	1.28461633319719\\
6335	1.36597822794308\\
6336	1.40965434994522\\
6337	1.46216916702351\\
6338	1.48762450170886\\
6339	1.51114242431265\\
6340	1.5019743392863\\
6341	1.40224780567657\\
6342	1.20639935923502\\
6343	1.10677913582642\\
6344	1.07213273481938\\
6345	1.05625510620986\\
6346	1.06951219926787\\
6347	1.07585959519414\\
6348	1.12317265120748\\
6349	1.1296675248566\\
6350	1.13172392171373\\
6351	1.13086035869675\\
6352	1.11263081019853\\
6353	1.10071524707479\\
6354	1.0641804341889\\
6355	1.05974108240756\\
6356	1.0547183621253\\
6357	1.07390513505327\\
6358	1.1073096936615\\
6359	1.18368237389488\\
6360	1.24850245359162\\
6361	1.30888703433087\\
6362	1.34637111717395\\
6363	1.36293965426417\\
6364	1.33876862211356\\
6365	1.2691763502699\\
6366	1.11850237493455\\
6367	1.02360173869168\\
6368	1.02511567763761\\
6369	1.04849241440744\\
6370	1.05786192018167\\
6371	1.07619231696406\\
6372	1.11613846045216\\
6373	1.13095913344736\\
6374	1.13104783995809\\
6375	1.11642625522885\\
6376	1.0861961940058\\
6377	1.05612535736652\\
6378	1.0435754824804\\
6379	1.02429948747093\\
6380	1.01284185188271\\
6381	1.06091694920792\\
6382	1.09140962127799\\
6383	1.15993819666373\\
6384	1.25520216686525\\
6385	1.32424633704283\\
6386	1.38276173377805\\
6387	1.40383550218864\\
6388	1.396857166497\\
6389	1.33549590379938\\
6390	1.18054798560843\\
6391	1.0841891163432\\
6392	1.07062072550032\\
6393	1.0669681010496\\
6394	1.07894532467343\\
6395	1.05547985580713\\
6396	1.07076327667464\\
6397	1.07164606045432\\
6398	1.08509892424757\\
6399	1.08816544315345\\
6400	1.09056934656639\\
6401	1.06342602307418\\
6402	1.08169116356397\\
6403	1.08876736935036\\
6404	1.07665504715491\\
6405	1.12382761091477\\
6406	1.15058808546595\\
6407	1.22224361334871\\
6408	1.30763003257696\\
6409	1.36521056448212\\
6410	1.39345205910147\\
6411	1.39926903500698\\
6412	1.38669492789855\\
6413	1.30606873480122\\
6414	1.15066832965282\\
6415	1.04890706033846\\
6416	1.03919585150504\\
6417	1.04677761128502\\
6418	1.06645927095258\\
6419	1.08165657769298\\
6420	1.13945879419626\\
6421	1.15679044298635\\
6422	1.17216953095948\\
6423	1.17372075740324\\
6424	1.16192648818333\\
6425	1.13650926808378\\
6426	1.09767289110602\\
6427	1.0888806500235\\
6428	1.07068577027992\\
6429	1.12065392151705\\
6430	1.18027551976377\\
6431	1.23238399747165\\
6432	1.2866599207292\\
6433	1.35111434009088\\
6434	1.39727528562092\\
6435	1.4152538640466\\
6436	1.39784717918361\\
6437	1.32896363715623\\
6438	1.16729278572394\\
6439	1.0557261221014\\
6440	1.05652997107488\\
6441	1.03958069307119\\
6442	1.06183240848527\\
6443	1.05986732720205\\
6444	1.05915659132188\\
6445	1.06552326424243\\
6446	1.05564741488684\\
6447	1.04975711557905\\
6448	1.02390639130751\\
6449	1.00743468819342\\
6450	1.00520263341083\\
6451	1.00147274727354\\
6452	1.02189511073797\\
6453	1.06522029500804\\
6454	1.06954918322061\\
6455	1.11697818325123\\
6456	1.19068440665822\\
6457	1.19987207687246\\
6458	1.2516398162406\\
6459	1.27969996655668\\
6460	1.28952704442446\\
6461	1.28450521566483\\
6462	1.23614002651555\\
6463	1.17410814087044\\
6464	1.12168097536531\\
6465	1.08657262610254\\
6466	1.06985940802181\\
6467	1.0811698917824\\
6468	1.1347521662066\\
6469	1.16590469344863\\
6470	1.19535333595407\\
6471	1.2084325391962\\
6472	1.18359640864364\\
6473	1.12548897465758\\
6474	1.09947214097822\\
6475	1.08176567244242\\
6476	1.06680200457655\\
6477	1.09058277901997\\
6478	1.12580729712709\\
6479	1.17756294098773\\
6480	1.2608748935644\\
6481	1.37621991794089\\
6482	1.4332060190306\\
6483	1.46631424720739\\
6484	1.47553989091465\\
6485	1.46236134174885\\
6486	1.40439536029458\\
6487	1.36500440021416\\
6488	1.33844322115599\\
6489	1.30502103963503\\
6490	1.30812624201994\\
6491	1.32939135662171\\
6492	1.37115015196389\\
6493	1.45355714095344\\
6494	1.4886351809948\\
6495	1.48111452370008\\
6496	1.43969476954237\\
6497	1.31199470069318\\
6498	1.18413965442275\\
6499	1.07784686310774\\
6500	1.05907048229507\\
6501	1.14837224729582\\
6502	1.19482294068329\\
6503	1.26669129173711\\
6504	1.3557140970899\\
6505	1.40646378046396\\
6506	1.45277781216671\\
6507	1.45459835894429\\
6508	1.4307583616594\\
6509	1.32722070155798\\
6510	1.16354212517854\\
6511	1.05787684044037\\
6512	1.04619865312199\\
6513	1.06153238358974\\
6514	1.07569129643722\\
6515	1.07675027897413\\
6516	1.1123057228909\\
6517	1.10885745184698\\
6518	1.10091085672828\\
6519	1.03534829360777\\
6520	1.0146662921039\\
6521	0.968181360991986\\
6522	0.953340839431368\\
6523	0.966718322843213\\
6524	0.989141898021704\\
6525	1.02091593940525\\
6526	1.06306912545735\\
6527	1.12761281205554\\
6528	1.21857610042671\\
6529	1.29287428809378\\
6530	1.34872478889377\\
6531	1.3411028215546\\
6532	1.34122349366009\\
6533	1.27697486402254\\
6534	1.06171837896335\\
6535	0.972001227496914\\
6536	0.957372389217668\\
6537	0.951803328308159\\
6538	0.974665831623613\\
6539	0.994080137280165\\
6540	1.01028131838187\\
6541	1.01280003179309\\
6542	1.05316783730325\\
6543	1.0464016290494\\
6544	1.03401488664035\\
6545	0.9919435322977\\
6546	0.974019082196425\\
6547	0.962525427621652\\
6548	0.966396311433782\\
6549	0.997435296249056\\
6550	1.03752215402897\\
6551	1.10403313810877\\
6552	1.20991494825769\\
6553	1.26310349113423\\
6554	1.32077104046606\\
6555	1.34201045454009\\
6556	1.32154917291404\\
6557	1.25711820616351\\
6558	1.10681082470751\\
6559	1.00303856246414\\
6560	1.02707005678177\\
6561	1.03733824741815\\
6562	1.06804005753761\\
6563	1.05819545602494\\
6564	1.1028996244047\\
6565	1.1006121154901\\
6566	1.10327200925608\\
6567	1.09796082841257\\
6568	1.07828091286787\\
6569	1.0262524457742\\
6570	0.996889087741182\\
6571	1.00709860226146\\
6572	1.01949688616563\\
6573	1.01273060640143\\
6574	1.03566121558723\\
6575	1.10524254464352\\
6576	1.18205784854624\\
6577	1.23675427129827\\
6578	1.26390832129631\\
6579	1.2900256994354\\
6580	1.27516742144448\\
6581	1.20974111506368\\
6582	1.06453032497273\\
6583	0.9717649230554\\
6584	0.966932372127469\\
6585	0.965721249812264\\
6586	0.975361095887292\\
6587	0.974085724355514\\
6588	0.991157129573679\\
6589	0.998546382756248\\
6590	0.990945045985716\\
6591	0.982027036702531\\
6592	0.976214682038873\\
6593	0.965597683521814\\
6594	0.969216403666555\\
6595	0.972754029985539\\
6596	0.977966626348916\\
6597	0.995353790690302\\
6598	1.0289972906188\\
6599	1.10074297291895\\
6600	1.17611771804873\\
6601	1.22196575920732\\
6602	1.25571212957983\\
6603	1.27784807928109\\
6604	1.26292805954421\\
6605	1.203996115373\\
6606	1.06818789429423\\
6607	0.962063296549035\\
6608	0.977326963047017\\
6609	0.995620510095195\\
6610	1.02935597522665\\
6611	1.02061400902273\\
6612	1.07952338207332\\
6613	1.09257008437647\\
6614	1.09422362423743\\
6615	1.09479707971062\\
6616	1.06500122875222\\
6617	1.02936005442554\\
6618	1.01570387942216\\
6619	1.0156528248611\\
6620	0.999163839785109\\
6621	1.0299421730448\\
6622	1.06214803024598\\
6623	1.11491709384988\\
6624	1.22073734551062\\
6625	1.29499796668145\\
6626	1.36315224818882\\
6627	1.40455797039164\\
6628	1.4610325034006\\
6629	1.47842148317372\\
6630	1.4211420219806\\
6631	1.34687935451874\\
6632	1.32526732443771\\
6633	1.33523733085211\\
6634	1.35229751311211\\
6635	1.36378854590468\\
6636	1.40626169933279\\
6637	1.42566813840641\\
6638	1.43771422185611\\
6639	1.3865811321534\\
6640	1.33176549422743\\
6641	1.24584239115482\\
6642	1.21165192524454\\
6643	1.22295827150337\\
6644	1.20886470558974\\
6645	1.22859046759303\\
6646	1.27631999517119\\
6647	1.31994316516016\\
6648	1.4064365074846\\
6649	1.47637547049649\\
6650	1.51738956116994\\
6651	1.52421971262825\\
6652	1.52026886626543\\
6653	1.48474594519533\\
6654	1.43000408768385\\
6655	1.35703554121798\\
6656	1.31558004551559\\
6657	1.26120129361356\\
6658	1.22232426844006\\
6659	1.19725150667439\\
6660	1.19813698254526\\
6661	1.2285323397458\\
6662	1.26835856061023\\
6663	1.26550275754198\\
6664	1.23366153116343\\
6665	1.17765489586021\\
6666	1.10384797983668\\
6667	1.10720834450425\\
6668	1.11754591859181\\
6669	1.10522173023921\\
6670	1.14042736484063\\
6671	1.19736634260638\\
6672	1.28298096604447\\
6673	1.35006931068142\\
6674	1.39436881100193\\
6675	1.39545136445762\\
6676	1.37656072209548\\
6677	1.29861035441242\\
6678	1.11534421401455\\
6679	0.989740393677629\\
6680	0.996439112395265\\
6681	0.991134463374204\\
6682	1.01813670806303\\
6683	1.03172031083869\\
6684	1.05820127099352\\
6685	1.03415387762129\\
6686	1.03661115447253\\
6687	1.07045434793573\\
6688	1.05746018526789\\
6689	1.03421673059398\\
6690	1.03469560491722\\
6691	1.07826780902852\\
6692	1.11409373433527\\
6693	1.16327259088569\\
6694	1.20322366181909\\
6695	1.28097851957772\\
6696	1.38393179420443\\
6697	1.48593377113569\\
6698	1.53834970928921\\
6699	1.56558916206962\\
6700	1.53645185387728\\
6701	1.42889130852181\\
6702	1.23309110220794\\
6703	1.08352016234384\\
6704	1.1127055714725\\
6705	1.13392548025908\\
6706	1.16389088303529\\
6707	1.15661225978082\\
6708	1.24433275789985\\
6709	1.21557665195322\\
6710	1.24144727270521\\
6711	1.2475966186886\\
6712	1.24011880046443\\
6713	1.17393181241229\\
6714	1.10457197869373\\
6715	1.10847768914979\\
6716	1.10983348046282\\
6717	1.1393993600164\\
6718	1.18133739779865\\
6719	1.235507223003\\
6720	1.33126258717378\\
6721	1.39650505108989\\
6722	1.43994378449407\\
6723	1.44056023770341\\
6724	1.41372943364306\\
6725	1.33688301877372\\
6726	1.18270347729878\\
6727	1.08817262712472\\
6728	1.07567862042529\\
6729	1.04085748795842\\
6730	1.04607098835442\\
6731	1.03526269560239\\
6732	1.0294420172673\\
6733	1.05013222146517\\
6734	1.08024223923421\\
6735	1.1072310867184\\
6736	1.12905648175428\\
6737	1.10733714565771\\
6738	1.08782017910056\\
6739	1.10504124164922\\
6740	1.11157900284245\\
6741	1.15502953263868\\
6742	1.17198091377341\\
6743	1.24323410758501\\
6744	1.33661360689791\\
6745	1.40487173403863\\
6746	1.44115160086132\\
6747	1.46252645664046\\
6748	1.45062743644539\\
6749	1.36568564425024\\
6750	1.19449021941435\\
6751	1.08871494267826\\
6752	1.05662092797719\\
6753	1.09417255438799\\
6754	1.09296409702309\\
6755	1.11420920700696\\
6756	1.20140443986314\\
6757	1.19332758617604\\
6758	1.16522375894883\\
6759	1.14680374608857\\
6760	1.13650829841715\\
6761	1.09499575094029\\
6762	1.10776188516708\\
6763	1.08681959981613\\
6764	1.11242844480959\\
6765	1.1384449799993\\
6766	1.18297722484497\\
6767	1.23204060617058\\
6768	1.3409530198561\\
6769	1.41685419230562\\
6770	1.47929966747219\\
6771	1.4879489898328\\
6772	1.47003481546988\\
6773	1.36383820309061\\
6774	1.17628742414817\\
6775	1.04615382733413\\
6776	1.06810896178931\\
6777	1.0766482968258\\
6778	1.09301365261411\\
6779	1.06360550716379\\
6780	1.12454575229459\\
6781	1.13174936458881\\
6782	1.13038205275117\\
6783	1.14158321667999\\
6784	1.12463489047063\\
6785	1.08686408817342\\
6786	1.04904596189162\\
6787	1.03149061014403\\
6788	1.02147845839543\\
6789	1.06319031328911\\
6790	1.07594177844274\\
6791	1.1078125314193\\
6792	1.2128541373209\\
6793	1.30201435157995\\
6794	1.35358548160911\\
6795	1.38417977696096\\
6796	1.39255985051169\\
6797	1.36590750810118\\
6798	1.30223819084585\\
6799	1.22073479911604\\
6800	1.1716833749041\\
6801	1.14864986558153\\
6802	1.14152920743883\\
6803	1.14952527380694\\
6804	1.17455271985975\\
6805	1.18674163919165\\
6806	1.22681536083519\\
6807	1.23293206026219\\
6808	1.21746829889102\\
6809	1.185672205983\\
6810	1.1685146271354\\
6811	1.12583216730776\\
6812	1.15132432682927\\
6813	1.18434723790799\\
6814	1.23059473177806\\
6815	1.27475610249237\\
6816	1.32448151483397\\
6817	1.39200846167806\\
6818	1.42539448649117\\
6819	1.43392287391302\\
6820	1.43447275226822\\
6821	1.40449324407957\\
6822	1.36485168507107\\
6823	1.30323442787665\\
6824	1.26803978159475\\
6825	1.24232058772887\\
6826	1.23212433388416\\
6827	1.2546734352652\\
6828	1.27907713128063\\
6829	1.31906698045827\\
6830	1.35102753246507\\
6831	1.35337925982347\\
6832	1.33623788921047\\
6833	1.29221617639035\\
6834	1.25922631765158\\
6835	1.23010685601932\\
6836	1.28405405537422\\
6837	1.31118539330367\\
6838	1.30940430563317\\
6839	1.35040461803892\\
6840	1.41162934222784\\
6841	1.46826413480329\\
6842	1.50447551030341\\
6843	1.48603789873669\\
6844	1.4462016474513\\
6845	1.34531453481803\\
6846	1.18971406850158\\
6847	1.10628752075115\\
6848	1.12107124400362\\
6849	1.15341841040044\\
6850	1.15145286814253\\
6851	1.13813557787785\\
6852	1.16063061824991\\
6853	1.15468712140335\\
6854	1.1585751195495\\
6855	1.15579847897405\\
6856	1.14400480235947\\
6857	1.10576690878591\\
6858	1.07943299662117\\
6859	1.07002945794879\\
6860	1.10554883968955\\
6861	1.13577321315801\\
6862	1.14659581354189\\
6863	1.21337867464061\\
6864	1.31687066907833\\
6865	1.41213216127112\\
6866	1.46243282553604\\
6867	1.46033559666298\\
6868	1.45196885758168\\
6869	1.35656220369682\\
6870	1.19329837142654\\
6871	1.05621824963868\\
6872	1.08262425756697\\
6873	1.10922887085\\
6874	1.14650891007591\\
6875	1.14385133863084\\
6876	1.19310148384587\\
6877	1.16687713577619\\
6878	1.15848928243148\\
6879	1.13743961958386\\
6880	1.11129335403783\\
6881	1.05810327364324\\
6882	1.02876846069083\\
6883	1.00776107351858\\
6884	1.02885280347141\\
6885	1.06592686594645\\
6886	1.08137788075266\\
6887	1.14412944144124\\
6888	1.21736679236901\\
6889	1.22725581080623\\
6890	1.26210115514195\\
6891	1.26827516103396\\
6892	1.23879804930237\\
6893	1.17743677800395\\
6894	1.03784036751909\\
6895	0.94088416929177\\
6896	0.999429611507393\\
6897	1.01431655137818\\
6898	1.01266210450956\\
6899	1.01399080766388\\
6900	1.04199240055136\\
6901	1.04450627652745\\
6902	1.04842995397843\\
6903	1.04029991087836\\
6904	1.02665413714874\\
6905	0.994330478815297\\
6906	0.97010776729696\\
6907	0.950309371831047\\
6908	0.984427497075079\\
6909	1.0311092857701\\
6910	1.05456159736668\\
6911	1.11043089315896\\
6912	1.21560880139525\\
6913	1.28301193945745\\
6914	1.34264503710169\\
6915	1.35703329946491\\
6916	1.35717988984943\\
6917	1.30091964139476\\
6918	1.15161521146225\\
6919	1.02706240569151\\
6920	1.03736260182257\\
6921	1.03203931935318\\
6922	1.03404140340997\\
6923	1.04645906384557\\
6924	1.06512219752681\\
6925	1.0754379248127\\
6926	1.09042300223007\\
6927	1.08479861652835\\
6928	1.08217902966293\\
6929	1.02969001269196\\
6930	0.992431622124358\\
6931	1.00655407317896\\
6932	1.02346021408403\\
6933	1.04847029064283\\
6934	1.06516121859179\\
6935	1.10553919341309\\
6936	1.22267018536434\\
6937	1.29648243326967\\
6938	1.33725948585811\\
6939	1.34538768537787\\
6940	1.3197033411302\\
6941	1.25443787983769\\
6942	1.11955460878253\\
6943	1.00876635299961\\
6944	1.00088609972708\\
6945	1.01168624018989\\
6946	1.02718902799072\\
6947	1.05932214040482\\
6948	1.0982458112139\\
6949	1.10648409953122\\
6950	1.1202625618742\\
6951	1.11429458682805\\
6952	1.08728674435378\\
6953	1.06137644167798\\
6954	1.02179831326305\\
6955	1.01328224905639\\
6956	1.04088374727933\\
6957	1.10704093437705\\
6958	1.1237992440107\\
6959	1.17869015578986\\
6960	1.29541910390175\\
6961	1.37890615271704\\
6962	1.47367469224211\\
6963	1.51029997186562\\
6964	1.54548378909121\\
6965	1.55479422578487\\
6966	1.46492129408149\\
6967	1.35257623637463\\
6968	1.29809689644421\\
6969	1.29854143644691\\
6970	1.26002325934467\\
6971	1.25582135561944\\
6972	1.32297772260869\\
6973	1.36154790661459\\
6974	1.39003332474836\\
6975	1.37369642692658\\
6976	1.34498267289797\\
6977	1.24605494148905\\
6978	1.14844846882675\\
6979	1.15473713351984\\
6980	1.22119711096354\\
6981	1.27081307429859\\
6982	1.26633022186902\\
6983	1.31878229212258\\
6984	1.42835542741908\\
6985	1.5076104141312\\
6986	1.59748136727985\\
6987	1.64524638938996\\
6988	1.64042581999128\\
6989	1.62970565627301\\
6990	1.57130553741902\\
6991	1.49025491031799\\
6992	1.43839899187725\\
6993	1.42569185822938\\
6994	1.41692303544156\\
6995	1.42182753380738\\
6996	1.46190510273933\\
6997	1.56228161502629\\
6998	1.5813937892952\\
6999	1.57322422744615\\
7000	1.56550569113687\\
7001	1.44014498890027\\
7002	1.33914413522088\\
7003	1.25345411008865\\
7004	1.27199479311024\\
7005	1.30789686073201\\
7006	1.34851944709077\\
7007	1.39725787265236\\
7008	1.54354399129661\\
7009	1.61824252799252\\
7010	1.63565672887708\\
7011	1.64914052194\\
7012	1.60893962006132\\
7013	1.49274729780713\\
7014	1.26134037701063\\
7015	1.09230571420108\\
7016	1.07192570619372\\
7017	1.07528368022789\\
7018	1.08720826557139\\
7019	1.09236344249377\\
7020	1.13056517463954\\
7021	1.12479946470986\\
7022	1.12880503954555\\
7023	1.12880366691605\\
7024	1.11965364349117\\
7025	1.07950540228618\\
7026	1.10173163023126\\
7027	1.115917746875\\
7028	1.1341791044516\\
7029	1.22229376146049\\
7030	1.2328862264288\\
7031	1.28844126319324\\
7032	1.36478840117516\\
7033	1.46198557362463\\
7034	1.52117866264049\\
7035	1.55706191288974\\
7036	1.53271367090381\\
7037	1.45028916796151\\
7038	1.26586965080933\\
7039	1.14190837677352\\
7040	1.12465335123485\\
7041	1.1273659102006\\
7042	1.13018451629937\\
7043	1.0964068325515\\
7044	1.14104832447419\\
7045	1.14951308681747\\
7046	1.16722406680986\\
7047	1.19077010251335\\
7048	1.19392003566595\\
7049	1.14751343958625\\
7050	1.150830716494\\
7051	1.17085666462176\\
7052	1.25232667478089\\
7053	1.30870079073767\\
7054	1.33206553527051\\
7055	1.39684265958892\\
7056	1.47455881362571\\
7057	1.56783254945078\\
7058	1.62166012291419\\
7059	1.64160936151336\\
7060	1.60486193122236\\
7061	1.50908860104034\\
7062	1.28972387665833\\
7063	1.13890642271139\\
7064	1.14733250016793\\
7065	1.16984774504266\\
7066	1.19617812624426\\
7067	1.19737843566587\\
7068	1.21605177298939\\
7069	1.23804363635746\\
7070	1.23623052346014\\
7071	1.20999717857827\\
7072	1.19019014609573\\
7073	1.15622639618842\\
7074	1.06764745855317\\
7075	1.08885904081132\\
7076	1.08437365204746\\
7077	1.14770529467111\\
7078	1.18852065960063\\
7079	1.26938820956142\\
7080	1.36271647648119\\
7081	1.4283164943781\\
7082	1.48496129157021\\
7083	1.50252933454902\\
7084	1.48204719680036\\
7085	1.4021679961566\\
7086	1.23529427734295\\
7087	1.11219602509224\\
7088	1.12146100434064\\
7089	1.13667434906374\\
7090	1.14891715763596\\
7091	1.14377467177243\\
7092	1.17578284086238\\
7093	1.16391003475416\\
7094	1.17728573696249\\
7095	1.15453612543353\\
7096	1.15192540198619\\
7097	1.13451644717194\\
7098	1.14466888456913\\
7099	1.14036902141265\\
7100	1.19866590556623\\
7101	1.23948796957708\\
7102	1.26810551566091\\
7103	1.3143893651321\\
7104	1.40489980758166\\
7105	1.50563200105343\\
7106	1.57004892804628\\
7107	1.59682086346622\\
7108	1.58125895181316\\
7109	1.49280893594251\\
7110	1.31183327000214\\
7111	1.21192972233685\\
7112	1.18176335146459\\
7113	1.14679925067967\\
7114	1.15029777239556\\
7115	1.14196647196734\\
7116	1.13020773802295\\
7117	1.15835365515408\\
7118	1.18832270711159\\
7119	1.18496767818014\\
7120	1.18099564984477\\
7121	1.16829484397023\\
7122	1.11744601776078\\
7123	1.18219759526424\\
7124	1.20543229998669\\
7125	1.26252792223447\\
7126	1.25278692000894\\
7127	1.22324669353896\\
7128	1.30858807521286\\
7129	1.38090878147762\\
7130	1.42875828703677\\
7131	1.45684924323796\\
7132	1.45504415411736\\
7133	1.42322829819961\\
7134	1.35424716259617\\
7135	1.2856539525844\\
7136	1.20102787980958\\
7137	1.18226501364758\\
7138	1.18247701948999\\
7139	1.16849944340784\\
7140	1.16179786905739\\
7141	1.25165954439495\\
7142	1.29360645674658\\
7143	1.30618514385679\\
7144	1.27968649141031\\
7145	1.21661791373529\\
7146	1.16425065104958\\
7147	1.20505020695239\\
7148	1.22680771272074\\
7149	1.27888650552904\\
7150	1.2714643627725\\
7151	1.28653772982765\\
7152	1.40892117028263\\
7153	1.50990560355304\\
7154	1.57490158711406\\
7155	1.59641867872458\\
7156	1.6125041902525\\
7157	1.60612061593765\\
7158	1.61612958414921\\
7159	1.56410612815446\\
7160	1.51095107722316\\
7161	1.45705897862537\\
7162	1.45991103933224\\
7163	1.4904206829563\\
7164	1.49230245856183\\
7165	1.50237205826144\\
7166	1.54456520444981\\
7167	1.54910980535535\\
7168	1.4859662347439\\
7169	1.45815439900055\\
7170	1.36571789763384\\
7171	1.27163710271483\\
7172	1.30464337296201\\
7173	1.38243247272\\
7174	1.40998766867369\\
7175	1.43188219852521\\
7176	1.42386199637341\\
7177	1.54595280698592\\
7178	1.63495486515114\\
7179	1.68692601952083\\
7180	1.68248507056633\\
7181	1.63720349220878\\
7182	1.52252436056007\\
7183	1.33503335701666\\
7184	1.23144808692181\\
7185	1.22203074458404\\
7186	1.23201171515131\\
7187	1.25698904540119\\
7188	1.31130751287233\\
7189	1.30245654909858\\
7190	1.30017853485907\\
7191	1.29745844869234\\
7192	1.26116644693038\\
7193	1.20082142580462\\
7194	1.16389617602985\\
7195	1.10290795857398\\
7196	1.15113167337087\\
7197	1.17151663783608\\
7198	1.23753568468017\\
7199	1.27706643936252\\
7200	1.31837113758805\\
7201	1.41722099504143\\
7202	1.50709774477095\\
7203	1.55848654413372\\
7204	1.58324766603661\\
7205	1.55133698583864\\
7206	1.46064927158724\\
7207	1.2887071765976\\
7208	1.21026211169725\\
7209	1.21958233481824\\
7210	1.24627006555265\\
7211	1.25974197753527\\
7212	1.27202018754474\\
7213	1.29839540385934\\
7214	1.29215277205447\\
7215	1.28710362158937\\
7216	1.24281032182655\\
7217	1.17941221336937\\
7218	1.12548633946012\\
7219	1.10507767827764\\
7220	1.13522651810542\\
7221	1.18055767906335\\
7222	1.24580573421381\\
7223	1.25435693710732\\
7224	1.29055265800647\\
7225	1.40662603645394\\
7226	1.48733019071202\\
7227	1.5425725669344\\
7228	1.55084653359096\\
7229	1.52407074856935\\
7230	1.43015089387623\\
7231	1.26875486293798\\
7232	1.17795082217821\\
7233	1.12424499809694\\
7234	1.1207202928696\\
7235	1.09654300070716\\
7236	1.07386928638935\\
7237	1.06940698643026\\
7238	1.06805835714938\\
7239	1.07173507328946\\
7240	1.07136410129226\\
7241	1.06135289816503\\
7242	1.05133249864721\\
7243	1.04443994433303\\
7244	1.08935355767761\\
7245	1.11666876809462\\
7246	1.15746165539138\\
7247	1.17755755131115\\
7248	1.1995628284929\\
7249	1.28864368082486\\
7250	1.36993102059566\\
7251	1.43614808287578\\
7252	1.44573869883087\\
7253	1.42944238653414\\
7254	1.35290935253116\\
7255	1.21328207196265\\
7256	1.16534336193683\\
7257	1.13370399003733\\
7258	1.11364423317598\\
7259	1.1304697297083\\
7260	1.12948931878582\\
7261	1.16417766818839\\
7262	1.16110147722216\\
7263	1.16776763450796\\
7264	1.16246816617096\\
7265	1.15380812136816\\
7266	1.10720418758389\\
7267	1.05213079173931\\
7268	1.10335237509004\\
7269	1.18851917385072\\
7270	1.2370957846523\\
7271	1.25997778472207\\
7272	1.2818493321489\\
7273	1.39597654164446\\
7274	1.4975914283859\\
7275	1.56966990739023\\
7276	1.59187850442756\\
7277	1.5581663529297\\
7278	1.46578311042\\
7279	1.29275710227124\\
7280	1.20772599445982\\
7281	1.18890555476759\\
7282	1.19446968330768\\
7283	1.23296996912909\\
7284	1.24576891155082\\
7285	1.26532526462999\\
7286	1.28520541443232\\
7287	1.29325916148401\\
7288	1.28277812068838\\
7289	1.25796692579806\\
7290	1.20268297005137\\
7291	1.17982612296326\\
7292	1.21132467828415\\
7293	1.28231549112805\\
7294	1.37169613686359\\
7295	1.37186220431969\\
7296	1.3494025094415\\
7297	1.48585874367668\\
7298	1.58036042801981\\
7299	1.65850831842235\\
7300	1.69690336294527\\
7301	1.71633406805881\\
7302	1.70064447503923\\
7303	1.63480002445063\\
7304	1.56785629722909\\
7305	1.55039692015983\\
7306	1.57515333913234\\
7307	1.59922363444031\\
7308	1.60360642238278\\
7309	1.58057979105424\\
7310	1.59191055582433\\
7311	1.60337748712402\\
7312	1.5806833249041\\
7313	1.48217664062599\\
7314	1.36572668449007\\
7315	1.34759493146839\\
7316	1.37270528856187\\
7317	1.42038373701682\\
7318	1.41700233550744\\
7319	1.38677709682858\\
7320	1.39817165255664\\
7321	1.54923863078182\\
7322	1.66786631602759\\
7323	1.74508426582008\\
7324	1.82106305470517\\
7325	1.84013817665236\\
7326	1.82551715756503\\
7327	1.74649698445822\\
7328	1.72171660619636\\
7329	1.71268443786388\\
7330	1.72776746188214\\
7331	1.7410367907147\\
7332	1.74340015031653\\
7333	1.69690267172663\\
7334	1.71152487451682\\
7335	1.6837941885451\\
7336	1.64387234030824\\
7337	1.59445038391844\\
7338	1.49588791868314\\
7339	1.43373698836182\\
7340	1.45092536545902\\
7341	1.51083125751249\\
7342	1.49278159732246\\
7343	1.50352051146666\\
7344	1.52973820363999\\
7345	1.68041366329489\\
7346	1.79173805384269\\
7347	1.85829655215958\\
7348	1.88815315236732\\
7349	1.84011079844167\\
7350	1.69604410525833\\
7351	1.45658687523597\\
7352	1.29768710490835\\
7353	1.29699347183618\\
7354	1.30975322814131\\
7355	1.27680187876811\\
7356	1.28657282556883\\
7357	1.29142590779471\\
7358	1.27195484691779\\
7359	1.26241574599898\\
7360	1.25762244653707\\
7361	1.21707647244036\\
7362	1.16504246941153\\
7363	1.16603041177295\\
7364	1.17767291644263\\
7365	1.23441727583983\\
7366	1.28080416362077\\
7367	1.27536638774925\\
7368	1.28995510917611\\
7369	1.37561398532459\\
7370	1.43894380576157\\
7371	1.48269908783761\\
7372	1.47295279557277\\
7373	1.44749046634773\\
7374	1.36307481398959\\
7375	1.20964535787835\\
7376	1.13096130502688\\
7377	1.14049206803113\\
7378	1.10567684146393\\
7379	1.07259311668044\\
7380	1.0660872691596\\
7381	1.07078107942011\\
7382	1.06679335249707\\
7383	1.06452123697932\\
7384	1.04786165135552\\
7385	1.06658797357636\\
7386	1.00341314625985\\
7387	1.0026421812786\\
7388	1.04956516541799\\
7389	1.05686508478737\\
7390	1.11745431076727\\
7391	1.13510522118355\\
7392	1.1563814181477\\
7393	1.23686588083633\\
7394	1.30125513204759\\
7395	1.34461067502714\\
7396	1.35125798766251\\
7397	1.33675320597229\\
7398	1.27375141864752\\
7399	1.13185698462149\\
7400	1.03063665229302\\
7401	1.03219434529899\\
7402	1.0446457214945\\
7403	1.0363380780069\\
7404	1.05276097770104\\
7405	1.05262931785551\\
7406	1.05266150985609\\
7407	1.05650792706577\\
7408	1.03978210320453\\
7409	1.02624769567328\\
7410	1.01482465683812\\
7411	1.0067272512266\\
7412	1.06277730703351\\
7413	1.05322154382293\\
7414	1.10756838881264\\
7415	1.12326998897477\\
7416	1.18764519825793\\
7417	1.25499416019032\\
7418	1.32845004042068\\
7419	1.37219754384602\\
7420	1.38939284642073\\
7421	1.38016812432982\\
7422	1.31600878386599\\
7423	1.18298398585468\\
7424	1.11290848555478\\
7425	1.1290864348999\\
7426	1.11459022114474\\
7427	1.10998349554398\\
7428	1.13438497806373\\
7429	1.13860936048646\\
7430	1.14293849745297\\
7431	1.15361182401636\\
7432	1.15360880670594\\
7433	1.14485284177677\\
7434	1.14055670982877\\
7435	1.11678370723856\\
7436	1.18719995861357\\
7437	1.20079706339676\\
7438	1.2819052948161\\
7439	1.27996025372244\\
7440	1.32243662074768\\
7441	1.45468773235101\\
7442	1.55840560736773\\
7443	1.63965021501366\\
7444	1.68396425975453\\
7445	1.67013483573619\\
7446	1.57162012980327\\
7447	1.37031627882753\\
7448	1.25755576054567\\
7449	1.23340041545807\\
7450	1.21467096137474\\
7451	1.21004209239358\\
7452	1.18576675337776\\
7453	1.19617104516924\\
7454	1.19959426413406\\
7455	1.21320186404023\\
7456	1.22647058757037\\
7457	1.255323785144\\
7458	1.22175035755911\\
7459	1.1921440127787\\
7460	1.21401175338454\\
7461	1.28479088212639\\
7462	1.34618798124034\\
7463	1.32080637866028\\
7464	1.30379193819666\\
7465	1.41760394696993\\
7466	1.52013214428761\\
7467	1.59008099081273\\
7468	1.66109546208216\\
7469	1.66780400699777\\
7470	1.60568720163971\\
7471	1.5370070294487\\
7472	1.4542050002841\\
7473	1.40739158847877\\
7474	1.39867274552525\\
7475	1.46541317274872\\
7476	1.46906789918025\\
7477	1.45942177014339\\
7478	1.47245748393445\\
7479	1.46161621600085\\
7480	1.44145495324817\\
7481	1.38802538434543\\
7482	1.32371165523007\\
7483	1.2755838605189\\
7484	1.34220098321629\\
7485	1.37301347804693\\
7486	1.39585272475636\\
7487	1.39056425626508\\
7488	1.40838271088115\\
7489	1.5005084824343\\
7490	1.60329105284511\\
7491	1.66678190067631\\
7492	1.69084373164267\\
7493	1.70365540100961\\
7494	1.67086103029063\\
7495	1.63075102506058\\
7496	1.59818651000119\\
7497	1.5528058345542\\
7498	1.49900901367461\\
7499	1.4637446916856\\
7500	1.40418187820297\\
7501	1.37353661600883\\
7502	1.36442309511314\\
7503	1.35969619548758\\
7504	1.34232061173291\\
7505	1.31893192543179\\
7506	1.29189719677127\\
7507	1.25442204560014\\
7508	1.2506011025794\\
7509	1.29101540488475\\
7510	1.3005921306706\\
7511	1.30918506391781\\
7512	1.33059310100441\\
7513	1.38621958503115\\
7514	1.47453053090137\\
7515	1.54105706170118\\
7516	1.56873011121501\\
7517	1.56082179129312\\
7518	1.50051113853253\\
7519	1.37882258188227\\
7520	1.32877943561405\\
7521	1.30540188487521\\
7522	1.3507825008099\\
7523	1.34659618466961\\
7524	1.36147129350223\\
7525	1.41027021147284\\
7526	1.40785688806561\\
7527	1.3941390292872\\
7528	1.35124199652387\\
7529	1.32359642464898\\
7530	1.22575821203442\\
7531	1.21351111275873\\
7532	1.26750000070935\\
7533	1.28959730824382\\
7534	1.34894342306033\\
7535	1.28823683450481\\
7536	1.32565186872595\\
7537	1.43065062478183\\
7538	1.53100563361599\\
7539	1.59071439069412\\
7540	1.63120480990362\\
7541	1.63700291426659\\
7542	1.61677165471014\\
7543	1.5530037443191\\
7544	1.59344302939817\\
7545	1.59136974072039\\
7546	1.5514322076895\\
7547	1.49002136546822\\
7548	1.45831697509721\\
7549	1.43243421318708\\
7550	1.46731502014876\\
7551	1.50088123923361\\
7552	1.47538501373738\\
7553	1.42660523137031\\
7554	1.38009578583954\\
7555	1.29449690608938\\
7556	1.30535486532593\\
7557	1.30490170283861\\
7558	1.36223472137256\\
7559	1.33164068469841\\
7560	1.35226617490796\\
7561	1.48171378342658\\
7562	1.59604116073204\\
7563	1.63775385094809\\
7564	1.63928258672474\\
7565	1.5913155071352\\
7566	1.47741897259339\\
7567	1.26117491957637\\
7568	1.12605243635897\\
7569	1.14512019356463\\
7570	1.13321601039193\\
7571	1.12124950548218\\
7572	1.11945455340099\\
7573	1.12142491767927\\
7574	1.10528985966913\\
7575	1.10517053831621\\
7576	1.10064008727927\\
7577	1.11311709758911\\
7578	1.03517626510538\\
7579	1.02437154386706\\
7580	1.06721021608131\\
7581	1.08908533692766\\
7582	1.14829209920267\\
7583	1.14173213993756\\
7584	1.18128092379569\\
7585	1.24833652856367\\
7586	1.3227713441275\\
7587	1.38070400662217\\
7588	1.39429687274772\\
7589	1.36818779085545\\
7590	1.30234107521328\\
7591	1.15737135318966\\
7592	1.09444762715177\\
7593	1.10557806307501\\
7594	1.1343311299089\\
7595	1.20137045606924\\
7596	1.19690719082193\\
7597	1.20444956406719\\
7598	1.19142229168113\\
7599	1.16203267650688\\
7600	1.16725996853544\\
7601	1.1310107567699\\
7602	1.04243267184555\\
7603	1.02938940467584\\
7604	1.08643636064711\\
7605	1.11768444806023\\
7606	1.18523569812686\\
7607	1.18595917931612\\
7608	1.21370284379914\\
7609	1.33374749886165\\
7610	1.4315139677899\\
7611	1.49902694979398\\
7612	1.51958198785278\\
7613	1.51424205351877\\
7614	1.41704125215453\\
7615	1.23828056051572\\
7616	1.12544788750909\\
7617	1.11353787798353\\
7618	1.09916448001672\\
7619	1.11577132451857\\
7620	1.08918157928777\\
7621	1.1117759264456\\
7622	1.08666427789288\\
7623	1.07670363911564\\
7624	1.07087010772358\\
7625	1.05298155961967\\
7626	1.05420635291141\\
7627	1.05284713728117\\
7628	1.0884174890969\\
7629	1.08452233896528\\
7630	1.14784226385094\\
7631	1.13036801004804\\
7632	1.18626912477169\\
7633	1.29813167214559\\
7634	1.36389847424077\\
7635	1.43367441109569\\
7636	1.47102102320681\\
7637	1.47871381953933\\
7638	1.45147674895472\\
7639	1.39666012860607\\
7640	1.32379561509785\\
7641	1.25314364158755\\
7642	1.20890191324126\\
7643	1.19307557732079\\
7644	1.17653519144751\\
7645	1.16964383052246\\
7646	1.18915077896739\\
7647	1.20979676555704\\
7648	1.22307719150447\\
7649	1.23665244971736\\
7650	1.21800858994917\\
7651	1.20045188190836\\
7652	1.17263602801937\\
7653	1.23160112334579\\
7654	1.29757750556523\\
7655	1.28071144717034\\
7656	1.28596391642507\\
7657	1.3674399057841\\
7658	1.4471660537761\\
7659	1.5129291598983\\
7660	1.54083611171858\\
7661	1.5419362368935\\
7662	1.54640362331748\\
7663	1.5040239048673\\
7664	1.45635952438758\\
7665	1.40199160597668\\
7666	1.3146484173776\\
7667	1.28530317884864\\
7668	1.25960639049589\\
7669	1.23550416383464\\
7670	1.24710479107221\\
7671	1.27204293312177\\
7672	1.26209399620845\\
7673	1.25906630489445\\
7674	1.20965949252367\\
7675	1.19069378772299\\
7676	1.18303745784681\\
7677	1.21495795283603\\
7678	1.24328286706077\\
7679	1.25387719283333\\
7680	1.27960862802315\\
7681	1.37137252953886\\
7682	1.441763021884\\
7683	1.48267704716921\\
7684	1.51211480495453\\
7685	1.49900015511564\\
7686	1.39277094084816\\
7687	1.22243709246244\\
7688	1.11029568243563\\
7689	1.0954600556972\\
7690	1.07789236049355\\
7691	1.08787309385529\\
7692	1.08634566246869\\
7693	1.110442494182\\
7694	1.09696358061795\\
7695	1.06516828066073\\
7696	1.04907254227579\\
7697	1.05954974368782\\
7698	1.00206114046926\\
7699	1.00360406684952\\
7700	1.06509626403912\\
7701	1.09425079790958\\
7702	1.15545085390075\\
7703	1.15015803959366\\
7704	1.1878444937259\\
7705	1.26375385253127\\
7706	1.33729077297691\\
7707	1.38984977145949\\
7708	1.40971698041716\\
7709	1.38919868445432\\
7710	1.30557127640296\\
7711	1.15442902866486\\
7712	1.0474645799112\\
7713	1.05615235942739\\
7714	1.06961146311399\\
7715	1.07725685385346\\
7716	1.07791132606382\\
7717	1.07764716390248\\
7718	1.0629695195443\\
7719	1.06385667051485\\
7720	1.03914281220779\\
7721	1.05869252689062\\
7722	1.0134096381461\\
7723	1.01743073301375\\
7724	1.03941560268363\\
7725	1.0632431683985\\
7726	1.12100278119352\\
7727	1.11989935053898\\
7728	1.1558531840246\\
7729	1.2471011577619\\
7730	1.31548813231623\\
7731	1.37006401423018\\
7732	1.39081287217043\\
7733	1.36868836089931\\
7734	1.30324978876611\\
7735	1.13707436974597\\
7736	1.02108955588854\\
7737	1.0387189220905\\
7738	1.02916605783834\\
7739	1.03256298470261\\
7740	1.03304347665922\\
7741	1.03870192258634\\
7742	1.02186813126664\\
7743	1.03325064125023\\
7744	1.0255193184383\\
7745	1.04857566132759\\
7746	1.02235649025934\\
7747	1.03060424069912\\
7748	1.06941358661043\\
7749	1.05969587448474\\
7750	1.12289442700254\\
7751	1.14871445429994\\
7752	1.18524655583246\\
7753	1.2343487650033\\
7754	1.3112801719613\\
7755	1.34432140809764\\
7756	1.36450659319946\\
7757	1.34465926349191\\
7758	1.27828702119909\\
7759	1.14594857663227\\
7760	1.04456556069701\\
7761	1.06313520487951\\
7762	1.07815414373704\\
7763	1.106534619516\\
7764	1.1212196616789\\
7765	1.12176918987848\\
7766	1.10884375290027\\
7767	1.10272654721681\\
7768	1.09024526735803\\
7769	1.0733347749563\\
7770	1.05604240177076\\
7771	1.05186400691022\\
7772	1.08682569733765\\
7773	1.09341669493135\\
7774	1.16044805396111\\
7775	1.17388620419271\\
7776	1.18143798982648\\
7777	1.25425629476532\\
7778	1.32501784943678\\
7779	1.36248271953444\\
7780	1.3811693840802\\
7781	1.35806542901804\\
7782	1.29893922465766\\
7783	1.15662789386553\\
7784	1.08420580676942\\
7785	1.05800822096083\\
7786	1.05854394207158\\
7787	1.05780571107523\\
7788	1.07176408035707\\
7789	1.08948508090298\\
7790	1.06699622155154\\
7791	1.08423553796223\\
7792	1.08524005301516\\
7793	1.06686457722924\\
7794	1.05552911669761\\
7795	1.05668341155985\\
7796	1.09002042413566\\
7797	1.13894232232077\\
7798	1.19861386564586\\
7799	1.18471523917354\\
7800	1.19768544044825\\
7801	1.29030398014069\\
7802	1.39021410894584\\
7803	1.4657974669166\\
7804	1.51038017421218\\
7805	1.53284982794108\\
7806	1.51635710934188\\
7807	1.47141365231208\\
7808	1.35599667361449\\
7809	1.30395699074751\\
7810	1.27258388216706\\
7811	1.33993902431494\\
7812	1.33597696875151\\
7813	1.32971636577593\\
7814	1.33343083217445\\
7815	1.34060820620004\\
7816	1.32041755754789\\
7817	1.26450818623\\
7818	1.24206757483882\\
7819	1.21095424393698\\
7820	1.20310062248499\\
7821	1.25437087307427\\
7822	1.30711440476669\\
7823	1.29670118293397\\
7824	1.29680050800515\\
7825	1.40956601501911\\
7826	1.49681929467887\\
7827	1.56608517900745\\
7828	1.60512195382431\\
7829	1.60407656373789\\
7830	1.58346029187581\\
7831	1.53672164738418\\
7832	1.50539922504569\\
7833	1.47359159162758\\
7834	1.44420652075804\\
7835	1.44063522149435\\
7836	1.40983983765842\\
7837	1.40745871013928\\
7838	1.43038374460364\\
7839	1.44040801401633\\
7840	1.42624637334458\\
7841	1.36531196889223\\
7842	1.34679147133657\\
7843	1.32630840792285\\
7844	1.28846365225749\\
7845	1.31864377157413\\
7846	1.39662952485392\\
7847	1.39480758791067\\
7848	1.40574249250964\\
7849	1.50504938184915\\
7850	1.56060218418323\\
7851	1.61003418793898\\
7852	1.61476856388513\\
7853	1.58462426942826\\
7854	1.49402332380897\\
7855	1.3099495143442\\
7856	1.19011064056304\\
7857	1.14632233792337\\
7858	1.14839240698732\\
7859	1.16861664819457\\
7860	1.15784407785445\\
7861	1.18914574474172\\
7862	1.21357025934924\\
7863	1.17750319582641\\
7864	1.17058524222801\\
7865	1.14346267740398\\
7866	1.09829877870926\\
7867	1.09303676128223\\
7868	1.11461960309142\\
7869	1.16118714131962\\
7870	1.23033644651135\\
7871	1.21367116680877\\
7872	1.24320936674117\\
7873	1.32750043110479\\
7874	1.39457232656152\\
7875	1.44582091450807\\
7876	1.46930176246939\\
7877	1.45574745931507\\
7878	1.38232155258887\\
7879	1.21895884746888\\
7880	1.10739831192362\\
7881	1.09755337263761\\
7882	1.11719706589047\\
7883	1.13464965755995\\
7884	1.14166392335622\\
7885	1.16912571678647\\
7886	1.15657824413073\\
7887	1.15013098687249\\
7888	1.1517920726982\\
7889	1.12993864826319\\
7890	1.08045829975623\\
7891	1.09658775263752\\
7892	1.10840922387191\\
7893	1.14902777838844\\
7894	1.21403746877215\\
7895	1.20385202063875\\
7896	1.22847860668493\\
7897	1.32028226929814\\
7898	1.39147293037083\\
7899	1.44499642740551\\
7900	1.46004739095762\\
7901	1.4354175257291\\
7902	1.3624413475812\\
7903	1.20642129075911\\
7904	1.1020507549484\\
7905	1.06786273830811\\
7906	1.05717644901794\\
7907	1.03809641179063\\
7908	1.02847482904692\\
7909	1.02757667233323\\
7910	1.02130045487999\\
7911	1.02015126240255\\
7912	1.01570533231853\\
7913	1.0073432300734\\
7914	1.01078808685437\\
7915	1.00517968245639\\
7916	1.0427019530266\\
7917	1.10526384259548\\
7918	1.14053482944537\\
7919	1.14317894287816\\
7920	1.16598108400041\\
7921	1.24691121497714\\
7922	1.32303470537658\\
7923	1.37929965594756\\
7924	1.36398755702996\\
7925	1.35195674500407\\
7926	1.29321904322495\\
7927	1.14290138098221\\
7928	1.04500005803602\\
7929	1.01444295546442\\
7930	1.04502060509009\\
7931	1.05440046184668\\
7932	1.02568138399726\\
7933	1.05079178914671\\
7934	1.04291396497014\\
7935	1.04977889115115\\
7936	1.04688701067462\\
7937	1.03604551578657\\
7938	1.03363182528548\\
7939	1.03964580491907\\
7940	1.04083065344723\\
7941	1.08011665538616\\
7942	1.1499983308061\\
7943	1.14576645051718\\
7944	1.17590264989348\\
7945	1.29982922962705\\
7946	1.3820575296193\\
7947	1.44528941049646\\
7948	1.48122922953787\\
7949	1.47380647500263\\
7950	1.38932478127899\\
7951	1.22825976182295\\
7952	1.16774644573713\\
7953	1.15688066730892\\
7954	1.16278554204201\\
7955	1.15955009856453\\
7956	1.15108188155975\\
7957	1.16794213076964\\
7958	1.1823108314222\\
7959	1.1891521991248\\
7960	1.18914964600505\\
7961	1.15113734856266\\
7962	1.16542283487871\\
7963	1.17852756061514\\
7964	1.18236279705618\\
7965	1.21962503818641\\
7966	1.27804156714253\\
7967	1.24915687458076\\
7968	1.25201229524308\\
7969	1.32090740410289\\
7970	1.40121201120829\\
7971	1.47561980941002\\
7972	1.52032852497847\\
7973	1.52515710571453\\
7974	1.51073501095657\\
7975	1.43689222466292\\
7976	1.35645315347259\\
7977	1.30908310632896\\
7978	1.27219072563634\\
7979	1.28891018923673\\
7980	1.30752247386622\\
7981	1.31043773421744\\
7982	1.32303043093813\\
7983	1.31518829183638\\
7984	1.2731492095661\\
7985	1.25411524724276\\
7986	1.2078148175986\\
7987	1.20643498070998\\
7988	1.26074342060625\\
7989	1.28746698168377\\
7990	1.2980901643087\\
7991	1.26454780341631\\
7992	1.24284822751565\\
7993	1.2853144847359\\
7994	1.35203760491879\\
7995	1.40877184751804\\
7996	1.44454498621928\\
7997	1.46351451227182\\
7998	1.45636412299606\\
7999	1.41636292089711\\
8000	1.38569379734371\\
8001	1.32714338990184\\
};
\addplot [color=mycolor1,line width=1.3pt,solid,forget plot]
  table[row sep=crcr]{%
8001	1.32714338990184\\
8002	1.2674075212338\\
8003	1.14362774899525\\
8004	1.13319750850523\\
8005	1.12447080811752\\
8006	1.15237038635345\\
8007	1.16015641768922\\
8008	1.1388931776476\\
8009	1.09150916218999\\
8010	1.0860738346635\\
8011	1.08138613468958\\
8012	1.06862009194893\\
8013	1.0804327780723\\
8014	1.14203677837624\\
8015	1.14193768572\\
8016	1.1680851209585\\
8017	1.24605609852322\\
8018	1.2976089673785\\
8019	1.36880748681882\\
8020	1.40818123806881\\
8021	1.39562889135193\\
8022	1.32937673970234\\
8023	1.16603230454341\\
8024	1.05427790767217\\
8025	1.00864756110756\\
8026	0.978957808151817\\
8027	0.968801154952644\\
8028	0.948475176961589\\
8029	0.948871398038586\\
8030	0.957600391326347\\
8031	0.944034951921649\\
8032	0.944337176684138\\
8033	0.945376561641493\\
8034	0.936550968501996\\
8035	0.947457915088902\\
8036	0.983822115177679\\
8037	0.982765590698779\\
8038	1.04284271141051\\
8039	1.04815171113407\\
8040	1.06503009210528\\
8041	1.10235603353149\\
8042	1.15213588622339\\
8043	1.1879960280464\\
8044	1.22055266170344\\
8045	1.33679125272692\\
8046	1.28660543185161\\
8047	1.13609739429769\\
8048	1.04523929503981\\
8049	1.03381907691649\\
8050	1.01870914209306\\
8051	1.01674731509505\\
8052	1.0240966384098\\
8053	1.01920201584531\\
8054	1.05626220069971\\
8055	1.04158817276111\\
8056	1.03006444666983\\
8057	1.0211111188244\\
8058	1.01424793275292\\
8059	1.02290240701807\\
8060	1.04327057337923\\
8061	1.08226376946989\\
8062	1.15813989231641\\
8063	1.159215648338\\
8064	1.15563858637609\\
8065	1.20893363087363\\
8066	1.26700888111112\\
8067	1.32741012006269\\
8068	1.35028200343963\\
8069	1.34938406688676\\
8070	1.28913265899389\\
8071	1.15878622002352\\
8072	1.05468381697732\\
8073	1.03795838810899\\
8074	1.03728350250654\\
8075	1.04520038241315\\
8076	1.04058198601095\\
8077	1.03928826158422\\
8078	1.02430520104429\\
8079	1.01997280945339\\
8080	0.994768047331192\\
8081	0.988564731027834\\
8082	0.990786051662645\\
8083	1.0017059501101\\
8084	1.01892099257461\\
8085	1.03216424814607\\
8086	1.0742264008316\\
8087	1.07728897467048\\
8088	1.09973077996003\\
8089	1.12288956837821\\
8090	1.17015407913711\\
8091	1.20996676400709\\
8092	1.24468255847432\\
8093	1.24007561745903\\
8094	1.19871446217658\\
8095	1.07766813704215\\
8096	0.997548267556559\\
8097	0.975779094581519\\
8098	0.975910833539629\\
8099	0.97715941708481\\
8100	0.959052621125101\\
8101	0.959024284284988\\
8102	0.958549032833736\\
8103	0.966191306550613\\
8104	0.962570188330324\\
8105	0.971775966357862\\
8106	0.943372835236049\\
8107	0.948769958911037\\
8108	0.961989949917675\\
8109	0.983741306653651\\
8110	1.04530761677083\\
8111	1.0400872316946\\
8112	1.04121526815226\\
8113	1.13955654336282\\
8114	1.16091790884612\\
8115	1.21321208664552\\
8116	1.24549972200872\\
8117	1.24962218733111\\
8118	1.20064984198152\\
8119	1.07799264454433\\
8120	1.00771601136051\\
8121	1.01790651164035\\
8122	0.998234475490073\\
8123	0.995316411288128\\
8124	0.995942184187268\\
8125	1.0067564788663\\
8126	1.01272572851109\\
8127	1.02714353228813\\
8128	1.030239263904\\
8129	1.01672414834098\\
8130	1.03114569653947\\
8131	1.05482100366943\\
8132	1.0627005401402\\
8133	1.12401527916026\\
8134	1.12315344344446\\
8135	1.11127057848047\\
8136	1.11536411846945\\
8137	1.17074092777347\\
8138	1.2381944951011\\
8139	1.31172244699663\\
8140	1.36015639969534\\
8141	1.39210556517492\\
8142	1.38400518319083\\
8143	1.31899715201476\\
8144	1.25744622202481\\
8145	1.19911255721701\\
8146	1.14612280374359\\
8147	1.13665084539821\\
8148	1.14152381787197\\
8149	1.14854800760858\\
8150	1.15060856031094\\
8151	1.16029829013551\\
8152	1.13074894134994\\
8153	1.09733511545445\\
8154	1.06285499208605\\
8155	1.05832928549834\\
8156	1.0585657608774\\
8157	1.08875361714719\\
8158	1.13217957623135\\
8159	1.12745231394739\\
8160	1.09258788344239\\
8161	1.14767427824545\\
8162	1.22348169054036\\
8163	1.28882118088677\\
8164	1.34814187825097\\
8165	1.39012747324226\\
8166	1.402245041811\\
8167	1.37583035980577\\
8168	1.37047522725175\\
8169	1.324938721044\\
8170	1.27518138566532\\
8171	1.22230717944232\\
8172	1.18893603171069\\
8173	1.17215879003556\\
8174	1.23004960575322\\
8175	1.25283774461116\\
8176	1.25834268306939\\
8177	1.2244318467072\\
8178	1.1942764558703\\
8179	1.19395429171517\\
8180	1.18401143359973\\
8181	1.17894483335176\\
8182	1.19597314372777\\
8183	1.1811631465486\\
8184	1.23686267241558\\
8185	1.34079306583217\\
8186	1.38818655366673\\
8187	1.45827111097811\\
8188	1.50924258077561\\
8189	1.50074630337809\\
8190	1.41490704168261\\
8191	1.23576563543977\\
8192	1.12832285638168\\
8193	1.06716449781852\\
8194	1.07212568982805\\
8195	1.08800158883547\\
8196	1.06363692176055\\
8197	1.04652860330476\\
8198	1.03977130059883\\
8199	1.02181097305879\\
8200	1.01649517240806\\
8201	1.00437072473902\\
8202	1.00651946246229\\
8203	0.988621238831433\\
8204	1.00898284020649\\
8205	1.0609416932194\\
8206	1.13114759457025\\
8207	1.11688143383646\\
8208	1.10710772603147\\
8209	1.19335839216978\\
8210	1.27490091079931\\
8211	1.32069316372277\\
8212	1.35812153779607\\
8213	1.33822121644125\\
8214	1.25465353509658\\
8215	1.10787211993941\\
8216	1.02853696673467\\
8217	0.992984851679174\\
8218	1.02547171963309\\
8219	1.01765998403277\\
8220	1.04302997615157\\
8221	1.03627457211846\\
8222	1.0273910585751\\
8223	1.03501838777361\\
8224	1.03107472764692\\
8225	1.02608232563071\\
8226	1.01913157797651\\
8227	1.04076470993463\\
8228	1.05591225568127\\
8229	1.13147724864141\\
8230	1.20050583128273\\
8231	1.21092004779116\\
8232	1.22727827958888\\
8233	1.3240600780942\\
8234	1.39244069774258\\
8235	1.46893212946179\\
8236	1.51602808039719\\
8237	1.52175378833141\\
8238	1.43381744711612\\
8239	1.25634251202815\\
8240	1.14411278300328\\
8241	1.07893922014659\\
8242	1.08249595157842\\
8243	1.10415301431682\\
8244	1.1113530615556\\
8245	1.14208344108835\\
8246	1.08988252774405\\
8247	1.08617933022818\\
8248	1.04891010714129\\
8249	1.01556956518918\\
8250	0.999107687409188\\
8251	1.01842916826334\\
8252	1.06919734801792\\
8253	1.08387837940913\\
8254	1.15665081923592\\
8255	1.15330301936905\\
8256	1.17803541395921\\
8257	1.26429577434067\\
8258	1.34052850892139\\
8259	1.41256064764692\\
8260	1.44716990994975\\
8261	1.43721383384624\\
8262	1.37121170446572\\
8263	1.23522503551443\\
8264	1.1133259236224\\
8265	1.08142880353185\\
8266	1.08747662709196\\
8267	1.07803823342975\\
8268	1.06843666002356\\
8269	1.07649942296848\\
8270	1.08007528732661\\
8271	1.0792310016126\\
8272	1.07066918070485\\
8273	1.06065849314233\\
8274	1.05005123990128\\
8275	1.06528723050508\\
8276	1.09501865839647\\
8277	1.11625793537462\\
8278	1.18815815615281\\
8279	1.19418736696983\\
8280	1.19066111681011\\
8281	1.28812568652554\\
8282	1.3974591587015\\
8283	1.45265425694484\\
8284	1.49036324833349\\
8285	1.48575315317391\\
8286	1.38683067106767\\
8287	1.22948229372524\\
8288	1.16061057236918\\
8289	1.15459011140171\\
8290	1.14218205517834\\
8291	1.11944930565554\\
8292	1.09814471223264\\
8293	1.10554553795501\\
8294	1.09699381947782\\
8295	1.08958299008165\\
8296	1.08903542471775\\
8297	1.05818312542576\\
8298	1.04918147400521\\
8299	1.030114273205\\
8300	1.07187759113221\\
8301	1.12385533098121\\
8302	1.16768367581928\\
8303	1.13128033532842\\
8304	1.13331632012856\\
8305	1.16607084424252\\
8306	1.23996021528096\\
8307	1.29753690410023\\
8308	1.33288953889236\\
8309	1.34051851350107\\
8310	1.32543300519182\\
8311	1.27855664308297\\
8312	1.2358580158805\\
8313	1.15269755251246\\
8314	1.12869765102761\\
8315	1.10875086506352\\
8316	1.11254384505299\\
8317	1.10585300144546\\
8318	1.11250514115532\\
8319	1.12213870815384\\
8320	1.10136397921569\\
8321	1.08226518655766\\
8322	1.07657003148479\\
8323	1.05820614973157\\
8324	1.09812647082653\\
8325	1.11773055954434\\
8326	1.16464990679266\\
8327	1.14966554450328\\
8328	1.14209261532454\\
8329	1.18590287644316\\
8330	1.24126338634169\\
8331	1.30121598283468\\
8332	1.34914177341647\\
8333	1.37740338956257\\
8334	1.38204674766572\\
8335	1.37089946391536\\
8336	1.36054424513417\\
8337	1.30026539428208\\
8338	1.25745577688457\\
8339	1.23938454639579\\
8340	1.22055288027389\\
8341	1.22601534039678\\
8342	1.2849899515783\\
8343	1.33184408450181\\
8344	1.29442679312928\\
8345	1.28112211739862\\
8346	1.21566747353943\\
8347	1.22138876627376\\
8348	1.26603493800594\\
8349	1.27025876091986\\
8350	1.3148742411285\\
8351	1.31747493451106\\
8352	1.33278817947977\\
8353	1.42557127688401\\
8354	1.51020999642886\\
8355	1.60622540199943\\
8356	1.68378323256844\\
8357	1.71338842621877\\
8358	1.67589858178971\\
8359	1.54792308216008\\
8360	1.45266439934437\\
8361	1.308101939105\\
8362	1.25410605809905\\
8363	1.19043037019656\\
8364	1.15923746168449\\
8365	1.17205581921283\\
8366	1.17146147575814\\
8367	1.18108884682878\\
8368	1.16070300319883\\
8369	1.13339031460694\\
8370	1.09649213410852\\
8371	1.10769519506942\\
8372	1.15531958448714\\
8373	1.24305770510821\\
8374	1.28411074585031\\
8375	1.26335544616121\\
8376	1.23779270026121\\
8377	1.32010525514882\\
8378	1.3996910418652\\
8379	1.46404775056346\\
8380	1.5084051681402\\
8381	1.50479530229249\\
8382	1.39752987611604\\
8383	1.18472360957616\\
8384	1.07249215929381\\
8385	1.03712495993264\\
8386	1.02894388101007\\
8387	1.04458459161933\\
8388	1.04251022364219\\
8389	1.05815952982376\\
8390	1.05453340453167\\
8391	1.04754112505741\\
8392	1.03317961007428\\
8393	1.02257155804158\\
8394	0.986993648029434\\
8395	0.984962378485203\\
8396	1.02362841987835\\
8397	1.03253805226395\\
8398	1.09600406978466\\
8399	1.0956979111307\\
8400	1.12439629125105\\
8401	1.23018345709125\\
8402	1.31264756102896\\
8403	1.40208533058254\\
8404	1.46212724082334\\
8405	1.49165601845845\\
8406	1.43656257889744\\
8407	1.27845727830626\\
8408	1.15322625014044\\
8409	1.0922551306894\\
8410	1.05534376284256\\
8411	1.04096593692628\\
8412	1.0232686110976\\
8413	1.02892795159319\\
8414	1.03048711374206\\
8415	1.02580990730217\\
8416	1.01886154516542\\
8417	0.992714311123493\\
8418	1.00162689959648\\
8419	1.00449952899897\\
8420	1.00270074324297\\
8421	1.06673401605805\\
8422	1.13652494192413\\
8423	1.177302564008\\
8424	1.17141882258827\\
8425	1.2511291782941\\
8426	1.34926432302792\\
8427	1.39853156336043\\
8428	1.43986736229519\\
8429	1.40144495491194\\
8430	1.3347464326993\\
8431	1.18287644649255\\
8432	1.11894665419411\\
8433	1.06826228576687\\
8434	1.16830444486746\\
8435	1.10876621003511\\
8436	1.09314059582075\\
8437	1.14521394076938\\
8438	1.11562659754462\\
8439	1.12927978444503\\
8440	1.13542358446419\\
8441	1.13299949651878\\
8442	1.137949720879\\
8443	1.10969009750241\\
8444	1.15358102137531\\
8445	1.2224167726239\\
8446	1.28541871278212\\
8447	1.27720183637854\\
8448	1.29077603760857\\
8449	1.39967049287388\\
8450	1.50875732661334\\
8451	1.60602079502665\\
8452	1.64724221348206\\
8453	1.64845068132474\\
8454	1.54502851175127\\
8455	1.35538279760793\\
8456	1.22728904570039\\
8457	1.1472869656789\\
8458	1.13966371185971\\
8459	1.1307546013623\\
8460	1.09404945033474\\
8461	1.07699413201875\\
8462	1.1154962028211\\
8463	1.14154793026088\\
8464	1.15660642498816\\
8465	1.15679095273787\\
8466	1.13392032562916\\
8467	1.1346093279584\\
8468	1.18951449088106\\
8469	1.24671693819689\\
8470	1.31108049510035\\
8471	1.27574426105955\\
8472	1.28452005859231\\
8473	1.45371625270884\\
8474	1.55444945752713\\
8475	1.64057248550187\\
8476	1.69454801532458\\
8477	1.72785420148592\\
8478	1.72842357489229\\
8479	1.66605015908381\\
8480	1.62697308854512\\
8481	1.50538761492827\\
8482	1.51401945432959\\
8483	1.49103287206764\\
8484	1.49173369002875\\
8485	1.49748666246559\\
8486	1.48120856615619\\
8487	1.4910369563797\\
8488	1.48429980518813\\
8489	1.44260824512219\\
8490	1.4244266050626\\
8491	1.37846846415113\\
8492	1.36404095639259\\
8493	1.40844480370983\\
8494	1.43482703637698\\
8495	1.40629640592344\\
8496	1.37247022689176\\
8497	1.45323711664685\\
8498	1.53183890857515\\
8499	1.60440030521016\\
8500	1.65373226866551\\
8501	1.66408306627363\\
8502	1.67121211674714\\
8503	1.6440584366952\\
8504	1.63037160358514\\
8505	1.56272482737773\\
8506	1.50352009382737\\
8507	1.46497699804384\\
8508	1.43831164789581\\
8509	1.42264660417006\\
8510	1.41628290800412\\
8511	1.46353933686541\\
8512	1.48018948358867\\
8513	1.46902353603382\\
8514	1.4937460865193\\
8515	1.48754359185873\\
8516	1.48282342416843\\
8517	1.47783415601633\\
8518	1.52278540880518\\
8519	1.49688531073387\\
8520	1.50751631022361\\
8521	1.64911409088117\\
8522	1.7380399270882\\
8523	1.84278219277516\\
8524	1.91102603910148\\
8525	1.88982056334449\\
8526	1.78308413693715\\
8527	1.58976825208901\\
8528	1.46459399453981\\
8529	1.33729922148427\\
8530	1.31338353420483\\
8531	1.29344127781202\\
8532	1.27974128008175\\
8533	1.33744107224697\\
8534	1.34986664334976\\
8535	1.38561570601464\\
8536	1.40619150985377\\
8537	1.40140447905447\\
8538	1.40487682906923\\
8539	1.37813571807168\\
8540	1.45959588515223\\
8541	1.5062553992216\\
8542	1.52724195454186\\
8543	1.49829696700234\\
8544	1.50436584696816\\
8545	1.64174260562922\\
8546	1.77467793906556\\
8547	1.87990479018297\\
8548	1.92268195856248\\
8549	1.90743694064729\\
8550	1.81309794258038\\
8551	1.57370382939131\\
8552	1.46080151982663\\
8553	1.35210966754684\\
8554	1.3371826231411\\
8555	1.33529123118354\\
8556	1.33106044673427\\
8557	1.30626440778397\\
8558	1.32526240118229\\
8559	1.34490157267694\\
8560	1.33478393415608\\
8561	1.34499941511857\\
8562	1.31686967045448\\
8563	1.34250411074167\\
8564	1.34247915014425\\
8565	1.43732234032891\\
8566	1.50367726988001\\
8567	1.46049385895492\\
8568	1.48084589004699\\
8569	1.62525815585464\\
8570	1.78623268844535\\
8571	1.90465573570297\\
8572	1.97750488782112\\
8573	1.98915052955196\\
8574	1.91493708276279\\
8575	1.76732578903348\\
8576	1.67432087898782\\
8577	1.54154792305925\\
8578	1.49003088867765\\
8579	1.44273791779057\\
8580	1.42963955181686\\
8581	1.426840692727\\
8582	1.36649313281802\\
8583	1.37435226912007\\
8584	1.37034179505831\\
8585	1.38464622224157\\
8586	1.4187142441262\\
8587	1.41363638995712\\
8588	1.44774574600538\\
8589	1.51733232224308\\
8590	1.54289111628408\\
8591	1.47906401168228\\
8592	1.3373111019866\\
8593	1.4181156898758\\
8594	1.51661824133831\\
8595	1.6331008785419\\
8596	1.72602632666105\\
8597	1.76685662112437\\
8598	1.76687434399964\\
8599	1.73689244027631\\
8600	1.76018739273318\\
8601	1.69901416154988\\
8602	1.65404534540226\\
8603	1.59499036599155\\
8604	1.55495011923529\\
8605	1.52346443941059\\
8606	1.52208670164402\\
8607	1.52343872800384\\
8608	1.51176478191814\\
8609	1.49477712082175\\
8610	1.47038786551795\\
8611	1.44654468840921\\
8612	1.40319965167553\\
8613	1.39319031460402\\
8614	1.36545934586638\\
8615	1.28848127907015\\
8616	1.27387531757695\\
8617	1.37441531619016\\
8618	1.44671595986635\\
8619	1.51160569489657\\
8620	1.54184993262004\\
8621	1.52687481016744\\
8622	1.46762806637376\\
8623	1.37037682837617\\
8624	1.32753517981743\\
8625	1.2539304156495\\
8626	1.20860842275527\\
8627	1.18510279314986\\
8628	1.18860509464683\\
8629	1.19615268691176\\
8630	1.19993024969369\\
8631	1.18405164532763\\
8632	1.18209855574803\\
8633	1.17345005485258\\
8634	1.19418586716758\\
8635	1.20054728178339\\
8636	1.25373724715499\\
8637	1.29167717473871\\
8638	1.37683124961439\\
8639	1.33296865343761\\
8640	1.33756531417263\\
8641	1.42181667740224\\
8642	1.52798157421027\\
8643	1.63215013860872\\
8644	1.70711197062328\\
8645	1.71713852234749\\
8646	1.69099616008878\\
8647	1.62720733793361\\
8648	1.56788242793241\\
8649	1.46159645706353\\
8650	1.3887735711199\\
8651	1.32469768234208\\
8652	1.2813917876019\\
8653	1.27992309397903\\
8654	1.27157889541129\\
8655	1.28813660915137\\
8656	1.29025021914606\\
8657	1.29645075301038\\
8658	1.28356367416061\\
8659	1.28464153738118\\
8660	1.28931907440499\\
8661	1.31981032823594\\
8662	1.3663827499791\\
8663	1.34595913322719\\
8664	1.32182417817365\\
8665	1.41534059051828\\
8666	1.50373159945767\\
8667	1.57557945415222\\
8668	1.64875183624293\\
8669	1.68894818416681\\
8670	1.68789467932532\\
8671	1.63702247323077\\
8672	1.62543211153994\\
8673	1.53699596934418\\
8674	1.45226720361628\\
8675	1.42473165753451\\
8676	1.40272487441901\\
8677	1.38947610209476\\
8678	1.39421554951891\\
8679	1.37161245835094\\
8680	1.30944101650075\\
8681	1.247635651852\\
8682	1.18096573031712\\
8683	1.16639149896853\\
8684	1.17607403372336\\
8685	1.16380512695592\\
8686	1.18021427257023\\
8687	1.18503415715792\\
8688	1.14404970588347\\
8689	1.21431908874673\\
8690	1.29065111632322\\
8691	1.37458449659015\\
8692	1.43538413224968\\
8693	1.46415401467346\\
8694	1.4461163806062\\
8695	1.35749888266681\\
8696	1.30812927646506\\
8697	1.21321622168228\\
8698	1.21657193783287\\
8699	1.21091471709704\\
8700	1.20401693677836\\
8701	1.21976546691653\\
8702	1.21608718284478\\
8703	1.22338064446681\\
8704	1.20445092203241\\
8705	1.18508735322561\\
8706	1.12926075635526\\
8707	1.10922701088948\\
8708	1.12026200956422\\
8709	1.14839566014159\\
8710	1.1845788772179\\
8711	1.14680187464326\\
8712	1.15168090420463\\
8713	1.23439092099838\\
8714	1.29764072001228\\
8715	1.37056622249755\\
8716	1.41740022872829\\
8717	1.42291367809524\\
8718	1.38408785617229\\
8719	1.29978002028562\\
8720	1.2508312619189\\
8721	1.18371747958621\\
8722	1.15673300303895\\
8723	1.15366871181385\\
8724	1.15643342164923\\
8725	1.14618583267885\\
8726	1.14457399673234\\
8727	1.15368258138542\\
8728	1.14743425371603\\
8729	1.12586706954808\\
8730	1.09176196233481\\
8731	1.10101955832463\\
8732	1.13645231798567\\
8733	1.15185910820279\\
8734	1.20781463026626\\
8735	1.16425463265425\\
8736	1.16208789058946\\
8737	1.2516302016378\\
8738	1.32688103504547\\
8739	1.39402901621521\\
8740	1.42823601179647\\
8741	1.44294004489756\\
8742	1.41427969104101\\
8743	1.33319658219332\\
8744	1.28331933389827\\
8745	1.20406101031944\\
8746	1.17747007448541\\
8747	1.20503401423872\\
8748	1.20825319533468\\
8749	1.22950298882453\\
8750	1.20746979830123\\
8751	1.22318878187661\\
8752	1.21691420791927\\
8753	1.19401951496614\\
8754	1.19550291861363\\
8755	1.21334587223175\\
8756	1.28311931989187\\
8757	1.2904425736041\\
8758	1.33421323961541\\
8759	1.28308881880662\\
8760	1.24069928656258\\
};
\end{axis}
\end{tikzpicture}%
    \caption{Ratio between the maximal Belgian production and the real demand level (the demand level minus the power coming from renewable energies)}
    \label{fig:ratio}
\end{figure}

\newpage
\subsection{Question 1}
% QUESTION 1

The profit margin of each generator in the system are shown on figures [\ref{fig:biomass}],[\ref{fig:nuclear}], [\ref{fig:gas}], and [\ref{fig:oil}]. The dotted lines represent the investment cost for each type of generator. Apparently, generators tend to have a profit margin below their investment cost. Although one or two generators per technology play their cards right, shareholders are far from a return on their investments. The question is therefore, whether or not it is wise to continue to invest, based on those results. Hard to say. Take for example the nuclear generators. Even if it's not doing so bad, continue to invest in may reduce profitability. Indeed, nuclear generators are usually used to satisfy the most stable part of the demand. However, if it has already been fully sated, investing in nuclear will not increase profitability as new generators will not be used very often.\\

The second model leads to more profitable generators, even if the improvement isn't incredible. The only reason is that the demand level is higher than in the first model since the amount of energy imported from neighboring markets are not deduce anymore from the demand level.\\

Notice that almost all generators tends to be more profitable in the ORDC model. It comes from the fact that the ORDC model tends to ask for far more reserve than the fixed reserve requirement of the first two models. And since reserves have physical no cost, the generators generate huge profits without producing anything. \\

Finally, figure [\ref{fig:margin}] shows the average profit margin for each kind of generators and for each model. The conclusion is the same for all of them : generators are not money-making.

\begin{minipage}{0.495\textwidth} 
\begin{figure}[H]
    \centering
    \setlength\fheight{4cm}
    \setlength\fwidth{0.8\textwidth}
    % This file was created by matlab2tikz.
% Minimal pgfplots version: 1.3
%
%The latest updates can be retrieved from
%  http://www.mathworks.com/matlabcentral/fileexchange/22022-matlab2tikz
%where you can also make suggestions and rate matlab2tikz.
%
\definecolor{mycolor1}{rgb}{0.84706,0.16078,0.00000}%
\definecolor{mycolor2}{rgb}{0.04314,0.51765,0.78039}%
\definecolor{mycolor3}{rgb}{0.87059,0.49020,0.00000}%
%
\begin{tikzpicture}

\begin{axis}[%
width=\fwidth,
height=\fheight,
at={(\fwidth,0\fheight)},
scale only axis,
area legend,
separate axis lines,
every outer x axis line/.append style={black},
every x tick label/.append style={font=\color{black}},
xmin=0,
xmax=10,
xtick={1, 2, 3, 4, 5, 6, 7, 8, 9},
xlabel={Biomass Generator},
every outer y axis line/.append style={black},
every y tick label/.append style={font=\color{black}},
ymin=0,
ymax=1300,
title={Profit margin [\euro/ MW-day]},
ymajorgrids,
legend style={at={(0.5,0.97)},anchor=north,legend columns=4,legend cell align=left,align=left,draw=white,fill=white}
]
\addplot[ybar,bar width=0.02\fwidth,bar shift=-0.02\fwidth,draw=black,fill=mycolor1] plot table[row sep=crcr] {%
1	890.144524\\
2	469.454365\\
3	402.610557\\
4	331.182026\\
5	161.417783\\
6	113.799985\\
7	98.931328\\
8	1.094311\\
9	0.547255\\
};
\addlegendentry{ EDR};

\addplot [color=black,solid,forget plot]
  table[row sep=crcr]{%
0	0\\
10	0\\
};
\addplot[ybar,bar width=0.02\fwidth,draw=black,fill=mycolor2] plot table[row sep=crcr] {%
1	1000.319513\\
2	579.629353\\
3	509.511515\\
4	441.318705\\
5	260.517653\\
6	199.421224\\
7	178.554357\\
8	15.075046\\
9	9.041929\\
};
\addlegendentry{ ImpExp};

\addplot[ybar,bar width=0.02\fwidth,bar shift=0.02\fwidth,draw=black,fill=mycolor3] plot table[row sep=crcr] {%
1	1066.69968\\
2	646.009491\\
3	573.442362\\
4	507.712942\\
5	332.210293\\
6	269.651539\\
7	248.87988\\
8	61.88185\\
9	36.98737\\
};
\addlegendentry{ ORDC};

\addplot [color=black,dotted,line width=2.0pt]
  table[row sep=crcr]{%
0	669.6\\
10	669.6\\
};

\end{axis}
\end{tikzpicture}%
    \caption{Profit margin of biomass generators for each model}
    \label{fig:biomass}
\end{figure}
\end{minipage}
\begin{minipage}{0.495\textwidth} 
\begin{figure}[H]
    \centering
    \setlength\fheight{4cm}
    \setlength\fwidth{0.8\textwidth}
    % This file was created by matlab2tikz.
% Minimal pgfplots version: 1.3
%
%The latest updates can be retrieved from
%  http://www.mathworks.com/matlabcentral/fileexchange/22022-matlab2tikz
%where you can also make suggestions and rate matlab2tikz.
%
\definecolor{mycolor1}{rgb}{0.84706,0.16078,0.00000}%
\definecolor{mycolor2}{rgb}{0.04314,0.51765,0.78039}%
\definecolor{mycolor3}{rgb}{0.87059,0.49020,0.00000}%
%
\begin{tikzpicture}

\begin{axis}[%
width=\fwidth,
height=\fheight,
at={(\fwidth,\fheight)},
scale only axis,
area legend,
separate axis lines,
every outer x axis line/.append style={black},
every x tick label/.append style={font=\color{black}},
xmin=0,
xmax=8,
xlabel={Nuclear generator},
xtick={1, 2, 3, 4, 5, 6, 7},
every outer y axis line/.append style={black},
every y tick label/.append style={font=\color{black}},
ymin=0,
ymax=1000,
title={Profit margin [\euro / MW-day]},
ymajorgrids,
legend style={at={(0.5,0.97)},anchor=north,legend columns=4,legend cell align=left,align=left,draw=white,fill=white}
]
\addplot[ybar,bar width=0.02\fwidth,bar shift=-0.02\fwidth,draw=black,fill=mycolor1] plot table[row sep=crcr] {%
1	391.841753\\
2	625.488172\\
3	199.950964\\
4	662.371119\\
5	684.183826\\
6	194.318354\\
7	693.458351\\
};
\addlegendentry{ EDR};

\addplot [color=black,solid,forget plot]
  table[row sep=crcr]{%
0	0\\
8	0\\
};
\addplot[ybar,bar width=0.02\fwidth,draw=black,fill=mycolor2] plot table[row sep=crcr] {%
1	393.435515\\
2	635.629893\\
3	181.945509\\
4	681.775726\\
5	696.216683\\
6	176.455306\\
7	708.18522\\
};
\addlegendentry{ ImpExp};

\addplot[ybar,bar width=0.02\fwidth,bar shift=0.02\fwidth,draw=black,fill=mycolor3] plot table[row sep=crcr] {%
1	408.735275\\
2	666.686268\\
3	183.37229\\
4	736.209532\\
5	749.013116\\
6	177.860866\\
7	762.168769\\
};
\addlegendentry{ ORDC};

\addplot [color=black,dotted,line width=2.0pt]
  table[row sep=crcr]{%
0	762.2\\
8	762.2\\
};

\end{axis}
\end{tikzpicture}%
    \caption{Profit margin of nuclear generators for each model}
    \label{fig:nuclear}
\end{figure}
\end{minipage}\\
\begin{minipage}{\textwidth} 
\begin{figure}[H]
    \centering
    \setlength\fheight{4cm}
    \setlength\fwidth{0.9\textwidth}
    % This file was created by matlab2tikz.
% Minimal pgfplots version: 1.3
%
%The latest updates can be retrieved from
%  http://www.mathworks.com/matlabcentral/fileexchange/22022-matlab2tikz
%where you can also make suggestions and rate matlab2tikz.
%
\definecolor{mycolor1}{rgb}{0.84706,0.16078,0.00000}%
\definecolor{mycolor2}{rgb}{0.04314,0.51765,0.78039}%
\definecolor{mycolor3}{rgb}{0.87059,0.49020,0.00000}%
%
\begin{tikzpicture}

\begin{axis}[%
width=\fwidth,
height=\fheight,
at={(\fwidth,\fheight)},
scale only axis,
area legend,
separate axis lines,
every outer x axis line/.append style={black},
every x tick label/.append style={font=\color{black}},
xmin=0,
xmax=44,
xlabel={Gas generator},
every outer y axis line/.append style={black},
every y tick label/.append style={font=\color{black}},
ymin=0,
ymax=520,
title={Profit margin [\euro / MW-day]},
ymajorgrids,
legend style={at={(0.5,0.97)},anchor=north,legend columns=4,legend cell align=left,align=left,draw=white,fill=white}
]
\addplot[ybar,bar width=0.007\fwidth,bar shift=-0.007\fwidth,draw=black,fill=mycolor1] plot table[row sep=crcr] {%
1	446.137971\\
2	119.121063\\
3	121.601622\\
4	118.063942\\
5	94.394361\\
6	54.435866\\
7	63.681298\\
8	70.719834\\
9	68.925917\\
10	33.929164\\
11	20.890385\\
12	20.890385\\
13	14.703568\\
14	16.87723\\
15	16.717246\\
16	16.717246\\
17	7.133388\\
18	9.343958\\
19	5.982131\\
20	6.132066\\
21	4.521989\\
22	5.86518\\
23	5.026359\\
24	4.855727\\
25	3.288458\\
26	5.476274\\
27	3.697205\\
28	3.620683\\
29	3.503884\\
30	2.42636\\
31	2.044421\\
32	3.044049\\
33	5.078886\\
34	4.766866\\
35	4.766866\\
36	4.766866\\
37	4.74255\\
38	1.162388\\
39	1.012686\\
40	1.923808\\
41	4.676752\\
42	0.942268\\
43	4.664245\\
};
\addlegendentry{ EDR};

\addplot [color=black,solid,forget plot]
  table[row sep=crcr]{%
0	0\\
44	0\\
};
\addplot[ybar,bar width=0.007\fwidth,draw=black,fill=mycolor2] plot table[row sep=crcr] {%
1	458.38911\\
2	134.362042\\
3	140.619298\\
4	135.147026\\
5	111.628856\\
6	70.629623\\
7	81.120977\\
8	90.368348\\
9	88.540415\\
10	48.420255\\
11	35.482169\\
12	35.482169\\
13	26.236998\\
14	29.870809\\
15	29.632263\\
16	29.632266\\
17	12.442325\\
18	17.711461\\
19	11.147036\\
20	11.544586\\
21	11.024107\\
22	9.992967\\
23	8.975957\\
24	8.29519\\
25	4.490975\\
26	10.280113\\
27	5.049382\\
28	4.873917\\
29	4.622847\\
30	2.165265\\
31	2.053126\\
32	3.93939\\
33	9.462649\\
34	9.406491\\
35	9.406491\\
36	9.406492\\
37	9.40497\\
38	1.880904\\
39	1.880904\\
40	3.761807\\
41	9.404519\\
42	1.880903\\
43	9.404519\\
};
\addlegendentry{ ImpExp};

\addplot[ybar,bar width=0.007\fwidth,bar shift=0.007\fwidth,draw=black,fill=mycolor3] plot table[row sep=crcr] {%
1	509.031622\\
2	180.263545\\
3	196.525709\\
4	199.771903\\
5	173.423449\\
6	130.915863\\
7	157.82485\\
8	148.145281\\
9	146.553509\\
10	126.084697\\
11	106.28886\\
12	123.884776\\
13	102.467043\\
14	101.699373\\
15	101.500337\\
16	101.500409\\
17	68.059127\\
18	90.974734\\
19	91.417278\\
20	92.323003\\
21	84.675595\\
22	82.150777\\
23	77.675397\\
24	77.065699\\
25	50.51984\\
26	120.344399\\
27	68.36467\\
28	67.527385\\
29	66.255322\\
30	43.68047\\
31	39.731264\\
32	61.232983\\
33	116.482618\\
34	113.584585\\
35	113.584585\\
36	113.584605\\
37	113.39449\\
38	31.550451\\
39	30.131534\\
40	50.733724\\
41	112.7646\\
42	29.230676\\
43	112.537817\\
};
\addlegendentry{ ORDC};

\addplot [color=black,dotted,line width=2.0pt]
  table[row sep=crcr]{%
0	122.4\\
44	122.4\\
};

\end{axis}
\end{tikzpicture}%
        \caption{Profit margin of gas generators for each model}
    \label{fig:gas}
\end{figure}
\end{minipage} \\
\begin{minipage}{0.495\textwidth} 
\begin{figure}[H]
    \centering
    \setlength\fheight{4cm}
    \setlength\fwidth{0.8\textwidth}
    % This file was created by matlab2tikz.
% Minimal pgfplots version: 1.3
%
%The latest updates can be retrieved from
%  http://www.mathworks.com/matlabcentral/fileexchange/22022-matlab2tikz
%where you can also make suggestions and rate matlab2tikz.
%
\definecolor{mycolor1}{rgb}{0.84706,0.16078,0.00000}%
\definecolor{mycolor2}{rgb}{0.04314,0.51765,0.78039}%
\definecolor{mycolor3}{rgb}{0.87059,0.49020,0.00000}%
%
\begin{tikzpicture}

\begin{axis}[%
width=\fwidth,
height=\fheight,
at={(0\fwidth,0\fheight)},
scale only axis,
area legend,
separate axis lines,
every outer x axis line/.append style={black},
every x tick label/.append style={font=\color{black}},
xmin=0,
xmax=4,
xlabel={Oil generator},
every outer y axis line/.append style={black},
every y tick label/.append style={font=\color{black}},
ymin=0,
ymax=60,
title={Profit margin [\euro / MW-day]},
ymajorgrids,
legend style={at={(0.5,0.97)},anchor=north,legend columns=4,legend cell align=left,align=left,draw=white,fill=white}
]
\addplot[ybar,bar width=0.029\fwidth,bar shift=-0.037\fwidth,draw=black,fill=mycolor1] plot table[row sep=crcr] {%
1	0.932849\\
2	0.932849\\
3	0.932849\\
};
\addlegendentry{ EDR};

\addplot [color=black,solid,forget plot]
  table[row sep=crcr]{%
0	0\\
4	0\\
};
\addplot[ybar,bar width=0.029\fwidth,draw=black,fill=mycolor2] plot table[row sep=crcr] {%
1	1.880901\\
2	1.880902\\
3	1.880904\\
};
\addlegendentry{ ImpExp};

\addplot[ybar,bar width=0.029\fwidth,bar shift=0.037\fwidth,draw=black,fill=mycolor3] plot table[row sep=crcr] {%
1	24.773702\\
2	24.820957\\
3	24.225142\\
};
\addlegendentry{ ORDC};

\addplot [color=black,dotted,line width=2.0pt]
  table[row sep=crcr]{%
0	40.8\\
4	40.8\\
};

\end{axis}
\end{tikzpicture}%
        \caption{Profit margin of oil generators for each model}
    \label{fig:oil}
\end{figure}
\end{minipage}
\begin{minipage}{0.495\textwidth} 
\begin{figure}[H]
    \centering
    \setlength\fheight{4cm}
    \setlength\fwidth{0.8\textwidth}
    % This file was created by matlab2tikz.
% Minimal pgfplots version: 1.3
%
%The latest updates can be retrieved from
%  http://www.mathworks.com/matlabcentral/fileexchange/22022-matlab2tikz
%where you can also make suggestions and rate matlab2tikz.
%
\definecolor{mycolor1}{rgb}{0.84706,0.16078,0.00000}%
\definecolor{mycolor2}{rgb}{0.04314,0.51765,0.78039}%
\definecolor{mycolor3}{rgb}{0.87059,0.49020,0.00000}%
%
\begin{tikzpicture}

\begin{axis}[%
width=\fwidth,
height=\fheight,
at={(\fwidth,\fheight)},
scale only axis,
area legend,
separate axis lines,
every outer x axis line/.append style={black},
every x tick label/.append style={font=\color{black}},
xmin=0,
xmax=5,
xtick={0,1,2,3,4,5},
xticklabels={{},{Bio.},{Nucl.},{Gas},{Oil},{}},
xlabel={Generator type},
every outer y axis line/.append style={black},
every y tick label/.append style={font=\color{black}},
ymin=0,
ymax=650,
title={Average profit margin [\euro / MW-day]},
ymajorgrids,
legend style={at={(0.5,0.97)},anchor=north,legend columns=3,legend cell align=left,align=left,draw=white,fill=white}
]
\addplot[ybar,bar width=0.029662\fwidth,bar shift=-0.037077\fwidth,draw=black,fill=mycolor1] plot table[row sep=crcr] {%
1	278.961397333333\\
2	493.087505571429\\
3	32.8435676976744\\
4	0.932849\\
};
\addlegendentry{ EDR};

\addplot [color=black,solid,forget plot]
  table[row sep=crcr]{%
0	0\\
5	0\\
};
\addplot[ybar,bar width=0.029662\fwidth,draw=black,fill=mycolor2] plot table[row sep=crcr] {%
1	286.350823666667\\
2	496.234836\\
3	40.6853934186047\\
4	1.88090233333333\\
};
\addlegendentry{ ImpExp};

\addplot[ybar,bar width=0.029662\fwidth,bar shift=0.037077\fwidth,draw=black,fill=mycolor3] plot table[row sep=crcr] {%
1	337.644301222222\\
2	526.292302285714\\
3	109.894391953488\\
4	24.6066003333333\\
};
\addlegendentry{ ORDC};

\end{axis}
\end{tikzpicture}%
        \caption{Average profit margin of biomass generators for each model}
    \label{fig:margin}
\end{figure}
\end{minipage}

\subsection{Question 2}
% QUESTION 2

Figure [\ref{fig:EDR_IMPEXP}] shows the predicted and the realized market prices and the table \ref{tab1} computes some statistics about them. Notice the higher volatility of the market price versus the predicted price. The variance is significantly larger and the gap between the lowest and the highest prices is bigger. How to explain those differences. \\

Our models are quite simple and doesn't take into account many factors. Here is a non-exhaustive list :

\begin{itemize}
\item[•] The transportation of power over grids is dictated by physical laws rather than decisions.
\item[•] The economic dispatch problem is not the only problem to take into account. The scheduling of electric energy production is not independent across time periods. Generators are subject to various operating constraints that create dependencies in operations from one hour to the next. The unit commitment problem schedule the on-off status of units, while accounting for time-dependent operating constraints and costs. This explain why the lowest realized price is negative. The system operator preferred selling at loss rather than shut down a generator to restart it later, which would have been costly.
\item[•] The financial instruments for trading a commodity in a price fixed in advance as callable forward contract or financial transmission rights, are unknown. Since the Belgian market has been strained during winter 2014, it's possible that they take a big part in the Belgian market in order to hedge risk.
\end{itemize}

\begin{table}[H]
\centering
\begin{tabular}{l | c  c  c  c}
model & mean & variance & min & max \\
\hline
EDR & $37.34$ &  $37.09$ & $24.74$ &  $74.63$ \\
ImpExp & $37.99$ &  $18.95$ & $28.51$ &  $53.4$ \\
\hline
Market & $40.79$ &  $160.79$ & $ -0.01$ &  $200$ \\
\end{tabular}
\caption{Statistics on the realized and the predicted energy prices}
\label{tab1}
\end{table}

\begin{figure}[H]
    \centering
    \setlength\fheight{0.6\textwidth}
    \setlength\fwidth{0.85\textwidth}
    % This file was created by matlab2tikz.
% Minimal pgfplots version: 1.3
%
%The latest updates can be retrieved from
%  http://www.mathworks.com/matlabcentral/fileexchange/22022-matlab2tikz
%where you can also make suggestions and rate matlab2tikz.
%
\definecolor{mycolor1}{rgb}{0,0.501960784313725,0.501960784313725}%
\definecolor{mycolor2}{rgb}{0.84706,0.16078,0.00000}%
\definecolor{mycolor3}{rgb}{0.04314,0.51765,0.78039}%
%
\begin{tikzpicture}

\begin{axis}[%
width=\fwidth,
height=\fheight,
at={(0\fwidth,0\fheight)},
scale only axis,
separate axis lines,
every outer x axis line/.append style={black},
every x tick label/.append style={font=\color{black}},
xmin=0,
xmax=8760,
xlabel={time [hour]},
xtick={0,1000,2000,3000,4000,5000,6000,7000,8000},
xmajorgrids,
every outer y axis line/.append style={black},
every y tick label/.append style={font=\color{black}},
ymin=0,
ymax=120,
ylabel={Energy price [\euro/MWh]},
ymajorgrids,
legend style={at={(0.5,0.97)},anchor=north,legend columns=4,legend cell align=left,align=left,draw=white,fill=white}
]
\addplot [color=mycolor1,line width=2.0pt,mark size=0.3pt,only marks,mark=*,mark options={solid},forget plot]
  table[row sep=crcr]{%
1	15.15\\
2	12.96\\
4	11.7\\
6	11.35\\
8	9.54\\
10	11.64\\
12	13.15\\
13	15.24\\
15	12.43\\
16	9.92\\
17	12.12\\
18	15.24\\
20	17.73\\
21	15.63\\
23	15.1\\
24	12.95\\
25	9.62\\
26	7.64\\
27	4.96\\
28	0.06\\
30	7.08\\
31	12.5\\
32	21.31\\
33	30.44\\
34	35.48\\
35	33.06\\
37	37.97\\
39	36.24\\
40	32.18\\
42	52.94\\
43	66.7\\
44	53.53\\
45	39.54\\
46	35.9\\
48	30.64\\
49	27.4\\
50	25.23\\
51	15.63\\
52	8.74\\
53	11.28\\
55	26.02\\
56	31.66\\
58	31.94\\
60	31.45\\
61	39.05\\
62	30.99\\
64	30.43\\
66	36.98\\
67	34.97\\
68	42.18\\
69	30.39\\
70	26.32\\
71	31.82\\
73	11.94\\
75	7.93\\
76	5.23\\
78	8.96\\
80	9.91\\
82	12.76\\
84	14.03\\
86	14.24\\
88	30\\
89	12.86\\
90	16.79\\
91	21.08\\
92	18.1\\
93	16.14\\
94	13.84\\
96	17.34\\
97	15.46\\
99	13.33\\
100	10.96\\
102	11.66\\
104	11.38\\
105	13.77\\
106	27.42\\
107	30.9\\
109	32.07\\
110	29.91\\
112	16.89\\
114	46.14\\
115	40.13\\
117	34.67\\
118	29.4\\
120	22.08\\
121	13.84\\
122	11.89\\
123	9.85\\
124	5.09\\
126	8.24\\
127	17.29\\
128	48.6\\
129	46.6\\
130	40\\
131	67.8\\
133	46.08\\
135	34.5\\
136	46.08\\
137	48.99\\
138	84.71\\
139	54.57\\
140	50.7\\
141	41.75\\
142	28.34\\
143	38.52\\
144	33.69\\
145	9.26\\
147	3.88\\
148	0.07\\
149	2.05\\
150	6.52\\
151	26.07\\
152	34.99\\
153	49.42\\
154	52\\
156	54.36\\
158	49.99\\
159	45.93\\
160	43\\
161	46\\
162	54.44\\
163	61.93\\
164	56.29\\
165	40.02\\
166	31.58\\
167	36.08\\
168	33.51\\
169	22.01\\
170	14.61\\
171	12.67\\
173	9.99\\
174	22.41\\
175	30.31\\
176	46.42\\
177	57.96\\
179	53.7\\
181	56.46\\
182	54.3\\
183	51.11\\
184	47\\
185	52.06\\
186	70.01\\
187	65\\
188	61.89\\
189	48.39\\
190	38.97\\
192	38.79\\
193	27.18\\
195	24.83\\
196	18.34\\
197	12.96\\
198	25.04\\
199	34.65\\
200	48\\
201	56.97\\
202	48.54\\
204	47.38\\
206	46.65\\
207	41.82\\
208	37.84\\
209	43.83\\
210	52.66\\
211	62.06\\
212	59.78\\
213	47.89\\
214	40.66\\
215	45.35\\
217	42.64\\
218	38\\
219	32.2\\
220	16.27\\
222	33.77\\
223	46.7\\
224	62.87\\
225	59.51\\
226	61.62\\
228	58.32\\
230	53.76\\
231	50.84\\
232	46.87\\
233	53.65\\
234	74.86\\
235	66.02\\
236	61.5\\
237	57.3\\
238	46.82\\
239	50\\
240	47.33\\
241	40.36\\
242	28.17\\
243	35.03\\
244	28.26\\
246	28.27\\
247	34.12\\
248	38.34\\
249	43.65\\
251	46.36\\
253	49.72\\
254	41.89\\
255	37.94\\
257	40.28\\
258	51.79\\
259	59.12\\
261	50.94\\
262	44.8\\
263	52.81\\
264	48.69\\
265	43.47\\
266	40\\
267	25.86\\
268	16.97\\
269	14.51\\
270	16.47\\
272	23.1\\
273	31.5\\
274	44.33\\
275	25.75\\
276	23.98\\
277	29.73\\
278	24.67\\
280	27.07\\
281	31.11\\
282	39.66\\
283	41.96\\
284	46.09\\
285	37.35\\
286	32.85\\
287	37.62\\
288	32.17\\
289	26\\
290	27.86\\
292	13.45\\
294	26.71\\
295	33.11\\
296	50.84\\
298	53.89\\
299	64.36\\
301	55.89\\
303	54.94\\
305	57.93\\
306	67.48\\
307	64.41\\
308	60.71\\
309	51.62\\
310	44.67\\
312	43.21\\
313	31.14\\
315	29.56\\
316	24.17\\
317	19.72\\
318	30.34\\
319	36.98\\
320	60.09\\
321	65.94\\
323	66.95\\
325	60.02\\
326	55.01\\
327	52.54\\
329	55.88\\
330	70.68\\
331	68\\
332	53.7\\
334	45.14\\
336	39.99\\
337	33.03\\
338	30.66\\
340	29.63\\
342	30.65\\
343	39.94\\
344	55.67\\
345	62.65\\
346	58.07\\
347	55.47\\
349	54.16\\
351	50.75\\
353	52.1\\
354	56.49\\
355	62.96\\
357	47\\
358	38.6\\
360	35.75\\
361	30.06\\
363	25.6\\
364	21.42\\
365	18.42\\
366	28.66\\
367	33.35\\
368	50.09\\
369	48.34\\
370	51.57\\
372	52.2\\
374	55\\
376	53.62\\
378	79.94\\
379	55.55\\
381	46.02\\
382	35.23\\
383	37.69\\
385	29.53\\
387	19.58\\
388	11.23\\
390	20.74\\
391	34.03\\
392	48.97\\
394	47.71\\
396	47.96\\
398	46.17\\
399	42.32\\
400	47.44\\
402	68.99\\
403	52.75\\
404	49.43\\
405	43.74\\
406	33.33\\
408	35.9\\
409	30.5\\
411	28.58\\
412	26.79\\
413	18.63\\
414	22.25\\
416	30.25\\
417	33.34\\
418	39.96\\
419	35.17\\
420	32.44\\
422	30.89\\
424	30.88\\
425	33.17\\
426	54.53\\
427	45.05\\
428	39.72\\
429	30.21\\
431	30.4\\
432	38.63\\
433	36.1\\
434	30.14\\
435	26.15\\
436	9.48\\
437	12.8\\
438	16.41\\
439	12.18\\
440	15.76\\
441	29.99\\
442	37.57\\
443	44.38\\
444	46.47\\
445	49.96\\
446	44.12\\
447	35.64\\
448	27.42\\
449	29.67\\
450	60.71\\
452	46.54\\
453	42.51\\
454	40.02\\
455	41.86\\
456	38\\
457	34.75\\
458	29.46\\
460	14.99\\
461	29.33\\
462	25.5\\
463	45.93\\
464	71.97\\
466	64.5\\
468	64.44\\
469	59.61\\
471	54.16\\
472	49.7\\
473	52.53\\
474	64.91\\
475	77.99\\
476	63.97\\
478	50.4\\
479	58.17\\
480	49.38\\
481	44.79\\
482	34.35\\
483	30.65\\
486	33.68\\
487	45.86\\
488	59.91\\
489	62.98\\
490	60.19\\
491	63.32\\
492	65.65\\
493	62.91\\
495	57.41\\
496	50.81\\
497	54.09\\
498	60.23\\
500	62.97\\
501	52.97\\
502	48.03\\
503	51.45\\
504	47.44\\
505	45.67\\
506	43.09\\
507	34.97\\
508	31.02\\
510	35.41\\
511	48.5\\
512	67.49\\
514	65.66\\
515	60.88\\
516	62.79\\
518	59.37\\
520	54.76\\
521	61.62\\
522	71.58\\
523	80.06\\
524	64.04\\
525	62\\
526	53.5\\
527	61.58\\
528	53.5\\
529	41.81\\
531	39.22\\
532	31.32\\
534	34.03\\
535	43.13\\
536	60.59\\
537	65.1\\
539	64.92\\
540	62.17\\
542	60.91\\
543	54.77\\
544	49.69\\
546	55.17\\
547	63.96\\
548	57.53\\
549	53.78\\
550	45.45\\
551	53.8\\
552	46.91\\
553	35.05\\
554	32.29\\
555	30.34\\
557	30.53\\
559	43.42\\
560	61.6\\
562	63.3\\
564	63.28\\
565	60.39\\
567	58.97\\
568	53.19\\
570	64.41\\
572	63.71\\
573	54.64\\
574	50.15\\
575	47.09\\
576	49\\
577	37.68\\
578	32.05\\
580	30.32\\
582	29.04\\
583	30.9\\
585	39.27\\
586	42.04\\
588	44.11\\
590	37.68\\
591	32.69\\
593	33.57\\
594	38.13\\
595	48.04\\
596	43.14\\
597	32.57\\
598	30.05\\
599	32.86\\
600	38.23\\
601	22.11\\
602	19.14\\
603	14.85\\
605	13.22\\
607	9.53\\
608	14.05\\
610	22.19\\
612	22.85\\
614	19.08\\
615	15.23\\
616	12.89\\
617	8.26\\
618	17.66\\
619	22.42\\
620	26.7\\
621	31.96\\
622	23.75\\
623	28.24\\
624	20.29\\
625	12.63\\
627	12.81\\
628	7.1\\
630	13.09\\
631	30.14\\
632	47.43\\
633	51.07\\
635	53.47\\
637	56.05\\
638	50.71\\
640	51.92\\
642	55.53\\
643	79.92\\
644	64.25\\
645	47.25\\
646	38.71\\
648	38.51\\
649	36.41\\
650	32.92\\
651	30.28\\
653	30.16\\
655	41.47\\
656	55.05\\
658	54.52\\
659	52\\
661	49.94\\
662	47.44\\
664	48\\
665	49.77\\
666	57.75\\
667	70\\
668	62.09\\
669	56.94\\
670	47.4\\
671	53.42\\
672	48.28\\
673	38.65\\
674	30.54\\
676	28.98\\
678	31.08\\
679	37.1\\
680	55.38\\
681	52.95\\
683	57.44\\
685	53.55\\
687	50.01\\
688	47.44\\
690	61.52\\
691	83.27\\
692	72.99\\
693	54.43\\
694	49.5\\
695	54.11\\
696	52.05\\
697	37.75\\
698	34.94\\
700	29.92\\
702	34.43\\
703	45.2\\
704	53.07\\
706	64.78\\
707	59.99\\
708	64.16\\
709	59.72\\
710	54.48\\
711	51.89\\
713	51.56\\
714	55.12\\
715	67.98\\
716	59.67\\
717	50.5\\
718	45.81\\
720	47.1\\
721	42.96\\
723	36.54\\
724	33.06\\
726	40.03\\
727	46.9\\
728	69.94\\
729	71.95\\
731	65.88\\
732	60.93\\
733	52.93\\
734	48.14\\
735	45\\
737	44.23\\
738	52.92\\
739	61.76\\
740	56.35\\
741	49.08\\
742	40\\
743	41.99\\
744	39.17\\
745	31.04\\
746	28.94\\
747	26.63\\
748	20.87\\
750	20.71\\
751	26.82\\
752	28.7\\
753	37.27\\
754	42.65\\
756	44.55\\
757	48.59\\
758	40.24\\
759	32.87\\
761	33.28\\
762	43.56\\
763	50.02\\
764	44.94\\
765	41.24\\
766	35.81\\
768	35.96\\
769	41.62\\
770	31.96\\
771	25.8\\
772	19.38\\
773	13.18\\
775	16.54\\
776	18.85\\
777	22.07\\
778	30.83\\
780	31.82\\
781	34.15\\
782	32.14\\
784	31.03\\
785	32.95\\
786	36.15\\
787	48.68\\
788	54.57\\
790	45.76\\
791	49.81\\
793	44.37\\
794	37.8\\
795	36\\
796	31.56\\
798	36.19\\
799	54.1\\
800	71.89\\
802	67.5\\
803	57.18\\
804	54.63\\
806	51.96\\
807	46.82\\
809	46.76\\
810	54.09\\
811	67.8\\
812	59.96\\
813	57.98\\
814	48.72\\
815	52.2\\
817	44.42\\
819	40.9\\
820	30.63\\
822	32.38\\
823	47.13\\
824	60.3\\
825	58.49\\
826	53.88\\
827	51.99\\
829	54.68\\
830	51.68\\
831	48.05\\
833	48.03\\
834	53.05\\
835	63.99\\
836	55.73\\
837	48\\
838	41.26\\
839	44.08\\
841	30.53\\
843	28.99\\
844	25.71\\
845	18.11\\
846	28.08\\
847	33.51\\
848	45.16\\
849	49.94\\
850	52.61\\
851	58.57\\
852	54.16\\
853	48.08\\
854	52.2\\
855	50.09\\
856	43.92\\
858	49.69\\
859	67.38\\
860	53.32\\
861	36.79\\
862	57.1\\
863	52\\
864	45\\
865	42.28\\
866	39.94\\
867	34.79\\
868	28.27\\
870	32.2\\
871	44\\
872	55.87\\
874	58.84\\
875	61.77\\
877	57.28\\
879	55.29\\
880	49.94\\
881	47.89\\
882	55.89\\
883	66.01\\
884	60.02\\
885	48.05\\
886	37.12\\
887	33.71\\
888	39.61\\
889	23.64\\
890	19.51\\
891	14.66\\
892	12.18\\
893	8.54\\
894	15.42\\
895	30.68\\
896	47.79\\
897	45.71\\
898	47.63\\
900	51.34\\
901	48.37\\
902	42.94\\
903	36.86\\
904	32.31\\
906	36.16\\
907	59.92\\
908	54.71\\
909	43.31\\
910	30.49\\
911	38.37\\
912	40.84\\
913	28\\
915	25.9\\
917	10.52\\
918	16.2\\
919	24.04\\
920	27.13\\
921	29.51\\
923	28.07\\
925	28.03\\
926	25.5\\
928	28.81\\
930	31.17\\
931	42.58\\
932	33.88\\
933	27.57\\
934	24.29\\
935	27.55\\
936	23.6\\
937	14.23\\
939	12.87\\
941	9.72\\
943	10.63\\
945	11.04\\
946	19.16\\
948	17.29\\
949	14.68\\
951	10.5\\
953	12.42\\
954	14.69\\
955	35.79\\
957	31.23\\
958	27.97\\
959	30.04\\
960	25.31\\
961	36.13\\
962	33.87\\
963	27.22\\
964	23.86\\
966	33.18\\
967	45.43\\
968	59.42\\
970	58.79\\
972	54.96\\
974	56.82\\
976	51.47\\
977	47.44\\
978	54.94\\
979	79.49\\
980	54.32\\
981	50.06\\
982	46.57\\
984	42.83\\
985	30.75\\
987	29.45\\
989	28.05\\
990	30.72\\
991	41.12\\
992	53.8\\
994	56.62\\
995	54.1\\
996	51.74\\
998	49.25\\
999	42.49\\
1000	39.56\\
1002	51.9\\
1003	58.97\\
1004	67.49\\
1005	50\\
1006	46.48\\
1007	49.48\\
1008	44.09\\
1010	41.94\\
1011	37.84\\
1012	28.43\\
1014	36.72\\
1015	45.54\\
1016	59.3\\
1017	65.77\\
1018	68.42\\
1019	60.87\\
1020	55.9\\
1021	50.33\\
1023	45.92\\
1024	43.34\\
1025	40.93\\
1026	44.99\\
1027	58.94\\
1028	54.37\\
1029	52.03\\
1030	42.78\\
1031	46.13\\
1032	44.03\\
1034	34.54\\
1035	32.23\\
1036	18.33\\
1037	12.47\\
1038	26.67\\
1039	42.44\\
1040	55.88\\
1041	53.54\\
1042	57.79\\
1044	57.98\\
1045	55\\
1046	52.89\\
1048	45.25\\
1049	42.91\\
1050	48.66\\
1051	68.92\\
1052	58.69\\
1053	52.81\\
1054	42\\
1056	38.04\\
1057	33\\
1059	29.33\\
1060	24.42\\
1062	29.44\\
1063	46.4\\
1064	61.12\\
1065	64.8\\
1066	69.25\\
1067	62.12\\
1068	55\\
1069	52.08\\
1070	47.1\\
1071	44.22\\
1072	36.2\\
1073	40\\
1074	46.18\\
1075	53.48\\
1076	48.73\\
1077	36.97\\
1078	30.39\\
1080	42.44\\
1081	27.73\\
1082	17.99\\
1083	12.38\\
1084	10.36\\
1086	10.33\\
1088	13.34\\
1089	18.26\\
1090	22.66\\
1092	23.08\\
1094	22.48\\
1096	19.33\\
1098	27.73\\
1099	39.29\\
1100	44.85\\
1101	28.93\\
1103	41.5\\
1105	25.49\\
1106	14.46\\
1107	12.35\\
1108	9.06\\
1109	5.3\\
1110	8.08\\
1111	11.12\\
1113	10.73\\
1114	13.37\\
1115	16.81\\
1117	18.23\\
1118	13.15\\
1119	11.38\\
1121	12.22\\
1122	15.26\\
1123	40.15\\
1124	47.98\\
1126	34.73\\
1127	39.08\\
1128	31.53\\
1130	25.69\\
1131	23.33\\
1132	17.52\\
1133	19.63\\
1134	26.47\\
1135	43.79\\
1136	58.75\\
1137	53.31\\
1139	49.32\\
1141	44.22\\
1142	42.07\\
1143	34.96\\
1144	37.25\\
1145	40.33\\
1146	43.03\\
1147	82.04\\
1148	71.74\\
1149	56.51\\
1150	47.19\\
1151	49.94\\
1152	43.43\\
1153	31.95\\
1155	30.09\\
1157	29.2\\
1158	31.3\\
1159	42.03\\
1160	56.19\\
1162	53.79\\
1163	51.9\\
1165	51.42\\
1166	48.48\\
1167	43.27\\
1169	45.72\\
1170	53.91\\
1171	73.68\\
1172	59.98\\
1173	45.94\\
1174	39.68\\
1175	41.9\\
1176	36.47\\
1177	32.16\\
1178	29.65\\
1180	27.67\\
1182	29.34\\
1183	38.19\\
1184	47.45\\
1185	49.44\\
1186	52.21\\
1188	53.96\\
1189	50.76\\
1191	49.37\\
1193	44.71\\
1194	47.44\\
1195	70\\
1196	59.54\\
1197	54.27\\
1198	47.02\\
1199	44.18\\
1201	31.28\\
1202	29.47\\
1204	27.75\\
1205	25.59\\
1206	27.85\\
1207	34.51\\
1208	46.28\\
1209	48.34\\
1211	44.93\\
1212	46.95\\
1214	44.94\\
1215	40.02\\
1216	34.96\\
1217	32.45\\
1218	41.78\\
1219	55\\
1220	52.17\\
1221	47.44\\
1222	39.95\\
1223	41.78\\
1225	32.91\\
1226	22.15\\
1228	19.42\\
1230	26.64\\
1231	35.83\\
1232	46.31\\
1233	49.03\\
1234	52.95\\
1235	50.07\\
1237	42.88\\
1238	37.61\\
1239	32.16\\
1241	31.68\\
1242	42.9\\
1243	49.44\\
1245	49.08\\
1246	40.71\\
1248	41\\
1249	35.11\\
1250	29.96\\
1251	26.09\\
1252	21.02\\
1253	18.67\\
1254	21.15\\
1255	27.93\\
1257	32.92\\
1258	37.43\\
1260	36.86\\
1261	38.85\\
1262	35.47\\
1263	29.59\\
1264	31.99\\
1266	37.71\\
1267	50.91\\
1268	53.1\\
1269	44.22\\
1270	36.72\\
1271	40.34\\
1272	36.96\\
1273	26.11\\
1274	21.63\\
1276	19.48\\
1277	16.28\\
1278	19.86\\
1279	15.73\\
1281	17.1\\
1283	17.46\\
1285	14.68\\
1287	11.26\\
1288	7.45\\
1289	10.84\\
1291	24.31\\
1292	35.49\\
1293	27.81\\
1294	23.11\\
1295	27.1\\
1297	14.5\\
1299	17.22\\
1300	13.68\\
1302	18.12\\
1303	35.66\\
1304	46.95\\
1305	50\\
1306	46.65\\
1308	41.79\\
1310	39.32\\
1311	35.99\\
1312	31.07\\
1313	32.97\\
1314	46.16\\
1315	64.95\\
1317	48.99\\
1318	35.43\\
1319	37.44\\
1320	33.87\\
1321	26.22\\
1322	23.87\\
1324	20.29\\
1326	25.94\\
1327	36.97\\
1328	46.95\\
1330	46.12\\
1332	44\\
1334	40.49\\
1335	36.09\\
1337	33.71\\
1338	42.61\\
1339	72.68\\
1340	62.52\\
1341	46.87\\
1342	39.88\\
1343	42.89\\
1344	39.59\\
1345	36.05\\
1346	30\\
1347	27.29\\
1349	27.86\\
1351	39.22\\
1352	45.05\\
1353	49.24\\
1354	51.3\\
1356	50\\
1357	46.91\\
1359	44.96\\
1361	41.81\\
1362	43.99\\
1363	71.19\\
1364	72.95\\
1365	49.85\\
1366	45.59\\
1367	42.44\\
1368	40.05\\
1369	33.99\\
1370	32.03\\
1371	30.28\\
1373	29.48\\
1374	32.18\\
1375	41.54\\
1376	53.34\\
1377	50.1\\
1378	47.1\\
1380	46\\
1381	44.24\\
1383	43.22\\
1385	40.62\\
1386	51.76\\
1387	64.96\\
1388	57.83\\
1389	48.66\\
1390	43.21\\
1391	41.45\\
1393	36.25\\
1395	30.19\\
1396	28.03\\
1398	29.94\\
1399	39.6\\
1400	47.45\\
1401	49.45\\
1403	49.12\\
1405	46.72\\
1406	44.7\\
1407	42.31\\
1409	42.98\\
1410	47.3\\
1411	64.68\\
1412	59.92\\
1413	47.28\\
1414	39.75\\
1415	42.95\\
1417	40.61\\
1418	35.72\\
1419	33.79\\
1420	30.22\\
1422	29.33\\
1424	33.71\\
1425	39.51\\
1426	47.4\\
1427	50.34\\
1428	48.18\\
1430	43.25\\
1431	38.63\\
1432	36.46\\
1433	33.77\\
1434	41.27\\
1435	63.21\\
1436	74.76\\
1437	45.93\\
1438	42.66\\
1439	48.38\\
1440	44.97\\
1442	39.4\\
1443	32.28\\
1444	29.31\\
1445	27.57\\
1447	27.33\\
1449	27.49\\
1450	32.39\\
1452	33.67\\
1454	27.45\\
1455	23.55\\
1457	25.42\\
1458	28.22\\
1459	33.96\\
1460	42.62\\
1461	39.44\\
1462	31.89\\
1463	36.16\\
1464	32.59\\
1465	26.52\\
1467	24.08\\
1468	22.14\\
1469	24.03\\
1470	27.01\\
1471	43.93\\
1472	49.94\\
1474	53.68\\
1475	51.43\\
1476	57.13\\
1477	47.83\\
1479	46.48\\
1480	38.95\\
1481	40.98\\
1482	44.96\\
1483	72.91\\
1484	80.69\\
1485	47.81\\
1486	43.99\\
1488	43.34\\
1490	39.19\\
1491	33.29\\
1492	28.99\\
1494	33.28\\
1495	47.18\\
1496	53.9\\
1497	57.9\\
1499	51.02\\
1501	52.3\\
1502	50\\
1504	46.51\\
1506	45.87\\
1507	75.64\\
1508	79.38\\
1509	54.77\\
1510	50.68\\
1512	50.54\\
1513	40.21\\
1515	42.1\\
1516	34.72\\
1518	41.28\\
1519	47.32\\
1520	62.16\\
1521	67.76\\
1523	53.01\\
1524	44.96\\
1526	44.19\\
1527	40.21\\
1528	38.02\\
1530	46.54\\
1531	58.03\\
1532	68.37\\
1533	49.04\\
1534	42.58\\
1535	45.99\\
1536	44.1\\
1537	37.43\\
1539	34.14\\
1540	28.81\\
1542	33.98\\
1543	42.85\\
1544	55.33\\
1546	56.22\\
1547	50\\
1549	46.41\\
1551	40.51\\
1552	34.11\\
1553	37\\
1554	45\\
1555	61.59\\
1556	69.75\\
1557	53.9\\
1558	42.17\\
1559	45.73\\
1560	40.97\\
1562	33.98\\
1563	32.22\\
1564	28.9\\
1566	32.06\\
1567	40.9\\
1568	48.26\\
1569	53.76\\
1570	62\\
1571	49.17\\
1572	45.36\\
1574	37.93\\
1575	35.64\\
1576	31.97\\
1578	39.36\\
1579	47.09\\
1580	49.26\\
1581	45.95\\
1582	41.92\\
1583	44.28\\
1585	39.97\\
1586	33.31\\
1587	30.44\\
1588	25.71\\
1590	25.82\\
1592	29.6\\
1593	31.96\\
1595	28.19\\
1596	26.17\\
1597	24.42\\
1598	20.02\\
1599	15.91\\
1600	12.03\\
1601	16.02\\
1602	22.55\\
1603	35.58\\
1604	42.77\\
1605	30.03\\
1606	25.14\\
1608	23.65\\
1609	15.67\\
1610	13.75\\
1612	11.4\\
1615	11.49\\
1617	11.55\\
1619	11.54\\
1621	12.26\\
1622	10.29\\
1623	6.2\\
1624	1.75\\
1626	9.83\\
1627	22.41\\
1628	41.73\\
1629	34.86\\
1630	28.07\\
1631	30.09\\
1632	26.92\\
1633	24.26\\
1635	22.73\\
1637	22.11\\
1638	25.57\\
1639	43.5\\
1640	48.28\\
1642	44.76\\
1643	40.9\\
1644	37.42\\
1646	35.85\\
1647	33.8\\
1648	30.13\\
1650	36.36\\
1651	56.07\\
1652	75.46\\
1653	46.32\\
1654	38.68\\
1655	35.17\\
1656	30.93\\
1657	25.83\\
1659	24.63\\
1661	23.78\\
1662	26.5\\
1663	35.08\\
1664	42.55\\
1665	44.58\\
1666	40.03\\
1668	36.07\\
1669	31.88\\
1671	28.18\\
1673	30.91\\
1674	36.26\\
1675	49.94\\
1676	54.1\\
1677	42.56\\
1678	34.04\\
1679	40.03\\
1680	32.99\\
1681	28.87\\
1683	26.65\\
1685	25.69\\
1686	28.09\\
1687	35.8\\
1688	44.54\\
1690	41.27\\
1691	32.59\\
1693	30\\
1695	28.42\\
1697	30.79\\
1698	37.04\\
1699	49\\
1700	75.98\\
1701	48.43\\
1702	42.01\\
1703	38.28\\
1704	32.08\\
1706	30.65\\
1708	30\\
1710	31.43\\
1711	43.03\\
1712	53.46\\
1713	49.84\\
1714	41.27\\
1715	34.74\\
1716	29.77\\
1718	28.42\\
1720	30.95\\
1721	35.54\\
1722	30.84\\
1723	52.76\\
1724	75.55\\
1725	48.69\\
1726	43.63\\
1728	40.69\\
1729	34.96\\
1730	32.01\\
1732	29.57\\
1734	32.89\\
1735	42.04\\
1736	49.94\\
1737	47.93\\
1738	42.94\\
1740	38.83\\
1741	37.01\\
1742	34.91\\
1743	30.96\\
1745	30.91\\
1746	37.91\\
1747	45.07\\
1748	53.95\\
1749	49.94\\
1750	38.64\\
1751	42.44\\
1752	38.99\\
1753	28.64\\
1754	18.86\\
1755	21.1\\
1756	11.3\\
1757	7.58\\
1758	11.81\\
1760	14.02\\
1761	22.83\\
1762	20.01\\
1763	16.52\\
1765	21.57\\
1766	15.2\\
1767	12.55\\
1769	11.26\\
1770	18.81\\
1771	44.94\\
1772	48\\
1773	25.54\\
1774	23.29\\
1775	20.33\\
1777	15.89\\
1779	13.29\\
1781	11.11\\
1783	11.23\\
1785	11.44\\
1787	12.34\\
1789	13.95\\
1790	11.07\\
1791	9.33\\
1792	3.07\\
1793	5.41\\
1794	9.43\\
1795	22.78\\
1796	52\\
1797	49\\
1798	21.31\\
1800	28.62\\
1802	26.33\\
1803	23.93\\
1804	20.73\\
1806	26.41\\
1807	42\\
1808	50.81\\
1809	49\\
1810	56.84\\
1811	59.16\\
1812	65.27\\
1813	50\\
1815	49.96\\
1817	51.87\\
1818	67.78\\
1819	47.44\\
1820	69.94\\
1821	54.86\\
1822	44\\
1824	44.25\\
1825	31.57\\
1827	29.99\\
1828	24.07\\
1830	28.06\\
1831	40.44\\
1832	49.49\\
1834	63.77\\
1835	55\\
1836	61.99\\
1837	58.13\\
1838	52.24\\
1839	45.13\\
1840	42\\
1842	42.44\\
1843	39.68\\
1844	46.28\\
1845	41.55\\
1846	39\\
1847	32.73\\
1848	30.33\\
1849	22.2\\
1851	14.3\\
1852	22.46\\
1853	29.66\\
1854	27.44\\
1855	42.36\\
1856	49.79\\
1857	47.99\\
1859	39.01\\
1860	34.72\\
1862	32.59\\
1864	28.19\\
1865	42.44\\
1866	47.59\\
1867	42.44\\
1868	54.97\\
1869	49\\
1870	45\\
1872	41.93\\
1873	26.47\\
1875	24.54\\
1876	17.18\\
1878	24.57\\
1879	37.54\\
1880	46.15\\
1881	41.33\\
1882	35.3\\
1883	31.3\\
1884	29.01\\
1885	26.8\\
1887	25.76\\
1889	26.84\\
1890	28.95\\
1891	36.94\\
1892	45.07\\
1893	32\\
1894	27.84\\
1895	31.24\\
1897	21.24\\
1898	12.08\\
1900	10.25\\
1902	18.63\\
1903	29.62\\
1904	40\\
1906	38.33\\
1907	44.54\\
1908	63.62\\
1909	57.66\\
1910	55.89\\
1911	45\\
1912	39.96\\
1914	38.93\\
1915	42.43\\
1916	55\\
1917	47.39\\
1918	43.08\\
1920	36.5\\
1921	33.94\\
1922	28.54\\
1923	24.08\\
1924	19.07\\
1925	13.66\\
1926	17.92\\
1927	22.76\\
1929	29.83\\
1930	32.37\\
1932	32.48\\
1934	27.59\\
1936	26.45\\
1938	31.39\\
1939	38.04\\
1940	45.01\\
1941	36.93\\
1942	31.7\\
1944	24.69\\
1945	22.91\\
1946	20.54\\
1948	20.59\\
1949	17.65\\
1950	19.82\\
1951	17.05\\
1953	21.36\\
1954	24.42\\
1955	26.9\\
1956	30.58\\
1958	24.04\\
1960	18.55\\
1962	19.29\\
1963	40\\
1964	45.12\\
1965	42\\
1966	37.1\\
1967	40.51\\
1968	35.55\\
1969	32.64\\
1970	30\\
1971	27.8\\
1972	25.08\\
1974	28.82\\
1975	40.58\\
1976	57.32\\
1978	54.94\\
1979	42.91\\
1981	43.18\\
1983	43.91\\
1984	41.46\\
1986	40.93\\
1987	47.23\\
1988	96.69\\
1989	48.73\\
1990	41.12\\
1991	44.62\\
1992	41.69\\
1993	37.3\\
1995	33.3\\
1996	30.65\\
1998	35.52\\
1999	42.44\\
2000	45\\
2001	49.47\\
2003	51.15\\
2004	45.85\\
2006	42.97\\
2008	39.82\\
2010	44.07\\
2011	49.08\\
2012	81.94\\
2013	48.12\\
2014	43.42\\
2016	46.99\\
2017	39.53\\
2019	40.42\\
2020	32.22\\
2022	34.68\\
2023	44.49\\
2024	48.93\\
2025	52.06\\
2027	54.01\\
2028	48.32\\
2030	43.52\\
2032	41.09\\
2034	42.24\\
2035	48.68\\
2036	86.74\\
2037	46.5\\
2038	42.37\\
2040	42.44\\
2043	38.41\\
2044	35.02\\
2046	38.08\\
2047	46.51\\
2048	51.57\\
2049	59.9\\
2050	52.44\\
2052	50.58\\
2053	48.79\\
2055	46.77\\
2056	43.27\\
2058	44.45\\
2059	50\\
2060	81.51\\
2061	62.32\\
2062	46.92\\
2064	45.4\\
2065	40.77\\
2067	41.04\\
2068	37.01\\
2069	35.1\\
2070	40.21\\
2071	45.87\\
2072	51.76\\
2073	58.1\\
2075	47.4\\
2076	45\\
2077	42.06\\
2079	40.81\\
2081	40\\
2082	43\\
2084	70\\
2085	45\\
2086	42.44\\
2088	40.06\\
2089	43.99\\
2090	32.81\\
2091	31.02\\
2093	32\\
2095	36.89\\
2096	39.94\\
2097	43\\
2098	47.44\\
2099	45.62\\
2100	42.62\\
2101	40\\
2102	38\\
2103	34.99\\
2105	39.96\\
2106	43\\
2108	46.17\\
2109	54.96\\
2110	46\\
2112	40.56\\
2113	44\\
2114	34.94\\
2115	29.49\\
2117	30.7\\
2118	35.04\\
2120	39.33\\
2122	41\\
2123	38.83\\
2125	36.99\\
2126	32.44\\
2128	31.8\\
2129	39.96\\
2131	45\\
2132	69.97\\
2133	59.94\\
2134	55\\
2135	42.44\\
2136	30.91\\
2138	31.28\\
2140	35.77\\
2141	28.14\\
2142	40.69\\
2143	69.53\\
2144	55.47\\
2146	48.33\\
2147	46.08\\
2149	43.73\\
2150	41.79\\
2151	40\\
2152	42.22\\
2154	40.47\\
2155	48.47\\
2156	54.61\\
2157	50\\
2158	44.94\\
2160	54.48\\
2161	43.64\\
2162	41.06\\
2164	36.26\\
2165	44.94\\
2166	48\\
2167	57.44\\
2168	62.19\\
2169	57.89\\
2170	47.29\\
2172	37.77\\
2173	40\\
2175	38.76\\
2177	42.11\\
2178	73.81\\
2179	50.12\\
2180	45\\
2181	47\\
2182	120\\
2184	65\\
2185	37.69\\
2187	37.53\\
2189	36.93\\
2190	44\\
2191	49.22\\
2192	62.17\\
2193	54.35\\
2194	48\\
2195	45\\
2197	44\\
2198	42.12\\
2200	44\\
2201	47.44\\
2203	52.92\\
2204	58.31\\
2205	53.12\\
2206	44.94\\
2207	40.98\\
2208	34.99\\
2209	31.27\\
2211	32.81\\
2212	28.3\\
2214	38.64\\
2215	55.5\\
2216	52.09\\
2217	49.94\\
2219	51\\
2220	47.44\\
2222	44.94\\
2224	45\\
2225	50\\
2226	69.97\\
2227	44.94\\
2228	50\\
2229	47.44\\
2230	63.41\\
2232	37.23\\
2233	34.37\\
2235	30.51\\
2237	31.4\\
2238	37.77\\
2239	46.07\\
2240	51.91\\
2241	56.15\\
2242	63.46\\
2243	120\\
2244	58.9\\
2245	47.44\\
2246	44.94\\
2247	40\\
2250	39.94\\
2252	41.76\\
2253	38.72\\
2254	65.37\\
2255	43\\
2256	34.3\\
2257	37.84\\
2258	34\\
2260	30.74\\
2262	29.5\\
2263	34.44\\
2264	43.18\\
2265	46\\
2267	47.44\\
2268	42.8\\
2269	35.87\\
2271	34.44\\
2273	35\\
2274	42.44\\
2276	49.99\\
2277	47.44\\
2278	42\\
2279	44.01\\
2280	35.16\\
2281	42.16\\
2282	34.94\\
2284	31.5\\
2285	33.97\\
2286	24.25\\
2288	25.31\\
2289	34.94\\
2290	38.99\\
2291	41.96\\
2292	47.44\\
2293	39.94\\
2294	35.15\\
2296	34.94\\
2297	39.96\\
2298	42.17\\
2299	44.94\\
2300	59.94\\
2301	54.96\\
2302	48\\
2303	35.99\\
2304	29.88\\
2305	27.73\\
2307	29.85\\
2309	29.99\\
2310	41\\
2311	49.94\\
2312	53.05\\
2313	51.06\\
2314	47.45\\
2315	44.61\\
2316	38.94\\
2317	37.09\\
2319	36.26\\
2321	39.12\\
2322	45.31\\
2324	48.75\\
2325	38.58\\
2326	34.86\\
2327	30.07\\
2328	28.01\\
2330	21.92\\
2332	22.69\\
2334	31.84\\
2335	38.99\\
2336	43.96\\
2337	45.8\\
2338	48.41\\
2339	53.7\\
2340	43.88\\
2342	45.32\\
2343	40.42\\
2345	44.46\\
2346	39.94\\
2348	66.56\\
2349	63.54\\
2351	55\\
2352	41.06\\
2353	37.77\\
2354	35.97\\
2356	37.06\\
2357	73.11\\
2358	80.78\\
2359	50.12\\
2360	55.91\\
2361	59.81\\
2362	52.44\\
2363	47.12\\
2364	44\\
2366	41.22\\
2367	42.98\\
2369	45.87\\
2370	55\\
2371	50.12\\
2372	54.25\\
2373	42.44\\
2375	70\\
2376	81.93\\
2377	39.75\\
2378	60.01\\
2379	39.15\\
2381	38.44\\
2382	69.9\\
2383	59.94\\
2384	63.98\\
2385	60.89\\
2386	54.91\\
2387	60\\
2388	45\\
2390	72.04\\
2392	69.9\\
2393	75.33\\
2395	50\\
2397	55\\
2398	69.9\\
2399	39.78\\
2400	50\\
2401	100\\
2402	55\\
2403	40.1\\
2404	59.94\\
2405	37.34\\
2406	42.91\\
2407	65\\
2408	70.93\\
2410	69.54\\
2411	52.44\\
2412	68.76\\
2413	50\\
2414	42.86\\
2415	39.99\\
2416	38\\
2417	39.94\\
2419	44.44\\
2421	42.5\\
2422	45\\
2423	42.02\\
2424	79.9\\
2425	41.06\\
2426	39.05\\
2428	38.27\\
2430	39.54\\
2431	42.44\\
2432	46.24\\
2433	50.4\\
2434	60.52\\
2435	56.09\\
2436	45.12\\
2437	40\\
2439	37.84\\
2441	39.94\\
2442	42.44\\
2444	44.94\\
2445	53.12\\
2446	50.31\\
2448	40.88\\
2449	30\\
2450	19.99\\
2451	30\\
2452	34.61\\
2454	34.92\\
2456	39.65\\
2458	39.94\\
2460	39.99\\
2462	35\\
2463	32.96\\
2464	34.94\\
2465	38.37\\
2466	41.55\\
2468	49.96\\
2469	48\\
2470	40\\
2471	31.42\\
2472	20.12\\
2473	14.94\\
2475	15.78\\
2476	19.96\\
2477	22.54\\
2478	37.14\\
2479	43.2\\
2481	48.65\\
2483	46.67\\
2484	42.9\\
2486	45\\
2488	44.96\\
2489	55.74\\
2490	48\\
2492	50\\
2493	55\\
2494	49.04\\
2495	42.44\\
2496	38.11\\
2497	36.22\\
2498	34.31\\
2499	30\\
2501	27.44\\
2502	38.07\\
2503	44.28\\
2504	50.97\\
2505	58.5\\
2506	51\\
2508	47.44\\
2509	49.94\\
2511	43\\
2513	41.06\\
2515	40\\
2516	42.32\\
2517	45\\
2519	42.14\\
2520	36.96\\
2521	34.48\\
2523	33.94\\
2525	37.36\\
2526	40.92\\
2527	55.1\\
2528	58.87\\
2529	63.9\\
2530	47\\
2531	42.44\\
2532	39.94\\
2535	42\\
2536	44\\
2537	49.31\\
2538	45\\
2540	47.44\\
2542	72.29\\
2543	42.39\\
2544	37.73\\
2545	35.49\\
2547	33.04\\
2548	29.25\\
2549	31.76\\
2550	38.36\\
2551	46.11\\
2552	51.52\\
2554	49.94\\
2555	55.02\\
2556	44.94\\
2558	42.44\\
2561	42.44\\
2562	39.99\\
2563	47.44\\
2565	54.25\\
2566	45\\
2567	41.45\\
2568	36.83\\
2569	31.73\\
2571	29.95\\
2574	36.83\\
2576	42.48\\
2577	46.72\\
2579	51.9\\
2580	43.45\\
2582	39.94\\
2585	42.12\\
2586	39.99\\
2588	39.94\\
2589	42.44\\
2591	43.84\\
2592	35.91\\
2593	29.1\\
2595	27.98\\
2597	24.71\\
2598	27.72\\
2599	29.99\\
2600	35.65\\
2601	39.94\\
2603	38.12\\
2604	36\\
2606	32.99\\
2607	35\\
2609	35.65\\
2610	38.38\\
2612	37.84\\
2614	38.39\\
2616	22.76\\
2617	18.19\\
2619	19.75\\
2621	19.9\\
2623	12.92\\
2624	27.21\\
2626	30.4\\
2628	31.3\\
2629	27.46\\
2630	21.68\\
2632	24.96\\
2633	32.16\\
2634	39.94\\
2635	43\\
2636	50\\
2637	54.94\\
2639	42\\
2640	27.46\\
2641	23\\
2642	19.94\\
2644	20.09\\
2645	28.74\\
2647	20.61\\
2648	29.79\\
2649	32.44\\
2650	34.99\\
2652	33.86\\
2653	30.65\\
2655	14.99\\
2656	27.99\\
2657	34.94\\
2658	40\\
2660	42.44\\
2661	47.44\\
2662	49.94\\
2663	40.41\\
2664	36.1\\
2666	31.81\\
2668	29.7\\
2670	40.67\\
2671	50.24\\
2672	54.16\\
2674	51\\
2675	48\\
2677	46.72\\
2678	44.26\\
2680	44.16\\
2681	47.44\\
2682	44.94\\
2684	45.67\\
2686	56.96\\
2687	43.75\\
2688	31.96\\
2691	31.95\\
2692	28.94\\
2693	34.33\\
2694	40.59\\
2695	46.49\\
2696	50.8\\
2697	44\\
2699	47.44\\
2700	44.94\\
2703	43.82\\
2705	47.44\\
2706	45\\
2708	47.44\\
2710	44\\
2711	37\\
2713	32.29\\
2715	31.38\\
2717	32.16\\
2718	39.83\\
2719	45.42\\
2721	42.56\\
2722	45\\
2723	48.28\\
2725	44.94\\
2726	41.33\\
2728	40.87\\
2730	39.97\\
2732	39.45\\
2733	42.44\\
2735	35.98\\
2736	29.5\\
2738	29.66\\
2740	29.99\\
2741	31.95\\
2742	35.82\\
2743	44.4\\
2745	57.24\\
2746	53.99\\
2747	44.94\\
2748	37.44\\
2749	39.94\\
2751	43.67\\
2753	50.11\\
2755	45\\
2757	48.9\\
2758	39.94\\
2759	42.39\\
2760	40.16\\
2761	33.49\\
2762	31.75\\
2763	28.02\\
2765	28.08\\
2767	30.35\\
2768	32.79\\
2769	35\\
2771	35.95\\
2773	29.7\\
2774	31.75\\
2775	27.9\\
2777	34.58\\
2778	40\\
2779	42.44\\
2780	44.94\\
2781	50.48\\
2782	54.94\\
2783	45\\
2784	35.28\\
2785	29.68\\
2787	27.15\\
2788	25.23\\
2789	19.19\\
2791	27.9\\
2792	22.51\\
2793	32.5\\
2794	34.96\\
2795	38.46\\
2797	40\\
2798	35.23\\
2799	19.79\\
2800	16.22\\
2801	42\\
2803	41\\
2804	38\\
2805	44.96\\
2806	41\\
2808	33.28\\
2809	29.68\\
2812	28.42\\
2814	38.1\\
2815	55.52\\
2816	52.44\\
2817	55.54\\
2819	54.02\\
2821	42.57\\
2823	39.27\\
2824	36.13\\
2825	38.23\\
2826	43.38\\
2827	53.12\\
2828	44.95\\
2829	50.07\\
2830	47.44\\
2831	44.94\\
2832	35.84\\
2834	34.99\\
2836	33.53\\
2838	38.97\\
2839	47.92\\
2840	50.38\\
2842	47.44\\
2843	49.32\\
2844	46.74\\
2845	50.58\\
2846	48\\
2847	44.34\\
2849	48\\
2850	45\\
2852	46.16\\
2854	49.96\\
2855	43.59\\
2856	31.85\\
2857	30.01\\
2859	31.23\\
2860	34.51\\
2862	35.03\\
2863	44.93\\
2864	52.26\\
2865	49.11\\
2867	49.94\\
2868	43.59\\
2869	48.28\\
2871	44\\
2872	40.56\\
2873	44.94\\
2874	40.12\\
2875	37.44\\
2877	42.44\\
2878	45\\
2879	39.94\\
2880	34.99\\
2881	38.29\\
2882	33.38\\
2885	30.76\\
2887	31.05\\
2888	34.78\\
2889	40\\
2890	42.44\\
2891	45\\
2892	48\\
2893	43\\
2894	25\\
2895	35\\
2896	42.44\\
2899	42.44\\
2900	40\\
2901	47.44\\
2902	50\\
2903	37.99\\
2904	47.44\\
2905	36.49\\
2906	32.07\\
2908	29.15\\
2909	31.5\\
2910	33.78\\
2911	38\\
2912	42\\
2913	51.27\\
2914	56.13\\
2915	50.88\\
2916	46\\
2917	35\\
2919	26.06\\
2921	35\\
2923	33.26\\
2925	35\\
2927	34.99\\
2929	26.73\\
2930	23.06\\
2931	30.43\\
2934	28.29\\
2935	31.77\\
2936	38\\
2937	42\\
2939	39.96\\
2940	35.61\\
2941	25.47\\
2942	23.28\\
2944	32.44\\
2945	37.44\\
2946	44.96\\
2947	42.44\\
2948	40\\
2949	49.94\\
2950	53.76\\
2951	49.94\\
2953	41.19\\
2954	32.44\\
2955	30.7\\
2956	35.92\\
2957	37.74\\
2958	33.16\\
2959	25.68\\
2960	30.7\\
2961	33.07\\
2963	32.44\\
2965	11.22\\
2966	5.34\\
2968	7.3\\
2969	32.75\\
2970	40\\
2972	42.44\\
2973	49.94\\
2974	54.96\\
2975	45.99\\
2976	30.83\\
2978	34.33\\
2980	36.45\\
2982	40.09\\
2983	44.08\\
2984	48.09\\
2985	43.94\\
2987	47\\
2988	39.99\\
2989	47.44\\
2990	42.44\\
2992	43.96\\
2993	47.63\\
2994	52.44\\
2995	48\\
2997	53.94\\
2998	49.94\\
2999	35.02\\
3000	36.98\\
3001	34.44\\
3002	31.05\\
3003	24.89\\
3004	30.65\\
3005	27.45\\
3006	34.37\\
3007	37.98\\
3009	44.94\\
3010	57.71\\
3012	44.94\\
3013	42.44\\
3014	40\\
3016	41.27\\
3017	47\\
3018	37.04\\
3019	38.94\\
3021	40.12\\
3022	45\\
3023	41.45\\
3024	32.58\\
3025	30.22\\
3027	28.89\\
3029	30.74\\
3030	38.02\\
3031	44.82\\
3032	47.07\\
3033	59.49\\
3034	57.46\\
3036	54.39\\
3037	47\\
3038	44.94\\
3040	42.44\\
3041	47.25\\
3042	42.41\\
3043	38.44\\
3045	45\\
3046	48.87\\
3047	32\\
3048	29.67\\
3049	26.87\\
3051	26.07\\
3053	27.03\\
3054	35.78\\
3055	38.09\\
3057	44.94\\
3058	47.44\\
3061	48.87\\
3062	46.56\\
3064	45.26\\
3065	53.26\\
3066	47\\
3067	40\\
3068	47.6\\
3069	43.18\\
3070	36.69\\
3071	34.43\\
3072	36.43\\
3073	30.11\\
3074	22.82\\
3075	27.23\\
3076	23.15\\
3077	29.35\\
3078	34.86\\
3079	39.94\\
3081	47\\
3082	48.88\\
3083	45\\
3084	42.44\\
3085	37.44\\
3086	34.94\\
3088	34.3\\
3090	33.28\\
3092	35.27\\
3094	47.44\\
3095	40.83\\
3097	30.52\\
3099	30.06\\
3100	25.29\\
3101	27.99\\
3102	23.53\\
3103	28.35\\
3104	34.94\\
3105	45\\
3106	55.62\\
3107	50\\
3108	45\\
3109	40\\
3110	33.7\\
3112	32.25\\
3113	35\\
3115	33.99\\
3117	31.63\\
3119	23.55\\
3121	21.56\\
3123	6.38\\
3125	2.8\\
3126	-0.01\\
3128	2.28\\
3129	8.06\\
3130	10.09\\
3132	10.82\\
3134	20.01\\
3135	3.51\\
3136	20.01\\
3137	2.47\\
3138	35\\
3140	33.73\\
3142	32.76\\
3143	24.48\\
3145	23.01\\
3147	22.7\\
3149	31.8\\
3150	37.69\\
3151	39.94\\
3152	42.44\\
3153	48\\
3154	57.71\\
3156	44.99\\
3157	52\\
3159	48.4\\
3161	64.96\\
3162	42.44\\
3164	42.92\\
3165	48.87\\
3166	57.44\\
3167	49.01\\
3169	35.16\\
3170	32.75\\
3172	31.91\\
3173	29.26\\
3174	37.17\\
3175	50.3\\
3176	58.51\\
3177	63.46\\
3178	51.85\\
3180	48.87\\
3181	42.65\\
3182	39.78\\
3183	44.17\\
3184	38\\
3186	39.96\\
3187	41.79\\
3189	41.03\\
3191	34.07\\
3192	31.83\\
3193	27.44\\
3194	31.12\\
3196	31.57\\
3197	58.74\\
3198	33.71\\
3199	59.35\\
3200	49.11\\
3201	44.94\\
3202	42.44\\
3203	40.33\\
3205	39.94\\
3206	42.44\\
3208	46.67\\
3209	49.11\\
3210	43\\
3211	39.99\\
3213	40.66\\
3214	72.25\\
3215	47\\
3216	31.77\\
3218	31.89\\
3220	30\\
3222	38.64\\
3223	47.42\\
3224	55.64\\
3225	51.69\\
3226	47.39\\
3228	40.37\\
3230	37.98\\
3231	35.2\\
3233	37.44\\
3234	40\\
3237	43.2\\
3238	70.58\\
3239	40\\
3240	32.34\\
3242	33.72\\
3244	30.65\\
3246	34.09\\
3247	55.13\\
3248	44.96\\
3249	47\\
3250	45\\
3251	41\\
3253	40\\
3255	38\\
3256	39.94\\
3257	42.44\\
3259	40\\
3261	39.94\\
3263	40\\
3264	31.5\\
3265	27.88\\
3266	32.98\\
3268	33.67\\
3269	31.68\\
3270	26.31\\
3271	30.39\\
3272	34.96\\
3273	42.44\\
3274	37.44\\
3275	34.83\\
3277	31.68\\
3279	29.99\\
3280	31.98\\
3281	36.4\\
3282	44.94\\
3283	46.83\\
3285	50\\
3287	48.06\\
3288	52.35\\
3289	32\\
3290	35.13\\
3291	33.33\\
3293	30.75\\
3294	17.97\\
3295	20.01\\
3297	29\\
3298	30.75\\
3300	29.96\\
3301	22\\
3302	31.03\\
3303	33.2\\
3305	31.77\\
3306	37.44\\
3308	39.94\\
3309	45\\
3310	49\\
3311	39.94\\
3312	30.9\\
3314	30.57\\
3316	28.12\\
3318	38.44\\
3319	49.95\\
3320	53.11\\
3321	50.39\\
3322	46.96\\
3324	44\\
3326	39.45\\
3327	37.44\\
3329	41.18\\
3330	46.5\\
3331	53.46\\
3332	49.47\\
3333	47.44\\
3334	39.98\\
3335	33.29\\
3337	30.59\\
3339	29.42\\
3341	31.5\\
3342	42.77\\
3343	53.94\\
3345	46.82\\
3346	42.7\\
3348	37.99\\
3350	36.08\\
3352	41\\
3353	57.21\\
3354	48\\
3356	53.11\\
3357	46.04\\
3359	48.01\\
3360	45.85\\
3361	35.72\\
3362	33.52\\
3364	32.7\\
3366	41.19\\
3367	48.83\\
3368	52.91\\
3370	44\\
3371	68.83\\
3372	39.94\\
3373	44.94\\
3374	42.3\\
3375	39.94\\
3376	44.94\\
3377	53.11\\
3378	43.12\\
3379	47.27\\
3380	43.46\\
3381	37.97\\
3382	44.94\\
3384	28.25\\
3385	24\\
3386	20.41\\
3387	15.13\\
3388	18.41\\
3389	22.91\\
3390	29.25\\
3391	36.55\\
3392	39.66\\
3394	40.5\\
3395	42.29\\
3397	39.94\\
3398	42.29\\
3400	48\\
3402	54.24\\
3403	44.7\\
3405	46.63\\
3406	55\\
3407	47.01\\
3408	30.05\\
3409	34.8\\
3410	31.81\\
3413	30.75\\
3414	34.04\\
3415	49.95\\
3417	52.97\\
3419	57.04\\
3420	47.98\\
3421	45.23\\
3422	40.53\\
3423	37.38\\
3424	35\\
3425	37.54\\
3426	42\\
3427	43.98\\
3428	41.92\\
3430	42.58\\
3431	49.78\\
3432	47.24\\
3433	34.57\\
3435	33.19\\
3437	30.58\\
3439	32.44\\
3440	39.94\\
3442	48\\
3444	44.94\\
3445	35\\
3447	32.44\\
3449	37.44\\
3450	40\\
3451	42.44\\
3452	45\\
3453	40\\
3454	49.96\\
3455	44.96\\
3456	32.94\\
3457	24.96\\
3458	20.34\\
3459	16.18\\
3461	15.79\\
3462	18.94\\
3463	27\\
3464	30.85\\
3465	37.99\\
3467	38.99\\
3469	32.94\\
3470	13.43\\
3472	30\\
3473	42.44\\
3474	44.94\\
3475	50\\
3478	55\\
3479	44.94\\
3480	50\\
3481	32.75\\
3483	29.99\\
3484	32.75\\
3485	29.99\\
3486	37.12\\
3487	46.58\\
3488	42.44\\
3490	42.4\\
3491	60\\
3492	52.08\\
3493	70.42\\
3495	69.99\\
3496	49.04\\
3497	60\\
3498	55\\
3499	45.21\\
3500	42.29\\
3501	44.94\\
3502	69.99\\
3503	41.75\\
3504	30.03\\
3505	31.82\\
3507	33.85\\
3509	50\\
3510	33.63\\
3511	43.97\\
3512	47.44\\
3513	50.01\\
3516	44.71\\
3518	45.42\\
3519	41.89\\
3521	45\\
3522	50.12\\
3523	40.94\\
3525	41.67\\
3527	37.58\\
3528	200\\
3529	34.62\\
3530	30.85\\
3532	31.63\\
3534	33.85\\
3535	44.94\\
3537	45.17\\
3538	54\\
3539	50\\
3540	60\\
3541	50\\
3542	45.06\\
3543	43.01\\
3544	46.4\\
3545	54.31\\
3546	44.94\\
3547	41.2\\
3548	38.13\\
3549	41.96\\
3550	54\\
3551	44.94\\
3552	39.94\\
3553	33.94\\
3555	33.58\\
3557	29.42\\
3559	29.96\\
3561	32.44\\
3562	34.37\\
3563	37.44\\
3564	39.94\\
3565	34.94\\
3567	33\\
3569	34.33\\
3570	44.94\\
3572	49.83\\
3573	44.53\\
3574	47.94\\
3575	45\\
3576	33.05\\
3577	30.06\\
3579	31.35\\
3582	32.37\\
3583	38.73\\
3584	42.42\\
3585	44.94\\
3586	42.44\\
3588	40.95\\
3589	35.63\\
3591	34.69\\
3593	37.11\\
3594	47.44\\
3595	45.24\\
3597	42.76\\
3599	39.94\\
3600	36.47\\
3601	44.25\\
3602	35\\
3603	32.87\\
3606	32.51\\
3607	35\\
3609	43.15\\
3610	37.12\\
3612	34.94\\
3613	38.75\\
3614	34.25\\
3616	34.94\\
3617	38.75\\
3618	41\\
3620	39.94\\
3622	44.94\\
3624	38.23\\
3625	34.99\\
3626	38.79\\
3627	34.29\\
3628	32.29\\
3630	29.3\\
3632	31.35\\
3634	33.56\\
3635	36\\
3636	38\\
3637	33.67\\
3638	31.68\\
3639	35\\
3640	30.37\\
3642	36\\
3643	39\\
3644	44\\
3646	45\\
3647	46.9\\
3648	49.23\\
3649	41.08\\
3650	37.04\\
3651	34.04\\
3653	30.58\\
3654	38.33\\
3655	42.04\\
3657	43.5\\
3659	44.99\\
3660	41.01\\
3661	50\\
3662	47.7\\
3663	43.58\\
3665	54.85\\
3666	40.76\\
3667	42.89\\
3669	41.91\\
3670	48.93\\
3671	45.32\\
3672	32.46\\
3674	31.47\\
3676	30.79\\
3678	40.94\\
3679	48.29\\
3680	53.15\\
3681	50.94\\
3682	47.86\\
3683	51.25\\
3684	47.69\\
3685	44.99\\
3687	44.37\\
3689	47.02\\
3691	48.97\\
3693	44.48\\
3695	37.5\\
3696	32.49\\
3698	33.2\\
3700	33.25\\
3702	37.66\\
3703	47.44\\
3704	50.44\\
3706	53.15\\
3707	55.55\\
3709	55.12\\
3710	48.37\\
3711	45.17\\
3712	47\\
3713	54.94\\
3714	48\\
3715	41.91\\
3716	39.67\\
3717	37.24\\
3718	45.51\\
3719	39.94\\
3720	28.95\\
3722	27.58\\
3724	24.07\\
3726	30.97\\
3727	37.67\\
3728	42\\
3729	46.44\\
3730	49\\
3732	42\\
3733	44.94\\
3734	40.12\\
3735	37.44\\
3737	43.56\\
3738	39.94\\
3740	37.2\\
3742	45.41\\
3743	41.99\\
3744	31.38\\
3746	31.41\\
3748	31.39\\
3750	40.01\\
3752	44.94\\
3753	47.18\\
3754	45\\
3756	40\\
3758	39.94\\
3759	38\\
3761	42.44\\
3763	39.44\\
3764	37.68\\
3766	43.7\\
3767	32.3\\
3768	28.34\\
3769	26.07\\
3771	24.73\\
3773	24.09\\
3774	10.35\\
3775	15.98\\
3776	29.19\\
3777	31.86\\
3778	34.94\\
3779	32.14\\
3781	20.01\\
3782	17.38\\
3783	29.58\\
3785	33.86\\
3786	38.43\\
3788	38\\
3789	33.87\\
3790	39.94\\
3791	30.57\\
3793	27.3\\
3794	20.12\\
3795	25.04\\
3797	16.32\\
3799	15\\
3801	18.13\\
3802	27.41\\
3804	18.45\\
3806	14.06\\
3808	16.55\\
3809	20.08\\
3810	27.51\\
3811	37\\
3813	34.93\\
3814	37.11\\
3816	31.37\\
3817	28.36\\
3818	25.04\\
3820	24.99\\
3821	22.28\\
3822	24.33\\
3823	27.99\\
3824	26.09\\
3826	26.46\\
3827	28.89\\
3829	26.71\\
3831	21.61\\
3832	31.09\\
3833	36\\
3835	39.94\\
3836	45\\
3837	39.94\\
3838	45.37\\
3839	35.99\\
3840	30.47\\
3842	30.01\\
3844	28.66\\
3845	21.77\\
3846	30.01\\
3847	40.12\\
3849	45.39\\
3850	48\\
3851	49.88\\
3852	45\\
3854	44.94\\
3856	44.49\\
3857	49.69\\
3858	43.84\\
3860	39.91\\
3861	36.92\\
3862	41.84\\
3864	31.07\\
3866	30.68\\
3868	30.71\\
3870	34.92\\
3871	44.99\\
3872	49.54\\
3874	49.88\\
3876	49.4\\
3877	44.98\\
3879	41.81\\
3880	39.57\\
3882	47.96\\
3883	44.94\\
3884	39.6\\
3885	36.24\\
3886	43\\
3888	30.95\\
3890	29.39\\
3892	28.74\\
3894	33.09\\
3895	44.07\\
3896	46.37\\
3898	46.12\\
3900	42.94\\
3901	41.05\\
3903	38.35\\
3905	39.4\\
3906	42\\
3907	44.77\\
3909	40.51\\
3911	39\\
3912	32.37\\
3914	31.39\\
3916	31.11\\
3918	36\\
3919	44.82\\
3921	53.88\\
3922	51.42\\
3923	55\\
3924	45.6\\
3925	43\\
3926	47.75\\
3927	39.94\\
3928	38\\
3929	44\\
3930	40\\
3931	35.12\\
3932	47\\
3933	32.36\\
3934	49.52\\
3935	45\\
3937	32\\
3939	29.73\\
3940	27.34\\
3941	20\\
3942	22.33\\
3943	29.34\\
3944	32.03\\
3945	48.57\\
3947	49.94\\
3948	42.44\\
3949	37.44\\
3950	33.32\\
3952	31.87\\
3954	36\\
3955	39.94\\
3956	36\\
3958	45.1\\
3959	48.33\\
3960	31.61\\
3962	28.82\\
3964	27.46\\
3965	23.12\\
3966	16.43\\
3967	22.89\\
3968	27.44\\
3969	31.24\\
3971	32.91\\
3973	31.6\\
3975	22.89\\
3977	30.65\\
3979	33.11\\
3980	35.25\\
3981	31.59\\
3982	34.96\\
3983	31.63\\
3985	27.8\\
3986	25.84\\
3987	21.49\\
3988	27.46\\
3989	24.89\\
3990	33.93\\
3991	41.17\\
3993	40\\
3994	42.44\\
3995	47.44\\
3996	45\\
3998	37.49\\
3999	34.99\\
4000	33.09\\
4002	35.11\\
4004	35.99\\
4006	36.05\\
4007	30.9\\
4009	30.87\\
4011	31.36\\
4013	28.57\\
4014	36.96\\
4015	44.71\\
4016	51.99\\
4018	50\\
4020	42.98\\
4022	41.45\\
4024	37.44\\
4026	39.84\\
4028	41.22\\
4029	36.31\\
4030	40.48\\
4031	35.65\\
4032	29.87\\
4034	31.32\\
4036	31.05\\
4037	28.22\\
4038	34.97\\
4039	48.51\\
4040	43.23\\
4041	48.51\\
4044	62.13\\
4045	39\\
4047	37.44\\
4049	60.01\\
4050	65.5\\
4051	57.01\\
4052	60\\
4053	44.99\\
4054	69.47\\
4055	47.44\\
4056	31.33\\
4058	31.22\\
4060	30.86\\
4062	30.52\\
4063	38\\
4064	39.94\\
4065	43.73\\
4066	49.47\\
4067	55\\
4068	40\\
4069	42\\
4070	44.57\\
4071	39.94\\
4072	37.44\\
4073	39.94\\
4074	50.01\\
4075	36.07\\
4076	45.95\\
4077	36.09\\
4078	60\\
4079	35.5\\
4080	30.84\\
4082	30.99\\
4084	30.25\\
4085	27.02\\
4086	30.43\\
4087	35\\
4089	40\\
4090	42.44\\
4091	49.88\\
4092	42.44\\
4093	44.94\\
4094	42.44\\
4095	35\\
4098	36.01\\
4100	35.52\\
4101	33.51\\
4102	43\\
4103	44.83\\
4104	31.49\\
4107	30.68\\
4108	26.71\\
4109	18.63\\
4110	27.4\\
4112	31.01\\
4114	34.09\\
4116	34.31\\
4118	33\\
4120	31.41\\
4121	34.94\\
4123	39.94\\
4125	32.44\\
4126	42\\
4127	36\\
4128	31.49\\
4130	31.01\\
4131	17.58\\
4133	14.59\\
4135	14.24\\
4137	25.04\\
4138	29\\
4140	30.55\\
4142	29.75\\
4144	29.96\\
4145	32.85\\
4146	34.94\\
4147	44.94\\
4148	50\\
4149	40\\
4150	50.5\\
4151	45.01\\
4152	31.24\\
4154	30.98\\
4155	27.4\\
4156	20.18\\
4157	23.88\\
4158	31.21\\
4159	39.01\\
4160	35.96\\
4162	33.99\\
4163	35.9\\
4165	38.02\\
4167	39.23\\
4169	41.94\\
4170	47.91\\
4172	43.98\\
4173	38.48\\
4174	41.92\\
4175	36.31\\
4176	30.36\\
4178	28.54\\
4180	27.87\\
4182	36.12\\
4183	44.9\\
4184	48\\
4186	50.61\\
4187	54.11\\
4188	47.85\\
4189	45.23\\
4190	41.98\\
4191	39.94\\
4193	40.79\\
4195	42.93\\
4197	38.86\\
4199	36.02\\
4200	30.98\\
4201	28.71\\
4203	26.9\\
4205	28.87\\
4206	35.06\\
4207	44.4\\
4209	44.81\\
4211	45.98\\
4212	41.2\\
4214	39.47\\
4216	38.41\\
4217	40.45\\
4219	43.22\\
4220	41.23\\
4221	38.86\\
4222	40.63\\
4223	34.96\\
4224	31.98\\
4225	29.69\\
4227	27.76\\
4229	29.11\\
4230	36.71\\
4231	44.69\\
4232	47.55\\
4233	44.06\\
4235	45.33\\
4236	42.85\\
4238	44.98\\
4240	41.9\\
4241	44.83\\
4242	48.59\\
4244	45.59\\
4245	41.96\\
4247	36.74\\
4248	32.48\\
4249	29.67\\
4251	28.14\\
4253	29.08\\
4254	35.39\\
4255	41.97\\
4256	45.41\\
4258	43.79\\
4260	40.96\\
4261	38.98\\
4262	36.07\\
4263	34.24\\
4265	37.93\\
4266	40.47\\
4268	38.9\\
4269	36\\
4270	41.96\\
4271	38.54\\
4272	46.71\\
4273	30.33\\
4275	28.02\\
4277	27.07\\
4279	29.99\\
4280	31.78\\
4281	45.12\\
4282	56.38\\
4284	54.25\\
4285	42.44\\
4286	38.79\\
4287	35.33\\
4289	40\\
4290	50\\
4291	40\\
4292	36\\
4293	39.96\\
4294	48.46\\
4295	46.09\\
4296	31.22\\
4298	24.58\\
4300	28.47\\
4301	22.51\\
4303	22.78\\
4304	29\\
4306	35\\
4307	39.04\\
4309	33.97\\
4310	30\\
4312	29.94\\
4313	36\\
4315	35.2\\
4317	37\\
4318	33.98\\
4319	31.37\\
4320	28.08\\
4322	25\\
4324	24.4\\
4325	26.78\\
4326	34.08\\
4327	41.26\\
4329	42.29\\
4331	47\\
4332	42.55\\
4334	39.94\\
4335	36\\
4336	33.87\\
4338	37\\
4339	40.37\\
4341	36.23\\
4343	37\\
4344	31.75\\
4346	30.1\\
4348	25.97\\
4350	33.59\\
4351	41.06\\
4353	42.44\\
4354	40\\
4356	35.69\\
4358	36\\
4360	34.94\\
4362	39\\
4363	41.92\\
4364	40.09\\
4366	39.38\\
4367	36\\
4368	30.26\\
4370	29.37\\
4371	25.15\\
4373	27.97\\
4374	31.63\\
4375	39.91\\
4377	39.61\\
4379	40.43\\
4380	38.41\\
4382	36\\
4384	33.82\\
4385	36.08\\
4386	38.59\\
4388	39.02\\
4389	36.9\\
4390	38.88\\
4391	32.59\\
4392	30.03\\
4394	29.39\\
4396	29.52\\
4397	26.87\\
4398	31.4\\
4399	36.92\\
4400	38.73\\
4401	36.95\\
4403	36.3\\
4404	33.97\\
4406	33.79\\
4407	31.83\\
4409	36\\
4410	38.79\\
4412	39.3\\
4414	39.43\\
4415	33.12\\
4416	29.63\\
4417	26.87\\
4419	25.06\\
4421	26.04\\
4422	30.28\\
4423	37\\
4425	37.44\\
4427	41\\
4428	37.44\\
4430	38\\
4432	37.61\\
4433	44.98\\
4434	36.94\\
4436	36.93\\
4437	32.9\\
4439	30.15\\
4441	34.8\\
4442	30.23\\
4444	28.75\\
4445	24.99\\
4447	27.6\\
4448	30\\
4449	34\\
4450	30\\
4453	30.26\\
4454	20.03\\
4456	19.35\\
4457	28.32\\
4460	33.2\\
4462	31.85\\
4463	28.86\\
4464	47\\
4465	18\\
4466	23.55\\
4468	22.02\\
4469	4.79\\
4470	0.29\\
4471	22.79\\
4472	14.21\\
4473	26.1\\
4474	28.23\\
4476	30.32\\
4477	32.44\\
4478	30\\
4480	29.94\\
4482	43\\
4483	47.27\\
4485	40.96\\
4486	42.95\\
4487	31.6\\
4488	28.23\\
4490	25.35\\
4492	24.14\\
4493	26.02\\
4494	33.09\\
4495	36.93\\
4497	36.94\\
4499	38.13\\
4501	36.97\\
4503	35.29\\
4505	37.97\\
4507	40\\
4509	36.17\\
4511	29.64\\
4513	27.55\\
4515	25.07\\
4517	25.93\\
4518	30.87\\
4519	40.26\\
4520	42.98\\
4522	45.09\\
4524	42.03\\
4526	38.55\\
4528	36.18\\
4530	33.98\\
4532	30.83\\
4534	29.15\\
4536	25.22\\
4538	22.04\\
4540	21.7\\
4541	24.2\\
4542	27.61\\
4543	34.95\\
4544	39.08\\
4546	40\\
4547	41.91\\
4548	38.64\\
4550	36.94\\
4551	35.1\\
4552	32\\
4553	34.9\\
4555	35.19\\
4556	33.41\\
4558	30.74\\
4560	25.07\\
4561	21.85\\
4563	21.37\\
4565	25.56\\
4566	29.4\\
4567	36.46\\
4568	40.18\\
4570	39.68\\
4572	34.58\\
4574	31.88\\
4576	32.77\\
4577	34.94\\
4579	36\\
4580	33.8\\
4582	39.99\\
4583	29.48\\
4584	25.95\\
4586	23.84\\
4588	23.59\\
4589	25.86\\
4590	30.09\\
4591	36.74\\
4593	36.48\\
4595	36.27\\
4596	39.1\\
4597	35\\
4598	38\\
4599	35\\
4602	33.53\\
4604	35.97\\
4606	35.98\\
4607	29.88\\
4609	26.5\\
4611	25.07\\
4613	24.58\\
4615	29.79\\
4616	32.93\\
4618	34.96\\
4619	47.44\\
4620	40\\
4621	34\\
4622	29.94\\
4623	27.55\\
4624	29.94\\
4626	32.86\\
4627	35.96\\
4629	35.91\\
4631	29.89\\
4632	27.72\\
4633	25.27\\
4635	22.69\\
4637	19.61\\
4639	19.99\\
4640	23.66\\
4641	25.62\\
4642	28.08\\
4643	34.94\\
4644	39.94\\
4645	34\\
4646	29.89\\
4647	28.04\\
4649	34.94\\
4651	35\\
4653	32.99\\
4654	42.44\\
4655	34\\
4657	27.4\\
4658	17.57\\
4659	24.01\\
4660	26.88\\
4662	34\\
4663	41.46\\
4665	40.78\\
4666	38.01\\
4667	39.94\\
4668	33\\
4670	29.94\\
4673	33.97\\
4674	37.5\\
4676	38.08\\
4677	35.93\\
4679	30\\
4680	28.21\\
4681	26.1\\
4683	25.07\\
4685	27.05\\
4686	29.62\\
4688	36.1\\
4690	35.23\\
4692	35.93\\
4694	34\\
4696	40\\
4697	42.44\\
4698	37.49\\
4699	39.7\\
4701	38.98\\
4703	33.54\\
4704	29.35\\
4706	28.94\\
4708	26.79\\
4710	33.74\\
4711	40.3\\
4713	44.94\\
4715	46.42\\
4716	40\\
4717	42.44\\
4718	39.94\\
4720	34.94\\
4722	38.47\\
4723	42.28\\
4725	39.24\\
4727	34.45\\
4728	31.36\\
4729	28.56\\
4731	28.31\\
4733	29.59\\
4734	35.85\\
4735	42.1\\
4737	46.73\\
4739	41\\
4740	49.67\\
4741	54.94\\
4742	50.11\\
4744	50\\
4746	42.36\\
4748	40.65\\
4750	38.89\\
4751	33.15\\
4752	31.09\\
4754	28.4\\
4756	27.69\\
4757	29.44\\
4758	35.44\\
4759	41.75\\
4761	40.72\\
4763	38.1\\
4765	34.26\\
4767	34.84\\
4769	37.93\\
4770	41.16\\
4771	43.03\\
4772	40.73\\
4773	36.93\\
4774	39.5\\
4776	34.47\\
4777	31.42\\
4779	30.68\\
4781	26.24\\
4782	9.74\\
4783	14.54\\
4784	22.03\\
4785	27.3\\
4786	29.32\\
4787	27.31\\
4789	23.27\\
4791	26\\
4793	30.91\\
4794	35\\
4796	37.44\\
4797	32.35\\
4798	46.7\\
4800	47.76\\
4801	32.87\\
4802	30.24\\
4803	21.62\\
4804	18.42\\
4805	15.67\\
4807	20.38\\
4809	31.13\\
4810	35\\
4811	41.63\\
4812	35\\
4814	33\\
4816	34.02\\
4818	45\\
4819	40\\
4820	42.44\\
4821	39.1\\
4823	31.16\\
4824	34.94\\
4825	23.86\\
4827	22.99\\
4829	15.17\\
4830	30.6\\
4831	24.77\\
4832	31.35\\
4833	37.19\\
4834	42.86\\
4836	43.51\\
4838	38.44\\
4839	35.96\\
4840	34.14\\
4842	35.52\\
4844	32.87\\
4845	30.33\\
4847	29.04\\
4849	26.17\\
4850	21.92\\
4851	25.09\\
4852	11.67\\
4853	19.67\\
4854	29.73\\
4855	37.92\\
4856	42.69\\
4858	42.92\\
4860	38.04\\
4861	35.03\\
4863	32.39\\
4865	35\\
4867	37.94\\
4868	35\\
4869	33.07\\
4870	35.95\\
4871	31.88\\
4872	29.98\\
4874	25.81\\
4876	25.94\\
4878	30.9\\
4879	38.67\\
4881	36.5\\
4883	36.67\\
4884	34\\
4886	32.23\\
4888	34.03\\
4889	44.94\\
4890	36.4\\
4891	39.08\\
4892	35.93\\
4893	32.88\\
4894	44.92\\
4895	34\\
4896	29.63\\
4897	26.64\\
4899	25.05\\
4901	27.32\\
4902	29.62\\
4903	36.49\\
4905	35.45\\
4907	36.41\\
4909	36.55\\
4911	34.99\\
4913	39.94\\
4915	41.36\\
4917	39.95\\
4919	40.5\\
4920	31.13\\
4922	30.45\\
4924	30.14\\
4926	32.66\\
4927	40\\
4929	46.03\\
4930	48.96\\
4933	49.99\\
4935	46.46\\
4937	39.99\\
4939	38.16\\
4940	36.12\\
4942	36.91\\
4943	35\\
4945	32.09\\
4946	30.11\\
4947	27.98\\
4949	27.67\\
4951	30.1\\
4952	32.07\\
4953	36\\
4954	40\\
4955	46.56\\
4956	40\\
4957	36\\
4958	32.46\\
4960	31.91\\
4962	35.91\\
4963	38.05\\
4965	38.1\\
4966	47.03\\
4967	36\\
4968	31.04\\
4969	29.23\\
4971	27.95\\
4973	26.83\\
4975	26.04\\
4977	29.17\\
4978	17.03\\
4979	19.28\\
4980	22.48\\
4981	18.59\\
4982	15.91\\
4984	13.87\\
4985	32.46\\
4986	34.94\\
4987	37.11\\
4989	44.11\\
4990	48.96\\
4991	37.11\\
4993	30.64\\
4995	27.67\\
4997	28.97\\
4998	36.98\\
4999	41.97\\
5001	41.28\\
5003	47\\
5004	42.44\\
5006	40.23\\
5007	37.44\\
5009	41.02\\
5011	42.95\\
5012	39.93\\
5013	35.99\\
5015	30.44\\
5017	28.58\\
5018	11.2\\
5019	5.23\\
5020	25.85\\
5021	29.31\\
5022	32.48\\
5023	39.41\\
5024	42.5\\
5027	42.5\\
5030	36\\
5032	34.94\\
5034	36.5\\
5036	35.94\\
5038	48.97\\
5039	34.1\\
5041	33.7\\
5043	31.97\\
5045	31.54\\
5047	35.98\\
5048	41.03\\
5050	39.98\\
5052	37.44\\
5054	38\\
5057	48.96\\
5059	44.94\\
5060	39.94\\
5061	34.1\\
5062	49.94\\
5063	44.58\\
5064	32.59\\
5066	27.19\\
5067	31.41\\
5069	31.71\\
5071	36.93\\
5073	34.94\\
5074	38\\
5075	34.94\\
5077	35.81\\
5078	38\\
5079	39.94\\
5080	43.05\\
5081	44.94\\
5082	35\\
5083	38\\
5084	40\\
5085	35.97\\
5086	41.15\\
5087	38\\
5088	31.93\\
5090	31.58\\
5092	31.61\\
5094	37.78\\
5095	40.3\\
5097	44.03\\
5098	48.97\\
5099	46.44\\
5100	39.99\\
5102	37.44\\
5103	35.02\\
5104	42.97\\
5105	47.51\\
5107	45.26\\
5108	48.9\\
5110	57.44\\
5111	44.13\\
5112	32.03\\
5114	34.77\\
5115	32.03\\
5117	29.99\\
5118	27.27\\
5119	29.06\\
5121	38.1\\
5122	41\\
5123	44.94\\
5124	30.01\\
5125	42.44\\
5127	37.73\\
5129	41.45\\
5130	49\\
5131	39.78\\
5132	45.63\\
5134	49.28\\
5135	39.78\\
5136	31.31\\
5137	28.48\\
5139	29.2\\
5141	27.15\\
5142	10.59\\
5144	12.93\\
5146	13.73\\
5148	15.73\\
5150	14.44\\
5152	13.5\\
5154	17.39\\
5155	30.79\\
5156	38\\
5157	46.48\\
5158	52.44\\
5159	39.8\\
5160	27.59\\
5162	17.91\\
5163	27.56\\
5164	29.4\\
5166	32.6\\
5167	38.09\\
5169	38.7\\
5171	43.4\\
5172	39.41\\
5174	35.4\\
5176	35\\
5177	40\\
5178	36.94\\
5180	38.09\\
5181	39.94\\
5182	47.44\\
5183	32.73\\
5184	29.24\\
5186	29.19\\
5188	29.95\\
5190	32.58\\
5191	38.6\\
5192	41.04\\
5193	45.08\\
5194	49.69\\
5195	45.11\\
5196	47\\
5197	45.19\\
5199	39.02\\
5200	42\\
5201	52.44\\
5202	39.49\\
5204	38.99\\
5206	35\\
5207	30.75\\
5208	28.95\\
5210	28.22\\
5212	27.78\\
5214	30.76\\
5215	37.73\\
5216	40.19\\
5217	36.82\\
5219	40\\
5220	36.82\\
5221	34.94\\
5222	32.65\\
5224	32.44\\
5226	37.9\\
5228	38.76\\
5230	32.75\\
5231	30.35\\
5233	28.64\\
5235	29.06\\
5237	28.5\\
5238	30.61\\
5239	35.62\\
5240	37.95\\
5241	41\\
5242	47\\
5243	44.94\\
5244	39.94\\
5246	39.5\\
5247	37\\
5248	42.44\\
5249	47.55\\
5251	39.22\\
5253	38.3\\
5254	47\\
5255	39.99\\
5256	33.71\\
5257	31.2\\
5259	30.45\\
5261	29.94\\
5262	39.97\\
5263	44\\
5264	42.03\\
5265	44.44\\
5267	44.99\\
5268	42.44\\
5269	59.94\\
5270	52.63\\
5273	52.27\\
5274	44.94\\
5275	41.07\\
5276	36.87\\
5277	34.53\\
5279	38.16\\
5280	47.52\\
5281	30\\
5282	27.44\\
5284	25.3\\
5286	24.1\\
5287	26.03\\
5288	29.98\\
5289	34\\
5291	35\\
5292	33.1\\
5294	30.62\\
5296	32.7\\
5297	35\\
5299	39.94\\
5300	35.66\\
5301	43.78\\
5302	47.44\\
5303	42.44\\
5304	34.41\\
5307	30.43\\
5309	27.27\\
5311	27.2\\
5313	29.94\\
5314	32.44\\
5316	33\\
5318	30.43\\
5320	29.94\\
5321	32.44\\
5323	33\\
5325	43.26\\
5326	47.36\\
5327	29.19\\
5328	26.17\\
5329	13.45\\
5330	8.99\\
5331	5.37\\
5332	21.95\\
5333	24.8\\
5334	29.4\\
5335	31.97\\
5336	34.96\\
5337	32.9\\
5339	34.94\\
5341	35.37\\
5342	33\\
};
\addplot [color=mycolor1,line width=2.0pt,mark size=0.3pt,only marks,mark=*,mark options={solid}]
  table[row sep=crcr]{%
5344	35.7\\
5345	37.87\\
5346	34.94\\
5347	37.27\\
5349	38.32\\
5350	35.7\\
5351	29.94\\
5353	29.87\\
5354	26\\
5356	27.48\\
5358	35.01\\
5359	41.05\\
5360	39.18\\
5362	35.12\\
5363	37.2\\
5364	33.3\\
5366	34.94\\
5368	37.2\\
5369	43\\
5370	41\\
5372	42.44\\
5373	39.86\\
5375	37.2\\
5376	30.86\\
5378	30.54\\
5380	30.05\\
5382	33.95\\
5383	38.19\\
5384	39.98\\
5385	41.78\\
5386	44.8\\
5387	48\\
5388	42.11\\
5390	41.97\\
5392	40.1\\
5393	45.51\\
5394	42.97\\
5395	39.97\\
5396	37.23\\
5398	37\\
5399	34\\
5400	30.82\\
5402	29.86\\
5404	28.97\\
5406	33.99\\
5407	39.03\\
5409	37.27\\
5410	42.44\\
5412	39.94\\
5415	39.94\\
5416	42.44\\
5417	47.44\\
5418	39.94\\
5420	41.08\\
5422	37.43\\
5424	31.15\\
5426	30.3\\
5428	29.79\\
5430	31.36\\
5431	36.18\\
5433	37.97\\
5434	40.07\\
5436	37.97\\
5438	35\\
5439	33.17\\
5441	34.99\\
5442	37.91\\
5443	40\\
5445	37.97\\
5446	42.44\\
5447	39.94\\
5448	46.95\\
5449	32.5\\
5451	31.9\\
5453	31.64\\
5454	29.35\\
5455	31.9\\
5456	34.99\\
5457	45\\
5458	49.94\\
5460	45.12\\
5461	37.44\\
5462	32.2\\
5464	31.72\\
5466	40\\
5467	44\\
5468	41.11\\
5469	47.19\\
5470	50.56\\
5471	32.5\\
5473	29.64\\
5475	27.08\\
5476	5.51\\
5477	8.06\\
5478	18.23\\
5479	24.33\\
5481	31.31\\
5483	30.11\\
5484	31.9\\
5486	30.12\\
5487	17.44\\
5488	31.49\\
5490	39.96\\
5491	47\\
5492	44.94\\
5493	40\\
5495	32.45\\
5496	41.99\\
5497	28.51\\
5499	28.09\\
5501	30.03\\
5502	33.24\\
5503	35\\
5504	40\\
5505	43.78\\
5506	41.89\\
5507	45.85\\
5508	41.76\\
5509	48.53\\
5510	44.72\\
5511	42.22\\
5513	47.44\\
5514	43.78\\
5516	45\\
5517	48.97\\
5519	48.53\\
5520	38.38\\
5521	36.11\\
5522	34.28\\
5524	33.45\\
5526	33.96\\
5528	40.64\\
5529	52.44\\
5531	55.69\\
5532	52.44\\
5533	40\\
5534	44.96\\
5535	39.99\\
5537	35.03\\
5538	38\\
5539	39.94\\
5540	47.44\\
5541	37.18\\
5542	35.29\\
5543	30.63\\
5545	29.94\\
5547	29.46\\
5548	31.43\\
5550	33.36\\
5551	40\\
5553	52.44\\
5555	55.04\\
5556	39.94\\
5557	49.99\\
5558	39.94\\
5560	44.94\\
5561	58.25\\
5562	38.77\\
5563	42.33\\
5565	45\\
5567	39.5\\
5568	34.01\\
5570	32.11\\
5571	27.96\\
5573	29.16\\
5574	39.37\\
5575	66.87\\
5576	43.35\\
5577	52.2\\
5578	55\\
5579	59.3\\
5580	55\\
5581	40.17\\
5582	48\\
5583	43.78\\
5585	45\\
5586	38.71\\
5587	41.02\\
5589	39.98\\
5590	37.09\\
5591	32.94\\
5592	29.92\\
5594	28.47\\
5596	28.09\\
5598	32.82\\
5599	38.91\\
5600	41.43\\
5601	38.2\\
5602	35.7\\
5603	37.44\\
5604	34.96\\
5605	39.37\\
5607	38.3\\
5609	44.94\\
5611	40\\
5613	42.5\\
5614	39.94\\
5616	37.3\\
5617	39.97\\
5619	32.7\\
5621	33.37\\
5624	34.99\\
5626	40\\
5628	34.94\\
5630	34.99\\
5631	32.54\\
5633	39.94\\
5634	35.09\\
5636	40\\
5637	44.94\\
5639	48.75\\
5641	39.45\\
5642	36.94\\
5643	32.7\\
5645	32.42\\
5647	31.74\\
5649	33.59\\
5650	39.36\\
5651	42.44\\
5652	48\\
5653	40.5\\
5654	37.44\\
5655	35\\
5656	37.12\\
5657	44.94\\
5658	39.94\\
5659	34.94\\
5660	39.94\\
5661	57.44\\
5662	50.23\\
5663	47.06\\
5664	39.38\\
5665	33.87\\
5667	29.3\\
5669	30.13\\
5670	37.91\\
5671	42\\
5673	43.24\\
5674	49.45\\
5675	60\\
5676	64.44\\
5678	48.96\\
5679	43.52\\
5681	46.39\\
5682	54.79\\
5683	44.01\\
5685	40.45\\
5686	38.37\\
5688	32.28\\
5690	31.69\\
5692	33.29\\
5694	36.46\\
5695	43.9\\
5696	48.86\\
5697	51.98\\
5699	54.27\\
5700	48.96\\
5701	47.17\\
5702	44.08\\
5704	42.09\\
5706	43.74\\
5708	41.97\\
5709	45\\
5711	35.99\\
5712	32.49\\
5714	32.5\\
5716	32.64\\
5718	35.99\\
5719	41.99\\
5720	45.03\\
5721	49.5\\
5723	49.94\\
5724	47.44\\
5726	48.75\\
5727	43.27\\
5729	50.95\\
5730	47.44\\
5731	43.6\\
5733	40.65\\
5735	37.33\\
5737	32.55\\
5739	29.47\\
5741	31.84\\
5742	35.77\\
5743	44\\
5744	41.63\\
5745	44.96\\
5746	49.41\\
5747	51.42\\
5748	44\\
5749	46\\
5751	48.96\\
5753	49.41\\
5754	46\\
5755	41.99\\
5756	46\\
5757	51.42\\
5758	39.99\\
5759	44.59\\
5760	35.14\\
5761	32.31\\
5763	31.14\\
5765	29.96\\
5766	37.46\\
5767	44.5\\
5768	47.81\\
5769	54.44\\
5771	54.94\\
5772	45\\
5773	39.94\\
5775	37.46\\
5777	37.44\\
5779	37.71\\
5781	36.72\\
5782	40\\
5783	35.14\\
5785	33\\
5786	29.78\\
5788	25.1\\
5789	27.7\\
5790	32.99\\
5792	34.96\\
5793	44.96\\
5794	47.44\\
5795	49.5\\
5796	45\\
5797	38.49\\
5798	36.25\\
5800	35\\
5801	39.94\\
5802	42.44\\
5803	44.96\\
5804	51.99\\
5805	54.94\\
5806	49.94\\
5807	42.48\\
5808	35.03\\
5810	33.9\\
5811	31.41\\
5813	31.27\\
5815	31.74\\
5817	34.06\\
5819	35.05\\
5821	34.9\\
5822	32.64\\
5824	32.4\\
5825	34.9\\
5827	45\\
5828	52.44\\
5830	50\\
5831	42.68\\
5832	34.89\\
5834	34.22\\
5836	34.88\\
5838	41.03\\
5840	43.42\\
5842	43.76\\
5843	47.44\\
5844	50.44\\
5845	47.44\\
5846	44.94\\
5848	48.96\\
5849	44\\
5850	40.21\\
5851	50\\
5852	44.94\\
5853	48.96\\
5855	44.99\\
5856	37.88\\
5858	37.55\\
5860	37.07\\
5862	41.2\\
5863	74.36\\
5864	55.39\\
5865	59.94\\
5866	62.29\\
5868	59.96\\
5870	58.2\\
5871	55.24\\
5872	52.21\\
5873	64.94\\
5874	42.44\\
5876	53.11\\
5877	45.2\\
5878	52\\
5879	47.44\\
5880	49.99\\
5881	41\\
5882	38.96\\
5884	38.62\\
5886	42.54\\
5887	57\\
5889	60.5\\
5891	55.78\\
5892	50\\
5894	47.82\\
5895	52.07\\
5896	55.78\\
5897	69.9\\
5898	41\\
5900	55.78\\
5902	60.39\\
5903	42.44\\
5904	37.18\\
5906	35.5\\
5908	35.03\\
5910	39.3\\
5911	47.44\\
5912	45.68\\
5913	54.7\\
5914	44.94\\
5915	55.78\\
5916	44.96\\
5917	50\\
5919	47.03\\
5920	54.96\\
5922	55.31\\
5923	50\\
5924	52.44\\
5925	44.12\\
5926	39.97\\
5927	49.99\\
5928	39.5\\
5930	35.28\\
5932	37.5\\
5933	50\\
5934	38.26\\
5935	55.78\\
5936	49.48\\
5937	60.75\\
5938	63.99\\
5939	61\\
5940	55\\
5941	60\\
5942	54.94\\
5943	49.05\\
5945	55.83\\
5947	42.36\\
5948	47.44\\
5950	55\\
5951	46.9\\
5952	60.12\\
5953	49.99\\
5954	46.01\\
5955	42.44\\
5956	39.92\\
5958	41.49\\
5960	45.71\\
5961	49.91\\
5962	60\\
5964	52.44\\
5965	39\\
5967	36.57\\
5969	39.81\\
5971	39.94\\
5972	55\\
5973	49.99\\
5974	47.44\\
5975	40\\
5976	37.13\\
5978	36.29\\
5979	31.97\\
5981	31.68\\
5983	31.83\\
5985	36.91\\
5986	39.94\\
5988	40\\
5990	37.07\\
5992	36.53\\
5993	42.44\\
5994	49.78\\
5996	59.94\\
5997	57\\
5998	46\\
5999	36.67\\
6000	31.9\\
6002	30.83\\
6004	31.34\\
6006	54.99\\
6007	60\\
6008	53.11\\
6011	43.21\\
6013	42.77\\
6014	45.13\\
6016	45.54\\
6018	54\\
6020	65.15\\
6021	70\\
6022	57.11\\
6023	36.92\\
6025	39\\
6026	41.42\\
6027	39\\
6028	35.47\\
6030	37.95\\
6031	45.76\\
6032	50.95\\
6033	54\\
6035	53.1\\
6036	45\\
6038	48\\
6039	45.9\\
6041	60\\
6042	45.38\\
6044	53.78\\
6045	41.89\\
6046	44.17\\
6047	37.17\\
6048	35.24\\
6049	31.53\\
6051	31.27\\
6053	32.61\\
6054	43.04\\
6055	57.11\\
6057	46.17\\
6058	44.01\\
6060	42.45\\
6061	50\\
6062	45\\
6063	43.24\\
6064	46.09\\
6065	42.98\\
6066	55.1\\
6067	46.24\\
6068	56.23\\
6069	42.88\\
6070	72.09\\
6071	46\\
6072	35.39\\
6074	31.55\\
6076	34.71\\
6078	56\\
6079	45.96\\
6081	65.4\\
6082	45.85\\
6084	44.42\\
6085	42.47\\
6087	39.66\\
6089	41.46\\
6090	45.91\\
6092	46.18\\
6093	42.64\\
6094	37.8\\
6096	32.54\\
6098	32.14\\
6100	32.22\\
6102	40.52\\
6103	44.94\\
6104	47.39\\
6106	47.32\\
6108	46.24\\
6109	42.91\\
6110	39.34\\
6112	35.85\\
6113	42\\
6114	39.38\\
6116	50\\
6117	43.74\\
6118	50\\
6119	44\\
6120	36.27\\
6121	34.34\\
6123	32.81\\
6125	32.84\\
6127	34.29\\
6128	41.52\\
6129	44.94\\
6131	40\\
6132	37.93\\
6134	35.2\\
6136	35.18\\
6137	38.52\\
6138	43\\
6139	49.94\\
6140	59.94\\
6141	50\\
6142	40.88\\
6143	37.99\\
6144	34.33\\
6146	31.95\\
6148	32.22\\
6150	32.51\\
6152	33.38\\
6153	37.44\\
6154	40\\
6155	44.94\\
6156	49.99\\
6157	38.5\\
6159	34.96\\
6161	39.94\\
6163	49.11\\
6164	59.94\\
6165	49.94\\
6166	48\\
6167	46\\
6169	36.87\\
6171	35.41\\
6173	35.23\\
6174	43.27\\
6175	45.01\\
6176	47.92\\
6178	49\\
6180	47.75\\
6182	48.56\\
6183	46.74\\
6184	59.58\\
6185	69.94\\
6186	53.11\\
6187	49.95\\
6189	41.48\\
6191	38.43\\
6192	34.99\\
6194	33.24\\
6196	33.57\\
6198	45.01\\
6199	74.87\\
6200	57.11\\
6203	50.55\\
6204	47.23\\
6206	46.91\\
6207	57.11\\
6208	49.94\\
6209	57.05\\
6211	57.11\\
6212	74.95\\
6213	60\\
6214	57.11\\
6215	40\\
6216	37.56\\
6217	33.14\\
6219	32.53\\
6221	34.01\\
6222	58.13\\
6223	55.78\\
6224	69.21\\
6225	71\\
6226	55.78\\
6228	52.5\\
6229	57.04\\
6232	57.04\\
6234	47.74\\
6235	55.78\\
6237	53.12\\
6238	46.46\\
6240	37.61\\
6241	34.21\\
6244	37.61\\
6246	42\\
6247	49.71\\
6248	55.78\\
6250	57.04\\
6252	48.7\\
6254	47.93\\
6256	45.74\\
6257	55.78\\
6260	55.78\\
6262	55.12\\
6263	42\\
6264	70\\
6265	37.95\\
6268	37.95\\
6269	42\\
6270	60\\
6271	51.78\\
6274	58.95\\
6275	62\\
6276	48.51\\
6278	57.04\\
6279	46.38\\
6280	50\\
6281	57.04\\
6282	51.78\\
6284	60.01\\
6285	46.1\\
6286	52.44\\
6287	59.94\\
6288	53\\
6289	37.85\\
6291	33.59\\
6293	36.57\\
6294	39.99\\
6296	46.44\\
6297	57.05\\
6298	53\\
6300	48.24\\
6301	42\\
6302	40.24\\
6303	38.16\\
6305	46.44\\
6306	52.44\\
6307	57.05\\
6308	69.96\\
6309	52.44\\
6310	44.94\\
6312	42\\
6313	36.63\\
6315	32.31\\
6317	31.92\\
6318	34\\
6320	31.7\\
6321	35.28\\
6323	34.99\\
6324	37.08\\
6325	33.27\\
6327	33\\
6329	35.18\\
6330	42\\
6331	51\\
6332	60\\
6333	51\\
6334	42\\
6335	35.47\\
6337	36.63\\
6338	33.73\\
6340	35.59\\
6341	33.73\\
6342	44.99\\
6343	76.99\\
6344	54\\
6346	49.31\\
6348	51.24\\
6349	49.14\\
6351	47.95\\
6353	46.66\\
6354	54\\
6356	70\\
6357	54\\
6358	44.58\\
6360	59.99\\
6361	45\\
6362	38.4\\
6365	44.99\\
6366	72.2\\
6367	74.92\\
6368	60.27\\
6369	52.44\\
6370	50.44\\
6371	47.49\\
6372	44.06\\
6374	43\\
6375	45.94\\
6377	60.22\\
6378	72.2\\
6379	69.3\\
6380	64.94\\
6381	47.33\\
6382	65\\
6383	76.66\\
6384	40.43\\
6385	37.1\\
6386	33.3\\
6388	33.14\\
6390	42\\
6391	52.43\\
6393	57.78\\
6394	55.49\\
6395	57.84\\
6396	51.56\\
6397	53.65\\
6398	50.26\\
6399	48.26\\
6401	46.91\\
6402	59.17\\
6404	52.1\\
6405	48.36\\
6407	44.38\\
6408	47.7\\
6409	37.87\\
6410	32.61\\
6412	32.9\\
6413	35.76\\
6414	51.7\\
6415	55.5\\
6416	61\\
6417	69.32\\
6418	60.21\\
6419	57.33\\
6420	49.95\\
6422	48.5\\
6424	46.05\\
6426	51.78\\
6427	53.93\\
6428	74.57\\
6429	49.96\\
6431	45.03\\
6433	37.87\\
6435	35.6\\
6437	34.33\\
6438	40.06\\
6439	52.48\\
6441	57.04\\
6442	59.94\\
6443	56\\
6444	52.13\\
6446	48.5\\
6449	52.06\\
6451	62.13\\
6452	57.44\\
6453	55\\
6454	50.44\\
6455	48.5\\
6456	59.46\\
6457	45.91\\
6458	40.36\\
6460	38.87\\
6462	40.11\\
6463	46.93\\
6464	50.84\\
6465	55\\
6466	59.94\\
6467	57.05\\
6468	50.44\\
6469	42.44\\
6470	39.99\\
6472	40.69\\
6473	48.76\\
6474	55\\
6475	61\\
6477	50.44\\
6478	43.95\\
6479	41.43\\
6481	35.3\\
6483	33.95\\
6485	32.35\\
6488	37.67\\
6489	43.4\\
6490	47.44\\
6491	52.44\\
6493	41.12\\
6494	37.83\\
6496	36.53\\
6497	44.94\\
6498	53.95\\
6499	59.92\\
6500	75\\
6501	59.94\\
6502	50\\
6503	42.44\\
6504	32.74\\
6507	32.23\\
6509	33.47\\
6510	44.95\\
6511	53.97\\
6512	57.54\\
6514	50.12\\
6516	51.67\\
6518	48.23\\
6520	48.34\\
6521	60.01\\
6523	63.24\\
6524	57.05\\
6525	47.5\\
6527	40.7\\
6528	36.51\\
6530	36.57\\
6532	37.6\\
6533	34.15\\
6534	44.49\\
6535	54.76\\
6536	58.41\\
6537	56.6\\
6538	54.56\\
6540	52\\
6542	48.59\\
6543	46.31\\
6544	48.67\\
6545	50.97\\
6546	53.11\\
6547	66.93\\
6548	64\\
6549	52\\
6551	43\\
6552	46.44\\
6553	38.09\\
6554	32.01\\
6556	32.13\\
6557	35.71\\
6558	48.11\\
6559	58\\
6560	60.88\\
6561	65\\
6563	65.19\\
6565	64.81\\
6566	60.9\\
6567	58\\
6568	52.7\\
6569	54.74\\
6570	56.75\\
6571	69.45\\
6573	50.59\\
6574	54\\
6575	48\\
6576	38.29\\
6577	40.81\\
6578	38.29\\
6580	39.88\\
6582	54.3\\
6583	59.98\\
6585	55.15\\
6587	54.3\\
6589	54.37\\
6591	56\\
6593	70\\
6594	54.3\\
6595	63.38\\
6596	61.04\\
6597	54.3\\
6600	50.98\\
6601	45\\
6602	41.47\\
6604	41.52\\
6605	49\\
6607	56.03\\
6609	59.53\\
6611	60.43\\
6613	56.48\\
6614	54\\
6615	50.27\\
6617	51.88\\
6618	54\\
6619	74.6\\
6620	79.03\\
6621	54\\
6622	56.96\\
6624	38.65\\
6626	38.69\\
6627	33.2\\
6628	31.35\\
6630	30.83\\
6631	34.02\\
6632	38.19\\
6633	44.42\\
6634	42.44\\
6636	41.86\\
6638	38.41\\
6640	38.16\\
6641	44.94\\
6642	52.44\\
6643	64.94\\
6644	53.11\\
6645	48.8\\
6646	46.5\\
6648	38.91\\
6649	36.45\\
6650	32.43\\
6651	35.2\\
6653	35.8\\
6654	32.8\\
6656	38.64\\
6657	43\\
6659	45.2\\
6661	43\\
6662	39.94\\
6664	41.32\\
6665	48.72\\
6666	46\\
6667	60\\
6669	53.11\\
6670	45.97\\
6672	43.52\\
6673	37.27\\
6675	37.6\\
6677	32.43\\
6678	43.24\\
6679	52.78\\
6680	56.79\\
6682	57.6\\
6684	59.31\\
6685	62.6\\
6686	59\\
6687	54.61\\
6688	51.78\\
6690	59.3\\
6691	84.15\\
6692	68.49\\
6693	48.66\\
6694	51.71\\
6695	49.15\\
6696	40.5\\
6697	36.63\\
6698	32.74\\
6700	32.9\\
6701	38.9\\
6703	56.79\\
6704	60.08\\
6706	58.12\\
6708	57.21\\
6709	54.92\\
6710	48.29\\
6711	46.47\\
6712	44.42\\
6714	47.62\\
6715	65.01\\
6716	57.46\\
6717	49.99\\
6719	48.69\\
6720	41.26\\
6721	37.88\\
6723	33.16\\
6724	37.02\\
6726	38.51\\
6727	51.96\\
6728	55.31\\
6730	53.77\\
6732	51.29\\
6733	46.07\\
6735	42.26\\
6736	40\\
6738	41.14\\
6739	54.95\\
6740	51\\
6741	44.94\\
6742	42\\
6743	38.47\\
6744	32.53\\
6745	34.94\\
6746	33.12\\
6748	33\\
6749	37.23\\
6751	51.74\\
6753	57.44\\
6754	51.17\\
6756	52.94\\
6757	50.67\\
6758	48\\
6759	43.3\\
6760	41.08\\
6761	43.79\\
6762	51.83\\
6763	61.68\\
6764	53.11\\
6765	42.8\\
6766	46.48\\
6768	37.88\\
6769	32.27\\
6770	28.93\\
6772	29.64\\
6774	37.41\\
6775	51.94\\
6776	54.15\\
6778	53.23\\
6780	52.73\\
6782	49.55\\
6783	47.09\\
6784	45\\
6785	49.89\\
6786	52.22\\
6787	58.41\\
6788	52.06\\
6789	42.94\\
6791	40.76\\
6792	38.85\\
6794	36.05\\
6795	33.54\\
6797	35.25\\
6798	39.15\\
6799	41\\
6800	46.36\\
6801	56.8\\
6802	62.23\\
6804	49.5\\
6805	44.94\\
6806	42\\
6808	44.94\\
6810	43\\
6811	53.11\\
6812	47\\
6813	39.31\\
6815	38.62\\
6816	34.75\\
6817	30.96\\
6818	33.85\\
6820	32.69\\
6822	31.16\\
6824	34.94\\
6825	38.7\\
6827	40.25\\
6829	39.94\\
6831	37.64\\
6833	44.94\\
6834	47.44\\
6835	54.99\\
6836	49\\
6837	40\\
6839	35.52\\
6840	29.4\\
6842	30.71\\
6844	32.16\\
6846	51.8\\
6847	54.94\\
6849	56.66\\
6850	51.84\\
6852	51.2\\
6853	47.06\\
6854	43.26\\
6856	44.92\\
6857	47.24\\
6858	51.8\\
6859	62.47\\
6860	59.91\\
6861	49.37\\
6863	46.46\\
6864	36.68\\
6865	39.06\\
6866	31.83\\
6868	36.67\\
6869	39.62\\
6871	54.87\\
6872	57.34\\
6874	53.79\\
6876	54.22\\
6878	50.16\\
6879	47.32\\
6880	42.86\\
6881	53.6\\
6882	57.77\\
6883	67.8\\
6884	59.89\\
6885	49.5\\
6887	51.3\\
6888	40.28\\
6890	41.3\\
6892	51.36\\
6895	97.81\\
6896	54.15\\
6897	150\\
6898	77.1\\
6899	60.74\\
6900	54.66\\
6902	51.5\\
6903	60.74\\
6904	70\\
6905	80\\
6906	55\\
6907	120\\
6908	52.32\\
6909	47.5\\
6912	40.23\\
6913	51.1\\
6914	44.99\\
6915	41\\
6916	39.17\\
6917	32.55\\
6918	40.98\\
6919	52.98\\
6920	54.94\\
6922	56.55\\
6924	52.66\\
6926	49\\
6929	51.4\\
6931	57.98\\
6932	51.78\\
6933	41\\
6935	48.4\\
6937	41.5\\
6938	47.1\\
6939	45.33\\
6940	47.1\\
6942	51.2\\
6943	55.78\\
6944	53.87\\
6946	54.11\\
6947	51.93\\
6948	49.18\\
6949	45.35\\
6951	41.5\\
6952	48\\
6953	78.4\\
6954	45.94\\
6955	52.29\\
6956	45.15\\
6957	38.28\\
6958	45.5\\
6959	39.94\\
6961	36.6\\
6962	32.8\\
6963	31\\
6965	30.67\\
6967	38.15\\
6968	40.25\\
6969	46.3\\
6971	45\\
6972	42\\
6973	40.2\\
6974	38.09\\
6976	40.3\\
6977	46.3\\
6979	54.94\\
6980	51.77\\
6981	46.3\\
6983	44.88\\
6984	26.25\\
6985	24.38\\
6986	26.55\\
6988	22.76\\
6990	26.31\\
6992	28.89\\
6993	36.6\\
6995	34.94\\
6996	30.92\\
6998	29.87\\
7000	29.83\\
7001	34.8\\
7002	42\\
7003	52.6\\
7004	49.8\\
7005	43\\
7006	37.99\\
7007	32.02\\
7008	20.59\\
7009	18.1\\
7011	10.05\\
7013	15.38\\
7014	34.94\\
7015	60\\
7016	45.61\\
7017	48.65\\
7018	46.76\\
7019	49.72\\
7020	53.61\\
7022	52.95\\
7023	49.67\\
7025	60.6\\
7026	50\\
7027	57.98\\
7028	50\\
7029	39.99\\
7031	39.96\\
7032	35.14\\
7034	34.68\\
7035	31.95\\
7036	30\\
7038	37.44\\
7039	46.44\\
7040	48.93\\
7041	46.19\\
7043	50.12\\
7044	46.54\\
7045	49\\
7046	47.01\\
7047	44.62\\
7048	42.44\\
7049	47.71\\
7050	42.12\\
7051	49.53\\
7052	39.96\\
7053	37.1\\
7054	39.71\\
7055	33.92\\
7056	42.44\\
7057	35.06\\
7059	31.88\\
7061	34.47\\
7062	42.94\\
7063	50.5\\
7065	51.78\\
7066	59\\
7067	50\\
7069	45.84\\
7071	45\\
7072	49.94\\
7073	54.53\\
7074	59\\
7075	66.4\\
7076	59\\
7077	50\\
7078	53.12\\
7079	51.3\\
7080	48.41\\
7081	45.7\\
7082	41.49\\
7083	36.97\\
7085	42.94\\
7086	52.58\\
7087	63.64\\
7088	70\\
7089	68.06\\
7090	61.05\\
7091	63.64\\
7092	57.02\\
7093	52.15\\
7095	50.52\\
7097	52.03\\
7098	61.05\\
7099	69\\
7101	53.57\\
7102	47.75\\
7103	42.48\\
7104	38.5\\
7105	32.91\\
7106	30.79\\
7108	24.99\\
7109	29.99\\
7110	40\\
7111	53.49\\
7113	53.11\\
7115	59.94\\
7116	52.44\\
7118	50.25\\
7120	49.51\\
7122	49.93\\
7124	50.5\\
7125	47.07\\
7126	49.99\\
7127	48.23\\
7129	39.21\\
7130	33.32\\
7131	28.98\\
7132	33.15\\
7134	23.34\\
7135	42\\
7136	46.59\\
7137	52.44\\
7138	55.78\\
7140	57.5\\
7141	46.68\\
7142	39.94\\
7144	42.44\\
7145	50.97\\
7146	57.5\\
7148	55.78\\
7149	42.09\\
7150	45\\
7152	44.99\\
7153	38.37\\
7154	33.7\\
7155	35.74\\
7156	28.39\\
7157	20\\
7159	23.36\\
7160	25.56\\
7162	37.44\\
7164	42\\
7165	44.94\\
7166	38.3\\
7167	34.94\\
7169	38.41\\
7170	48.29\\
7171	59.37\\
7173	46.44\\
7174	43.71\\
7176	43.9\\
7177	39.27\\
7178	32.75\\
7179	27.82\\
7180	22.1\\
7181	18.6\\
7182	22.93\\
7183	39.53\\
7184	44.8\\
7185	50.03\\
7186	51.86\\
7187	49.97\\
7189	50\\
7191	47.85\\
7193	52.04\\
7194	60.51\\
7196	56.95\\
7197	47.65\\
7198	44.94\\
7200	40.5\\
7201	44.46\\
7202	38.6\\
7203	32.74\\
7205	25.48\\
7206	31.84\\
7207	41.58\\
7208	48.66\\
7209	52.6\\
7211	51.69\\
7213	47.15\\
7215	49.24\\
7216	46.32\\
7217	49.42\\
7218	58.4\\
7219	64.02\\
7220	62.19\\
7221	52.6\\
7222	45\\
7223	49.98\\
7225	42.68\\
7226	38.17\\
7227	28.48\\
7228	24.8\\
7230	28.42\\
7231	47.61\\
7232	50.38\\
7234	52.87\\
7236	52\\
7238	52.79\\
7240	52.25\\
7241	47.99\\
7242	54.35\\
7243	69.96\\
7244	60.97\\
7245	52.44\\
7246	46.93\\
7248	47.11\\
7249	42.28\\
7250	32.36\\
7252	29.94\\
7254	34.29\\
7255	47\\
7256	54.56\\
7257	58.32\\
7259	56.97\\
7261	54.18\\
7263	53.45\\
7265	52\\
7266	56.42\\
7267	64.1\\
7268	54.91\\
7269	49.62\\
7270	45.92\\
7271	49.97\\
7272	43.83\\
7273	37.59\\
7274	30.64\\
7276	28.83\\
7278	30.64\\
7279	39.96\\
7280	50.03\\
7282	49.11\\
7284	50\\
7286	48.38\\
7288	45\\
7289	49.11\\
7290	58.49\\
7292	46.14\\
7293	49.94\\
7295	44.94\\
7296	36.17\\
7297	27.24\\
7298	19.67\\
7299	17.25\\
7300	15.34\\
7302	13.52\\
7303	16.2\\
7305	20.83\\
7306	30.97\\
7307	33.62\\
7309	33.67\\
7311	32.78\\
7312	34.68\\
7313	39.99\\
7314	50.22\\
7315	63.4\\
7316	51.37\\
7317	42.44\\
7318	37\\
7320	32\\
7321	16.11\\
7323	13.03\\
7324	10.78\\
7326	11.88\\
7328	13.55\\
7330	15.1\\
7332	16.9\\
7333	21\\
7335	14.69\\
7336	21\\
7337	31.18\\
7338	42.44\\
7339	46.25\\
7340	40\\
7341	34.94\\
7342	32.76\\
7343	22.47\\
7344	19.16\\
7346	14.5\\
7347	6.33\\
7348	0.12\\
7349	2.58\\
7350	13.48\\
7351	28.97\\
7352	39.8\\
7353	43.38\\
7354	46.64\\
7356	48.09\\
7358	46.54\\
7359	43.06\\
7361	40.77\\
7362	49.23\\
7363	62.74\\
7364	58.03\\
7365	50.46\\
7366	54.69\\
7367	49.05\\
7368	46.8\\
7369	51.02\\
7370	42.57\\
7371	40.21\\
7372	34.18\\
7374	39.44\\
7375	51.32\\
7376	62.7\\
7377	58.79\\
7379	55.96\\
7380	54.2\\
7381	52\\
7382	49.52\\
7383	45.51\\
7385	48.93\\
7386	57.47\\
7387	62.19\\
7388	63.95\\
7389	57.1\\
7390	44.52\\
7391	46.96\\
7392	43\\
7393	35.81\\
7395	33.91\\
7396	31.65\\
7398	30.71\\
7399	43.36\\
7400	53.75\\
7401	56.83\\
7403	55.31\\
7405	53.52\\
7407	51.15\\
7409	53.5\\
7410	61.3\\
7411	75.05\\
7412	67.79\\
7413	52.43\\
7414	46.78\\
7415	49.41\\
7417	43.3\\
7418	40.63\\
7419	35.08\\
7420	30.21\\
7422	38.74\\
7423	46.31\\
7424	55.66\\
7425	60.56\\
7427	60.42\\
7429	55.41\\
7431	53.5\\
7433	55.98\\
7434	63.72\\
7435	77.92\\
7436	63.5\\
7437	54.92\\
7438	45.94\\
7440	43.3\\
7441	29.88\\
7442	27.86\\
7444	25.91\\
7446	26.7\\
7447	34.76\\
7448	49.04\\
7450	47.31\\
7451	44.96\\
7453	43.3\\
7455	41.37\\
7457	43.3\\
7458	61.3\\
7460	52.44\\
7461	44.96\\
7462	39.89\\
7463	42.15\\
7465	34.94\\
7466	30.33\\
7468	23.15\\
7470	23.43\\
7471	28.75\\
7472	35\\
7473	39.98\\
7474	44.02\\
7476	42.33\\
7477	44.75\\
7478	41.77\\
7479	37.7\\
7481	39.23\\
7482	52.79\\
7484	51.18\\
7485	44\\
7486	38.79\\
7487	41.43\\
7489	38.38\\
7490	30.95\\
7491	20.78\\
7493	21.11\\
7495	22.7\\
7497	23.34\\
7498	26\\
7499	33\\
7500	34.94\\
7501	40.46\\
7502	37.44\\
7503	33.47\\
7504	37.32\\
7505	39.94\\
7506	54.94\\
7507	64.99\\
7508	55\\
7509	48.06\\
7510	41.76\\
7511	44.22\\
7512	40.85\\
7513	38.43\\
7514	28.71\\
7515	25.14\\
7516	29.1\\
7517	24.08\\
7518	29.98\\
7519	42.92\\
7520	50.13\\
7522	50.06\\
7524	49.73\\
7526	48.72\\
7528	48.62\\
7530	53.53\\
7531	60.33\\
7532	53.36\\
7533	44.96\\
7534	37.89\\
7535	40.38\\
7536	34.26\\
7538	28.81\\
7540	25.6\\
7542	27.98\\
7543	31.29\\
7544	41.13\\
7546	42.87\\
7548	44.94\\
7549	43.11\\
7551	43.5\\
7553	57.44\\
7554	60.99\\
7555	57.8\\
7556	51.91\\
7557	55\\
7558	42.36\\
7560	35.38\\
7561	25.14\\
7563	23.74\\
7565	29.66\\
7566	25.85\\
7567	37.01\\
7568	46.68\\
7570	49.77\\
7572	50.58\\
7574	46.98\\
7576	49.15\\
7578	60.7\\
7579	68.13\\
7580	57.81\\
7581	49.51\\
7583	47.44\\
7584	43.13\\
7585	39.96\\
7586	35.13\\
7588	33.32\\
7590	31.71\\
7591	39.97\\
7592	54.09\\
7594	55\\
7596	53.8\\
7597	51.62\\
7599	47.42\\
7601	48.05\\
7602	58.01\\
7604	52.97\\
7605	47.18\\
7606	37.93\\
7607	36.13\\
7609	41.63\\
7611	39.71\\
7612	32.44\\
7613	28.14\\
7615	37.07\\
7616	47.3\\
7618	46.21\\
7620	44.03\\
7622	47.44\\
7623	56.41\\
7625	56.43\\
7626	71.2\\
7628	56.1\\
7629	51.37\\
7631	52.44\\
7632	44\\
7633	35.91\\
7634	39.99\\
7635	37.48\\
7636	35.17\\
7638	33.5\\
7639	37.66\\
7640	39.94\\
7641	49.94\\
7642	44.99\\
7644	50.5\\
7646	45.2\\
7648	43.27\\
7649	45.35\\
7650	68.9\\
7652	57\\
7653	40.44\\
7654	42.21\\
7655	47.44\\
7656	43.91\\
7657	40.44\\
7659	39.16\\
7660	34.99\\
7662	33.69\\
7664	31.31\\
7665	39.16\\
7666	43.75\\
7667	49.9\\
7668	53.49\\
7670	44.94\\
7671	40.71\\
7673	40.66\\
7674	59.96\\
7675	68.9\\
7676	51.78\\
7677	40.79\\
7679	44.75\\
7680	40.83\\
7681	35.17\\
7683	33.43\\
7686	33.21\\
7687	44.94\\
7688	51.32\\
7689	54.29\\
7691	52.43\\
7693	51.98\\
7694	54.15\\
7695	56.25\\
7697	57.41\\
7698	70.11\\
7699	67.52\\
7700	54.99\\
7701	49.78\\
7702	42.07\\
7703	46.57\\
7705	42.69\\
7707	39.03\\
7708	36.55\\
7710	37.17\\
7711	44.83\\
7712	53.5\\
7713	55.5\\
7715	54.76\\
7717	52.88\\
7719	54.5\\
7721	57.02\\
7722	78.61\\
7723	63.3\\
7724	58.3\\
7725	50.89\\
7727	50.8\\
7729	42.02\\
7730	38.43\\
7731	41.43\\
7733	41.42\\
7735	50.5\\
7736	55.19\\
7737	59.94\\
7739	58.75\\
7741	54.41\\
7743	52.98\\
7744	50.5\\
7745	54.96\\
7746	75\\
7747	66.97\\
7748	59.45\\
7749	52.26\\
7750	47.61\\
7751	50.5\\
7753	42.44\\
7755	40.6\\
7758	41.57\\
7759	44.75\\
7760	58.69\\
7762	56.37\\
7763	54.06\\
7764	56.15\\
7765	51\\
7767	49.97\\
7769	56.48\\
7770	75\\
7771	61.47\\
7772	58.34\\
7773	52.74\\
7774	46.85\\
7776	46.4\\
7777	49.99\\
7778	41.74\\
7780	38.63\\
7781	40.76\\
7783	44.36\\
7784	54.94\\
7786	56.37\\
7787	60\\
7788	64\\
7789	53.94\\
7790	56.36\\
7791	50.29\\
7793	53.5\\
7794	65.12\\
7795	55.37\\
7796	52.85\\
7797	50.61\\
7799	53.5\\
7801	40.34\\
7802	35.88\\
7804	35.13\\
7805	32.64\\
7807	31.56\\
7808	39.96\\
7809	44.7\\
7811	52.38\\
7812	54.94\\
7814	44.94\\
7815	50\\
7817	48.94\\
7818	66.4\\
7819	59.94\\
7820	47.39\\
7821	49.99\\
7822	45.23\\
7823	41.92\\
7825	31.94\\
7826	28.91\\
7828	31.87\\
7829	28.12\\
7831	30\\
7833	31\\
7834	36.57\\
7835	41.64\\
7836	50.21\\
7837	57.39\\
7838	46.79\\
7839	42.11\\
7841	52.62\\
7842	69\\
7843	64.94\\
7844	57.39\\
7846	44.5\\
7847	40.46\\
7848	37.93\\
7849	31.69\\
7851	34.2\\
7853	34.62\\
7854	30.34\\
7855	38.11\\
7856	51.34\\
7858	49.44\\
7860	50.32\\
7861	48.5\\
7863	49.95\\
7865	51.51\\
7866	60.16\\
7867	63.34\\
7868	57.47\\
7869	51\\
7870	48.58\\
7872	44.88\\
7873	46.7\\
7874	41.79\\
7876	37.7\\
7877	39.94\\
7878	35.7\\
7879	46.9\\
7880	56.27\\
7882	55.94\\
7883	53.99\\
7885	53.01\\
7887	55\\
7888	57.29\\
7890	69.44\\
7892	57.72\\
7893	52.22\\
7894	47\\
7895	50.7\\
7897	35.33\\
7899	34.72\\
7901	34.45\\
7903	46.1\\
7904	59.99\\
7905	54.49\\
7906	52.5\\
7907	64.63\\
7908	78.9\\
7909	69.01\\
7911	63\\
7912	69.06\\
7913	78.9\\
7914	84.63\\
7915	78.9\\
7916	54.35\\
7917	49.48\\
7918	78.9\\
7920	55.72\\
7921	49.64\\
7922	43.01\\
7923	39.3\\
7924	35.7\\
7926	42.94\\
7927	46.54\\
7928	60\\
7929	57.47\\
7931	58\\
7933	57.92\\
7934	55.5\\
7937	57.26\\
7938	80\\
7939	55.95\\
7941	56\\
7942	50.06\\
7944	44.19\\
7945	47.17\\
7946	42.49\\
7947	39.94\\
7948	35\\
7950	37.08\\
7951	44.12\\
7952	56.33\\
7953	53\\
7954	55.5\\
7957	52.5\\
7959	49.96\\
7961	54.78\\
7962	72.32\\
7963	55.5\\
7964	50.06\\
7965	55\\
7966	50.06\\
7967	47.3\\
7968	44.94\\
7969	42.18\\
7970	37.01\\
7972	32.84\\
7973	30.88\\
7975	31.13\\
7976	38.25\\
7977	43.41\\
7978	46.8\\
7980	48\\
7981	45\\
7983	44.72\\
7985	50.12\\
7986	76.4\\
7988	52.44\\
7989	46.79\\
7990	45\\
7991	47.59\\
7993	37.01\\
7994	34.51\\
7996	34.24\\
7997	30.26\\
7999	31.49\\
8001	31.29\\
8002	40\\
8003	42.31\\
8005	42.41\\
8007	40.12\\
8008	34.94\\
8009	42.44\\
8010	65\\
8011	60.79\\
8012	54.06\\
8013	52\\
8014	42.62\\
8016	32.24\\
8017	41.78\\
8018	34.77\\
8019	38\\
8020	33.74\\
8021	41.67\\
8022	39.16\\
8023	51.7\\
8025	53\\
8028	56.36\\
8030	57\\
8031	53.64\\
8033	52.7\\
8034	60\\
8035	80.74\\
8036	59.71\\
8037	54.46\\
8039	57\\
8041	54.4\\
8042	51.4\\
8045	51.4\\
8047	50\\
8048	57.76\\
8049	64.42\\
8051	64.43\\
8053	60.4\\
8055	58.08\\
8057	61.84\\
8058	77.52\\
8059	68.19\\
8060	60.09\\
8061	54.87\\
8062	48.06\\
8064	45.87\\
8065	49.05\\
8066	46.3\\
8067	42.36\\
8068	31.67\\
8070	35.85\\
8071	48.7\\
8072	63\\
8074	64.9\\
8076	69.8\\
8077	62.3\\
8078	64.62\\
8079	67.76\\
8080	65.66\\
8082	82.87\\
8083	72.72\\
8084	69.91\\
8085	58.96\\
8086	50.6\\
8087	58\\
8089	46.04\\
8091	49.5\\
8092	43.62\\
8093	39.57\\
8094	47.54\\
8095	51.43\\
8096	60.87\\
8097	64\\
8098	67.64\\
8100	67.7\\
8102	67.47\\
8104	65.83\\
8105	72.23\\
8106	87.97\\
8107	73.59\\
8108	64.67\\
8109	57.5\\
8111	54.96\\
8112	49.4\\
8113	57.62\\
8114	54\\
8115	47.99\\
8116	45.17\\
8117	43.36\\
8118	47.16\\
8119	50.91\\
8120	66.64\\
8121	70.41\\
8123	69.77\\
8124	66.39\\
8126	64.39\\
8127	58.67\\
8129	59.98\\
8130	70.61\\
8131	64.9\\
8133	58.64\\
8134	48.04\\
8135	53.96\\
8137	50.1\\
8138	46.49\\
8139	40\\
8141	33.91\\
8143	37.02\\
8144	47.46\\
8145	49.52\\
8147	54.06\\
8148	59.99\\
8149	66.5\\
8150	50.03\\
8152	48.91\\
8154	62.17\\
8155	59.96\\
8156	53.97\\
8157	51.36\\
8158	48.69\\
8160	48.66\\
8161	46.89\\
8162	42.43\\
8164	30.07\\
8166	31.33\\
8167	29.53\\
8168	33.89\\
8170	42.89\\
8171	49.36\\
8172	53.36\\
8173	57.13\\
8174	52.87\\
8175	45.5\\
8176	41.27\\
8177	45\\
8178	60\\
8179	64.5\\
8180	52.72\\
8182	51.4\\
8183	54\\
8184	50\\
8186	46.82\\
8187	42\\
8188	39.14\\
8189	36.16\\
8190	40.78\\
8191	47.2\\
8192	57.53\\
8193	50.95\\
8195	52\\
8196	54.06\\
8198	51.5\\
8201	56.29\\
8202	73\\
8203	61.36\\
8204	56.25\\
8205	51.29\\
8206	45\\
8207	49.99\\
8209	45.45\\
8210	41.37\\
8211	39.53\\
8212	31.52\\
8214	39.85\\
8215	46.77\\
8216	62.52\\
8217	67.41\\
8218	69.85\\
8219	63.71\\
8220	61.28\\
8221	56.84\\
8223	60.06\\
8225	62.57\\
8226	74.69\\
8227	72.42\\
8228	62.32\\
8229	55.87\\
8230	50.24\\
8232	46.92\\
8234	43.2\\
8235	41.21\\
8236	36.27\\
8237	31.09\\
8238	39.49\\
8239	44.39\\
8240	52.78\\
8241	58.71\\
8243	55.1\\
8244	57.96\\
8246	56.84\\
8248	55.93\\
8250	59.72\\
8252	59.68\\
8253	53.05\\
8254	50.45\\
8256	50.73\\
8258	45.34\\
8259	41.06\\
8260	36.53\\
8261	33.2\\
8262	38.12\\
8263	45.17\\
8264	55.16\\
8266	52.02\\
8268	53.01\\
8270	52.59\\
8271	48.88\\
8272	46.96\\
8273	49.76\\
8274	59.99\\
8275	54.7\\
8277	49.45\\
8278	45\\
8280	46.62\\
8282	41.66\\
8283	37.41\\
8284	31.22\\
8286	37.93\\
8287	46.13\\
8288	54.96\\
8289	50.98\\
8291	51.99\\
8292	59.94\\
8293	52.88\\
8294	50.22\\
8296	48.85\\
8297	53.8\\
8298	64.94\\
8299	56.03\\
8300	53.98\\
8301	50.66\\
8302	46.06\\
8303	51.44\\
8304	48.9\\
8305	51.39\\
8306	49.04\\
8307	41.93\\
8309	41.76\\
8310	43.5\\
8312	47.17\\
8314	60.32\\
8315	62.9\\
8317	52.1\\
8319	48.3\\
8320	45.67\\
8322	62.9\\
8323	55.69\\
8324	51.08\\
8325	47.44\\
8326	42.5\\
8327	48.39\\
8329	49.43\\
8330	44.37\\
8331	42.08\\
8332	38.76\\
8333	34.9\\
8335	36.9\\
8336	39.99\\
8338	44.99\\
8339	47.44\\
8340	50.44\\
8341	52.35\\
8343	52.03\\
8344	50.2\\
8346	59.2\\
8348	57.44\\
8349	53.39\\
8350	49.6\\
8351	47.79\\
8353	47.51\\
8354	38.65\\
8355	34.38\\
8356	30.42\\
8358	36.5\\
8359	47.17\\
8360	55.54\\
8361	59.94\\
8363	61.6\\
8365	59.99\\
8367	57.14\\
8368	54.13\\
8369	59.21\\
8370	74.27\\
8371	71.23\\
8372	68\\
8373	57.59\\
8374	51.6\\
8376	50.27\\
8377	47.44\\
8378	43.5\\
8379	41.37\\
8380	33.75\\
8382	36\\
8383	42.07\\
8384	54.75\\
8386	55.96\\
8388	55.64\\
8390	53.74\\
8392	51.42\\
8394	60.99\\
8396	58.07\\
8397	55.3\\
8398	47.46\\
8399	55.3\\
8400	47.12\\
8401	42.72\\
8402	40.44\\
8403	35.83\\
8404	32.35\\
8405	28.51\\
8407	39.06\\
8408	50.5\\
8410	51.02\\
8412	52.44\\
8414	54.3\\
8416	57.39\\
8418	75\\
8419	57.61\\
8420	54.3\\
8421	48.54\\
8422	45\\
8423	51.36\\
8424	47.16\\
8425	41.07\\
8427	33.72\\
8429	29.36\\
8431	39.96\\
8432	51.13\\
8434	63.7\\
8435	61.21\\
8437	46.46\\
8438	52.59\\
8439	42.44\\
8441	42.5\\
8442	49.5\\
8444	48.73\\
8445	41.9\\
8446	40\\
8448	37.43\\
8449	32.82\\
8450	29.15\\
8452	29.49\\
8453	27.43\\
8455	33.69\\
8456	52.25\\
8457	47.39\\
8458	68\\
8460	66.53\\
8462	49.67\\
8463	44.97\\
8464	43.03\\
8465	47.39\\
8466	67.23\\
8467	59.96\\
8468	51.99\\
8470	48.81\\
8472	44.94\\
8473	32.64\\
8474	27.91\\
8476	24.69\\
8477	19.96\\
8478	13.15\\
8479	19.68\\
8481	38.59\\
8482	36.08\\
8483	40\\
8486	40\\
8488	40.18\\
8489	53.38\\
8490	70.5\\
8492	67.21\\
8493	48\\
8494	40.72\\
8496	40.18\\
8498	34.85\\
8500	34.83\\
8502	32.64\\
8503	30.74\\
8505	30.16\\
8506	38.06\\
8507	43.86\\
8509	63.18\\
8510	59.99\\
8511	40.34\\
8513	39.49\\
8514	54.94\\
8515	63.18\\
8516	49.94\\
8517	47.44\\
8518	41.2\\
8520	30\\
8521	16.57\\
8523	12.99\\
8524	6.31\\
8525	3.46\\
8526	13.24\\
8527	19.22\\
8528	40.66\\
8529	44.39\\
8531	45\\
8532	54.94\\
8533	47\\
8534	45\\
8535	54.43\\
8536	48\\
8537	55\\
8538	59.94\\
8539	45.81\\
8541	49.94\\
8543	40.28\\
8545	28.49\\
8547	27.28\\
8549	27\\
8550	23.17\\
8551	32.95\\
8552	40\\
8553	45\\
8555	47.39\\
8557	43.16\\
8559	41.81\\
8561	47.39\\
8562	70\\
8563	47.39\\
8564	45\\
8565	47.39\\
8566	40.32\\
8567	44.94\\
8568	40\\
8569	29.63\\
8571	28.49\\
8573	16.36\\
8574	13.09\\
8575	34.6\\
8576	38.92\\
8577	44.5\\
8578	47.44\\
8580	60\\
8582	56.55\\
8583	49.34\\
8584	44.55\\
8585	47.39\\
8586	55\\
8587	47.39\\
8588	45\\
8590	40.87\\
8591	44.34\\
8592	47.39\\
8593	37.18\\
8594	29\\
8596	28.4\\
8598	26.17\\
8599	18.85\\
8600	15\\
8601	26.03\\
8602	33.26\\
8603	37.83\\
8605	39.69\\
8607	38.45\\
8609	39.99\\
8610	59.25\\
8611	50\\
8614	41.09\\
8616	42.44\\
8617	39.27\\
8618	31.15\\
8621	29.54\\
8622	31.75\\
8624	35.56\\
8625	40.29\\
8626	45\\
8627	49.4\\
8628	53.23\\
8629	49.99\\
8630	52\\
8631	45\\
8632	42.97\\
8633	44.99\\
8634	69.94\\
8635	52.71\\
8636	50.97\\
8637	44.41\\
8638	38.72\\
8639	41.76\\
8640	37.29\\
8641	32\\
8642	27.17\\
8643	22.64\\
8644	28.72\\
8645	20.14\\
8647	29.3\\
8648	26\\
8649	34.22\\
8650	44.94\\
8651	47.97\\
8652	55.5\\
8653	58.2\\
8654	59.94\\
8655	42.44\\
8657	44.94\\
8658	70.61\\
8660	45.39\\
8662	39.96\\
8664	36.64\\
8665	29.99\\
8667	20.73\\
8668	27.44\\
8669	29.54\\
8671	29.87\\
8673	27.51\\
8674	32.83\\
8676	39.94\\
8678	41.54\\
8680	45.12\\
8681	49.94\\
8682	65\\
8684	52.39\\
8685	60\\
8686	48.72\\
8687	51.85\\
8688	66\\
8689	49.95\\
8690	41.25\\
8692	39.05\\
8693	35.05\\
8694	38.73\\
8695	40.74\\
8696	46.66\\
8698	54.07\\
8700	57.05\\
8702	52.74\\
8703	50.6\\
8704	47.31\\
8705	51.88\\
8706	73.2\\
8707	60.12\\
8708	56.97\\
8709	53.8\\
8710	50.14\\
8712	49.79\\
8713	43.36\\
8714	40.06\\
8715	35.27\\
8717	33.71\\
8718	36.22\\
8719	40.25\\
8720	43.7\\
8721	49.04\\
8723	51.3\\
8725	50.84\\
8726	49.03\\
8727	47.06\\
8728	44.4\\
8729	47.53\\
8730	55.01\\
8731	58.15\\
8732	55.54\\
8733	51.63\\
8734	46.39\\
8735	49.73\\
8737	48.76\\
8738	43.43\\
8739	41.19\\
8740	34.94\\
8741	32.32\\
8742	36.08\\
8743	41\\
8744	46.31\\
8746	50.39\\
8747	52.35\\
8749	51.77\\
8751	45.75\\
8752	41.38\\
8754	48.76\\
8755	52.92\\
8757	48.01\\
8758	45.39\\
8759	48.43\\
8760	49.64\\
};
\addlegendentry{Market price  };

\addplot [color=mycolor2,line width=2.0pt,mark size=0.3pt,only marks,mark=*,mark options={solid},forget plot]
  table[row sep=crcr]{%
1	42.895618\\
3	41.952874\\
4	37.710522\\
6	37.954992\\
8	39.773874\\
9	42.895618\\
11	42.224425\\
13	42.895618\\
15	43.173757\\
17	43.713932\\
19	44.584215\\
21	43.173757\\
23	42.224425\\
25	39.773874\\
26	37.954992\\
28	34.354\\
29	37.710522\\
31	41.952874\\
33	44.248252\\
34	46.02029\\
36	47.443191\\
39	47.443191\\
42	47.443191\\
45	47.443191\\
46	45.242233\\
48	45.142689\\
50	41.952874\\
51	39.773874\\
53	39.377892\\
55	43.173757\\
57	44.248252\\
59	44.584215\\
61	45.242233\\
63	45.142689\\
65	47.443191\\
68	47.443191\\
70	46.02029\\
72	44.248252\\
73	42.060469\\
74	39.773874\\
77	39.773874\\
79	44.248252\\
81	45.142689\\
83	44.248252\\
85	45.142689\\
87	42.895618\\
90	45.142689\\
93	44.248252\\
95	43.173757\\
98	42.060469\\
99	39.377892\\
101	39.773874\\
103	44.248252\\
105	44.584215\\
107	44.248252\\
109	44.584215\\
111	42.224425\\
113	44.248252\\
115	45.142689\\
117	43.173757\\
119	42.060469\\
121	41.952874\\
122	39.773874\\
124	39.377892\\
126	42.060469\\
128	44.248252\\
130	44.584215\\
132	45.142689\\
135	45.142689\\
137	45.242233\\
138	47.443191\\
141	47.443191\\
142	45.242233\\
144	43.597553\\
146	39.773874\\
148	39.377892\\
150	39.773874\\
151	43.173757\\
152	45.242233\\
153	48.22564\\
155	49.717343\\
158	48.22564\\
160	49.717343\\
163	49.717343\\
165	49.153013\\
167	47.443191\\
170	42.060469\\
171	39.773874\\
174	42.060469\\
175	47.443191\\
176	49.717343\\
179	49.717343\\
182	51.119017\\
184	52.700017\\
186	53.342665\\
188	54.597218\\
189	52.700017\\
190	49.717343\\
193	47.443191\\
194	44.248252\\
197	44.248252\\
199	47.443191\\
200	49.717343\\
203	49.717343\\
206	49.717343\\
209	49.717343\\
212	49.717343\\
215	48.918059\\
217	47.957016\\
219	43.597553\\
221	44.248252\\
223	47.957016\\
224	49.717343\\
227	49.717343\\
230	49.717343\\
233	49.717343\\
236	49.717343\\
239	49.717343\\
241	47.443191\\
243	42.895618\\
245	42.060469\\
247	45.142689\\
248	47.935057\\
250	49.717343\\
253	49.717343\\
256	49.717343\\
259	49.717343\\
262	49.717343\\
264	48.918059\\
266	47.443191\\
269	42.895618\\
272	43.713932\\
273	46.02029\\
275	47.935057\\
278	47.851616\\
279	43.597553\\
281	47.443191\\
283	48.22564\\
285	47.935057\\
287	47.443191\\
289	43.713932\\
290	39.773874\\
292	39.377892\\
294	39.773874\\
295	44.248252\\
296	47.443191\\
298	47.935057\\
300	47.443191\\
302	47.851616\\
305	47.935057\\
307	49.717343\\
309	48.22564\\
311	47.443191\\
312	44.248252\\
314	47.443191\\
315	42.060469\\
318	42.895618\\
319	47.957016\\
320	49.717343\\
323	49.717343\\
326	49.717343\\
329	49.717343\\
332	49.717343\\
335	49.153013\\
337	47.443191\\
339	45.142689\\
342	44.584215\\
343	47.443191\\
344	49.717343\\
347	49.717343\\
350	49.717343\\
353	49.717343\\
356	49.717343\\
358	47.443191\\
360	47.957016\\
362	42.895618\\
364	41.952874\\
366	42.060469\\
367	47.443191\\
369	47.851616\\
371	49.717343\\
374	49.717343\\
376	48.22564\\
379	49.717343\\
381	48.22564\\
383	47.443191\\
384	43.173757\\
385	45.142689\\
386	41.952874\\
387	39.773874\\
389	41.952874\\
391	45.242233\\
392	47.935057\\
393	49.717343\\
395	48.22564\\
397	47.443191\\
400	47.851616\\
402	49.717343\\
405	48.22564\\
407	47.443191\\
409	45.142689\\
411	42.060469\\
412	39.773874\\
414	41.952874\\
415	45.142689\\
418	45.142689\\
421	44.584215\\
423	44.248252\\
425	43.173757\\
426	45.242233\\
427	47.443191\\
428	45.242233\\
430	45.142689\\
433	44.248252\\
435	39.377892\\
437	37.954992\\
439	43.597553\\
441	44.248252\\
443	45.142689\\
445	45.242233\\
447	43.713932\\
449	43.597553\\
450	47.443191\\
453	47.443191\\
455	45.242233\\
457	42.224425\\
458	39.773874\\
460	39.377892\\
462	42.895618\\
463	46.02029\\
464	47.935057\\
465	49.717343\\
468	49.717343\\
471	49.153013\\
474	49.717343\\
477	49.717343\\
478	47.443191\\
480	46.02029\\
481	44.248252\\
482	42.060469\\
483	37.954992\\
485	39.377892\\
487	46.02029\\
488	47.851616\\
490	49.717343\\
493	49.717343\\
496	49.717343\\
499	49.717343\\
502	48.22564\\
504	47.443191\\
505	44.584215\\
506	41.952874\\
508	39.773874\\
510	42.895618\\
511	47.443191\\
512	49.717343\\
515	49.717343\\
518	49.717343\\
521	49.717343\\
523	52.049319\\
524	50.249467\\
526	49.717343\\
529	47.443191\\
530	42.224425\\
532	41.952874\\
534	42.060469\\
535	47.443191\\
537	49.717343\\
540	49.717343\\
542	48.918059\\
544	48.22564\\
546	49.717343\\
549	49.717343\\
550	47.935057\\
552	47.443191\\
554	42.060469\\
556	41.952874\\
558	44.248252\\
559	47.443191\\
560	49.717343\\
563	49.717343\\
566	49.717343\\
569	49.717343\\
572	49.717343\\
574	47.957016\\
576	47.851616\\
577	45.242233\\
579	44.248252\\
580	41.952874\\
581	39.773874\\
584	43.713932\\
586	45.242233\\
588	47.443191\\
591	45.242233\\
593	45.142689\\
595	47.443191\\
596	45.142689\\
598	44.248252\\
600	43.173757\\
602	41.952874\\
604	39.773874\\
607	42.060469\\
609	44.248252\\
611	45.142689\\
613	45.242233\\
616	45.242233\\
618	45.142689\\
621	45.142689\\
623	43.173757\\
626	42.060469\\
627	39.773874\\
630	41.952874\\
631	45.242233\\
632	47.935057\\
634	49.717343\\
637	49.717343\\
640	49.717343\\
643	49.717343\\
646	49.717343\\
648	48.22564\\
650	43.173757\\
652	41.952874\\
654	42.060469\\
655	47.443191\\
657	49.153013\\
659	48.918059\\
661	49.717343\\
663	48.22564\\
666	49.717343\\
669	49.717343\\
672	48.22564\\
673	44.584215\\
674	42.224425\\
676	42.060469\\
679	47.443191\\
680	49.717343\\
682	48.22564\\
685	48.22564\\
687	47.935057\\
689	48.22564\\
691	49.717343\\
694	48.918059\\
696	47.935057\\
698	43.173757\\
700	42.060469\\
702	45.142689\\
703	47.443191\\
704	49.717343\\
707	49.717343\\
710	49.717343\\
713	49.717343\\
716	49.717343\\
719	49.153013\\
721	47.443191\\
722	45.142689\\
723	42.895618\\
725	42.060469\\
727	47.935057\\
728	49.717343\\
731	49.717343\\
733	48.22564\\
735	47.851616\\
737	48.22564\\
739	49.717343\\
742	47.443191\\
744	43.21201\\
745	40.60801\\
747	38.309342\\
748	36.226675\\
750	35.866009\\
752	40.60801\\
753	43.21201\\
756	43.58401\\
758	43.21201\\
761	43.21201\\
763	43.58401\\
765	43.21201\\
767	41.207343\\
769	41.116676\\
772	38.309342\\
774	38.211342\\
776	41.116676\\
779	41.116676\\
782	41.116676\\
784	38.458676\\
786	41.91601\\
788	43.21201\\
791	43.21201\\
793	41.207343\\
795	38.309342\\
798	38.309342\\
799	43.21201\\
801	43.924677\\
804	43.58401\\
806	43.21201\\
809	43.58401\\
811	45.283344\\
813	44.769344\\
815	43.924677\\
817	43.21201\\
818	40.30201\\
819	38.309342\\
822	41.116676\\
823	43.58401\\
825	45.283344\\
828	45.283344\\
830	43.21201\\
833	43.21201\\
835	45.283344\\
837	43.924677\\
839	43.21201\\
841	41.116676\\
843	38.211342\\
844	36.226675\\
845	38.211342\\
847	43.21201\\
848	45.283344\\
851	45.283344\\
854	45.283344\\
856	43.68001\\
858	45.283344\\
861	45.283344\\
864	45.283344\\
865	39.323343\\
867	38.309342\\
870	40.30201\\
871	43.21201\\
873	43.924677\\
876	44.769344\\
878	43.924677\\
880	43.21201\\
882	45.283344\\
885	43.924677\\
887	43.21201\\
888	41.207343\\
890	38.309342\\
893	38.309342\\
895	41.116676\\
896	43.21201\\
898	43.58401\\
900	43.68001\\
903	43.66001\\
905	43.21201\\
907	43.66001\\
909	43.21201\\
912	43.21201\\
913	41.116676\\
915	38.309342\\
917	38.211342\\
918	36.226675\\
920	40.30201\\
922	41.116676\\
924	41.207343\\
927	41.116676\\
928	38.309342\\
930	40.60801\\
932	41.207343\\
934	40.60801\\
936	40.30201\\
938	39.323343\\
940	38.309342\\
943	38.309342\\
944	40.60801\\
947	39.815343\\
949	39.070009\\
951	38.309342\\
953	38.458676\\
954	40.30201\\
956	41.207343\\
959	41.207343\\
961	38.458676\\
963	38.211342\\
965	41.116676\\
966	38.309342\\
967	43.21201\\
969	43.58401\\
971	45.283344\\
974	43.924677\\
977	44.555344\\
979	45.283344\\
981	43.924677\\
983	43.21201\\
985	41.116676\\
986	38.309342\\
988	38.211342\\
990	38.458676\\
991	43.21201\\
992	45.283344\\
994	43.924677\\
997	43.66001\\
999	43.21201\\
1001	43.66001\\
1003	45.283344\\
1006	43.924677\\
1008	43.21201\\
1009	39.709343\\
1011	38.309342\\
1014	41.116676\\
1015	43.21201\\
1016	45.283344\\
1019	43.924677\\
1021	43.68001\\
1023	43.21201\\
1026	45.283344\\
1029	45.283344\\
1031	43.21201\\
1033	39.070009\\
1035	38.211342\\
1036	36.226675\\
1038	41.116676\\
1039	43.21201\\
1040	45.283344\\
1043	45.283344\\
1046	45.283344\\
1049	45.283344\\
1052	45.283344\\
1055	43.66001\\
1057	39.709343\\
1059	38.309342\\
1062	41.116676\\
1063	43.21201\\
1065	43.924677\\
1067	43.68001\\
1070	43.66001\\
1072	43.21201\\
1074	43.924677\\
1076	43.21201\\
1077	41.207343\\
1079	39.070009\\
1081	39.323343\\
1083	36.226675\\
1085	38.211342\\
1088	41.116676\\
1090	40.60801\\
1092	40.30201\\
1094	39.815343\\
1096	38.309342\\
1098	40.30201\\
1100	41.207343\\
1102	40.30201\\
1104	41.116676\\
1105	39.323343\\
1107	38.211342\\
1109	36.226675\\
1111	38.211342\\
1113	39.815343\\
1115	39.323343\\
1117	39.815343\\
1119	39.070009\\
1122	39.070009\\
1123	41.116676\\
1125	43.21201\\
1126	41.207343\\
1128	41.116676\\
1130	38.211342\\
1131	36.226675\\
1134	36.226675\\
1135	41.207343\\
1136	43.66001\\
1138	43.924677\\
1140	43.21201\\
1143	43.21201\\
1144	41.207343\\
1145	43.21201\\
1147	45.283344\\
1150	43.924677\\
1152	43.21201\\
1153	41.116676\\
1154	38.309342\\
1156	38.211342\\
1158	39.709343\\
1159	43.21201\\
1160	45.283344\\
1163	45.283344\\
1166	45.283344\\
1169	45.283344\\
1172	45.283344\\
1174	43.924677\\
1176	43.58401\\
1177	41.116676\\
1178	38.309342\\
1181	38.309342\\
1182	41.116676\\
1183	43.21201\\
1184	45.283344\\
1187	45.283344\\
1190	43.924677\\
1192	44.555344\\
1194	45.283344\\
1197	45.283344\\
1199	43.66001\\
1201	41.116676\\
1202	36.226675\\
1205	36.226675\\
1206	40.30201\\
1208	43.21201\\
1211	43.21201\\
1214	43.21201\\
1216	41.91601\\
1218	43.21201\\
1220	43.58401\\
1222	43.21201\\
1224	41.116676\\
1225	39.070009\\
1227	36.226675\\
1230	40.30201\\
1232	43.21201\\
1235	43.21201\\
1237	41.116676\\
1239	40.60801\\
1241	41.116676\\
1242	43.21201\\
1244	43.58401\\
1246	43.21201\\
1248	41.116676\\
1250	39.070009\\
1252	35.866009\\
1254	36.226675\\
1256	40.30201\\
1258	41.116676\\
1260	40.30201\\
1262	39.070009\\
1265	39.709343\\
1267	43.21201\\
1270	41.116676\\
1272	40.30201\\
1274	39.070009\\
1276	35.866009\\
1278	36.226675\\
1280	39.709343\\
1282	39.070009\\
1284	38.309342\\
1285	36.226675\\
1286	34.347342\\
1289	36.226675\\
1290	38.458676\\
1292	40.30201\\
1294	39.815343\\
1296	39.070009\\
1298	36.226675\\
1300	34.570008\\
1302	39.070009\\
1303	41.116676\\
1304	43.21201\\
1307	41.116676\\
1309	38.458676\\
1311	38.309342\\
1313	39.323343\\
1314	41.116676\\
1315	43.21201\\
1318	43.21201\\
1319	41.207343\\
1321	40.30201\\
1322	38.309342\\
1323	36.226675\\
1326	36.226675\\
1327	41.116676\\
1328	43.21201\\
1330	43.68001\\
1333	43.21201\\
1336	43.21201\\
1338	43.58401\\
1340	44.769344\\
1342	43.21201\\
1345	39.070009\\
1347	38.309342\\
1350	41.116676\\
1351	43.21201\\
1352	45.283344\\
1354	43.924677\\
1357	43.58401\\
1359	43.21201\\
1361	43.924677\\
1363	45.283344\\
1366	45.283344\\
1368	43.21201\\
1369	40.30201\\
1370	38.309342\\
1373	38.309342\\
1375	43.21201\\
1377	43.924677\\
1379	45.283344\\
1382	45.283344\\
1384	44.555344\\
1386	45.283344\\
1389	45.283344\\
1391	44.555344\\
1393	41.207343\\
1394	38.309342\\
1397	38.309342\\
1398	43.21201\\
1400	45.283344\\
1403	45.283344\\
1406	45.283344\\
1409	45.283344\\
1412	45.283344\\
1415	45.283344\\
1416	43.056901\\
1417	41.087409\\
1418	39.095096\\
1419	37.149057\\
1422	36.567781\\
1423	41.087409\\
1425	41.513382\\
1427	41.441119\\
1430	41.087409\\
1432	41.441119\\
1434	41.532399\\
1436	41.765036\\
1438	41.087409\\
1441	41.087409\\
1443	39.095096\\
1444	36.567781\\
1446	37.389935\\
1448	39.181305\\
1450	37.857745\\
1452	36.42579\\
1454	34.354\\
1456	36.332608\\
1458	38.320484\\
1460	41.087409\\
1462	39.181305\\
1465	39.095096\\
1466	36.332608\\
1468	34.445521\\
1470	37.389935\\
1471	41.087409\\
1473	43.056901\\
1475	41.765036\\
1477	41.087409\\
1480	41.087409\\
1482	41.513382\\
1484	43.056901\\
1486	41.765036\\
1489	39.095096\\
1490	37.149057\\
1492	36.42579\\
1494	39.095096\\
1495	41.765036\\
1497	43.056901\\
1500	43.056901\\
1502	41.765036\\
1504	42.364695\\
1506	43.056901\\
1509	43.056901\\
1512	43.056901\\
1514	41.532399\\
1515	39.181305\\
1517	38.611439\\
1519	43.056901\\
1522	43.056901\\
1525	43.056901\\
1527	41.765036\\
1529	41.532399\\
1531	43.056901\\
1534	43.056901\\
1537	41.513382\\
1539	38.320484\\
1541	37.149057\\
1543	41.765036\\
1545	43.056901\\
1547	41.532399\\
1549	41.087409\\
1552	41.087409\\
1554	41.441119\\
1556	43.056901\\
1558	42.364695\\
1560	41.765036\\
1562	36.567781\\
1564	36.42579\\
1566	37.389935\\
1567	41.765036\\
1569	43.056901\\
1571	41.765036\\
1573	41.513382\\
1576	41.087409\\
1578	41.765036\\
1580	43.056901\\
1583	43.056901\\
1585	41.087409\\
1587	39.181305\\
1588	37.389935\\
1591	38.611439\\
1593	38.320484\\
1595	37.389935\\
1597	36.567781\\
1599	36.42579\\
1602	36.42579\\
1603	38.320484\\
1606	37.857745\\
1608	37.389935\\
1610	36.332608\\
1612	36.42579\\
1614	37.389935\\
1616	39.095096\\
1618	38.320484\\
1620	36.42579\\
1622	34.102587\\
1624	36.42579\\
1626	37.149057\\
1628	38.320484\\
1631	38.320484\\
1633	36.42579\\
1634	34.445521\\
1637	34.445521\\
1638	38.320484\\
1639	41.087409\\
1642	41.087409\\
1644	41.513382\\
1646	41.087409\\
1649	41.087409\\
1651	43.056901\\
1654	41.513382\\
1656	41.087409\\
1657	37.756957\\
1659	36.42579\\
1662	39.095096\\
1663	41.087409\\
1666	41.513382\\
1668	41.441119\\
1670	39.855129\\
1672	38.320484\\
1674	39.181305\\
1675	41.087409\\
1678	41.087409\\
1681	39.095096\\
1682	36.42579\\
1685	36.42579\\
1687	41.087409\\
1689	41.513382\\
1691	41.087409\\
1694	41.087409\\
1695	39.095096\\
1697	39.181305\\
1698	41.087409\\
1700	43.056901\\
1702	41.513382\\
1704	41.087409\\
1705	37.389935\\
1707	36.42579\\
1709	36.567781\\
1710	39.095096\\
1711	41.087409\\
1713	41.765036\\
1715	42.364695\\
1717	41.765036\\
1719	41.532399\\
1721	41.441119\\
1723	43.056901\\
1726	41.087409\\
1729	39.095096\\
1730	36.42579\\
1731	34.445521\\
1734	38.320484\\
1735	41.087409\\
1738	41.087409\\
1739	39.095096\\
1741	38.611439\\
1743	37.389935\\
1745	39.095096\\
1747	41.087409\\
1749	39.095096\\
1750	41.087409\\
1752	39.095096\\
1754	37.389935\\
1756	36.567781\\
1758	36.42579\\
1760	37.756957\\
1762	38.320484\\
1765	38.611439\\
1767	36.567781\\
1770	37.149057\\
1772	39.095096\\
1774	37.857745\\
1776	37.389935\\
1778	34.354\\
1780	34.445521\\
1782	37.389935\\
1784	37.857745\\
1786	38.611439\\
1788	39.095096\\
1790	36.42579\\
1791	34.445521\\
1794	36.332608\\
1796	38.320484\\
1798	37.756957\\
1800	37.149057\\
1801	34.445521\\
1803	34.354\\
1805	34.445521\\
1806	37.756957\\
1808	41.087409\\
1810	41.513382\\
1812	41.765036\\
1814	41.087409\\
1817	41.087409\\
1819	43.056901\\
1822	41.765036\\
1824	41.087409\\
1825	38.320484\\
1826	36.332608\\
1829	34.445521\\
1830	36.332608\\
1831	41.087409\\
1833	42.364695\\
1835	42.568173\\
1837	41.532399\\
1839	41.513382\\
1842	41.513382\\
1844	42.364695\\
1846	41.087409\\
1848	39.181305\\
1850	36.332608\\
1851	34.445521\\
1854	37.389935\\
1855	39.181305\\
1856	41.087409\\
1859	43.056901\\
1862	43.056901\\
1864	41.532399\\
1866	43.056901\\
1869	43.056901\\
1871	41.441119\\
1873	41.087409\\
1874	37.389935\\
1876	37.149057\\
1878	36.42579\\
1880	39.095096\\
1882	38.611439\\
1885	38.320484\\
1886	36.42579\\
1889	37.149057\\
1891	39.095096\\
1892	41.087409\\
1895	39.095096\\
1897	36.42579\\
1899	34.445521\\
1902	37.149057\\
1903	39.095096\\
1904	41.087409\\
1906	43.056901\\
1908	41.765036\\
1910	41.532399\\
1912	41.087409\\
1915	41.513382\\
1917	42.364695\\
1919	41.087409\\
1921	38.320484\\
1923	34.445521\\
1926	36.332608\\
1927	34.445521\\
1928	37.149057\\
1931	37.389935\\
1933	37.756957\\
1935	36.567781\\
1937	37.389935\\
1939	38.320484\\
1941	39.095096\\
1943	38.320484\\
1945	38.611439\\
1947	36.42579\\
1949	36.332608\\
1951	37.149057\\
1953	38.611439\\
1955	39.095096\\
1957	39.181305\\
1958	36.567781\\
1960	36.42579\\
1962	37.857745\\
1964	39.095096\\
1967	38.320484\\
1969	37.389935\\
1970	34.445521\\
1972	34.354\\
1974	36.42579\\
1975	39.181305\\
1976	41.087409\\
1978	41.765036\\
1980	41.441119\\
1983	41.087409\\
1986	41.087409\\
1988	43.056901\\
1990	41.532399\\
1992	41.087409\\
1995	38.611439\\
1997	38.320484\\
1999	41.532399\\
2001	43.056901\\
2003	41.513382\\
2005	41.765036\\
2007	41.513382\\
2009	41.765036\\
2011	43.056901\\
2014	41.532399\\
2016	41.087409\\
2018	38.320484\\
2019	36.42579\\
2022	39.095096\\
2023	41.087409\\
2025	41.765036\\
2027	41.513382\\
2030	43.056901\\
2033	43.056901\\
2035	43.517739\\
2036	57.004\\
2037	53.784203\\
2038	47.283039\\
2040	47.275432\\
2041	41.532399\\
2043	37.389935\\
2045	37.149057\\
2046	41.087409\\
2048	43.056901\\
2051	43.056901\\
2053	42.364695\\
2055	43.056901\\
2058	43.056901\\
2060	45.639999\\
2062	43.056901\\
2065	41.532399\\
2067	39.095096\\
2069	37.756957\\
2071	43.056901\\
2074	43.056901\\
2077	43.056901\\
2080	43.056901\\
2083	43.056901\\
2086	43.056901\\
2088	42.364695\\
2090	39.095096\\
2092	37.389935\\
2094	39.095096\\
2096	41.087409\\
2099	41.087409\\
2102	39.855129\\
2105	39.181305\\
2106	41.087409\\
2108	41.441119\\
2110	41.087409\\
2112	39.181305\\
2114	38.320484\\
2116	37.857745\\
2118	37.756957\\
2120	41.087409\\
2123	41.087409\\
2126	41.087409\\
2129	41.087409\\
2131	41.513382\\
2134	41.087409\\
2136	43.056901\\
2137	41.087409\\
2139	39.095096\\
2141	41.441119\\
2143	43.056901\\
2146	43.056901\\
2149	43.056901\\
2152	43.056901\\
2155	44.270799\\
2157	43.056901\\
2160	36.820227\\
2161	35.034828\\
2164	36.820227\\
2166	38.585176\\
2169	38.585176\\
2172	38.585176\\
2174	37.964859\\
2176	38.585176\\
2179	38.585176\\
2182	38.585176\\
2184	36.820227\\
2186	34.340665\\
2188	33.506759\\
2189	36.820227\\
2191	38.585176\\
2194	38.585176\\
2197	38.585176\\
2200	38.585176\\
2203	38.585176\\
2205	39.902497\\
2207	38.585176\\
2208	35.715927\\
2210	33.925984\\
2212	33.425411\\
2214	36.820227\\
2215	38.585176\\
2218	37.427479\\
2220	37.219002\\
2222	37.427479\\
2225	38.585176\\
2228	38.585176\\
2231	37.964859\\
2233	36.820227\\
2234	34.354\\
2236	35.034828\\
2238	38.585176\\
2241	38.585176\\
2244	38.585176\\
2247	38.585176\\
2250	38.585176\\
2253	38.585176\\
2256	36.820227\\
2257	35.034828\\
2259	34.340665\\
2261	34.601402\\
2262	36.820227\\
2265	37.201961\\
2267	36.820227\\
2270	36.820227\\
2273	36.820227\\
2276	36.820227\\
2279	36.820227\\
2281	35.112084\\
2283	34.340665\\
2285	33.925984\\
2286	36.820227\\
2289	36.820227\\
2292	36.820227\\
2295	36.820227\\
2297	37.219002\\
2300	37.201961\\
2303	36.820227\\
2304	35.034828\\
2306	33.425411\\
2308	33.506759\\
2310	36.820227\\
2312	38.585176\\
2315	38.585176\\
2318	37.427479\\
2321	37.427479\\
2323	38.585176\\
2326	37.427479\\
2328	36.820227\\
2330	35.034828\\
2332	35.112084\\
2334	37.427479\\
2336	38.585176\\
2339	37.964859\\
2342	37.427479\\
2344	37.964859\\
2346	37.427479\\
2348	38.585176\\
2351	38.147205\\
2353	36.820227\\
2354	35.034828\\
2357	36.820227\\
2358	38.585176\\
2361	38.585176\\
2364	38.585176\\
2367	37.201961\\
2369	37.427479\\
2371	38.585176\\
2374	38.585176\\
2376	36.820227\\
2378	35.034828\\
2381	36.820227\\
2382	38.585176\\
2385	38.585176\\
2388	37.964859\\
2390	37.427479\\
2393	38.585176\\
2396	38.585176\\
2399	38.585176\\
2400	36.820227\\
2402	35.034828\\
2405	36.820227\\
2407	38.585176\\
2410	38.585176\\
2412	38.147205\\
2414	37.427479\\
2416	38.585176\\
2419	38.585176\\
2422	38.585176\\
2424	36.820227\\
2426	35.034828\\
2428	34.340665\\
2430	35.034828\\
2432	36.820227\\
2435	36.820227\\
2437	35.112084\\
2439	36.820227\\
2442	36.820227\\
2445	36.820227\\
2447	35.112084\\
2448	32.642746\\
2450	32.559242\\
2452	30.560823\\
2454	30.868141\\
2456	32.769991\\
2458	33.506759\\
2460	34.340665\\
2462	33.290898\\
2464	33.425411\\
2466	34.354\\
2468	35.034828\\
2471	34.354\\
2473	33.290898\\
2474	30.868141\\
2477	32.559242\\
2479	35.112084\\
2481	36.820227\\
2484	36.820227\\
2486	35.112084\\
2488	35.034828\\
2490	35.112084\\
2492	36.820227\\
2495	36.820227\\
2496	35.034828\\
2498	33.506759\\
2500	34.340665\\
2502	36.820227\\
2504	37.219002\\
2506	36.820227\\
2509	36.820227\\
2512	36.820227\\
2514	37.427479\\
2517	37.427479\\
2519	36.820227\\
2521	35.034828\\
2523	33.506759\\
2525	35.034828\\
2526	36.820227\\
2528	37.219002\\
2530	37.137202\\
2533	37.137202\\
2535	36.820227\\
2538	36.820227\\
2541	37.201961\\
2543	36.820227\\
2545	34.340665\\
2547	33.425411\\
2549	33.506759\\
2551	36.820227\\
2553	37.137202\\
2555	36.820227\\
2558	36.820227\\
2561	36.820227\\
2563	37.427479\\
2565	37.137202\\
2567	36.820227\\
2569	34.340665\\
2571	33.506759\\
2573	33.835663\\
2575	36.820227\\
2578	36.820227\\
2581	36.820227\\
2584	36.820227\\
2587	36.820227\\
2590	36.820227\\
2592	35.034828\\
2594	33.835663\\
2596	33.506759\\
2598	34.340665\\
2600	35.034828\\
2602	35.325495\\
2604	36.820227\\
2605	35.034828\\
2607	33.425411\\
2609	35.034828\\
2610	36.820227\\
2613	36.820227\\
2615	35.112084\\
2617	34.354\\
2619	32.642746\\
2622	33.506759\\
2624	35.034828\\
2627	34.354\\
2629	34.601402\\
2631	34.340665\\
2633	34.354\\
2635	34.340665\\
2637	35.034828\\
2639	34.354\\
2641	33.290898\\
2643	32.642746\\
2646	33.425411\\
2648	35.034828\\
2651	34.354\\
2654	35.112084\\
2656	36.820227\\
2659	36.820227\\
2662	36.820227\\
2664	34.354\\
2666	32.642746\\
2668	33.290898\\
2670	36.820227\\
2672	37.219002\\
2674	36.820227\\
2677	36.820227\\
2680	36.820227\\
2683	36.820227\\
2686	36.820227\\
2688	35.112084\\
2690	33.506759\\
2693	34.354\\
2694	37.201961\\
2696	38.585176\\
2698	37.964859\\
2700	37.427479\\
2703	37.427479\\
2705	38.585176\\
2707	37.427479\\
2709	38.585176\\
2711	37.219002\\
2713	34.354\\
2715	33.425411\\
2717	35.034828\\
2718	36.820227\\
2720	38.585176\\
2723	38.585176\\
2726	37.964859\\
2728	38.147205\\
2730	37.964859\\
2733	38.585176\\
2735	36.820227\\
2736	35.034828\\
2738	33.425411\\
2741	33.506759\\
2742	36.820227\\
2744	37.427479\\
2746	37.219002\\
2749	37.201961\\
2751	37.427479\\
2754	38.585176\\
2756	37.427479\\
2759	37.137202\\
2761	35.034828\\
2763	33.506759\\
2765	33.425411\\
2767	35.034828\\
2768	36.820227\\
2771	36.820227\\
2773	35.325495\\
2775	36.820227\\
2778	36.820227\\
2781	36.820227\\
2782	35.034828\\
2784	33.506759\\
2786	32.642746\\
2788	32.559242\\
2790	32.769991\\
2792	34.340665\\
2794	34.601402\\
2796	35.034828\\
2798	34.340665\\
2800	34.354\\
2802	36.820227\\
2805	36.820227\\
2807	35.034828\\
2809	32.769991\\
2811	32.642746\\
2813	33.425411\\
2814	36.820227\\
2815	38.585176\\
2818	38.585176\\
2821	38.585176\\
2824	38.585176\\
2827	38.585176\\
2830	37.964859\\
2832	36.820227\\
2833	33.925984\\
2835	33.290898\\
2837	33.425411\\
2838	37.201961\\
2840	38.585176\\
2843	38.585176\\
2846	37.427479\\
2848	37.201961\\
2850	36.820227\\
2853	36.820227\\
2856	35.034828\\
2858	33.425411\\
2860	32.769991\\
2862	36.820227\\
2864	38.585176\\
2867	38.585176\\
2869	37.201961\\
2871	36.820227\\
2874	36.820227\\
2877	36.820227\\
2880	32.807673\\
2882	30.567655\\
2885	31.376742\\
2887	32.807673\\
2889	33.425411\\
2891	34.479574\\
2894	34.479574\\
2897	34.479574\\
2900	34.479574\\
2903	34.479574\\
2905	32.157637\\
2907	30.68681\\
2909	31.376742\\
2911	33.445474\\
2913	34.479574\\
2915	33.425411\\
2917	32.807673\\
2919	32.157637\\
2921	33.425411\\
2923	34.354\\
2925	33.425411\\
2927	32.880017\\
2929	31.376742\\
2931	28.90586\\
2933	30.489459\\
2935	32.157637\\
2937	32.880017\\
2939	32.807673\\
2942	32.807673\\
2944	32.401799\\
2946	32.880017\\
2948	33.425411\\
2950	32.880017\\
2952	32.157637\\
2954	31.174603\\
2956	30.567655\\
2958	28.90586\\
2959	31.769317\\
2961	32.401799\\
2963	31.684738\\
2965	31.376742\\
2967	30.68681\\
2969	32.157637\\
2971	33.425411\\
2974	32.880017\\
2976	31.174603\\
2978	28.90586\\
2980	30.567655\\
2982	32.807673\\
2984	34.354\\
2986	33.425411\\
2988	32.880017\\
2990	32.807673\\
2993	33.425411\\
2995	34.354\\
2997	33.425411\\
2999	32.807673\\
3000	30.567655\\
3002	28.90586\\
3004	30.489459\\
3006	32.401799\\
3007	34.354\\
3009	34.479574\\
3011	34.776399\\
3013	34.479574\\
3016	34.354\\
3018	34.479574\\
3021	34.479574\\
3023	33.425411\\
3025	32.157637\\
3027	31.376742\\
3029	31.769317\\
3031	34.479574\\
3034	34.479574\\
3037	34.837041\\
3039	34.479574\\
3042	34.479574\\
3045	34.479574\\
3048	32.880017\\
3050	30.68681\\
3052	32.157637\\
3054	34.354\\
3057	32.880017\\
3059	34.479574\\
3062	34.479574\\
3065	34.479574\\
3068	34.479574\\
3070	34.354\\
3072	32.157637\\
3075	31.376742\\
3077	31.684738\\
3078	33.425411\\
3080	34.479574\\
3083	34.479574\\
3085	33.425411\\
3088	33.425411\\
3090	34.354\\
3092	32.880017\\
3094	34.479574\\
3096	34.354\\
3098	31.684738\\
3100	30.567655\\
3102	32.807673\\
3104	34.479574\\
3107	34.479574\\
3109	33.445474\\
3111	33.425411\\
3113	32.807673\\
3116	31.769317\\
3118	31.684738\\
3120	30.68681\\
3122	28.90586\\
3125	30.489459\\
3126	27.583979\\
3128	30.567655\\
3131	30.567655\\
3133	30.489459\\
3134	27.583979\\
3137	30.567655\\
3139	31.376742\\
3141	30.68681\\
3143	30.567655\\
3145	28.618079\\
3148	28.90586\\
3150	31.376742\\
3152	33.425411\\
3154	32.807673\\
3156	32.880017\\
3159	32.880017\\
3161	34.354\\
3163	34.479574\\
3166	34.479574\\
3168	30.567655\\
3170	28.618079\\
3173	31.174603\\
3175	33.425411\\
3177	34.479574\\
3179	34.837041\\
3182	34.479574\\
3185	34.479574\\
3187	34.354\\
3189	33.425411\\
3192	31.769317\\
3194	30.567655\\
3195	28.618079\\
3197	30.567655\\
3199	33.445474\\
3201	32.880017\\
3203	33.445474\\
3205	34.354\\
3207	32.880017\\
3210	32.807673\\
3213	32.807673\\
3216	32.157637\\
3218	28.90586\\
3221	31.376742\\
3223	34.479574\\
3226	34.479574\\
3228	34.354\\
3230	32.880017\\
3233	32.880017\\
3236	32.807673\\
3238	33.425411\\
3240	31.376742\\
3242	28.90586\\
3245	31.174603\\
3247	33.425411\\
3250	33.425411\\
3252	32.880017\\
3255	32.807673\\
3257	32.880017\\
3259	33.425411\\
3261	32.807673\\
3263	32.157637\\
3265	32.401799\\
3266	30.489459\\
3268	28.90586\\
3270	32.807673\\
3272	34.479574\\
3275	34.354\\
3277	32.807673\\
3280	32.880017\\
3282	34.354\\
3285	34.354\\
3287	32.880017\\
3289	31.174603\\
3291	30.567655\\
3293	28.90586\\
3294	30.68681\\
3296	32.157637\\
3298	31.769317\\
3300	31.174603\\
3302	30.567655\\
3305	31.684738\\
3307	32.880017\\
3309	32.807673\\
3311	32.157637\\
3313	30.489459\\
3315	28.618079\\
3317	30.567655\\
3319	33.425411\\
3322	33.425411\\
3324	33.445474\\
3326	34.354\\
3328	33.425411\\
3331	33.425411\\
3333	32.401799\\
3335	32.157637\\
3337	31.174603\\
3339	30.567655\\
3341	31.174603\\
3342	33.425411\\
3344	34.479574\\
3346	34.354\\
3349	34.354\\
3351	34.479574\\
3354	34.479574\\
3357	34.479574\\
3359	34.354\\
3360	32.157637\\
3362	30.567655\\
3364	30.489459\\
3366	32.807673\\
3368	34.479574\\
3371	34.479574\\
3374	34.479574\\
3377	34.479574\\
3380	34.479574\\
3383	34.479574\\
3384	32.157637\\
3386	28.90586\\
3388	28.618079\\
3389	30.489459\\
3391	32.157637\\
3393	32.807673\\
3395	31.174603\\
3397	31.684738\\
3399	31.376742\\
3401	32.880017\\
3404	32.401799\\
3406	32.807673\\
3408	31.174603\\
3409	28.90586\\
3412	28.90586\\
3414	31.174603\\
3416	34.354\\
3418	33.425411\\
3420	34.354\\
3422	33.425411\\
3424	34.354\\
3427	34.354\\
3429	32.807673\\
3431	32.401799\\
3433	31.174603\\
3435	30.489459\\
3437	28.90586\\
3439	30.567655\\
3441	32.157637\\
3443	31.376742\\
3445	31.174603\\
3447	30.567655\\
3449	32.807673\\
3451	33.425411\\
3453	32.807673\\
3455	31.684738\\
3457	30.567655\\
3459	30.489459\\
3461	28.90586\\
3463	30.567655\\
3465	31.684738\\
3467	31.174603\\
3469	30.68681\\
3471	31.174603\\
3473	32.807673\\
3475	33.425411\\
3477	32.880017\\
3479	32.807673\\
3481	30.567655\\
3483	30.489459\\
3485	30.567655\\
3487	33.425411\\
3489	34.479574\\
3492	34.479574\\
3494	33.425411\\
3497	33.445474\\
3499	32.880017\\
3501	31.174603\\
3504	30.68681\\
3506	28.90586\\
3508	28.618079\\
3510	30.567655\\
3511	32.880017\\
3513	34.479574\\
3516	34.479574\\
3519	34.479574\\
3522	34.479574\\
3524	32.807673\\
3527	31.376742\\
3529	30.567655\\
3531	31.174603\\
3533	30.567655\\
3535	33.425411\\
3537	34.479574\\
3540	34.479574\\
3543	34.479574\\
3546	34.479574\\
3548	33.425411\\
3551	32.807673\\
3553	31.376742\\
3555	30.567655\\
3557	28.90586\\
3559	30.567655\\
3561	31.376742\\
3563	31.769317\\
3566	31.376742\\
3569	32.157637\\
3571	32.401799\\
3573	32.157637\\
3575	31.769317\\
3577	30.567655\\
3579	28.618079\\
3581	28.90586\\
3582	31.174603\\
3584	32.807673\\
3587	33.425411\\
3590	32.807673\\
3593	33.425411\\
3595	34.479574\\
3597	33.425411\\
3599	32.880017\\
3601	30.567655\\
3603	30.489459\\
3606	30.567655\\
3608	32.807673\\
3610	33.425411\\
3612	34.354\\
3614	32.401799\\
3616	32.807673\\
3618	33.425411\\
3621	32.880017\\
3624	28.361966\\
3626	27.630618\\
3629	27.630618\\
3631	27.738325\\
3633	29.655408\\
3635	29.067829\\
3637	28.361966\\
3639	27.630618\\
3641	29.655408\\
3643	31.166668\\
3646	31.166668\\
3648	28.716821\\
3650	27.630618\\
3653	27.738325\\
3655	31.166668\\
3657	31.434973\\
3659	31.489788\\
3661	31.166668\\
3664	31.166668\\
3667	31.166668\\
3670	31.166668\\
3673	28.640368\\
3675	27.738325\\
3677	28.361966\\
3678	31.166668\\
3680	32.660618\\
3682	31.680679\\
3684	31.166668\\
3686	31.434973\\
3689	31.680679\\
3691	31.504213\\
3693	31.166668\\
3696	29.067829\\
3698	27.738325\\
3700	28.361966\\
3702	31.166668\\
3704	32.660618\\
3706	33.425411\\
3709	32.660618\\
3712	32.660618\\
3715	32.660618\\
3717	31.166668\\
3719	28.361966\\
3721	27.630618\\
3723	26.128494\\
3725	27.630618\\
3727	31.166668\\
3730	31.166668\\
3732	29.655408\\
3735	29.655408\\
3737	30.231927\\
3739	31.166668\\
3741	29.655408\\
3744	28.716821\\
3746	27.630618\\
3749	28.179249\\
3751	31.166668\\
3753	31.504213\\
3755	31.489788\\
3758	31.166668\\
3761	31.166668\\
3764	31.166668\\
3766	29.720801\\
3768	30.231927\\
3770	28.640368\\
3772	27.738325\\
3774	28.179249\\
3776	29.720801\\
3778	31.166668\\
3781	29.720801\\
3783	29.655408\\
3785	29.720801\\
3787	31.166668\\
3789	29.720801\\
3792	29.067829\\
3794	28.179249\\
3796	27.630618\\
3798	27.559936\\
3800	29.288532\\
3802	29.134334\\
3804	28.716821\\
3806	27.738325\\
3808	28.361966\\
3810	28.716821\\
3813	28.716821\\
3815	28.640368\\
3817	27.630618\\
3820	27.630618\\
3823	27.630618\\
3825	29.655408\\
3827	30.231927\\
3830	29.655408\\
3833	30.231927\\
3835	31.166668\\
3838	31.166668\\
3840	29.655408\\
3842	28.361966\\
3844	27.630618\\
3846	31.166668\\
3848	31.489788\\
3850	31.680679\\
3853	32.135547\\
3855	31.166668\\
3857	32.289895\\
3859	31.680679\\
3861	31.166668\\
3864	31.166668\\
3866	29.288532\\
3868	28.179249\\
3870	31.166668\\
3872	33.581414\\
3874	33.775671\\
3877	33.775671\\
3879	33.425411\\
3881	33.775671\\
3883	32.660618\\
3885	31.680679\\
3888	31.166668\\
3890	29.655408\\
3892	28.640368\\
3894	31.166668\\
3896	32.660618\\
3899	32.660618\\
3901	32.289895\\
3903	31.680679\\
3905	32.660618\\
3907	31.680679\\
3909	31.489788\\
3911	31.504213\\
3913	31.166668\\
3916	28.640368\\
3918	31.166668\\
3920	31.680679\\
3922	32.660618\\
3925	32.660618\\
3927	31.680679\\
3929	32.135547\\
3931	31.680679\\
3933	31.166668\\
3935	31.489788\\
3937	29.655408\\
3940	29.655408\\
3942	28.179249\\
3943	31.166668\\
3946	31.166668\\
3949	30.231927\\
3951	29.655408\\
3953	31.166668\\
3956	31.166668\\
3959	31.166668\\
3962	31.166668\\
3964	29.655408\\
3966	28.640368\\
3968	31.166668\\
3970	29.655408\\
3973	29.655408\\
3976	29.655408\\
3978	31.166668\\
3981	31.166668\\
3984	28.716821\\
3986	27.630618\\
3988	27.559936\\
3990	28.361966\\
3992	31.166668\\
3994	29.720801\\
3996	29.655408\\
3999	29.288532\\
4001	29.655408\\
4004	29.655408\\
4006	29.134334\\
4008	27.630618\\
4010	27.559936\\
4012	26.128494\\
4014	29.720801\\
4016	31.489788\\
4018	31.680679\\
4021	31.166668\\
4023	29.655408\\
4026	28.716821\\
4028	29.067829\\
4030	29.655408\\
4032	28.361966\\
4034	26.128494\\
4036	27.559936\\
4038	28.640368\\
4040	29.655408\\
4042	29.288532\\
4045	29.655408\\
4048	29.655408\\
4050	29.720801\\
4052	29.655408\\
4054	29.720801\\
4056	29.655408\\
4058	27.738325\\
4060	27.630618\\
4062	29.655408\\
4064	31.489788\\
4066	31.166668\\
4069	29.720801\\
4071	29.655408\\
4073	31.166668\\
4076	30.231927\\
4078	29.720801\\
4080	29.655408\\
4082	27.630618\\
4085	28.361966\\
4087	31.166668\\
4090	31.166668\\
4093	29.720801\\
4095	31.166668\\
4098	31.166668\\
4101	31.166668\\
4104	31.166668\\
4106	29.655408\\
4108	29.288532\\
4110	28.361966\\
4111	31.166668\\
4113	31.489788\\
4115	31.166668\\
4118	31.166668\\
4121	31.166668\\
4124	31.166668\\
4127	31.166668\\
4129	29.720801\\
4131	27.738325\\
4133	27.630618\\
4135	28.179249\\
4136	31.166668\\
4139	31.166668\\
4141	30.231927\\
4143	29.720801\\
4145	31.166668\\
4148	31.166668\\
4151	31.166668\\
4153	29.655408\\
4154	27.630618\\
4156	27.738325\\
4157	29.655408\\
4159	31.434973\\
4161	31.680679\\
4164	31.680679\\
4167	31.680679\\
4169	32.135547\\
4172	31.680679\\
4174	31.504213\\
4176	31.166668\\
4177	27.630618\\
4180	27.630618\\
4182	31.166668\\
4184	32.660618\\
4187	32.660618\\
4190	32.660618\\
4192	32.135547\\
4194	31.680679\\
4197	31.166668\\
4200	31.166668\\
4202	28.179249\\
4205	28.361966\\
4206	31.166668\\
4208	32.660618\\
4211	32.660618\\
4213	31.504213\\
4215	31.489788\\
4218	31.489788\\
4221	31.166668\\
4224	29.655408\\
4225	27.630618\\
4228	27.630618\\
4230	31.166668\\
4232	32.660618\\
4235	32.660618\\
4237	32.135547\\
4239	32.660618\\
4242	32.660618\\
4245	31.680679\\
4247	31.166668\\
4249	29.067829\\
4251	27.630618\\
4253	27.738325\\
4254	29.655408\\
4256	31.434973\\
4258	31.166668\\
4260	31.489788\\
4262	31.166668\\
4265	31.166668\\
4268	31.166668\\
4270	29.720801\\
4273	29.655408\\
4275	29.067829\\
4277	28.179249\\
4279	29.655408\\
4282	29.720801\\
4284	31.166668\\
4286	30.231927\\
4288	31.166668\\
4291	31.166668\\
4293	30.231927\\
4296	29.720801\\
4298	29.655408\\
4300	27.738325\\
4303	28.361966\\
4305	29.655408\\
4307	31.166668\\
4310	31.166668\\
4312	30.231927\\
4314	29.720801\\
4316	31.166668\\
4319	31.166668\\
4321	29.288532\\
4323	28.361966\\
4325	29.288532\\
4326	31.166668\\
4328	31.489788\\
4331	31.680679\\
4333	31.504213\\
4335	31.166668\\
4338	31.504213\\
4340	31.680679\\
4342	31.166668\\
4344	29.510215\\
4346	27.522926\\
4348	27.731898\\
4350	29.510215\\
4352	29.996907\\
4355	29.996907\\
4358	29.76426\\
4360	29.510215\\
4362	29.996907\\
4364	30.573744\\
4366	28.625154\\
4368	29.510215\\
4370	26.681572\\
4372	26.1621\\
4374	28.141193\\
4376	29.510215\\
4379	29.510215\\
4382	29.510215\\
4385	29.510215\\
4387	29.996907\\
4389	29.82982\\
4391	29.510215\\
4393	28.079276\\
4394	26.264082\\
4396	26.1621\\
4398	27.190573\\
4399	29.510215\\
4401	28.141193\\
4404	28.079276\\
4406	27.522926\\
4408	28.079276\\
4410	28.625154\\
4412	29.510215\\
4414	28.141193\\
4416	27.190573\\
4418	26.681572\\
4421	27.522926\\
4423	29.510215\\
4426	29.510215\\
4429	29.510215\\
4432	28.141193\\
4435	28.141193\\
4437	28.079276\\
4439	27.522926\\
4441	28.079276\\
4443	26.854578\\
4445	26.1621\\
4447	26.681572\\
4449	28.141193\\
4451	29.510215\\
4454	28.141193\\
4457	28.079276\\
4459	27.522926\\
4461	28.079276\\
4464	27.731898\\
4466	26.1621\\
4468	24.739811\\
4470	26.1621\\
4472	26.854578\\
4475	27.522926\\
4478	27.190573\\
4480	27.522926\\
4482	28.079276\\
4484	27.731898\\
4486	27.522926\\
4488	27.118184\\
4490	26.1621\\
4493	26.681572\\
4495	29.510215\\
4498	29.510215\\
4501	29.510215\\
4504	29.510215\\
4507	29.510215\\
4510	28.079276\\
4512	27.522926\\
4514	26.681572\\
4516	26.1621\\
4518	28.141193\\
4520	29.996907\\
4522	30.924764\\
4525	30.924764\\
4527	29.996907\\
4529	29.82982\\
4531	29.76426\\
4533	29.510215\\
4536	27.118184\\
4538	26.1621\\
4540	26.095174\\
4542	28.079276\\
4543	29.82982\\
4545	30.924764\\
4548	30.924764\\
4551	30.924764\\
4554	30.573744\\
4556	29.510215\\
4558	27.522926\\
4561	27.522926\\
4563	24.739811\\
4565	26.681572\\
4567	29.76426\\
4569	30.924764\\
4572	30.924764\\
4574	29.996907\\
4576	29.510215\\
4579	29.76426\\
4581	29.510215\\
4583	28.625154\\
4585	26.264082\\
4587	26.1621\\
4589	28.079276\\
4591	29.996907\\
4593	30.924764\\
4595	30.573744\\
4597	29.996907\\
4599	29.82982\\
4601	29.816161\\
4603	29.996907\\
4605	29.816161\\
4608	29.510215\\
4611	28.141193\\
4613	28.079276\\
4615	28.141193\\
4617	29.816161\\
4620	29.76426\\
4622	29.510215\\
4625	29.510215\\
4627	29.996907\\
4629	29.82982\\
4631	29.76426\\
4633	28.625154\\
4635	27.522926\\
4637	26.264082\\
4639	26.854578\\
4641	28.141193\\
4643	29.510215\\
4646	28.141193\\
4648	28.079276\\
4650	29.510215\\
4653	29.510215\\
4656	29.510215\\
4657	27.522926\\
4659	26.854578\\
4661	26.681572\\
4662	29.510215\\
4664	29.996907\\
4666	29.82982\\
4668	29.76426\\
4670	29.510215\\
4673	29.510215\\
4675	29.996907\\
4677	29.816161\\
4679	28.141193\\
4681	27.731898\\
4683	26.264082\\
4685	26.1621\\
4686	29.510215\\
4688	29.996907\\
4690	29.510215\\
4692	29.816161\\
4694	29.510215\\
4696	29.76426\\
4698	29.816161\\
4700	29.996907\\
4702	29.82982\\
4704	28.079276\\
4706	26.681572\\
4708	26.264082\\
4709	28.079276\\
4711	29.816161\\
4713	30.924764\\
4715	30.573744\\
4717	29.996907\\
4720	29.996907\\
4723	30.924764\\
4725	29.82982\\
4727	29.510215\\
4729	27.522926\\
4731	26.1621\\
4733	26.681572\\
4734	29.510215\\
4736	29.996907\\
4738	30.573744\\
4740	29.996907\\
4743	30.4276\\
4745	29.996907\\
4747	30.573744\\
4749	29.996907\\
4751	28.141193\\
4753	26.681572\\
4755	26.264082\\
4757	28.079276\\
4759	29.82982\\
4761	29.996907\\
4764	29.816161\\
4766	29.510215\\
4769	29.510215\\
4771	29.996907\\
4773	29.82982\\
4775	29.76426\\
4776	27.731898\\
4778	26.1621\\
4781	26.095174\\
4783	27.522926\\
4785	27.190573\\
4787	27.731898\\
4789	28.079276\\
4791	27.522926\\
4793	28.079276\\
4795	29.510215\\
4797	28.141193\\
4799	27.731898\\
4801	26.681572\\
4803	26.1621\\
4806	26.1621\\
4808	26.854578\\
4810	27.522926\\
4813	27.190573\\
4815	26.681572\\
4818	27.190573\\
4820	28.079276\\
4823	27.190573\\
4825	26.1621\\
4827	26.095174\\
4829	26.681572\\
4832	28.079276\\
4834	28.625154\\
4836	29.510215\\
4838	28.625154\\
4840	28.141193\\
4843	28.141193\\
4845	26.264082\\
4847	26.1621\\
4850	26.1621\\
4852	24.739811\\
4854	26.854578\\
4856	29.510215\\
4859	29.510215\\
4862	28.079276\\
4864	27.522926\\
4867	27.731898\\
4869	27.190573\\
4871	27.522926\\
4873	26.264082\\
4875	24.739811\\
4877	26.1621\\
4879	28.079276\\
4881	29.510215\\
4883	28.141193\\
4885	28.079276\\
4888	28.079276\\
4890	27.522926\\
4893	26.854578\\
4895	27.522926\\
4897	26.264082\\
4899	26.095174\\
4901	26.1621\\
4903	28.141193\\
4905	28.079276\\
4908	28.079276\\
4910	27.731898\\
4912	27.522926\\
4914	28.079276\\
4916	28.141193\\
4918	28.079276\\
4920	26.854578\\
4922	26.1621\\
4925	26.1621\\
4927	28.079276\\
4929	29.510215\\
4932	29.510215\\
4934	28.141193\\
4936	27.522926\\
4938	27.190573\\
4940	28.079276\\
4942	28.141193\\
4944	27.118184\\
4946	26.854578\\
4948	26.1621\\
4950	26.264082\\
4952	26.854578\\
4954	27.731898\\
4956	28.079276\\
4958	27.118184\\
4960	27.190573\\
4962	27.731898\\
4964	28.079276\\
4967	27.522926\\
4969	26.681572\\
4971	26.1621\\
4974	26.1621\\
4976	26.854578\\
4979	26.264082\\
4981	26.1621\\
4984	26.1621\\
4986	27.118184\\
4988	28.079276\\
4990	28.141193\\
4992	27.522926\\
4994	26.1621\\
4997	26.854578\\
4999	27.522926\\
5001	29.510215\\
5004	29.510215\\
5006	28.079276\\
5008	27.522926\\
5010	28.141193\\
5012	28.079276\\
5015	27.522926\\
5017	26.854578\\
5019	26.264082\\
5021	26.1621\\
5023	28.141193\\
5025	29.510215\\
5028	29.510215\\
5031	29.510215\\
5034	29.510215\\
5036	28.079276\\
5038	28.141193\\
5040	27.118184\\
5042	26.1621\\
5045	26.1621\\
5047	28.625154\\
5049	29.510215\\
5052	28.625154\\
5055	28.141193\\
5057	29.510215\\
5060	29.510215\\
5063	28.141193\\
5065	27.522926\\
5067	26.1621\\
5070	26.681572\\
5072	28.079276\\
5074	28.141193\\
5077	28.079276\\
5080	28.079276\\
5083	28.079276\\
5086	28.079276\\
5088	27.898569\\
5090	27.79024\\
5092	27.71915\\
5094	28.525813\\
5096	31.346718\\
5099	30.406578\\
5101	29.892499\\
5104	29.892499\\
5106	31.346718\\
5109	29.826728\\
5112	29.134334\\
5114	27.898569\\
5116	27.79024\\
5119	27.79024\\
5121	29.235755\\
5123	29.892499\\
5125	29.826728\\
5128	29.826728\\
5130	31.346718\\
5132	29.892499\\
5134	29.826728\\
5136	29.235755\\
5138	28.342041\\
5140	27.79024\\
5142	27.898569\\
5144	29.235755\\
5146	28.805824\\
5148	28.525813\\
5150	27.79024\\
5152	27.898569\\
5154	29.235755\\
5156	29.826728\\
5159	29.235755\\
5161	28.342041\\
5163	27.79024\\
5165	28.342041\\
5167	31.346718\\
5169	31.671705\\
5171	31.346718\\
5174	30.406578\\
5177	31.346718\\
5180	31.346718\\
5183	29.826728\\
5185	29.235755\\
5187	28.342041\\
5190	29.457732\\
5192	31.346718\\
5195	31.346718\\
5197	32.849298\\
5200	32.849298\\
5203	32.849298\\
5206	32.849298\\
5208	31.346718\\
5211	31.346718\\
5214	31.616573\\
5216	31.863699\\
5218	32.849298\\
5221	32.321195\\
5223	32.849298\\
5226	32.849298\\
5229	32.849298\\
5231	31.346718\\
5233	29.826728\\
5235	28.882718\\
5237	29.892499\\
5239	31.616573\\
5241	32.849298\\
5244	32.849298\\
5247	32.849298\\
5250	32.849298\\
5252	32.476434\\
5254	31.863699\\
5256	29.892499\\
5258	27.79024\\
5261	28.525813\\
5263	31.616573\\
5265	32.849298\\
5268	32.849298\\
5270	31.863699\\
5272	32.849298\\
5275	32.849298\\
5277	31.863699\\
5279	31.346718\\
5280	28.882718\\
5282	27.898569\\
5284	27.79024\\
5287	27.79024\\
5289	28.525813\\
5292	27.898569\\
5295	27.898569\\
5297	28.342041\\
5299	29.235755\\
5301	29.457732\\
5303	29.826728\\
5304	27.898569\\
5307	27.898569\\
5309	28.342041\\
5311	27.79024\\
5313	28.342041\\
5315	29.235755\\
5317	28.342041\\
5319	27.79024\\
5321	28.342041\\
5324	27.898569\\
5327	27.898569\\
5329	27.79024\\
5332	27.79024\\
5335	27.898569\\
5337	28.525813\\
5340	28.342041\\
5343	27.898569\\
5346	27.898569\\
5348	28.525813\\
5350	27.79024\\
5352	27.898569\\
5354	27.79024\\
5356	26.279439\\
5358	27.898569\\
5359	29.826728\\
5362	29.826728\\
5365	29.892499\\
5367	29.826728\\
5369	29.892499\\
5371	31.346718\\
5374	30.406578\\
5376	27.898569\\
5378	26.279439\\
5380	27.79024\\
5382	28.525813\\
5384	31.346718\\
5387	31.686213\\
5389	31.671705\\
5391	31.616573\\
5393	31.346718\\
5395	31.671705\\
5397	31.346718\\
5399	29.826728\\
5401	27.79024\\
5404	27.79024\\
5406	28.525813\\
5407	30.406578\\
5409	31.686213\\
5411	31.863699\\
5414	31.671705\\
5417	31.671705\\
5420	31.346718\\
5423	29.826728\\
5424	27.898569\\
5426	27.79024\\
5429	27.898569\\
5431	28.525813\\
5433	28.882718\\
5435	28.525813\\
5437	28.342041\\
5440	27.898569\\
5442	27.79024\\
5444	27.898569\\
5446	29.235755\\
5448	28.342041\\
5450	27.79024\\
5452	27.71915\\
5454	27.898569\\
5456	29.134334\\
5458	29.892499\\
5460	29.826728\\
5463	29.826728\\
5466	29.892499\\
5469	29.892499\\
5471	29.826728\\
5472	27.898569\\
5475	27.79024\\
5478	27.79024\\
5481	28.525813\\
5483	27.898569\\
5485	28.342041\\
5487	27.898569\\
5489	28.882718\\
5491	29.826728\\
5493	30.406578\\
5495	29.457732\\
5497	27.79024\\
5500	27.79024\\
5502	29.457732\\
5503	31.616573\\
5505	31.671705\\
5507	32.321195\\
5509	32.476434\\
5511	31.863699\\
5513	32.476434\\
5515	32.849298\\
5518	32.849298\\
5520	29.892499\\
5522	27.898569\\
5524	27.79024\\
5526	29.826728\\
5528	31.346718\\
5530	31.863699\\
5532	31.346718\\
5535	31.346718\\
5538	31.346718\\
5540	31.616573\\
5543	31.346718\\
5545	29.826728\\
5547	28.525813\\
5549	28.805824\\
5550	31.346718\\
5552	32.849298\\
5555	32.849298\\
5558	32.849298\\
5561	32.849298\\
5564	32.849298\\
5567	31.616573\\
5569	29.235755\\
5571	28.525813\\
5573	29.457732\\
5574	31.346718\\
5576	31.863699\\
5578	32.476434\\
5580	32.849298\\
5582	31.671705\\
5584	31.616573\\
5586	31.671705\\
5589	31.616573\\
5591	31.346718\\
5593	28.882718\\
5595	27.79024\\
5597	28.805824\\
5598	31.346718\\
5600	31.616573\\
5602	31.671705\\
5604	31.686213\\
5606	31.671705\\
5608	31.346718\\
5610	31.671705\\
5612	31.863699\\
5614	31.671705\\
5616	31.346718\\
5619	31.346718\\
5622	31.346718\\
5624	31.863699\\
5626	31.686213\\
5628	31.671705\\
5630	31.346718\\
5632	31.616573\\
5635	31.863699\\
5637	32.476434\\
5639	31.671705\\
5641	29.235755\\
5643	27.79024\\
5646	27.79024\\
5648	29.457732\\
5650	29.826728\\
5652	29.892499\\
5654	28.525813\\
5656	29.235755\\
5658	31.346718\\
5660	31.671705\\
5662	31.346718\\
5665	29.826728\\
5667	28.525813\\
5669	30.406578\\
5671	32.849298\\
5674	32.849298\\
5677	33.425411\\
5679	32.849298\\
5681	34.354\\
5684	33.425411\\
5686	32.849298\\
5688	31.346718\\
5690	29.826728\\
5693	29.826728\\
5694	31.671705\\
5696	32.849298\\
5698	34.820014\\
5700	34.354\\
5703	34.354\\
5705	33.775414\\
5707	32.849298\\
5710	32.849298\\
5712	29.892499\\
5714	28.525813\\
5716	28.342041\\
5718	31.616573\\
5720	32.849298\\
5723	32.849298\\
5726	32.849298\\
5729	32.849298\\
5732	32.849298\\
5735	32.849298\\
5737	29.892499\\
5739	29.235755\\
5741	29.826728\\
5743	32.849298\\
5746	32.849298\\
5749	32.849298\\
5752	32.849298\\
5755	32.849298\\
5758	32.849298\\
5760	31.346718\\
5762	28.805824\\
5765	29.892499\\
5767	31.863699\\
5769	32.849298\\
5772	32.476434\\
5774	31.863699\\
5776	32.476434\\
5779	32.476434\\
5781	32.849298\\
5783	29.826728\\
5784	27.79024\\
5786	29.892499\\
5788	29.826728\\
5790	29.892499\\
5792	31.346718\\
5794	31.863699\\
5797	31.863699\\
5800	31.863699\\
5802	32.849298\\
5805	32.849298\\
5807	31.686213\\
5809	30.406578\\
5811	29.235755\\
5814	29.457732\\
5816	31.346718\\
5818	31.686213\\
5820	31.863699\\
5822	31.346718\\
5825	31.346718\\
5827	32.476434\\
5829	32.849298\\
5831	31.863699\\
5832	37.342355\\
5834	35.60999\\
5836	35.325495\\
5837	37.342355\\
5839	39.132332\\
5842	39.132332\\
5845	39.132332\\
5848	39.132332\\
5851	39.132332\\
5854	39.132332\\
5857	37.342355\\
5860	37.342355\\
5862	39.132332\\
5864	39.551164\\
5865	41.479983\\
5868	41.479983\\
5870	40.468333\\
5873	41.479983\\
5875	39.132332\\
5878	39.132332\\
5880	37.729502\\
5882	37.342355\\
5885	37.729502\\
5887	39.132332\\
5890	39.132332\\
5893	39.132332\\
5896	39.132332\\
5899	39.132332\\
5902	39.132332\\
5904	37.342355\\
5907	37.342355\\
5910	39.132332\\
5912	41.479983\\
5915	41.479983\\
5917	39.132332\\
5920	39.132332\\
5921	41.479983\\
5923	39.132332\\
5924	40.967821\\
5926	39.132332\\
5929	37.663825\\
5931	37.342355\\
5933	37.663825\\
5935	39.132332\\
5936	41.479983\\
5939	41.479983\\
5941	39.132332\\
5944	39.132332\\
5947	39.132332\\
5950	39.132332\\
5953	37.342355\\
5956	35.531639\\
5958	37.342355\\
5960	39.132332\\
5963	40.468333\\
5965	39.132332\\
5968	39.132332\\
5971	39.132332\\
5974	39.132332\\
5977	39.132332\\
5979	37.663825\\
5981	37.729502\\
5983	39.132332\\
5986	39.132332\\
5989	39.132332\\
5992	38.503218\\
5994	39.132332\\
5997	39.132332\\
5999	37.342355\\
6001	35.531639\\
6003	34.354\\
6005	37.342355\\
6006	39.132332\\
6008	41.32501\\
6010	40.468333\\
6013	39.551164\\
6015	39.132332\\
6018	39.551164\\
6020	39.132332\\
6023	37.958218\\
6025	36.222395\\
6027	35.531639\\
6029	37.342355\\
6030	39.132332\\
6031	41.479983\\
6034	40.235584\\
6036	40.468333\\
6038	39.132332\\
6041	40.235584\\
6043	39.132332\\
6046	39.132332\\
6048	37.342355\\
6051	37.342355\\
6054	39.132332\\
6056	40.468333\\
6058	39.132332\\
6061	39.132332\\
6064	39.132332\\
6067	39.132332\\
6070	39.132332\\
6072	37.342355\\
6073	35.60999\\
6075	35.531639\\
6077	37.342355\\
6078	39.132332\\
6080	40.468333\\
6082	39.132332\\
6084	39.551164\\
6086	39.132332\\
6089	39.132332\\
6092	39.132332\\
6095	39.132332\\
6096	37.342355\\
6098	35.531639\\
6100	35.60999\\
6101	37.342355\\
6102	39.132332\\
6104	40.967821\\
6105	39.132332\\
6108	39.132332\\
6111	39.132332\\
6114	39.132332\\
6116	40.235584\\
6118	39.132332\\
6120	37.342355\\
6122	35.60999\\
6124	35.531639\\
6126	37.342355\\
6128	39.132332\\
6131	39.132332\\
6133	37.663825\\
6135	37.342355\\
6137	37.663825\\
6139	39.132332\\
6142	38.503218\\
6144	37.342355\\
6147	34.354\\
6150	34.827631\\
6151	37.342355\\
6153	37.746785\\
6155	38.68815\\
6158	37.958218\\
6160	37.729502\\
6162	37.746785\\
6164	39.132332\\
6166	38.503218\\
6168	37.342355\\
6170	36.222395\\
6172	37.342355\\
6174	39.132332\\
6176	39.551164\\
6178	41.479983\\
6180	39.132332\\
6183	39.132332\\
6186	39.551164\\
6188	40.468333\\
6190	39.132332\\
6192	37.729502\\
6194	37.342355\\
6197	37.663825\\
6199	40.468333\\
6202	39.132332\\
6205	39.132332\\
6208	39.132332\\
6210	40.468333\\
6212	41.32501\\
6213	39.132332\\
6216	37.746785\\
6218	37.342355\\
6221	37.342355\\
6222	39.132332\\
6225	39.132332\\
6228	39.132332\\
6231	39.132332\\
6234	39.132332\\
6237	39.132332\\
6240	37.342355\\
6242	35.60999\\
6244	36.222395\\
6246	39.132332\\
6248	41.479983\\
6251	42.966349\\
6253	41.479983\\
6256	41.479983\\
6259	40.468333\\
6261	39.132332\\
6264	37.746785\\
6266	37.342355\\
6269	37.342355\\
6270	39.132332\\
6272	41.479983\\
6275	41.479983\\
6277	40.468333\\
6279	39.551164\\
6281	40.468333\\
6283	39.132332\\
6286	39.132332\\
6288	37.729502\\
6290	35.60999\\
6292	34.827631\\
6294	35.60999\\
6295	37.342355\\
6297	39.132332\\
6300	39.132332\\
6303	39.132332\\
6306	39.132332\\
6309	39.132332\\
6311	37.342355\\
6313	35.092066\\
6315	33.234684\\
6317	33.105636\\
6319	34.354\\
6321	37.342355\\
6324	35.60999\\
6326	34.827631\\
6329	34.827631\\
6330	37.342355\\
6332	37.663825\\
6334	37.342355\\
6335	35.60999\\
6337	34.354\\
6339	33.762978\\
6341	34.354\\
6342	37.342355\\
6343	39.132332\\
6346	39.132332\\
6349	39.132332\\
6352	39.132332\\
6355	39.132332\\
6358	39.132332\\
6360	37.729502\\
6362	37.342355\\
6363	35.60999\\
6364	37.342355\\
6366	39.132332\\
6367	41.479983\\
6369	39.551164\\
6371	39.132332\\
6374	39.132332\\
6377	39.132332\\
6379	41.479983\\
6381	39.551164\\
6383	39.132332\\
6385	37.342355\\
6386	35.60999\\
6389	37.342355\\
6390	39.132332\\
6392	40.468333\\
6394	40.967821\\
6396	41.479983\\
6398	40.235584\\
6400	39.132332\\
6402	40.468333\\
6405	39.132332\\
6408	37.342355\\
6409	35.60999\\
6412	35.60999\\
6413	37.342355\\
6414	39.132332\\
6416	40.468333\\
6418	39.551164\\
6420	39.132332\\
6423	39.132332\\
6426	39.132332\\
6429	39.132332\\
6431	38.503218\\
6433	37.342355\\
6434	35.60999\\
6436	36.222395\\
6438	39.132332\\
6440	40.468333\\
6443	40.468333\\
6445	40.235584\\
6447	39.551164\\
6448	41.479983\\
6451	41.479983\\
6454	39.132332\\
6457	39.132332\\
6459	37.342355\\
6462	39.132332\\
6465	39.132332\\
6467	39.551164\\
6469	39.132332\\
6472	39.132332\\
6475	39.132332\\
6478	39.132332\\
6480	37.958218\\
6481	36.222395\\
6483	34.827631\\
6486	35.60999\\
6487	37.342355\\
6489	37.663825\\
6491	37.342355\\
6494	37.342355\\
6497	37.342355\\
6498	39.132332\\
6500	39.551164\\
6502	37.729502\\
6504	37.342355\\
6506	35.325495\\
6508	35.531639\\
6509	37.342355\\
6511	41.479983\\
6513	40.468333\\
6516	39.132332\\
6518	41.32501\\
6520	42.966349\\
6522	43.208316\\
6524	42.973263\\
6526	39.132332\\
6528	37.729502\\
6530	34.827631\\
6533	37.342355\\
6534	39.132332\\
6535	47.807985\\
6537	43.208316\\
6539	41.985809\\
6541	41.479983\\
6544	41.479983\\
6546	42.966349\\
6548	42.973263\\
6550	40.468333\\
6552	40.30219\\
6553	38.458702\\
6554	36.593854\\
6557	38.458702\\
6558	40.30219\\
6559	44.257945\\
6561	44.500026\\
6563	45.466566\\
6565	44.257945\\
6567	42.720025\\
6569	44.257945\\
6571	50.343176\\
6573	44.257945\\
6575	41.438424\\
6577	40.30219\\
6579	39.84473\\
6581	40.30219\\
6582	44.257945\\
6583	62.745036\\
6584	60.594961\\
6585	69.11735\\
6586	58.20188\\
6589	58.20188\\
6590	56.4575\\
6592	58.20188\\
6595	74.633069\\
6596	58.20188\\
6597	53.400031\\
6598	50.343176\\
6599	42.720025\\
6600	40.30219\\
6601	38.458702\\
6604	38.458702\\
6606	40.30219\\
6607	44.257945\\
6609	44.250825\\
6611	43.240971\\
6613	42.720025\\
6615	41.678131\\
6617	42.720025\\
6619	44.257945\\
6620	49.237202\\
6621	42.720025\\
6623	41.678131\\
6624	39.092976\\
6626	36.593854\\
6628	34.354\\
6630	38.458702\\
6632	38.789782\\
6634	38.458702\\
6637	38.789782\\
6639	38.857422\\
6641	39.654269\\
6643	40.30219\\
6646	39.092976\\
6648	39.654269\\
6650	38.458702\\
6651	36.593854\\
6654	38.458702\\
6656	39.84473\\
6658	40.30219\\
6660	40.733543\\
6662	40.30219\\
6665	41.678131\\
6667	42.891498\\
6669	42.720025\\
6670	40.30219\\
6672	39.092976\\
6674	35.435667\\
6677	38.458702\\
6678	40.30219\\
6679	43.240971\\
6681	44.257945\\
6684	44.257945\\
6687	42.720025\\
6689	41.678131\\
6691	43.240971\\
6693	40.30219\\
6696	40.30219\\
6697	36.593854\\
6699	35.868801\\
6702	39.092976\\
6703	42.720025\\
6706	42.720025\\
6709	42.720025\\
6710	40.30219\\
6713	40.30219\\
6714	42.720025\\
6716	44.250825\\
6717	41.678131\\
6719	40.30219\\
6721	37.305261\\
6723	36.593854\\
6725	37.305261\\
6726	40.30219\\
6728	42.720025\\
6730	44.257945\\
6733	42.720025\\
6735	40.30219\\
6738	40.30219\\
6739	42.720025\\
6741	40.30219\\
6744	38.857422\\
6746	38.458702\\
6748	36.674548\\
6749	38.458702\\
6750	40.30219\\
6751	42.720025\\
6754	42.720025\\
6756	41.678131\\
6758	40.733543\\
6760	40.30219\\
6762	42.720025\\
6765	40.30219\\
6768	39.092976\\
6770	36.593854\\
6773	36.674548\\
6774	40.30219\\
6775	44.257945\\
6777	42.891498\\
6779	44.250825\\
6781	42.720025\\
6783	41.678131\\
6786	44.250825\\
6788	44.500026\\
6789	42.720025\\
6791	40.30219\\
6793	38.857422\\
6795	38.458702\\
6798	38.857422\\
6800	40.30219\\
6803	40.30219\\
6806	40.30219\\
6809	40.30219\\
6812	40.30219\\
6815	40.30219\\
6817	38.458702\\
6818	35.868801\\
6821	35.868801\\
6822	38.458702\\
6824	39.092976\\
6826	40.30219\\
6828	39.84473\\
6830	38.857422\\
6832	39.092976\\
6834	40.30219\\
6837	39.654269\\
6839	39.092976\\
6841	35.868801\\
6843	34.997787\\
6845	36.141141\\
6846	40.30219\\
6849	40.30219\\
6852	40.30219\\
6855	40.30219\\
6858	42.560418\\
6860	42.720025\\
6861	40.30219\\
6864	38.458702\\
6865	36.674548\\
6867	36.593854\\
6869	38.458702\\
6870	40.30219\\
6871	42.720025\\
6873	41.678131\\
6875	41.438424\\
6877	40.30219\\
6880	40.30219\\
6882	42.720025\\
6885	40.733543\\
6887	40.30219\\
6889	38.857422\\
6891	38.458702\\
6893	40.30219\\
6894	42.720025\\
6895	53.400031\\
6896	44.257945\\
6898	42.720025\\
6901	42.720025\\
6904	42.720025\\
6906	44.500026\\
6907	52.740837\\
6908	44.257945\\
6910	41.678131\\
6912	40.30219\\
6914	38.458702\\
6917	38.857422\\
6919	42.720025\\
6922	42.720025\\
6925	41.678131\\
6927	40.30219\\
6929	42.720025\\
6931	44.500026\\
6933	42.192551\\
6935	40.30219\\
6937	38.857422\\
6939	38.458702\\
6941	39.092976\\
6943	42.720025\\
6945	43.240971\\
6947	42.720025\\
6948	40.733543\\
6950	40.30219\\
6953	41.678131\\
6955	42.720025\\
6957	40.30219\\
6959	39.092976\\
6961	38.458702\\
6963	36.593854\\
6965	36.674548\\
6966	38.458702\\
6968	39.092976\\
6970	39.84473\\
6972	38.857422\\
6974	38.458702\\
6977	38.857422\\
6979	40.30219\\
6982	39.84473\\
6984	38.458702\\
6986	35.435667\\
6988	34.997787\\
6990	36.674548\\
6991	38.458702\\
6993	38.789782\\
6995	38.857422\\
6997	38.458702\\
7000	38.458702\\
7002	40.30219\\
7005	39.84473\\
7007	38.789782\\
7009	38.458702\\
7010	35.868801\\
7012	36.141141\\
7013	38.458702\\
7015	40.30219\\
7018	40.30219\\
7021	40.30219\\
7024	40.30219\\
7027	40.30219\\
7029	38.857422\\
7032	38.458702\\
7033	35.868801\\
7035	34.095326\\
7037	35.435667\\
7038	38.857422\\
7040	40.30219\\
7043	40.30219\\
7046	40.30219\\
7049	40.30219\\
7052	40.30219\\
7053	38.458702\\
7056	38.458702\\
7057	36.593854\\
7058	34.77232\\
7061	35.868801\\
7062	38.857422\\
7064	40.30219\\
7067	40.30219\\
7070	40.30219\\
7073	40.30219\\
7075	44.257945\\
7077	40.30219\\
7079	39.092976\\
7081	36.674548\\
7083	34.354\\
7085	38.458702\\
7086	40.733543\\
7087	44.500026\\
7089	44.257945\\
7091	42.891498\\
7093	40.30219\\
7095	41.678131\\
7097	41.438424\\
7098	43.382185\\
7100	44.257945\\
7102	40.30219\\
7104	38.458702\\
7105	34.77232\\
7107	34.095326\\
7109	34.77232\\
7110	38.458702\\
7111	40.30219\\
7114	40.30219\\
7117	40.30219\\
7120	40.30219\\
7123	40.30219\\
7125	39.092976\\
7127	40.30219\\
7129	38.458702\\
7131	36.593854\\
7133	36.674548\\
7134	38.875222\\
7136	40.30219\\
7139	40.30219\\
7141	39.84473\\
7143	39.092976\\
7145	40.30219\\
7148	40.30219\\
7150	39.092976\\
7152	38.458702\\
7153	35.868801\\
7155	34.77232\\
7157	34.354\\
7159	34.997787\\
7161	38.458702\\
7164	38.458702\\
7166	36.674548\\
7167	38.458702\\
7169	38.789782\\
7171	40.30219\\
7173	38.875222\\
7175	38.458702\\
7176	36.674548\\
7178	34.997787\\
7180	34.095326\\
7182	34.228233\\
7183	39.092976\\
7185	40.30219\\
7188	40.30219\\
7191	40.30219\\
7194	41.678131\\
7196	42.720025\\
7197	40.30219\\
7199	38.857422\\
7201	38.458702\\
7202	36.141141\\
7204	34.095326\\
7206	34.997787\\
7207	39.092976\\
7209	40.30219\\
7212	40.30219\\
7215	40.30219\\
7218	44.257945\\
7220	44.250825\\
7222	40.30219\\
7225	38.458702\\
7226	36.593854\\
7228	34.228233\\
7230	34.77232\\
7231	38.789782\\
7233	40.30219\\
7236	40.30219\\
7238	40.733543\\
7240	41.438424\\
7242	42.720025\\
7245	41.678131\\
7247	40.30219\\
7249	39.092976\\
7251	34.997787\\
7253	35.341327\\
7255	40.30219\\
7258	40.30219\\
7261	40.30219\\
7264	40.30219\\
7267	43.240971\\
7269	40.30219\\
7271	39.84473\\
7273	38.458702\\
7274	35.435667\\
7276	34.095326\\
7278	34.997787\\
7279	38.789782\\
7281	40.30219\\
7283	39.092976\\
7286	39.092976\\
7288	38.875222\\
7290	40.30219\\
7293	40.30219\\
7295	38.857422\\
7296	40.61926\\
7298	38.649648\\
7300	36.151129\\
7302	36.010756\\
7303	38.649648\\
7304	40.61926\\
7307	40.61926\\
7310	38.734875\\
7312	38.649648\\
7313	40.61926\\
7315	41.289167\\
7317	40.61926\\
7320	39.401021\\
7322	36.010756\\
7325	36.010756\\
7328	36.010756\\
7330	36.725782\\
7332	36.151129\\
7334	36.725782\\
7336	36.963915\\
7338	38.649648\\
7339	40.61926\\
7342	38.649648\\
7344	37.883862\\
7346	36.151129\\
7348	35.918636\\
7350	36.010756\\
7351	40.61926\\
7353	41.289167\\
7355	42.566313\\
7358	42.566313\\
7361	42.566313\\
7363	44.019552\\
7365	42.566313\\
7368	42.566313\\
7370	39.401021\\
7371	36.725782\\
7374	38.649648\\
7375	40.61926\\
7376	42.566313\\
7379	45.119978\\
7381	44.562872\\
7383	42.566313\\
7386	45.119978\\
7389	42.566313\\
7392	40.96894\\
7394	37.326755\\
7396	36.010756\\
7398	37.883862\\
7399	42.083153\\
7400	45.819338\\
7401	52.003281\\
7402	46.999977\\
7404	52.003281\\
7405	46.744297\\
7407	45.589978\\
7409	45.819338\\
7410	52.003281\\
7412	46.744297\\
7414	42.566313\\
7417	41.05918\\
7419	37.883862\\
7421	36.963915\\
7423	42.566313\\
7424	46.744297\\
7426	45.589978\\
7428	45.819338\\
7430	46.736777\\
7432	45.119978\\
7434	45.670191\\
7436	46.736777\\
7438	42.566313\\
7440	40.96894\\
7441	38.734875\\
7443	36.010756\\
7446	36.963915\\
7447	40.61926\\
7449	42.566313\\
7452	42.566313\\
7455	42.083153\\
7457	41.289167\\
7459	42.566313\\
7461	41.289167\\
7463	40.61926\\
7466	40.61926\\
7467	36.725782\\
7470	37.883862\\
7472	40.96894\\
7474	41.289167\\
7476	40.61926\\
7479	40.61926\\
7482	42.083153\\
7484	42.566313\\
7486	40.61926\\
7489	40.61926\\
7491	38.649648\\
7492	36.010756\\
7494	36.725782\\
7496	40.61926\\
7499	40.61926\\
7502	41.04038\\
7504	41.05918\\
7506	42.566313\\
7509	42.566313\\
7511	41.289167\\
7513	40.61926\\
7515	36.725782\\
7517	36.010756\\
7518	38.649648\\
7519	40.96894\\
7521	42.566313\\
7524	42.566313\\
7527	42.566313\\
7530	42.566313\\
7533	41.289167\\
7535	41.04038\\
7537	40.61926\\
7538	38.649648\\
7539	36.725782\\
7542	37.883862\\
7543	41.05918\\
7545	42.566313\\
7547	41.881993\\
7549	42.566313\\
7551	41.289167\\
7553	42.566313\\
7556	42.566313\\
7558	41.289167\\
7560	41.04038\\
7562	36.725782\\
7564	36.010756\\
7566	38.734875\\
7567	41.04038\\
7569	42.566313\\
7572	42.566313\\
7575	42.566313\\
7578	42.566313\\
7581	42.566313\\
7584	42.566313\\
7586	40.61926\\
7587	38.649648\\
7590	40.61926\\
7592	42.566313\\
7595	42.083153\\
7598	42.083153\\
7600	42.566313\\
7603	42.566313\\
7606	42.566313\\
7609	41.881993\\
7611	38.649648\\
7613	37.883862\\
7614	40.61926\\
7616	42.566313\\
7619	42.566313\\
7622	42.566313\\
7625	43.766379\\
7627	42.566313\\
7630	42.566313\\
7632	41.289167\\
7634	38.734875\\
7636	37.426395\\
7638	37.883862\\
7640	40.61926\\
7642	42.566313\\
7645	42.566313\\
7648	42.566313\\
7651	42.566313\\
7654	41.05918\\
7656	40.61926\\
7658	37.883862\\
7659	36.151129\\
7662	36.151129\\
7663	38.649648\\
7664	40.61926\\
7666	41.04038\\
7668	41.289167\\
7671	41.04038\\
7673	40.96894\\
7675	41.04038\\
7677	40.61926\\
7680	40.61926\\
7682	37.883862\\
7684	36.151129\\
7686	37.883862\\
7687	40.61926\\
7688	42.566313\\
7691	42.566313\\
7694	42.566313\\
7696	43.766379\\
7698	46.736777\\
7700	45.119978\\
7701	42.566313\\
7703	41.289167\\
7705	39.401021\\
7706	36.725782\\
7708	36.010756\\
7710	38.649648\\
7711	41.289167\\
7713	45.119978\\
7715	43.021899\\
7717	42.566313\\
7720	44.019552\\
7722	45.119978\\
7725	42.566313\\
7727	41.881993\\
7729	40.61926\\
7730	37.426395\\
7732	36.010756\\
7734	37.426395\\
7735	41.05918\\
7736	44.019552\\
7738	45.119978\\
7741	45.119978\\
7744	44.019552\\
7746	45.119978\\
7748	44.562872\\
7749	42.566313\\
7751	41.04038\\
7753	40.61926\\
7754	36.963915\\
7756	36.010756\\
7758	38.171501\\
7759	41.05918\\
7761	44.562872\\
7763	42.566313\\
7766	42.566313\\
7769	44.019552\\
7771	45.119978\\
7773	42.566313\\
7775	40.61926\\
7778	38.649648\\
7780	37.426395\\
7782	38.649648\\
7783	41.04038\\
7785	42.566313\\
7788	42.566313\\
7791	42.566313\\
7794	42.566313\\
7797	41.289167\\
7799	40.61926\\
7802	40.61926\\
7803	37.883862\\
7804	36.151129\\
7806	36.963915\\
7807	40.61926\\
7809	42.566313\\
7812	42.566313\\
7815	42.566313\\
7818	42.566313\\
7821	42.566313\\
7823	40.96894\\
7825	40.61926\\
7826	38.734875\\
7827	36.963915\\
7829	36.151129\\
7831	38.171501\\
7832	40.61926\\
7835	40.61926\\
7837	41.04038\\
7840	41.04038\\
7842	42.566313\\
7845	41.289167\\
7847	40.61926\\
7848	38.734875\\
7850	37.426395\\
7852	36.010756\\
7854	38.734875\\
7855	41.04038\\
7857	42.566313\\
7860	42.566313\\
7863	42.566313\\
7866	42.566313\\
7869	42.566313\\
7871	41.289167\\
7873	40.61926\\
7874	38.734875\\
7875	36.725782\\
7877	36.010756\\
7878	38.649648\\
7879	41.05918\\
7881	42.566313\\
7884	42.566313\\
7887	42.566313\\
7890	44.019552\\
7892	42.566313\\
7894	41.289167\\
7896	40.61926\\
7898	37.883862\\
7899	36.151129\\
7901	36.725782\\
7903	40.61926\\
7904	42.566313\\
7907	42.566313\\
7910	42.566313\\
7913	43.766379\\
7916	42.566313\\
7918	42.083153\\
7920	41.289167\\
7922	38.649648\\
7923	36.725782\\
7925	36.963915\\
7927	41.05918\\
7929	43.766379\\
7931	42.566313\\
7934	42.566313\\
7937	42.566313\\
7940	42.566313\\
7943	41.289167\\
7945	40.61926\\
7946	37.883862\\
7947	36.010756\\
7950	36.963915\\
7951	40.61926\\
7952	42.566313\\
7955	42.566313\\
7958	41.881993\\
7960	41.289167\\
7962	42.566313\\
7964	41.289167\\
7966	40.61926\\
7969	40.61926\\
7970	38.649648\\
7972	36.963915\\
7974	36.725782\\
7975	38.734875\\
7976	40.61926\\
7978	41.289167\\
7980	41.04038\\
7983	41.04038\\
7985	42.566313\\
7988	41.289167\\
7990	40.96894\\
7992	41.04038\\
7994	38.649648\\
7995	36.725782\\
7998	36.725782\\
8000	38.734875\\
8001	41.289167\\
8003	42.566313\\
8005	44.019552\\
8007	42.566313\\
8009	45.119978\\
8012	45.119978\\
8014	42.566313\\
8016	43.075788\\
8017	41.105432\\
8019	37.165352\\
8021	36.441768\\
8023	41.550617\\
8025	43.075788\\
8026	45.660018\\
8028	47.296169\\
8030	47.303779\\
8032	45.660018\\
8034	52.625708\\
8037	45.660018\\
8038	43.075788\\
8041	43.075788\\
8043	41.459297\\
8045	36.583821\\
8046	38.337293\\
8047	41.550617\\
8049	45.660018\\
8052	45.660018\\
8055	45.660018\\
8057	45.843293\\
8059	47.303779\\
8061	43.075788\\
8064	43.075788\\
8066	41.105432\\
8067	39.112245\\
8069	38.337293\\
8070	41.459297\\
8072	47.296169\\
8073	52.625708\\
8075	47.562519\\
8077	47.689353\\
8078	52.625708\\
8079	55.183303\\
8081	56.4575\\
8083	55.183303\\
8085	47.562519\\
8086	43.075788\\
8089	43.075788\\
8091	41.105432\\
8093	39.872611\\
8095	43.075788\\
8096	47.562519\\
8097	54.012631\\
8099	52.625708\\
8101	55.183303\\
8103	54.012631\\
8105	56.4575\\
8107	57.004\\
8109	47.303779\\
8111	43.075788\\
8114	42.383278\\
8116	41.105432\\
8119	43.075788\\
8120	47.303779\\
8121	54.012631\\
8123	53.807795\\
8125	52.625708\\
8126	47.562519\\
8128	47.303779\\
8130	52.625708\\
8132	47.562519\\
8134	43.536827\\
8136	43.075788\\
8138	42.586845\\
8140	41.105432\\
8141	38.337293\\
8143	41.459297\\
8145	42.586845\\
8147	43.075788\\
8149	42.586845\\
8152	43.075788\\
8155	43.075788\\
8158	43.075788\\
8160	41.783356\\
8162	41.105432\\
8165	37.406336\\
8167	36.583821\\
8169	41.105432\\
8171	42.586845\\
8173	43.075788\\
8176	43.075788\\
8179	43.075788\\
8182	43.075788\\
8184	41.783356\\
8186	41.105432\\
8188	38.628376\\
8190	41.105432\\
8191	43.075788\\
8194	43.075788\\
8197	43.075788\\
8199	44.546421\\
8201	45.660018\\
8203	46.216817\\
8205	44.546421\\
8207	43.075788\\
8210	41.783356\\
8211	39.198492\\
8213	39.112245\\
8214	41.105432\\
8215	43.075788\\
8216	47.562519\\
8217	55.183303\\
8219	47.303779\\
8222	47.562519\\
8225	54.012631\\
8226	57.075023\\
8228	56.4575\\
8229	47.303779\\
8230	43.075788\\
8233	43.075788\\
8235	41.105432\\
8236	39.112245\\
8238	41.105432\\
8239	43.075788\\
8240	45.660018\\
8242	46.367749\\
8244	45.660018\\
8246	46.135644\\
8248	47.562519\\
8249	55.183303\\
8251	57.004\\
8253	55.183303\\
8254	47.303779\\
8256	43.075788\\
8258	41.783356\\
8260	39.112245\\
8262	39.872611\\
8263	43.075788\\
8264	45.096244\\
8266	43.075788\\
8269	43.075788\\
8272	43.075788\\
8274	44.546421\\
8276	45.660018\\
8277	43.075788\\
8280	43.075788\\
8282	41.105432\\
8283	37.773519\\
8285	37.165352\\
8286	39.112245\\
8287	43.075788\\
8290	43.075788\\
8293	43.075788\\
8296	43.075788\\
8298	45.660018\\
8300	43.075788\\
8303	43.075788\\
8305	42.383278\\
8307	41.105432\\
8308	39.112245\\
8311	41.105432\\
8313	43.075788\\
8316	43.075788\\
8319	43.075788\\
8322	43.075788\\
8325	43.075788\\
8327	41.783356\\
8329	41.105432\\
8331	39.198492\\
8332	36.441768\\
8335	36.583821\\
8336	39.112245\\
8337	41.105432\\
8340	41.783356\\
8343	41.531592\\
8345	41.459297\\
8347	41.531592\\
8350	41.105432\\
8353	41.783356\\
8355	39.198492\\
8356	37.406336\\
8358	39.198492\\
8359	41.105432\\
8361	41.550617\\
8363	43.075788\\
8366	43.075788\\
8369	43.075788\\
8372	43.075788\\
8374	41.459297\\
8376	41.105432\\
8377	39.112245\\
8378	36.583821\\
8380	34.460631\\
8382	36.441768\\
8383	41.105432\\
8384	43.075788\\
8385	45.660018\\
8386	43.075788\\
8389	43.075788\\
8392	43.075788\\
8394	45.660018\\
8397	43.075788\\
8400	41.783356\\
8402	39.198492\\
8404	37.406336\\
8406	37.874351\\
8407	41.105432\\
8408	43.075788\\
8411	45.660018\\
8414	45.660018\\
8417	47.303779\\
8420	47.296169\\
8421	43.075788\\
8424	41.783356\\
8426	39.112245\\
8429	41.105432\\
8431	43.075788\\
8434	43.075788\\
8437	43.075788\\
8440	43.075788\\
8443	43.075788\\
8446	42.383278\\
8448	41.105432\\
8451	36.583821\\
8453	36.441768\\
8455	41.105432\\
8456	43.075788\\
8459	43.075788\\
8462	43.075788\\
8464	41.550617\\
8467	41.783356\\
8469	41.105432\\
8472	41.105432\\
8474	37.874351\\
8476	36.583821\\
8478	36.441768\\
8479	38.337293\\
8480	41.105432\\
8483	41.105432\\
8485	41.459297\\
8487	41.105432\\
8490	41.459297\\
8493	41.105432\\
8496	39.872611\\
8498	38.337293\\
8499	36.441768\\
8502	36.441768\\
8504	39.112245\\
8506	39.198492\\
8507	41.105432\\
8509	41.550617\\
8511	41.105432\\
8514	41.105432\\
8517	41.105432\\
8519	39.198492\\
8522	39.112245\\
8523	36.583821\\
8525	36.441768\\
8527	39.198492\\
8528	41.105432\\
8530	41.459297\\
8532	41.550617\\
8534	41.105432\\
8537	41.105432\\
8539	41.459297\\
8541	41.105432\\
8544	39.198492\\
};
\addplot [color=mycolor2,line width=2.0pt,mark size=0.3pt,only marks,mark=*,mark options={solid}]
  table[row sep=crcr]{%
8546	37.165352\\
8548	36.348545\\
8550	36.441768\\
8551	39.112245\\
8552	41.105432\\
8555	41.105432\\
8558	41.105432\\
8559	39.198492\\
8561	41.105432\\
8564	41.105432\\
8567	41.105432\\
8568	39.198492\\
8570	37.165352\\
8572	34.460631\\
8574	36.441768\\
8576	38.628376\\
8578	39.198492\\
8579	41.105432\\
8582	41.105432\\
8584	39.112245\\
8587	38.337293\\
8589	38.628376\\
8591	38.337293\\
8592	41.105432\\
8593	39.112245\\
8595	37.165352\\
8597	34.460631\\
8599	36.441768\\
8601	37.874351\\
8603	37.406336\\
8605	38.337293\\
8608	37.773519\\
8610	39.112245\\
8612	39.198492\\
8613	41.105432\\
8616	41.105432\\
8617	38.337293\\
8619	37.165352\\
8621	36.441768\\
8623	39.112245\\
8625	39.198492\\
8626	41.105432\\
8629	41.105432\\
8632	41.105432\\
8635	41.459297\\
8637	41.105432\\
8640	41.105432\\
8641	39.112245\\
8643	37.773519\\
8645	36.348545\\
8647	36.583821\\
8648	39.112245\\
8649	41.105432\\
8652	41.459297\\
8654	41.531592\\
8657	41.550617\\
8659	41.783356\\
8661	41.531592\\
8663	41.105432\\
8665	39.112245\\
8667	37.165352\\
8669	34.460631\\
8671	36.441768\\
8673	39.872611\\
8675	41.105432\\
8678	41.105432\\
8681	41.105432\\
8683	41.550617\\
8685	41.783356\\
8687	41.105432\\
8690	41.105432\\
8691	39.112245\\
8692	36.583821\\
8694	37.406336\\
8695	39.198492\\
8696	41.105432\\
8698	41.783356\\
8701	41.550617\\
8703	41.105432\\
8705	43.075788\\
8708	43.075788\\
8710	41.783356\\
8712	41.531592\\
8714	41.105432\\
8715	39.198492\\
8716	37.165352\\
8718	37.874351\\
8719	41.459297\\
8721	43.075788\\
8724	43.075788\\
8727	43.075788\\
8730	43.075788\\
8733	43.075788\\
8736	43.075788\\
8738	41.459297\\
8740	38.337293\\
8741	36.441768\\
8743	38.628376\\
8744	41.531592\\
8746	43.075788\\
8749	43.075788\\
8751	41.550617\\
8753	41.531592\\
8755	43.075788\\
8757	41.783356\\
8760	41.783356\\
};
\addlegendentry{EDR  };

\addplot [color=mycolor3,line width=2.0pt,mark size=0.3pt,only marks,mark=*,mark options={solid},forget plot]
  table[row sep=crcr]{%
1	37.954992\\
2	35.666536\\
4	33.172569\\
6	31.483214\\
8	30.097389\\
10	29.357979\\
12	30.581084\\
14	32.667633\\
16	32.820685\\
18	34.840683\\
20	33.425411\\
22	33.661626\\
24	35.660924\\
25	33.410042\\
26	31.039003\\
28	28.521642\\
30	31.191448\\
31	34.354\\
32	36.791966\\
34	39.773874\\
36	40.50787\\
38	40.975738\\
40	41.170046\\
42	41.952874\\
45	40.07928\\
47	39.773874\\
49	37.954991\\
50	35.437307\\
52	33.425411\\
54	34.354\\
55	37.433597\\
57	39.377892\\
59	39.773874\\
61	40.318348\\
63	39.445978\\
65	40.237385\\
67	39.773874\\
70	39.138304\\
72	39.773874\\
73	37.954985\\
75	36.279051\\
77	35.082312\\
79	36.853643\\
81	38.875721\\
83	39.773874\\
85	39.377944\\
87	39.790985\\
89	40.995022\\
91	40.799514\\
93	39.080727\\
95	39.773874\\
97	40.005877\\
99	37.777296\\
101	37.710522\\
104	37.710522\\
106	38.836655\\
108	39.474033\\
110	39.773874\\
113	40.170838\\
114	42.060469\\
116	41.454921\\
118	37.955052\\
120	38.34923\\
122	34.958473\\
124	33.324974\\
126	35.36147\\
127	38.774067\\
128	42.060446\\
130	41.952874\\
132	42.623795\\
134	42.224425\\
137	42.890885\\
139	43.173757\\
141	42.248102\\
143	42.052816\\
145	39.377892\\
147	36.838046\\
149	36.666703\\
151	40.74513\\
152	42.84134\\
154	43.472079\\
156	42.895618\\
158	42.54532\\
160	44.248252\\
162	45.754056\\
164	44.272713\\
166	42.895618\\
168	43.375925\\
169	41.220951\\
170	39.377892\\
172	37.345986\\
174	39.773862\\
175	42.62942\\
177	46.262407\\
179	44.248253\\
181	45.142689\\
182	47.443191\\
184	48.22564\\
186	49.717343\\
188	48.848329\\
190	47.443191\\
192	47.353401\\
193	41.952893\\
194	39.773874\\
196	37.715814\\
198	39.377892\\
199	42.060468\\
200	44.362287\\
202	44.248252\\
204	44.314704\\
206	45.242233\\
208	45.156566\\
210	45.242233\\
212	44.748805\\
214	43.459242\\
216	43.830409\\
218	41.298991\\
220	39.773874\\
222	41.952874\\
223	44.51185\\
224	47.443191\\
226	47.33453\\
228	47.190125\\
229	45.227579\\
231	46.038272\\
233	47.443191\\
236	46.020292\\
238	44.439267\\
240	45.142689\\
241	42.060473\\
243	39.785378\\
245	39.233115\\
247	39.773874\\
249	42.280364\\
251	43.597554\\
253	43.713932\\
255	43.021303\\
257	43.793375\\
259	45.010937\\
261	42.80906\\
263	44.248238\\
264	42.060469\\
267	41.516701\\
269	39.806232\\
271	39.989951\\
273	41.952874\\
275	42.060469\\
277	41.899427\\
279	40.133107\\
281	41.952874\\
283	42.895619\\
285	41.663981\\
286	39.773874\\
288	41.952874\\
289	39.377894\\
291	37.684974\\
293	36.933598\\
295	42.060469\\
297	43.713932\\
299	45.142689\\
301	44.584216\\
303	44.582723\\
305	45.142689\\
307	46.300825\\
309	44.247704\\
311	43.173757\\
313	41.497526\\
315	39.377892\\
317	39.16332\\
319	42.895618\\
320	45.242232\\
322	46.071578\\
324	46.72764\\
326	45.724363\\
328	47.001994\\
330	47.443191\\
332	47.071236\\
334	44.586684\\
336	44.543178\\
338	42.060469\\
340	39.773874\\
342	41.206076\\
344	45.080572\\
346	44.942141\\
348	45.142689\\
350	45.142688\\
352	44.626899\\
354	45.242233\\
356	45.142688\\
358	42.787447\\
360	42.895618\\
362	39.773874\\
364	37.954992\\
366	39.377892\\
367	42.060469\\
369	43.182661\\
371	43.642027\\
373	44.786411\\
375	45.242233\\
377	45.142688\\
379	45.197127\\
381	44.24817\\
383	42.825635\\
385	41.053827\\
387	38.051285\\
389	37.954996\\
391	42.060469\\
393	43.173757\\
395	43.597554\\
397	42.576751\\
399	42.895618\\
401	44.297832\\
403	44.703077\\
405	43.542957\\
407	42.895618\\
409	41.952874\\
410	39.773874\\
412	38.237189\\
414	37.954992\\
416	39.377892\\
418	41.100386\\
420	40.060957\\
422	39.774937\\
424	40.377183\\
426	42.224425\\
428	42.060469\\
430	41.952874\\
432	42.060469\\
433	40.17228\\
435	37.888327\\
437	37.572508\\
439	37.954991\\
441	39.391446\\
443	41.952874\\
445	42.060469\\
447	42.045948\\
449	42.060469\\
451	43.113568\\
453	42.224425\\
455	42.254506\\
457	41.952874\\
459	39.687312\\
461	39.377902\\
462	41.148497\\
463	43.325711\\
464	45.644542\\
466	46.987057\\
468	47.443191\\
471	47.443191\\
474	47.443191\\
476	45.790167\\
478	43.597553\\
480	43.145747\\
482	41.878818\\
483	39.874197\\
485	39.773876\\
486	41.863464\\
487	43.597555\\
488	46.11508\\
490	45.074781\\
492	45.242233\\
494	45.466267\\
496	45.944605\\
498	47.443191\\
500	46.541722\\
502	43.597553\\
504	43.041527\\
506	39.969575\\
508	38.875221\\
510	39.773874\\
511	42.224425\\
512	44.428691\\
514	43.713933\\
516	43.173708\\
518	43.680961\\
520	44.749072\\
522	46.942722\\
524	45.727247\\
526	43.81593\\
528	44.248252\\
529	42.224425\\
531	41.024678\\
533	39.783672\\
534	41.838474\\
536	45.242233\\
538	45.150597\\
540	45.142689\\
543	45.142689\\
546	45.97114\\
548	44.248252\\
550	42.168912\\
552	42.224425\\
554	39.773874\\
556	38.499613\\
558	40.245474\\
559	43.173757\\
560	45.370514\\
562	45.005257\\
564	44.248234\\
566	43.597554\\
568	43.881207\\
570	45.215864\\
572	44.248252\\
574	42.060469\\
576	42.60593\\
578	41.344607\\
580	39.255625\\
582	37.954992\\
584	39.647446\\
585	41.415714\\
587	42.060469\\
590	42.060469\\
592	41.579303\\
594	42.060469\\
596	41.470819\\
598	38.677136\\
600	39.447169\\
601	37.576291\\
602	34.944541\\
604	34.468179\\
606	35.244981\\
608	37.604354\\
610	39.377892\\
612	39.773874\\
615	38.974557\\
617	38.489077\\
619	38.720465\\
621	39.773874\\
623	39.377892\\
625	37.710815\\
627	35.319786\\
629	35.736398\\
630	37.710522\\
631	40.883947\\
632	42.840966\\
634	42.895618\\
637	42.895618\\
639	42.908659\\
641	43.579779\\
643	43.828004\\
645	42.895724\\
647	42.060469\\
649	40.642385\\
651	37.954992\\
653	37.710522\\
655	41.952874\\
657	42.546376\\
659	42.698482\\
661	42.463658\\
663	43.173759\\
665	43.502342\\
667	44.000033\\
669	43.057017\\
671	42.224425\\
673	40.810416\\
675	38.856634\\
677	37.954992\\
679	42.060469\\
681	43.173757\\
683	42.22439\\
685	42.060469\\
688	42.895618\\
690	44.056649\\
692	43.969621\\
694	42.224425\\
696	42.895618\\
698	40.942138\\
700	39.773874\\
702	40.821013\\
703	43.173757\\
704	45.242235\\
706	45.201005\\
708	44.972934\\
710	44.248252\\
712	44.857567\\
714	47.120988\\
716	45.242232\\
718	44.248164\\
720	44.584215\\
722	42.060469\\
724	41.307178\\
726	42.060469\\
727	44.248252\\
728	46.499103\\
729	44.619342\\
731	43.17381\\
733	42.060469\\
736	42.059302\\
738	42.895618\\
740	42.060469\\
742	39.868013\\
744	38.458676\\
746	35.866009\\
748	34.347342\\
750	33.958525\\
752	34.672312\\
754	38.309342\\
756	38.458676\\
758	38.309342\\
760	37.659266\\
762	38.309342\\
765	38.309342\\
767	38.159708\\
769	36.226675\\
771	34.651729\\
773	34.347342\\
775	34.354\\
777	34.570008\\
779	35.866011\\
781	35.801503\\
783	35.748443\\
785	36.226677\\
786	38.30934\\
788	38.458676\\
790	38.309342\\
792	38.420943\\
794	37.319391\\
796	36.226675\\
798	38.050755\\
799	39.81534\\
800	41.867482\\
802	40.608009\\
804	39.070009\\
806	39.323343\\
808	40.502217\\
810	43.211992\\
812	42.811845\\
814	40.30201\\
817	38.688025\\
819	37.017576\\
821	36.226675\\
823	39.559552\\
825	40.836656\\
827	40.60801\\
829	39.663595\\
831	39.42859\\
833	40.608008\\
835	42.487023\\
837	40.722623\\
839	39.323343\\
841	38.211342\\
842	36.226675\\
844	34.570008\\
846	35.866024\\
847	38.309342\\
849	40.865433\\
851	41.08334\\
853	40.30201\\
855	39.535181\\
857	40.591585\\
859	41.383674\\
861	39.815343\\
863	39.323343\\
865	38.309342\\
866	36.423213\\
868	35.68821\\
870	36.226675\\
871	38.363294\\
873	39.338893\\
875	39.815343\\
877	40.161514\\
879	40.657469\\
881	41.116676\\
883	41.207343\\
885	39.323342\\
887	39.070011\\
889	38.198098\\
890	36.226675\\
892	34.731966\\
894	35.866009\\
895	38.309342\\
897	39.323343\\
899	40.548659\\
901	40.713562\\
903	40.607983\\
905	39.709343\\
907	40.608476\\
909	39.734521\\
911	39.323345\\
913	38.243083\\
914	36.226675\\
916	34.354\\
918	34.347342\\
920	34.570008\\
922	36.561009\\
924	38.309342\\
927	37.500413\\
929	36.889887\\
931	38.21186\\
933	35.866009\\
935	36.199939\\
937	35.490099\\
939	33.945998\\
941	33.213651\\
943	33.678019\\
945	34.347342\\
947	35.294265\\
949	35.866009\\
951	34.570008\\
953	35.231259\\
955	38.309342\\
957	38.211343\\
959	37.636155\\
961	37.959126\\
963	36.226675\\
966	38.193725\\
968	43.147835\\
970	43.21201\\
972	42.984832\\
974	43.21201\\
977	43.21201\\
979	43.66001\\
981	42.524666\\
983	41.207343\\
985	39.32334\\
987	37.897504\\
989	36.622423\\
991	39.447137\\
993	40.30201\\
995	39.348469\\
997	39.070953\\
999	39.709343\\
1001	40.602888\\
1003	41.91601\\
1005	40.873973\\
1007	40.421441\\
1009	38.340781\\
1011	36.226675\\
1014	37.900353\\
1016	42.951786\\
1017	41.116684\\
1019	38.500819\\
1021	38.309343\\
1023	39.070009\\
1025	39.815343\\
1027	41.116681\\
1029	39.678933\\
1031	39.070009\\
1033	37.576789\\
1035	35.415038\\
1037	35.328924\\
1039	38.866484\\
1041	40.30201\\
1043	41.932278\\
1045	43.21201\\
1048	43.21201\\
1051	43.21201\\
1053	41.207347\\
1055	41.116676\\
1057	39.323343\\
1059	38.309342\\
1062	38.510815\\
1063	41.155746\\
1065	41.095249\\
1067	40.30201\\
1069	40.418541\\
1071	41.116676\\
1073	40.905636\\
1075	40.380546\\
1077	39.180731\\
1079	38.309342\\
1081	36.332409\\
1083	34.388296\\
1085	34.347342\\
1087	34.570008\\
1089	35.866009\\
1091	36.710838\\
1093	35.866072\\
1095	35.706164\\
1097	36.226675\\
1099	38.171379\\
1101	36.226675\\
1104	37.403603\\
1106	34.813689\\
1108	34.347342\\
1111	34.354\\
1113	34.879251\\
1115	35.488827\\
1117	35.242139\\
1119	34.407702\\
1121	35.691434\\
1123	38.309342\\
1125	39.070007\\
1127	39.070009\\
1129	38.309342\\
1131	36.226675\\
1133	35.893613\\
1135	39.589444\\
1136	41.689459\\
1138	40.847576\\
1140	40.857804\\
1142	39.729329\\
1144	40.043516\\
1146	40.302011\\
1147	42.371379\\
1149	41.207343\\
1151	39.884793\\
1153	38.309342\\
1155	36.226675\\
1158	38.203369\\
1160	41.116676\\
1163	41.207343\\
1165	41.831154\\
1167	42.077541\\
1169	42.429437\\
1171	43.21201\\
1173	41.648528\\
1175	40.737062\\
1177	38.661853\\
1179	37.324805\\
1181	37.045714\\
1183	40.538979\\
1185	41.310164\\
1187	42.828563\\
1189	41.969708\\
1191	41.938\\
1193	42.518039\\
1195	43.21201\\
1197	43.015665\\
1198	41.207343\\
1200	41.116676\\
1201	38.698198\\
1203	36.459906\\
1205	35.866008\\
1207	38.584163\\
1209	39.07001\\
1211	39.447859\\
1213	40.302015\\
1215	40.497315\\
1217	40.906705\\
1219	41.523286\\
1221	40.30201\\
1223	40.033547\\
1225	38.457887\\
1227	36.226711\\
1229	36.226675\\
1230	38.211342\\
1232	40.870927\\
1234	40.30201\\
1236	40.420205\\
1238	39.096296\\
1240	39.070009\\
1242	39.957388\\
1244	40.86213\\
1246	38.458676\\
1248	39.070009\\
1250	36.226675\\
1252	34.570008\\
1255	35.308887\\
1257	37.243223\\
1259	38.458676\\
1261	38.309342\\
1263	37.307727\\
1265	38.163793\\
1267	39.069994\\
1269	38.309342\\
1271	37.854246\\
1273	36.872109\\
1275	34.570008\\
1277	34.347342\\
1280	34.347342\\
1282	33.595294\\
1284	34.014362\\
1286	33.552904\\
1288	33.91619\\
1290	35.720424\\
1292	38.211342\\
1294	37.599082\\
1296	37.541217\\
1298	34.570008\\
1300	34.318556\\
1302	35.146719\\
1303	38.309342\\
1305	38.484147\\
1307	38.309342\\
1309	35.860599\\
1311	37.694867\\
1313	38.458676\\
1315	40.405633\\
1317	39.932677\\
1319	39.00475\\
1321	36.520672\\
1323	34.570008\\
1325	34.441188\\
1327	38.309343\\
1328	40.302009\\
1330	39.070009\\
1332	38.458676\\
1334	39.070009\\
1336	39.324398\\
1338	40.608179\\
1340	41.116677\\
1342	39.323403\\
1344	39.070009\\
1346	37.563022\\
1348	36.226675\\
1350	38.212675\\
1351	40.045833\\
1353	41.116676\\
1356	40.969954\\
1358	40.123311\\
1360	40.30201\\
1362	40.302011\\
1364	42.276763\\
1366	40.30201\\
1368	40.295336\\
1369	38.309343\\
1371	36.226675\\
1374	36.768097\\
1375	39.070009\\
1377	38.568157\\
1379	39.121713\\
1381	39.905491\\
1383	40.60801\\
1385	40.9903\\
1387	41.817237\\
1389	41.596153\\
1391	41.242527\\
1393	39.064058\\
1395	37.518649\\
1397	37.306626\\
1399	41.116676\\
1401	41.116677\\
1403	42.633974\\
1405	43.21201\\
1408	43.21201\\
1411	43.211682\\
1413	42.104567\\
1415	41.736223\\
1417	38.320484\\
1419	36.42579\\
1421	36.332608\\
1423	36.42579\\
1425	36.601311\\
1427	37.516059\\
1429	38.611439\\
1431	37.418222\\
1433	37.389936\\
1435	38.930572\\
1437	37.67949\\
1439	37.857744\\
1441	37.149056\\
1443	35.456125\\
1445	34.445521\\
1448	34.445521\\
1450	33.821961\\
1452	32.870307\\
1454	32.251326\\
1456	32.658588\\
1458	34.370766\\
1460	35.891153\\
1462	35.711091\\
1464	35.964542\\
1466	33.425411\\
1468	32.832483\\
1470	34.354002\\
1471	36.521852\\
1473	37.659021\\
1475	37.445182\\
1477	36.853746\\
1479	37.389935\\
1481	38.164173\\
1483	39.113396\\
1485	39.095096\\
1488	39.095096\\
1490	37.149055\\
1492	36.42579\\
1494	37.534792\\
1495	39.460974\\
1497	40.551119\\
1499	41.064109\\
1501	40.231062\\
1503	40.547891\\
1505	41.087409\\
1507	41.441119\\
1509	41.087409\\
1512	41.087409\\
1514	38.611439\\
1516	37.202261\\
1518	38.320484\\
1519	41.087408\\
1521	41.087409\\
1523	40.972448\\
1525	39.181305\\
1528	39.84756\\
1530	41.087409\\
1533	41.151156\\
1535	41.087409\\
1537	39.181304\\
1539	36.425832\\
1541	36.42579\\
1543	39.102723\\
1545	38.839507\\
1547	38.600763\\
1549	37.149057\\
1551	36.608076\\
1553	37.891247\\
1555	39.272195\\
1557	41.086913\\
1558	39.095096\\
1560	38.733692\\
1561	36.567781\\
1563	34.445521\\
1565	34.52733\\
1566	36.332608\\
1568	37.326335\\
1570	37.389646\\
1572	37.389932\\
1574	36.598388\\
1576	37.427753\\
1578	39.095096\\
1580	39.181305\\
1582	38.611437\\
1584	39.095096\\
1586	36.42579\\
1588	34.445521\\
1591	34.460059\\
1593	35.469955\\
1595	34.445521\\
1597	33.929568\\
1599	33.03593\\
1601	34.44552\\
1602	36.291906\\
1604	36.42579\\
1606	34.445521\\
1608	35.155562\\
1610	33.124355\\
1612	32.658588\\
1614	32.658587\\
1616	32.856087\\
1618	32.870307\\
1620	32.658588\\
1622	31.48067\\
1624	32.187755\\
1626	33.802358\\
1628	36.42579\\
1630	35.848752\\
1632	35.589091\\
1634	34.26237\\
1636	33.616992\\
1638	35.679237\\
1639	38.37366\\
1641	38.321327\\
1643	37.86117\\
1645	36.87803\\
1647	36.86918\\
1649	37.756958\\
1651	40.811093\\
1653	39.095096\\
1655	38.320484\\
1657	36.42579\\
1658	34.445521\\
1660	34.354087\\
1662	36.332608\\
1664	37.896287\\
1666	38.611439\\
1668	39.181305\\
1670	37.149057\\
1672	36.787555\\
1674	38.47918\\
1676	39.181305\\
1678	38.146561\\
1680	38.074337\\
1682	36.226802\\
1684	34.93585\\
1686	36.425791\\
1687	39.095096\\
1689	38.320484\\
1691	38.290727\\
1693	36.983767\\
1695	36.567782\\
1697	37.902783\\
1699	39.188714\\
1700	41.087409\\
1702	39.181305\\
1704	38.771034\\
1706	36.42579\\
1708	36.160417\\
1710	36.714235\\
1711	39.269132\\
1713	40.223069\\
1715	39.095096\\
1717	37.857722\\
1719	37.756957\\
1721	38.161006\\
1723	39.095096\\
1725	39.819969\\
1727	39.181305\\
1729	37.149057\\
1731	36.332608\\
1734	36.567781\\
1735	39.095096\\
1737	38.905545\\
1739	38.649094\\
1741	37.389935\\
1743	36.523582\\
1745	37.565543\\
1747	39.181304\\
1749	39.095096\\
1751	39.91413\\
1753	37.149057\\
1755	36.425705\\
1757	35.613567\\
1759	36.42579\\
1761	37.149057\\
1763	38.320475\\
1765	38.237512\\
1767	36.567875\\
1769	36.939191\\
1771	38.415074\\
1773	38.533849\\
1775	37.389935\\
1777	36.42579\\
1779	34.382157\\
1781	34.402031\\
1783	34.445521\\
1785	35.156503\\
1787	36.42579\\
1790	35.325501\\
1792	34.445521\\
1794	36.42579\\
1796	39.063293\\
1798	37.756956\\
1800	37.581834\\
1802	36.332608\\
1804	35.30622\\
1806	36.953409\\
1807	39.713341\\
1809	40.991566\\
1811	41.399794\\
1813	41.087409\\
1816	41.087409\\
1818	41.765036\\
1820	41.980247\\
1822	41.087409\\
1825	39.095096\\
1827	37.148873\\
1829	37.149055\\
1831	41.087408\\
1833	41.087409\\
1836	41.249351\\
1838	41.087409\\
1841	41.087409\\
1844	41.087409\\
1846	39.181305\\
1848	39.095096\\
1850	36.42579\\
1853	36.42579\\
1855	40.038771\\
1857	39.855128\\
1859	41.087409\\
1862	41.010933\\
1864	40.759825\\
1866	41.513382\\
1868	41.706203\\
1870	41.513382\\
1872	40.768615\\
1874	37.857745\\
1876	37.149057\\
1879	38.320484\\
1881	37.389936\\
1883	37.149057\\
1885	36.425804\\
1887	35.903742\\
1889	36.567786\\
1891	38.026491\\
1893	39.181305\\
1895	38.62507\\
1897	36.425791\\
1899	35.080357\\
1901	36.332608\\
1903	39.095096\\
1904	41.087409\\
1907	41.08741\\
1909	41.439352\\
1911	40.781175\\
1913	39.925665\\
1915	39.181305\\
1917	39.712649\\
1919	39.855131\\
1921	37.543478\\
1923	35.779242\\
1925	34.445521\\
1927	34.541571\\
1929	35.740758\\
1931	36.587868\\
1933	37.011437\\
1935	36.42579\\
1938	36.663975\\
1940	37.756957\\
1942	37.578135\\
1944	37.756957\\
1946	36.42579\\
1948	34.817519\\
1950	35.5282\\
1952	35.556666\\
1954	36.42579\\
1956	36.567781\\
1958	36.42579\\
1960	36.425788\\
1962	37.389935\\
1964	38.569716\\
1966	37.981855\\
1968	38.868593\\
1970	36.42579\\
1973	36.42579\\
1975	40.552943\\
1977	40.808677\\
1979	40.092608\\
1981	39.85516\\
1983	39.653808\\
1985	41.087409\\
1988	41.765035\\
1990	41.087409\\
1993	39.095096\\
1995	37.389935\\
1998	38.611439\\
1999	41.087409\\
2002	41.087409\\
2004	39.334575\\
2006	39.110201\\
2008	39.977869\\
2010	41.087409\\
2012	41.259388\\
2014	41.087406\\
2016	39.855129\\
2018	37.619713\\
2020	36.567781\\
2022	38.478895\\
2023	41.087409\\
2026	41.087408\\
2028	41.087409\\
2030	41.788675\\
2032	42.992896\\
2034	43.056901\\
2036	44.526883\\
2038	43.517669\\
2040	43.24967\\
2042	43.056885\\
2044	41.765036\\
2046	43.056901\\
2048	43.354647\\
2050	43.056902\\
2053	43.056901\\
2056	43.056901\\
2058	44.677572\\
2060	45.639999\\
2062	43.418007\\
2064	43.056905\\
2066	43.056901\\
2068	41.765036\\
2070	43.056901\\
2072	43.056902\\
2075	43.056901\\
2078	43.056901\\
2081	43.056901\\
2083	43.45613\\
2085	43.056902\\
2087	43.056901\\
2089	42.469129\\
2091	41.087409\\
2094	41.087409\\
2097	41.441119\\
2099	41.087409\\
2101	41.087406\\
2103	39.380894\\
2105	41.087409\\
2107	41.854168\\
2109	43.056901\\
2111	41.727247\\
2113	41.087409\\
2115	39.854174\\
2117	39.59845\\
2119	40.462214\\
2121	41.087409\\
2123	40.790953\\
2125	39.095096\\
2127	38.82831\\
2129	40.593795\\
2131	41.513382\\
2133	41.765036\\
2135	41.441118\\
2137	41.087409\\
2139	40.964109\\
2141	41.513382\\
2143	43.371432\\
2145	43.056901\\
2148	42.814716\\
2150	42.575148\\
2152	43.056901\\
2155	43.056901\\
2157	43.056902\\
2159	43.056901\\
2160	38.585176\\
2163	38.585176\\
2166	40.862582\\
2168	40.900002\\
2170	40.57018\\
2172	39.379661\\
2174	39.380676\\
2176	40.084881\\
2178	41.06417\\
2180	40.900003\\
2183	40.883917\\
2184	38.585176\\
2186	37.964927\\
2188	37.597164\\
2190	39.902497\\
2192	40.900002\\
2194	40.864782\\
2196	39.902497\\
2198	39.540167\\
2200	39.497987\\
2202	40.900002\\
2205	40.900002\\
2207	39.673003\\
2209	38.297906\\
2211	37.427479\\
2213	38.147208\\
2215	40.658005\\
2217	39.902497\\
2220	38.998152\\
2223	38.824168\\
2225	40.900002\\
2227	41.398755\\
2229	40.900002\\
2231	40.085569\\
2233	38.585176\\
2235	37.712696\\
2237	38.585176\\
2239	40.901208\\
2241	41.482076\\
2243	42.181752\\
2245	40.900002\\
2247	39.902497\\
2249	40.900002\\
2251	40.667313\\
2253	40.015412\\
2255	40.242924\\
2257	38.585176\\
2259	37.427479\\
2261	37.500102\\
2263	38.585176\\
2266	38.585176\\
2269	38.585176\\
2271	37.955538\\
2273	38.585173\\
2275	38.585176\\
2278	38.585176\\
2280	38.19134\\
2282	36.820227\\
2284	36.820224\\
2286	36.820227\\
2289	36.820228\\
2291	37.219003\\
2293	37.219002\\
2295	37.137202\\
2297	37.427479\\
2299	38.585176\\
2302	38.585176\\
2304	37.427479\\
2306	36.820227\\
2309	37.396938\\
2311	38.585176\\
2313	40.319345\\
2315	40.601296\\
2317	39.56035\\
2319	39.310548\\
2321	39.902494\\
2323	39.00867\\
2325	40.097726\\
2327	38.58518\\
2329	37.399516\\
2331	36.899764\\
2333	38.146249\\
2335	38.981568\\
2337	39.673003\\
2339	39.700178\\
2341	38.585176\\
2344	38.585252\\
2346	38.68604\\
2348	40.900002\\
2351	40.899794\\
2352	38.585176\\
2355	38.585175\\
2357	38.585176\\
2358	40.900002\\
2360	40.90117\\
2362	40.900002\\
2364	39.137628\\
2366	38.585176\\
2368	38.720911\\
2370	40.900002\\
2372	40.747196\\
2374	40.900002\\
2376	38.720006\\
2378	38.585176\\
2381	38.585176\\
2382	40.900002\\
2384	41.355215\\
2386	40.900002\\
2388	40.155923\\
2390	40.033393\\
2392	40.48144\\
2394	41.180492\\
2396	40.900002\\
2399	40.900002\\
2400	38.685801\\
2402	38.585176\\
2405	38.585176\\
2407	40.900002\\
2410	40.900002\\
2412	40.394887\\
2414	39.195\\
2416	39.672948\\
2418	40.509217\\
2420	40.310191\\
2422	40.899995\\
2424	38.585176\\
2426	38.585175\\
2428	38.229081\\
2430	38.585176\\
2433	38.585186\\
2435	38.585176\\
2437	37.427479\\
2439	37.427478\\
2441	38.585176\\
2444	38.585176\\
2447	38.585176\\
2448	36.820227\\
2450	35.034828\\
2452	34.988033\\
2454	35.112272\\
2456	36.748167\\
2458	36.228685\\
2460	35.986835\\
2462	34.35407\\
2464	34.561372\\
2466	36.820227\\
2468	37.201961\\
2470	37.427479\\
2472	35.831872\\
2474	34.614343\\
2476	34.534875\\
2478	37.137202\\
2480	38.585175\\
2482	38.429034\\
2484	37.600907\\
2486	37.427478\\
2488	37.427473\\
2490	38.585176\\
2493	38.585176\\
2496	37.42748\\
2498	36.820227\\
2500	37.064761\\
2502	38.585176\\
2504	39.317978\\
2506	38.585176\\
2509	38.585176\\
2512	38.585176\\
2515	38.585176\\
2517	38.585179\\
2519	38.585176\\
2521	37.588258\\
2523	37.201961\\
2525	38.147205\\
2527	38.585198\\
2529	38.585176\\
2532	38.58444\\
2534	37.427479\\
2536	38.16672\\
2538	38.585176\\
2540	38.585214\\
2542	38.997964\\
2544	38.585176\\
2546	36.902789\\
2548	36.820227\\
2550	38.585176\\
2553	38.585176\\
2556	37.690519\\
2558	37.201961\\
2560	37.427479\\
2562	38.585176\\
2565	38.585176\\
2568	37.854441\\
2570	36.820227\\
2573	36.820227\\
2575	38.585176\\
2578	38.585176\\
2580	37.965087\\
2582	37.427479\\
2584	37.219002\\
2586	38.073457\\
2588	37.436873\\
2590	38.585176\\
2592	37.427479\\
2594	36.820227\\
2597	36.820227\\
2600	37.034263\\
2602	37.20196\\
2604	36.820227\\
2606	35.112084\\
2609	36.258283\\
2611	36.820227\\
2614	36.820227\\
2616	35.799614\\
2618	34.340665\\
2620	33.795048\\
2622	34.340665\\
2624	35.034828\\
2626	35.27663\\
2628	35.112084\\
2630	34.601402\\
2632	34.354\\
2634	36.752253\\
2636	36.820227\\
2638	37.13487\\
2640	36.820227\\
2642	35.112084\\
2644	35.034828\\
2646	35.166199\\
2648	35.173366\\
2650	35.574912\\
2652	35.034829\\
2654	34.6014\\
2656	35.037568\\
2657	36.820227\\
2659	37.220445\\
2661	37.137191\\
2663	37.427479\\
2665	36.820227\\
2667	35.563461\\
2669	36.820227\\
2670	38.585176\\
2673	38.585176\\
2676	38.585176\\
2679	38.585176\\
2682	38.585181\\
2684	38.585176\\
2686	38.624394\\
2688	38.252204\\
2690	36.820227\\
2693	37.427479\\
2695	38.585176\\
2698	38.585176\\
2701	38.585176\\
2704	38.585176\\
2706	38.998152\\
2708	38.585176\\
2710	39.287794\\
2712	38.04519\\
2714	36.820227\\
2717	37.21901\\
2719	38.998152\\
2721	39.902497\\
2724	39.174089\\
2726	38.585176\\
2728	38.74963\\
2730	39.414571\\
2732	38.585176\\
2735	38.359186\\
2737	36.774747\\
2739	35.158896\\
2741	36.820227\\
2743	38.585176\\
2746	37.624847\\
2748	36.820227\\
2750	37.201961\\
2752	37.427479\\
2754	38.585176\\
2757	38.585176\\
2759	38.137364\\
2761	36.725454\\
2763	35.112084\\
2766	35.112084\\
2768	36.776723\\
2770	36.589513\\
2772	35.112085\\
2774	35.034828\\
2776	36.09208\\
2778	36.820227\\
2780	37.170529\\
2782	36.820227\\
2784	35.715926\\
2786	34.340665\\
2788	34.181235\\
2790	34.278045\\
2792	34.485188\\
2794	35.034828\\
2796	35.112084\\
2798	34.549081\\
2800	34.763555\\
2802	36.820224\\
2804	36.820227\\
2807	36.820227\\
2809	35.112084\\
2811	34.519096\\
2813	36.820227\\
2815	38.585176\\
2818	38.585176\\
2821	38.585176\\
2824	38.585176\\
2826	38.585181\\
2828	38.585176\\
2831	38.585176\\
2833	36.820227\\
2836	36.820227\\
2838	38.585176\\
2840	38.585178\\
2842	38.82639\\
2844	38.998151\\
2846	39.674572\\
2848	38.99851\\
2850	38.585185\\
2852	38.585176\\
2855	38.585176\\
2857	36.820227\\
2860	36.820227\\
2862	38.585176\\
2865	38.585187\\
2867	38.585176\\
2870	38.585176\\
2872	38.578633\\
2874	38.585176\\
2877	38.585176\\
2880	35.551439\\
2882	34.479574\\
2885	34.479574\\
2887	34.256966\\
2889	34.479574\\
2892	34.479574\\
2895	34.479574\\
2897	34.852999\\
2899	36.018316\\
2901	36.132325\\
2903	36.132326\\
2905	34.479574\\
2907	33.983015\\
2909	34.479563\\
2911	34.479574\\
2913	34.776399\\
2915	34.479574\\
2917	33.776542\\
2919	33.425407\\
2921	34.479574\\
2923	34.741164\\
2925	34.837041\\
2927	34.852999\\
2929	33.47661\\
2931	32.880017\\
2933	33.425402\\
2935	34.353998\\
2937	34.479574\\
2939	34.354\\
2941	33.425411\\
2943	33.192984\\
2945	34.479574\\
2947	34.776399\\
2949	34.479574\\
2952	34.776399\\
2954	34.354\\
2957	34.354\\
2959	34.479227\\
2961	34.354\\
2963	33.425411\\
2965	32.807673\\
2967	32.157637\\
2969	34.259294\\
2971	35.048223\\
2974	35.232245\\
2976	34.479574\\
2978	33.68531\\
2980	34.327706\\
2982	35.801837\\
2984	36.132326\\
2987	35.551443\\
2989	35.048223\\
2992	35.559348\\
2994	36.132326\\
2997	36.132326\\
2999	36.107005\\
3001	34.354\\
3003	33.482416\\
3005	34.479574\\
3007	36.132326\\
3009	36.299728\\
3011	36.132326\\
3014	36.132326\\
3017	36.132326\\
3020	36.132326\\
3023	36.132326\\
3025	34.479575\\
3027	34.479574\\
3029	34.837041\\
3031	36.132326\\
3034	36.132326\\
3037	36.132326\\
3040	36.132326\\
3043	36.132326\\
3046	36.132329\\
3048	35.048223\\
3050	34.479574\\
3053	34.479574\\
3055	36.132326\\
3058	36.13275\\
3060	36.132336\\
3062	36.519047\\
3064	36.604545\\
3066	37.066351\\
3068	36.132326\\
3071	35.80721\\
3073	34.479574\\
3075	34.234779\\
3077	34.479574\\
3079	36.132326\\
3082	36.132326\\
3084	35.933955\\
3086	34.852999\\
3088	34.837041\\
3090	35.080297\\
3092	34.831617\\
3094	36.132325\\
3096	35.048223\\
3098	34.479574\\
3100	34.179583\\
3102	34.354\\
3104	34.852999\\
3106	35.701234\\
3108	34.852999\\
3110	34.479574\\
3112	34.479555\\
3114	34.479574\\
3116	34.354\\
3118	34.479574\\
3120	33.206185\\
3122	31.376742\\
3124	31.149202\\
3126	30.68681\\
3128	31.376742\\
3130	31.684738\\
3132	31.571822\\
3134	30.68681\\
3136	30.915215\\
3138	32.880017\\
3140	33.227202\\
3142	34.028441\\
3144	32.807673\\
3146	31.769317\\
3148	32.355195\\
3150	34.479574\\
3152	35.547123\\
3154	36.087184\\
3156	35.165443\\
3158	35.863179\\
3160	36.132326\\
3163	36.132326\\
3166	36.519044\\
3168	35.722196\\
3170	34.479574\\
3173	34.852999\\
3175	36.132326\\
3177	37.150998\\
3179	36.132327\\
3181	36.132326\\
3184	36.132326\\
3186	36.159544\\
3188	36.132326\\
3190	36.704051\\
3192	36.132326\\
3194	34.776399\\
3196	34.837041\\
3198	36.132326\\
3200	36.396789\\
3202	36.132326\\
3205	36.132326\\
3208	36.132326\\
3211	36.132326\\
3214	36.132328\\
3216	35.217495\\
3218	34.479574\\
3220	34.652879\\
3222	36.132326\\
3224	36.231617\\
3226	36.132332\\
3228	36.132326\\
3230	35.142146\\
3232	35.851842\\
3234	36.132326\\
3237	36.132326\\
3240	35.551442\\
3242	34.479574\\
3244	34.479575\\
3246	36.132326\\
3248	36.132327\\
3250	36.132326\\
3252	35.513963\\
3254	35.048223\\
3257	35.723921\\
3259	36.132326\\
3262	36.132326\\
3265	35.048219\\
3267	34.479574\\
3270	34.479574\\
3272	35.048223\\
3274	34.837041\\
3276	34.479574\\
3278	33.877829\\
3280	34.479574\\
3282	35.048223\\
3284	35.722196\\
3286	35.610918\\
3288	35.048223\\
3290	34.13372\\
3292	33.425411\\
3294	33.258942\\
3296	33.651263\\
3298	33.425411\\
3300	32.880017\\
3302	32.228877\\
3304	32.807673\\
3306	34.479574\\
3309	34.479574\\
3312	34.354\\
3314	32.880017\\
3316	33.425411\\
3318	34.776399\\
3320	36.132326\\
3323	36.132326\\
3325	35.840323\\
3327	36.132325\\
3329	36.132326\\
3332	36.132326\\
3335	35.720771\\
3337	34.479574\\
3339	34.354\\
3341	34.479574\\
3343	36.132326\\
3346	36.132326\\
3349	36.132326\\
3352	36.132326\\
3354	36.519049\\
3356	36.132326\\
3359	36.132326\\
3361	35.151377\\
3363	34.776399\\
3365	35.048223\\
3367	36.132326\\
3369	36.736137\\
3371	37.319828\\
3373	36.390349\\
3375	36.982669\\
3377	36.55186\\
3379	36.132327\\
3381	36.132326\\
3384	34.479574\\
3386	32.752628\\
3388	32.807673\\
3390	34.479574\\
3392	35.722222\\
3394	35.778187\\
3396	35.048223\\
3398	35.130688\\
3400	36.132326\\
3403	36.132326\\
3406	36.132326\\
3408	34.633212\\
3410	33.995808\\
3412	34.354\\
3414	35.024846\\
3416	36.132326\\
3419	36.132326\\
3422	36.132326\\
3424	36.044281\\
3426	36.132326\\
3428	35.722196\\
3430	35.722256\\
3432	35.048223\\
3434	34.354001\\
3436	33.445474\\
3438	33.325648\\
3440	34.479574\\
3443	34.354\\
3445	33.211418\\
3447	32.807673\\
3449	34.354\\
3451	34.479574\\
3454	34.479574\\
3456	34.354\\
3458	32.880017\\
3460	32.870003\\
3462	32.807674\\
3464	33.562764\\
3466	33.425411\\
3468	32.807681\\
3470	32.157637\\
3472	33.019932\\
3474	34.479574\\
3477	34.479574\\
3480	34.448984\\
3482	32.880017\\
3484	33.353886\\
3486	34.689603\\
3488	36.132327\\
3490	37.167134\\
3492	36.341982\\
3494	36.132326\\
3497	36.132326\\
3500	36.132326\\
3503	35.552623\\
3505	34.479574\\
3508	34.479574\\
3510	35.048344\\
3512	36.132327\\
3514	37.365905\\
3516	36.132326\\
3519	36.519049\\
3522	36.519049\\
3524	36.132326\\
3527	36.030119\\
3529	34.479574\\
3532	34.479574\\
3534	35.615279\\
3536	36.132326\\
3538	37.365905\\
3541	37.150999\\
3543	36.132326\\
3545	36.132366\\
3547	36.132326\\
3550	36.132326\\
3552	35.048229\\
3554	34.479574\\
3556	34.354\\
3558	33.425419\\
3560	34.354\\
3562	34.479574\\
3565	34.479574\\
3567	33.557179\\
3569	34.098285\\
3571	34.479574\\
3573	34.836655\\
3575	34.479574\\
3577	33.425411\\
3579	32.880017\\
3581	33.425411\\
3583	34.479574\\
3586	34.479574\\
3588	34.479561\\
3590	34.172189\\
3592	34.16688\\
3594	34.479574\\
3596	34.526578\\
3598	34.776399\\
3600	34.479574\\
3602	33.434907\\
3604	33.445474\\
3606	33.49583\\
3608	34.479574\\
3610	34.47957\\
3612	34.408794\\
3614	33.652734\\
3616	34.354\\
3618	34.479574\\
3620	34.776399\\
3622	34.837041\\
3624	32.087009\\
3626	31.166668\\
3629	31.166668\\
3632	31.166668\\
3634	31.489788\\
3636	31.249718\\
3638	31.166668\\
3641	31.434973\\
3643	32.660618\\
3646	32.660618\\
3648	31.829365\\
3650	31.166668\\
3652	31.434973\\
3654	32.660618\\
3656	34.354\\
3658	33.775671\\
3660	33.388493\\
3662	33.425411\\
3664	33.765813\\
3666	33.780767\\
3668	33.739935\\
3670	34.526429\\
3672	32.660618\\
3674	31.8066\\
3676	31.680679\\
3678	32.660618\\
3680	34.620014\\
3682	34.468783\\
3684	33.425411\\
3686	33.148873\\
3688	34.020854\\
3690	34.620014\\
3692	34.301127\\
3694	34.354\\
3696	32.660618\\
3699	32.60558\\
3701	32.660618\\
3703	34.620014\\
3705	35.866335\\
3707	35.993515\\
3709	36.062515\\
3711	35.012206\\
3713	35.042182\\
3715	34.934454\\
3717	33.94082\\
3719	32.660618\\
3721	31.16667\\
3723	31.166668\\
3726	31.944561\\
3728	33.425411\\
3730	32.660618\\
3733	32.660618\\
3735	31.943722\\
3737	32.660618\\
3739	33.425411\\
3741	32.749123\\
3743	33.425411\\
3745	32.289901\\
3747	31.680679\\
3749	32.586812\\
3751	34.620014\\
3753	34.354\\
3755	33.484916\\
3757	32.960949\\
3759	32.660618\\
3761	33.425411\\
3763	33.581414\\
3765	32.660618\\
3768	32.255287\\
3770	31.166668\\
3773	30.769477\\
3775	31.166668\\
3777	31.736099\\
3779	32.135548\\
3781	31.504213\\
3783	31.434973\\
3785	31.995327\\
3787	32.660618\\
3790	32.660618\\
3792	32.135547\\
3794	31.166668\\
3797	31.166668\\
3800	31.489788\\
3802	31.680679\\
3804	31.489788\\
3806	31.166666\\
3808	31.166668\\
3810	31.504283\\
3812	31.702461\\
3814	31.831777\\
3816	31.489788\\
3818	31.161079\\
3820	30.528904\\
3822	30.231928\\
3824	31.166668\\
3827	31.434973\\
3829	31.166668\\
3832	31.16667\\
3834	32.130252\\
3836	32.660618\\
3839	32.660618\\
3841	31.854102\\
3843	31.489788\\
3845	31.680679\\
3847	34.354\\
3849	35.156607\\
3851	35.042186\\
3853	34.620014\\
3856	34.620014\\
3858	35.154194\\
3860	34.620014\\
3863	34.620013\\
3865	32.847695\\
3867	32.660618\\
3869	32.706253\\
3870	34.620009\\
3872	35.866335\\
3874	35.042185\\
3876	34.620014\\
3879	34.620014\\
3881	35.156622\\
3883	35.325293\\
3885	34.620014\\
3888	33.77567\\
3890	32.660618\\
3893	32.660621\\
3895	35.251918\\
3897	34.885883\\
3899	34.620014\\
3901	34.490172\\
3903	34.354\\
3905	34.620014\\
3907	35.38657\\
3909	34.705595\\
3911	34.620014\\
3913	33.775671\\
3915	33.010184\\
3917	32.660628\\
3919	35.784633\\
3921	35.574252\\
3923	35.036854\\
3925	34.620014\\
3927	34.354\\
3929	34.620014\\
3932	34.620014\\
3934	34.64459\\
3936	33.775671\\
3938	33.310306\\
3940	32.660618\\
3943	32.660618\\
3945	33.635773\\
3947	33.339801\\
3949	32.660618\\
3951	31.680679\\
3953	32.660618\\
3955	33.425408\\
3957	32.903376\\
3959	33.425411\\
3961	32.660618\\
3963	31.680679\\
3965	31.576259\\
3967	31.943961\\
3969	32.380922\\
3971	32.408227\\
3973	31.504213\\
3975	31.166668\\
3977	31.640252\\
3979	32.660618\\
3982	33.249787\\
3984	32.660618\\
3986	31.434973\\
3988	31.567539\\
3990	32.660618\\
3991	34.417644\\
3993	34.620014\\
3996	34.613507\\
3998	33.81052\\
4000	33.808306\\
4002	34.383053\\
4004	33.775671\\
4006	33.954892\\
4008	32.660618\\
4010	32.291956\\
4012	32.579216\\
4014	33.425411\\
4016	34.907237\\
4018	34.620014\\
4020	33.775671\\
4022	32.964345\\
4024	33.168862\\
4026	32.786837\\
4028	33.425411\\
4030	33.818827\\
4032	32.660618\\
4034	32.437878\\
4036	32.660618\\
4038	32.706397\\
4040	34.354\\
4042	34.192553\\
4044	34.198233\\
4046	33.010184\\
4048	33.241781\\
4050	33.775671\\
4052	33.581519\\
4054	33.775671\\
4056	32.660618\\
4059	32.660618\\
4062	33.179415\\
4064	34.620014\\
4067	34.353204\\
4069	33.425411\\
4071	33.425414\\
4073	34.582998\\
4075	34.354\\
4077	33.365252\\
4079	33.109886\\
4081	32.660618\\
4084	32.660618\\
4087	33.425411\\
4089	34.174132\\
4091	34.174752\\
4093	33.304616\\
4095	33.581413\\
4097	34.1683\\
4099	34.354\\
4101	33.775671\\
4103	34.187487\\
4105	32.660618\\
4108	32.660618\\
4111	32.660618\\
4113	33.28272\\
4115	32.660641\\
4117	32.660618\\
4120	32.660618\\
4123	32.814808\\
4125	32.794899\\
4127	33.298664\\
4129	32.660618\\
4131	32.289895\\
4133	31.865753\\
4135	32.537891\\
4137	32.660618\\
4140	32.660617\\
4142	31.680679\\
4144	32.225355\\
4146	32.660618\\
4149	32.660619\\
4151	32.983962\\
4153	32.660618\\
4155	32.585212\\
4157	32.660618\\
4159	34.620014\\
4162	34.598554\\
4164	34.212401\\
4166	33.775671\\
4168	34.354\\
4170	34.620014\\
4173	33.775671\\
4175	34.620014\\
4177	32.660618\\
4180	32.660618\\
4182	33.348591\\
4184	34.620014\\
4187	35.071936\\
4189	34.980753\\
4191	34.620014\\
4194	34.114339\\
4196	34.620014\\
4199	34.531769\\
4201	32.660619\\
4203	32.660618\\
4206	32.660619\\
4208	34.192773\\
4210	33.775671\\
4212	33.425411\\
4214	33.769517\\
4216	33.775671\\
4218	34.354\\
4220	34.295461\\
4222	34.620014\\
4224	32.86962\\
4226	32.660618\\
4229	32.660618\\
4231	34.114141\\
4233	34.620014\\
4235	34.49067\\
4237	34.133239\\
4239	33.904574\\
4241	34.620014\\
4243	33.949238\\
4245	33.775671\\
4247	34.354001\\
4249	32.660618\\
4252	32.660618\\
4254	33.010185\\
4256	34.354\\
4258	35.047095\\
4260	34.620014\\
4263	33.425411\\
4265	33.775671\\
4267	33.425411\\
4269	33.124686\\
4271	32.918624\\
4273	32.660618\\
4275	31.680679\\
4277	31.504213\\
4279	32.492384\\
4281	32.660618\\
4284	32.660618\\
4287	32.660618\\
4290	32.853434\\
4292	32.660618\\
4295	32.660618\\
4297	32.63329\\
4299	31.680679\\
4301	31.489788\\
4303	31.628817\\
4305	32.135547\\
4307	32.660618\\
4309	32.323144\\
4311	31.680679\\
4313	31.866398\\
4315	32.660618\\
4318	32.660618\\
4320	32.289905\\
4322	31.489788\\
4325	31.680679\\
4327	33.010183\\
4329	33.64646\\
4331	33.775671\\
4333	33.425411\\
4335	32.81174\\
4337	33.56182\\
4339	34.354\\
4341	33.781576\\
4343	33.775671\\
4344	31.796621\\
4346	30.924764\\
4349	30.924764\\
4351	33.179756\\
4353	33.124009\\
4355	32.780022\\
4357	31.980884\\
4359	31.812912\\
4361	32.780022\\
4363	32.579187\\
4365	30.924764\\
4368	30.924764\\
4371	30.924764\\
4374	30.924764\\
4376	31.980553\\
4378	31.796619\\
4380	30.924764\\
4383	30.924764\\
4385	31.255751\\
4387	31.980553\\
4389	31.980775\\
4391	31.401504\\
4393	30.924764\\
4396	30.924764\\
4399	30.924767\\
4401	31.980553\\
4403	30.924769\\
4405	30.924764\\
4408	30.924764\\
4410	31.094991\\
4412	30.924765\\
4414	30.924872\\
4416	30.924764\\
4419	30.924763\\
4421	30.924764\\
4423	31.475705\\
4425	31.994332\\
4427	32.375313\\
4429	31.980553\\
4431	31.817016\\
4433	31.796621\\
4436	31.255752\\
4438	30.924801\\
4440	30.924764\\
4442	29.99691\\
4444	29.816176\\
4446	29.510215\\
4448	29.816177\\
4450	30.411035\\
4452	30.573737\\
4454	29.996907\\
4457	29.996907\\
4459	29.816162\\
4461	30.57375\\
4463	30.924761\\
4465	29.816161\\
4467	29.510215\\
4469	28.625152\\
4471	29.510215\\
4473	29.76426\\
4475	29.829813\\
4477	29.996907\\
4479	29.816161\\
4481	30.444036\\
4483	30.924764\\
4486	30.924764\\
4488	29.76426\\
4490	29.510215\\
4493	29.82982\\
4495	30.924764\\
4497	30.924765\\
4500	30.924764\\
4503	30.967815\\
4505	31.273102\\
4507	31.940872\\
4509	31.980554\\
4511	31.702406\\
4513	30.924764\\
4516	30.924764\\
4519	31.828142\\
4521	33.425411\\
4523	33.960103\\
4525	33.425411\\
4527	33.179756\\
4530	32.911563\\
4532	32.780022\\
4535	31.980553\\
4537	30.924764\\
4539	30.597798\\
4541	30.573744\\
4543	30.924764\\
4545	31.980553\\
4547	32.780022\\
4550	32.780022\\
4552	32.54328\\
4554	31.980554\\
4556	30.924764\\
4559	30.924764\\
4561	29.996907\\
4563	29.809579\\
4565	30.037157\\
4567	30.924764\\
4569	31.980554\\
4571	31.799182\\
4573	31.980554\\
4575	31.547883\\
4577	31.796621\\
4579	31.554564\\
4581	30.995057\\
4583	30.924764\\
4586	30.924763\\
4588	30.778643\\
4590	30.924764\\
4592	30.924774\\
4594	30.924764\\
4595	32.771754\\
4597	31.980553\\
4599	30.924766\\
4601	31.368226\\
4603	30.924766\\
4605	30.924783\\
4607	31.796621\\
4609	30.924764\\
4612	30.924764\\
4615	30.924764\\
4618	30.924767\\
4620	30.924764\\
4623	30.924764\\
4626	30.924764\\
4629	30.924764\\
4632	30.924764\\
4634	30.4276\\
4636	29.82982\\
4638	29.76426\\
4640	30.087271\\
4642	30.924764\\
4645	30.924764\\
4647	29.996909\\
4649	30.924763\\
4651	30.924764\\
4654	30.924764\\
4657	29.82982\\
4659	29.57906\\
4661	29.996907\\
4663	30.924764\\
4665	30.924784\\
4667	32.106101\\
4669	31.7627\\
4671	30.924764\\
4673	31.796621\\
4675	31.665846\\
4677	30.924896\\
4679	31.083463\\
4681	30.716459\\
4683	29.963584\\
4685	29.996907\\
4687	30.924764\\
4689	30.924771\\
4691	32.662189\\
4693	32.532586\\
4695	32.419498\\
4697	32.780022\\
4699	31.980554\\
4701	31.980555\\
4703	32.780022\\
4705	30.936454\\
4707	30.924764\\
4710	31.980553\\
4712	33.615111\\
4714	33.425411\\
4716	32.780022\\
4718	32.029963\\
4720	32.466856\\
4722	32.849064\\
4724	31.980554\\
4726	32.780022\\
4728	30.924966\\
4730	30.924764\\
4733	30.924764\\
4735	31.980553\\
4737	32.780022\\
4739	32.034789\\
4741	31.591451\\
4743	31.796621\\
4745	32.657552\\
4747	32.65755\\
4749	31.980549\\
4751	31.255748\\
4753	30.573744\\
4755	29.996907\\
4757	30.427595\\
4759	30.924764\\
4761	31.519622\\
4763	31.980553\\
4765	30.924764\\
4768	30.924764\\
4770	31.255751\\
4772	31.743211\\
4774	31.980553\\
4776	30.924764\\
4778	29.82982\\
4780	29.510215\\
4783	29.510215\\
4786	29.76426\\
4788	29.510215\\
4791	29.510215\\
4793	29.996907\\
4795	30.924764\\
4798	30.924764\\
4800	30.924762\\
4802	29.816162\\
4804	29.556602\\
4806	29.510215\\
4808	29.764261\\
4810	30.226873\\
4812	29.816161\\
4814	29.699309\\
4816	29.510215\\
4818	29.996907\\
4820	30.924763\\
4822	30.924764\\
4824	29.996907\\
4826	29.510215\\
4829	29.510215\\
4832	29.510215\\
4834	29.816162\\
4836	29.623972\\
4838	29.696338\\
4840	29.510215\\
4842	29.82982\\
4844	29.996905\\
4846	30.012714\\
4848	29.510215\\
4850	28.921293\\
4852	28.587076\\
4854	29.580123\\
4856	30.924764\\
4859	30.924764\\
4862	30.583591\\
4864	30.222745\\
4866	30.924764\\
4869	30.924764\\
4872	30.353745\\
4874	29.510215\\
4877	29.816161\\
4879	30.924764\\
4882	30.924764\\
4884	30.671176\\
4886	30.427588\\
4888	30.924763\\
4890	30.924764\\
4893	30.924764\\
4896	30.4276\\
4898	29.653463\\
4900	29.76426\\
4902	30.446225\\
4904	30.924764\\
4907	30.924764\\
4909	30.753809\\
4911	30.52854\\
4913	30.924764\\
4916	30.924764\\
4919	30.924764\\
4921	29.996908\\
4923	29.82982\\
4925	29.996907\\
4927	30.924764\\
4930	30.924764\\
4933	30.924764\\
4936	30.924764\\
4939	30.924764\\
4942	30.924764\\
4945	30.310624\\
4947	29.816162\\
4949	29.76426\\
4951	29.996907\\
4953	30.924764\\
4956	30.924764\\
4958	29.996907\\
4960	29.996917\\
4962	30.924764\\
4965	30.924764\\
4968	30.260785\\
4970	29.76426\\
4972	29.510215\\
4975	29.510215\\
4977	29.816161\\
4979	29.510215\\
4982	29.448857\\
4984	29.510215\\
4986	29.996907\\
4988	30.924764\\
4991	30.924764\\
4993	29.996907\\
4995	29.76426\\
4997	29.996907\\
4999	30.924764\\
5002	30.924764\\
5005	30.924764\\
5008	30.924764\\
5011	30.924764\\
5014	30.924764\\
5016	30.263259\\
5018	29.510215\\
5021	29.779385\\
5023	30.924764\\
5026	30.924764\\
5029	30.924764\\
5032	30.924764\\
5035	30.924764\\
5038	30.924764\\
5041	30.192421\\
5043	29.816164\\
5045	29.996907\\
5047	30.924764\\
5050	30.924764\\
5053	30.924764\\
5056	30.924764\\
5059	30.924764\\
5062	30.924764\\
5065	29.996907\\
5067	29.76426\\
5069	29.996907\\
5071	30.924764\\
5074	30.924764\\
5077	30.924764\\
5080	30.924764\\
5083	30.924764\\
5086	30.924764\\
5088	31.863699\\
5090	31.346718\\
5093	31.513456\\
5095	32.849298\\
5098	32.849298\\
5100	32.511416\\
5102	32.185508\\
5104	32.476433\\
5106	32.849298\\
5109	32.849298\\
5112	32.437175\\
5114	31.346718\\
5117	31.346718\\
5120	31.616572\\
5122	31.863698\\
5124	31.827564\\
5126	31.616589\\
5128	31.863699\\
5130	32.849298\\
5133	32.849298\\
5136	31.688666\\
5138	31.346718\\
5141	31.346718\\
5144	31.346718\\
5147	31.346718\\
5149	29.892499\\
5151	29.826728\\
5153	31.324599\\
5155	31.589096\\
5157	31.627364\\
5159	31.346747\\
5161	31.346718\\
5164	31.346718\\
5166	32.849297\\
5168	32.849298\\
5170	32.849299\\
5172	32.849298\\
5175	32.849298\\
5178	32.849298\\
5181	32.849298\\
5184	32.476434\\
5186	31.616574\\
5188	31.346764\\
5190	32.616227\\
5192	32.849298\\
5195	32.849298\\
5198	32.849298\\
5201	33.425411\\
5203	33.970793\\
5205	34.137799\\
5207	33.425412\\
5209	32.849298\\
5212	32.67304\\
5214	32.849298\\
5216	33.066208\\
5218	33.087277\\
5220	32.849298\\
5223	32.849298\\
5225	33.425411\\
5227	33.188226\\
5229	32.849302\\
5231	32.849298\\
5234	32.849298\\
5237	32.849298\\
5239	33.425411\\
5241	34.820014\\
5243	34.354\\
5245	34.13783\\
5247	33.916714\\
5249	33.781114\\
5251	33.622557\\
5253	34.333777\\
5255	33.425411\\
5257	32.849298\\
5260	32.849298\\
5263	33.913715\\
5265	34.820014\\
5267	34.354\\
5270	33.970793\\
5272	34.354\\
5274	33.970793\\
5276	33.425412\\
5278	32.932266\\
5280	32.849298\\
5282	31.863699\\
5284	31.671705\\
5286	31.616573\\
5288	32.476434\\
5290	32.849298\\
5292	32.054101\\
5294	31.686211\\
5296	31.863699\\
5298	32.849298\\
5301	32.849298\\
5304	32.849298\\
5306	32.15661\\
5308	31.752278\\
5310	31.671705\\
5312	31.830524\\
5314	32.512735\\
5316	32.849298\\
5318	31.616573\\
5320	31.346718\\
5322	31.863699\\
5325	32.346045\\
5327	31.863701\\
5329	31.346718\\
5331	30.956819\\
5333	31.346718\\
5335	32.849298\\
5338	32.849298\\
5341	32.849298\\
5344	32.849298\\
5347	32.849298\\
5350	32.849298\\
5352	32.689828\\
5354	31.686204\\
5356	31.686213\\
5358	32.849298\\
5361	32.849299\\
5364	32.849298\\
5367	32.849298\\
5370	32.849299\\
5373	32.849299\\
5376	32.849298\\
5378	32.476434\\
5380	32.476452\\
5382	32.849298\\
5384	33.970438\\
5386	33.970793\\
5388	33.752566\\
5390	33.200885\\
5392	32.989186\\
5394	32.849299\\
5396	33.425411\\
5398	33.775414\\
5400	32.849298\\
5403	32.849298\\
5406	32.849298\\
5408	34.820014\\
5410	34.936586\\
5412	34.820014\\
5414	34.380072\\
5416	34.438252\\
5418	34.820014\\
5420	34.354\\
5422	34.353936\\
5424	32.849299\\
5426	32.849298\\
5429	32.849298\\
5432	32.849298\\
5435	32.849298\\
5437	32.849297\\
5439	31.863699\\
5441	32.321196\\
5443	32.849298\\
5446	32.849298\\
5449	32.479103\\
5451	31.742117\\
5453	31.863701\\
5455	32.849298\\
5458	32.849298\\
5461	32.849298\\
5463	32.476434\\
5465	32.849298\\
5468	32.849298\\
5471	32.849298\\
5473	31.616573\\
5475	31.346718\\
5478	31.346718\\
5481	31.616573\\
5483	31.671705\\
5485	31.346718\\
5488	31.346718\\
5490	31.901596\\
5492	32.849298\\
5495	32.848545\\
5497	31.346718\\
5500	31.346718\\
5502	32.849298\\
5504	32.849303\\
5506	32.849298\\
5509	32.849298\\
5511	32.84931\\
5513	33.970793\\
5515	34.820014\\
5518	34.820014\\
5520	33.073268\\
5522	32.849298\\
5525	32.849298\\
5527	33.970793\\
5529	34.820014\\
5531	34.689923\\
5533	33.425411\\
5536	33.774182\\
5538	34.820014\\
5540	34.820015\\
5542	34.820014\\
5544	33.425411\\
5546	32.849298\\
5549	32.849299\\
5551	34.930668\\
5553	35.167958\\
5555	34.820014\\
5557	34.484206\\
5559	34.651336\\
5561	35.207664\\
5563	35.114677\\
5565	36.062248\\
5567	34.820014\\
5569	32.849299\\
5571	32.849298\\
5574	34.354\\
5576	35.309359\\
5578	34.689923\\
5580	34.820014\\
5583	34.820014\\
5585	34.693415\\
5587	34.820014\\
5589	34.3579\\
5591	33.970788\\
5593	32.849298\\
5596	32.849298\\
5598	32.8493\\
5600	34.820014\\
5602	34.354\\
5604	34.090444\\
5606	34.354\\
5609	34.354\\
5611	34.820014\\
5613	34.820015\\
5615	34.653298\\
5617	32.849298\\
5619	33.200884\\
5621	32.849299\\
5623	33.425411\\
5625	33.425409\\
5627	32.849501\\
5629	32.849298\\
5632	32.849298\\
5634	33.970793\\
5636	34.820014\\
5638	34.689924\\
5640	33.425411\\
5642	32.239795\\
5644	31.863698\\
5646	32.061405\\
5648	32.849298\\
5651	32.849298\\
5653	31.863699\\
5656	31.871492\\
5658	32.849298\\
5660	33.425411\\
5662	33.896698\\
5664	32.849298\\
5667	32.849298\\
5670	33.719345\\
5672	34.95978\\
5674	36.067732\\
5676	36.270848\\
5679	36.63137\\
5681	36.479592\\
5683	36.067732\\
5685	34.820014\\
5687	33.923271\\
5689	32.849299\\
5691	32.849298\\
5694	34.389737\\
5696	36.179113\\
5698	38.228111\\
5700	37.757235\\
5702	36.660286\\
5704	37.426293\\
5706	36.427007\\
5708	36.073535\\
5710	35.144251\\
5712	33.049293\\
5714	32.849298\\
5717	32.849299\\
5719	34.821586\\
5721	36.073535\\
5723	35.072552\\
5725	34.820015\\
5728	34.959778\\
5730	36.073535\\
5732	35.707081\\
5734	35.990464\\
5736	33.054399\\
5738	32.849298\\
5741	32.849298\\
5743	34.820015\\
5745	36.073535\\
5747	36.128784\\
5749	35.761415\\
5751	36.067732\\
5753	35.132071\\
5755	35.244625\\
5757	34.820015\\
5759	33.970793\\
5761	32.849298\\
5764	32.849298\\
5766	33.775414\\
5768	34.959778\\
5770	34.820014\\
5772	34.022937\\
5774	33.425411\\
5776	32.849299\\
5778	33.951879\\
5780	33.970793\\
5782	33.425412\\
5784	32.849298\\
5787	32.849298\\
5790	32.849298\\
5792	33.11001\\
5794	34.820014\\
5797	34.820014\\
5799	34.625004\\
5801	34.820014\\
5803	35.244626\\
5805	34.820014\\
5807	34.354\\
5809	32.849299\\
5811	32.849298\\
5814	32.849299\\
5816	33.135325\\
5818	34.354\\
5821	33.660415\\
5823	32.849299\\
5825	33.206174\\
5827	34.914849\\
5829	34.865396\\
5831	34.820014\\
5832	39.132332\\
5834	38.641177\\
5836	38.873933\\
5838	40.22562\\
5840	41.479983\\
5842	41.350397\\
5844	40.632433\\
5846	40.967815\\
5848	41.479983\\
5851	41.985807\\
5853	41.479983\\
5855	40.402285\\
5857	39.132332\\
5860	39.132332\\
5862	40.468333\\
5864	41.985809\\
5866	41.479983\\
5868	40.950578\\
5870	40.867168\\
5872	41.320132\\
5874	41.479983\\
5877	41.479983\\
5879	40.468333\\
5881	39.132332\\
5884	39.132332\\
5886	40.468333\\
5888	41.989586\\
5890	41.479983\\
5892	40.244114\\
5894	39.173891\\
5896	40.402855\\
5898	41.479983\\
5901	41.479983\\
5903	39.132337\\
5905	39.132332\\
5908	39.132332\\
5910	40.468333\\
5912	42.09425\\
5914	41.479983\\
5916	40.977749\\
5918	40.468333\\
5920	41.479983\\
5922	42.782616\\
5924	42.973263\\
5926	41.886459\\
5928	39.133882\\
5930	39.132332\\
5933	39.132332\\
5935	42.555624\\
5937	43.208316\\
5939	42.857422\\
5941	41.479983\\
5944	41.479983\\
5946	41.646487\\
5948	41.57238\\
5950	41.325011\\
5952	41.479983\\
5953	39.736628\\
5955	39.132332\\
5958	39.132332\\
5960	40.662066\\
5962	41.985809\\
5964	42.122346\\
5966	41.479983\\
5969	42.008628\\
5971	41.94496\\
5973	41.985809\\
5975	41.479983\\
5977	40.468332\\
5979	40.141361\\
5981	40.078195\\
5983	40.468333\\
5985	41.479983\\
5988	41.414917\\
5990	39.845352\\
5992	39.132332\\
5994	40.183908\\
5996	41.479983\\
5998	41.32501\\
6000	39.132332\\
6003	39.132331\\
6005	39.132332\\
6006	41.479983\\
6008	43.39715\\
6010	43.208316\\
6012	42.450441\\
6014	41.479983\\
6016	41.912066\\
6018	42.97328\\
6020	42.596442\\
6022	41.479983\\
6024	39.551164\\
6026	39.132332\\
6029	39.132332\\
6030	41.479983\\
6032	42.896879\\
6034	41.985813\\
6036	41.479983\\
6039	41.479983\\
6041	42.966349\\
6043	42.171461\\
6045	41.939159\\
6047	40.730069\\
6049	39.132332\\
6052	39.132332\\
6054	41.479983\\
6056	43.100955\\
6058	41.811496\\
6060	41.479983\\
6062	41.158657\\
6064	41.479983\\
6066	42.673472\\
6068	42.011555\\
6070	41.479983\\
6072	39.132332\\
6075	39.132332\\
6078	41.479983\\
6080	42.890803\\
6082	42.954311\\
6084	41.985531\\
6086	41.479983\\
6089	42.073557\\
6091	41.88021\\
6093	41.912066\\
6095	40.468333\\
6097	39.132332\\
6100	39.132332\\
6102	41.479983\\
6104	42.820654\\
6106	41.479983\\
6108	40.235592\\
6110	39.64445\\
6112	40.235581\\
6114	41.479983\\
6116	41.985809\\
6118	41.479983\\
6120	39.132332\\
6122	39.132331\\
6124	38.313228\\
6126	39.132332\\
6128	39.433246\\
6130	39.505699\\
6132	39.132332\\
6134	38.641892\\
6136	38.946947\\
6138	39.132332\\
6141	39.132332\\
6144	39.132331\\
6146	37.958217\\
6148	37.737697\\
6150	38.503217\\
6152	39.132332\\
6154	39.454437\\
6156	39.132336\\
6158	39.132332\\
6161	39.132332\\
6164	39.730132\\
6166	39.132332\\
6169	39.132325\\
6171	38.671439\\
6173	39.132332\\
6175	41.985809\\
6177	42.966347\\
6179	42.33554\\
6181	41.479983\\
6184	41.607509\\
6186	42.973263\\
6189	42.122923\\
6191	40.512682\\
6193	39.132332\\
6196	39.132332\\
6198	41.479983\\
6200	43.208316\\
6202	42.336735\\
6204	41.479983\\
6207	41.479983\\
6209	42.966349\\
6211	42.973263\\
6213	42.947148\\
6215	41.325033\\
6216	39.132332\\
6219	39.132332\\
6222	41.479983\\
6224	42.766791\\
6226	41.891955\\
6228	41.479983\\
6230	40.905695\\
6232	41.479983\\
6234	42.455341\\
6236	42.969322\\
6238	41.479983\\
6240	39.132332\\
6243	39.132332\\
6246	41.325009\\
6248	43.208316\\
6250	44.157027\\
6252	43.414218\\
6254	43.208316\\
6256	42.973263\\
6258	43.323538\\
6260	43.208316\\
6262	41.479983\\
6264	39.317697\\
6266	39.132332\\
6269	39.132332\\
6270	41.479983\\
6272	43.240797\\
6274	43.32394\\
6276	42.973263\\
6278	42.953028\\
6280	42.891586\\
6282	42.973262\\
6284	42.966349\\
6286	41.479983\\
6288	39.132407\\
6290	39.132332\\
6293	39.132332\\
6296	39.132342\\
6298	40.235584\\
6300	39.491616\\
6302	39.132332\\
6305	39.551165\\
6307	40.277997\\
6309	39.728535\\
6311	39.132332\\
6314	38.68815\\
6316	37.663826\\
6318	37.717765\\
6320	39.124236\\
6322	39.132332\\
6325	37.958218\\
6327	37.342355\\
6329	38.503211\\
6331	39.132332\\
6334	39.132332\\
6336	38.92228\\
6338	37.889636\\
6340	37.746785\\
6342	40.138704\\
6344	42.014921\\
6346	42.122883\\
6348	41.479983\\
6351	41.479983\\
6354	42.18645\\
6356	42.629696\\
6358	41.479983\\
6360	39.132332\\
6363	39.132332\\
6366	41.479983\\
6368	43.208316\\
6370	42.862667\\
6372	41.479983\\
6375	41.479983\\
6377	42.945306\\
6379	43.208316\\
6381	42.71794\\
6383	41.042039\\
6384	39.132333\\
6386	39.132332\\
6389	39.132332\\
6391	41.985808\\
6393	42.433713\\
6395	42.96634\\
6397	42.216185\\
6399	41.912066\\
6401	42.599694\\
6403	41.911288\\
6405	41.479983\\
6407	39.580049\\
6409	39.132332\\
6412	39.132332\\
6414	41.331429\\
6416	42.973263\\
6418	42.457489\\
6420	41.479983\\
6422	40.617534\\
6424	40.96809\\
6426	41.646416\\
6428	42.260684\\
6430	40.468333\\
6432	39.132332\\
6435	39.132332\\
6438	41.066699\\
6439	42.966349\\
6441	42.973263\\
6443	42.891104\\
6445	42.635784\\
6447	42.973263\\
6449	43.844324\\
6451	44.14446\\
6453	42.649393\\
6455	41.479983\\
6457	40.468333\\
6459	39.132342\\
6461	39.132332\\
6463	41.173142\\
6465	41.985906\\
6467	42.125843\\
6469	41.384027\\
6471	40.468333\\
6473	41.479983\\
6475	42.122926\\
6477	41.985809\\
6479	41.051814\\
6481	39.132332\\
6483	37.958218\\
6485	37.958219\\
6487	39.132332\\
6490	39.132332\\
6493	38.150075\\
6495	37.958213\\
6497	39.132332\\
6499	42.264046\\
6501	41.479983\\
6503	39.132337\\
6505	38.943577\\
6507	38.125087\\
6509	39.132332\\
6510	41.241253\\
6512	42.973263\\
6514	42.200413\\
6516	41.479983\\
6518	41.646479\\
6520	43.487125\\
6521	45.856119\\
6523	45.935436\\
6525	43.234798\\
6527	41.479983\\
6529	39.132332\\
6532	39.132332\\
6534	42.954741\\
6535	45.583836\\
6537	46.641364\\
6539	44.476987\\
6541	43.608372\\
6543	42.973263\\
6545	44.600644\\
6547	46.164084\\
6549	44.318996\\
6551	41.609369\\
6553	40.30219\\
6556	40.30219\\
6558	42.720025\\
6559	44.500026\\
6561	43.944458\\
6563	43.240971\\
6565	42.720025\\
6568	42.720061\\
6570	44.500026\\
6572	44.257945\\
6574	44.023666\\
6576	41.432312\\
6578	40.30219\\
6581	40.643187\\
6582	43.164921\\
6583	45.662851\\
6585	45.989629\\
6587	45.538444\\
6589	44.500026\\
6591	45.149234\\
6593	45.996353\\
6595	45.609757\\
6597	44.511288\\
6599	42.720025\\
6601	40.302192\\
6603	40.30219\\
6605	40.733543\\
6606	43.011523\\
6607	46.189407\\
6609	44.50114\\
6611	44.257945\\
6613	42.720025\\
6616	43.153608\\
6618	44.257945\\
6620	44.500026\\
6622	43.201765\\
6624	40.344112\\
6626	40.20587\\
6628	39.028651\\
6630	39.093\\
6632	40.30219\\
6634	40.302189\\
6636	39.449922\\
6638	39.092976\\
6640	40.30219\\
6642	40.653105\\
6644	40.733454\\
6646	40.30219\\
6648	39.445476\\
6650	38.78977\\
6652	38.769363\\
6654	39.092976\\
6656	40.30219\\
6658	40.302626\\
6660	40.928742\\
6662	40.30219\\
6665	41.483817\\
6667	42.720025\\
6670	42.192554\\
6672	40.30219\\
6674	39.65427\\
6676	39.847763\\
6678	42.720024\\
6679	44.742037\\
6681	44.66374\\
6683	44.247335\\
6685	44.109459\\
6687	42.891498\\
6689	44.052272\\
6691	42.720025\\
6693	41.67813\\
6695	40.30219\\
6697	39.165481\\
6699	38.857422\\
6701	40.206849\\
6703	42.720025\\
6705	42.192451\\
6707	41.678131\\
6709	40.302192\\
6711	40.30219\\
6713	41.434605\\
6715	42.720024\\
6717	41.975394\\
6719	40.30219\\
6722	39.923578\\
6724	40.30219\\
6726	41.101652\\
6728	42.720025\\
6730	43.448376\\
6732	44.25081\\
6734	42.720025\\
6736	42.345564\\
6738	42.720025\\
6740	42.720024\\
6742	41.438421\\
6744	40.30219\\
6746	39.892886\\
6748	39.844728\\
6750	40.733543\\
6751	42.720025\\
6754	42.720025\\
6756	40.600301\\
6758	41.595339\\
6760	42.093127\\
6762	42.720024\\
6765	42.014193\\
6767	40.30219\\
6770	39.324632\\
6772	39.549302\\
6774	41.344854\\
6775	43.444386\\
6777	42.720025\\
6779	43.097056\\
6781	42.234433\\
6783	41.886856\\
6785	42.720025\\
6787	44.16095\\
6789	43.116418\\
6791	42.720025\\
6792	40.302199\\
6794	40.30219\\
6797	40.30219\\
6799	40.302191\\
6801	41.678131\\
6804	41.411063\\
6806	40.30219\\
6808	40.302203\\
6810	41.467886\\
6812	41.678131\\
6814	40.30219\\
6817	40.30219\\
6819	40.077351\\
6821	40.30219\\
6824	40.30219\\
6827	40.30219\\
6830	40.30219\\
6833	40.30219\\
6835	40.302191\\
6837	40.30219\\
6840	40.126305\\
6842	39.092974\\
6844	39.621347\\
6846	40.958631\\
6847	42.720024\\
6849	41.678131\\
6851	42.117905\\
6853	41.678131\\
6856	41.883579\\
6858	42.720025\\
6861	42.192539\\
6863	40.302195\\
6865	40.302189\\
6867	39.65427\\
6869	40.30219\\
6871	43.240971\\
6873	42.720024\\
6875	41.79526\\
6877	41.531214\\
6879	42.055135\\
6881	43.240971\\
6883	44.407793\\
6885	42.989081\\
6887	41.784054\\
6889	40.30219\\
6892	40.30219\\
6894	43.936669\\
6895	53.400031\\
6896	44.500026\\
6898	44.257946\\
6900	43.645695\\
6902	43.382182\\
6904	44.250825\\
6906	45.813428\\
6907	52.740837\\
6908	45.05475\\
6910	43.240971\\
6912	40.30219\\
6915	40.30219\\
6918	41.678131\\
6919	44.250825\\
6921	44.133768\\
6923	43.429679\\
6925	42.720025\\
6928	42.720025\\
6930	44.622894\\
6932	44.257945\\
6934	43.024642\\
6936	40.302191\\
6938	40.30219\\
6941	40.30219\\
6942	42.661877\\
6944	44.500026\\
6946	44.250826\\
6948	42.720025\\
6950	42.63211\\
6952	42.720025\\
6954	44.257945\\
6956	43.699599\\
6958	42.560094\\
6960	40.30219\\
6962	39.092976\\
6964	38.857422\\
6966	39.136442\\
6968	40.30219\\
6971	40.30219\\
6973	40.074751\\
6975	39.84473\\
6977	40.30219\\
6979	41.678131\\
6981	40.30219\\
6984	39.844729\\
6986	38.458702\\
6989	38.458702\\
6991	39.092976\\
6993	39.84473\\
6995	39.844742\\
6997	38.857422\\
7000	38.857422\\
7002	40.30219\\
7005	40.30219\\
7008	38.875223\\
7010	38.458702\\
7012	38.458941\\
7014	40.30219\\
7015	42.720025\\
7017	42.720029\\
7019	42.720025\\
7021	42.522331\\
7023	42.356024\\
7025	42.720025\\
7027	42.418201\\
7029	40.30219\\
7032	40.30219\\
7034	38.857422\\
7037	39.290239\\
7039	41.678131\\
7041	42.192217\\
7043	42.720025\\
7045	41.678131\\
7047	40.606609\\
7049	41.678131\\
7051	40.882321\\
7053	40.30219\\
7055	39.84473\\
7057	38.458702\\
7060	38.458702\\
7062	40.30219\\
7064	41.67813\\
7066	40.302193\\
7068	40.30219\\
7071	40.30219\\
7073	41.438424\\
7075	42.720025\\
7077	41.678125\\
7079	40.30219\\
7081	39.092976\\
7083	38.458751\\
7085	39.277251\\
7087	42.347254\\
7089	41.678131\\
7091	41.563853\\
7093	40.835516\\
7095	41.252221\\
7097	41.678131\\
7099	41.67812\\
7101	40.30219\\
7104	39.194412\\
7106	38.458702\\
7109	38.692621\\
7111	40.30219\\
7113	41.438571\\
7115	41.646532\\
7117	41.380708\\
7119	40.457583\\
7121	40.957961\\
7123	40.57137\\
7125	40.30219\\
7128	40.30219\\
7130	39.654271\\
7132	39.093771\\
7134	40.30219\\
7136	40.302193\\
7138	40.733544\\
7140	41.438423\\
7142	40.30219\\
7145	40.30219\\
7148	40.30219\\
7151	40.30219\\
7153	38.789782\\
7155	38.458702\\
7158	38.458702\\
7160	38.789782\\
7162	39.092976\\
7164	38.857422\\
7166	38.458702\\
7168	38.875222\\
7170	40.30219\\
7173	39.725296\\
7175	39.092976\\
7177	38.458702\\
7179	37.040934\\
7181	37.941205\\
7183	40.30219\\
7186	40.30219\\
7189	40.30219\\
7192	40.30219\\
7194	40.836127\\
7195	42.717159\\
7197	40.694971\\
7199	40.30219\\
7201	39.092976\\
7203	38.458702\\
7206	38.857422\\
7208	40.30219\\
7211	40.30219\\
7214	40.30219\\
7217	40.353626\\
7219	42.613136\\
7221	40.305327\\
7223	40.30219\\
7225	39.142241\\
7227	38.458702\\
7230	39.092976\\
7232	40.416489\\
7234	41.992538\\
7236	42.720025\\
7239	42.720025\\
7241	42.891498\\
7243	43.306245\\
7245	42.185748\\
7247	40.433402\\
7249	40.30219\\
7251	39.092976\\
7254	40.302189\\
7256	40.77239\\
7258	42.277409\\
7260	41.678131\\
7262	40.959661\\
7264	40.899175\\
7266	42.560417\\
7268	42.720025\\
7269	40.302199\\
7271	40.30219\\
7273	39.654262\\
7275	38.458702\\
7278	38.874204\\
7280	40.30219\\
7282	40.302192\\
7284	40.30219\\
7287	40.30219\\
7290	40.30219\\
7293	40.30219\\
7295	39.84715\\
7297	40.394432\\
7299	38.171499\\
7301	37.674893\\
7303	38.542487\\
7305	38.948669\\
7307	38.649648\\
7309	38.734875\\
7311	38.649648\\
7313	40.491167\\
7315	41.04038\\
7317	40.61926\\
7320	40.61926\\
7322	38.053535\\
7324	36.725786\\
7326	36.725782\\
7328	37.569504\\
7330	37.451821\\
7332	37.366755\\
7334	37.769616\\
7336	38.347163\\
7338	40.133361\\
7340	40.61926\\
7342	40.213847\\
7344	39.401021\\
7346	36.963915\\
7348	36.151129\\
7350	37.883862\\
7351	40.61926\\
7353	41.289366\\
7355	41.881995\\
7357	41.477141\\
7359	42.131332\\
7361	42.566313\\
7364	42.566313\\
7366	41.848688\\
7368	41.528226\\
7370	40.61926\\
7373	40.61926\\
7375	42.566313\\
7377	42.566322\\
7379	44.575974\\
7381	44.664471\\
7383	44.951404\\
7385	44.761102\\
7387	45.695488\\
7389	45.119978\\
7390	43.16646\\
7392	42.566313\\
7394	41.289167\\
7396	41.04038\\
7398	41.93867\\
7400	45.119978\\
7403	45.119978\\
7406	45.119978\\
7409	45.119978\\
7411	45.670191\\
7413	45.119978\\
7415	43.021899\\
7417	42.402365\\
7419	40.842063\\
7421	40.619482\\
7423	42.566313\\
7425	42.611193\\
7427	43.328975\\
7429	42.56632\\
7431	42.566313\\
7434	42.566315\\
7436	42.566313\\
7439	42.566313\\
7441	40.61926\\
7443	38.662923\\
7445	38.649648\\
7447	41.187289\\
7449	42.566313\\
7452	42.566313\\
7455	42.566313\\
7458	42.566313\\
7461	42.566312\\
7463	41.881776\\
7465	41.04038\\
7467	39.784022\\
7469	38.734875\\
7471	40.61926\\
7473	41.05918\\
7475	40.61926\\
7478	40.61926\\
7480	40.746704\\
7482	42.083153\\
7484	41.703839\\
7486	41.05918\\
7488	41.04038\\
7490	39.485121\\
7492	38.649648\\
7494	38.734875\\
7496	39.600034\\
7498	40.61926\\
7500	41.040383\\
7502	41.289167\\
7504	41.700009\\
7506	42.566313\\
7509	42.566313\\
7511	42.412296\\
7513	41.258906\\
7515	40.61926\\
7517	40.464076\\
7519	41.289167\\
7521	42.539966\\
7523	41.563565\\
7525	41.04038\\
7527	41.05918\\
7529	42.083153\\
7531	42.566313\\
7534	41.489026\\
7536	42.079803\\
7538	40.61926\\
7540	39.044624\\
7542	39.360311\\
7544	39.707479\\
7546	40.61926\\
7549	40.968933\\
7551	40.61926\\
7553	40.96894\\
7555	42.566313\\
7557	42.55615\\
7559	41.884869\\
7561	40.61926\\
7563	38.903215\\
7565	39.755878\\
7567	42.566313\\
7569	42.566318\\
7571	43.021901\\
7573	43.021899\\
7575	43.758187\\
7577	43.386006\\
7578	45.119978\\
7580	44.740505\\
7582	42.566314\\
7584	42.566313\\
7586	42.083153\\
7588	41.05918\\
7590	42.566277\\
7592	44.019546\\
7594	42.609601\\
7596	42.566313\\
7599	42.566313\\
7601	42.758843\\
7602	45.119978\\
7604	44.019552\\
7606	42.566313\\
7609	41.881993\\
7611	40.61926\\
7614	41.04038\\
7616	43.010757\\
7618	43.868649\\
7620	44.019552\\
7623	44.317735\\
7625	45.119978\\
7628	44.019552\\
7630	42.566314\\
7632	42.566313\\
7634	41.289167\\
7636	40.61926\\
7639	41.04038\\
7641	42.566313\\
7644	42.566313\\
7647	42.566313\\
7650	42.566313\\
7653	42.566313\\
7656	42.566312\\
7658	40.61926\\
7660	40.223156\\
7662	40.083578\\
7664	40.61926\\
7666	41.953266\\
7668	42.566313\\
7671	42.566313\\
7674	42.566313\\
7677	42.566313\\
7680	42.566313\\
7682	40.61926\\
7685	40.61926\\
7687	42.566313\\
7689	43.867295\\
7691	44.019552\\
7693	43.30673\\
7695	44.769099\\
7697	44.982282\\
7699	45.670192\\
7701	43.927554\\
7703	42.566313\\
7706	41.344094\\
7708	40.96894\\
7710	42.083202\\
7712	45.119978\\
7714	44.539408\\
7716	44.109388\\
7718	44.834578\\
7720	45.119978\\
7722	45.460195\\
7724	45.119978\\
7726	42.992656\\
7728	42.566313\\
7730	41.924261\\
7732	41.04038\\
7734	42.135473\\
7736	45.193173\\
7738	45.119978\\
7741	45.119978\\
7744	45.119978\\
7746	45.119979\\
7748	44.549616\\
7750	42.902886\\
7752	42.566313\\
7754	42.069832\\
7756	41.289166\\
7758	42.566313\\
7760	45.119978\\
7762	44.096907\\
7764	42.982349\\
7766	43.384292\\
7768	44.01955\\
7770	45.118855\\
7772	44.019552\\
7774	42.566313\\
7777	42.566313\\
7779	41.289167\\
7782	42.287577\\
7784	44.019552\\
7786	44.989437\\
7788	44.427245\\
7790	44.680636\\
7792	44.019552\\
7794	45.119978\\
7796	44.01955\\
7798	42.566313\\
7801	42.566302\\
7803	40.61926\\
7805	40.425143\\
7807	40.61926\\
7809	42.51895\\
7811	41.289362\\
7813	41.598515\\
7815	41.28918\\
7817	42.566313\\
7820	42.566313\\
7822	42.083153\\
7824	42.363422\\
7826	40.61926\\
7828	38.845639\\
7830	39.354749\\
7832	40.61926\\
7835	40.61926\\
7837	40.96894\\
7839	40.61926\\
7841	41.289163\\
7843	41.713947\\
7845	41.881993\\
7847	41.04038\\
7849	40.61926\\
7851	38.735152\\
7853	39.327038\\
7855	42.083153\\
7857	42.566313\\
7860	42.566313\\
7863	42.566313\\
7866	43.766379\\
7868	43.105129\\
7870	42.566313\\
7873	41.673504\\
7875	40.61926\\
7878	41.05918\\
7880	43.454608\\
7882	43.0219\\
7884	42.566314\\
7886	42.566313\\
7888	42.566314\\
7890	44.019553\\
7892	43.40541\\
7894	42.566313\\
7897	41.881992\\
7899	40.61926\\
7902	41.289167\\
7904	43.716355\\
7906	45.062434\\
7908	45.119978\\
7910	45.144911\\
7912	45.326968\\
7914	45.589978\\
7916	45.119978\\
7918	42.566313\\
7921	42.566313\\
7923	41.05918\\
7925	41.315739\\
7927	42.566323\\
7928	45.119978\\
7931	45.119978\\
7934	45.119978\\
7937	45.119978\\
7940	45.119978\\
7942	42.566313\\
7945	42.256089\\
7947	40.61926\\
7950	41.04038\\
7952	42.566313\\
7955	42.566313\\
7958	42.566313\\
7961	42.566313\\
7964	42.566313\\
7967	42.566313\\
7969	41.881928\\
7971	40.61926\\
7973	40.618743\\
7975	40.61926\\
7977	42.083153\\
7979	42.566286\\
7981	42.083105\\
7983	41.934614\\
7985	42.566313\\
7988	42.566313\\
7990	42.317658\\
7992	42.566313\\
7994	41.289167\\
7996	40.619261\\
7998	40.61926\\
8000	41.04038\\
8002	42.566313\\
8004	43.021726\\
8006	42.566313\\
8008	42.793929\\
8010	44.167215\\
8012	44.921724\\
8014	42.755577\\
8016	43.075788\\
8019	41.928153\\
8021	41.783356\\
8023	43.075788\\
8024	45.525916\\
8026	46.926219\\
8028	47.562519\\
8030	47.303779\\
8032	47.562519\\
8034	47.86205\\
8036	46.654747\\
8038	45.660018\\
8040	45.095741\\
8042	43.075788\\
8045	41.783356\\
8047	43.075788\\
8048	45.660018\\
8051	45.660018\\
8054	45.191432\\
8056	45.660018\\
8058	45.660019\\
8060	45.660018\\
8062	43.075788\\
8065	43.075788\\
8067	41.783356\\
8069	41.783355\\
8071	43.075788\\
8072	45.276068\\
8074	45.660018\\
8077	45.660018\\
8080	46.216817\\
8082	46.367749\\
8084	45.660018\\
8086	44.546421\\
8088	43.830707\\
8090	43.075788\\
8093	43.075788\\
8095	44.546421\\
8097	47.253642\\
8099	47.167168\\
8101	47.366222\\
8103	47.303779\\
8105	47.296169\\
8107	47.562519\\
8109	46.758153\\
8111	45.660018\\
8113	43.075788\\
8116	43.075788\\
8119	44.546421\\
8121	45.660018\\
8123	46.216817\\
8125	45.850293\\
8127	45.660018\\
8130	45.660018\\
8132	44.902606\\
8133	43.07579\\
8135	43.536828\\
8137	43.075788\\
8139	42.574398\\
8141	41.531592\\
8143	42.383278\\
8145	43.075788\\
8147	43.075789\\
8149	43.075788\\
8152	43.075788\\
8154	45.023784\\
8156	45.189373\\
8158	43.075801\\
8160	44.391093\\
8162	43.075788\\
8164	41.519616\\
8166	41.105432\\
8168	41.17589\\
8170	42.186251\\
8172	43.075788\\
8175	43.075788\\
8178	43.075788\\
8181	43.075788\\
8184	43.075788\\
8186	41.459297\\
8188	41.034711\\
8190	41.105432\\
8191	43.075788\\
8193	44.667474\\
8195	44.290218\\
8197	45.63734\\
8199	45.660018\\
8201	45.990679\\
8203	46.458417\\
8205	44.995794\\
8206	43.075789\\
8208	43.536828\\
8210	43.07576\\
8212	41.578716\\
8214	43.075788\\
8216	45.660018\\
8219	45.660018\\
8222	45.660018\\
8225	45.660018\\
8228	45.210175\\
8229	43.075789\\
8231	43.075788\\
8233	41.79093\\
8235	41.105432\\
8237	40.709301\\
8239	43.075788\\
8241	44.546421\\
8243	43.613153\\
8245	43.075788\\
8247	43.695279\\
8249	45.660018\\
8252	44.359206\\
8254	43.075788\\
8257	42.863283\\
8259	41.105432\\
8262	41.459297\\
8264	43.075789\\
8266	43.632255\\
8268	44.397418\\
8270	43.993851\\
8272	44.290626\\
8274	45.18614\\
8276	43.461511\\
8278	43.075788\\
8281	42.383272\\
8283	41.105432\\
8286	41.230493\\
8287	43.075788\\
8290	43.075788\\
8292	43.312335\\
8294	43.367155\\
8296	43.556732\\
8298	45.231443\\
8300	44.290218\\
8302	43.075788\\
8305	43.075788\\
8307	42.075876\\
8309	41.577657\\
8311	42.57225\\
8313	43.075788\\
8315	43.075789\\
8317	43.075811\\
8319	43.075788\\
8321	43.886348\\
8323	44.818805\\
8325	43.075789\\
8327	43.075788\\
8330	43.075788\\
8332	41.550617\\
8334	41.37372\\
8336	41.531592\\
8338	43.075782\\
8340	43.075788\\
8342	42.383283\\
8344	42.182135\\
8346	43.075788\\
8348	42.800934\\
8350	41.783356\\
8352	41.783345\\
8354	40.627896\\
8356	38.619118\\
8358	38.628378\\
8360	41.105432\\
8362	43.075785\\
8364	43.075788\\
8367	43.075788\\
8370	43.391089\\
8372	43.075788\\
8374	42.383378\\
8376	43.075788\\
8378	41.105432\\
8380	40.67336\\
8382	41.105432\\
8383	43.075788\\
8385	45.660018\\
8387	45.472131\\
8389	44.821194\\
8391	45.317088\\
8393	45.660018\\
8395	46.517827\\
8397	45.660018\\
8398	43.414396\\
8400	43.075788\\
8402	41.783356\\
8404	41.105432\\
8407	42.575627\\
8409	43.536827\\
8411	45.583007\\
8413	45.660018\\
8416	45.660018\\
8418	45.841343\\
8420	45.784281\\
8422	43.075788\\
8425	43.075788\\
8427	41.327004\\
8429	41.30782\\
8431	43.075788\\
8433	44.505454\\
8435	43.075789\\
8437	43.075788\\
8440	43.075788\\
8443	43.075788\\
8446	42.383279\\
8448	42.307517\\
8450	40.664481\\
8452	39.112244\\
8454	39.872611\\
8456	43.075788\\
8459	43.075788\\
8461	44.147994\\
8463	43.075788\\
8466	43.075788\\
8469	43.075785\\
8471	42.383282\\
};
\addplot [color=mycolor3,line width=2.0pt,mark size=0.3pt,only marks,mark=*,mark options={solid}]
  table[row sep=crcr]{%
8473	40.483043\\
8475	38.089499\\
8477	37.165353\\
8479	37.773519\\
8481	39.198531\\
8483	39.580857\\
8485	39.399977\\
8487	39.580742\\
8489	40.814506\\
8491	41.105432\\
8494	41.049754\\
8496	41.105432\\
8498	39.112247\\
8500	37.874351\\
8502	37.773519\\
8504	38.328066\\
8506	39.232286\\
8508	40.94409\\
8510	41.105431\\
8512	39.872605\\
8514	39.504623\\
8516	39.813219\\
8518	39.198492\\
8521	37.891961\\
8523	36.441768\\
8526	36.583825\\
8527	38.628374\\
8529	41.452598\\
8531	41.783356\\
8533	41.531592\\
8535	41.105432\\
8538	41.105432\\
8540	40.745112\\
8542	39.198492\\
8544	39.596394\\
8546	36.84657\\
8548	36.382303\\
8550	36.582263\\
8551	38.337293\\
8553	41.105432\\
8556	41.105432\\
8559	41.105432\\
8562	41.105435\\
8564	41.524103\\
8566	39.616159\\
8568	40.106746\\
8569	38.337293\\
8571	36.441768\\
8573	35.800695\\
8575	36.998776\\
8577	39.15624\\
8579	41.105431\\
8581	41.105432\\
8584	41.105679\\
8586	41.105431\\
8588	40.365953\\
8590	39.112245\\
8592	41.550617\\
8594	40.145992\\
8596	37.965977\\
8598	37.611232\\
8600	37.737273\\
8602	38.63072\\
8604	39.368333\\
8606	40.007125\\
8608	40.270083\\
8610	41.105432\\
8613	41.105432\\
8615	42.209785\\
8617	41.367315\\
8619	40.274164\\
8621	39.886404\\
8623	41.459293\\
8625	43.0757\\
8627	43.075788\\
8630	43.075788\\
8633	43.075788\\
8636	43.075782\\
8638	41.292475\\
8640	41.550617\\
8642	39.872706\\
8644	38.337123\\
8646	38.337293\\
8648	39.198492\\
8649	41.105432\\
8651	41.783356\\
8653	42.383282\\
8655	42.221959\\
8657	41.930115\\
8659	42.345762\\
8661	41.783356\\
8663	41.531592\\
8665	41.105432\\
8667	39.198492\\
8669	38.337293\\
8671	38.998816\\
8673	39.808709\\
8675	41.105432\\
8677	41.105433\\
8679	41.454319\\
8681	43.075788\\
8684	43.075788\\
8687	43.075788\\
8689	43.067261\\
8691	41.105432\\
8693	40.249099\\
8695	41.105432\\
8697	43.075738\\
8699	43.075788\\
8702	43.075788\\
8705	43.075788\\
8707	43.075795\\
8709	43.075788\\
8712	43.075788\\
8714	41.888651\\
8716	41.105432\\
8718	41.105445\\
8720	43.075787\\
8722	43.075788\\
8725	43.075788\\
8728	43.075788\\
8730	43.567644\\
8732	43.075788\\
8735	43.075788\\
8737	43.075786\\
8739	41.105432\\
8742	41.105432\\
8744	42.382782\\
8746	43.075788\\
8749	43.075788\\
8752	43.075788\\
8755	43.075788\\
8757	42.140638\\
8759	42.383231\\
8760	43.075788\\
};
\addlegendentry{ImpExp};

\end{axis}
\end{tikzpicture}%
    \caption{The realized and the predicted energy prices for 2014}
    \label{fig:EDR_IMPEXP}
\end{figure}
\subsection{Question 3}
% QUESTION 3

\begin{table}[H]
\centering
\begin{tabular}{l | c  c  c  c}
model & mean & variance & min & max \\
\hline
EDR & $37.34$ &  $37.09$ & $24.74$ &  $74.63$ \\
ImpExp & $37.99$ &  $18.95$ & $28.51$ &  $53.4$ \\
ORDC & $40.299$ & $163.29$ & $28.52$ & $754.63$ \\
ORDC2 & $39.95$ & $31.05$ & $28.52$ & $70$ \\
\hline
Market & $40.79$ &  $160.79$ & $ -0.01$ &  $200$ \\
\end{tabular}
\caption{Statistics on the realized and the predicted energy prices}
\end{table}

\begin{figure}[H]
    \centering
    \setlength\fheight{4cm}
    \setlength\fwidth{0.8\textwidth}
    \input{images/ORDC_price.tikz}
    \caption{TRUC}
    \label{fig:capa}
\end{figure}

\begin{figure}[H]
    \centering
    \setlength\fheight{4cm}
    \setlength\fwidth{0.8\textwidth}
    % This file was created by matlab2tikz.
% Minimal pgfplots version: 1.3
%
%The latest updates can be retrieved from
%  http://www.mathworks.com/matlabcentral/fileexchange/22022-matlab2tikz
%where you can also make suggestions and rate matlab2tikz.
%
\definecolor{mycolor1}{rgb}{0.87059,0.49020,0.00000}%
%
\begin{tikzpicture}
\begin{axis}[%
width=\fwidth,
height=\fheight,
at={(0\fwidth,0\fheight)},
scale only axis,
separate axis lines,
every outer x axis line/.append style={black},
every x tick label/.append style={font=\color{black}},
xmin=0,
xmax=8760,
xlabel={time [hour]},
xmajorgrids,
every outer y axis line/.append style={black},
every y tick label/.append style={font=\color{black}},
ymin=20,
ymax=70,
ylabel={ORDC price [\euro/MWh]},
ymajorgrids
]
\addplot [color=mycolor1,solid,line width=0.5pt,forget plot]
  table[row sep=crcr]{%
1	38\\
2	35.7\\
4	33.2\\
6	31.5\\
8	30.1\\
10	29.4\\
12	30.6\\
14	32.7\\
16	32.8\\
18	34.8\\
20	33.4\\
22	33.7\\
24	35.7\\
25	33.4\\
26	31\\
28	28.5\\
30	31.2\\
31	34.4\\
32	36.8\\
34	39.8\\
36	40.5\\
38	41\\
40	41.2\\
42	42\\
44	42\\
45	40.1\\
47	39.8\\
49	38\\
50	35.4\\
52	33.4\\
54	34.4\\
55	37.4\\
57	39.4\\
59	39.8\\
61	40.3\\
63	39.4\\
65	40.2\\
67	39.8\\
69	39.8\\
71	39.8\\
73	38\\
75	36.3\\
77	35.1\\
79	36.9\\
81	38.9\\
83	39.8\\
85	39.4\\
87	39.8\\
89	41\\
91	40.8\\
93	39.1\\
95	39.8\\
97	40\\
99	37.8\\
101	37.7\\
103	37.7\\
105	38\\
107	39.4\\
109	39.8\\
111	39.4\\
113	40.2\\
114	42.1\\
116	41.5\\
118	38\\
120	38.3\\
122	35\\
124	33.3\\
126	35.4\\
127	38.8\\
128	42.1\\
130	42\\
132	42.6\\
134	42.2\\
136	42.2\\
138	44.3\\
140	42.1\\
142	42\\
144	40.9\\
146	37.7\\
148	35.9\\
150	38\\
151	40.7\\
152	42.8\\
154	43.5\\
156	42.9\\
158	42.5\\
160	44.6\\
162	47.4\\
164	45.1\\
166	42.9\\
168	44.4\\
169	41.6\\
170	39.5\\
171	37.7\\
173	37.7\\
174	39.8\\
175	42.6\\
176	44.9\\
177	47.4\\
179	45\\
181	46.2\\
182	49.7\\
184	49.7\\
186	50.5\\
188	49.7\\
190	49\\
192	47.9\\
193	42.1\\
194	39.8\\
196	37.7\\
198	39.4\\
199	42.1\\
200	45.4\\
201	47.2\\
202	45.1\\
204	45.3\\
206	47.1\\
208	46.9\\
210	47\\
212	45.6\\
213	43.6\\
215	44.5\\
217	42.6\\
219	39.8\\
221	40.3\\
223	45.2\\
224	48.6\\
226	47.9\\
228	47.9\\
230	47.4\\
232	47.6\\
234	49.3\\
236	47.4\\
238	45.1\\
240	46.1\\
241	42.1\\
243	39.8\\
245	39.2\\
247	39.8\\
249	42.3\\
251	43.8\\
253	44.1\\
255	43.4\\
257	44.2\\
258	46.9\\
260	44.7\\
261	42.8\\
263	44.6\\
264	42.1\\
266	42.1\\
268	40.2\\
270	39.8\\
272	40.7\\
274	42.1\\
276	42\\
278	40.9\\
280	41\\
282	43.3\\
284	42.3\\
286	39.8\\
288	42\\
289	39.4\\
291	37.7\\
293	36.9\\
295	42.1\\
297	43.7\\
299	45.3\\
301	44.8\\
303	44.6\\
305	45.7\\
307	47.4\\
308	44.9\\
310	42.9\\
312	42.9\\
314	39.8\\
316	38.6\\
318	40.3\\
319	42.9\\
320	46.2\\
322	47.4\\
325	47.4\\
328	47.4\\
330	49.1\\
332	47.5\\
334	44.8\\
336	44.6\\
338	42.1\\
340	39.8\\
342	41.2\\
344	45.1\\
346	45.1\\
348	46\\
350	45.4\\
352	44.8\\
354	46.3\\
356	45.4\\
358	42.8\\
360	42.9\\
362	39.8\\
364	38\\
366	39.4\\
367	42.1\\
369	43.2\\
371	43.6\\
373	45\\
374	47.1\\
376	45.9\\
378	47.4\\
380	45.1\\
382	42.6\\
384	42.2\\
386	39.5\\
388	38\\
390	39.4\\
391	42.1\\
393	43.2\\
395	43.6\\
397	42.6\\
399	42.9\\
401	44.5\\
402	46.4\\
404	44.1\\
406	42.3\\
408	42.9\\
410	39.8\\
412	38.2\\
414	38\\
416	39.4\\
418	41.1\\
420	40.1\\
422	39.8\\
424	40.4\\
426	42.2\\
428	42.1\\
430	42\\
432	42.1\\
433	40.2\\
435	37.9\\
437	37.6\\
439	38\\
441	39.4\\
443	42\\
445	42.1\\
447	42\\
449	42.1\\
451	43.1\\
453	42.2\\
455	42.3\\
457	42\\
459	39.7\\
461	39.4\\
462	41.1\\
463	43.3\\
464	47.2\\
466	47.4\\
468	47.9\\
470	48\\
473	48\\
475	48\\
477	47.4\\
478	43.6\\
480	43.1\\
482	41.9\\
483	39.9\\
485	39.8\\
486	41.9\\
487	43.6\\
488	47.4\\
490	45.1\\
492	46.6\\
494	47\\
496	47.4\\
498	48.6\\
500	47.4\\
502	43.6\\
504	43\\
506	40\\
508	38.9\\
510	39.8\\
511	42.2\\
512	44.6\\
514	43.7\\
516	43.2\\
518	43.7\\
520	45\\
522	47.4\\
524	47.3\\
526	43.8\\
528	44.2\\
529	42.2\\
531	41\\
533	39.8\\
534	41.8\\
536	46.7\\
538	46.1\\
540	45.5\\
542	45.8\\
544	45.5\\
546	47.4\\
547	45.6\\
549	43.2\\
551	42.5\\
553	41.2\\
555	39\\
557	39.4\\
559	43.2\\
560	46.9\\
562	45.1\\
564	44.2\\
566	43.6\\
568	43.9\\
570	46.2\\
572	44.3\\
574	42.1\\
576	42.6\\
578	41.3\\
580	39.3\\
582	38\\
584	39.6\\
585	41.4\\
587	42.1\\
590	42.1\\
592	41.6\\
594	42.1\\
596	41.5\\
598	38.7\\
600	39.4\\
601	37.6\\
602	34.9\\
604	34.5\\
606	35.2\\
608	37.6\\
610	39.4\\
612	39.8\\
614	39.8\\
616	38.6\\
618	39.8\\
620	38.5\\
622	39.3\\
624	39.2\\
626	36.6\\
628	35.2\\
630	37.7\\
631	40.9\\
632	42.8\\
634	42.9\\
636	42.9\\
638	42.9\\
640	43.2\\
642	43.2\\
644	42.9\\
646	42.1\\
648	42.1\\
650	39.3\\
652	37.7\\
654	38.3\\
655	42\\
657	42.5\\
659	42.7\\
661	42.5\\
663	43.2\\
665	43.5\\
667	44\\
669	43.1\\
671	42.2\\
673	40.8\\
675	38.9\\
677	38\\
679	42.1\\
681	43.2\\
683	42.2\\
685	42.1\\
687	42.1\\
689	43.5\\
691	45.1\\
693	43.2\\
695	42.9\\
697	42.1\\
699	39.8\\
701	39.8\\
703	43.2\\
704	46.8\\
705	44.6\\
707	44.6\\
709	44.1\\
711	44.5\\
713	46.6\\
715	47.4\\
717	46.1\\
718	44.2\\
720	44.7\\
721	42.9\\
723	42\\
725	41.4\\
727	44.2\\
728	47.4\\
729	44.8\\
731	43.2\\
733	42.1\\
735	42.1\\
737	42.1\\
739	43.2\\
741	42.1\\
742	39.9\\
744	38.5\\
746	35.9\\
748	34.3\\
750	34\\
752	34.7\\
754	38.3\\
756	38.5\\
758	38.3\\
760	37.7\\
762	38.3\\
764	38.3\\
766	37.4\\
768	38.3\\
769	36.2\\
771	34.7\\
773	34.3\\
775	34.4\\
777	34.6\\
779	35.9\\
781	35.8\\
783	35.7\\
785	36.2\\
786	38.3\\
788	38.5\\
790	38.3\\
792	38.4\\
794	37.3\\
796	36.2\\
798	38.1\\
799	39.8\\
800	43.2\\
802	40.7\\
804	39.1\\
806	39.3\\
808	40.6\\
810	43.6\\
812	43.2\\
814	40.3\\
816	40.3\\
818	38.3\\
820	36.2\\
822	37.7\\
823	39.6\\
824	42.8\\
825	41\\
827	40.7\\
829	39.7\\
831	39.4\\
833	40.7\\
834	42.5\\
836	41.6\\
838	39.1\\
840	39.1\\
842	36.2\\
844	34.6\\
846	35.9\\
847	38.3\\
849	41.1\\
851	41.1\\
853	40.3\\
855	39.5\\
857	40.6\\
858	43.2\\
860	41.1\\
862	39.1\\
864	39.1\\
866	36.4\\
868	35.7\\
870	36.2\\
871	38.4\\
873	39.3\\
875	39.8\\
877	40.2\\
879	40.9\\
881	41.4\\
882	43.2\\
884	40.9\\
886	38.5\\
888	38.8\\
890	36.2\\
892	34.7\\
894	35.9\\
895	38.3\\
897	39.3\\
899	40.6\\
901	41.1\\
903	41\\
905	39.7\\
907	40.8\\
909	39.7\\
911	39.3\\
913	38.2\\
914	36.2\\
916	34.4\\
918	34.3\\
920	34.6\\
922	36.6\\
924	38.3\\
926	38.3\\
928	37.2\\
930	37.3\\
932	37.4\\
934	35.9\\
936	37.1\\
938	34.4\\
940	33.4\\
942	33.2\\
944	34.3\\
946	34.7\\
948	35.9\\
950	35.1\\
952	34.6\\
954	36.2\\
955	38.3\\
957	38.2\\
959	37.6\\
961	38\\
963	36.2\\
965	36.2\\
966	38.2\\
968	43.6\\
970	43.6\\
972	43.4\\
974	43.7\\
976	43.9\\
978	45.3\\
980	44.5\\
982	42\\
984	41.7\\
985	39.3\\
987	37.9\\
989	36.6\\
991	39.4\\
993	40.3\\
995	39.3\\
997	39.1\\
999	39.7\\
1e+03	40.6\\
1e+03	42.9\\
1e+03	42.5\\
1.01e+03	40.3\\
1.01e+03	40.3\\
1.01e+03	38.3\\
1.01e+03	36.2\\
1.01e+03	36.2\\
1.02e+03	39.3\\
1.02e+03	43.4\\
1.02e+03	40.3\\
1.02e+03	38.5\\
1.02e+03	38.3\\
1.02e+03	39.1\\
1.02e+03	39.8\\
1.03e+03	41.6\\
1.03e+03	40.4\\
1.03e+03	39\\
1.03e+03	38.5\\
1.03e+03	36.2\\
1.04e+03	34.8\\
1.04e+03	36.4\\
1.04e+03	38.9\\
1.04e+03	40.3\\
1.04e+03	43.2\\
1.04e+03	43.7\\
1.05e+03	43.9\\
1.05e+03	43.7\\
1.05e+03	44.5\\
1.05e+03	43.7\\
1.05e+03	41.1\\
1.06e+03	40.8\\
1.06e+03	38.5\\
1.06e+03	38.2\\
1.06e+03	38.5\\
1.06e+03	42.1\\
1.06e+03	41.1\\
1.07e+03	40.3\\
1.07e+03	40.6\\
1.07e+03	41.5\\
1.07e+03	41.1\\
1.08e+03	40.6\\
1.08e+03	39.2\\
1.08e+03	38.3\\
1.08e+03	36.3\\
1.08e+03	34.4\\
1.08e+03	34.3\\
1.09e+03	34.6\\
1.09e+03	35.9\\
1.09e+03	36.7\\
1.09e+03	35.9\\
1.1e+03	35.7\\
1.1e+03	36.2\\
1.1e+03	38.2\\
1.1e+03	36.2\\
1.1e+03	36.2\\
1.10e+03	36.2\\
1.11e+03	34.4\\
1.11e+03	34.3\\
1.11e+03	34.4\\
1.11e+03	34.9\\
1.12e+03	35.5\\
1.12e+03	35.2\\
1.12e+03	34.4\\
1.12e+03	35.7\\
1.12e+03	38.3\\
1.12e+03	39.1\\
1.13e+03	39.1\\
1.13e+03	37\\
1.13e+03	35.9\\
1.13e+03	37.4\\
1.14e+03	39.6\\
1.14e+03	43.2\\
1.14e+03	41.1\\
1.14e+03	41.1\\
1.14e+03	39.7\\
1.14e+03	40\\
1.15e+03	40.5\\
1.15e+03	43.2\\
1.15e+03	42.7\\
1.15e+03	40.1\\
1.15e+03	39.6\\
1.15e+03	37.7\\
1.16e+03	36\\
1.16e+03	38.2\\
1.16e+03	41.3\\
1.16e+03	41.4\\
1.16e+03	43.2\\
1.17e+03	43.2\\
1.17e+03	43.8\\
1.17e+03	43.3\\
1.17e+03	40.3\\
1.18e+03	40.3\\
1.18e+03	38.3\\
1.18e+03	36.8\\
1.18e+03	38.3\\
1.18e+03	40.6\\
1.18e+03	43.2\\
1.19e+03	43.2\\
1.19e+03	43.7\\
1.19e+03	43.2\\
1.19e+03	43.2\\
1.19e+03	43.7\\
1.2e+03	43.7\\
1.2e+03	42.2\\
1.2e+03	41.3\\
1.2e+03	38.7\\
1.2e+03	36.5\\
1.20e+03	35.9\\
1.21e+03	38.6\\
1.21e+03	39.1\\
1.21e+03	39.4\\
1.21e+03	40.5\\
1.22e+03	40.6\\
1.22e+03	41.1\\
1.22e+03	43.1\\
1.22e+03	40.4\\
1.22e+03	40\\
1.22e+03	38.5\\
1.23e+03	36.2\\
1.23e+03	36.2\\
1.23e+03	38.2\\
1.23e+03	41.1\\
1.23e+03	40.3\\
1.24e+03	40.6\\
1.24e+03	39.1\\
1.24e+03	39.1\\
1.24e+03	40\\
1.24e+03	42.6\\
1.24e+03	39.7\\
1.25e+03	39.1\\
1.25e+03	38.2\\
1.25e+03	36.2\\
1.25e+03	34.6\\
1.26e+03	35.3\\
1.26e+03	37.2\\
1.26e+03	38.5\\
1.26e+03	38.3\\
1.26e+03	37.3\\
1.26e+03	38.2\\
1.27e+03	39.1\\
1.27e+03	38.3\\
1.27e+03	37.9\\
1.27e+03	36.9\\
1.28e+03	34.6\\
1.28e+03	34.3\\
1.28e+03	34.3\\
1.28e+03	33.3\\
1.28e+03	34.3\\
1.28e+03	33.4\\
1.29e+03	33.9\\
1.29e+03	35.7\\
1.29e+03	38.2\\
1.29e+03	37.6\\
1.3e+03	37.5\\
1.3e+03	34.6\\
1.3e+03	34.3\\
1.3e+03	35.1\\
1.3e+03	38.3\\
1.30e+03	38.5\\
1.31e+03	38.3\\
1.31e+03	35.9\\
1.31e+03	37.7\\
1.31e+03	38.5\\
1.32e+03	40.6\\
1.32e+03	39.9\\
1.32e+03	39\\
1.32e+03	36.5\\
1.32e+03	34.6\\
1.32e+03	34.4\\
1.33e+03	38.3\\
1.33e+03	40.3\\
1.33e+03	39.1\\
1.33e+03	38.5\\
1.33e+03	39.1\\
1.34e+03	39.3\\
1.34e+03	40.8\\
1.34e+03	41.8\\
1.34e+03	39.8\\
1.34e+03	39.3\\
1.34e+03	38.3\\
1.35e+03	36.4\\
1.35e+03	36.4\\
1.35e+03	38.2\\
1.35e+03	40\\
1.35e+03	41.3\\
1.36e+03	41.7\\
1.36e+03	40.3\\
1.36e+03	39.9\\
1.36e+03	39.9\\
1.36e+03	43.2\\
1.36e+03	41.4\\
1.37e+03	40.6\\
1.37e+03	38.3\\
1.37e+03	36.2\\
1.37e+03	36.2\\
1.38e+03	39.1\\
1.38e+03	38.6\\
1.38e+03	39.1\\
1.38e+03	39.9\\
1.38e+03	40.7\\
1.38e+03	41.1\\
1.39e+03	43.2\\
1.39e+03	43.1\\
1.39e+03	42.8\\
1.39e+03	39.1\\
1.4e+03	37.5\\
1.4e+03	37.3\\
1.4e+03	41.7\\
1.4e+03	41.9\\
1.4e+03	43.2\\
1.40e+03	43.7\\
1.41e+03	43.7\\
1.41e+03	43.4\\
1.41e+03	43.6\\
1.41e+03	43.2\\
1.42e+03	43.2\\
1.42e+03	41.4\\
1.42e+03	38.3\\
1.42e+03	36.4\\
1.42e+03	36.3\\
1.42e+03	36.4\\
1.42e+03	36.6\\
1.43e+03	37.5\\
1.43e+03	38.8\\
1.43e+03	37.4\\
1.43e+03	37.4\\
1.44e+03	39.1\\
1.44e+03	37.7\\
1.44e+03	37.9\\
1.44e+03	37.1\\
1.44e+03	35.5\\
1.44e+03	34.4\\
1.45e+03	34.4\\
1.45e+03	33.8\\
1.45e+03	32.9\\
1.45e+03	32.3\\
1.46e+03	32.7\\
1.46e+03	34.4\\
1.46e+03	35.9\\
1.46e+03	35.7\\
1.46e+03	36\\
1.47e+03	33.4\\
1.47e+03	32.8\\
1.47e+03	34.4\\
1.47e+03	36.5\\
1.47e+03	37.7\\
1.48e+03	37.4\\
1.48e+03	36.9\\
1.48e+03	37.4\\
1.48e+03	38.2\\
1.48e+03	40.1\\
1.48e+03	39.2\\
1.49e+03	39.3\\
1.49e+03	38.1\\
1.49e+03	36.4\\
1.49e+03	36.6\\
1.5e+03	41\\
1.5e+03	41.1\\
1.5e+03	41.4\\
1.5e+03	41.1\\
1.5e+03	41.1\\
1.50e+03	41.5\\
1.51e+03	43.1\\
1.51e+03	42.5\\
1.51e+03	42.5\\
1.51e+03	40.9\\
1.51e+03	38.7\\
1.52e+03	37.2\\
1.52e+03	38.4\\
1.52e+03	41.4\\
1.52e+03	42\\
1.52e+03	41.4\\
1.52e+03	40.5\\
1.53e+03	40.3\\
1.53e+03	41.1\\
1.53e+03	41.4\\
1.53e+03	43.1\\
1.54e+03	42.2\\
1.54e+03	40.2\\
1.54e+03	38\\
1.54e+03	36.4\\
1.54e+03	37.4\\
1.54e+03	40\\
1.54e+03	39.1\\
1.55e+03	38.6\\
1.55e+03	37.1\\
1.55e+03	36.6\\
1.55e+03	37.9\\
1.55e+03	40.3\\
1.56e+03	41.2\\
1.56e+03	39.4\\
1.56e+03	38.9\\
1.56e+03	36.6\\
1.56e+03	34.4\\
1.56e+03	34.5\\
1.57e+03	36.3\\
1.57e+03	37.3\\
1.57e+03	37.4\\
1.57e+03	37.4\\
1.57e+03	36.6\\
1.58e+03	37.4\\
1.58e+03	39.3\\
1.58e+03	40.5\\
1.58e+03	38.7\\
1.58e+03	39.5\\
1.58e+03	37.4\\
1.59e+03	34.8\\
1.59e+03	34.4\\
1.59e+03	34.5\\
1.59e+03	35.5\\
1.6e+03	34.4\\
1.6e+03	33.9\\
1.6e+03	33\\
1.6e+03	34.4\\
1.6e+03	36.3\\
1.6e+03	36.4\\
1.61e+03	34.4\\
1.61e+03	35.2\\
1.61e+03	33.1\\
1.61e+03	32.7\\
1.61e+03	32.7\\
1.62e+03	32.9\\
1.62e+03	32.9\\
1.62e+03	32.7\\
1.62e+03	31.5\\
1.62e+03	32.2\\
1.63e+03	33.8\\
1.63e+03	36.4\\
1.63e+03	35.8\\
1.63e+03	35.6\\
1.63e+03	34.3\\
1.64e+03	33.6\\
1.64e+03	35.7\\
1.64e+03	39.4\\
1.64e+03	39.3\\
1.64e+03	38.6\\
1.64e+03	36.9\\
1.65e+03	36.9\\
1.65e+03	38.3\\
1.65e+03	40.9\\
1.65e+03	41.1\\
1.65e+03	38.9\\
1.66e+03	38.3\\
1.66e+03	36.4\\
1.66e+03	34.4\\
1.66e+03	34.4\\
1.66e+03	36.3\\
1.66e+03	38\\
1.67e+03	39\\
1.67e+03	41.1\\
1.67e+03	38.1\\
1.67e+03	36.4\\
1.67e+03	37.4\\
1.68e+03	41\\
1.68e+03	40.6\\
1.68e+03	38.2\\
1.68e+03	38.1\\
1.68e+03	36.2\\
1.68e+03	34.9\\
1.69e+03	36.4\\
1.69e+03	39.7\\
1.69e+03	38.6\\
1.69e+03	38.3\\
1.69e+03	37\\
1.7e+03	36.6\\
1.7e+03	38\\
1.7e+03	40.3\\
1.7e+03	41.5\\
1.7e+03	40.8\\
1.7e+03	39.1\\
1.70e+03	37.2\\
1.71e+03	36.3\\
1.71e+03	36.4\\
1.71e+03	41.1\\
1.71e+03	41.1\\
1.72e+03	40.1\\
1.72e+03	37.9\\
1.72e+03	37.8\\
1.72e+03	38.2\\
1.72e+03	40\\
1.72e+03	41.1\\
1.73e+03	40.5\\
1.73e+03	37.1\\
1.73e+03	36.3\\
1.73e+03	36.6\\
1.74e+03	39.8\\
1.74e+03	39.3\\
1.74e+03	39.1\\
1.74e+03	37.4\\
1.74e+03	36.5\\
1.74e+03	37.6\\
1.75e+03	40.5\\
1.75e+03	39.9\\
1.75e+03	41.1\\
1.75e+03	37.1\\
1.76e+03	36.4\\
1.76e+03	35.6\\
1.76e+03	36.4\\
1.76e+03	37.1\\
1.76e+03	38.3\\
1.76e+03	38.2\\
1.77e+03	36.6\\
1.77e+03	36.9\\
1.77e+03	38.6\\
1.77e+03	38.6\\
1.78e+03	37.4\\
1.78e+03	36.4\\
1.78e+03	34.4\\
1.78e+03	34.4\\
1.78e+03	34.4\\
1.78e+03	35.2\\
1.79e+03	36.4\\
1.79e+03	36.4\\
1.79e+03	34.4\\
1.79e+03	34.5\\
1.79e+03	36.4\\
1.8e+03	39.1\\
1.8e+03	37.8\\
1.8e+03	37.6\\
1.8e+03	36.3\\
1.8e+03	35.3\\
1.81e+03	37\\
1.81e+03	41.1\\
1.81e+03	41.4\\
1.81e+03	43.1\\
1.81e+03	42.1\\
1.82e+03	41.5\\
1.82e+03	42.7\\
1.82e+03	43.1\\
1.82e+03	42.6\\
1.82e+03	41.5\\
1.82e+03	39.6\\
1.83e+03	37.9\\
1.83e+03	36.8\\
1.83e+03	37.9\\
1.83e+03	41.4\\
1.83e+03	42.6\\
1.84e+03	42.3\\
1.84e+03	42.6\\
1.84e+03	41.6\\
1.84e+03	41.5\\
1.84e+03	42.4\\
1.84e+03	41.1\\
1.85e+03	40.9\\
1.85e+03	37.4\\
1.85e+03	36.4\\
1.85e+03	36.4\\
1.86e+03	41.1\\
1.86e+03	41\\
1.86e+03	42.8\\
1.86e+03	41.5\\
1.86e+03	41.1\\
1.86e+03	41.5\\
1.87e+03	42.6\\
1.87e+03	43.1\\
1.87e+03	41.9\\
1.87e+03	39.9\\
1.87e+03	37.9\\
1.88e+03	37.1\\
1.88e+03	37.1\\
1.88e+03	40.4\\
1.88e+03	37.4\\
1.88e+03	37.1\\
1.88e+03	36.4\\
1.89e+03	35.9\\
1.89e+03	36.6\\
1.89e+03	38\\
1.89e+03	40.6\\
1.9e+03	38.8\\
1.9e+03	36.4\\
1.9e+03	35.1\\
1.9e+03	36.3\\
1.9e+03	40\\
1.90e+03	42\\
1.91e+03	42.9\\
1.91e+03	43.1\\
1.91e+03	41.2\\
1.91e+03	41.1\\
1.92e+03	40.5\\
1.92e+03	41.1\\
1.92e+03	41.1\\
1.92e+03	37.5\\
1.92e+03	35.8\\
1.92e+03	34.4\\
1.93e+03	34.5\\
1.93e+03	35.7\\
1.93e+03	36.6\\
1.93e+03	37\\
1.94e+03	36.4\\
1.94e+03	36.4\\
1.94e+03	36.5\\
1.94e+03	37.1\\
1.94e+03	37.5\\
1.94e+03	37.1\\
1.95e+03	35.7\\
1.95e+03	34.6\\
1.95e+03	36\\
1.95e+03	35.9\\
1.96e+03	36.4\\
1.96e+03	37.1\\
1.96e+03	36.3\\
1.96e+03	36.4\\
1.96e+03	38.7\\
1.96e+03	38.6\\
1.97e+03	37.8\\
1.97e+03	37.2\\
1.97e+03	36.4\\
1.97e+03	37.7\\
1.98e+03	41.1\\
1.98e+03	43\\
1.98e+03	41.2\\
1.98e+03	41.1\\
1.98e+03	41.1\\
1.98e+03	41.2\\
1.99e+03	43.1\\
1.99e+03	43.1\\
1.99e+03	42.1\\
1.99e+03	41.6\\
1.99e+03	39.8\\
1.99e+03	38.1\\
2e+03	37.1\\
2e+03	38.7\\
2e+03	41.5\\
2e+03	42.1\\
2e+03	41.1\\
2.00e+03	39.9\\
2.01e+03	40.4\\
2.01e+03	41.3\\
2.01e+03	43.1\\
2.01e+03	42.8\\
2.02e+03	41.1\\
2.02e+03	39\\
2.02e+03	37.1\\
2.02e+03	37.1\\
2.02e+03	41.5\\
2.02e+03	41.5\\
2.03e+03	41.5\\
2.03e+03	41.4\\
2.03e+03	43.1\\
2.03e+03	43.9\\
2.04e+03	45.2\\
2.04e+03	47.3\\
2.04e+03	46.6\\
2.04e+03	46.3\\
2.04e+03	43.3\\
2.04e+03	43.1\\
2.05e+03	43.4\\
2.05e+03	45.6\\
2.05e+03	46.2\\
2.05e+03	46.2\\
2.05e+03	45.2\\
2.06e+03	45.6\\
2.06e+03	46.1\\
2.06e+03	49.5\\
2.06e+03	47.5\\
2.06e+03	47\\
2.06e+03	45.1\\
2.07e+03	43.1\\
2.07e+03	44.1\\
2.07e+03	46\\
2.07e+03	45.8\\
2.08e+03	45.6\\
2.08e+03	45.6\\
2.08e+03	45.5\\
2.08e+03	45.6\\
2.08e+03	46.5\\
2.08e+03	45.8\\
2.09e+03	45.6\\
2.09e+03	43.1\\
2.09e+03	41.5\\
2.09e+03	41.5\\
2.1e+03	42.3\\
2.1e+03	43.1\\
2.1e+03	42.5\\
2.1e+03	41.4\\
2.1e+03	40.9\\
2.10e+03	41.5\\
2.11e+03	43.1\\
2.11e+03	43.4\\
2.11e+03	43.1\\
2.11e+03	41.9\\
2.12e+03	41.1\\
2.12e+03	41.1\\
2.12e+03	41.1\\
2.12e+03	41.6\\
2.12e+03	41.2\\
2.12e+03	39.8\\
2.13e+03	39\\
2.13e+03	41.1\\
2.13e+03	43.1\\
2.13e+03	43.1\\
2.14e+03	42.4\\
2.14e+03	41.4\\
2.14e+03	41.5\\
2.14e+03	44.9\\
2.14e+03	47.3\\
2.14e+03	45.6\\
2.15e+03	44.7\\
2.15e+03	43.1\\
2.15e+03	43.1\\
2.15e+03	45.3\\
2.16e+03	45.6\\
2.16e+03	47.5\\
2.16e+03	45.7\\
2.16e+03	44.5\\
2.16e+03	41.4\\
2.16e+03	40.9\\
2.16e+03	40.9\\
2.17e+03	43.5\\
2.17e+03	43.6\\
2.17e+03	43.1\\
2.17e+03	42.4\\
2.17e+03	42.4\\
2.18e+03	42.6\\
2.18e+03	44.4\\
2.18e+03	44.4\\
2.18e+03	45.3\\
2.18e+03	43.5\\
2.18e+03	41.3\\
2.19e+03	39.5\\
2.19e+03	39\\
2.19e+03	42.6\\
2.19e+03	44.3\\
2.19e+03	43.5\\
2.2e+03	42.8\\
2.2e+03	42.4\\
2.2e+03	42.2\\
2.2e+03	42.6\\
2.2e+03	44.2\\
2.20e+03	44.6\\
2.21e+03	42.4\\
2.21e+03	39.9\\
2.21e+03	38.6\\
2.21e+03	39.7\\
2.21e+03	41.6\\
2.22e+03	42.7\\
2.22e+03	42.4\\
2.22e+03	42.1\\
2.22e+03	42.1\\
2.22e+03	42.4\\
2.23e+03	45.4\\
2.23e+03	45.7\\
2.23e+03	43.7\\
2.23e+03	40.9\\
2.23e+03	39.1\\
2.24e+03	39.7\\
2.24e+03	42.6\\
2.24e+03	45.4\\
2.24e+03	47.1\\
2.24e+03	48.2\\
2.24e+03	51.1\\
2.24e+03	45.1\\
2.25e+03	43.2\\
2.25e+03	42.6\\
2.25e+03	44.6\\
2.25e+03	43.7\\
2.25e+03	43.7\\
2.26e+03	40.9\\
2.26e+03	39\\
2.26e+03	38.6\\
2.26e+03	39.9\\
2.26e+03	40.9\\
2.27e+03	40.9\\
2.27e+03	40.9\\
2.27e+03	39.6\\
2.27e+03	39.5\\
2.27e+03	40.9\\
2.28e+03	41.3\\
2.28e+03	40.9\\
2.28e+03	38.6\\
2.28e+03	38.1\\
2.28e+03	37.9\\
2.29e+03	38.2\\
2.29e+03	38.6\\
2.29e+03	38.6\\
2.29e+03	38.6\\
2.3e+03	38.6\\
2.3e+03	39.9\\
2.3e+03	40.9\\
2.3e+03	40.6\\
2.3e+03	38.6\\
2.31e+03	38.6\\
2.31e+03	38.6\\
2.31e+03	40.9\\
2.31e+03	42.6\\
2.31e+03	43.4\\
2.32e+03	42.4\\
2.32e+03	42.3\\
2.32e+03	42.4\\
2.32e+03	42.4\\
2.32e+03	42.4\\
2.33e+03	42.6\\
2.33e+03	39.9\\
2.33e+03	38.6\\
2.33e+03	39.7\\
2.34e+03	42\\
2.34e+03	42.4\\
2.34e+03	42.4\\
2.34e+03	40.9\\
2.34e+03	41.1\\
2.34e+03	42.4\\
2.35e+03	42.4\\
2.35e+03	45\\
2.35e+03	45.4\\
2.35e+03	42.1\\
2.35e+03	40.8\\
2.36e+03	40.9\\
2.36e+03	44.3\\
2.36e+03	45.7\\
2.36e+03	44.6\\
2.36e+03	42.4\\
2.37e+03	41.6\\
2.37e+03	42.4\\
2.37e+03	44.3\\
2.37e+03	43.6\\
2.37e+03	44.5\\
2.38e+03	42.4\\
2.38e+03	40.9\\
2.38e+03	41.4\\
2.38e+03	44.4\\
2.38e+03	47.1\\
2.38e+03	45.7\\
2.39e+03	45.5\\
2.39e+03	43\\
2.39e+03	42.8\\
2.39e+03	43.3\\
2.39e+03	46.1\\
2.4e+03	45.1\\
2.4e+03	45.1\\
2.4e+03	41.7\\
2.4e+03	40.3\\
2.4e+03	40.4\\
2.41e+03	42.7\\
2.41e+03	44.5\\
2.41e+03	45\\
2.41e+03	44.2\\
2.41e+03	42.4\\
2.42e+03	42.4\\
2.42e+03	42.6\\
2.42e+03	42.6\\
2.42e+03	43.5\\
2.42e+03	42.4\\
2.42e+03	40.9\\
2.43e+03	39.7\\
2.43e+03	39.9\\
2.43e+03	40.9\\
2.43e+03	41.4\\
2.44e+03	40.9\\
2.44e+03	38.8\\
2.44e+03	38.6\\
2.44e+03	40.1\\
2.44e+03	40.9\\
2.45e+03	40.9\\
2.45e+03	38.1\\
2.45e+03	36.7\\
2.45e+03	36.1\\
2.45e+03	36.8\\
2.46e+03	37.5\\
2.46e+03	37.2\\
2.46e+03	37.2\\
2.46e+03	35.3\\
2.46e+03	35.5\\
2.47e+03	38.1\\
2.47e+03	38.6\\
2.47e+03	38.6\\
2.47e+03	37.1\\
2.47e+03	35.7\\
2.48e+03	35.4\\
2.48e+03	38.6\\
2.48e+03	39.9\\
2.48e+03	39.9\\
2.48e+03	39\\
2.49e+03	38.6\\
2.49e+03	38.6\\
2.49e+03	40.3\\
2.49e+03	40.8\\
2.49e+03	40.9\\
2.5e+03	39\\
2.5e+03	38.6\\
2.5e+03	38.8\\
2.5e+03	40.9\\
2.5e+03	42.4\\
2.51e+03	41.3\\
2.51e+03	40.9\\
2.51e+03	40.3\\
2.51e+03	39.9\\
2.51e+03	40.9\\
2.52e+03	41.3\\
2.52e+03	42.1\\
2.52e+03	40.4\\
2.52e+03	38.6\\
2.52e+03	38.6\\
2.53e+03	41.1\\
2.53e+03	40.9\\
2.53e+03	40.9\\
2.53e+03	39.9\\
2.53e+03	38.9\\
2.54e+03	39.8\\
2.54e+03	41.1\\
2.54e+03	41.6\\
2.54e+03	42\\
2.54e+03	40\\
2.55e+03	38.6\\
2.55e+03	38.6\\
2.55e+03	40.4\\
2.55e+03	40.9\\
2.56e+03	40.6\\
2.56e+03	38.6\\
2.56e+03	38.6\\
2.56e+03	40.4\\
2.56e+03	40.9\\
2.57e+03	40.9\\
2.57e+03	39.3\\
2.57e+03	38.6\\
2.57e+03	38.1\\
2.57e+03	39.7\\
2.58e+03	40.5\\
2.58e+03	39.9\\
2.58e+03	39.5\\
2.58e+03	38.6\\
2.58e+03	38.6\\
2.59e+03	39.6\\
2.59e+03	39\\
2.59e+03	40.9\\
2.59e+03	38.6\\
2.59e+03	37.6\\
2.6e+03	37.6\\
2.6e+03	38\\
2.6e+03	38.6\\
2.6e+03	38.6\\
2.60e+03	37.5\\
2.61e+03	36.6\\
2.61e+03	37.2\\
2.61e+03	38.6\\
2.61e+03	38.4\\
2.62e+03	38.1\\
2.62e+03	36\\
2.62e+03	34.4\\
2.62e+03	34.4\\
2.62e+03	34.9\\
2.62e+03	36.6\\
2.63e+03	36.8\\
2.63e+03	36.8\\
2.63e+03	35.7\\
2.63e+03	35\\
2.63e+03	36.8\\
2.64e+03	38.6\\
2.64e+03	38.6\\
2.64e+03	38.5\\
2.64e+03	36.8\\
2.64e+03	36.4\\
2.65e+03	36.8\\
2.65e+03	36.8\\
2.65e+03	36.8\\
2.65e+03	35.2\\
2.66e+03	35.7\\
2.66e+03	38\\
2.66e+03	38.6\\
2.66e+03	38.6\\
2.66e+03	37.6\\
2.67e+03	36.9\\
2.67e+03	38.6\\
2.67e+03	41.4\\
2.67e+03	40.9\\
2.68e+03	40.9\\
2.68e+03	40.3\\
2.68e+03	40.3\\
2.68e+03	40.9\\
2.68e+03	41.8\\
2.68e+03	41.4\\
2.69e+03	40.9\\
2.69e+03	38.6\\
2.69e+03	38.6\\
2.69e+03	40.9\\
2.7e+03	40.9\\
2.7e+03	40.9\\
2.7e+03	40.9\\
2.7e+03	40.9\\
2.7e+03	40.9\\
2.71e+03	42.1\\
2.71e+03	41.3\\
2.71e+03	42.4\\
2.71e+03	39.6\\
2.71e+03	38.6\\
2.72e+03	38.6\\
2.72e+03	40.9\\
2.72e+03	42.4\\
2.72e+03	42.4\\
2.72e+03	42.4\\
2.73e+03	41.4\\
2.73e+03	41.8\\
2.73e+03	42.4\\
2.73e+03	41.3\\
2.73e+03	40.9\\
2.74e+03	38.6\\
2.74e+03	37.1\\
2.74e+03	36.9\\
2.74e+03	38.6\\
2.74e+03	39.9\\
2.75e+03	39\\
2.75e+03	38.6\\
2.75e+03	38.6\\
2.75e+03	39.8\\
2.76e+03	40.9\\
2.76e+03	40.7\\
2.76e+03	39.7\\
2.76e+03	37.4\\
2.76e+03	36.8\\
2.77e+03	36.8\\
2.77e+03	37.5\\
2.77e+03	37.3\\
2.77e+03	36.8\\
2.77e+03	36.7\\
2.78e+03	37.2\\
2.78e+03	38.6\\
2.78e+03	38.6\\
2.78e+03	38.6\\
2.78e+03	36.1\\
2.79e+03	34.4\\
2.79e+03	34.5\\
2.79e+03	34.6\\
2.79e+03	36.4\\
2.8e+03	36.8\\
2.8e+03	35.7\\
2.8e+03	35.4\\
2.8e+03	36.8\\
2.8e+03	38.6\\
2.80e+03	38.6\\
2.81e+03	38.6\\
2.81e+03	36.8\\
2.81e+03	36.3\\
2.81e+03	38\\
2.82e+03	40.7\\
2.82e+03	40.1\\
2.82e+03	40.9\\
2.82e+03	40.9\\
2.82e+03	40.9\\
2.83e+03	41.4\\
2.83e+03	41.4\\
2.83e+03	41.1\\
2.83e+03	38.8\\
2.83e+03	38.6\\
2.84e+03	38.6\\
2.84e+03	39.9\\
2.84e+03	41.5\\
2.84e+03	41.8\\
2.84e+03	42.1\\
2.85e+03	42.4\\
2.85e+03	42.2\\
2.85e+03	41.5\\
2.85e+03	40.9\\
2.86e+03	39.9\\
2.86e+03	38.6\\
2.86e+03	38\\
2.86e+03	38.6\\
2.86e+03	40.9\\
2.86e+03	41.4\\
2.87e+03	41\\
2.87e+03	40.9\\
2.87e+03	39.9\\
2.87e+03	40.5\\
2.88e+03	40.7\\
2.88e+03	40.9\\
2.88e+03	40.2\\
2.88e+03	37\\
2.88e+03	36.1\\
2.88e+03	35.6\\
2.89e+03	35.1\\
2.89e+03	34.9\\
2.89e+03	35.6\\
2.89e+03	36.1\\
2.89e+03	36.1\\
2.9e+03	36.1\\
2.9e+03	37.2\\
2.9e+03	37.3\\
2.9e+03	38.3\\
2.9e+03	36.1\\
2.91e+03	34.8\\
2.91e+03	34.8\\
2.91e+03	36.1\\
2.91e+03	36.1\\
2.92e+03	36.1\\
2.92e+03	34.7\\
2.92e+03	34.5\\
2.92e+03	35.6\\
2.92e+03	36.1\\
2.93e+03	36.1\\
2.93e+03	35.7\\
2.93e+03	34.5\\
2.93e+03	34.4\\
2.93e+03	34.5\\
2.94e+03	35.7\\
2.94e+03	35.7\\
2.94e+03	34.8\\
2.94e+03	34.5\\
2.94e+03	34.5\\
2.95e+03	36.1\\
2.95e+03	36.1\\
2.95e+03	36.1\\
2.95e+03	35.7\\
2.96e+03	34.8\\
2.96e+03	34.9\\
2.96e+03	35.2\\
2.96e+03	34.9\\
2.96e+03	34.5\\
2.96e+03	33.4\\
2.97e+03	32.8\\
2.97e+03	34.8\\
2.97e+03	36.1\\
2.97e+03	36.1\\
2.98e+03	36.1\\
2.98e+03	35.2\\
2.98e+03	34.5\\
2.98e+03	36.1\\
2.98e+03	38.7\\
2.98e+03	38.3\\
2.99e+03	37.1\\
2.99e+03	36.1\\
2.99e+03	36.1\\
2.99e+03	38.3\\
3e+03	38.8\\
3e+03	38.3\\
3e+03	37.4\\
3e+03	34.9\\
3e+03	34.5\\
3.00e+03	36\\
3.01e+03	38.3\\
3.01e+03	39.3\\
3.01e+03	38.9\\
3.01e+03	38.3\\
3.02e+03	37.8\\
3.02e+03	38.3\\
3.02e+03	38.6\\
3.02e+03	38.3\\
3.02e+03	38.3\\
3.02e+03	36.1\\
3.03e+03	35.9\\
3.03e+03	36.1\\
3.03e+03	37.4\\
3.03e+03	39.3\\
3.03e+03	38.7\\
3.04e+03	38.5\\
3.04e+03	38.3\\
3.04e+03	38.2\\
3.04e+03	38.3\\
3.04e+03	38.3\\
3.05e+03	38.3\\
3.05e+03	36.1\\
3.05e+03	35.6\\
3.05e+03	35.7\\
3.05e+03	37.5\\
3.06e+03	38.3\\
3.06e+03	39.2\\
3.06e+03	39\\
3.06e+03	39.6\\
3.06e+03	39.7\\
3.07e+03	39.7\\
3.07e+03	38.3\\
3.07e+03	38.2\\
3.07e+03	36.1\\
3.07e+03	34.8\\
3.08e+03	34.8\\
3.08e+03	37\\
3.08e+03	38.4\\
3.08e+03	38.3\\
3.08e+03	37.4\\
3.09e+03	36.1\\
3.09e+03	36.1\\
3.09e+03	36.7\\
3.09e+03	36.1\\
3.1e+03	37.2\\
3.1e+03	36.1\\
3.1e+03	34.9\\
3.1e+03	34.8\\
3.1e+03	35.9\\
3.10e+03	36.5\\
3.11e+03	36.5\\
3.11e+03	36.1\\
3.11e+03	36.1\\
3.11e+03	35.4\\
3.12e+03	35.7\\
3.12e+03	34.9\\
3.12e+03	34.9\\
3.12e+03	32.8\\
3.12e+03	31.2\\
3.12e+03	31.1\\
3.13e+03	30.9\\
3.13e+03	32.1\\
3.13e+03	32.4\\
3.13e+03	31.3\\
3.14e+03	30.6\\
3.14e+03	32.4\\
3.14e+03	34.4\\
3.14e+03	34.5\\
3.14e+03	34.8\\
3.14e+03	33.6\\
3.15e+03	32.4\\
3.15e+03	33.3\\
3.15e+03	36.1\\
3.15e+03	37.1\\
3.15e+03	37.8\\
3.16e+03	36.7\\
3.16e+03	37.6\\
3.16e+03	38.3\\
3.16e+03	38.5\\
3.16e+03	39.7\\
3.17e+03	38.9\\
3.17e+03	36.1\\
3.17e+03	36.1\\
3.17e+03	36.1\\
3.17e+03	38\\
3.18e+03	39.7\\
3.18e+03	38.5\\
3.18e+03	39.7\\
3.18e+03	38.3\\
3.18e+03	38.5\\
3.19e+03	39.2\\
3.19e+03	38.7\\
3.19e+03	39.7\\
3.19e+03	37.7\\
3.19e+03	36.1\\
3.2e+03	37\\
3.2e+03	39.7\\
3.2e+03	38.5\\
3.2e+03	38.7\\
3.20e+03	38.3\\
3.21e+03	38.2\\
3.21e+03	38.3\\
3.21e+03	38.3\\
3.21e+03	39.1\\
3.22e+03	36.6\\
3.22e+03	36.1\\
3.22e+03	36.1\\
3.22e+03	38.3\\
3.22e+03	39.3\\
3.23e+03	39\\
3.23e+03	37.8\\
3.23e+03	36.5\\
3.23e+03	37.4\\
3.23e+03	38.3\\
3.24e+03	38.3\\
3.24e+03	39.1\\
3.24e+03	37.1\\
3.24e+03	36.1\\
3.24e+03	36.1\\
3.25e+03	38\\
3.25e+03	38.8\\
3.25e+03	38.3\\
3.25e+03	36.9\\
3.25e+03	36.1\\
3.26e+03	36.1\\
3.26e+03	38.2\\
3.26e+03	38.3\\
3.26e+03	38.3\\
3.26e+03	37.4\\
3.27e+03	36.1\\
3.27e+03	36.1\\
3.27e+03	36.1\\
3.27e+03	36.2\\
3.27e+03	36.1\\
3.28e+03	36.1\\
3.28e+03	34.8\\
3.28e+03	35.7\\
3.28e+03	36.1\\
3.28e+03	37.2\\
3.29e+03	37.2\\
3.29e+03	36.1\\
3.29e+03	34.8\\
3.29e+03	34.5\\
3.29e+03	34.5\\
3.3e+03	34.6\\
3.3e+03	34.5\\
3.3e+03	34.3\\
3.3e+03	33\\
3.3e+03	33.6\\
3.31e+03	35.5\\
3.31e+03	36.1\\
3.31e+03	36.1\\
3.31e+03	34.9\\
3.31e+03	34.4\\
3.32e+03	34.5\\
3.32e+03	36.1\\
3.32e+03	38.2\\
3.32e+03	38.3\\
3.32e+03	37.4\\
3.33e+03	37.4\\
3.33e+03	38.3\\
3.33e+03	38.3\\
3.33e+03	37.4\\
3.34e+03	37.2\\
3.34e+03	36.1\\
3.34e+03	34.8\\
3.34e+03	36.1\\
3.34e+03	38.2\\
3.34e+03	38.3\\
3.35e+03	37.8\\
3.35e+03	38.1\\
3.35e+03	38.3\\
3.35e+03	39.7\\
3.36e+03	38.9\\
3.36e+03	38.9\\
3.36e+03	37.8\\
3.36e+03	36.1\\
3.36e+03	36.1\\
3.37e+03	38.2\\
3.37e+03	38.9\\
3.37e+03	38.9\\
3.37e+03	39.7\\
3.37e+03	39.6\\
3.38e+03	39.7\\
3.38e+03	39.6\\
3.38e+03	38.4\\
3.38e+03	38.8\\
3.38e+03	36.1\\
3.38e+03	34.4\\
3.39e+03	33.4\\
3.39e+03	34.5\\
3.39e+03	36.2\\
3.39e+03	37.4\\
3.4e+03	36.6\\
3.4e+03	36.3\\
3.4e+03	36.8\\
3.4e+03	37.9\\
3.4e+03	38.3\\
3.40e+03	38.3\\
3.41e+03	37.4\\
3.41e+03	35.6\\
3.41e+03	34.7\\
3.41e+03	35.7\\
3.42e+03	38.3\\
3.42e+03	38.3\\
3.42e+03	38.1\\
3.42e+03	37.6\\
3.42e+03	37.4\\
3.43e+03	38.1\\
3.43e+03	37.2\\
3.43e+03	37.3\\
3.43e+03	36.1\\
3.43e+03	35\\
3.44e+03	34.5\\
3.44e+03	34.5\\
3.44e+03	35.7\\
3.44e+03	35.6\\
3.44e+03	34.8\\
3.45e+03	33.4\\
3.45e+03	34.5\\
3.45e+03	35.6\\
3.45e+03	36.1\\
3.45e+03	36.1\\
3.46e+03	34.9\\
3.46e+03	34.3\\
3.46e+03	34.2\\
3.46e+03	34.1\\
3.46e+03	34.5\\
3.47e+03	34.5\\
3.47e+03	34.1\\
3.47e+03	32.8\\
3.47e+03	34.4\\
3.47e+03	36.1\\
3.48e+03	36.1\\
3.48e+03	36.1\\
3.48e+03	34.5\\
3.48e+03	34.4\\
3.48e+03	34.8\\
3.49e+03	38.3\\
3.49e+03	39.7\\
3.49e+03	39.9\\
3.49e+03	38.5\\
3.5e+03	38.4\\
3.5e+03	38.8\\
3.5e+03	38.3\\
3.5e+03	37.4\\
3.5e+03	37.1\\
3.50e+03	36.1\\
3.51e+03	35.5\\
3.51e+03	36.1\\
3.51e+03	38.3\\
3.51e+03	39.7\\
3.52e+03	39.7\\
3.52e+03	38.7\\
3.52e+03	39.6\\
3.52e+03	39.7\\
3.52e+03	38.7\\
3.52e+03	38.3\\
3.53e+03	37.4\\
3.53e+03	36.1\\
3.53e+03	35.8\\
3.53e+03	36.1\\
3.54e+03	38.4\\
3.54e+03	39.7\\
3.54e+03	39.7\\
3.54e+03	39.7\\
3.54e+03	38.9\\
3.54e+03	39.1\\
3.55e+03	38.7\\
3.55e+03	38.3\\
3.55e+03	36.5\\
3.55e+03	36.1\\
3.56e+03	34.9\\
3.56e+03	34.5\\
3.56e+03	34.9\\
3.56e+03	36.1\\
3.56e+03	35.6\\
3.57e+03	34.5\\
3.57e+03	34.8\\
3.57e+03	36.1\\
3.57e+03	36.1\\
3.58e+03	36.1\\
3.58e+03	34.5\\
3.58e+03	34.4\\
3.58e+03	34.5\\
3.58e+03	36.1\\
3.58e+03	36.1\\
3.59e+03	35.2\\
3.59e+03	34.8\\
3.59e+03	34.8\\
3.59e+03	36.1\\
3.6e+03	36.1\\
3.6e+03	36.1\\
3.6e+03	34.5\\
3.6e+03	34.5\\
3.61e+03	34.5\\
3.61e+03	35.7\\
3.61e+03	35.2\\
3.61e+03	35.1\\
3.61e+03	34.6\\
3.62e+03	34.9\\
3.62e+03	36.1\\
3.62e+03	36.1\\
3.62e+03	36.1\\
3.62e+03	33.3\\
3.63e+03	32.7\\
3.63e+03	32.7\\
3.63e+03	32.7\\
3.63e+03	32.7\\
3.63e+03	33.1\\
3.64e+03	32.7\\
3.64e+03	32.7\\
3.64e+03	32.7\\
3.64e+03	34.3\\
3.64e+03	34.6\\
3.65e+03	34.6\\
3.65e+03	33\\
3.65e+03	32.7\\
3.65e+03	33.6\\
3.66e+03	36.1\\
3.66e+03	36.4\\
3.66e+03	35.9\\
3.66e+03	35.8\\
3.66e+03	35.9\\
3.66e+03	36.7\\
3.67e+03	36.2\\
3.67e+03	35.9\\
3.67e+03	35.9\\
3.67e+03	34.4\\
3.68e+03	33.6\\
3.68e+03	33.8\\
3.68e+03	36.6\\
3.68e+03	37.9\\
3.68e+03	36.8\\
3.68e+03	35\\
3.69e+03	35.9\\
3.69e+03	36.1\\
3.69e+03	37.6\\
3.69e+03	35.9\\
3.7e+03	35.9\\
3.7e+03	34.6\\
3.7e+03	34.4\\
3.7e+03	34.6\\
3.7e+03	38.3\\
3.7e+03	40.9\\
3.71e+03	41.8\\
3.71e+03	41.8\\
3.71e+03	39.9\\
3.71e+03	39.4\\
3.71e+03	40.7\\
3.72e+03	37.4\\
3.72e+03	35.6\\
3.72e+03	33.4\\
3.72e+03	32.7\\
3.72e+03	32.7\\
3.73e+03	34.8\\
3.73e+03	35.6\\
3.73e+03	34.6\\
3.73e+03	34.4\\
3.74e+03	33.8\\
3.74e+03	34.6\\
3.74e+03	35.9\\
3.74e+03	35\\
3.74e+03	35.6\\
3.74e+03	34.2\\
3.75e+03	33.6\\
3.75e+03	34.4\\
3.75e+03	37.3\\
3.75e+03	36.6\\
3.76e+03	35.9\\
3.76e+03	35\\
3.76e+03	34.6\\
3.76e+03	35.6\\
3.76e+03	35.9\\
3.76e+03	34.6\\
3.77e+03	34\\
3.77e+03	32.7\\
3.77e+03	32.7\\
3.77e+03	32.3\\
3.78e+03	33.2\\
3.78e+03	33.9\\
3.78e+03	33.8\\
3.78e+03	32.7\\
3.78e+03	33.4\\
3.79e+03	34.4\\
3.79e+03	34.6\\
3.79e+03	34.6\\
3.79e+03	33.8\\
3.79e+03	32.7\\
3.8e+03	32.7\\
3.8e+03	32.7\\
3.8e+03	33.4\\
3.8e+03	33.4\\
3.80e+03	32.7\\
3.81e+03	32\\
3.81e+03	32.7\\
3.81e+03	33.6\\
3.81e+03	33.9\\
3.82e+03	33.8\\
3.82e+03	32.7\\
3.82e+03	32.2\\
3.82e+03	31.5\\
3.82e+03	32.5\\
3.82e+03	32.7\\
3.83e+03	32.7\\
3.83e+03	32.7\\
3.83e+03	32.7\\
3.83e+03	33.8\\
3.84e+03	34.6\\
3.84e+03	34.6\\
3.84e+03	34.4\\
3.84e+03	33.6\\
3.84e+03	33\\
3.85e+03	34.7\\
3.85e+03	36.5\\
3.85e+03	39.4\\
3.85e+03	39.5\\
3.85e+03	38.9\\
3.85e+03	38.7\\
3.86e+03	37.9\\
3.86e+03	39.9\\
3.86e+03	38.4\\
3.86e+03	38.7\\
3.86e+03	35.5\\
3.87e+03	34.6\\
3.87e+03	34.7\\
3.87e+03	37.2\\
3.87e+03	39.9\\
3.87e+03	40.9\\
3.88e+03	39.3\\
3.88e+03	38.6\\
3.88e+03	37.7\\
3.88e+03	39.9\\
3.88e+03	40\\
3.88e+03	38.8\\
3.89e+03	38.7\\
3.89e+03	35.9\\
3.89e+03	34.6\\
3.89e+03	34.9\\
3.9e+03	40\\
3.9e+03	39.5\\
3.9e+03	38.5\\
3.9e+03	36.7\\
3.9e+03	36.5\\
3.9e+03	38.1\\
3.91e+03	40.2\\
3.91e+03	39.8\\
3.91e+03	40.2\\
3.91e+03	37.2\\
3.91e+03	35.5\\
3.92e+03	34.6\\
3.92e+03	36.6\\
3.92e+03	40.5\\
3.92e+03	40.3\\
3.92e+03	39.7\\
3.92e+03	38.4\\
3.93e+03	36.8\\
3.93e+03	38.1\\
3.93e+03	38.9\\
3.93e+03	37.7\\
3.94e+03	37.6\\
3.94e+03	35\\
3.94e+03	35\\
3.94e+03	34.6\\
3.94e+03	34.9\\
3.94e+03	35.9\\
3.95e+03	35.5\\
3.95e+03	34.6\\
3.95e+03	33.8\\
3.95e+03	34.6\\
3.96e+03	35.6\\
3.96e+03	35\\
3.96e+03	35.7\\
3.96e+03	34.6\\
3.96e+03	33.8\\
3.96e+03	33.4\\
3.97e+03	33.8\\
3.97e+03	34.3\\
3.97e+03	34.3\\
3.97e+03	33.4\\
3.98e+03	32.7\\
3.98e+03	33.4\\
3.98e+03	34.6\\
3.98e+03	34.8\\
3.98e+03	34.8\\
3.98e+03	33.4\\
3.99e+03	33.1\\
3.99e+03	34.1\\
3.99e+03	36.9\\
3.99e+03	37.4\\
4e+03	37.3\\
4e+03	36.6\\
4e+03	36.1\\
4e+03	36.2\\
4e+03	36.2\\
4.00e+03	35.9\\
4.01e+03	35.9\\
4.01e+03	34.6\\
4.01e+03	34.4\\
4.01e+03	34.6\\
4.02e+03	37.2\\
4.02e+03	39.5\\
4.02e+03	37.9\\
4.02e+03	36\\
4.02e+03	35\\
4.02e+03	35.3\\
4.03e+03	35\\
4.03e+03	35.7\\
4.03e+03	36.1\\
4.03e+03	34.6\\
4.03e+03	33.6\\
4.04e+03	33.8\\
4.04e+03	35\\
4.04e+03	36.8\\
4.04e+03	36.2\\
4.04e+03	36.2\\
4.05e+03	35.2\\
4.05e+03	35.4\\
4.05e+03	36.1\\
4.05e+03	35.9\\
4.05e+03	36.1\\
4.06e+03	34.6\\
4.06e+03	33.8\\
4.06e+03	33.8\\
4.06e+03	35.3\\
4.06e+03	38.2\\
4.07e+03	37.5\\
4.07e+03	35.9\\
4.07e+03	35.9\\
4.07e+03	36.1\\
4.07e+03	38.2\\
4.08e+03	35.9\\
4.08e+03	35.9\\
4.08e+03	34.6\\
4.08e+03	33.7\\
4.08e+03	33.6\\
4.09e+03	34.6\\
4.09e+03	36.1\\
4.09e+03	36.5\\
4.09e+03	35.9\\
4.09e+03	35.9\\
4.1e+03	36.1\\
4.1e+03	36.2\\
4.1e+03	36.2\\
4.1e+03	37\\
4.1e+03	35\\
4.11e+03	34.4\\
4.11e+03	33.6\\
4.11e+03	33.4\\
4.11e+03	35.5\\
4.11e+03	35.9\\
4.12e+03	34.6\\
4.12e+03	34.4\\
4.12e+03	34.5\\
4.12e+03	34.6\\
4.12e+03	35\\
4.13e+03	35.5\\
4.13e+03	34.6\\
4.13e+03	33.4\\
4.13e+03	33.2\\
4.13e+03	32.7\\
4.14e+03	33.5\\
4.14e+03	33.8\\
4.14e+03	33.4\\
4.14e+03	32.7\\
4.14e+03	33\\
4.15e+03	34.4\\
4.15e+03	34.6\\
4.15e+03	35.9\\
4.15e+03	34.6\\
4.15e+03	33.4\\
4.16e+03	33.8\\
4.16e+03	37.3\\
4.16e+03	38.4\\
4.16e+03	36.9\\
4.16e+03	36.1\\
4.17e+03	36.1\\
4.17e+03	37.5\\
4.17e+03	38.3\\
4.17e+03	36.1\\
4.18e+03	37.6\\
4.18e+03	35.8\\
4.18e+03	34.6\\
4.18e+03	34.6\\
4.18e+03	35.5\\
4.18e+03	37.4\\
4.19e+03	38.6\\
4.19e+03	38.7\\
4.19e+03	38.7\\
4.19e+03	37.6\\
4.19e+03	36.1\\
4.2e+03	37.8\\
4.2e+03	38.7\\
4.2e+03	35.9\\
4.2e+03	33.9\\
4.2e+03	34.1\\
4.21e+03	35\\
4.21e+03	36.2\\
4.21e+03	36.1\\
4.21e+03	35.8\\
4.21e+03	35.9\\
4.22e+03	36\\
4.22e+03	36.5\\
4.22e+03	36.3\\
4.22e+03	37.3\\
4.22e+03	35\\
4.23e+03	34.4\\
4.23e+03	34.6\\
4.23e+03	35.9\\
4.23e+03	36.6\\
4.23e+03	38\\
4.24e+03	36.5\\
4.24e+03	36.1\\
4.24e+03	36.7\\
4.24e+03	37.8\\
4.24e+03	36.1\\
4.25e+03	37.3\\
4.25e+03	35.6\\
4.25e+03	34.5\\
4.25e+03	34.3\\
4.25e+03	35.2\\
4.26e+03	36.5\\
4.26e+03	39.9\\
4.26e+03	38.9\\
4.26e+03	37.3\\
4.26e+03	35.9\\
4.27e+03	35.9\\
4.27e+03	35.3\\
4.27e+03	35.8\\
4.27e+03	34.2\\
4.27e+03	32.7\\
4.28e+03	32.7\\
4.28e+03	32.7\\
4.28e+03	34.4\\
4.28e+03	34.6\\
4.28e+03	34.6\\
4.29e+03	34.6\\
4.29e+03	34.6\\
4.29e+03	34.5\\
4.29e+03	34.6\\
4.3e+03	33.7\\
4.3e+03	32.7\\
4.3e+03	32.7\\
4.3e+03	32.7\\
4.31e+03	32.7\\
4.31e+03	34\\
4.31e+03	33.3\\
4.31e+03	32.7\\
4.31e+03	33.8\\
4.32e+03	33.8\\
4.32e+03	34.2\\
4.32e+03	33.6\\
4.32e+03	32.7\\
4.32e+03	33\\
4.33e+03	35.1\\
4.33e+03	35.9\\
4.33e+03	36.1\\
4.33e+03	35.7\\
4.34e+03	35\\
4.34e+03	35.9\\
4.34e+03	36.5\\
4.34e+03	36.1\\
4.34e+03	34\\
4.35e+03	32.9\\
4.35e+03	32.9\\
4.35e+03	34.1\\
4.35e+03	36.9\\
4.35e+03	36.7\\
4.36e+03	34.5\\
4.36e+03	34.1\\
4.36e+03	34\\
4.36e+03	34.5\\
4.36e+03	34.4\\
4.36e+03	33.3\\
4.37e+03	33.3\\
4.37e+03	32.7\\
4.37e+03	31.3\\
4.37e+03	32\\
4.38e+03	34\\
4.38e+03	34.1\\
4.38e+03	33.9\\
4.38e+03	33\\
4.38e+03	32.9\\
4.38e+03	33.8\\
4.39e+03	34.1\\
4.39e+03	34.1\\
4.39e+03	34\\
4.39e+03	32.7\\
4.4e+03	31.4\\
4.4e+03	31.8\\
4.4e+03	33.4\\
4.4e+03	34\\
4.4e+03	33.4\\
4.40e+03	32.8\\
4.41e+03	32.8\\
4.41e+03	33.6\\
4.41e+03	33.4\\
4.41e+03	33.4\\
4.42e+03	32.7\\
4.42e+03	31.2\\
4.42e+03	31.3\\
4.42e+03	32.8\\
4.42e+03	34.1\\
4.43e+03	34.4\\
4.43e+03	34.1\\
4.43e+03	34\\
4.43e+03	34.1\\
4.44e+03	33.9\\
4.44e+03	33.4\\
4.44e+03	32.2\\
4.44e+03	30.9\\
4.44e+03	30.9\\
4.45e+03	30.8\\
4.45e+03	30.9\\
4.45e+03	30.9\\
4.45e+03	30.9\\
4.45e+03	30.9\\
4.46e+03	30.9\\
4.46e+03	30.9\\
4.46e+03	31.7\\
4.46e+03	30.9\\
4.47e+03	30.8\\
4.47e+03	29.5\\
4.47e+03	29.8\\
4.47e+03	30.9\\
4.48e+03	30.9\\
4.48e+03	30.9\\
4.48e+03	30.9\\
4.48e+03	31.8\\
4.48e+03	31.7\\
4.49e+03	31.6\\
4.49e+03	30.9\\
4.49e+03	30.4\\
4.49e+03	30.8\\
4.49e+03	31.3\\
4.5e+03	33.3\\
4.5e+03	33.3\\
4.5e+03	32.9\\
4.5e+03	33.4\\
4.5e+03	34\\
4.51e+03	34.1\\
4.51e+03	34.2\\
4.51e+03	34.4\\
4.51e+03	32.8\\
4.51e+03	31.8\\
4.52e+03	31.3\\
4.52e+03	32.8\\
4.52e+03	35.9\\
4.52e+03	37.9\\
4.52e+03	37.7\\
4.53e+03	37.5\\
4.53e+03	36.3\\
4.53e+03	36.3\\
4.53e+03	34.5\\
4.53e+03	34.5\\
4.54e+03	32.8\\
4.54e+03	31.4\\
4.54e+03	30.9\\
4.54e+03	32\\
4.54e+03	34\\
4.55e+03	34.4\\
4.55e+03	35.2\\
4.55e+03	34.7\\
4.55e+03	34.4\\
4.55e+03	34.1\\
4.56e+03	32.9\\
4.56e+03	33.1\\
4.56e+03	31.3\\
4.56e+03	30.9\\
4.56e+03	30.9\\
4.57e+03	32.8\\
4.57e+03	34.1\\
4.57e+03	34\\
4.57e+03	34.1\\
4.58e+03	34\\
4.58e+03	34\\
4.58e+03	34\\
4.58e+03	33.5\\
4.58e+03	33.2\\
4.58e+03	31.5\\
4.59e+03	30.9\\
4.59e+03	31.6\\
4.59e+03	33.4\\
4.59e+03	33.3\\
4.6e+03	34.4\\
4.6e+03	34\\
4.6e+03	33.4\\
4.6e+03	34\\
4.6e+03	33.4\\
4.60e+03	33.4\\
4.61e+03	34\\
4.61e+03	32.8\\
4.61e+03	31.9\\
4.61e+03	31.2\\
4.62e+03	32\\
4.62e+03	33.2\\
4.62e+03	33.4\\
4.62e+03	32.8\\
4.62e+03	32.4\\
4.62e+03	32.8\\
4.63e+03	33.3\\
4.63e+03	33.1\\
4.63e+03	32.8\\
4.63e+03	31.6\\
4.64e+03	30.9\\
4.64e+03	30.9\\
4.64e+03	30.9\\
4.64e+03	31.3\\
4.64e+03	32.4\\
4.64e+03	31.3\\
4.65e+03	30.9\\
4.65e+03	31\\
4.65e+03	32.8\\
4.65e+03	32.8\\
4.66e+03	32\\
4.66e+03	30.9\\
4.66e+03	30.9\\
4.66e+03	31.8\\
4.66e+03	33.9\\
4.67e+03	34\\
4.67e+03	34.2\\
4.67e+03	33.7\\
4.67e+03	33.5\\
4.67e+03	34.1\\
4.68e+03	34\\
4.68e+03	34\\
4.68e+03	32.8\\
4.68e+03	30.9\\
4.68e+03	30.9\\
4.69e+03	32.5\\
4.69e+03	33.2\\
4.69e+03	34.1\\
4.69e+03	34.5\\
4.69e+03	34.2\\
4.7e+03	34.3\\
4.7e+03	35\\
4.7e+03	34.1\\
4.7e+03	34.5\\
4.7e+03	34.1\\
4.71e+03	32.9\\
4.71e+03	32.8\\
4.71e+03	34.1\\
4.71e+03	36.3\\
4.71e+03	38\\
4.72e+03	36.9\\
4.72e+03	35\\
4.72e+03	34.1\\
4.72e+03	34.4\\
4.72e+03	36.3\\
4.72e+03	34.1\\
4.73e+03	35.2\\
4.73e+03	33.4\\
4.73e+03	32.8\\
4.73e+03	32.6\\
4.73e+03	33.3\\
4.74e+03	35.2\\
4.74e+03	34.4\\
4.74e+03	34.1\\
4.74e+03	34\\
4.74e+03	34\\
4.75e+03	34.8\\
4.75e+03	34\\
4.75e+03	34.4\\
4.75e+03	32.8\\
4.75e+03	31.3\\
4.76e+03	31.1\\
4.76e+03	32.8\\
4.76e+03	34\\
4.76e+03	34\\
4.76e+03	33.2\\
4.77e+03	32.8\\
4.77e+03	32.8\\
4.77e+03	33.8\\
4.77e+03	34\\
4.77e+03	34.1\\
4.78e+03	32.8\\
4.78e+03	30.9\\
4.78e+03	30.9\\
4.78e+03	30.1\\
4.78e+03	30.4\\
4.78e+03	30.9\\
4.79e+03	29.9\\
4.79e+03	29.8\\
4.79e+03	29.9\\
4.79e+03	30.9\\
4.8e+03	32.3\\
4.8e+03	32\\
4.8e+03	32\\
4.8e+03	30.9\\
4.8e+03	30.9\\
4.80e+03	30.9\\
4.81e+03	30.9\\
4.81e+03	30.9\\
4.81e+03	30.9\\
4.81e+03	30.9\\
4.82e+03	30.8\\
4.82e+03	30.9\\
4.82e+03	31\\
4.82e+03	31.5\\
4.82e+03	30.9\\
4.83e+03	30.6\\
4.83e+03	29.8\\
4.83e+03	29.8\\
4.83e+03	30.2\\
4.83e+03	30.9\\
4.84e+03	30.9\\
4.84e+03	30.9\\
4.84e+03	30.9\\
4.84e+03	30.9\\
4.85e+03	30.9\\
4.85e+03	30.7\\
4.85e+03	29.5\\
4.85e+03	29.8\\
4.86e+03	30.9\\
4.86e+03	32.8\\
4.86e+03	32.7\\
4.86e+03	31.3\\
4.86e+03	30.9\\
4.86e+03	31.7\\
4.87e+03	32\\
4.87e+03	32\\
4.87e+03	32\\
4.87e+03	30.9\\
4.88e+03	30.9\\
4.88e+03	30.9\\
4.88e+03	31.9\\
4.88e+03	32.6\\
4.88e+03	32\\
4.88e+03	30.9\\
4.89e+03	31\\
4.89e+03	31.9\\
4.89e+03	32\\
4.89e+03	32.8\\
4.9e+03	30.9\\
4.9e+03	30.9\\
4.9e+03	30.9\\
4.9e+03	32.1\\
4.90e+03	32\\
4.91e+03	31.3\\
4.91e+03	30.9\\
4.91e+03	30.9\\
4.91e+03	31.8\\
4.92e+03	32\\
4.92e+03	32.4\\
4.92e+03	32.5\\
4.92e+03	30.9\\
4.92e+03	30.9\\
4.92e+03	30.9\\
4.93e+03	31.9\\
4.93e+03	32.8\\
4.93e+03	32.5\\
4.93e+03	32\\
4.94e+03	32\\
4.94e+03	32\\
4.94e+03	32.1\\
4.94e+03	32.8\\
4.94e+03	31.5\\
4.95e+03	30.9\\
4.95e+03	30.9\\
4.95e+03	30.9\\
4.95e+03	31.5\\
4.96e+03	31.1\\
4.96e+03	30.9\\
4.96e+03	30.9\\
4.96e+03	32\\
4.96e+03	32\\
4.97e+03	31.9\\
4.97e+03	30.9\\
4.97e+03	30.9\\
4.97e+03	30.8\\
4.98e+03	30.9\\
4.98e+03	30.9\\
4.98e+03	30.6\\
4.98e+03	29.8\\
4.98e+03	29.8\\
4.98e+03	30.9\\
4.99e+03	31.3\\
4.99e+03	32\\
4.99e+03	31.4\\
4.99e+03	30.9\\
5e+03	30.9\\
5e+03	30.9\\
5e+03	32.8\\
5e+03	33\\
5e+03	32.9\\
5.01e+03	32.8\\
5.01e+03	32.8\\
5.01e+03	32.6\\
5.01e+03	32\\
5.02e+03	31.8\\
5.02e+03	30.9\\
5.02e+03	30.8\\
5.02e+03	30.9\\
5.02e+03	31.4\\
5.02e+03	32\\
5.03e+03	32\\
5.03e+03	31.9\\
5.03e+03	31.1\\
5.03e+03	32\\
5.04e+03	32\\
5.04e+03	32.3\\
5.04e+03	32.5\\
5.04e+03	30.9\\
5.04e+03	30.9\\
5.04e+03	30.9\\
5.05e+03	32.4\\
5.05e+03	32.8\\
5.05e+03	32.1\\
5.05e+03	31.4\\
5.06e+03	31.8\\
5.06e+03	32.8\\
5.06e+03	32.8\\
5.06e+03	32.9\\
5.06e+03	31.3\\
5.07e+03	30.9\\
5.07e+03	30.9\\
5.07e+03	32.1\\
5.07e+03	32.8\\
5.08e+03	32.4\\
5.08e+03	32\\
5.08e+03	32\\
5.08e+03	32.4\\
5.08e+03	32.8\\
5.09e+03	33\\
5.09e+03	32.8\\
5.09e+03	32.6\\
5.09e+03	32.5\\
5.09e+03	32.8\\
5.1e+03	33.9\\
5.1e+03	33.2\\
5.1e+03	32.8\\
5.1e+03	32.8\\
5.10e+03	33.2\\
5.11e+03	34\\
5.11e+03	34\\
5.11e+03	33.8\\
5.11e+03	32.8\\
5.12e+03	32.1\\
5.12e+03	31.7\\
5.12e+03	32.2\\
5.12e+03	32.8\\
5.12e+03	32.8\\
5.13e+03	32.8\\
5.13e+03	32.9\\
5.13e+03	33.4\\
5.13e+03	33.6\\
5.14e+03	33.2\\
5.14e+03	32.8\\
5.14e+03	31.7\\
5.14e+03	31.7\\
5.14e+03	31.8\\
5.14e+03	31.7\\
5.15e+03	31.6\\
5.15e+03	31.1\\
5.15e+03	30.2\\
5.15e+03	31.5\\
5.16e+03	32.8\\
5.16e+03	32.8\\
5.16e+03	32.8\\
5.16e+03	31.6\\
5.16e+03	31.7\\
5.17e+03	32.8\\
5.17e+03	34.8\\
5.17e+03	34.8\\
5.17e+03	34.4\\
5.17e+03	33.4\\
5.18e+03	33.4\\
5.18e+03	34\\
5.18e+03	34.4\\
5.18e+03	34.8\\
5.18e+03	32.8\\
5.19e+03	32.8\\
5.19e+03	32.8\\
5.19e+03	32.8\\
5.19e+03	34.4\\
5.19e+03	34\\
5.2e+03	34.8\\
5.2e+03	34.8\\
5.2e+03	34.8\\
5.2e+03	36.1\\
5.2e+03	36.3\\
5.21e+03	37.2\\
5.21e+03	34.8\\
5.21e+03	33.4\\
5.21e+03	32.8\\
5.21e+03	34\\
5.22e+03	35.2\\
5.22e+03	35.2\\
5.22e+03	34.8\\
5.22e+03	34.8\\
5.22e+03	35.2\\
5.23e+03	36.1\\
5.23e+03	35.1\\
5.23e+03	35.5\\
5.23e+03	34.4\\
5.23e+03	33.3\\
5.24e+03	33.4\\
5.24e+03	34.8\\
5.24e+03	36.1\\
5.24e+03	37.9\\
5.24e+03	36.6\\
5.24e+03	36.3\\
5.25e+03	36.1\\
5.25e+03	36.2\\
5.25e+03	36.3\\
5.25e+03	36.6\\
5.26e+03	34.8\\
5.26e+03	34\\
5.26e+03	33.8\\
5.26e+03	34.8\\
5.26e+03	37\\
5.27e+03	36.8\\
5.27e+03	36.3\\
5.27e+03	36.3\\
5.27e+03	36.8\\
5.27e+03	36.3\\
5.28e+03	36\\
5.28e+03	35.2\\
5.28e+03	33.9\\
5.28e+03	32.8\\
5.28e+03	32.8\\
5.29e+03	32.8\\
5.29e+03	33.2\\
5.29e+03	33.2\\
5.29e+03	32.8\\
5.3e+03	32.8\\
5.3e+03	33.5\\
5.3e+03	34\\
5.3e+03	34.7\\
5.3e+03	33.8\\
5.31e+03	32.8\\
5.31e+03	32.8\\
5.31e+03	32.8\\
5.31e+03	32.8\\
5.32e+03	33.2\\
5.32e+03	32.8\\
5.32e+03	32.6\\
5.32e+03	32.8\\
5.32e+03	32.8\\
5.32e+03	32.8\\
5.33e+03	32.8\\
5.33e+03	31.7\\
5.33e+03	31.3\\
5.33e+03	32.8\\
5.34e+03	33.5\\
5.34e+03	34\\
5.34e+03	33.8\\
5.34e+03	33.4\\
5.34e+03	33.4\\
5.34e+03	33.6\\
5.35e+03	34.4\\
5.35e+03	34.8\\
5.35e+03	32.8\\
5.35e+03	32.8\\
5.36e+03	32.8\\
5.36e+03	34.7\\
5.36e+03	34.8\\
5.36e+03	34.4\\
5.37e+03	34\\
5.37e+03	34.7\\
5.37e+03	35.2\\
5.37e+03	34.8\\
5.37e+03	35.4\\
5.38e+03	34\\
5.38e+03	32.8\\
5.38e+03	32.8\\
5.38e+03	34.5\\
5.38e+03	36.1\\
5.39e+03	36.3\\
5.39e+03	36.1\\
5.39e+03	35.4\\
5.39e+03	35.2\\
5.39e+03	35.2\\
5.4e+03	35.7\\
5.4e+03	36.1\\
5.4e+03	34.4\\
5.4e+03	33.4\\
5.40e+03	34.2\\
5.41e+03	36.2\\
5.41e+03	38.4\\
5.41e+03	39.4\\
5.41e+03	37.5\\
5.42e+03	36.8\\
5.42e+03	37\\
5.42e+03	37.1\\
5.42e+03	36.8\\
5.42e+03	36.1\\
5.42e+03	34.3\\
5.43e+03	33.5\\
5.43e+03	33.4\\
5.43e+03	33.4\\
5.43e+03	34\\
5.44e+03	33.6\\
5.44e+03	32.8\\
5.44e+03	32.8\\
5.44e+03	34\\
5.44e+03	34.8\\
5.45e+03	34.6\\
5.45e+03	32.8\\
5.45e+03	32.8\\
5.45e+03	32.8\\
5.46e+03	33.4\\
5.46e+03	34.4\\
5.46e+03	34\\
5.46e+03	32.9\\
5.46e+03	33.1\\
5.47e+03	34.4\\
5.47e+03	34\\
5.47e+03	34.3\\
5.47e+03	32.8\\
5.47e+03	32.6\\
5.48e+03	31.7\\
5.48e+03	31.8\\
5.48e+03	32.6\\
5.48e+03	32.8\\
5.48e+03	32.8\\
5.49e+03	32.5\\
5.49e+03	32.8\\
5.49e+03	33.3\\
5.49e+03	33.4\\
5.5e+03	33.1\\
5.5e+03	32.8\\
5.5e+03	32.4\\
5.5e+03	32.8\\
5.5e+03	34.8\\
5.50e+03	34.8\\
5.51e+03	34.8\\
5.51e+03	34.8\\
5.51e+03	35.2\\
5.51e+03	36.3\\
5.52e+03	37.7\\
5.52e+03	37.7\\
5.52e+03	36.7\\
5.52e+03	34.8\\
5.52e+03	34\\
5.52e+03	34.7\\
5.53e+03	36.3\\
5.53e+03	38.4\\
5.53e+03	37.2\\
5.53e+03	35.7\\
5.54e+03	36.1\\
5.54e+03	36.1\\
5.54e+03	38.2\\
5.54e+03	38.5\\
5.54e+03	36.5\\
5.54e+03	34.8\\
5.55e+03	34.8\\
5.55e+03	35\\
5.55e+03	39.4\\
5.55e+03	39.8\\
5.56e+03	38.7\\
5.56e+03	37\\
5.56e+03	37.2\\
5.56e+03	39.9\\
5.56e+03	39.7\\
5.56e+03	40.8\\
5.57e+03	38.2\\
5.57e+03	36.1\\
5.57e+03	34.8\\
5.57e+03	34.8\\
5.57e+03	36.4\\
5.58e+03	39.4\\
5.58e+03	38.1\\
5.58e+03	37.1\\
5.58e+03	36.8\\
5.58e+03	37.4\\
5.58e+03	37.3\\
5.59e+03	39.2\\
5.59e+03	37.8\\
5.59e+03	36.8\\
5.59e+03	34.8\\
5.59e+03	33.8\\
5.6e+03	33.4\\
5.6e+03	35\\
5.6e+03	37.8\\
5.6e+03	36.4\\
5.6e+03	36.3\\
5.61e+03	36.5\\
5.61e+03	36.4\\
5.61e+03	36.7\\
5.61e+03	38.6\\
5.61e+03	39.3\\
5.62e+03	37.2\\
5.62e+03	35.2\\
5.62e+03	36.1\\
5.62e+03	35\\
5.62e+03	35.2\\
5.62e+03	36.1\\
5.63e+03	35.6\\
5.63e+03	34.8\\
5.63e+03	34.8\\
5.63e+03	34.8\\
5.63e+03	36.1\\
5.64e+03	38.3\\
5.64e+03	37.2\\
5.64e+03	35.9\\
5.64e+03	33.6\\
5.64e+03	32.8\\
5.65e+03	32.8\\
5.65e+03	33.4\\
5.65e+03	33.4\\
5.65e+03	33.4\\
5.65e+03	32.8\\
5.66e+03	32.8\\
5.66e+03	34.8\\
5.66e+03	35.9\\
5.66e+03	36.1\\
5.66e+03	34.8\\
5.67e+03	33.7\\
5.67e+03	33.8\\
5.67e+03	36.1\\
5.67e+03	39.6\\
5.67e+03	40.9\\
5.68e+03	42\\
5.68e+03	42.1\\
5.68e+03	42.9\\
5.68e+03	42.1\\
5.68e+03	40.1\\
5.69e+03	38.2\\
5.69e+03	36.3\\
5.69e+03	34.8\\
5.69e+03	34.6\\
5.69e+03	34.8\\
5.69e+03	37.2\\
5.7e+03	40.1\\
5.7e+03	43.5\\
5.7e+03	43.5\\
5.7e+03	43.1\\
5.7e+03	43.4\\
5.70e+03	43.4\\
5.71e+03	42.1\\
5.71e+03	40.5\\
5.71e+03	38.1\\
5.71e+03	35.9\\
5.71e+03	34.7\\
5.72e+03	34.8\\
5.72e+03	37.3\\
5.72e+03	39.6\\
5.72e+03	41.7\\
5.72e+03	40.4\\
5.72e+03	39.3\\
5.73e+03	39.1\\
5.73e+03	39.8\\
5.73e+03	41.3\\
5.73e+03	40.4\\
5.73e+03	40.7\\
5.74e+03	38\\
5.74e+03	35.9\\
5.74e+03	34.6\\
5.74e+03	34.4\\
5.74e+03	36.3\\
5.74e+03	39.1\\
5.74e+03	41.1\\
5.75e+03	41.7\\
5.75e+03	40.5\\
5.75e+03	40.9\\
5.75e+03	40.1\\
5.76e+03	40\\
5.76e+03	39.1\\
5.76e+03	36.3\\
5.76e+03	34.5\\
5.76e+03	34\\
5.76e+03	34.8\\
5.77e+03	37.8\\
5.77e+03	38.7\\
5.77e+03	36.9\\
5.77e+03	36.1\\
5.78e+03	35\\
5.78e+03	35.5\\
5.78e+03	36.1\\
5.78e+03	36.1\\
5.78e+03	35\\
5.78e+03	33.4\\
5.79e+03	34.4\\
5.79e+03	34\\
5.79e+03	34.8\\
5.79e+03	36.3\\
5.8e+03	38.4\\
5.8e+03	37.5\\
5.8e+03	37.1\\
5.8e+03	38.8\\
5.8e+03	40.1\\
5.80e+03	38.9\\
5.81e+03	36.7\\
5.81e+03	34.8\\
5.81e+03	34.7\\
5.81e+03	34.7\\
5.82e+03	34.8\\
5.82e+03	36.1\\
5.82e+03	37.3\\
5.82e+03	36.1\\
5.82e+03	34.8\\
5.82e+03	35.4\\
5.83e+03	39.4\\
5.83e+03	39.3\\
5.83e+03	38.3\\
5.83e+03	40.5\\
5.83e+03	39.1\\
5.84e+03	39.1\\
5.84e+03	43\\
5.84e+03	45.2\\
5.84e+03	44.5\\
5.84e+03	43.5\\
5.84e+03	43.2\\
5.85e+03	43.8\\
5.85e+03	45.1\\
5.85e+03	46.7\\
5.85e+03	45.3\\
5.86e+03	43\\
5.86e+03	41.5\\
5.86e+03	40.5\\
5.86e+03	41.5\\
5.86e+03	47.8\\
5.86e+03	46.3\\
5.87e+03	44.3\\
5.87e+03	43.2\\
5.87e+03	43.2\\
5.87e+03	44.7\\
5.88e+03	45.4\\
5.88e+03	45.5\\
5.88e+03	43.1\\
5.88e+03	41.5\\
5.88e+03	40.5\\
5.88e+03	41.5\\
5.89e+03	47.8\\
5.89e+03	46.2\\
5.89e+03	43.8\\
5.89e+03	42.8\\
5.9e+03	42.3\\
5.9e+03	43.9\\
5.9e+03	44.2\\
5.9e+03	45.9\\
5.9e+03	42\\
5.90e+03	40.6\\
5.91e+03	40.5\\
5.91e+03	41.5\\
5.91e+03	48.9\\
5.91e+03	47.8\\
5.91e+03	45.9\\
5.92e+03	43.5\\
5.92e+03	43.2\\
5.92e+03	44.8\\
5.92e+03	51.8\\
5.92e+03	51.2\\
5.93e+03	46.5\\
5.93e+03	43.3\\
5.93e+03	41.5\\
5.93e+03	41.5\\
5.93e+03	41.5\\
5.93e+03	43.3\\
5.94e+03	51.8\\
5.94e+03	56.5\\
5.94e+03	52.6\\
5.94e+03	45.8\\
5.94e+03	44.8\\
5.94e+03	44.9\\
5.95e+03	46.3\\
5.95e+03	46.1\\
5.95e+03	43.9\\
5.95e+03	44.3\\
5.95e+03	41.6\\
5.96e+03	41.5\\
5.96e+03	42\\
5.96e+03	43.2\\
5.96e+03	46.8\\
5.96e+03	47.1\\
5.97e+03	45.3\\
5.97e+03	45\\
5.97e+03	47\\
5.97e+03	46.6\\
5.97e+03	46.8\\
5.98e+03	45.5\\
5.98e+03	43\\
5.98e+03	43\\
5.98e+03	43\\
5.98e+03	43.2\\
5.98e+03	44.1\\
5.99e+03	44.9\\
5.99e+03	43\\
5.99e+03	43\\
5.99e+03	41.6\\
6e+03	43.4\\
6e+03	44.6\\
6e+03	42.9\\
6e+03	41.3\\
6e+03	40.8\\
6.00e+03	41.9\\
6.01e+03	45.3\\
6.01e+03	51.8\\
6.01e+03	56.5\\
6.01e+03	51.8\\
6.01e+03	51.8\\
6.01e+03	47.8\\
6.01e+03	45.5\\
6.02e+03	46.6\\
6.02e+03	50.1\\
6.02e+03	51.8\\
6.02e+03	47.8\\
6.02e+03	45.9\\
6.02e+03	44.1\\
6.02e+03	41.6\\
6.03e+03	41.5\\
6.03e+03	42\\
6.03e+03	45.1\\
6.03e+03	48.9\\
6.03e+03	46.1\\
6.04e+03	47.8\\
6.04e+03	45.7\\
6.04e+03	45.5\\
6.04e+03	46.6\\
6.04e+03	50.1\\
6.04e+03	47.3\\
6.04e+03	46.6\\
6.05e+03	43.3\\
6.05e+03	41.5\\
6.05e+03	41.5\\
6.05e+03	41.7\\
6.05e+03	44.3\\
6.06e+03	51.8\\
6.06e+03	48.9\\
6.06e+03	46.4\\
6.06e+03	44.9\\
6.06e+03	43.7\\
6.06e+03	45.7\\
6.07e+03	50.1\\
6.07e+03	46.5\\
6.07e+03	48.9\\
6.07e+03	45.1\\
6.07e+03	43.2\\
6.07e+03	41.5\\
6.08e+03	41.3\\
6.08e+03	41.9\\
6.08e+03	45.1\\
6.08e+03	50.1\\
6.08e+03	51.8\\
6.08e+03	46.7\\
6.09e+03	45.7\\
6.09e+03	45.3\\
6.09e+03	47.8\\
6.09e+03	46.5\\
6.09e+03	46.6\\
6.1e+03	43.2\\
6.1e+03	41.5\\
6.1e+03	41\\
6.1e+03	41.5\\
6.1e+03	44.2\\
6.1e+03	46.7\\
6.1e+03	50.1\\
6.10e+03	46.7\\
6.11e+03	43.2\\
6.11e+03	43\\
6.11e+03	42.3\\
6.11e+03	43.7\\
6.12e+03	45.4\\
6.12e+03	45.1\\
6.12e+03	43.2\\
6.12e+03	41.5\\
6.12e+03	40.5\\
6.12e+03	40.4\\
6.13e+03	41.5\\
6.13e+03	43\\
6.13e+03	42\\
6.13e+03	41.5\\
6.14e+03	40.2\\
6.14e+03	41.5\\
6.14e+03	42\\
6.14e+03	41.7\\
6.14e+03	41.5\\
6.14e+03	40.3\\
6.15e+03	39.1\\
6.15e+03	39.3\\
6.15e+03	40.5\\
6.15e+03	42\\
6.16e+03	42.9\\
6.16e+03	41.5\\
6.16e+03	40.5\\
6.16e+03	41.5\\
6.16e+03	42\\
6.16e+03	42.1\\
6.17e+03	41.5\\
6.17e+03	39.9\\
6.17e+03	39.3\\
6.17e+03	41.4\\
6.17e+03	43.2\\
6.18e+03	46.9\\
6.18e+03	50.1\\
6.18e+03	51.2\\
6.18e+03	48.9\\
6.18e+03	45.4\\
6.18e+03	45.1\\
6.18e+03	46.1\\
6.19e+03	51.8\\
6.19e+03	51.8\\
6.19e+03	47.8\\
6.19e+03	45.7\\
6.19e+03	43.2\\
6.19e+03	41.5\\
6.2e+03	41.4\\
6.2e+03	41.9\\
6.2e+03	45.3\\
6.2e+03	51.8\\
6.2e+03	56.5\\
6.2e+03	51.8\\
6.2e+03	48.9\\
6.2e+03	45\\
6.21e+03	44.5\\
6.21e+03	46.6\\
6.21e+03	51.8\\
6.21e+03	51.8\\
6.21e+03	56.5\\
6.21e+03	51.8\\
6.21e+03	46.5\\
6.22e+03	43.9\\
6.22e+03	41.9\\
6.22e+03	41.3\\
6.22e+03	41.5\\
6.22e+03	44.6\\
6.22e+03	50.1\\
6.22e+03	46.7\\
6.23e+03	46.4\\
6.23e+03	44.2\\
6.23e+03	43.4\\
6.23e+03	44.3\\
6.23e+03	49.1\\
6.24e+03	51.8\\
6.24e+03	46.6\\
6.24e+03	44.8\\
6.24e+03	43\\
6.24e+03	41.5\\
6.24e+03	40.7\\
6.24e+03	41.5\\
6.25e+03	43.9\\
6.25e+03	51.8\\
6.25e+03	56.5\\
6.25e+03	56.5\\
6.25e+03	72.5\\
6.25e+03	56.5\\
6.25e+03	56.5\\
6.26e+03	51.8\\
6.26e+03	56.5\\
6.26e+03	51.8\\
6.26e+03	56.5\\
6.26e+03	49.1\\
6.26e+03	45.6\\
6.26e+03	43.4\\
6.26e+03	41.5\\
6.27e+03	41.5\\
6.27e+03	41.9\\
6.27e+03	45\\
6.27e+03	56.5\\
6.27e+03	56.5\\
6.28e+03	56.5\\
6.28e+03	51.8\\
6.28e+03	51.8\\
6.28e+03	51.8\\
6.28e+03	51.8\\
6.28e+03	47.8\\
6.28e+03	51.8\\
6.28e+03	46.3\\
6.29e+03	43.9\\
6.29e+03	42.1\\
6.29e+03	40.9\\
6.29e+03	40.3\\
6.29e+03	40.9\\
6.3e+03	42.1\\
6.3e+03	43\\
6.3e+03	42.5\\
6.3e+03	42\\
6.3e+03	42\\
6.31e+03	43\\
6.31e+03	43.5\\
6.31e+03	42.6\\
6.31e+03	41\\
6.31e+03	39.1\\
6.32e+03	39.1\\
6.32e+03	39.1\\
6.32e+03	40.8\\
6.32e+03	41.5\\
6.32e+03	39.1\\
6.33e+03	39.1\\
6.33e+03	39.1\\
6.33e+03	41\\
6.33e+03	41.5\\
6.33e+03	41.5\\
6.34e+03	40.5\\
6.34e+03	39.1\\
6.34e+03	39.1\\
6.34e+03	43\\
6.34e+03	45.6\\
6.34e+03	47.8\\
6.34e+03	50.1\\
6.35e+03	47.8\\
6.35e+03	44.9\\
6.35e+03	44.5\\
6.35e+03	45.3\\
6.35e+03	47.8\\
6.36e+03	50.1\\
6.36e+03	46.8\\
6.36e+03	43.2\\
6.36e+03	41.5\\
6.36e+03	41\\
6.36e+03	41.7\\
6.37e+03	45.2\\
6.37e+03	56.5\\
6.37e+03	51.8\\
6.37e+03	47.8\\
6.37e+03	45.3\\
6.37e+03	44.7\\
6.38e+03	46.7\\
6.38e+03	51.8\\
6.38e+03	56.5\\
6.38e+03	50.1\\
6.38e+03	46.4\\
6.38e+03	43.6\\
6.38e+03	41.5\\
6.39e+03	40.5\\
6.39e+03	41.5\\
6.39e+03	46.7\\
6.39e+03	48.9\\
6.39e+03	47\\
6.4e+03	51.8\\
6.4e+03	47.8\\
6.4e+03	46.7\\
6.4e+03	46.5\\
6.4e+03	50.1\\
6.4e+03	46.9\\
6.4e+03	47.8\\
6.40e+03	45\\
6.41e+03	42.8\\
6.41e+03	40.9\\
6.41e+03	40.5\\
6.41e+03	41.5\\
6.41e+03	43.9\\
6.42e+03	51.8\\
6.42e+03	49.1\\
6.42e+03	46.9\\
6.42e+03	44.4\\
6.42e+03	43.2\\
6.42e+03	43.5\\
6.43e+03	46.1\\
6.43e+03	48.3\\
6.43e+03	45.2\\
6.43e+03	43.2\\
6.43e+03	41.6\\
6.43e+03	40.6\\
6.44e+03	40.6\\
6.44e+03	43.6\\
6.44e+03	51.8\\
6.44e+03	51.8\\
6.44e+03	51.8\\
6.44e+03	50.1\\
6.45e+03	51.8\\
6.45e+03	56.5\\
6.45e+03	56.5\\
6.45e+03	56.5\\
6.45e+03	50.1\\
6.46e+03	45.5\\
6.46e+03	43.2\\
6.46e+03	42.7\\
6.46e+03	42\\
6.46e+03	43\\
6.46e+03	45.6\\
6.46e+03	47.7\\
6.47e+03	50.1\\
6.47e+03	47.8\\
6.47e+03	45.1\\
6.47e+03	43.2\\
6.47e+03	43.4\\
6.47e+03	45.4\\
6.48e+03	47.8\\
6.48e+03	50.1\\
6.48e+03	46.8\\
6.48e+03	43.6\\
6.48e+03	41.2\\
6.48e+03	39.8\\
6.48e+03	39.9\\
6.49e+03	41.4\\
6.49e+03	41.5\\
6.49e+03	41.3\\
6.49e+03	39.5\\
6.5e+03	40.3\\
6.5e+03	43.2\\
6.5e+03	47.8\\
6.5e+03	51.8\\
6.5e+03	44.3\\
6.5e+03	42\\
6.50e+03	40.5\\
6.51e+03	40.1\\
6.51e+03	41.5\\
6.51e+03	43.8\\
6.51e+03	51.8\\
6.51e+03	50.1\\
6.51e+03	47.8\\
6.52e+03	45.7\\
6.52e+03	46.2\\
6.52e+03	51.8\\
6.52e+03	56.5\\
6.52e+03	179\\
6.52e+03	224\\
6.52e+03	179\\
6.52e+03	57.5\\
6.53e+03	50.1\\
6.53e+03	45.1\\
6.53e+03	43\\
6.53e+03	41.5\\
6.53e+03	41.9\\
6.53e+03	51.8\\
6.54e+03	79.2\\
6.54e+03	224\\
6.54e+03	69.1\\
6.54e+03	57\\
6.54e+03	56.5\\
6.54e+03	51.8\\
6.54e+03	51.8\\
6.54e+03	57.5\\
6.55e+03	72.5\\
6.55e+03	193\\
6.55e+03	179\\
6.55e+03	56.5\\
6.55e+03	51.8\\
6.55e+03	46.1\\
6.55e+03	43.7\\
6.55e+03	42.7\\
6.56e+03	42.6\\
6.56e+03	46\\
6.56e+03	58.2\\
6.56e+03	56.5\\
6.56e+03	53.4\\
6.56e+03	50.5\\
6.56e+03	53.4\\
6.56e+03	46.1\\
6.57e+03	46.1\\
6.57e+03	49.2\\
6.57e+03	56.5\\
6.57e+03	58.2\\
6.57e+03	57\\
6.57e+03	53.4\\
6.58e+03	46\\
6.58e+03	44.3\\
6.58e+03	42.7\\
6.58e+03	43.7\\
6.58e+03	51.6\\
6.58e+03	113\\
6.58e+03	179\\
6.59e+03	73.9\\
6.59e+03	59.1\\
6.59e+03	59.1\\
6.59e+03	62.7\\
6.59e+03	70.6\\
6.59e+03	179\\
6.59e+03	164\\
6.6e+03	93.2\\
6.6e+03	70.6\\
6.6e+03	58.2\\
6.6e+03	56.5\\
6.6e+03	46.2\\
6.6e+03	44.3\\
6.6e+03	42.8\\
6.6e+03	42.7\\
6.61e+03	50.5\\
6.61e+03	199\\
6.61e+03	70.6\\
6.61e+03	58.2\\
6.61e+03	56.5\\
6.61e+03	49.2\\
6.61e+03	46.6\\
6.62e+03	46.5\\
6.62e+03	51.6\\
6.62e+03	56.5\\
6.62e+03	58.2\\
6.62e+03	56.5\\
6.62e+03	51.6\\
6.62e+03	45.7\\
6.62e+03	43.4\\
6.63e+03	41.7\\
6.63e+03	40.3\\
6.63e+03	41\\
6.63e+03	42.6\\
6.63e+03	42\\
6.64e+03	41.4\\
6.64e+03	40.7\\
6.64e+03	42.5\\
6.64e+03	43.7\\
6.64e+03	43.8\\
6.65e+03	42.7\\
6.65e+03	41.4\\
6.65e+03	40.3\\
6.65e+03	40.3\\
6.66e+03	41.9\\
6.66e+03	42.7\\
6.66e+03	44.2\\
6.66e+03	43.2\\
6.66e+03	42.7\\
6.66e+03	44.3\\
6.67e+03	46.1\\
6.67e+03	45.6\\
6.67e+03	44.7\\
6.67e+03	42.7\\
6.67e+03	41.5\\
6.68e+03	41.7\\
6.68e+03	45.7\\
6.68e+03	56.5\\
6.68e+03	53.7\\
6.68e+03	56.5\\
6.68e+03	53.4\\
6.68e+03	50.5\\
6.68e+03	48.1\\
6.68e+03	50.3\\
6.69e+03	47.5\\
6.69e+03	49.2\\
6.69e+03	47.1\\
6.69e+03	44.3\\
6.7e+03	42.7\\
6.7e+03	40.3\\
6.7e+03	40.3\\
6.7e+03	40.3\\
6.7e+03	43\\
6.7e+03	46.8\\
6.70e+03	44.7\\
6.71e+03	44.4\\
6.71e+03	43.2\\
6.71e+03	42.7\\
6.71e+03	44.3\\
6.72e+03	45.7\\
6.72e+03	44.5\\
6.72e+03	42.9\\
6.72e+03	41\\
6.72e+03	40.3\\
6.72e+03	41.8\\
6.73e+03	44.3\\
6.73e+03	46.6\\
6.73e+03	48.8\\
6.73e+03	49.2\\
6.73e+03	48.4\\
6.74e+03	45.8\\
6.74e+03	45.8\\
6.74e+03	45.9\\
6.74e+03	44.4\\
6.74e+03	42.8\\
6.74e+03	40.8\\
6.75e+03	40.3\\
6.75e+03	41.7\\
6.75e+03	43.9\\
6.75e+03	46.6\\
6.75e+03	46.3\\
6.76e+03	45.5\\
6.76e+03	43.6\\
6.76e+03	44.3\\
6.76e+03	44.6\\
6.76e+03	45.7\\
6.76e+03	45.5\\
6.77e+03	44.3\\
6.77e+03	41.7\\
6.77e+03	40.3\\
6.77e+03	41.7\\
6.77e+03	44.3\\
6.78e+03	48.5\\
6.78e+03	47.1\\
6.78e+03	47.7\\
6.78e+03	45.1\\
6.78e+03	44.8\\
6.78e+03	45.1\\
6.79e+03	48.4\\
6.79e+03	50.3\\
6.79e+03	52.7\\
6.79e+03	47.7\\
6.79e+03	45.7\\
6.79e+03	43.3\\
6.79e+03	41.7\\
6.8e+03	41.1\\
6.8e+03	42.7\\
6.8e+03	44.3\\
6.8e+03	44.5\\
6.8e+03	44.3\\
6.81e+03	43.2\\
6.81e+03	43.2\\
6.81e+03	44.3\\
6.81e+03	44.5\\
6.81e+03	43.1\\
6.82e+03	42.2\\
6.82e+03	40.4\\
6.82e+03	40.3\\
6.82e+03	41.7\\
6.82e+03	42.7\\
6.83e+03	43\\
6.83e+03	42.7\\
6.83e+03	41.7\\
6.83e+03	41.9\\
6.83e+03	42.7\\
6.84e+03	42.7\\
6.84e+03	42.6\\
6.84e+03	40.8\\
6.84e+03	40.3\\
6.84e+03	40.3\\
6.85e+03	44.2\\
6.85e+03	45.3\\
6.85e+03	44.5\\
6.85e+03	44.3\\
6.85e+03	44.4\\
6.86e+03	44.5\\
6.86e+03	53.4\\
6.86e+03	45.9\\
6.86e+03	44.5\\
6.86e+03	42.4\\
6.87e+03	40.3\\
6.87e+03	41.7\\
6.87e+03	43.9\\
6.87e+03	53.4\\
6.87e+03	51.6\\
6.87e+03	45.7\\
6.88e+03	44.5\\
6.88e+03	44.3\\
6.88e+03	44.6\\
6.88e+03	53.4\\
6.88e+03	58.2\\
6.88e+03	53.4\\
6.89e+03	51.6\\
6.89e+03	44.5\\
6.89e+03	43.2\\
6.89e+03	42.7\\
6.89e+03	44.3\\
6.89e+03	58.2\\
6.9e+03	755\\
6.9e+03	69.1\\
6.9e+03	58.2\\
6.9e+03	58.2\\
6.9e+03	56.5\\
6.9e+03	56.5\\
6.9e+03	58.2\\
6.90e+03	74.6\\
6.91e+03	224\\
6.91e+03	605\\
6.91e+03	179\\
6.91e+03	58.2\\
6.91e+03	53.4\\
6.91e+03	45.6\\
6.91e+03	43.2\\
6.91e+03	41.7\\
6.92e+03	41.7\\
6.92e+03	44.5\\
6.92e+03	58.2\\
6.92e+03	58.2\\
6.92e+03	56.5\\
6.92e+03	53.4\\
6.93e+03	50.3\\
6.93e+03	51.6\\
6.93e+03	58.2\\
6.93e+03	88.1\\
6.93e+03	59.2\\
6.93e+03	56.5\\
6.93e+03	53.4\\
6.94e+03	45.8\\
6.94e+03	43.2\\
6.94e+03	41.8\\
6.94e+03	42.3\\
6.94e+03	45.3\\
6.94e+03	59.2\\
6.94e+03	62.7\\
6.94e+03	58.2\\
6.95e+03	53.4\\
6.95e+03	49.2\\
6.95e+03	45.8\\
6.95e+03	45.5\\
6.95e+03	50.7\\
6.95e+03	53.4\\
6.95e+03	50.3\\
6.96e+03	48.8\\
6.96e+03	45.8\\
6.96e+03	44.3\\
6.96e+03	41.5\\
6.96e+03	40.3\\
6.97e+03	40.3\\
6.97e+03	42.7\\
6.97e+03	42.8\\
6.97e+03	42.7\\
6.97e+03	41.7\\
6.98e+03	42.2\\
6.98e+03	44.5\\
6.98e+03	43.2\\
6.98e+03	42.7\\
6.98e+03	40.6\\
6.99e+03	40.3\\
6.99e+03	39.8\\
6.99e+03	40.3\\
6.99e+03	40.3\\
6.99e+03	40.7\\
7e+03	40.3\\
7e+03	40.3\\
7e+03	40.3\\
7e+03	41.8\\
7e+03	42.7\\
7.01e+03	41.7\\
7.01e+03	40.3\\
7.01e+03	40\\
7.01e+03	40.3\\
7.01e+03	42.7\\
7.02e+03	46.4\\
7.02e+03	47.2\\
7.02e+03	46.4\\
7.02e+03	45.1\\
7.02e+03	44.9\\
7.02e+03	47\\
7.03e+03	45\\
7.03e+03	42.7\\
7.03e+03	42.6\\
7.03e+03	40.3\\
7.04e+03	40.3\\
7.04e+03	40.3\\
7.04e+03	42.7\\
7.04e+03	44.5\\
7.04e+03	44.7\\
7.04e+03	46\\
7.04e+03	44.3\\
7.05e+03	43.6\\
7.05e+03	44.4\\
7.05e+03	44.1\\
7.05e+03	41.7\\
7.06e+03	40.3\\
7.06e+03	40.3\\
7.06e+03	39.1\\
7.06e+03	40.3\\
7.06e+03	42.6\\
7.06e+03	44.5\\
7.06e+03	44.1\\
7.07e+03	43.2\\
7.07e+03	42.7\\
7.07e+03	43\\
7.07e+03	44.3\\
7.07e+03	47.2\\
7.08e+03	46.4\\
7.08e+03	44.3\\
7.08e+03	42.4\\
7.08e+03	40.3\\
7.08e+03	40.3\\
7.08e+03	40.3\\
7.09e+03	42.7\\
7.09e+03	44.9\\
7.09e+03	44.3\\
7.09e+03	44.3\\
7.09e+03	44.1\\
7.1e+03	44.3\\
7.1e+03	44.4\\
7.1e+03	44.3\\
7.1e+03	42.7\\
7.1e+03	41.5\\
7.10e+03	40.3\\
7.11e+03	38.9\\
7.11e+03	40.3\\
7.11e+03	42.7\\
7.11e+03	44.3\\
7.12e+03	44.5\\
7.12e+03	43.4\\
7.12e+03	43.6\\
7.12e+03	44.9\\
7.12e+03	43.1\\
7.13e+03	42.7\\
7.13e+03	42.1\\
7.13e+03	40.3\\
7.13e+03	40.3\\
7.14e+03	42.7\\
7.14e+03	43.9\\
7.14e+03	44.3\\
7.14e+03	42.7\\
7.14e+03	41.7\\
7.14e+03	42.8\\
7.15e+03	43.2\\
7.15e+03	42.5\\
7.15e+03	42.3\\
7.15e+03	40.3\\
7.15e+03	39.7\\
7.16e+03	38.9\\
7.16e+03	38.9\\
7.16e+03	40.3\\
7.16e+03	40.3\\
7.17e+03	40.2\\
7.17e+03	40.3\\
7.17e+03	40.6\\
7.17e+03	42.6\\
7.17e+03	40.3\\
7.18e+03	40.3\\
7.18e+03	38.5\\
7.18e+03	38.5\\
7.18e+03	40\\
7.18e+03	42.7\\
7.19e+03	42.7\\
7.19e+03	41.7\\
7.19e+03	41.7\\
7.19e+03	43\\
7.2e+03	45.3\\
7.2e+03	43.7\\
7.2e+03	42.2\\
7.2e+03	40.3\\
7.2e+03	39.4\\
7.20e+03	39.5\\
7.21e+03	41.7\\
7.21e+03	42.7\\
7.21e+03	42.7\\
7.21e+03	41.7\\
7.22e+03	41.8\\
7.22e+03	43.4\\
7.22e+03	45.2\\
7.22e+03	43.4\\
7.22e+03	42.7\\
7.22e+03	40.3\\
7.23e+03	39.7\\
7.23e+03	39.9\\
7.23e+03	42.4\\
7.23e+03	44.5\\
7.24e+03	45.6\\
7.24e+03	47\\
7.24e+03	46.9\\
7.24e+03	47.4\\
7.24e+03	49.2\\
7.24e+03	46\\
7.25e+03	44.3\\
7.25e+03	43\\
7.25e+03	40.3\\
7.25e+03	40.3\\
7.26e+03	42.7\\
7.26e+03	44.5\\
7.26e+03	44.5\\
7.26e+03	44\\
7.26e+03	43.9\\
7.26e+03	44.3\\
7.27e+03	48\\
7.27e+03	45.5\\
7.27e+03	43.2\\
7.27e+03	42.7\\
7.27e+03	40.3\\
7.28e+03	39.4\\
7.28e+03	39.6\\
7.28e+03	41.8\\
7.28e+03	43.2\\
7.28e+03	42.7\\
7.28e+03	42.6\\
7.29e+03	41.7\\
7.29e+03	42.7\\
7.29e+03	43.4\\
7.29e+03	42\\
7.3e+03	40.3\\
7.3e+03	42.6\\
7.3e+03	40.6\\
7.3e+03	38.6\\
7.3e+03	38.5\\
7.3e+03	40.6\\
7.31e+03	40.6\\
7.31e+03	40.4\\
7.31e+03	40.6\\
7.31e+03	40.6\\
7.31e+03	42.6\\
7.32e+03	42.6\\
7.32e+03	42.6\\
7.32e+03	40.6\\
7.32e+03	37.9\\
7.32e+03	36.7\\
7.33e+03	37.9\\
7.33e+03	38.3\\
7.33e+03	37.9\\
7.33e+03	38.6\\
7.34e+03	38.7\\
7.34e+03	40.6\\
7.34e+03	42.2\\
7.34e+03	41\\
7.34e+03	41\\
7.34e+03	38.7\\
7.35e+03	36.6\\
7.35e+03	36.7\\
7.35e+03	38.6\\
7.35e+03	41.9\\
7.35e+03	42.9\\
7.36e+03	43.4\\
7.36e+03	43\\
7.36e+03	43.8\\
7.36e+03	44.7\\
7.36e+03	45.1\\
7.36e+03	44.1\\
7.37e+03	43.5\\
7.37e+03	42.6\\
7.37e+03	41.2\\
7.37e+03	42\\
7.38e+03	45\\
7.38e+03	45.7\\
7.38e+03	47.1\\
7.38e+03	47.2\\
7.38e+03	47.5\\
7.38e+03	47.3\\
7.39e+03	56.4\\
7.39e+03	48.1\\
7.39e+03	46.4\\
7.39e+03	45.3\\
7.39e+03	42.7\\
7.4e+03	42.6\\
7.4e+03	43.5\\
7.4e+03	45.8\\
7.4e+03	49.2\\
7.4e+03	48.4\\
7.4e+03	48\\
7.41e+03	48\\
7.41e+03	48.7\\
7.41e+03	52\\
7.41e+03	54.5\\
7.41e+03	56.4\\
7.41e+03	47.5\\
7.41e+03	46.7\\
7.42e+03	45.1\\
7.42e+03	42.6\\
7.42e+03	42.6\\
7.42e+03	45.1\\
7.42e+03	45.7\\
7.43e+03	46.5\\
7.43e+03	45.6\\
7.43e+03	45.1\\
7.43e+03	45.4\\
7.44e+03	46.2\\
7.44e+03	45\\
7.44e+03	43\\
7.44e+03	42.6\\
7.44e+03	40.6\\
7.44e+03	38.7\\
7.45e+03	40.6\\
7.45e+03	42.6\\
7.45e+03	44\\
7.45e+03	44.7\\
7.45e+03	45.1\\
7.46e+03	44.6\\
7.46e+03	43.8\\
7.46e+03	45.1\\
7.46e+03	42.9\\
7.46e+03	42.6\\
7.46e+03	42.6\\
7.47e+03	40.6\\
7.47e+03	39.8\\
7.47e+03	41.1\\
7.47e+03	42.6\\
7.48e+03	42.4\\
7.48e+03	42.5\\
7.48e+03	42.5\\
7.48e+03	42.6\\
7.48e+03	43.5\\
7.48e+03	42.6\\
7.49e+03	42.6\\
7.49e+03	40.6\\
7.49e+03	39.3\\
7.49e+03	39.8\\
7.5e+03	40.6\\
7.5e+03	41.6\\
7.5e+03	42.6\\
7.5e+03	42.6\\
7.50e+03	42.6\\
7.51e+03	44\\
7.51e+03	43\\
7.51e+03	42.6\\
7.51e+03	42.6\\
7.52e+03	41\\
7.52e+03	40.9\\
7.52e+03	42.6\\
7.52e+03	42.6\\
7.52e+03	42.6\\
7.52e+03	42.6\\
7.53e+03	42.6\\
7.53e+03	44.5\\
7.53e+03	43.8\\
7.53e+03	42.6\\
7.54e+03	42.6\\
7.54e+03	40.6\\
7.54e+03	40.5\\
7.54e+03	41\\
7.54e+03	40.6\\
7.55e+03	41.8\\
7.55e+03	42.6\\
7.55e+03	41.5\\
7.55e+03	42.6\\
7.56e+03	43\\
7.56e+03	42.7\\
7.56e+03	42.6\\
7.56e+03	41.9\\
7.56e+03	40.4\\
7.56e+03	40.6\\
7.57e+03	43.8\\
7.57e+03	46\\
7.57e+03	45.8\\
7.57e+03	46.3\\
7.57e+03	46.7\\
7.58e+03	46.7\\
7.58e+03	48.9\\
7.58e+03	52\\
7.58e+03	47.3\\
7.58e+03	45.5\\
7.58e+03	45.1\\
7.59e+03	42.6\\
7.59e+03	42.6\\
7.59e+03	42.8\\
7.59e+03	45.2\\
7.59e+03	46.7\\
7.6e+03	45.1\\
7.6e+03	45.1\\
7.6e+03	45.1\\
7.6e+03	45.8\\
7.6e+03	48.5\\
7.6e+03	46.9\\
7.61e+03	45.1\\
7.61e+03	44.9\\
7.61e+03	42.6\\
7.61e+03	41.6\\
7.61e+03	41.2\\
7.62e+03	44\\
7.62e+03	46\\
7.62e+03	46.7\\
7.62e+03	46.8\\
7.62e+03	46.9\\
7.62e+03	47.1\\
7.63e+03	47.9\\
7.63e+03	46.8\\
7.63e+03	45.5\\
7.63e+03	45.1\\
7.63e+03	42.6\\
7.64e+03	42.4\\
7.64e+03	41.4\\
7.64e+03	42.6\\
7.64e+03	43.8\\
7.64e+03	45.1\\
7.65e+03	45.1\\
7.65e+03	44.2\\
7.65e+03	44.4\\
7.65e+03	45.1\\
7.65e+03	42.6\\
7.66e+03	42.9\\
7.66e+03	42.1\\
7.66e+03	40.6\\
7.66e+03	40.6\\
7.66e+03	41.9\\
7.67e+03	42.6\\
7.67e+03	43.6\\
};
\addplot [color=mycolor1,solid,line width=0.5pt,forget plot]
  table[row sep=crcr]{%
7.67e+03	43.6\\
7.67e+03	43.9\\
7.67e+03	43.5\\
7.67e+03	44.7\\
7.68e+03	45.1\\
7.68e+03	44\\
7.68e+03	43\\
7.68e+03	42.2\\
7.68e+03	41\\
7.69e+03	42.6\\
7.69e+03	46.5\\
7.69e+03	47\\
7.69e+03	46.8\\
7.69e+03	46.7\\
7.7e+03	48.1\\
7.7e+03	54.5\\
7.7e+03	47.4\\
7.7e+03	45.1\\
7.70e+03	43.5\\
7.71e+03	42.6\\
7.71e+03	42.6\\
7.71e+03	45.1\\
7.71e+03	48.2\\
7.71e+03	47.1\\
7.72e+03	47\\
7.72e+03	47.4\\
7.72e+03	48.6\\
7.72e+03	52\\
7.72e+03	49.8\\
7.72e+03	47.4\\
7.73e+03	46.1\\
7.73e+03	43.9\\
7.73e+03	42.6\\
7.73e+03	42.6\\
7.74e+03	45.7\\
7.74e+03	49.7\\
7.74e+03	49.2\\
7.74e+03	49\\
7.74e+03	49.6\\
7.74e+03	49.4\\
7.75e+03	49.6\\
7.75e+03	47.1\\
7.75e+03	45.9\\
7.75e+03	45.1\\
7.75e+03	42.6\\
7.76e+03	42.6\\
7.76e+03	45.3\\
7.76e+03	48.3\\
7.76e+03	47\\
7.76e+03	46\\
7.77e+03	46.6\\
7.77e+03	46.7\\
7.77e+03	47.7\\
7.77e+03	46.7\\
7.77e+03	45.1\\
7.78e+03	43.8\\
7.78e+03	42.6\\
7.78e+03	42.6\\
7.78e+03	45.1\\
7.78e+03	46.9\\
7.79e+03	47.6\\
7.79e+03	47\\
7.79e+03	47.2\\
7.79e+03	46.9\\
7.79e+03	47.8\\
7.8e+03	46.7\\
7.8e+03	45.1\\
7.8e+03	45.1\\
7.8e+03	42.7\\
7.8e+03	41.8\\
7.80e+03	40.8\\
7.81e+03	41.6\\
7.81e+03	42.6\\
7.81e+03	42.6\\
7.81e+03	42.6\\
7.82e+03	43.5\\
7.82e+03	44.7\\
7.82e+03	43.8\\
7.82e+03	42.6\\
7.82e+03	42.6\\
7.83e+03	40.6\\
7.83e+03	40.4\\
7.83e+03	40.7\\
7.83e+03	41.6\\
7.84e+03	42.2\\
7.84e+03	42.6\\
7.84e+03	42.2\\
7.84e+03	42.6\\
7.84e+03	42.8\\
7.85e+03	42.6\\
7.85e+03	42.6\\
7.85e+03	40.6\\
7.85e+03	40.2\\
7.85e+03	41.1\\
7.86e+03	45.1\\
7.86e+03	45.3\\
7.86e+03	45.1\\
7.86e+03	44.6\\
7.86e+03	45.1\\
7.87e+03	46.7\\
7.87e+03	46.3\\
7.87e+03	44\\
7.87e+03	44\\
7.87e+03	42.6\\
7.88e+03	41.7\\
7.88e+03	42.6\\
7.88e+03	44.4\\
7.88e+03	46.7\\
7.88e+03	46.2\\
7.88e+03	45.5\\
7.89e+03	45.1\\
7.89e+03	45.2\\
7.89e+03	47\\
7.89e+03	46.6\\
7.89e+03	44.6\\
7.9e+03	44\\
7.9e+03	42.6\\
7.9e+03	41.9\\
7.9e+03	42.6\\
7.9e+03	44.8\\
7.9e+03	46.7\\
7.91e+03	47.7\\
7.91e+03	53.2\\
7.91e+03	54.5\\
7.91e+03	56.4\\
7.91e+03	56.4\\
7.92e+03	48.4\\
7.92e+03	45.6\\
7.92e+03	45.1\\
7.92e+03	42.6\\
7.92e+03	42.6\\
7.93e+03	45.6\\
7.93e+03	48.4\\
7.93e+03	52\\
7.93e+03	48.3\\
7.93e+03	49.4\\
7.93e+03	48.4\\
7.94e+03	48.2\\
7.94e+03	48.9\\
7.94e+03	48.5\\
7.94e+03	45.3\\
7.94e+03	45.1\\
7.94e+03	42.6\\
7.95e+03	42.1\\
7.95e+03	41.6\\
7.95e+03	44\\
7.95e+03	45.1\\
7.96e+03	45.2\\
7.96e+03	45.1\\
7.96e+03	45.3\\
7.96e+03	45.1\\
7.96e+03	44.4\\
7.97e+03	43.9\\
7.97e+03	42.6\\
7.97e+03	41.5\\
7.97e+03	41\\
7.98e+03	42.3\\
7.98e+03	42.6\\
7.98e+03	42.8\\
7.98e+03	42.6\\
7.98e+03	43.1\\
7.99e+03	44.8\\
7.99e+03	43.6\\
7.99e+03	42.6\\
7.99e+03	44\\
7.99e+03	42.6\\
8e+03	42.1\\
8e+03	41.9\\
8e+03	42.6\\
8e+03	43.3\\
8e+03	45.7\\
8.00e+03	46.4\\
8.01e+03	45.4\\
8.01e+03	47\\
8.01e+03	47\\
8.01e+03	47\\
8.02e+03	45.8\\
8.02e+03	44.5\\
8.02e+03	43.1\\
8.02e+03	43.1\\
8.02e+03	45.7\\
8.02e+03	48.1\\
8.02e+03	54\\
8.03e+03	57.1\\
8.03e+03	62.2\\
8.03e+03	62.2\\
8.03e+03	62.2\\
8.03e+03	63.3\\
8.04e+03	57.1\\
8.04e+03	48.6\\
8.04e+03	47.6\\
8.04e+03	45.8\\
8.04e+03	45.2\\
8.04e+03	43.1\\
8.05e+03	45.8\\
8.05e+03	48.3\\
8.05e+03	49.8\\
8.05e+03	49.5\\
8.05e+03	47.7\\
8.06e+03	49.1\\
8.06e+03	50\\
8.06e+03	48.4\\
8.06e+03	45.7\\
8.06e+03	44.8\\
8.07e+03	43.1\\
8.07e+03	43.1\\
8.07e+03	45.7\\
8.07e+03	47.8\\
8.07e+03	48.7\\
8.08e+03	48.6\\
8.08e+03	49.5\\
8.08e+03	55.2\\
8.08e+03	56.5\\
8.08e+03	53.8\\
8.08e+03	49.8\\
8.09e+03	47.4\\
8.09e+03	47\\
8.09e+03	45.7\\
8.09e+03	44.2\\
8.09e+03	45.2\\
8.1e+03	47.3\\
8.1e+03	57.1\\
8.1e+03	62.2\\
8.1e+03	62.2\\
8.1e+03	63.3\\
8.1e+03	62.2\\
8.10e+03	57.1\\
8.11e+03	62.2\\
8.11e+03	57.1\\
8.11e+03	48.3\\
8.11e+03	48.5\\
8.11e+03	45.8\\
8.12e+03	44.6\\
8.12e+03	44\\
8.12e+03	47.3\\
8.12e+03	52.6\\
8.12e+03	49.8\\
8.12e+03	55.2\\
8.12e+03	55.2\\
8.12e+03	52.6\\
8.13e+03	50.1\\
8.13e+03	49.1\\
8.13e+03	49.1\\
8.13e+03	47.6\\
8.13e+03	46.2\\
8.14e+03	46.5\\
8.14e+03	44.5\\
8.14e+03	43.1\\
8.14e+03	43.1\\
8.14e+03	44.2\\
8.15e+03	45.8\\
8.15e+03	45.8\\
8.15e+03	45.7\\
8.15e+03	46.2\\
8.15e+03	47.6\\
8.16e+03	47.7\\
8.16e+03	46.1\\
8.16e+03	47.3\\
8.16e+03	45\\
8.16e+03	43.1\\
8.17e+03	43.1\\
8.17e+03	44\\
8.17e+03	45.7\\
8.17e+03	44.5\\
8.18e+03	44.1\\
8.18e+03	45.6\\
8.18e+03	45.7\\
8.18e+03	45.6\\
8.18e+03	44.3\\
8.19e+03	43.1\\
8.19e+03	41.4\\
8.19e+03	42.9\\
8.19e+03	46.1\\
8.19e+03	47.5\\
8.2e+03	47.6\\
8.2e+03	48.6\\
8.2e+03	52.6\\
8.2e+03	55.2\\
8.2e+03	57.1\\
8.2e+03	54\\
8.20e+03	47.6\\
8.21e+03	46.3\\
8.21e+03	45.4\\
8.21e+03	43.3\\
8.21e+03	43.1\\
8.21e+03	43.8\\
8.22e+03	46.6\\
8.22e+03	49.2\\
8.22e+03	56.5\\
8.22e+03	49.4\\
8.22e+03	48.4\\
8.22e+03	49.3\\
8.22e+03	49.1\\
8.23e+03	49.8\\
8.23e+03	47.7\\
8.23e+03	46\\
8.23e+03	44.7\\
8.23e+03	43.1\\
8.24e+03	41.7\\
8.24e+03	41.1\\
8.24e+03	43.8\\
8.24e+03	45.7\\
8.24e+03	47.3\\
8.24e+03	46.5\\
8.25e+03	47.3\\
8.25e+03	48.1\\
8.25e+03	52.6\\
8.25e+03	56.5\\
8.25e+03	49.7\\
8.25e+03	47.6\\
8.25e+03	45.7\\
8.26e+03	43.1\\
8.26e+03	42.7\\
8.26e+03	42.3\\
8.26e+03	44\\
8.26e+03	46.5\\
8.27e+03	47.3\\
8.27e+03	47.6\\
8.27e+03	47.3\\
8.27e+03	47.6\\
8.27e+03	48\\
8.28e+03	47.3\\
8.28e+03	45.5\\
8.28e+03	45.4\\
8.28e+03	43.1\\
8.28e+03	41.9\\
8.28e+03	41.5\\
8.29e+03	44.2\\
8.29e+03	45.7\\
8.29e+03	46.3\\
8.29e+03	46.9\\
8.3e+03	47.3\\
8.3e+03	47.7\\
8.3e+03	49.1\\
8.3e+03	46.1\\
8.3e+03	45.8\\
8.30e+03	45.7\\
8.31e+03	43.8\\
8.31e+03	43.1\\
8.31e+03	43.1\\
8.31e+03	45.7\\
8.32e+03	46.7\\
8.32e+03	46.8\\
8.32e+03	46.2\\
8.32e+03	47.3\\
8.32e+03	47.7\\
8.32e+03	46.4\\
8.33e+03	45.7\\
8.33e+03	45.5\\
8.33e+03	43.8\\
8.33e+03	43.1\\
8.33e+03	43.1\\
8.34e+03	43.1\\
8.34e+03	43.4\\
8.34e+03	44.4\\
8.34e+03	43.1\\
8.34e+03	43.1\\
8.35e+03	44.4\\
8.35e+03	43.1\\
8.35e+03	43.1\\
8.35e+03	42.5\\
8.36e+03	40.3\\
8.36e+03	38.4\\
8.36e+03	41.1\\
8.36e+03	43.1\\
8.36e+03	45.4\\
8.36e+03	45.7\\
8.37e+03	45.7\\
8.37e+03	47.3\\
8.37e+03	45.7\\
8.37e+03	43.7\\
8.38e+03	43.2\\
8.38e+03	43.1\\
8.38e+03	41.6\\
8.38e+03	41.3\\
8.38e+03	45.7\\
8.38e+03	47.5\\
8.39e+03	49.2\\
8.39e+03	48.4\\
8.39e+03	47.8\\
8.39e+03	48.9\\
8.39e+03	57.1\\
8.4e+03	49.4\\
8.4e+03	47.3\\
8.4e+03	46.1\\
8.4e+03	44.2\\
8.4e+03	43\\
8.40e+03	41.5\\
8.41e+03	43.1\\
8.41e+03	45.7\\
8.41e+03	47.8\\
8.41e+03	49.4\\
8.41e+03	49\\
8.42e+03	49.7\\
8.42e+03	56.5\\
8.42e+03	55.2\\
8.42e+03	47.6\\
8.42e+03	45.7\\
8.42e+03	45.7\\
8.42e+03	43.5\\
8.43e+03	43.1\\
8.43e+03	43.1\\
8.43e+03	45.7\\
8.43e+03	47.6\\
8.43e+03	45.7\\
8.44e+03	47.3\\
8.44e+03	46.4\\
8.44e+03	45.8\\
8.44e+03	45.7\\
8.44e+03	45.7\\
8.45e+03	43.1\\
8.45e+03	43.1\\
8.45e+03	41.2\\
8.45e+03	39.4\\
8.45e+03	41.1\\
8.46e+03	43.1\\
8.46e+03	45.7\\
8.46e+03	45.9\\
8.46e+03	47.3\\
8.46e+03	45.7\\
8.47e+03	45.7\\
8.47e+03	45.2\\
8.47e+03	43.5\\
8.47e+03	43.1\\
8.47e+03	41.1\\
8.48e+03	38.1\\
8.48e+03	37.2\\
8.48e+03	37.8\\
8.48e+03	40.9\\
8.48e+03	41.1\\
8.48e+03	41.1\\
8.49e+03	41.1\\
8.49e+03	41.4\\
8.49e+03	42.4\\
8.49e+03	41.6\\
8.5e+03	41.7\\
8.5e+03	41.1\\
8.5e+03	38.6\\
8.5e+03	37.8\\
8.5e+03	38\\
8.50e+03	39.4\\
8.51e+03	41.1\\
8.51e+03	41.5\\
8.51e+03	41.1\\
8.51e+03	41.1\\
8.52e+03	41.1\\
8.52e+03	41.1\\
8.52e+03	37.9\\
8.52e+03	36.4\\
8.53e+03	36.6\\
8.53e+03	38.9\\
8.53e+03	41.1\\
8.53e+03	43.1\\
8.53e+03	43.1\\
8.54e+03	42.5\\
8.54e+03	42.2\\
8.54e+03	42.6\\
8.54e+03	41.1\\
8.54e+03	41.1\\
8.54e+03	38.3\\
8.55e+03	36.4\\
8.55e+03	36.4\\
8.55e+03	39\\
8.55e+03	41.1\\
8.55e+03	42.9\\
8.56e+03	43.1\\
8.56e+03	43.1\\
8.56e+03	43.1\\
8.56e+03	43.1\\
8.56e+03	43.1\\
8.56e+03	41.5\\
8.57e+03	41.1\\
8.57e+03	38.3\\
8.57e+03	36.6\\
8.57e+03	36\\
8.57e+03	36.4\\
8.58e+03	37.8\\
8.58e+03	40.1\\
8.58e+03	41.5\\
8.58e+03	41.6\\
8.58e+03	43\\
8.58e+03	42.7\\
8.59e+03	41.5\\
8.59e+03	40.4\\
8.59e+03	41.1\\
8.59e+03	43.1\\
8.59e+03	41.1\\
8.6e+03	38\\
8.6e+03	37.6\\
8.6e+03	37.7\\
8.6e+03	39.1\\
8.6e+03	41.1\\
8.61e+03	41.1\\
8.61e+03	41.1\\
8.61e+03	41.5\\
8.61e+03	42.9\\
8.61e+03	43.1\\
8.62e+03	43.1\\
8.62e+03	41.9\\
8.62e+03	41.1\\
8.62e+03	41.5\\
8.62e+03	43.1\\
8.63e+03	44.5\\
8.63e+03	45.1\\
8.63e+03	44.6\\
8.63e+03	45.3\\
8.63e+03	44.8\\
8.64e+03	43.5\\
8.64e+03	43.1\\
8.64e+03	42.5\\
8.64e+03	39.5\\
8.64e+03	38.2\\
8.65e+03	39.6\\
8.65e+03	41.6\\
8.65e+03	43.1\\
8.65e+03	43.1\\
8.66e+03	43.1\\
8.66e+03	43.1\\
8.66e+03	43.1\\
8.66e+03	43.1\\
8.67e+03	41.2\\
8.67e+03	39.1\\
8.67e+03	38.6\\
8.67e+03	39.6\\
8.67e+03	41.8\\
8.68e+03	42.9\\
8.68e+03	43.1\\
8.68e+03	43.1\\
8.68e+03	45.3\\
8.68e+03	45.5\\
8.69e+03	45.4\\
8.69e+03	45.7\\
8.69e+03	43.1\\
8.69e+03	42.1\\
8.69e+03	41.8\\
8.7e+03	43.1\\
8.7e+03	44.5\\
8.7e+03	44.5\\
8.7e+03	44.5\\
8.7e+03	44.5\\
8.71e+03	45.7\\
8.71e+03	45.9\\
8.71e+03	45.2\\
8.71e+03	45.7\\
8.71e+03	43.1\\
8.72e+03	42.6\\
8.72e+03	43.1\\
8.72e+03	43.5\\
8.72e+03	45.7\\
8.72e+03	45.7\\
8.73e+03	45.7\\
8.73e+03	46.8\\
8.73e+03	45.7\\
8.73e+03	44.5\\
8.74e+03	45.7\\
8.74e+03	43.5\\
8.74e+03	43.1\\
8.74e+03	42\\
8.74e+03	43.1\\
8.74e+03	44.5\\
8.75e+03	44.5\\
8.75e+03	44.3\\
8.75e+03	44.8\\
8.76e+03	44.5\\
8.76e+03	43.1\\
8.76e+03	43.1\\
8.76e+03	43.8\\
};
\end{axis}
\end{tikzpicture}%
    \caption{TRUC}
    \label{fig:ratio}
\end{figure}
\subsection{Question 4}
% QUESTION 4

The percent of generator revenue corresponding to the ancillary service is shown on table [\ref{pourcentage}] and figure [\ref{pourcent}]. We can see that quite logically, the cheapest generators (nuclear and biomass) are used for the production and not for the requirement of reserves. And the most expensive generators (oil) are almost exclusively used for making reserve. It makes sense since the reserves are cost-free, as opposed to the power production. Notice that since the ORDC model asks for more reserves than the two others, even the biomass generators have to participate in the procurement of reserves. Moreover, we can see (figure XXXX) that oil generators made a big leap towards profitability. The way the ORDC model guarantees the reliability of the system makes the oil generator far more valuable than expected.
 
\begin{table}[H]
\centering
\begin{tabular}{l | c  c  c}
type & EDR & ImpExp & ORDC \\
\hline
biomass &  $0.26$ & $0.53$ & $5.83$ \\
nuclear & $0$ & $0$ & $0$ \\
gas & $2$ & $3.42$ & $38.98$ \\
oil & $100$ & $100$ & $90.4$ \\
\end{tabular}
\caption{Percent of generator revenue corresponding to the ancillary services [\%]}
\label{pourcentage}
\end{table} 
 
\begin{figure}[H]
    \centering
    \setlength\fheight{4cm}
    \setlength\fwidth{0.8\textwidth}
    % This file was created by matlab2tikz.
% Minimal pgfplots version: 1.3
%
%The latest updates can be retrieved from
%  http://www.mathworks.com/matlabcentral/fileexchange/22022-matlab2tikz
%where you can also make suggestions and rate matlab2tikz.
%
\definecolor{mycolor1}{rgb}{0.84706,0.16078,0.00000}%
\definecolor{mycolor2}{rgb}{0.04314,0.51765,0.78039}%
\definecolor{mycolor3}{rgb}{0.87059,0.49020,0.00000}%
%
\begin{tikzpicture}

\begin{axis}[%
width=\fwidth,
height=\fheight,
at={(\fwidth,\fheight)},
scale only axis,
area legend,
separate axis lines,
every outer x axis line/.append style={black},
every x tick label/.append style={font=\color{black}},
xmin=0,
xmax=5,
xtick={0,1,2,3,4,5},
xticklabels={{},{Bio.},{Nucl.},{Gas},{Oil},{}},
xlabel={Generator type},
every outer y axis line/.append style={black},
every y tick label/.append style={font=\color{black}},
ymin=0,
ymax=110,
title={Percent of generator revenue corresponding to the ancillary services [\%]},
ymajorgrids,
legend style={at={(0.5,0.97)},anchor=north,legend columns=3,legend cell align=left,align=left,draw=white,fill=white}
]
\addplot[ybar,bar width=0.029662\fwidth,bar shift=-0.037077\fwidth,draw=black,fill=mycolor1] plot table[row sep=crcr] {%
1.00000	0.25971\\
2.00000	0.00000\\
3.00000	2.00764\\
4.00000	100.00000\\
};
\addlegendentry{ EDR};

\addplot [color=black,solid,forget plot]
  table[row sep=crcr]{%
0.50000	0.00000\\
4.50000	0.00000\\
};
\addplot[ybar,bar width=0.029662\fwidth,draw=black,fill=mycolor2] plot table[row sep=crcr] {%
1.00000	0.53238\\
2.00000	0.00000\\
3.00000	3.42201\\
4.00000	100.00005\\
};
\addlegendentry{ ImpExp};

\addplot[ybar,bar width=0.029662\fwidth,bar shift=0.037077\fwidth,draw=black,fill=mycolor3] plot table[row sep=crcr] {%
1.00000	5.82569\\
2.00000	0.00000\\
3.00000	38.97705\\
4.00000	90.40499\\
};
\addlegendentry{ ORDC};

\end{axis}
\end{tikzpicture}%
    \caption{Percent of generator revenue corresponding to the ancillary services [\%]}
    \label{pourcent}
\end{figure}

\subsection{Question 5}
% QUESTION 5

The reserve prices for the different models are available in figure [\ref{fig:EDR_R}], [\ref{fig:ImpExp_R}] et [\ref{fig:ORDC_R1}], and a zoom for the third model  is available on figure [\ref{fig:ORDC_R2}]. Notice that, as expected, the reserve prices are lower than energy prices. It can be interesting to  compare such results with the figure [\ref{fig:ratio}] which is re-displayed on figure [\ref{fig:ratio2}].

\begin{figure}[H]
    \centering
    \setlength\fheight{4cm}
    \setlength\fwidth{0.8\textwidth}
    % This file was created by matlab2tikz.
% Minimal pgfplots version: 1.3
%
%The latest updates can be retrieved from
%  http://www.mathworks.com/matlabcentral/fileexchange/22022-matlab2tikz
%where you can also make suggestions and rate matlab2tikz.
%
\definecolor{mycolor1}{rgb}{0.84706,0.16078,0.00000}%
%
\begin{tikzpicture}

\begin{axis}[%
width=\fwidth,
height=\fheight,
at={(0\fwidth,0\fheight)},
scale only axis,
separate axis lines,
every outer x axis line/.append style={black},
every x tick label/.append style={font=\color{black}},
xmin=0,
xmax=8760,
xlabel={time [hour]},
xmajorgrids,
every outer y axis line/.append style={black},
every y tick label/.append style={font=\color{black}},
ymin=0.5,
ymax=2.5,
ylabel={ratio level},
ymajorgrids
]
\addplot [color=mycolor1,line width=1.3pt,solid,forget plot]
  table[row sep=crcr]{%
1	1.76277299026698\\
2	1.9362109758773\\
3	2.01383811417918\\
4	2.14556960277483\\
5	2.21691525772622\\
6	2.25450999839374\\
7	2.26315544633488\\
8	2.35434350380979\\
9	2.43391137968296\\
10	2.41028010959221\\
11	2.32649951900646\\
12	2.31734256915795\\
13	2.27152828955165\\
14	2.17701182593871\\
15	2.18950187093352\\
16	2.16738441375517\\
17	2.10329602620842\\
18	1.97969107222188\\
19	2.0982931513597\\
20	2.12403009322197\\
21	2.06913982707518\\
22	2.07783314631548\\
23	1.97292246943499\\
24	1.93649148120353\\
25	2.13109399445936\\
26	2.28501729833992\\
27	2.40536610933119\\
28	2.48094107909005\\
29	2.44346375633962\\
30	2.27445515666959\\
31	2.03308065689467\\
32	1.88155208674845\\
33	1.73216866221003\\
34	1.64000652518427\\
35	1.60057681297788\\
36	1.58151613686274\\
37	1.58058487502011\\
38	1.56606783677838\\
39	1.58544314319611\\
40	1.55974041513492\\
41	1.52603227566969\\
42	1.53144300145539\\
43	1.525416005635\\
44	1.52953071537887\\
45	1.59593745591488\\
46	1.67644591024023\\
47	1.63861553513276\\
48	1.65121653230106\\
49	1.77613915853804\\
50	1.94773556379149\\
51	2.03533445565709\\
52	2.11733591042659\\
53	2.11885435849448\\
54	2.0276289081425\\
55	1.85174931598301\\
56	1.73599673433243\\
57	1.66927759045363\\
58	1.66690536535875\\
59	1.61999094793536\\
60	1.59120753866144\\
61	1.58786097022828\\
62	1.62368626516258\\
63	1.66078900659206\\
64	1.64645781444153\\
65	1.59058698874881\\
66	1.51986726206597\\
67	1.61304799908995\\
68	1.57010443545918\\
69	1.62070690520079\\
70	1.66772940795751\\
71	1.62282496964554\\
72	1.61791168836035\\
73	1.75267714540524\\
74	1.83123061439868\\
75	1.92988005157148\\
76	1.98364437652562\\
77	1.99543556846451\\
78	1.98695024960769\\
79	1.90051604156909\\
80	1.85169720657404\\
81	1.71267194465335\\
82	1.64541711806509\\
83	1.64862209604406\\
84	1.60332965175916\\
85	1.67281694661606\\
86	1.59948840821261\\
87	1.61107300828397\\
88	1.60686449608187\\
89	1.56867037987429\\
90	1.51548309597428\\
91	1.57540234481252\\
92	1.62941053054905\\
93	1.70433420248443\\
94	1.63735890708814\\
95	1.63629059667911\\
96	1.60060428716605\\
97	1.60333361776825\\
98	1.708673988582\\
99	1.80413887459463\\
100	1.84530409485739\\
101	1.85130687020377\\
102	1.83900078997816\\
103	1.84939800153386\\
104	1.83193932063438\\
105	1.77380157351207\\
106	1.7142700447578\\
107	1.69152794751149\\
108	1.66794040072872\\
109	1.65017016272179\\
110	1.6452922441227\\
111	1.68088411609648\\
112	1.64843811576025\\
113	1.59744676877715\\
114	1.47500979093346\\
115	1.50530933054361\\
116	1.55305918071202\\
117	1.62607089550037\\
118	1.77848352311283\\
119	1.81048840080216\\
120	1.77720217708164\\
121	1.92955755097854\\
122	2.0586650212972\\
123	2.1598769632186\\
124	2.24455928924168\\
125	2.21228853250866\\
126	2.03376000182603\\
127	1.75874791088956\\
128	1.54699661737417\\
129	1.56648080193148\\
130	1.55673122189553\\
131	1.45903852996141\\
132	1.45690692494384\\
133	1.49567169832483\\
134	1.48083543336548\\
135	1.4779962361954\\
136	1.47477542990651\\
137	1.44906650377054\\
138	1.3604406435506\\
139	1.41912403058445\\
140	1.50388994692369\\
141	1.45971957118769\\
142	1.54310988359145\\
143	1.53823670087808\\
144	1.5957179230879\\
145	1.71352765438218\\
146	1.83434967945918\\
147	1.9298084065673\\
148	1.97640821697432\\
149	1.93860794318979\\
150	1.8216968497538\\
151	1.60093419425915\\
152	1.44256959766703\\
153	1.36260426079029\\
154	1.41204413651093\\
155	1.43865777808738\\
156	1.43382352713922\\
157	1.47426399251346\\
158	1.45456417218019\\
159	1.43658453492889\\
160	1.36700176783911\\
161	1.36056436315477\\
162	1.28154614249717\\
163	1.27501489230533\\
164	1.35115937186708\\
165	1.40542391889273\\
166	1.42776817561145\\
167	1.41974512425605\\
168	1.36596777402959\\
169	1.49388492672713\\
170	1.61341759269041\\
171	1.76656294926166\\
172	1.80540856979635\\
173	1.74095449895641\\
174	1.62969123554603\\
175	1.44203464831519\\
176	1.35588876972628\\
177	1.26549609812125\\
178	1.31240747128363\\
179	1.35282654998826\\
180	1.37722093617662\\
181	1.32003445397444\\
182	1.20120839897165\\
183	1.17864165503985\\
184	1.14837152696739\\
185	1.1324719261063\\
186	1.09369475619244\\
187	1.08296583518112\\
188	1.13393964490555\\
189	1.1534939896481\\
190	1.21425820839644\\
191	1.2072493286163\\
192	1.24985209442554\\
193	1.47786225976837\\
194	1.60703148308646\\
195	1.68771620741077\\
196	1.71999475243835\\
197	1.71036030092028\\
198	1.63790889272756\\
199	1.47141800969484\\
200	1.33413454908006\\
201	1.28769053648502\\
202	1.34515586211072\\
203	1.34616681130741\\
204	1.33537541599863\\
205	1.2984571870324\\
206	1.29047608089168\\
207	1.30252820413333\\
208	1.29644384757354\\
209	1.30748395845728\\
210	1.29851816762616\\
211	1.28592995826083\\
212	1.33319453925039\\
213	1.40518548656661\\
214	1.40175012865366\\
215	1.37204876271802\\
216	1.37931051578157\\
217	1.44203407418592\\
218	1.5329550020917\\
219	1.58922488996188\\
220	1.60005729224535\\
221	1.56657308444984\\
222	1.50507658910994\\
223	1.344867776067\\
224	1.21421912255514\\
225	1.18686565116026\\
226	1.241115355355\\
227	1.26697313609213\\
228	1.24434454821355\\
229	1.30892654343811\\
230	1.28631726467607\\
231	1.27072022446387\\
232	1.25486438864895\\
233	1.23053138918279\\
234	1.19175647135829\\
235	1.21057509119057\\
236	1.27222433362268\\
237	1.31865765608143\\
238	1.34677117714842\\
239	1.32770463768483\\
240	1.32075888993452\\
241	1.46066681046678\\
242	1.547044957561\\
243	1.58624427321258\\
244	1.64651415851665\\
245	1.67388840923648\\
246	1.65220191938573\\
247	1.59427635094351\\
248	1.59892002270501\\
249	1.45262103531662\\
250	1.40824512293663\\
251	1.39409060694883\\
252	1.39127721090953\\
253	1.3862368597072\\
254	1.38264197700962\\
255	1.41432073847967\\
256	1.40449281104224\\
257	1.38033142510119\\
258	1.30227087688349\\
259	1.32648268347874\\
260	1.36001145131298\\
261	1.43664635278679\\
262	1.41519867049676\\
263	1.36716440482475\\
264	1.44093747579285\\
265	1.42708775687099\\
266	1.45579779437206\\
267	1.5018144498523\\
268	1.54531878932517\\
269	1.5607828473973\\
270	1.57020401355159\\
271	1.55422790806552\\
272	1.53007104958006\\
273	1.47780263037668\\
274	1.44850251576378\\
275	1.44763465213742\\
276	1.48436113014352\\
277	1.48922489452031\\
278	1.52269035611753\\
279	1.58907165256958\\
280	1.55955281541541\\
281	1.52052486096639\\
282	1.39567062084007\\
283	1.3939770892222\\
284	1.42990942972091\\
285	1.50906604656547\\
286	1.58245128202283\\
287	1.60741687258773\\
288	1.5207598905816\\
289	1.67452532482118\\
290	1.77546489765398\\
291	1.85957918589554\\
292	1.91402148458577\\
293	1.89650070517672\\
294	1.74696708840451\\
295	1.50501748465052\\
296	1.3798084036594\\
297	1.38313294903386\\
298	1.3611042249003\\
299	1.31881420134862\\
300	1.3090446410692\\
301	1.33623091364315\\
302	1.33123361366905\\
303	1.34069078970888\\
304	1.32718363395993\\
305	1.30857466833095\\
306	1.28471155305273\\
307	1.25691034598817\\
308	1.33335631294786\\
309	1.36409933389232\\
310	1.42725057367219\\
311	1.4094913890812\\
312	1.43161837728613\\
313	1.574892929793\\
314	1.66336113250588\\
315	1.71557384708476\\
316	1.74978059408488\\
317	1.72650176668612\\
318	1.6179268743252\\
319	1.43004668188753\\
320	1.29662255454809\\
321	1.28774957483874\\
322	1.26190178514174\\
323	1.25859273979119\\
324	1.24772169123354\\
325	1.26472785981096\\
326	1.27342532226228\\
327	1.25336353826995\\
328	1.24188587235901\\
329	1.2256149716125\\
330	1.18724701068376\\
331	1.21491913485517\\
332	1.24042163532429\\
333	1.26041684345534\\
334	1.33549368655742\\
335	1.31910208885942\\
336	1.34178377902336\\
337	1.4414563995484\\
338	1.51649026335007\\
339	1.58874500977674\\
340	1.64401231065145\\
341	1.64333927896388\\
342	1.58487919328203\\
343	1.44765681706216\\
344	1.32349924033057\\
345	1.32040389513651\\
346	1.32683655341044\\
347	1.30646473725458\\
348	1.30183153724242\\
349	1.31938225968908\\
350	1.31653810883307\\
351	1.33728867550053\\
352	1.33449907545004\\
353	1.32684582732477\\
354	1.29624514138962\\
355	1.30342826413165\\
356	1.31598558017386\\
357	1.37149665391699\\
358	1.44411083117731\\
359	1.42273409673338\\
360	1.43288212789452\\
361	1.56156681111608\\
362	1.65805066814055\\
363	1.73383723304027\\
364	1.80027569463498\\
365	1.79076921017304\\
366	1.70237334242708\\
367	1.50627034577754\\
368	1.38280450851465\\
369	1.40363232494979\\
370	1.38039305901844\\
371	1.38561609737659\\
372	1.35886004449793\\
373	1.33061081263968\\
374	1.2762345619689\\
375	1.28868787611307\\
376	1.30370296482384\\
377	1.31588247829049\\
378	1.26232366833322\\
379	1.29904372010335\\
380	1.32493243669322\\
381	1.36408016984428\\
382	1.45092710060791\\
383	1.44301840641233\\
384	1.46803816272251\\
385	1.59014635166389\\
386	1.69124572927249\\
387	1.77323684872064\\
388	1.79692178576831\\
389	1.77942376903925\\
390	1.70842889842567\\
391	1.52400821779961\\
392	1.43431186190209\\
393	1.41364201785654\\
394	1.37109301865032\\
395	1.38852269101093\\
396	1.41491375462163\\
397	1.45016793839661\\
398	1.43598470585223\\
399	1.43376523935839\\
400	1.39392912081024\\
401	1.34787697522538\\
402	1.29405727168018\\
403	1.33263931757675\\
404	1.36772197434555\\
405	1.39394917679652\\
406	1.45785006353957\\
407	1.43687687000337\\
408	1.43756091893541\\
409	1.54513831675944\\
410	1.64113865265369\\
411	1.71862852481592\\
412	1.76524859711787\\
413	1.78858206471907\\
414	1.78215758801291\\
415	1.72590314407838\\
416	1.70769707835731\\
417	1.65374752305444\\
418	1.58853190547876\\
419	1.60792899698593\\
420	1.62537315846665\\
421	1.67945378457126\\
422	1.63589096353438\\
423	1.63413072563917\\
424	1.61398532914327\\
425	1.5598769260042\\
426	1.46610498216442\\
427	1.45798021940989\\
428	1.51970504444844\\
429	1.50500070260286\\
430	1.55883645125382\\
431	1.5642349073119\\
432	1.51760651649515\\
433	1.62134594600579\\
434	1.7299694453504\\
435	1.82609237324793\\
436	1.88109326894015\\
437	1.89297688525953\\
438	1.86954481734432\\
439	1.81858381661576\\
440	1.77531511191274\\
441	1.6962042059528\\
442	1.61739867472062\\
443	1.54522851459845\\
444	1.51219411784021\\
445	1.49357737919357\\
446	1.50762819614386\\
447	1.53846278673096\\
448	1.5369003657501\\
449	1.52938868049174\\
450	1.41576693939692\\
451	1.41954990165677\\
452	1.42689013367824\\
453	1.46615712933264\\
454	1.46560038353042\\
455	1.45953229505914\\
456	1.45174464719133\\
457	1.54359611563728\\
458	1.61802584355109\\
459	1.68459873285971\\
460	1.72303378197357\\
461	1.69861027943118\\
462	1.58686706626664\\
463	1.39976813220679\\
464	1.27520468506693\\
465	1.2847985204491\\
466	1.24220220381518\\
467	1.22855651510009\\
468	1.22016794031514\\
469	1.2450583394897\\
470	1.21590632775143\\
471	1.2219561900253\\
472	1.2158114460006\\
473	1.21235429323892\\
474	1.1916377178075\\
475	1.21335468320486\\
476	1.27196216754927\\
477	1.26932299767581\\
478	1.3755625747909\\
479	1.35404410395878\\
480	1.40172457699525\\
481	1.45344584899125\\
482	1.53708673344626\\
483	1.60293159732125\\
484	1.62303194170461\\
485	1.60835691862427\\
486	1.53757066792223\\
487	1.37327195455553\\
488	1.25152693074666\\
489	1.27313193920395\\
490	1.31139085201476\\
491	1.31294834155599\\
492	1.2786600410626\\
493	1.29458152892861\\
494	1.26897251758148\\
495	1.26593196571905\\
496	1.25878626657506\\
497	1.24713192049216\\
498	1.19395580505083\\
499	1.23576849880719\\
500	1.24268104965154\\
501	1.28854048682447\\
502	1.37442629548452\\
503	1.35954938540214\\
504	1.40446036298426\\
505	1.50441005927358\\
506	1.5996712178095\\
507	1.65111979905229\\
508	1.70210305802094\\
509	1.69733858565748\\
510	1.6188021154037\\
511	1.44519807008918\\
512	1.3313803862551\\
513	1.39051310205313\\
514	1.36482779727784\\
515	1.40483236387018\\
516	1.40081751171981\\
517	1.41690477471042\\
518	1.36939238212201\\
519	1.34687223843381\\
520	1.31890100056159\\
521	1.29432370040765\\
522	1.23448013905997\\
523	1.25388636083406\\
524	1.26339458873812\\
525	1.30136029501626\\
526	1.36069355959173\\
527	1.334616455149\\
528	1.3479062319039\\
529	1.45481212646417\\
530	1.52166570116622\\
531	1.56446933306228\\
532	1.61141915937192\\
533	1.60603872291545\\
534	1.53835877765572\\
535	1.38940337070939\\
536	1.27584549408996\\
537	1.29285316959904\\
538	1.28897411425697\\
539	1.29391135078365\\
540	1.30131126627011\\
541	1.31787921464603\\
542	1.29506739435805\\
543	1.30635435807575\\
544	1.30187104975722\\
545	1.3027247225854\\
546	1.25822590655936\\
547	1.29906035509768\\
548	1.33939292922362\\
549	1.38578264052682\\
550	1.45717525502922\\
551	1.43312031928977\\
552	1.44900088710226\\
553	1.55937696839204\\
554	1.63501686817876\\
555	1.69825399513108\\
556	1.71673484929485\\
557	1.68000184351256\\
558	1.59031421552231\\
559	1.39909073282687\\
560	1.27103151627006\\
561	1.29705670329196\\
562	1.31298671718102\\
563	1.33931474977036\\
564	1.34971069486066\\
565	1.36991431763676\\
566	1.37300173925463\\
567	1.3676158718002\\
568	1.3590842523661\\
569	1.34043924350929\\
570	1.28751488318873\\
571	1.31159591883922\\
572	1.34074268343609\\
573	1.38428162882628\\
574	1.46728353059486\\
575	1.42766947528636\\
576	1.43077365205217\\
577	1.46965592667972\\
578	1.55409949860337\\
579	1.63112223478094\\
580	1.68753651701994\\
581	1.72385709531491\\
582	1.7486000407942\\
583	1.71485279364862\\
584	1.65341377580699\\
585	1.55181332510421\\
586	1.50246746378642\\
587	1.47591604266792\\
588	1.47817806382629\\
589	1.47648757596111\\
590	1.49668606436226\\
591	1.53429645667311\\
592	1.54657927475067\\
593	1.54819179671975\\
594	1.50107870369551\\
595	1.50691695757208\\
596	1.55004630281238\\
597	1.63408831096859\\
598	1.70978822012148\\
599	1.6944591711967\\
600	1.66074523024259\\
601	1.84524928648012\\
602	1.97427958129578\\
603	1.9660597947546\\
604	1.99934685014634\\
605	1.9965626800371\\
606	1.95879041668584\\
607	1.88856356217999\\
608	1.84397626118771\\
609	1.74380632244057\\
610	1.67052369384502\\
611	1.62711929326756\\
612	1.62432160287537\\
613	1.68293577762177\\
614	1.64192684402621\\
615	1.69827504430018\\
616	1.71281082957701\\
617	1.71714891336484\\
618	1.64253052822686\\
619	1.70810125833876\\
620	1.71641323052057\\
621	1.610939613585\\
622	1.68466596648788\\
623	1.66924164434493\\
624	1.68926615694046\\
625	1.79173315575833\\
626	1.8895456475136\\
627	1.95497158077937\\
628	1.9632395147859\\
629	1.93272569793698\\
630	1.8068722183789\\
631	1.56907480502023\\
632	1.42438792911335\\
633	1.41629012897633\\
634	1.42021381795637\\
635	1.41449757964959\\
636	1.42114761720527\\
637	1.41624134453463\\
638	1.4105352894036\\
639	1.40796312305481\\
640	1.38083416960464\\
641	1.35758903685862\\
642	1.37651597731509\\
643	1.34087984697575\\
644	1.3943740736046\\
645	1.40900782611426\\
646	1.48405206524128\\
647	1.46675979578271\\
648	1.50042115530657\\
649	1.57704352113774\\
650	1.68550491404402\\
651	1.75785485568288\\
652	1.80529597597514\\
653	1.80472682539044\\
654	1.72292484599695\\
655	1.52731434800408\\
656	1.39117531321251\\
657	1.43240077204004\\
658	1.43476001300265\\
659	1.4282522784179\\
660	1.41731447177099\\
661	1.43466689998475\\
662	1.39747231627165\\
663	1.38870433466752\\
664	1.38557623323142\\
665	1.37930929628494\\
666	1.36208197108929\\
667	1.35616450006906\\
668	1.41736774690988\\
669	1.40405316011646\\
670	1.47956539000948\\
671	1.41024674014638\\
672	1.42344418122247\\
673	1.52325803237152\\
674	1.59533141440017\\
675	1.6505564709669\\
676	1.69652727946484\\
677	1.68846195401973\\
678	1.62230223076158\\
679	1.48692803428247\\
680	1.36224369716512\\
681	1.3989263607249\\
682	1.41634710975467\\
683	1.45541238110692\\
684	1.46336381528691\\
685	1.50759412554392\\
686	1.46231899419032\\
687	1.46864593217488\\
688	1.41624323824673\\
689	1.38063134541111\\
690	1.35477776778578\\
691	1.3124969732399\\
692	1.35691058341962\\
693	1.39050102140604\\
694	1.45048734994841\\
695	1.41514120107166\\
696	1.41968223960922\\
697	1.50658799381731\\
698	1.56716720961811\\
699	1.61672363131233\\
700	1.64387382392022\\
701	1.64758434846582\\
702	1.57114311960646\\
703	1.39538546285337\\
704	1.27453165043858\\
705	1.32637338316898\\
706	1.28784333473782\\
707	1.32630073733953\\
708	1.31373001990013\\
709	1.35353770978764\\
710	1.3389170180987\\
711	1.33448210637232\\
712	1.31638979835614\\
713	1.27744058876725\\
714	1.23086904047905\\
715	1.24328449590893\\
716	1.28443134705923\\
717	1.28831422952243\\
718	1.34981910891928\\
719	1.31751590455564\\
720	1.3250984687833\\
721	1.41029381786496\\
722	1.47271568453849\\
723	1.52511327109967\\
724	1.55530556132148\\
725	1.55298156179936\\
726	1.49084713950088\\
727	1.34552300752499\\
728	1.24355906130386\\
729	1.32191630991776\\
730	1.36580484448721\\
731	1.38818902663135\\
732	1.42453607201295\\
733	1.50568883311062\\
734	1.48015718000949\\
735	1.50877943097414\\
736	1.51444399329173\\
737	1.46671678543549\\
738	1.41540005388023\\
739	1.39675921516997\\
740	1.47420084196088\\
741	1.51309090010367\\
742	1.60314345034369\\
743	1.55891283964275\\
744	1.5663111153011\\
745	1.69312795693061\\
746	1.81661551150348\\
747	1.90260740418457\\
748	2.01874083549618\\
749	2.03977513107709\\
750	2.06088406889545\\
751	1.98006914670823\\
752	1.88272667411938\\
753	1.74453497973873\\
754	1.62420122970536\\
755	1.57111266755559\\
756	1.55641415600064\\
757	1.56620462673971\\
758	1.61877454000787\\
759	1.67990588330399\\
760	1.68256976758404\\
761	1.64874445800944\\
762	1.59728238502265\\
763	1.57546667661705\\
764	1.62058435974641\\
765	1.61118186076976\\
766	1.69074968040058\\
767	1.66389200712825\\
768	1.63749096504245\\
769	1.74521546561771\\
770	1.82523954905672\\
771	1.88369980800598\\
772	1.94603156291641\\
773	1.9931706110492\\
774	1.97706548395676\\
775	1.9657631653116\\
776	1.93629872464328\\
777	1.92762780438199\\
778	1.85127747329721\\
779	1.80614912803815\\
780	1.84183323828688\\
781	1.82971978275727\\
782	1.80673575813586\\
783	1.83209099976229\\
784	1.80146496904007\\
785	1.74112018505493\\
786	1.63620329096007\\
787	1.56239758790357\\
788	1.56133371021367\\
789	1.56526549258313\\
790	1.60291230934364\\
791	1.58096706003214\\
792	1.57160079126442\\
793	1.65024041418316\\
794	1.69549572236993\\
795	1.76453354348502\\
796	1.77771965203256\\
797	1.76434613352481\\
798	1.66792294655795\\
799	1.4726994966479\\
800	1.35123674826025\\
801	1.3879343116216\\
802	1.42620771940288\\
803	1.49182620814653\\
804	1.52114691085046\\
805	1.53887394517554\\
806	1.5011412453587\\
807	1.47808953502875\\
808	1.43111661761871\\
809	1.38545348625869\\
810	1.31397426159115\\
811	1.27892673805515\\
812	1.32452516571302\\
813	1.36838401324018\\
814	1.44268179624416\\
815	1.4221425234759\\
816	1.44205624768149\\
817	1.54633180058829\\
818	1.634870377183\\
819	1.70714180752753\\
820	1.76368526898442\\
821	1.76324107776226\\
822	1.68098099772347\\
823	1.48712609375201\\
824	1.36581602185607\\
825	1.41633156847929\\
826	1.41757913236043\\
827	1.42518289338379\\
828	1.41870378206515\\
829	1.48406492380573\\
830	1.4818217618441\\
831	1.49099733914413\\
832	1.45618971466303\\
833	1.42669944015305\\
834	1.37498702569358\\
835	1.33216594053293\\
836	1.39644531816491\\
837	1.41938768124388\\
838	1.52003720578367\\
839	1.49704043314214\\
840	1.521003682554\\
841	1.64521534923316\\
842	1.74865687424519\\
843	1.844020267108\\
844	1.89804313849664\\
845	1.88111088606937\\
846	1.80505130214543\\
847	1.58980536237336\\
848	1.48629598035734\\
849	1.42788900608666\\
850	1.44257561170201\\
851	1.42186849609995\\
852	1.42859525501049\\
853	1.4633932732358\\
854	1.4658599392089\\
855	1.50307760055265\\
856	1.47631532693688\\
857	1.44151765967084\\
858	1.36654744516495\\
859	1.37347738727311\\
860	1.42753918173368\\
861	1.48569428473421\\
862	1.54035622844191\\
863	1.52261597863579\\
864	1.53469167066894\\
865	1.65119713434204\\
866	1.75721840912235\\
867	1.82476338763658\\
868	1.86716994115569\\
869	1.84120614219687\\
870	1.76403278965096\\
871	1.57350134328106\\
872	1.49106334555056\\
873	1.49366230267771\\
874	1.47881111906073\\
875	1.47012370188843\\
876	1.45904716839532\\
877	1.45729861984901\\
878	1.42416624889929\\
879	1.42113976209236\\
880	1.42193104176669\\
881	1.40334629441289\\
882	1.35390958758701\\
883	1.37994750221938\\
884	1.41930795610026\\
885	1.50889179359612\\
886	1.55461034613988\\
887	1.5214481089317\\
888	1.54144564960204\\
889	1.66247581609949\\
890	1.74837899723901\\
891	1.8313750521402\\
892	1.87991195674547\\
893	1.87487045198552\\
894	1.80906090903083\\
895	1.61538029761186\\
896	1.50656396502617\\
897	1.50562551414465\\
898	1.47992271558814\\
899	1.42984968458926\\
900	1.36647573655411\\
901	1.399265528488\\
902	1.39152636079161\\
903	1.40714185134752\\
904	1.43237364118373\\
905	1.4796397109341\\
906	1.45499891954891\\
907	1.42255533792565\\
908	1.4637792016313\\
909	1.47571063943682\\
910	1.54560279274649\\
911	1.49637063699369\\
912	1.50644402335816\\
913	1.64095134846136\\
914	1.76655506853929\\
915	1.85169610411146\\
916	1.9550732967038\\
917	1.98811424541056\\
918	2.0295579324086\\
919	1.99456912037404\\
920	1.90831344503783\\
921	1.88240140879383\\
922	1.72506653982239\\
923	1.64743988972254\\
924	1.62896914972356\\
925	1.63508577264848\\
926	1.62693591095608\\
927	1.68858655290345\\
928	1.70165051868589\\
929	1.71211723253997\\
930	1.69522605053988\\
931	1.64213539082683\\
932	1.69361384529426\\
933	1.81790138662435\\
934	1.80487043961167\\
935	1.78939048649405\\
936	1.70467365940436\\
937	1.8437246357153\\
938	1.96291987526031\\
939	2.06159363965088\\
940	2.11625509932\\
941	2.14305138463847\\
942	2.14450642632571\\
943	2.07689007586728\\
944	2.03632435745472\\
945	1.98279245571543\\
946	1.88143228691825\\
947	1.85378804321048\\
948	1.82342656930013\\
949	1.81059847081694\\
950	1.86246835786853\\
951	1.93080628877364\\
952	1.90292890755801\\
953	1.85667907649698\\
954	1.74221015370484\\
955	1.6089250089362\\
956	1.6560649401284\\
957	1.64607088083511\\
958	1.71418491384621\\
959	1.68344246835954\\
960	1.60882677268093\\
961	1.67132819091707\\
962	1.72555516899171\\
963	1.77395723700183\\
964	1.79974412687181\\
965	1.75722522627825\\
966	1.66263727634158\\
967	1.46491959477324\\
968	1.31671221874543\\
969	1.31000900144211\\
970	1.31105534121874\\
971	1.34150114675608\\
972	1.32049163449593\\
973	1.35208248471259\\
974	1.30694859248855\\
975	1.3056796611363\\
976	1.28985572848652\\
977	1.28163973646872\\
978	1.23708156357472\\
979	1.23263819568989\\
980	1.27655062245345\\
981	1.3312759378392\\
982	1.38596304679113\\
983	1.37763506164926\\
984	1.39538360242385\\
985	1.50894801764543\\
986	1.61664389923648\\
987	1.67362610521443\\
988	1.72611573487201\\
989	1.72263354742266\\
990	1.63730140148844\\
991	1.49044782470049\\
992	1.42038857785016\\
993	1.44938679205965\\
994	1.45322839661172\\
995	1.49337573802593\\
996	1.47235105704674\\
997	1.51741748295604\\
998	1.49343186233545\\
999	1.47857161361539\\
1000	1.45848335473745\\
1001	1.42836115621628\\
1002	1.36301259836134\\
1003	1.34882501794498\\
1004	1.37386120893641\\
1005	1.4153343116423\\
1006	1.45354418012722\\
1007	1.43332551826903\\
1008	1.4427330067618\\
1009	1.57424484634047\\
1010	1.67224650123079\\
1011	1.74454373587716\\
1012	1.7673704406926\\
1013	1.75048302581888\\
1014	1.67351969563248\\
1015	1.49510195567119\\
1016	1.33013726410298\\
1017	1.39709039000402\\
1018	1.45587032411577\\
1019	1.55232287633046\\
1020	1.5776603903491\\
1021	1.58374480790873\\
1022	1.54456317195354\\
1023	1.53128507799663\\
1024	1.50049415357034\\
1025	1.46935885884903\\
1026	1.39705523322851\\
1027	1.3860071014324\\
1028	1.44062515351358\\
1029	1.48361467760865\\
1030	1.53571763920509\\
1031	1.52097841994156\\
1032	1.55695828662321\\
1033	1.68568835994277\\
1034	1.78707439912066\\
1035	1.84713248447074\\
1036	1.8790455424182\\
1037	1.85105993934949\\
1038	1.73016525699352\\
1039	1.54066388131943\\
1040	1.46423862487333\\
1041	1.44409807798525\\
1042	1.37072846645109\\
1043	1.35937500557471\\
1044	1.30765262988547\\
1045	1.31455165436822\\
1046	1.30069896496736\\
1047	1.3132226273768\\
1048	1.31433358389226\\
1049	1.30290503098499\\
1050	1.28851264875118\\
1051	1.29992203540171\\
1052	1.31439882460924\\
1053	1.38449667324796\\
1054	1.43353217012013\\
1055	1.41875802244783\\
1056	1.44082788992884\\
1057	1.51876833470765\\
1058	1.59651308186906\\
1059	1.66506821178456\\
1060	1.67614379268271\\
1061	1.65868065520219\\
1062	1.57788380915965\\
1063	1.3999346145719\\
1064	1.36923496882436\\
1065	1.4094496309803\\
1066	1.42822400878646\\
1067	1.45115474338077\\
1068	1.44200252952516\\
1069	1.43340494507921\\
1070	1.40251185098932\\
1071	1.39892990072047\\
1072	1.40960483957274\\
1073	1.41448926120727\\
1074	1.37757632871137\\
1075	1.43444644083881\\
1076	1.44988064418772\\
1077	1.51389512199336\\
1078	1.61649070187475\\
1079	1.57783043164336\\
1080	1.57948373189773\\
1081	1.73418352616715\\
1082	1.84317561747041\\
1083	1.94674236292082\\
1084	2.00828716588167\\
1085	2.01860126077626\\
1086	1.99586568962558\\
1087	1.91253637441223\\
1088	1.84217069234444\\
1089	1.81418068537273\\
1090	1.81752531641432\\
1091	1.71914294164772\\
1092	1.74266396939076\\
1093	1.80468110878044\\
1094	1.79802574778965\\
1095	1.83398481761457\\
1096	1.80525511400656\\
1097	1.77457408845398\\
1098	1.7097823020377\\
1099	1.66346135515253\\
1100	1.70830133148827\\
1101	1.77548399670531\\
1102	1.82616298123159\\
1103	1.74282534486713\\
1104	1.69227453217884\\
1105	1.78154237439685\\
1106	1.87606952171848\\
1107	1.94776430679222\\
1108	2.00055212707833\\
1109	2.0317298550167\\
1110	2.02677108784173\\
1111	1.9612154012034\\
1112	1.93058484874829\\
1113	1.87299832091689\\
1114	1.84890387234109\\
1115	1.84378229039902\\
1116	1.82978613907045\\
1117	1.85617918758327\\
1118	1.89172542611858\\
1119	1.94576232372104\\
1120	1.891292049772\\
1121	1.83464558254731\\
1122	1.7361635451403\\
1123	1.59472095999948\\
1124	1.53626895064466\\
1125	1.53266254771631\\
1126	1.56936328762474\\
1127	1.53237270227949\\
1128	1.52780875711811\\
1129	1.62914697623003\\
1130	1.70976532785867\\
1131	1.77444675760729\\
1132	1.80208436312904\\
1133	1.80256284269429\\
1134	1.69381241785247\\
1135	1.48624529764477\\
1136	1.35558423369473\\
1137	1.37382352078158\\
1138	1.41603960956189\\
1139	1.43085764994221\\
1140	1.4157662484423\\
1141	1.46799235969682\\
1142	1.47586247120746\\
1143	1.47914936054792\\
1144	1.46064706104408\\
1145	1.49755421461314\\
1146	1.43841391533394\\
1147	1.33490754047782\\
1148	1.34741107276169\\
1149	1.36916578503131\\
1150	1.43938449404708\\
1151	1.44415666785907\\
1152	1.46413645182811\\
1153	1.56676835124536\\
1154	1.68134294892787\\
1155	1.72749169674902\\
1156	1.77113549880987\\
1157	1.74370605836107\\
1158	1.6622781808038\\
1159	1.46883382597432\\
1160	1.40505027198735\\
1161	1.42505117805576\\
1162	1.4031398690306\\
1163	1.37754224128762\\
1164	1.34516865499778\\
1165	1.35212186955506\\
1166	1.33989112400067\\
1167	1.34192471492456\\
1168	1.3381780203424\\
1169	1.33352972431269\\
1170	1.29173057333978\\
1171	1.28172427725849\\
1172	1.32270899815576\\
1173	1.35658770252284\\
1174	1.44256060065255\\
1175	1.41899999129688\\
1176	1.44455728308238\\
1177	1.54716654607358\\
1178	1.63895035997102\\
1179	1.69528826144815\\
1180	1.71753325155394\\
1181	1.70604927609225\\
1182	1.61311983379675\\
1183	1.43011351138937\\
1184	1.34096832613863\\
1185	1.36494041505172\\
1186	1.35692566990408\\
1187	1.32413428124812\\
1188	1.29832548065922\\
1189	1.34451838854499\\
1190	1.33642490005658\\
1191	1.34528306547003\\
1192	1.33719362356672\\
1193	1.33143253026307\\
1194	1.30717042483666\\
1195	1.28699753265233\\
1196	1.30071540197641\\
1197	1.31977526311931\\
1198	1.38211224027533\\
1199	1.37213276154943\\
1200	1.40544091565439\\
1201	1.54600757081468\\
1202	1.65571047132507\\
1203	1.72908688272841\\
1204	1.81375617109401\\
1205	1.82549156106991\\
1206	1.72296749314694\\
1207	1.54964971186277\\
1208	1.50512413012915\\
1209	1.52318854159098\\
1210	1.51934242244397\\
1211	1.4904264290003\\
1212	1.42015123159394\\
1213	1.43765546374575\\
1214	1.42709584723878\\
1215	1.43125046260825\\
1216	1.4294696397711\\
1217	1.41446075282807\\
1218	1.38750100881769\\
1219	1.35966737410752\\
1220	1.39556448354121\\
1221	1.43928024060472\\
1222	1.49117563825254\\
1223	1.46093066891589\\
1224	1.47431628851414\\
1225	1.5702892436199\\
1226	1.65342391651488\\
1227	1.73958356940577\\
1228	1.77581647152714\\
1229	1.75252308001367\\
1230	1.65004500457537\\
1231	1.46956108396099\\
1232	1.41541568032266\\
1233	1.4404840930802\\
1234	1.44425556851776\\
1235	1.43109241740852\\
1236	1.43335937218678\\
1237	1.50783226912512\\
1238	1.51647939277927\\
1239	1.53089432825976\\
1240	1.52569688970089\\
1241	1.50764107425346\\
1242	1.46310086532589\\
1243	1.37171466336432\\
1244	1.41565065859734\\
1245	1.48133044168859\\
1246	1.56079048037675\\
1247	1.51530596275611\\
1248	1.52590567359812\\
1249	1.65830738857164\\
1250	1.77284659278082\\
1251	1.84627269789947\\
1252	1.92916958691043\\
1253	1.9514692096126\\
1254	1.91936108516281\\
1255	1.85311838917957\\
1256	1.78960761691917\\
1257	1.6984197898694\\
1258	1.60234131220283\\
1259	1.56517960449799\\
1260	1.5601282652347\\
1261	1.59418739433646\\
1262	1.66455205067402\\
1263	1.69594281764828\\
1264	1.68524320143251\\
1265	1.6637412476646\\
1266	1.59013997143677\\
1267	1.53330745590325\\
1268	1.5305837609147\\
1269	1.62017425838932\\
1270	1.69456206914855\\
1271	1.67524295653315\\
1272	1.61997175275389\\
1273	1.7128122291888\\
1274	1.83892109633137\\
1275	1.92821007330604\\
1276	1.99709661667188\\
1277	2.02470207018891\\
1278	2.0167708069749\\
1279	1.99346655253803\\
1280	2.0020826797628\\
1281	2.13494179147894\\
1282	2.08165802325483\\
1283	2.02736485298379\\
1284	2.05772736749435\\
1285	2.11739969432957\\
1286	2.08410972896309\\
1287	2.12454525944562\\
1288	2.06328396489125\\
1289	1.96729191393079\\
1290	1.83334565737404\\
1291	1.70053965614033\\
1292	1.65554380700852\\
1293	1.6306205619437\\
1294	1.6848442640391\\
1295	1.6936906605813\\
1296	1.68703692432894\\
1297	1.81779042153918\\
1298	1.91734803120658\\
1299	1.99753635343761\\
1300	2.04069685995945\\
1301	2.0041323861599\\
1302	1.86057238095737\\
1303	1.62660400581672\\
1304	1.53441390584699\\
1305	1.55285993887301\\
1306	1.5771227598962\\
1307	1.60086900917826\\
1308	1.65509076792046\\
1309	1.82705438066427\\
1310	1.69636652960844\\
1311	1.68122719697985\\
1312	1.61561404023547\\
1313	1.56220533760339\\
1314	1.48000616457412\\
1315	1.43375860464433\\
1316	1.42809870762589\\
1317	1.46380640575576\\
1318	1.54736593938664\\
1319	1.53630099156625\\
1320	1.5772069827285\\
1321	1.72666829983797\\
1322	1.83308546624763\\
1323	1.92860021277753\\
1324	1.96987612553902\\
1325	1.94407367345058\\
1326	1.82148681139616\\
1327	1.58634071762336\\
1328	1.4516716026449\\
1329	1.48433081978454\\
1330	1.52435060148051\\
1331	1.56482778914137\\
1332	1.55894405155434\\
1333	1.55291902719336\\
1334	1.52759096941009\\
1335	1.52278695974096\\
1336	1.49419614550913\\
1337	1.47481670109982\\
1338	1.42302706677315\\
1339	1.3852507082783\\
1340	1.39280115762667\\
1341	1.47150274418752\\
1342	1.49513474110871\\
1343	1.50841716221653\\
1344	1.525557710434\\
1345	1.59360655822387\\
1346	1.68621000822805\\
1347	1.7324332833404\\
1348	1.75403959686966\\
1349	1.72959305903211\\
1350	1.64206705515076\\
1351	1.46058117212\\
1352	1.40086565834898\\
1353	1.40356664142018\\
1354	1.37432197351209\\
1355	1.39485641607943\\
1356	1.4127757958776\\
1357	1.45468730151603\\
1358	1.45838102350342\\
1359	1.46411677847241\\
1360	1.44145826443804\\
1361	1.46346584666619\\
1362	1.43813661068\\
1363	1.34657057071765\\
1364	1.33715903934482\\
1365	1.40124385755666\\
1366	1.44231491980583\\
1367	1.42938421110351\\
1368	1.45351774799229\\
1369	1.58138740191801\\
1370	1.67672615186208\\
1371	1.75003893501209\\
1372	1.7999463368322\\
1373	1.78177422217253\\
1374	1.7168898775182\\
1375	1.53226876994737\\
1376	1.46047049375954\\
1377	1.55016230516151\\
1378	1.53381096483716\\
1379	1.51570071552217\\
1380	1.49514090130997\\
1381	1.46458343098298\\
1382	1.43410343242429\\
1383	1.4254082577596\\
1384	1.42084665961113\\
1385	1.41223462718777\\
1386	1.38083500718421\\
1387	1.3524611704329\\
1388	1.34219379276061\\
1389	1.37150598513743\\
1390	1.39924721265692\\
1391	1.38068118253955\\
1392	1.40196038423815\\
1393	1.53443022052717\\
1394	1.62900197407407\\
1395	1.68789366093929\\
1396	1.72018749582986\\
1397	1.69598503345778\\
1398	1.58750606444366\\
1399	1.40888645852836\\
1400	1.38150107632478\\
1401	1.40453323949878\\
1402	1.37938233679319\\
1403	1.34109685051927\\
1404	1.31970945985732\\
1405	1.30821381759089\\
1406	1.30131621996911\\
1407	1.31146954924539\\
1408	1.31744114181831\\
1409	1.33283286992557\\
1410	1.32136484552733\\
1411	1.32673567568304\\
1412	1.33547790600582\\
1413	1.35427568205696\\
1414	1.41551005276263\\
1415	1.36807806449716\\
1416	1.37530124004129\\
1417	1.50168182683362\\
1418	1.59919907420464\\
1419	1.67558168973102\\
1420	1.71702768699532\\
1421	1.72778884535952\\
1422	1.72628401587013\\
1423	1.66516483714779\\
1424	1.62587546375015\\
1425	1.61584059243753\\
1426	1.58094368368277\\
1427	1.54991786658728\\
1428	1.53441076548761\\
1429	1.48038882631661\\
1430	1.5178678587332\\
1431	1.55305750163302\\
1432	1.54358904734156\\
1433	1.55809171406616\\
1434	1.54138039413382\\
1435	1.46920816632231\\
1436	1.46867907159527\\
1437	1.54470153094711\\
1438	1.56091513706473\\
1439	1.53123459730367\\
1440	1.5202089901251\\
1441	1.59507679777851\\
1442	1.69391979925913\\
1443	1.77004241593774\\
1444	1.82562484134006\\
1445	1.85078851097035\\
1446	1.85846245218051\\
1447	1.83123107282825\\
1448	1.85596066594132\\
1449	1.94614240362187\\
1450	1.95439267279642\\
1451	2.0162418271528\\
1452	2.0495945739848\\
1453	2.11247447490105\\
1454	2.20363672188879\\
1455	2.24523750606185\\
1456	2.15487445730973\\
1457	2.02541537529442\\
1458	1.87220184323161\\
1459	1.8099706619807\\
1460	1.75205688404772\\
1461	1.68059939721125\\
1462	1.75945674734015\\
1463	1.74431498905925\\
1464	1.74905871448169\\
1465	1.86794069700398\\
1466	1.99117587025208\\
1467	2.07166175494405\\
1468	2.10029676957498\\
1469	2.0574036182004\\
1470	1.87453530116228\\
1471	1.63672415725931\\
1472	1.58548761528739\\
1473	1.54535297946052\\
1474	1.54907060445952\\
1475	1.55219104674505\\
1476	1.55324063961403\\
1477	1.60710110950456\\
1478	1.58358759005959\\
1479	1.55924947305533\\
1480	1.54052679571469\\
1481	1.516130193406\\
1482	1.44536937598167\\
1483	1.43815502854824\\
1484	1.42050435975442\\
1485	1.46217656232679\\
1486	1.50563157807399\\
1487	1.45940346329604\\
1488	1.45625588716567\\
1489	1.51921781955863\\
1490	1.59530272691583\\
1491	1.65566292720738\\
1492	1.66716210392124\\
1493	1.63366374585093\\
1494	1.54931815976876\\
1495	1.41246795665539\\
1496	1.3608661082163\\
1497	1.37959752525329\\
1498	1.36994551408288\\
1499	1.37178802752807\\
1500	1.37133145141073\\
1501	1.3932283232757\\
1502	1.39840905176265\\
1503	1.38499570788644\\
1504	1.36463223234458\\
1505	1.35778212231777\\
1506	1.30438025078657\\
1507	1.30175930997948\\
1508	1.28240855959225\\
1509	1.33135790458699\\
1510	1.35951327912801\\
1511	1.33088752828115\\
1512	1.342391139714\\
1513	1.43289680289457\\
1514	1.50498245090555\\
1515	1.56782241172476\\
1516	1.60466047494602\\
1517	1.58679205182118\\
1518	1.51876461029249\\
1519	1.37924235170322\\
1520	1.33039402821076\\
1521	1.3467614928955\\
1522	1.3780063479666\\
1523	1.38433063631076\\
1524	1.40269719637695\\
1525	1.44315335939091\\
1526	1.45084889138621\\
1527	1.45034427293121\\
1528	1.41940821161614\\
1529	1.40702302933709\\
1530	1.35243138606194\\
1531	1.3750792919581\\
1532	1.31808484248431\\
1533	1.3071664277574\\
1534	1.35908917568277\\
1535	1.33758717364877\\
1536	1.36848470516466\\
1537	1.44748587889809\\
1538	1.53734398499614\\
1539	1.64070652290812\\
1540	1.66447559468948\\
1541	1.65754605684342\\
1542	1.56999054904156\\
1543	1.43844868792774\\
1544	1.46128237330689\\
1545	1.47183338338961\\
1546	1.52462883813733\\
1547	1.48498313811263\\
1548	1.52963159133752\\
1549	1.58118044932827\\
1550	1.60220806681203\\
1551	1.61560514922772\\
1552	1.56428594459024\\
1553	1.52454022558502\\
1554	1.4306329873745\\
1555	1.41750683392254\\
1556	1.37367168967247\\
1557	1.36594801328238\\
1558	1.45631697971245\\
1559	1.44573644002731\\
1560	1.47489564176708\\
1561	1.62429619946677\\
1562	1.75563355881954\\
1563	1.83242635719538\\
1564	1.85746755308446\\
1565	1.81079859515086\\
1566	1.72942121047204\\
1567	1.5514024310081\\
1568	1.57275683891228\\
1569	1.57451989411418\\
1570	1.57044797064252\\
1571	1.54331330397133\\
1572	1.56869765284999\\
1573	1.61762568556321\\
1574	1.61594345964667\\
1575	1.58556763611253\\
1576	1.55275106936093\\
1577	1.52228056295366\\
1578	1.45817308710399\\
1579	1.45906048934699\\
1580	1.42562645991018\\
1581	1.40727641957975\\
1582	1.48243487965044\\
1583	1.45595926221496\\
1584	1.45482759021898\\
1585	1.55796116279909\\
1586	1.67982594510891\\
1587	1.79965601897131\\
1588	1.85857412869265\\
1589	1.87814112660555\\
1590	1.85735614877135\\
1591	1.8137446257556\\
1592	1.81673758816237\\
1593	1.76946495946296\\
1594	1.77145999668182\\
1595	1.84168713422355\\
1596	1.90583768608713\\
1597	1.92096984840545\\
1598	1.98665755001086\\
1599	2.00126782872708\\
1600	1.93035282183811\\
1601	1.83997915583286\\
1602	1.71090617574316\\
1603	1.59501272086269\\
1604	1.63727209576367\\
1605	1.73746729081917\\
1606	1.80149809904322\\
1607	1.73936625913349\\
1608	1.75815772123314\\
1609	1.889125721908\\
1610	1.99648928035087\\
1611	2.09178559434901\\
1612	2.10224433036551\\
1613	2.12153766438367\\
1614	2.14307574426379\\
1615	2.03818677033229\\
1616	2.06879852842859\\
1617	2.07258354101384\\
1618	2.05514477123435\\
1619	2.10928573639396\\
1620	2.11430343264358\\
1621	2.12668447545066\\
1622	2.2223374708055\\
1623	2.27960964757037\\
1624	2.17608798243745\\
1625	2.09466618423347\\
1626	1.92733995660852\\
1627	1.76677144041151\\
1628	1.66081583639286\\
1629	1.68424303642453\\
1630	1.72863444219807\\
1631	1.72047354995512\\
1632	1.69798580893511\\
1633	1.7573074431079\\
1634	1.83454350168685\\
1635	1.8759460895251\\
1636	1.89081150032069\\
1637	1.84736403073938\\
1638	1.69439280796366\\
1639	1.46090413248887\\
1640	1.44259810938236\\
1641	1.46252539770256\\
1642	1.49768144745225\\
1643	1.49464022591201\\
1644	1.50765745379114\\
1645	1.57525491826685\\
1646	1.59259902917661\\
1647	1.57555918127084\\
1648	1.54754161563465\\
1649	1.50573011635047\\
1650	1.41748685900654\\
1651	1.35382705508541\\
1652	1.38603411627146\\
1653	1.42827621005414\\
1654	1.47936141172255\\
1655	1.47484564850755\\
1656	1.50490501093331\\
1657	1.67978292973052\\
1658	1.7981577212744\\
1659	1.84470118513926\\
1660	1.84672085351022\\
1661	1.82153524846446\\
1662	1.70754224882545\\
1663	1.51158281735357\\
1664	1.50251572595272\\
1665	1.4722029846499\\
1666	1.4606812537432\\
1667	1.38784862406108\\
1668	1.41389564012361\\
1669	1.49860619408095\\
1670	1.5636632955075\\
1671	1.61789955546484\\
1672	1.58629574662685\\
1673	1.54833784102172\\
1674	1.46721619480739\\
1675	1.4008898023453\\
1676	1.4107517345248\\
1677	1.41050519916173\\
1678	1.49491242404481\\
1679	1.49032547220823\\
1680	1.49709869112513\\
1681	1.62459376572139\\
1682	1.71348889133602\\
1683	1.74860267977442\\
1684	1.76739178464789\\
1685	1.72817095292387\\
1686	1.61784194857867\\
1687	1.43723725282999\\
1688	1.44762286715377\\
1689	1.4742994115643\\
1690	1.4193922585723\\
1691	1.49056537934615\\
1692	1.53343044564681\\
1693	1.57964579307061\\
1694	1.58458509917406\\
1695	1.59603181550982\\
1696	1.55532110478149\\
1697	1.50231737394992\\
1698	1.41871205569991\\
1699	1.39938652586993\\
1700	1.34189301702986\\
1701	1.38007774878638\\
1702	1.40720814113439\\
1703	1.39591370423512\\
1704	1.45267068634472\\
1705	1.55608494773627\\
1706	1.63550364731917\\
1707	1.70027730810307\\
1708	1.71612924848952\\
1709	1.67414420269927\\
1710	1.58879508325501\\
1711	1.39725264495193\\
1712	1.34934101068217\\
1713	1.36806039323772\\
1714	1.41053448082051\\
1715	1.44337589280634\\
1716	1.48571185434296\\
1717	1.53365149349081\\
1718	1.54148051801797\\
1719	1.54027966659052\\
1720	1.53708589607791\\
1721	1.49447594209228\\
1722	1.43938879701447\\
1723	1.42703874132109\\
1724	1.35914291202739\\
1725	1.38285705858426\\
1726	1.42253724482949\\
1727	1.4131204135611\\
1728	1.4560139904295\\
1729	1.55890674253538\\
1730	1.63968905791443\\
1731	1.69579117670868\\
1732	1.71194400437713\\
1733	1.69625089320463\\
1734	1.60662952037256\\
1735	1.43447022362948\\
1736	1.4289854354799\\
1737	1.4488415761731\\
1738	1.49761255845512\\
1739	1.45615943305603\\
1740	1.48639204715042\\
1741	1.54486715584096\\
1742	1.58661335982603\\
1743	1.61318336238898\\
1744	1.58403227864481\\
1745	1.52612286033176\\
1746	1.45793657418767\\
1747	1.41480591744411\\
1748	1.39575256811726\\
1749	1.43059351379548\\
1750	1.41988541097448\\
1751	1.4161615610348\\
1752	1.45388974709468\\
1753	1.63695102681476\\
1754	1.73040313815786\\
1755	1.80297173784277\\
1756	1.85460222516377\\
1757	1.86971670899039\\
1758	1.87577281153016\\
1759	1.79371222171891\\
1760	1.75440137238114\\
1761	1.65395889676411\\
1762	1.56832947627\\
1763	1.56956781480838\\
1764	1.56037696387873\\
1765	1.57407886400857\\
1766	1.62979009579208\\
1767	1.69591165327552\\
1768	1.70376595381493\\
1769	1.67967940519809\\
1770	1.60789575288788\\
1771	1.54692549774427\\
1772	1.50221644691561\\
1773	1.54286511280289\\
1774	1.6119384417947\\
1775	1.62466493238114\\
1776	1.68238935792879\\
1777	1.7716261324079\\
1778	1.89617713467987\\
1779	2.00149403133383\\
1780	1.99146210053675\\
1781	2.00035429406344\\
1782	2.0151633650197\\
1783	1.94455241046621\\
1784	1.9756084817401\\
1785	1.89417172990412\\
1786	1.81842054568902\\
1787	1.77107591621315\\
1788	1.77099075145364\\
1789	1.79406632340052\\
1790	1.88447332974286\\
1791	1.98023216388053\\
1792	1.9749635871867\\
1793	1.93109558246915\\
1794	1.77785467034383\\
1795	1.65937214111547\\
1796	1.51796872338027\\
1797	1.51907720118241\\
1798	1.6061512061118\\
1799	1.61089500340504\\
1800	1.61359189767765\\
1801	1.74477292406613\\
1802	1.81596505981108\\
1803	1.87428337693944\\
1804	1.88650880602629\\
1805	1.82925161869589\\
1806	1.69887468305512\\
1807	1.46081176752934\\
1808	1.40905385442238\\
1809	1.41698212611693\\
1810	1.38726289260196\\
1811	1.32953623371697\\
1812	1.31625068922819\\
1813	1.37406028610578\\
1814	1.38674306413162\\
1815	1.38915323755735\\
1816	1.37112887059202\\
1817	1.34990954387465\\
1818	1.28264094207133\\
1819	1.2948907869418\\
1820	1.26619633130976\\
1821	1.31417235963518\\
1822	1.35149569338848\\
1823	1.36351366194984\\
1824	1.40257153346212\\
1825	1.513730380931\\
1826	1.61184346581024\\
1827	1.69042144381051\\
1828	1.70696220373394\\
1829	1.68995510552026\\
1830	1.60962686805314\\
1831	1.41206296728\\
1832	1.33076296333963\\
1833	1.35415638685079\\
1834	1.33511776681012\\
1835	1.36880960651899\\
1836	1.33416931831329\\
1837	1.35087188155977\\
1838	1.35529756790403\\
1839	1.38873404053547\\
1840	1.40624836752587\\
1841	1.40640025527753\\
1842	1.37249567976758\\
1843	1.36457179369257\\
1844	1.38659718123026\\
1845	1.43022767252236\\
1846	1.48702558171287\\
1847	1.47069785490974\\
1848	1.51691200035896\\
1849	1.65776043623007\\
1850	1.77343192701197\\
1851	1.82962263700183\\
1852	1.81513112737509\\
1853	1.76217233279003\\
1854	1.63928394254754\\
1855	1.44541225264629\\
1856	1.34912929828007\\
1857	1.45550972294685\\
1858	1.46969093720202\\
1859	1.38295157817602\\
1860	1.39391058278457\\
1861	1.43253194617518\\
1862	1.46235390781029\\
1863	1.48778842532059\\
1864	1.47081604193697\\
1865	1.44189731502055\\
1866	1.35027618622874\\
1867	1.37984530549828\\
1868	1.31642659283041\\
1869	1.33257362343575\\
1870	1.33077043969421\\
1871	1.40860751743824\\
1872	1.45810393898916\\
1873	1.55666609183253\\
1874	1.67740201521435\\
1875	1.76195674399085\\
1876	1.76303645233266\\
1877	1.74590758165143\\
1878	1.71498270705072\\
1879	1.5642467433521\\
1880	1.48086815545516\\
1881	1.63186999945816\\
1882	1.64552008873262\\
1883	1.66812570587663\\
1884	1.65545381804819\\
1885	1.73524766941985\\
1886	1.85360323686218\\
1887	1.87011244689236\\
1888	1.86027920708547\\
1889	1.70774128933448\\
1890	1.60462052134116\\
1891	1.59016896340545\\
1892	1.52660433873764\\
1893	1.47526706351551\\
1894	1.52803785843894\\
1895	1.54016786165953\\
1896	1.58028571960507\\
1897	1.73722323833139\\
1898	1.85400486933866\\
1899	1.914012621466\\
1900	1.89844945688985\\
1901	1.83326394187926\\
1902	1.70799935722249\\
1903	1.49678027829448\\
1904	1.3866625615625\\
1905	1.37372337893363\\
1906	1.35298081420226\\
1907	1.33920660534979\\
1908	1.29387807222542\\
1909	1.32512290018497\\
1910	1.36154073707651\\
1911	1.41851612703381\\
1912	1.44218756656511\\
1913	1.44384577123015\\
1914	1.42282456896212\\
1915	1.47825970224873\\
1916	1.42451614475714\\
1917	1.455640042439\\
1918	1.45528664248027\\
1919	1.44711405815108\\
1920	1.47201243838848\\
1921	1.62427523638119\\
1922	1.77139730908032\\
1923	1.87642804588696\\
1924	1.93309496789292\\
1925	1.96368739176182\\
1926	1.97087770099373\\
1927	1.94297786447001\\
1928	1.92995087174754\\
1929	1.87838890952762\\
1930	1.7792711013168\\
1931	1.70495593015082\\
1932	1.69857004074966\\
1933	1.68734060669256\\
1934	1.73386443571451\\
1935	1.77158333940517\\
1936	1.78734876680459\\
1937	1.74923622113497\\
1938	1.70176367764267\\
1939	1.73137680924973\\
1940	1.61287202835278\\
1941	1.67513699926892\\
1942	1.62295451078069\\
1943	1.62451691787511\\
1944	1.61397209209273\\
1945	1.68504432476538\\
1946	1.79372126494185\\
1947	1.87793708267298\\
1948	1.92803420941212\\
1949	1.93971387575075\\
1950	1.88929353978743\\
1951	1.86741046636232\\
1952	1.88782585664846\\
1953	1.86843253941805\\
1954	1.78787893189809\\
1955	1.76739587526286\\
1956	1.72022086450511\\
1957	1.68343823817002\\
1958	1.77387747410252\\
1959	1.84814932559063\\
1960	1.81449959543903\\
1961	1.73988279226465\\
1962	1.63993718597877\\
1963	1.54325771283558\\
1964	1.54941545471038\\
1965	1.54685819726808\\
1966	1.59180220585576\\
1967	1.61389201025443\\
1968	1.53186214711856\\
1969	1.6562602399322\\
1970	1.73973087426389\\
1971	1.77469881772308\\
1972	1.80224551631018\\
1973	1.74198639501277\\
1974	1.61683462525867\\
1975	1.42518622353491\\
1976	1.33854902225415\\
1977	1.41771661631\\
1978	1.43249233844722\\
1979	1.43883217883858\\
1980	1.4197472322135\\
1981	1.44705159016528\\
1982	1.44803986314577\\
1983	1.45744461705446\\
1984	1.41842554060967\\
1985	1.39739711782646\\
1986	1.32792677382235\\
1987	1.3465854806629\\
1988	1.28873378215774\\
1989	1.31291499090452\\
1990	1.36914038505421\\
1991	1.33957901983481\\
1992	1.38285870321052\\
1993	1.50137579804359\\
1994	1.58880382475931\\
1995	1.64586414667484\\
1996	1.66845089465294\\
1997	1.64475240938137\\
1998	1.54337933980264\\
1999	1.39240325192606\\
2000	1.41530942843544\\
2001	1.36886325024338\\
2002	1.3991903815184\\
2003	1.4461309881888\\
2004	1.46731355563547\\
2005	1.49913292312822\\
2006	1.49346677469573\\
2007	1.4818674447843\\
2008	1.44227426169692\\
2009	1.41599871637536\\
2010	1.37803380494369\\
2011	1.33580948086746\\
2012	1.32973233531299\\
2013	1.34299186609224\\
2014	1.40885021948812\\
2015	1.43363026919636\\
2016	1.44849030411333\\
2017	1.53298500377149\\
2018	1.6213728883071\\
2019	1.6829265434353\\
2020	1.71262939401528\\
2021	1.6750984709377\\
2022	1.55257374226178\\
2023	1.38995897042979\\
2024	1.37186886176007\\
2025	1.39993810395868\\
2026	1.40729234965379\\
2027	1.39500260148793\\
2028	1.39210137607535\\
2029	1.41152915559117\\
2030	1.29528327554444\\
2031	1.28415362028209\\
2032	1.25524114061316\\
2033	1.23094160661644\\
2034	1.18621696033999\\
2035	1.18240410194899\\
2036	1.08190854109592\\
2037	1.0959119559347\\
2038	1.1110907052666\\
2039	1.10803684505969\\
2040	1.11870695774836\\
2041	1.20388727638715\\
2042	1.28345400580662\\
2043	1.33579816205302\\
2044	1.35753965449947\\
2045	1.34846482013577\\
2046	1.27394573378156\\
2047	1.14935677581808\\
2048	1.11481004433171\\
2049	1.11957739316284\\
2050	1.12774296246977\\
2051	1.12200784757781\\
2052	1.14358709638932\\
2053	1.20029010197787\\
2054	1.192008986553\\
2055	1.18905711322673\\
2056	1.15910108325716\\
2057	1.13247250461958\\
2058	1.06861594711045\\
2059	1.02435859472313\\
2060	1.04605749810883\\
2061	1.06953696383025\\
2062	1.11499742408368\\
2063	1.10356003400873\\
2064	1.12421519946254\\
2065	1.20550297220782\\
2066	1.27156801943399\\
2067	1.32379843934259\\
2068	1.34476083523002\\
2069	1.33067384790344\\
2070	1.25564244415345\\
2071	1.13540613306394\\
2072	1.12897895652548\\
2073	1.14262546619346\\
2074	1.14315007029485\\
2075	1.18076292552699\\
2076	1.16168889149222\\
2077	1.17463786893323\\
2078	1.16826641675761\\
2079	1.19778965302969\\
2080	1.19973839331625\\
2081	1.19193636341114\\
2082	1.15125961691329\\
2083	1.11459134180412\\
2084	1.14897550548535\\
2085	1.13939546387112\\
2086	1.16865299055247\\
2087	1.16855593726436\\
2088	1.20441153324046\\
2089	1.31019504851979\\
2090	1.43017588308937\\
2091	1.49651417678919\\
2092	1.52075077256715\\
2093	1.515327431826\\
2094	1.49482567213735\\
2095	1.46602043876633\\
2096	1.43641650361988\\
2097	1.40867863395091\\
2098	1.40984645130894\\
2099	1.45200760216002\\
2100	1.49058486185025\\
2101	1.52617497497367\\
2102	1.57058733074416\\
2103	1.60690511993328\\
2104	1.5511024605411\\
2105	1.49981354137923\\
2106	1.40400515963621\\
2107	1.33345929458362\\
2108	1.35794117531352\\
2109	1.28065221816755\\
2110	1.35855488120386\\
2111	1.36621796861262\\
2112	1.37540271279591\\
2113	1.48131222245934\\
2114	1.53520784111136\\
2115	1.58602173467951\\
2116	1.61765212465098\\
2117	1.5973120276463\\
2118	1.56232629722312\\
2119	1.55303591365258\\
2120	1.52180530071842\\
2121	1.49047009408056\\
2122	1.50646127778187\\
2123	1.53953291035341\\
2124	1.57175875396113\\
2125	1.66050574526957\\
2126	1.71470279432474\\
2127	1.70445121349626\\
2128	1.66397360886721\\
2129	1.5476028960032\\
2130	1.49356049276508\\
2131	1.38743658983168\\
2132	1.31340688851364\\
2133	1.36124882512449\\
2134	1.32843669503136\\
2135	1.40977084844161\\
2136	1.45729178838005\\
2137	1.49614580710222\\
2138	1.53040319507457\\
2139	1.53251453491788\\
2140	1.49887061218682\\
2141	1.3946010972877\\
2142	1.21411778591768\\
2143	1.116401845099\\
2144	1.15250902237845\\
2145	1.17246909136943\\
2146	1.17265142155019\\
2147	1.21818821656209\\
2148	1.29386290766152\\
2149	1.3053255586896\\
2150	1.30080585202713\\
2151	1.2917248429784\\
2152	1.26136541747359\\
2153	1.20162832028489\\
2154	1.15252295848605\\
2155	1.18468043273594\\
2156	1.0998735026785\\
2157	1.14642716719058\\
2158	1.16865406271068\\
2159	1.23024645139652\\
2160	1.25362231827604\\
2161	1.33211962576126\\
2162	1.34698380147525\\
2163	1.36215022747716\\
2164	1.33600244187693\\
2165	1.29916447238862\\
2166	1.15194948808992\\
2167	1.11001375524118\\
2168	1.14488673052728\\
2169	1.15313012647366\\
2170	1.16079152357812\\
2171	1.16449684309263\\
2172	1.21238486835621\\
2173	1.21192453892436\\
2174	1.21235838656915\\
2175	1.20110650988779\\
2176	1.17383139258003\\
2177	1.1299454653307\\
2178	1.10273476349998\\
2179	1.13038853649776\\
2180	1.1099338367366\\
2181	1.10920691776337\\
2182	1.11015718993717\\
2183	1.15144718553193\\
2184	1.25921382441375\\
2185	1.34552152785436\\
2186	1.41173597875105\\
2187	1.42964089933858\\
2188	1.42990686604993\\
2189	1.35415286007748\\
2190	1.18344299113468\\
2191	1.1310611122114\\
2192	1.14664927399403\\
2193	1.15261377398376\\
2194	1.15189778975754\\
2195	1.16844673333761\\
2196	1.1856195638371\\
2197	1.1941319571991\\
2198	1.20821120391057\\
2199	1.2215083603636\\
2200	1.20930524558915\\
2201	1.17409309155889\\
2202	1.14103215067002\\
2203	1.13545361352693\\
2204	1.13161443864014\\
2205	1.12579861085768\\
2206	1.14445226780974\\
2207	1.20138378661713\\
2208	1.30723863397228\\
2209	1.39188712462851\\
2210	1.45251254156875\\
2211	1.45986604251502\\
2212	1.47227118092665\\
2213	1.39934378319654\\
2214	1.23844028773193\\
2215	1.15869458417374\\
2216	1.16949786960425\\
2217	1.17979775915639\\
2218	1.19823130494182\\
2219	1.18753840368005\\
2220	1.2250886064538\\
2221	1.22382026913614\\
2222	1.22585497227867\\
2223	1.23222683499767\\
2224	1.20314067917292\\
2225	1.14961444781581\\
2226	1.10736311791139\\
2227	1.08797243137346\\
2228	1.10002363382955\\
2229	1.11953994464978\\
2230	1.14611233540154\\
2231	1.17381456324535\\
2232	1.2977478906328\\
2233	1.36016548490334\\
2234	1.42509786365864\\
2235	1.42572573009011\\
2236	1.40517902888599\\
2237	1.32597586857305\\
2238	1.18271299065312\\
2239	1.10802290768129\\
2240	1.08731028979321\\
2241	1.08401313491717\\
2242	1.08067338329244\\
2243	1.0680488126064\\
2244	1.1131930311788\\
2245	1.1393891337524\\
2246	1.15816461818544\\
2247	1.18205908237356\\
2248	1.17642841463357\\
2249	1.140908849017\\
2250	1.12550155873831\\
2251	1.15847280999027\\
2252	1.14592203534015\\
2253	1.1755330712329\\
2254	1.1464245299415\\
2255	1.16997837449482\\
2256	1.29697677213853\\
2257	1.37378687541612\\
2258	1.43806431148202\\
2259	1.46098595064787\\
2260	1.46717699397583\\
2261	1.43343890518131\\
2262	1.37333557445879\\
2263	1.34218151070145\\
2264	1.27716739116472\\
2265	1.27732974659422\\
2266	1.27764520201606\\
2267	1.27095159630006\\
2268	1.30034698258039\\
2269	1.36041604747814\\
2270	1.40725192248557\\
2271	1.41701527624907\\
2272	1.41124911975999\\
2273	1.37816287363181\\
2274	1.30485957530233\\
2275	1.32077692006871\\
2276	1.29343242075199\\
2277	1.25842041112445\\
2278	1.28668748016478\\
2279	1.30099890388597\\
2280	1.39556208281814\\
2281	1.5087290453171\\
2282	1.59114428559744\\
2283	1.6301729105654\\
2284	1.66097522562648\\
2285	1.64443779745848\\
2286	1.59584227386098\\
2287	1.61554922888718\\
2288	1.56978848943736\\
2289	1.54249362260821\\
2290	1.52615539719417\\
2291	1.48310846470841\\
2292	1.42417189743338\\
2293	1.47952096023\\
2294	1.50532579406946\\
2295	1.52022621244826\\
2296	1.51645721091484\\
2297	1.45170418964588\\
2298	1.37416633057681\\
2299	1.35027105963606\\
2300	1.33165496668819\\
2301	1.3283142329384\\
2302	1.3444170015292\\
2303	1.37678338153215\\
2304	1.46473610788984\\
2305	1.54421205665638\\
2306	1.59932854886974\\
2307	1.60793414190795\\
2308	1.57665388746556\\
2309	1.47174786592852\\
2310	1.29400885220314\\
2311	1.25523808863086\\
2312	1.18286109221035\\
2313	1.1681242605485\\
2314	1.15480082543752\\
2315	1.16004771549767\\
2316	1.211633796579\\
2317	1.20768841413582\\
2318	1.21984005454083\\
2319	1.21419095067177\\
2320	1.21761609855799\\
2321	1.19308111030553\\
2322	1.21677834888146\\
2323	1.22214340779357\\
2324	1.20742662800161\\
2325	1.18180087565348\\
2326	1.19656697335478\\
2327	1.26069390956358\\
2328	1.39497790571682\\
2329	1.50023165710484\\
2330	1.56544020403265\\
2331	1.56901095728355\\
2332	1.53259375725895\\
2333	1.42787275750208\\
2334	1.27413302687551\\
2335	1.23941383329599\\
2336	1.19917492384489\\
2337	1.21103994948329\\
2338	1.2255178326586\\
2339	1.20933794822475\\
2340	1.26732308358186\\
2341	1.29611964690236\\
2342	1.30383912423664\\
2343	1.28165410849572\\
2344	1.2533760913301\\
2345	1.2015410388857\\
2346	1.23596112182939\\
2347	1.20691228716618\\
2348	1.11549810896798\\
2349	1.11125423281417\\
2350	1.10636252310033\\
2351	1.14376279668373\\
2352	1.22712499281646\\
2353	1.29491079146687\\
2354	1.33190399029072\\
2355	1.34773059361839\\
2356	1.32938663110779\\
2357	1.24264483112728\\
2358	1.13293049534791\\
2359	1.11267699175638\\
2360	1.10057644635096\\
2361	1.09978611589633\\
2362	1.12602394011848\\
2363	1.13414669067561\\
2364	1.19785898931502\\
2365	1.20835748683161\\
2366	1.24089396469019\\
2367	1.23911736496541\\
2368	1.21392546699658\\
2369	1.1713382388073\\
2370	1.1328321329232\\
2371	1.13512934890657\\
2372	1.14831144470472\\
2373	1.11885250293989\\
2374	1.12834542423324\\
2375	1.13471019761957\\
2376	1.2139496538016\\
2377	1.27680322990487\\
2378	1.31740345964107\\
2379	1.33742271042549\\
2380	1.3220500545829\\
2381	1.25009340761784\\
2382	1.12937306303011\\
2383	1.08872555154845\\
2384	1.0859101409649\\
2385	1.09983895749211\\
2386	1.11292667325356\\
2387	1.10472141941404\\
2388	1.16518225274952\\
2389	1.16092851526386\\
2390	1.16820609957146\\
2391	1.16529179925058\\
2392	1.15591962091945\\
2393	1.11789864764514\\
2394	1.09138081307153\\
2395	1.12954848815279\\
2396	1.11476865433246\\
2397	1.10266350896588\\
2398	1.11299973377227\\
2399	1.13232597274871\\
2400	1.23596810017628\\
2401	1.30312620252703\\
2402	1.35557894683369\\
2403	1.35933114394581\\
2404	1.35076964330021\\
2405	1.2754276065423\\
2406	1.17084074956832\\
2407	1.12778048488893\\
2408	1.10816412634679\\
2409	1.11599554095574\\
2410	1.1227061693306\\
2411	1.13464506371662\\
2412	1.16594656474901\\
2413	1.19708970858241\\
2414	1.21722249427645\\
2415	1.20561389513338\\
2416	1.20463798048718\\
2417	1.18566024069001\\
2418	1.16225154205194\\
2419	1.17610162869819\\
2420	1.1683460842755\\
2421	1.15260895809329\\
2422	1.15072053200747\\
2423	1.19311220132861\\
2424	1.27377917554457\\
2425	1.32528548355039\\
2426	1.38071594944198\\
2427	1.39911935794532\\
2428	1.39425816247011\\
2429	1.38433294778029\\
2430	1.3538383965631\\
2431	1.33107747473093\\
2432	1.2774136360115\\
2433	1.24581401493381\\
2434	1.262437634681\\
2435	1.31408767459794\\
2436	1.37146152088835\\
2437	1.44419537863653\\
2438	1.47896079596764\\
2439	1.46626633746875\\
2440	1.42568565183467\\
2441	1.36225250767165\\
2442	1.30713130162413\\
2443	1.28644698761417\\
2444	1.32456780771198\\
2445	1.28844063274588\\
2446	1.29210304011601\\
2447	1.3168093075537\\
2448	1.54569738496007\\
2449	1.62736138347947\\
2450	1.7045923436738\\
2451	1.72674686779652\\
2452	1.7340870591381\\
2453	1.70518596964295\\
2454	1.65752649389264\\
2455	1.64511110629238\\
2456	1.57916352005726\\
2457	1.58175052099711\\
2458	1.60050300250743\\
2459	1.5820747930671\\
2460	1.61063580045866\\
2461	1.67252834397789\\
2462	1.77702014454312\\
2463	1.79703931306778\\
2464	1.76590888932577\\
2465	1.66667138274024\\
2466	1.5449015580938\\
2467	1.46074089378204\\
2468	1.43803710081372\\
2469	1.38838094894339\\
2470	1.40372923548479\\
2471	1.45584617975335\\
2472	1.61719600001167\\
2473	1.70775233333337\\
2474	1.75254506910591\\
2475	1.78607547524171\\
2476	1.76725290535933\\
2477	1.64183909110945\\
2478	1.45301256859489\\
2479	1.37636833656826\\
2480	1.33346978661795\\
2481	1.34392089353061\\
2482	1.3434472233046\\
2483	1.34745134594234\\
2484	1.37970314352013\\
2485	1.38316479514204\\
2486	1.41236118578886\\
2487	1.41388764191191\\
2488	1.41335883718781\\
2489	1.35896353754277\\
2490	1.31844429544859\\
2491	1.29305720559928\\
2492	1.30037472447523\\
2493	1.26697672779742\\
2494	1.26727038353837\\
2495	1.28824194297648\\
2496	1.38623373761712\\
2497	1.43934534523005\\
2498	1.48340133081076\\
2499	1.50217478062299\\
2500	1.46297310781972\\
2501	1.39116748016938\\
2502	1.26329772666588\\
2503	1.21895637108433\\
2504	1.1924497042498\\
2505	1.20370512168887\\
2506	1.23334766699142\\
2507	1.24999000590635\\
2508	1.28734932154567\\
2509	1.30151438558564\\
2510	1.31839038732055\\
2511	1.32686232028049\\
2512	1.332024832477\\
2513	1.30247769249356\\
2514	1.25415712774159\\
2515	1.23369692757872\\
2516	1.23312244960995\\
2517	1.22336680697958\\
2518	1.20166523346958\\
2519	1.22887194156125\\
2520	1.31450536298898\\
2521	1.38009457525389\\
2522	1.4359783829818\\
2523	1.44451415574379\\
2524	1.41915123222873\\
2525	1.35278771784686\\
2526	1.23761028834652\\
2527	1.21705014516083\\
2528	1.2466677047671\\
2529	1.26551694863836\\
2530	1.27246142292551\\
2531	1.31104831961575\\
2532	1.33884714159865\\
2533	1.39074561949833\\
2534	1.38985965783256\\
2535	1.37864150280566\\
2536	1.35118644883546\\
2537	1.30043760379654\\
2538	1.23791873695638\\
2539	1.22335178988923\\
2540	1.2150831203055\\
2541	1.22740174990761\\
2542	1.2041427860058\\
2543	1.22626414502963\\
2544	1.3257494308657\\
2545	1.41532957472805\\
2546	1.46862943375303\\
2547	1.49311595766811\\
2548	1.50071439546335\\
2549	1.44818591056557\\
2550	1.32255930881883\\
2551	1.31509650543724\\
2552	1.28811764301072\\
2553	1.29578260450617\\
2554	1.31393769432316\\
2555	1.30654375495691\\
2556	1.37693620950614\\
2557	1.40562418581113\\
2558	1.43701424903589\\
2559	1.45008136909763\\
2560	1.41032666258952\\
2561	1.35243965499214\\
2562	1.31210344809688\\
2563	1.27924474953415\\
2564	1.26669281962999\\
2565	1.26364900979838\\
2566	1.25422801384745\\
2567	1.26574055607666\\
2568	1.37190347980952\\
2569	1.46488785693239\\
2570	1.5218000207516\\
2571	1.55038106770243\\
2572	1.54502494835417\\
2573	1.48823078155585\\
2574	1.35395768790798\\
2575	1.3308065528321\\
2576	1.30948607967116\\
2577	1.32211777319854\\
2578	1.33263929982467\\
2579	1.34597165877552\\
2580	1.36396720694012\\
2581	1.37916767424388\\
2582	1.40989921998949\\
2583	1.44075327336176\\
2584	1.42214830077937\\
2585	1.40468679552339\\
2586	1.36040222312283\\
2587	1.3549085945791\\
2588	1.38480887868092\\
2589	1.33352683146\\
2590	1.29724101831336\\
2591	1.30261583551133\\
2592	1.40713128225155\\
2593	1.50871457425981\\
2594	1.57341550225123\\
2595	1.5973228399871\\
2596	1.57370639269312\\
2597	1.56121004509647\\
2598	1.55418459100415\\
2599	1.52751603309624\\
2600	1.46403494318029\\
2601	1.44210635919353\\
2602	1.44630151224562\\
2603	1.42585724321132\\
2604	1.47737977346363\\
2605	1.57612154921073\\
2606	1.66100078748186\\
2607	1.71211465293857\\
2608	1.67929253862999\\
2609	1.59927165455158\\
2610	1.52396571082729\\
2611	1.48713710546762\\
2612	1.52094339575594\\
2613	1.52911614714488\\
2614	1.54768260481383\\
2615	1.5416127649904\\
2616	1.61856833304569\\
2617	1.74106241600103\\
2618	1.83185623282691\\
2619	1.86828133254571\\
2620	1.89703668379607\\
2621	1.86807388539285\\
2622	1.81196320830395\\
2623	1.80843274479394\\
2624	1.71069176403488\\
2625	1.6479618801787\\
2626	1.64995776787393\\
2627	1.66726745470556\\
2628	1.66794981749934\\
2629	1.70408763318624\\
2630	1.75590622185129\\
2631	1.79656337300872\\
2632	1.79147744249027\\
2633	1.67742110609325\\
2634	1.57899791437189\\
2635	1.49631156078159\\
2636	1.50874887081565\\
2637	1.47504863725732\\
2638	1.46047923775862\\
2639	1.48338298520586\\
2640	1.52516311773139\\
2641	1.60570141587666\\
2642	1.67184350094325\\
2643	1.72181440921284\\
2644	1.72082342892872\\
2645	1.68697727278875\\
2646	1.65485873693513\\
2647	1.67827766747208\\
2648	1.65453975028823\\
2649	1.64363758853879\\
2650	1.63577564464782\\
2651	1.69424943735699\\
2652	1.70098717301376\\
2653	1.78295520099112\\
2654	1.76240185606654\\
2655	1.75210621846416\\
2656	1.68799302566996\\
2657	1.55672892586897\\
2658	1.4538390391255\\
2659	1.42124163222008\\
2660	1.44536597511798\\
2661	1.45961274024156\\
2662	1.41020908243657\\
2663	1.4048154155646\\
2664	1.4804967012563\\
2665	1.57313067912719\\
2666	1.6274398129024\\
2667	1.63627378609345\\
2668	1.60910554437361\\
2669	1.50282722502828\\
2670	1.32299779316529\\
2671	1.22669761984199\\
2672	1.22954978065047\\
2673	1.24780555541982\\
2674	1.25465639791746\\
2675	1.28042838891942\\
2676	1.31514594187815\\
2677	1.31761052018671\\
2678	1.31432434008686\\
2679	1.31722110046228\\
2680	1.29392284056994\\
2681	1.24717700919381\\
2682	1.21800507493495\\
2683	1.20879285070284\\
2684	1.23578407752829\\
2685	1.22996456377151\\
2686	1.21329049008808\\
2687	1.25293507399839\\
2688	1.34865456141543\\
2689	1.43130831560934\\
2690	1.49884483603819\\
2691	1.5024228229863\\
2692	1.4874919077175\\
2693	1.39517921630652\\
2694	1.26401747435107\\
2695	1.2353476057694\\
2696	1.24713172814904\\
2697	1.25815683882602\\
2698	1.26741469348328\\
2699	1.26154593379812\\
2700	1.29229590674641\\
2701	1.28764042680623\\
2702	1.2721902130548\\
2703	1.26205589542907\\
2704	1.24752587886291\\
2705	1.23890993771766\\
2706	1.2031434691523\\
2707	1.20218488132074\\
2708	1.23313693442622\\
2709	1.18176480239868\\
2710	1.19314769892681\\
2711	1.25258946910222\\
2712	1.36125299596497\\
2713	1.43402336888919\\
2714	1.48380376034148\\
2715	1.51524114552537\\
2716	1.49818136240814\\
2717	1.42163242990463\\
2718	1.27702806006731\\
2719	1.20178935662885\\
2720	1.178797053884\\
2721	1.1652975620232\\
2722	1.18551497147898\\
2723	1.16695964491392\\
2724	1.19578442930858\\
2725	1.20275222536312\\
2726	1.22174214673781\\
2727	1.21266160170127\\
2728	1.21030129589833\\
2729	1.19911763530019\\
2730	1.19022154734175\\
2731	1.20828219369449\\
2732	1.23070343073045\\
2733	1.24742136063187\\
2734	1.2824652174485\\
2735	1.33433815408127\\
2736	1.46301780501111\\
2737	1.55630565801715\\
2738	1.6057028590418\\
2739	1.62933942023015\\
2740	1.6141781325833\\
2741	1.53775276113273\\
2742	1.39323826000312\\
2743	1.30925036937679\\
2744	1.33483412649255\\
2745	1.31224352036548\\
2746	1.36642636398326\\
2747	1.39444728635259\\
2748	1.4780944632717\\
2749	1.4666870932021\\
2750	1.42378118244147\\
2751	1.42043143865979\\
2752	1.38587628785267\\
2753	1.34384151492922\\
2754	1.29459083551554\\
2755	1.29432903753025\\
2756	1.30822385187185\\
2757	1.3016902080609\\
2758	1.32136015835214\\
2759	1.35848239855843\\
2760	1.48119278531623\\
2761	1.58008463165729\\
2762	1.66767657818132\\
2763	1.67967815102297\\
2764	1.67730297004474\\
2765	1.67521556707195\\
2766	1.67268258171578\\
2767	1.63718700122009\\
2768	1.57800394495903\\
2769	1.56130106475819\\
2770	1.58562013348031\\
2771	1.61149536855122\\
2772	1.65821998136198\\
2773	1.68843575910541\\
2774	1.70417603672251\\
2775	1.67226143571594\\
2776	1.60621061603248\\
2777	1.52167646694611\\
2778	1.49131732340811\\
2779	1.43301030487338\\
2780	1.44817201370191\\
2781	1.44237868590036\\
2782	1.47691444354499\\
2783	1.49009983499636\\
2784	1.60427812771019\\
2785	1.70244621603674\\
2786	1.78007231815471\\
2787	1.80544104673792\\
2788	1.81311487548213\\
2789	1.80425101087162\\
2790	1.80808164568491\\
2791	1.79133119362592\\
2792	1.73739985306386\\
2793	1.68931564172938\\
2794	1.68320999129273\\
2795	1.66633265017385\\
2796	1.65071332945278\\
2797	1.72126732447105\\
2798	1.73434652428938\\
2799	1.73492416114588\\
2800	1.71420446289097\\
2801	1.6274714455784\\
2802	1.54627627379111\\
2803	1.4931559960267\\
2804	1.49774013818914\\
2805	1.44695407204197\\
2806	1.45828420864959\\
2807	1.47895861899845\\
2808	1.57787051033176\\
2809	1.64429877050676\\
2810	1.69582001933958\\
2811	1.71235534657726\\
2812	1.6707279994851\\
2813	1.54013008368874\\
2814	1.37200233921205\\
2815	1.29868981497081\\
2816	1.28382853347628\\
2817	1.31894654825127\\
2818	1.32458761099919\\
2819	1.26584671240289\\
2820	1.26554290570379\\
2821	1.27102103940332\\
2822	1.27993106051647\\
2823	1.26932686719795\\
2824	1.2543207748956\\
2825	1.24447648554614\\
2826	1.22474699411563\\
2827	1.21185486311818\\
2828	1.23045663192858\\
2829	1.25117558902969\\
2830	1.23885718857785\\
2831	1.29554839643957\\
2832	1.39332762852082\\
2833	1.46314333848957\\
2834	1.51566237110565\\
2835	1.5251588171127\\
2836	1.51705806008034\\
2837	1.44480533272068\\
2838	1.33719038076799\\
2839	1.26532296518722\\
2840	1.21677202227245\\
2841	1.19666266911192\\
2842	1.20940272807806\\
2843	1.18362737453917\\
2844	1.20438516914433\\
2845	1.18370504336814\\
2846	1.18110698296097\\
2847	1.19523799640249\\
2848	1.20085974816424\\
2849	1.21549219083678\\
2850	1.21124063265605\\
2851	1.24462962584249\\
2852	1.26546566979005\\
2853	1.25971113912951\\
2854	1.24892347626245\\
2855	1.32417431005052\\
2856	1.40261835661346\\
2857	1.47859156172959\\
2858	1.52578652882179\\
2859	1.53707213685171\\
2860	1.52280349600958\\
2861	1.44534021338878\\
2862	1.32389088326081\\
2863	1.2465290901564\\
2864	1.23412603175284\\
2865	1.21576758756779\\
2866	1.21904881118622\\
2867	1.2360235560359\\
2868	1.27114026938582\\
2869	1.27871632207873\\
2870	1.30627613491297\\
2871	1.32512585350739\\
2872	1.32816831982406\\
2873	1.30160400881182\\
2874	1.28942760009927\\
2875	1.29449212758005\\
2876	1.30693192822099\\
2877	1.27406527698735\\
2878	1.25841353095422\\
2879	1.30963771720873\\
2880	1.43010610189892\\
2881	1.52477329599021\\
2882	1.58874374482988\\
2883	1.60661241263019\\
2884	1.629757570769\\
2885	1.62853745639297\\
2886	1.65310480793013\\
2887	1.68766117742958\\
2888	1.6631226097365\\
2889	1.64186001712883\\
2890	1.6252510900762\\
2891	1.59061216967792\\
2892	1.55140129452212\\
2893	1.61738490604192\\
2894	1.61419472971525\\
2895	1.61495585383827\\
2896	1.57397704418468\\
2897	1.51098625450491\\
2898	1.43741773661076\\
2899	1.41809226156888\\
2900	1.428802803546\\
2901	1.40151283847672\\
2902	1.35279757587901\\
2903	1.39134562035848\\
2904	1.55456881074965\\
2905	1.63400570030979\\
2906	1.70249452680334\\
2907	1.73018143975509\\
2908	1.73168025654978\\
2909	1.6771247358467\\
2910	1.58843407935424\\
2911	1.57713899770444\\
2912	1.54596078333973\\
2913	1.52267116201845\\
2914	1.55617464973651\\
2915	1.59050857665978\\
2916	1.61239952680814\\
2917	1.70964493965081\\
2918	1.74440909301212\\
2919	1.76507481200469\\
2920	1.71473539823691\\
2921	1.62706576329385\\
2922	1.55550033549874\\
2923	1.52915932865571\\
2924	1.51395064158707\\
2925	1.50451546374351\\
2926	1.46820489367934\\
2927	1.49459043555808\\
2928	1.61781575794469\\
2929	1.72366304261695\\
2930	1.78333966311024\\
2931	1.81346255026161\\
2932	1.7987125301121\\
2933	1.76529966573959\\
2934	1.74601003137112\\
2935	1.68253730217259\\
2936	1.61694097676544\\
2937	1.5957957896605\\
2938	1.61851495369323\\
2939	1.66785373892567\\
2940	1.69630664634076\\
2941	1.75829569578537\\
2942	1.77759551935295\\
2943	1.77757332137522\\
2944	1.72582342555817\\
2945	1.62656786340615\\
2946	1.54097084354168\\
2947	1.52427231962874\\
2948	1.57560345589223\\
2949	1.55240481514119\\
2950	1.53471817787526\\
2951	1.5509424582395\\
2952	1.52351696296232\\
2953	1.61379966329205\\
2954	1.66273556517011\\
2955	1.67793909044954\\
2956	1.67848842832389\\
2957	1.66136948920028\\
2958	1.67549071026775\\
2959	1.65052711144532\\
2960	1.64931262430877\\
2961	1.66064439311481\\
2962	1.68715179093269\\
2963	1.75487471771076\\
2964	1.79954582857799\\
2965	1.86665912446352\\
2966	1.95414966060664\\
2967	1.94143281038642\\
2968	1.82860362202457\\
2969	1.6875560073716\\
2970	1.5384119903671\\
2971	1.46691585024388\\
2972	1.45954491751169\\
2973	1.46860504090133\\
2974	1.44197355214318\\
2975	1.48112909777243\\
2976	1.58900991947209\\
2977	1.6492061331502\\
2978	1.71388445829987\\
2979	1.72470307103565\\
2980	1.68447170322625\\
2981	1.55606142056463\\
2982	1.41841206424193\\
2983	1.29695004162293\\
2984	1.338195700182\\
2985	1.3297561639832\\
2986	1.37611483226351\\
2987	1.43497606678726\\
2988	1.49534379869371\\
2989	1.481761823544\\
2990	1.47382190330478\\
2991	1.47541053270482\\
2992	1.43336632820087\\
2993	1.36241554998217\\
2994	1.31087485207229\\
2995	1.29113743083812\\
2996	1.31682387785546\\
2997	1.32750182583552\\
2998	1.33048173685631\\
2999	1.40861353083552\\
3000	1.56076167455419\\
3001	1.67599651653322\\
3002	1.7294154584597\\
3003	1.74009422597288\\
3004	1.71774984621465\\
3005	1.61692322990636\\
3006	1.47850401254996\\
3007	1.35059384100487\\
3008	1.30459516062067\\
3009	1.26770404482015\\
3010	1.2762124884976\\
3011	1.28134582306326\\
3012	1.34426333785239\\
3013	1.36172003810569\\
3014	1.3887744028308\\
3015	1.38545924027411\\
3016	1.37086953195588\\
3017	1.32714928937876\\
3018	1.34245623679965\\
3019	1.29938150042557\\
3020	1.32426603645872\\
3021	1.35583972344898\\
3022	1.33596022024731\\
3023	1.3859898084714\\
3024	1.50200658174123\\
3025	1.58608762425269\\
3026	1.65562854565878\\
3027	1.66423576880101\\
3028	1.64481730942736\\
3029	1.54808780127294\\
3030	1.42007644774353\\
3031	1.30878371148241\\
3032	1.28166751973344\\
3033	1.29234008311136\\
3034	1.31002587401842\\
3035	1.30427608034439\\
3036	1.31755521035858\\
3037	1.32522874408011\\
3038	1.35854755281493\\
3039	1.38178429816443\\
3040	1.39097311358233\\
3041	1.36434427908645\\
3042	1.33124061005872\\
3043	1.31572314804926\\
3044	1.3297374337493\\
3045	1.32947915426585\\
3046	1.29635457056906\\
3047	1.35059207001644\\
3048	1.50032302888691\\
3049	1.60498624213676\\
3050	1.68662725343702\\
3051	1.70696386801105\\
3052	1.66736813734637\\
3053	1.61344195176821\\
3054	1.40603709271621\\
3055	1.34528064733439\\
3056	1.3221421471203\\
3057	1.32139106253662\\
3058	1.27997927227389\\
3059	1.25407539053781\\
3060	1.28372343744995\\
3061	1.26598023397519\\
3062	1.26862492807164\\
3063	1.26270468159325\\
3064	1.26218266888468\\
3065	1.25059564642472\\
3066	1.25021022269834\\
3067	1.3022864974074\\
3068	1.35508599186767\\
3069	1.3782350388799\\
3070	1.3820788579868\\
3071	1.43163050132078\\
3072	1.58988690472834\\
3073	1.67028533827484\\
3074	1.74015861573471\\
3075	1.73151310362504\\
3076	1.71638138196528\\
3077	1.63018572320245\\
3078	1.4530828198303\\
3079	1.31993906522812\\
3080	1.31742931785501\\
3081	1.32135672946574\\
3082	1.353382866227\\
3083	1.34634730731225\\
3084	1.42737527321303\\
3085	1.48612329758788\\
3086	1.51946319262438\\
3087	1.54306102752882\\
3088	1.53981889928724\\
3089	1.50032024807032\\
3090	1.47062430579373\\
3091	1.46360793638203\\
3092	1.51393992545702\\
3093	1.4852142636451\\
3094	1.39154309617949\\
3095	1.41843282478392\\
3096	1.49342054911632\\
3097	1.60355968662351\\
3098	1.67384770423102\\
3099	1.71152111854656\\
3100	1.73423616609154\\
3101	1.72163099094091\\
3102	1.71159207144974\\
3103	1.64152514705456\\
3104	1.51765348824139\\
3105	1.4716710304345\\
3106	1.44252826235223\\
3107	1.47004942929362\\
3108	1.5083871206395\\
3109	1.54624016048612\\
3110	1.5787221569081\\
3111	1.59213760118659\\
3112	1.6501451685875\\
3113	1.63955618459683\\
3114	1.61104604917153\\
3115	1.61745313446041\\
3116	1.67819384598397\\
3117	1.67334323146827\\
3118	1.63722299352282\\
3119	1.66283991279328\\
3120	1.77691201942842\\
3121	1.93607386954957\\
3122	2.06972624774105\\
3123	2.1043370835617\\
3124	2.12993053724072\\
3125	2.13577251994424\\
3126	2.19042992453352\\
3127	2.15147441283882\\
3128	2.05320727765085\\
3129	2.01139375427059\\
3130	2.01867439192536\\
3131	1.98362000411839\\
3132	2.02836476388767\\
3133	2.08034576463229\\
3134	2.17165314406387\\
3135	2.21223088022835\\
3136	2.14689964238336\\
3137	1.9835123396369\\
3138	1.78882966133314\\
3139	1.75795179311635\\
3140	1.74273403878859\\
3141	1.7152032102684\\
3142	1.66637260408444\\
3143	1.70496535135049\\
3144	1.83336886883493\\
3145	1.88258737399275\\
3146	1.95646249763428\\
3147	1.96413238889372\\
3148	1.88073177761805\\
3149	1.75069824773805\\
3150	1.54580987554458\\
3151	1.46514814535492\\
3152	1.40413632318222\\
3153	1.40051430053381\\
3154	1.37558542282945\\
3155	1.37418369419289\\
3156	1.41724063215675\\
3157	1.37128415405376\\
3158	1.38246983586313\\
3159	1.37225886363751\\
3160	1.3329702583026\\
3161	1.30363658520668\\
3162	1.29416241439231\\
3163	1.27440547797985\\
3164	1.24445558903586\\
3165	1.25528550322493\\
3166	1.23799437553153\\
3167	1.27905789398656\\
3168	1.39224546044384\\
3169	1.50986882628868\\
3170	1.55887661334803\\
3171	1.57809572222256\\
3172	1.5586766377644\\
3173	1.49435135921643\\
3174	1.37123558807571\\
3175	1.34305585762286\\
3176	1.24255569762498\\
3177	1.23537644425507\\
3178	1.29837813822632\\
3179	1.28337902347451\\
3180	1.24083145192207\\
3181	1.29444198365764\\
3182	1.32153674810697\\
3183	1.29774817817401\\
3184	1.30196184405084\\
3185	1.32814280381696\\
3186	1.26683662589449\\
3187	1.27569286217303\\
3188	1.29070143650631\\
3189	1.29198690585418\\
3190	1.24834314431301\\
3191	1.28950288584908\\
3192	1.38272853047242\\
3193	1.46599908062339\\
3194	1.52034396325737\\
3195	1.52995941965078\\
3196	1.50347131047847\\
3197	1.42955984741482\\
3198	1.32049766529779\\
3199	1.25054157981764\\
3200	1.26082302086434\\
3201	1.29941380018177\\
3202	1.28592943344824\\
3203	1.29045615936632\\
3204	1.30169471950475\\
3205	1.33496054951618\\
3206	1.3544019218714\\
3207	1.3639260846577\\
3208	1.37584623837483\\
3209	1.35214859245916\\
3210	1.33242692074987\\
3211	1.32746522494677\\
3212	1.33731681799662\\
3213	1.3263898735602\\
3214	1.26936111348618\\
3215	1.30445716949973\\
3216	1.44246011537526\\
3217	1.49851716302101\\
3218	1.55083629323821\\
3219	1.55366668843999\\
3220	1.53244213027414\\
3221	1.45920226390856\\
3222	1.35022700545099\\
3223	1.28515074242048\\
3224	1.2708480252462\\
3225	1.2329522749346\\
3226	1.27798153475977\\
3227	1.30246231641602\\
3228	1.38889942110729\\
3229	1.43364028199064\\
3230	1.44495080544145\\
3231	1.44600731876367\\
3232	1.40910499713108\\
3233	1.34890346315035\\
3234	1.30676732477788\\
3235	1.29386389211209\\
3236	1.30856256469901\\
3237	1.31006625289483\\
3238	1.26800212298789\\
3239	1.30865292632285\\
3240	1.42724570366435\\
3241	1.50679937714101\\
3242	1.55907504962773\\
3243	1.56906439538011\\
3244	1.54191781281377\\
3245	1.47319600560829\\
3246	1.37062417179471\\
3247	1.27718499194161\\
3248	1.28212003873568\\
3249	1.30493013743004\\
3250	1.32547037469605\\
3251	1.36550828585216\\
3252	1.43192875736553\\
3253	1.4619326759724\\
3254	1.47386320317288\\
3255	1.50279779829202\\
3256	1.46519943905474\\
3257	1.41323725301835\\
3258	1.36612455317384\\
3259	1.34119200998597\\
3260	1.36033342631902\\
3261	1.37056829218819\\
3262	1.35748584674218\\
3263	1.35648293354937\\
3264	1.41080157154587\\
3265	1.50125597383828\\
3266	1.56078635123696\\
3267	1.58762764953674\\
3268	1.61783774733454\\
3269	1.61701021316971\\
3270	1.60827152048674\\
3271	1.54497607805231\\
3272	1.48652498910574\\
3273	1.46826172757546\\
3274	1.53616065520112\\
3275	1.58586872548615\\
3276	1.60254407817294\\
3277	1.68784516402285\\
3278	1.74584272920093\\
3279	1.73444523898619\\
3280	1.65245852744748\\
3281	1.57340246974063\\
3282	1.48832378213865\\
3283	1.43984835709963\\
3284	1.43645397406676\\
3285	1.42232314629383\\
3286	1.44393386121314\\
3287	1.42786570356729\\
3288	1.49746540357466\\
3289	1.63496492738467\\
3290	1.69324669635898\\
3291	1.72066030650189\\
3292	1.74713149984665\\
3293	1.76529860950729\\
3294	1.77427419508714\\
3295	1.73549026407388\\
3296	1.71547203534736\\
3297	1.72431697247761\\
3298	1.73853371122601\\
3299	1.77535380920386\\
3300	1.820779951402\\
3301	1.89047697451506\\
3302	1.92386840185119\\
3303	1.93568387580577\\
3304	1.85495539768721\\
3305	1.7691041369076\\
3306	1.63628723145542\\
3307	1.59375246503559\\
3308	1.57118432292419\\
3309	1.54559225738105\\
3310	1.51385771298304\\
3311	1.54286481145688\\
3312	1.66803299437027\\
3313	1.74501142046612\\
3314	1.80178353154346\\
3315	1.82022141355844\\
3316	1.762166920183\\
3317	1.66618826096858\\
3318	1.52012671003818\\
3319	1.41327606614251\\
3320	1.36495471377756\\
3321	1.3640878188046\\
3322	1.33921797211575\\
3323	1.34136625115527\\
3324	1.40612392870974\\
3325	1.40946783764641\\
3326	1.40431330862732\\
3327	1.39879108613499\\
3328	1.39042418118953\\
3329	1.3476821251704\\
3330	1.32245442822313\\
3331	1.32356323368919\\
3332	1.36678957723635\\
3333	1.39687840039249\\
3334	1.36427347009581\\
3335	1.42005436657419\\
3336	1.51526753618535\\
3337	1.60788820461274\\
3338	1.69213000193478\\
3339	1.71212827575141\\
3340	1.67691086800042\\
3341	1.62648124545162\\
3342	1.47843370047879\\
3343	1.37848362399023\\
3344	1.34469831308842\\
3345	1.33398414041637\\
3346	1.32849985764695\\
3347	1.34741368236285\\
3348	1.39773196240349\\
3349	1.39399599906194\\
3350	1.38551068242474\\
3351	1.35717826909054\\
3352	1.33437154095154\\
3353	1.29041304807137\\
3354	1.26542765416139\\
3355	1.25039313661155\\
3356	1.28509087171595\\
3357	1.26343660807086\\
3358	1.28864420916392\\
3359	1.32428747942227\\
3360	1.39525390942568\\
3361	1.46630120959867\\
3362	1.5172129512798\\
3363	1.54764232693269\\
3364	1.54034598996163\\
3365	1.49173601371162\\
3366	1.38316075251922\\
3367	1.34367526566712\\
3368	1.28910195809662\\
3369	1.25789369144479\\
3370	1.2888901291317\\
3371	1.23852090080989\\
3372	1.25332475765947\\
3373	1.27201195479046\\
3374	1.26673365668897\\
3375	1.25153391965774\\
3376	1.25934842024774\\
3377	1.26268975711321\\
3378	1.26722588911801\\
3379	1.28095761601508\\
3380	1.31767504707716\\
3381	1.31407404516357\\
3382	1.29539281255859\\
3383	1.35987790300152\\
3384	1.61594707365686\\
3385	1.79132064893098\\
3386	1.88158807311251\\
3387	1.88498566944586\\
3388	1.83470898003691\\
3389	1.75172180260946\\
3390	1.61043732266182\\
3391	1.46174167686307\\
3392	1.41419331358836\\
3393	1.40621282217625\\
3394	1.41142860182528\\
3395	1.44241639462054\\
3396	1.47023232122599\\
3397	1.45786604259448\\
3398	1.44533032896692\\
3399	1.43725284207534\\
3400	1.39740530670913\\
3401	1.37551035472875\\
3402	1.31920626336905\\
3403	1.36069091555962\\
3404	1.38980705388244\\
3405	1.36027274468698\\
3406	1.32902099523441\\
3407	1.39468891860774\\
3408	1.53317494916422\\
3409	1.62609327187703\\
3410	1.69954087617078\\
3411	1.70771432402119\\
3412	1.67132046542995\\
3413	1.61681720885396\\
3414	1.48012681342022\\
3415	1.3568684019681\\
3416	1.36188409987195\\
3417	1.3117467075406\\
3418	1.33385717357527\\
3419	1.33656111021969\\
3420	1.3675393183701\\
3421	1.36149987419627\\
3422	1.38584414019356\\
3423	1.3966567294372\\
3424	1.4030710212163\\
3425	1.35784220568983\\
3426	1.36756383204136\\
3427	1.37976397616426\\
3428	1.41919030573347\\
3429	1.46161474451579\\
3430	1.41432579392631\\
3431	1.43317467021934\\
3432	1.47281290949161\\
3433	1.56262111655571\\
3434	1.65824144324321\\
3435	1.71897535347258\\
3436	1.73110599194389\\
3437	1.76780695619659\\
3438	1.77095005917909\\
3439	1.71205712485509\\
3440	1.61649351778959\\
3441	1.61658726191139\\
3442	1.62908229567601\\
3443	1.66303976607236\\
3444	1.67561359593678\\
3445	1.77665004896764\\
3446	1.87334909791026\\
3447	1.8752947251489\\
3448	1.77542451937641\\
3449	1.66532414885776\\
3450	1.63191509002869\\
3451	1.58629056808435\\
3452	1.58261894229468\\
3453	1.63859011570852\\
3454	1.59260664817838\\
3455	1.66645475579075\\
3456	1.66612758869444\\
3457	1.73972144184186\\
3458	1.81971841162414\\
3459	1.83114987830095\\
3460	1.82510456271422\\
3461	1.84343365726735\\
3462	1.83157298540778\\
3463	1.76263196798128\\
3464	1.71961242101704\\
3465	1.70314618114934\\
3466	1.74095215346314\\
3467	1.78390212485685\\
3468	1.83014626247074\\
3469	1.89367040396159\\
3470	1.93140397903628\\
3471	1.89052395293083\\
3472	1.78628779462794\\
3473	1.66958798776458\\
3474	1.58703982480947\\
3475	1.54581442190018\\
3476	1.5769731172769\\
3477	1.56518434112045\\
3478	1.53831612931946\\
3479	1.57039814742227\\
3480	1.6521681058744\\
3481	1.74690180514905\\
3482	1.80655700125968\\
3483	1.81393693080482\\
3484	1.76954660165159\\
3485	1.69021367543069\\
3486	1.5310755320018\\
3487	1.34391415225513\\
3488	1.27868654411925\\
3489	1.24700879416168\\
3490	1.23776786633892\\
3491	1.23157299031289\\
3492	1.26797862801856\\
3493	1.2972487467166\\
3494	1.3048590693374\\
3495	1.30339370910691\\
3496	1.29783702054218\\
3497	1.28531041336101\\
3498	1.29053900488729\\
3499	1.33223357432914\\
3500	1.37441568037042\\
3501	1.40784757122763\\
3502	1.38343535839022\\
3503	1.42552532145859\\
3504	1.53789818600151\\
3505	1.58998504082942\\
3506	1.62810408197475\\
3507	1.6384141514744\\
3508	1.61469155202715\\
3509	1.56937500998751\\
3510	1.44870688543115\\
3511	1.34463938739476\\
3512	1.28059928334195\\
3513	1.24417346306498\\
3514	1.2214141630151\\
3515	1.23325792239952\\
3516	1.30998976853912\\
3517	1.28920939929713\\
3518	1.29006933877187\\
3519	1.25517423987638\\
3520	1.25404998788724\\
3521	1.25404682091675\\
3522	1.25346041950707\\
3523	1.29198659276015\\
3524	1.32273717481753\\
3525	1.3563345266784\\
3526	1.37617768795751\\
3527	1.40351327182402\\
3528	1.47528592363244\\
3529	1.54977655640936\\
3530	1.60034656336183\\
3531	1.60954591059092\\
3532	1.60616978223731\\
3533	1.55074400062311\\
3534	1.42347663104524\\
3535	1.30492611218036\\
3536	1.27542846589857\\
3537	1.24733913451658\\
3538	1.22524108598256\\
3539	1.22925654942286\\
3540	1.2224559181606\\
3541	1.23483608598032\\
3542	1.25898616100278\\
3543	1.27368045401772\\
3544	1.28592727030291\\
3545	1.26844210878786\\
3546	1.26869286857035\\
3547	1.29235150789516\\
3548	1.33120280021136\\
3549	1.34974275682199\\
3550	1.2971371181664\\
3551	1.33792135016413\\
3552	1.44924036273717\\
3553	1.5322292704943\\
3554	1.59541124545483\\
3555	1.63692208253824\\
3556	1.67068094196564\\
3557	1.70789486634917\\
3558	1.73523597918368\\
3559	1.69363641761737\\
3560	1.66118589227309\\
3561	1.61452761340408\\
3562	1.5768185379352\\
3563	1.56290645534421\\
3564	1.56942468634392\\
3565	1.63121988180104\\
3566	1.69733271691534\\
3567	1.71987436393198\\
3568	1.72557267772143\\
3569	1.69485947596681\\
3570	1.61069916956613\\
3571	1.54356529479618\\
3572	1.54515463718691\\
3573	1.51367739373205\\
3574	1.48851466791849\\
3575	1.55164483091248\\
3576	1.65441397155151\\
3577	1.73799538310475\\
3578	1.79117709459355\\
3579	1.81466938479262\\
3580	1.80218643834785\\
3581	1.76458771335886\\
3582	1.67702915228013\\
3583	1.59336596965713\\
3584	1.57537732329565\\
3585	1.57210313093325\\
3586	1.58404664371216\\
3587	1.55727568008797\\
3588	1.64919111447361\\
3589	1.69701652389237\\
3590	1.69149932758696\\
3591	1.72908952688909\\
3592	1.69174026650414\\
3593	1.68209387234293\\
3594	1.56634409964128\\
3595	1.51817310532781\\
3596	1.53716074891189\\
3597	1.55808618572393\\
3598	1.52417399160467\\
3599	1.53782941496754\\
3600	1.58858663847764\\
3601	1.67199244323598\\
3602	1.73394240018804\\
3603	1.7479021388503\\
3604	1.73969212078291\\
3605	1.75539473342113\\
3606	1.73430482651318\\
3607	1.66235908240646\\
3608	1.6234061318884\\
3609	1.57803804549519\\
3610	1.65983879327477\\
3611	1.66175692227782\\
3612	1.66393352235026\\
3613	1.69016979981357\\
3614	1.72678397910067\\
3615	1.72777192426236\\
3616	1.67422774028148\\
3617	1.59633273805267\\
3618	1.55661358997438\\
3619	1.52934884427107\\
3620	1.53210255384682\\
3621	1.54015754546434\\
3622	1.51095469502889\\
3623	1.50711765102413\\
3624	1.61253361673398\\
3625	1.69744073510795\\
3626	1.75096968820139\\
3627	1.73850099942952\\
3628	1.74609267288591\\
3629	1.77611295202285\\
3630	1.79176103724345\\
3631	1.74673999066845\\
3632	1.69757602657492\\
3633	1.65146137170387\\
3634	1.63942105286224\\
3635	1.64507404734072\\
3636	1.68035714813859\\
3637	1.73625795890791\\
3638	1.78730805691229\\
3639	1.80976327351318\\
3640	1.77145973174921\\
3641	1.66016125248812\\
3642	1.53404974995119\\
3643	1.50320855497487\\
3644	1.48680998465372\\
3645	1.46942726060924\\
3646	1.48781914445541\\
3647	1.45103288749957\\
3648	1.56929971760489\\
3649	1.64793459294226\\
3650	1.6946471472596\\
3651	1.70286727209379\\
3652	1.66667934061087\\
3653	1.59714174342091\\
3654	1.44073901465412\\
3655	1.32601138932868\\
3656	1.30301323734043\\
3657	1.31067097561998\\
3658	1.3336125897619\\
3659	1.34137039082495\\
3660	1.38190246380524\\
3661	1.37328253615264\\
3662	1.36781987980151\\
3663	1.36881231923175\\
3664	1.3483000075915\\
3665	1.30116201115787\\
3666	1.32899703918848\\
3667	1.31806256180328\\
3668	1.34906816405225\\
3669	1.34621159344954\\
3670	1.2898806268131\\
3671	1.35759877201769\\
3672	1.46399048280467\\
3673	1.52135835656344\\
3674	1.57021480511099\\
3675	1.58797552576248\\
3676	1.59526621297103\\
3677	1.56297554627002\\
3678	1.43779593390847\\
3679	1.30479307475122\\
3680	1.27612548527471\\
3681	1.26843508623987\\
3682	1.29362971128189\\
3683	1.29868774040491\\
3684	1.36976050892584\\
3685	1.40934185923732\\
3686	1.38940747508346\\
3687	1.35794283354143\\
3688	1.3221085620893\\
3689	1.32261712363637\\
3690	1.27710695910869\\
3691	1.27773223158342\\
3692	1.31265085461586\\
3693	1.34551436082535\\
3694	1.29861435260488\\
3695	1.36251358002126\\
3696	1.4577450528693\\
3697	1.48727946773778\\
3698	1.5388317344878\\
3699	1.55452120281891\\
3700	1.52154061254977\\
3701	1.49885676147523\\
3702	1.37376504855579\\
3703	1.2702724617331\\
3704	1.19023874194263\\
3705	1.18413930661907\\
3706	1.1630157864384\\
3707	1.17537368861946\\
3708	1.16339919020576\\
3709	1.16581014547604\\
3710	1.21562816837501\\
3711	1.22277445958487\\
3712	1.2291607139089\\
3713	1.2214768169612\\
3714	1.19868898979943\\
3715	1.23700370048132\\
3716	1.29578889936315\\
3717	1.34048347163383\\
3718	1.37802974481406\\
3719	1.45822808239194\\
3720	1.62542618084639\\
3721	1.68730299142826\\
3722	1.74743473806237\\
3723	1.74205322279998\\
3724	1.72774932969227\\
3725	1.69494664100439\\
3726	1.56468557746798\\
3727	1.42094100815621\\
3728	1.37101936290912\\
3729	1.38033453822624\\
3730	1.42279287325755\\
3731	1.44723406471445\\
3732	1.49583214348397\\
3733	1.51296153745204\\
3734	1.54105483778736\\
3735	1.56471908591135\\
3736	1.51332333354539\\
3737	1.44311707519592\\
3738	1.39298210201162\\
3739	1.37038238834379\\
3740	1.39239817397019\\
3741	1.40831787750714\\
3742	1.34286800616276\\
3743	1.3790829393274\\
3744	1.48194479275096\\
3745	1.53725267940936\\
3746	1.58306753027281\\
3747	1.58803974076524\\
3748	1.57221098167797\\
3749	1.52507385934158\\
3750	1.40956711889265\\
3751	1.28344811093828\\
3752	1.26751691644711\\
3753	1.30322577240288\\
3754	1.33817366140415\\
3755	1.36196216099695\\
3756	1.37990997560199\\
3757	1.40149768005111\\
3758	1.44106140393456\\
3759	1.44733111485408\\
3760	1.42319366424452\\
3761	1.37965334897607\\
3762	1.36175150018695\\
3763	1.3550096747808\\
3764	1.39599200495646\\
3765	1.44448988154368\\
3766	1.38321296646044\\
3767	1.45470505099715\\
3768	1.54610360837696\\
3769	1.63285676431404\\
3770	1.70283788060351\\
3771	1.75752370073179\\
3772	1.77242811655421\\
3773	1.845139379414\\
3774	1.82172366476321\\
3775	1.72109187834806\\
3776	1.63617585851796\\
3777	1.57305554230484\\
3778	1.54900100824105\\
3779	1.55181755128365\\
3780	1.58192125794566\\
3781	1.6238039297677\\
3782	1.67430992750607\\
3783	1.66113008807146\\
3784	1.60872641528051\\
3785	1.56266072923787\\
3786	1.51852139790263\\
3787	1.4842384558708\\
3788	1.4962097154783\\
3789	1.48272336021206\\
3790	1.44702916047302\\
3791	1.47717380881303\\
3792	1.55382520480484\\
3793	1.64134450067479\\
3794	1.70012093148227\\
3795	1.72123594661883\\
3796	1.72419116668653\\
3797	1.77709232047157\\
3798	1.79434058597665\\
3799	1.72700128606725\\
3800	1.6533575496581\\
3801	1.60440696604998\\
3802	1.5920694477183\\
3803	1.60750928296351\\
3804	1.65012075701476\\
3805	1.76370012586292\\
3806	1.82167989339568\\
3807	1.84721942346852\\
3808	1.79380198615085\\
3809	1.70970455462316\\
3810	1.62113977619217\\
3811	1.59440463601407\\
3812	1.57441629961808\\
3813	1.55062003804769\\
3814	1.5692028266816\\
3815	1.56578860682774\\
3816	1.64695455299042\\
3817	1.73985531875917\\
3818	1.82364266918675\\
3819	1.83129097486363\\
3820	1.85859748173624\\
3821	1.89877486272914\\
3822	1.88054116138669\\
3823	1.80386395160727\\
3824	1.74011410639167\\
3825	1.68755906541981\\
3826	1.69116369818957\\
3827	1.67020393189327\\
3828	1.68406761660961\\
3829	1.6885789829241\\
3830	1.73537129877557\\
3831	1.73729078131874\\
3832	1.68708028136734\\
3833	1.62949588014114\\
3834	1.5572716274429\\
3835	1.51407816913073\\
3836	1.48910655890562\\
3837	1.4980440130649\\
3838	1.47082422529746\\
3839	1.49433991807984\\
3840	1.54052871657566\\
3841	1.60428631668962\\
3842	1.63308234734281\\
3843	1.6969558133662\\
3844	1.69496751641167\\
3845	1.62260350459049\\
3846	1.45243584365977\\
3847	1.3228292356722\\
3848	1.24263593780518\\
3849	1.22138283008873\\
3850	1.23827362048467\\
3851	1.22652485901527\\
3852	1.25400584046108\\
3853	1.26094667228788\\
3854	1.25876985005813\\
3855	1.27917477701699\\
3856	1.28073583119562\\
3857	1.25744547095255\\
3858	1.22171156985212\\
3859	1.22891116308928\\
3860	1.26689679245714\\
3861	1.29490158118341\\
3862	1.25986617616603\\
3863	1.30108763867455\\
3864	1.40407772180231\\
3865	1.43696861080253\\
3866	1.48318372415007\\
3867	1.49023083137904\\
3868	1.46064690797695\\
3869	1.44201113404697\\
3870	1.31289587789456\\
3871	1.23586127612009\\
3872	1.19120594942631\\
3873	1.1972638113535\\
3874	1.2398721899103\\
3875	1.25057422285247\\
3876	1.26354581703321\\
3877	1.27044968208399\\
3878	1.29451832747296\\
3879	1.29598308630525\\
3880	1.26422731235269\\
3881	1.2287300933331\\
3882	1.24917263820293\\
3883	1.21545397069644\\
3884	1.21868048553675\\
3885	1.25825698980613\\
3886	1.22510620536524\\
3887	1.26122920319733\\
3888	1.36195714581954\\
3889	1.4227788760954\\
3890	1.47472041636434\\
3891	1.48990667545931\\
3892	1.476084215177\\
3893	1.44091426659924\\
3894	1.33466955241681\\
3895	1.21733333402759\\
3896	1.19258667451659\\
3897	1.23896370481109\\
3898	1.24160896712122\\
3899	1.26536832115598\\
3900	1.31756116606996\\
3901	1.30802354090699\\
3902	1.32336025552507\\
3903	1.31606495486159\\
3904	1.27615938561948\\
3905	1.25592999667836\\
3906	1.21196267173485\\
3907	1.21370914781147\\
3908	1.22968850754999\\
3909	1.24736051735245\\
3910	1.21217929774862\\
3911	1.26447318691481\\
3912	1.3117676751058\\
3913	1.37717033794416\\
3914	1.42006009476563\\
3915	1.43749764543925\\
3916	1.48179465658843\\
3917	1.44029768075546\\
3918	1.32510423346029\\
3919	1.20718088208272\\
3920	1.1938366114696\\
3921	1.21247297405552\\
3922	1.22776046245526\\
3923	1.23688781852953\\
3924	1.26364601638577\\
3925	1.27125675304712\\
3926	1.2970995147342\\
3927	1.32115271201425\\
3928	1.29236568355441\\
3929	1.2808615329221\\
3930	1.26359392223323\\
3931	1.2599150213718\\
3932	1.27375197498322\\
3933	1.29237982149662\\
3934	1.25319915935664\\
3935	1.29640593669678\\
3936	1.35762389833102\\
3937	1.43199603588574\\
3938	1.4363765672514\\
3939	1.46439119466006\\
3940	1.49501738750928\\
3941	1.52625790685329\\
3942	1.5442944915498\\
3943	1.47455622136698\\
3944	1.4241270723373\\
3945	1.39859170343386\\
3946	1.42986701893472\\
3947	1.43529114793996\\
3948	1.48743775745037\\
3949	1.53862980548707\\
3950	1.61879637728744\\
3951	1.65263377390224\\
3952	1.62132448938724\\
3953	1.5367700767582\\
3954	1.50135255902522\\
3955	1.43043805642336\\
3956	1.44554491491109\\
3957	1.45880427597704\\
3958	1.38959317308818\\
3959	1.42658075524807\\
3960	1.4828942256248\\
3961	1.55767679106148\\
3962	1.61945715869091\\
3963	1.65400475023648\\
3964	1.65529879628369\\
3965	1.70097957904286\\
3966	1.68577766784164\\
3967	1.63784243485142\\
3968	1.60233028135369\\
3969	1.60036305899578\\
3970	1.59258204495623\\
3971	1.59913302706937\\
3972	1.66177063349327\\
3973	1.71252910337078\\
3974	1.79692445855988\\
3975	1.82149129977999\\
3976	1.80320919268491\\
3977	1.69772670913118\\
3978	1.57152389225968\\
3979	1.49236695004415\\
3980	1.4623126337071\\
3981	1.47111193680151\\
3982	1.4299062830571\\
3983	1.46931162556171\\
3984	1.52019967917529\\
3985	1.63273460985966\\
3986	1.67102491117197\\
3987	1.66487172877752\\
3988	1.62559085268999\\
3989	1.55969390873533\\
3990	1.42438571926197\\
3991	1.29459521637564\\
3992	1.26179069912829\\
3993	1.28198386945227\\
3994	1.28976391795728\\
3995	1.28250838589964\\
3996	1.28666614901737\\
3997	1.30609644898647\\
3998	1.32957544298294\\
3999	1.33215841507071\\
4000	1.32964287270559\\
4001	1.31618941572949\\
};
\addplot [color=mycolor1,line width=1.3pt,solid,forget plot]
  table[row sep=crcr]{%
4001	1.31618941572949\\
4002	1.29559453608764\\
4003	1.31608266758528\\
4004	1.33720879124765\\
4005	1.35632342689396\\
4006	1.32519426188851\\
4007	1.37028832738409\\
4008	1.4977389816997\\
4009	1.52743305227643\\
4010	1.55597518957749\\
4011	1.53741919955852\\
4012	1.52273775635947\\
4013	1.46338672809951\\
4014	1.38049536430936\\
4015	1.29299617862611\\
4016	1.22901716047218\\
4017	1.19161613294501\\
4018	1.26613794170925\\
4019	1.22752355109554\\
4020	1.32703310646884\\
4021	1.39272714364408\\
4022	1.38956252880666\\
4023	1.38002175167595\\
4024	1.37688430919111\\
4025	1.36598053189071\\
4026	1.39530392793119\\
4027	1.38888498612239\\
4028	1.37708044681498\\
4029	1.37742622847031\\
4030	1.32503918265647\\
4031	1.3589989478727\\
4032	1.46296190842913\\
4033	1.5489825040329\\
4034	1.57890942962955\\
4035	1.58848172196584\\
4036	1.56434640038085\\
4037	1.52575360967673\\
4038	1.39913292452421\\
4039	1.29331335102123\\
4040	1.28752944825253\\
4041	1.2788540262545\\
4042	1.30497155265534\\
4043	1.30607507532779\\
4044	1.30419532142317\\
4045	1.34798516447704\\
4046	1.36962170502019\\
4047	1.37979812705748\\
4048	1.36015576330416\\
4049	1.32533379648762\\
4050	1.30916242532408\\
4051	1.31116368606774\\
4052	1.32910174823948\\
4053	1.3550526159864\\
4054	1.3094079319432\\
4055	1.33401645056993\\
4056	1.41082974207069\\
4057	1.48754059550494\\
4058	1.5309922071067\\
4059	1.54488315117306\\
4060	1.51825912317412\\
4061	1.47730403143664\\
4062	1.36202861622388\\
4063	1.28094636819851\\
4064	1.23951616471639\\
4065	1.24932626328895\\
4066	1.25655299482283\\
4067	1.28800106024549\\
4068	1.33371615525912\\
4069	1.35012796744352\\
4070	1.32569268443033\\
4071	1.3389173000853\\
4072	1.30711015174073\\
4073	1.26613376257791\\
4074	1.24000279063002\\
4075	1.27633836640107\\
4076	1.32565494149845\\
4077	1.35646304604422\\
4078	1.31605339367937\\
4079	1.36412268946916\\
4080	1.43432212815354\\
4081	1.4941993713795\\
4082	1.54922884134155\\
4083	1.5543428566154\\
4084	1.5544027752952\\
4085	1.52800711819146\\
4086	1.41858508721773\\
4087	1.34264594284767\\
4088	1.30000514987923\\
4089	1.2943399879378\\
4090	1.28493495620388\\
4091	1.29432312435724\\
4092	1.33368370772\\
4093	1.35827401510497\\
4094	1.33482638389293\\
4095	1.34932563517064\\
4096	1.32686895532556\\
4097	1.30851215315966\\
4098	1.30846913604266\\
4099	1.30062398604511\\
4100	1.306734996058\\
4101	1.32419673563923\\
4102	1.28197832097944\\
4103	1.30794238415821\\
4104	1.39463851459985\\
4105	1.46871050131794\\
4106	1.52509604660416\\
4107	1.55920180150872\\
4108	1.59059373299992\\
4109	1.62114511776913\\
4110	1.60898600693006\\
4111	1.53307665113458\\
4112	1.3965255885567\\
4113	1.39976472205997\\
4114	1.38352885513267\\
4115	1.41543659306338\\
4116	1.45901980087057\\
4117	1.48301428051985\\
4118	1.51606729874238\\
4119	1.53200780221918\\
4120	1.50760054625678\\
4121	1.48277494561907\\
4122	1.43654859047325\\
4123	1.40512665027457\\
4124	1.40178010498662\\
4125	1.40579542166428\\
4126	1.38121023434916\\
4127	1.38277274932911\\
4128	1.48593139271085\\
4129	1.56677604790267\\
4130	1.61588910652981\\
4131	1.65770656662642\\
4132	1.64818003488658\\
4133	1.69122635443168\\
4134	1.70687931213232\\
4135	1.64176620241366\\
4136	1.6096159211088\\
4137	1.57633759815222\\
4138	1.57807174480689\\
4139	1.58478422790385\\
4140	1.62589802095437\\
4141	1.6836661485127\\
4142	1.7284461776359\\
4143	1.72569223282136\\
4144	1.66631834281753\\
4145	1.58010044680156\\
4146	1.51981623098922\\
4147	1.48057370980043\\
4148	1.43551043421745\\
4149	1.41322940987785\\
4150	1.36455491342459\\
4151	1.39947015660366\\
4152	1.50055314312815\\
4153	1.56776093409121\\
4154	1.62701992615919\\
4155	1.63960017321494\\
4156	1.6272269599899\\
4157	1.56153425268857\\
4158	1.39918335456132\\
4159	1.27818261784583\\
4160	1.25744054867085\\
4161	1.24922171512694\\
4162	1.28245111978724\\
4163	1.29013877996645\\
4164	1.31061269710729\\
4165	1.31554713075102\\
4166	1.3357969529238\\
4167	1.32733589432155\\
4168	1.30187054689532\\
4169	1.27456663012423\\
4170	1.23976968286901\\
4171	1.2517705133191\\
4172	1.27925702398332\\
4173	1.33134761763261\\
4174	1.27308223218856\\
4175	1.27153217639943\\
4176	1.37033879151016\\
4177	1.44889311637665\\
4178	1.50113441758113\\
4179	1.51444214493756\\
4180	1.50294851761252\\
4181	1.47894263827432\\
4182	1.38115129611963\\
4183	1.29453127780985\\
4184	1.27677338202901\\
4185	1.27940665477827\\
4186	1.2504679660399\\
4187	1.21070151197521\\
4188	1.24669560883242\\
4189	1.22152868883905\\
4190	1.24662258234907\\
4191	1.27121023677755\\
4192	1.27800735373782\\
4193	1.25807399495768\\
4194	1.32342616543423\\
4195	1.30588897356459\\
4196	1.27173147058311\\
4197	1.30617960339352\\
4198	1.24784257551314\\
4199	1.29168720544665\\
4200	1.36930254930289\\
4201	1.43860399795486\\
4202	1.54115706589538\\
4203	1.55251368551415\\
4204	1.5318000913884\\
4205	1.51940348360105\\
4206	1.40357034748643\\
4207	1.30020666305435\\
4208	1.30449963782686\\
4209	1.30616872966461\\
4210	1.3205758628378\\
4211	1.33279259540913\\
4212	1.36111402758336\\
4213	1.32482021991956\\
4214	1.337075747628\\
4215	1.33923316122851\\
4216	1.32707581540292\\
4217	1.30093783383335\\
4218	1.29503857335015\\
4219	1.29023073893879\\
4220	1.30142816785006\\
4221	1.32317031619275\\
4222	1.27423209953808\\
4223	1.3050185353304\\
4224	1.39381099869254\\
4225	1.47259620424479\\
4226	1.5099449957334\\
4227	1.52247529444703\\
4228	1.49103816559809\\
4229	1.45999060519064\\
4230	1.35686387895285\\
4231	1.31204839104068\\
4232	1.29633566490037\\
4233	1.2672602861788\\
4234	1.25653432272031\\
4235	1.28336056932423\\
4236	1.29869825623951\\
4237	1.31149363486506\\
4238	1.31865120458664\\
4239	1.31816688386565\\
4240	1.29365060998065\\
4241	1.26707548168283\\
4242	1.26294105358684\\
4243	1.31685809554586\\
4244	1.32349105841922\\
4245	1.32945601747519\\
4246	1.275263031251\\
4247	1.28937517252696\\
4248	1.37516975771258\\
4249	1.4507748715812\\
4250	1.50428529116204\\
4251	1.53615720441293\\
4252	1.53014174788755\\
4253	1.49786195506492\\
4254	1.3897027345737\\
4255	1.31703111984698\\
4256	1.30014302122484\\
4257	1.25306773624843\\
4258	1.21205868549745\\
4259	1.2507790566151\\
4260	1.24386710789626\\
4261	1.22504138173063\\
4262	1.27566476678454\\
4263	1.34992393602151\\
4264	1.33608599772793\\
4265	1.32150293864922\\
4266	1.34047699202803\\
4267	1.35594931500238\\
4268	1.36632389642256\\
4269	1.36836322203444\\
4270	1.35212261250182\\
4271	1.38069026526006\\
4272	1.51156179774314\\
4273	1.59642943063689\\
4274	1.6670570419327\\
4275	1.70023517190724\\
4276	1.71780830959404\\
4277	1.75055956912029\\
4278	1.70717881839584\\
4279	1.6357603237253\\
4280	1.55633829594116\\
4281	1.50071451922941\\
4282	1.46417798799434\\
4283	1.43906343395589\\
4284	1.4458943989111\\
4285	1.46657920835261\\
4286	1.48498592031279\\
4287	1.46871914608118\\
4288	1.45223764355814\\
4289	1.42466349117417\\
4290	1.37780666905603\\
4291	1.43576404493913\\
4292	1.47440429002814\\
4293	1.45412490436344\\
4294	1.41485761799737\\
4295	1.45086280797502\\
4296	1.54660668211826\\
4297	1.62009706720831\\
4298	1.67512918834857\\
4299	1.69817175230901\\
4300	1.71907775282803\\
4301	1.70966078713112\\
4302	1.70552380828391\\
4303	1.67176822278353\\
4304	1.69602012441316\\
4305	1.6557943748279\\
4306	1.63650795491492\\
4307	1.59000037935756\\
4308	1.53565679833737\\
4309	1.58370596013948\\
4310	1.62444714963466\\
4311	1.66236512823489\\
4312	1.6758571222268\\
4313	1.61938441273584\\
4314	1.55152733731941\\
4315	1.50427356818753\\
4316	1.54600304732303\\
4317	1.48005370451145\\
4318	1.52546271982928\\
4319	1.54500236718032\\
4320	1.587055480313\\
4321	1.63986582772198\\
4322	1.70339907407017\\
4323	1.72830732682964\\
4324	1.69774362438139\\
4325	1.63559591193005\\
4326	1.49868134539045\\
4327	1.38747759799134\\
4328	1.36542381918368\\
4329	1.34076686573725\\
4330	1.3475781680163\\
4331	1.32212203584888\\
4332	1.33947405285563\\
4333	1.36646827606846\\
4334	1.39118570783722\\
4335	1.39569368131295\\
4336	1.37213237382825\\
4337	1.34862122664019\\
4338	1.29610999902684\\
4339	1.31094281482464\\
4340	1.30068895924441\\
4341	1.33194401231678\\
4342	1.29866674637721\\
4343	1.33277409611082\\
4344	1.43308884336158\\
4345	1.48896079374823\\
4346	1.52982045067167\\
4347	1.54969139079214\\
4348	1.531677182099\\
4349	1.51350008767921\\
4350	1.40842986192962\\
4351	1.29780928154691\\
4352	1.2828514689639\\
4353	1.30218647487576\\
4354	1.34082400338246\\
4355	1.36965423479867\\
4356	1.40076715329339\\
4357	1.40450760353909\\
4358	1.43486190642438\\
4359	1.43109248568233\\
4360	1.4176997666995\\
4361	1.37036234046939\\
4362	1.33721781023052\\
4363	1.367751105602\\
4364	1.35365030427683\\
4365	1.45178023021802\\
4366	1.41961971994557\\
4367	1.45305889390258\\
4368	1.49734977561513\\
4369	1.56429917586771\\
4370	1.63773994047444\\
4371	1.66566453615465\\
4372	1.6631546211206\\
4373	1.61894937399649\\
4374	1.52040563907253\\
4375	1.41169724829369\\
4376	1.38452894205561\\
4377	1.38246120632814\\
4378	1.39874972921873\\
4379	1.41556952017938\\
4380	1.47433117163176\\
4381	1.47888991377449\\
4382	1.49539917989589\\
4383	1.48617607997818\\
4384	1.46343701365672\\
4385	1.41836549821394\\
4386	1.3814034767776\\
4387	1.38023389184722\\
4388	1.34897251739531\\
4389	1.3700680038937\\
4390	1.34944104354227\\
4391	1.41146659636917\\
4392	1.49375223043106\\
4393	1.56339877036334\\
4394	1.62130907283976\\
4395	1.65428426986773\\
4396	1.64991234127051\\
4397	1.62685934860187\\
4398	1.52303540110398\\
4399	1.44665786131594\\
4400	1.3806424174047\\
4401	1.38622069551985\\
4402	1.41906802928409\\
4403	1.43917135714619\\
4404	1.489123813568\\
4405	1.49428537407353\\
4406	1.52841632212816\\
4407	1.52110293208369\\
4408	1.50318279847849\\
4409	1.4525013863931\\
4410	1.42695151458626\\
4411	1.43628270996561\\
4412	1.44921013147167\\
4413	1.45153943630807\\
4414	1.43264545473599\\
4415	1.46572374638287\\
4416	1.56117729606036\\
4417	1.63494625125762\\
4418	1.66730014758874\\
4419	1.67817866278197\\
4420	1.65854488156546\\
4421	1.62582398741912\\
4422	1.51918364787478\\
4423	1.40919358901159\\
4424	1.37456566965475\\
4425	1.36938750307759\\
4426	1.36851598028361\\
4427	1.35721676062842\\
4428	1.37102426727749\\
4429	1.37814190348712\\
4430	1.38182402966115\\
4431	1.39357266810357\\
4432	1.41966709337951\\
4433	1.39570683499555\\
4434	1.37664814182431\\
4435	1.39789643702387\\
4436	1.41702305726996\\
4437	1.4529054790648\\
4438	1.43357288746509\\
4439	1.50340537907493\\
4440	1.58627882237049\\
4441	1.67139264916676\\
4442	1.74721595307551\\
4443	1.79126752408154\\
4444	1.81345383280937\\
4445	1.86712780340955\\
4446	1.89911390117386\\
4447	1.89159565700153\\
4448	1.81344855658169\\
4449	1.72682297899704\\
4450	1.72481604771903\\
4451	1.71935614221376\\
4452	1.70919850584017\\
4453	1.71116454181024\\
4454	1.75262711598611\\
4455	1.74152683904705\\
4456	1.75140940000694\\
4457	1.7664738419082\\
4458	1.80644016836022\\
4459	1.82056468696452\\
4460	1.73554379043239\\
4461	1.70088990979367\\
4462	1.6391046229217\\
4463	1.68387426488656\\
4464	1.74015729387358\\
4465	1.82618560642051\\
4466	1.89788463572942\\
4467	1.96911901879963\\
4468	2.06242203825983\\
4469	2.10460953356121\\
4470	2.07989674249205\\
4471	2.01015472071837\\
4472	1.91898202071455\\
4473	1.85366613666493\\
4474	1.79303880888077\\
4475	1.8115170474606\\
4476	1.77723305597322\\
4477	1.77420648400289\\
4478	1.83198703343779\\
4479	1.83157906652971\\
4480	1.77111175630129\\
4481	1.71601188032177\\
4482	1.63568537282106\\
4483	1.60292933405205\\
4484	1.63823393067935\\
4485	1.63616405576641\\
4486	1.64214708472491\\
4487	1.67968903212644\\
4488	1.85086342053808\\
4489	1.89213865348715\\
4490	1.93079290992442\\
4491	1.94097741474605\\
4492	1.90153297139307\\
4493	1.80962426593558\\
4494	1.66034015945357\\
4495	1.48322529934938\\
4496	1.45676969024054\\
4497	1.44872628459886\\
4498	1.45742112137186\\
4499	1.44927901326722\\
4500	1.48253062990321\\
4501	1.46266001076143\\
4502	1.44232624143255\\
4503	1.43095134873069\\
4504	1.39703991266497\\
4505	1.41541745428682\\
4506	1.38118832530763\\
4507	1.3898779969399\\
4508	1.3626614495872\\
4509	1.3711925161759\\
4510	1.346548670668\\
4511	1.40229422129295\\
4512	1.51908662170931\\
4513	1.56949102626469\\
4514	1.62673430339605\\
4515	1.65522774760241\\
4516	1.65453580538204\\
4517	1.60999520846384\\
4518	1.50902906260639\\
4519	1.39323995463441\\
4520	1.30075678956672\\
4521	1.25914719882269\\
4522	1.24204335334177\\
4523	1.22705456276675\\
4524	1.25434556855222\\
4525	1.25624565363706\\
4526	1.25845731507884\\
4527	1.27481015404865\\
4528	1.28916955303821\\
4529	1.27474672671309\\
4530	1.28876295501267\\
4531	1.31620092247046\\
4532	1.33875450324422\\
4533	1.36370401235384\\
4534	1.33986278526885\\
4535	1.38578034642895\\
4536	1.51576137808137\\
4537	1.59773572525181\\
4538	1.65084962291435\\
4539	1.69944063584339\\
4540	1.73141648004091\\
4541	1.70811144672443\\
4542	1.5989073885938\\
4543	1.47890520288579\\
4544	1.40163689680588\\
4545	1.38218241204369\\
4546	1.34398461819648\\
4547	1.30551437190779\\
4548	1.3189031496298\\
4549	1.32542099030167\\
4550	1.33210588862896\\
4551	1.34559537679521\\
4552	1.35219570478913\\
4553	1.35782837680865\\
4554	1.37188077595834\\
4555	1.43414803589739\\
4556	1.48657104940747\\
4557	1.49166120453397\\
4558	1.47427716095113\\
4559	1.53326563493965\\
4560	1.66026418908893\\
4561	1.74973464018032\\
4562	1.81410121529767\\
4563	1.83728822992848\\
4564	1.80951872470068\\
4565	1.74328870369438\\
4566	1.60808549317835\\
4567	1.49231831445394\\
4568	1.39357565458146\\
4569	1.3733343406562\\
4570	1.39607238352986\\
4571	1.39414024193057\\
4572	1.35756491810609\\
4573	1.37540724405425\\
4574	1.39858145675128\\
4575	1.41149710352945\\
4576	1.42380046459854\\
4577	1.39748082007369\\
4578	1.39476678655035\\
4579	1.40678593059262\\
4580	1.46203329074518\\
4581	1.43009229546106\\
4582	1.40716270325324\\
4583	1.45982886349861\\
4584	1.57076282773402\\
4585	1.64901028246195\\
4586	1.67704434967407\\
4587	1.68945641370108\\
4588	1.69143367106243\\
4589	1.64202550255197\\
4590	1.53348693424498\\
4591	1.43442636352814\\
4592	1.43777076953143\\
4593	1.45609350827763\\
4594	1.43578341500415\\
4595	1.34356191608332\\
4596	1.3517702214149\\
4597	1.38796859423923\\
4598	1.41721169923852\\
4599	1.43955606133263\\
4600	1.44328519536852\\
4601	1.41248836347422\\
4602	1.41099903478099\\
4603	1.44706819654924\\
4604	1.42452859406177\\
4605	1.43363840147953\\
4606	1.41099990155249\\
4607	1.4164633150049\\
4608	1.48959269632043\\
4609	1.56482829603809\\
4610	1.63321459409466\\
4611	1.65867567202244\\
4612	1.68885727694936\\
4613	1.71107093472958\\
4614	1.71447633192934\\
4615	1.64426748601285\\
4616	1.55598306130484\\
4617	1.49471001728977\\
4618	1.46140184814382\\
4619	1.46756423983987\\
4620	1.4825638511596\\
4621	1.53004746963212\\
4622	1.57171399952105\\
4623	1.61232305970308\\
4624	1.5937741016283\\
4625	1.54667022151225\\
4626	1.51328147473042\\
4627	1.4789564989673\\
4628	1.49766778947659\\
4629	1.49874646914091\\
4630	1.46692690083988\\
4631	1.51917442172625\\
4632	1.59643110658413\\
4633	1.68205665066274\\
4634	1.76562988138068\\
4635	1.82139779965719\\
4636	1.85680281667358\\
4637	1.88634762553177\\
4638	1.90810067247642\\
4639	1.85109102391815\\
4640	1.78710531015798\\
4641	1.70537669035585\\
4642	1.64310502112223\\
4643	1.60752547456017\\
4644	1.60950587106563\\
4645	1.69854894657558\\
4646	1.75441749964696\\
4647	1.7927196285187\\
4648	1.78622703592402\\
4649	1.72021499991657\\
4650	1.63249995086164\\
4651	1.57525568025951\\
4652	1.55943295739602\\
4653	1.55590215686009\\
4654	1.5337712576679\\
4655	1.54670991201498\\
4656	1.62824688657946\\
4657	1.80015002456493\\
4658	1.86067167656401\\
4659	1.86658151007515\\
4660	1.83526076366312\\
4661	1.758554196914\\
4662	1.63388134749322\\
4663	1.49312248466233\\
4664	1.4159921309297\\
4665	1.43717093063491\\
4666	1.40105205718623\\
4667	1.38121887229992\\
4668	1.379420586745\\
4669	1.41736641358679\\
4670	1.44220858110202\\
4671	1.47427231216578\\
4672	1.4463027784979\\
4673	1.41411635440644\\
4674	1.39281215045093\\
4675	1.42045892611969\\
4676	1.43197511380951\\
4677	1.45198940349357\\
4678	1.41051716632336\\
4679	1.44570208590064\\
4680	1.56441372331271\\
4681	1.64520619050764\\
4682	1.71998946591133\\
4683	1.73928363703857\\
4684	1.73862636926446\\
4685	1.71448323193354\\
4686	1.60099615075696\\
4687	1.52544375290271\\
4688	1.47758088760188\\
4689	1.45281651911276\\
4690	1.39621914128058\\
4691	1.36065591674509\\
4692	1.35367553326594\\
4693	1.36683856624594\\
4694	1.37709943143013\\
4695	1.3701975618758\\
4696	1.37400036009692\\
4697	1.34351682421745\\
4698	1.33653357345793\\
4699	1.38730797561503\\
4700	1.39614313272518\\
4701	1.38567934719607\\
4702	1.3537124141453\\
4703	1.32735883451147\\
4704	1.41632498789465\\
4705	1.47579846449675\\
4706	1.53419652670055\\
4707	1.55052427503633\\
4708	1.54491365307537\\
4709	1.49543251762708\\
4710	1.41997404107156\\
4711	1.31657633738457\\
4712	1.26389757590045\\
4713	1.25966400963326\\
4714	1.27244325770386\\
4715	1.29949523023552\\
4716	1.35503269895928\\
4717	1.37840477525719\\
4718	1.40393192892372\\
4719	1.37865827204192\\
4720	1.38819032009804\\
4721	1.36177115447024\\
4722	1.31702585062606\\
4723	1.33288011375792\\
4724	1.38944016745084\\
4725	1.38917997143364\\
4726	1.33370630894813\\
4727	1.37348820032172\\
4728	1.45637121478727\\
4729	1.54325492971439\\
4730	1.5860351347298\\
4731	1.60344269108749\\
4732	1.60241388249113\\
4733	1.55757492769731\\
4734	1.47908198576506\\
4735	1.39840907533089\\
4736	1.33405975485323\\
4737	1.35872118709408\\
4738	1.37368454933541\\
4739	1.38699198157114\\
4740	1.39437850851295\\
4741	1.42694219815011\\
4742	1.4358988985985\\
4743	1.41664217601781\\
4744	1.40526567403826\\
4745	1.36460645055754\\
4746	1.34707465386839\\
4747	1.3662306970077\\
4748	1.41009324785018\\
4749	1.40746891907867\\
4750	1.37287335755327\\
4751	1.44316172079412\\
4752	1.5367839624099\\
4753	1.6454164712439\\
4754	1.70175677134148\\
4755	1.71687733837094\\
4756	1.71489152994698\\
4757	1.66131631364338\\
4758	1.57673992238283\\
4759	1.48638285845249\\
4760	1.41529170963782\\
4761	1.42928132490828\\
4762	1.436169521318\\
4763	1.4079416405844\\
4764	1.49510271864436\\
4765	1.56972322470649\\
4766	1.57365237071526\\
4767	1.54420650195928\\
4768	1.51722272960743\\
4769	1.46358228578607\\
4770	1.44037034043947\\
4771	1.41449613677074\\
4772	1.4220252268564\\
4773	1.43262990789637\\
4774	1.39613239688236\\
4775	1.46160725966154\\
4776	1.55076214797843\\
4777	1.71113143705093\\
4778	1.80692979979074\\
4779	1.88286163290274\\
4780	1.89838451304531\\
4781	1.95803987751944\\
4782	1.98463940441448\\
4783	1.9316290137851\\
4784	1.88483927324364\\
4785	1.8828981461269\\
4786	1.84872705952539\\
4787	1.96785469738677\\
4788	2.00959217780517\\
4789	1.99696822998982\\
4790	2.01802723008759\\
4791	1.97163969996087\\
4792	1.86814419746603\\
4793	1.76161080696037\\
4794	1.64209334311521\\
4795	1.5830768789506\\
4796	1.56383980879565\\
4797	1.60328359816925\\
4798	1.61911437800292\\
4799	1.61268820009824\\
4800	1.6835510361411\\
4801	1.7612807848468\\
4802	1.82229998084255\\
4803	1.8409248704038\\
4804	1.86778733933902\\
4805	1.87055365313631\\
4806	1.90265920113458\\
4807	1.86220440035996\\
4808	1.84146216677447\\
4809	1.76736351882109\\
4810	1.73358058090014\\
4811	1.74994800436423\\
4812	1.83399759788768\\
4813	1.83020386314253\\
4814	1.86015127329546\\
4815	1.90747294279015\\
4816	1.89961049238829\\
4817	1.88669371902829\\
4818	1.77882847063897\\
4819	1.70620640241822\\
4820	1.67787494940387\\
4821	1.66948884160969\\
4822	1.64974644937316\\
4823	1.68528977251559\\
4824	1.76547050375868\\
4825	1.85371676425487\\
4826	1.91164660786981\\
4827	1.96169788708705\\
4828	1.98443216780835\\
4829	2.01042628229444\\
4830	2.0291803130122\\
4831	1.96268752508897\\
4832	1.95555077393018\\
4833	1.87744129479487\\
4834	1.82268453042181\\
4835	1.85740629918543\\
4836	1.86417469450505\\
4837	1.81862010501414\\
4838	1.86030961873876\\
4839	1.89485246827313\\
4840	1.90107418324606\\
4841	1.86574660751424\\
4842	1.80909577564819\\
4843	1.79614900925671\\
4844	1.78708631690126\\
4845	1.77774472343373\\
4846	1.74443404188909\\
4847	1.79909011978581\\
4848	1.905945297287\\
4849	2.00254192002816\\
4850	2.07444399903042\\
4851	2.11371501523038\\
4852	2.10876263914645\\
4853	2.01190351177934\\
4854	1.86652443172401\\
4855	1.70962054720054\\
4856	1.59960197208602\\
4857	1.5533109549666\\
4858	1.55666290725773\\
4859	1.56466343196595\\
4860	1.62347457918826\\
4861	1.66336974400224\\
4862	1.70007192927527\\
4863	1.72744409033513\\
4864	1.73463252445493\\
4865	1.63876537071008\\
4866	1.61406687849377\\
4867	1.59339160555086\\
4868	1.57454075451458\\
4869	1.5977752908448\\
4870	1.57013228994685\\
4871	1.59494133446264\\
4872	1.72744498833086\\
4873	1.81880330211071\\
4874	1.88289572349143\\
4875	1.88381917203693\\
4876	1.88413072412988\\
4877	1.82725561358592\\
4878	1.76391936942108\\
4879	1.62500287078914\\
4880	1.56160487694797\\
4881	1.56853141219304\\
4882	1.59004603994113\\
4883	1.61699074762484\\
4884	1.69618214621117\\
4885	1.70980129877867\\
4886	1.72364875130709\\
4887	1.70196836667236\\
4888	1.67992038484474\\
4889	1.64887251180275\\
4890	1.62341148980498\\
4891	1.61331082972964\\
4892	1.60901041745858\\
4893	1.6167888230751\\
4894	1.55846212306039\\
4895	1.60484565626097\\
4896	1.71848761219632\\
4897	1.81964169044996\\
4898	1.86259764681552\\
4899	1.87348395449316\\
4900	1.8486400472033\\
4901	1.80918121993009\\
4902	1.7159125307324\\
4903	1.58933785249964\\
4904	1.57539940516794\\
4905	1.6008950267197\\
4906	1.64448830657587\\
4907	1.66553026417196\\
4908	1.66102770693597\\
4909	1.69252862786778\\
4910	1.71373967047149\\
4911	1.71218786438991\\
4912	1.68208041113866\\
4913	1.63055307356014\\
4914	1.60345581232375\\
4915	1.60251020065602\\
4916	1.58834678928129\\
4917	1.57791460829857\\
4918	1.53825648636354\\
4919	1.57322272022105\\
4920	1.67313177336046\\
4921	1.7476778886419\\
4922	1.80357778950498\\
4923	1.8106174509066\\
4924	1.79506213672045\\
4925	1.75795560184864\\
4926	1.68637131227605\\
4927	1.62447745270393\\
4928	1.53127677518992\\
4929	1.53925109264216\\
4930	1.56063915451773\\
4931	1.57454007572243\\
4932	1.59184305316604\\
4933	1.59948669322999\\
4934	1.59971185235114\\
4935	1.61113527883714\\
4936	1.62112318995445\\
4937	1.61703895491777\\
4938	1.59367069216149\\
4939	1.58035644530255\\
4940	1.59262165839242\\
4941	1.58312168194018\\
4942	1.53903304680397\\
4943	1.60274772170575\\
4944	1.64911441524974\\
4945	1.7294289797867\\
4946	1.79796213842512\\
4947	1.81985052470703\\
4948	1.83681941723779\\
4949	1.84434999235046\\
4950	1.83840910137275\\
4951	1.77451277725081\\
4952	1.71896618126226\\
4953	1.66561937010653\\
4954	1.65600660459341\\
4955	1.65837546966231\\
4956	1.68535008702192\\
4957	1.72959621294691\\
4958	1.78384264688621\\
4959	1.7898996791458\\
4960	1.75857664710909\\
4961	1.69549324553552\\
4962	1.67630774439399\\
4963	1.63080625522925\\
4964	1.61860175928722\\
4965	1.61812386986629\\
4966	1.59020053971504\\
4967	1.62273209933302\\
4968	1.73172778380117\\
4969	1.79250251454019\\
4970	1.84765893362585\\
4971	1.88667572673274\\
4972	1.90966192632447\\
4973	1.89946962366138\\
4974	1.93171208021666\\
4975	1.88451749837483\\
4976	1.84799288624546\\
4977	1.82676415740296\\
4978	1.88176883690491\\
4979	1.9114559227512\\
4980	1.96439988250826\\
4981	2.0134450101113\\
4982	2.03685617223114\\
4983	2.02117428741974\\
4984	1.96159526698572\\
4985	1.85282989629305\\
4986	1.75235608781471\\
4987	1.66547847190056\\
4988	1.62968739978369\\
4989	1.60979722249612\\
4990	1.61731847002928\\
4991	1.65104421276981\\
4992	1.7166140394037\\
4993	1.77290814718587\\
4994	1.83086023545586\\
4995	1.85256267079421\\
4996	1.83476126083532\\
4997	1.7708657720389\\
4998	1.68846573948948\\
4999	1.58268012990512\\
5000	1.51692483792136\\
5001	1.4867689134258\\
5002	1.48082339913355\\
5003	1.48491734050944\\
5004	1.48656367523982\\
5005	1.5165270779851\\
5006	1.53479128015968\\
5007	1.55405516917485\\
5008	1.54591603977543\\
5009	1.51626416234671\\
5010	1.51873866376434\\
5011	1.56820193470871\\
5012	1.60033548418052\\
5013	1.61153750555251\\
5014	1.55051441309273\\
5015	1.63284546290318\\
5016	1.73161353122053\\
5017	1.8115714429989\\
5018	1.88663941310593\\
5019	1.90138271338261\\
5020	1.89523654531382\\
5021	1.83886058890445\\
5022	1.75810156451926\\
5023	1.65386977402118\\
5024	1.58363824714227\\
5025	1.61054422338444\\
5026	1.61820325095725\\
5027	1.61809156963678\\
5028	1.59822428142478\\
5029	1.62602384566583\\
5030	1.65797821105665\\
5031	1.67409499720823\\
5032	1.65487478905342\\
5033	1.60424140757313\\
5034	1.58385799538476\\
5035	1.59566172634704\\
5036	1.58107770666115\\
5037	1.58315851178155\\
5038	1.50773612723059\\
5039	1.57338895937081\\
5040	1.66757852717405\\
5041	1.73605181008837\\
5042	1.78926302476676\\
5043	1.81385147630244\\
5044	1.81714886431123\\
5045	1.77389946741239\\
5046	1.68559943745563\\
5047	1.57775611467238\\
5048	1.53408596122801\\
5049	1.55419071098099\\
5050	1.57374051980781\\
5051	1.59108689728736\\
5052	1.63258477886812\\
5053	1.65343020470828\\
5054	1.63720708520389\\
5055	1.62731791701827\\
5056	1.59598030319354\\
5057	1.54797866157849\\
5058	1.52501544343363\\
5059	1.52810935058777\\
5060	1.54040161133377\\
5061	1.55433938742125\\
5062	1.48162796157517\\
5063	1.57710233351075\\
5064	1.65633597826658\\
5065	1.74967760847099\\
5066	1.81280157818883\\
5067	1.84479626036456\\
5068	1.83051299987061\\
5069	1.78151888702967\\
5070	1.70081070096294\\
5071	1.58915840467819\\
5072	1.55371143179058\\
5073	1.55720093785001\\
5074	1.54160679574318\\
5075	1.5798640425226\\
5076	1.57797769900633\\
5077	1.60155006491717\\
5078	1.60781128641772\\
5079	1.60918592605525\\
5080	1.60020241066137\\
5081	1.60950851383479\\
5082	1.58003059675887\\
5083	1.58244441940473\\
5084	1.54937033730299\\
5085	1.54080672416639\\
5086	1.47777251373869\\
5087	1.56573471745639\\
5088	1.67376730318411\\
5089	1.74734826794991\\
5090	1.80316304759763\\
5091	1.81075107951983\\
5092	1.81920290373721\\
5093	1.76704775578893\\
5094	1.6872228453427\\
5095	1.57078783391539\\
5096	1.52792117185262\\
5097	1.55356740505943\\
5098	1.58575156778369\\
5099	1.63122357907021\\
5100	1.61838305709686\\
5101	1.63282691505675\\
5102	1.64717377608998\\
5103	1.63528336405328\\
5104	1.62734605726679\\
5105	1.58223498990548\\
5106	1.54462242122231\\
5107	1.52383532445451\\
5108	1.50714520842046\\
5109	1.51708310458728\\
5110	1.46890113334712\\
5111	1.53192625856298\\
5112	1.63014375188912\\
5113	1.70782700355481\\
5114	1.78572825605775\\
5115	1.84820045665323\\
5116	1.87760196749425\\
5117	1.88900060069232\\
5118	1.90579329997623\\
5119	1.8414552400793\\
5120	1.7602877270298\\
5121	1.71969329404051\\
5122	1.69483311578048\\
5123	1.68425881146882\\
5124	1.73071569784826\\
5125	1.74355719873472\\
5126	1.77856662515523\\
5127	1.74978479329568\\
5128	1.69096680071591\\
5129	1.62202069139607\\
5130	1.59635912547937\\
5131	1.59553773707085\\
5132	1.57030763268662\\
5133	1.56497115459285\\
5134	1.55809580496493\\
5135	1.61074479066161\\
5136	1.73734458827019\\
5137	1.82502577365379\\
5138	1.90396230228593\\
5139	1.93556170413346\\
5140	1.93900888593752\\
5141	1.93485378900737\\
5142	1.93870669227527\\
5143	1.89789308580299\\
5144	1.85210376287785\\
5145	1.88741557551948\\
5146	1.89308855224634\\
5147	1.90130613593707\\
5148	1.94403826309424\\
5149	2.02408397691402\\
5150	2.08274590293557\\
5151	2.08997143586383\\
5152	2.02954623931267\\
5153	1.92223915227928\\
5154	1.86717796399558\\
5155	1.76341568771223\\
5156	1.73276232162147\\
5157	1.74623382952743\\
5158	1.71870825328518\\
5159	1.77540407154056\\
5160	1.77514182658803\\
5161	1.84160606562223\\
5162	1.90191231805711\\
5163	1.91458906251578\\
5164	1.89212618992821\\
5165	1.80851442392841\\
5166	1.63040586827657\\
5167	1.53170629277662\\
5168	1.44267265534897\\
5169	1.41583933595077\\
5170	1.41751349899803\\
5171	1.45773185081765\\
5172	1.47580528765776\\
5173	1.52193617103804\\
5174	1.55810706303978\\
5175	1.56043556540382\\
5176	1.55725465195629\\
5177	1.5072182374253\\
5178	1.52077556169586\\
5179	1.50515625634747\\
5180	1.47830295546505\\
5181	1.47459773941874\\
5182	1.42633000886513\\
5183	1.50894137781055\\
5184	1.62274076310759\\
5185	1.69801185877577\\
5186	1.74880952266386\\
5187	1.77398741817662\\
5188	1.77601374373296\\
5189	1.71592068516447\\
5190	1.61416614154227\\
5191	1.50302543294371\\
5192	1.47718305484837\\
5193	1.4982703868072\\
5194	1.51744582661318\\
5195	1.55902534663892\\
5196	1.45401256465141\\
5197	1.46088106454156\\
5198	1.48345262545796\\
5199	1.47868671455192\\
5200	1.4452289492623\\
5201	1.39130027189605\\
5202	1.35498253973797\\
5203	1.33563073129264\\
5204	1.33843077741259\\
5205	1.32990050309456\\
5206	1.29594162689493\\
5207	1.37998028538411\\
5208	1.50931882402675\\
5209	1.58680840816009\\
5210	1.65598476044806\\
5211	1.67629046453695\\
5212	1.70248968516289\\
5213	1.64742319156116\\
5214	1.57341055106209\\
5215	1.47813585156727\\
5216	1.41288527457515\\
5217	1.40911653878314\\
5218	1.41218156137593\\
5219	1.41100737218043\\
5220	1.44147498107671\\
5221	1.45788734030126\\
5222	1.47189193635809\\
5223	1.44062433100851\\
5224	1.40938804083369\\
5225	1.38094763711316\\
5226	1.37807359820861\\
5227	1.4088191845344\\
5228	1.42594450663982\\
5229	1.42303374286449\\
5230	1.39577929791604\\
5231	1.47416928194715\\
5232	1.54607456866111\\
5233	1.61853287878565\\
5234	1.6597083530823\\
5235	1.66398410347946\\
5236	1.64110545788209\\
5237	1.57235570498506\\
5238	1.48756502800201\\
5239	1.38894035402197\\
5240	1.34817971626027\\
5241	1.27784859919047\\
5242	1.29449558505161\\
5243	1.31513916046334\\
5244	1.31519956871601\\
5245	1.32989959811272\\
5246	1.34304371648652\\
5247	1.35606821483396\\
5248	1.36840808778791\\
5249	1.36026057765293\\
5250	1.34531113652791\\
5251	1.37067186544196\\
5252	1.33394425717413\\
5253	1.32412324279651\\
5254	1.31609459044885\\
5255	1.38152678354068\\
5256	1.45710224135249\\
5257	1.5372800707245\\
5258	1.58846422888717\\
5259	1.61401269979424\\
5260	1.61227001863672\\
5261	1.55394062218637\\
5262	1.47137860616508\\
5263	1.35616052354667\\
5264	1.30441055706259\\
5265	1.28624112563431\\
5266	1.30870565903884\\
5267	1.30912789195062\\
5268	1.32655439890589\\
5269	1.32156378749173\\
5270	1.33743493343918\\
5271	1.31385326192454\\
5272	1.30910422411239\\
5273	1.3310897619535\\
5274	1.33886382177027\\
5275	1.34099752573996\\
5276	1.37947404027159\\
5277	1.41050511575643\\
5278	1.41737739296711\\
5279	1.48415213543186\\
5280	1.60159451357936\\
5281	1.69594425879808\\
5282	1.76807540132066\\
5283	1.83403835894509\\
5284	1.85823354886763\\
5285	1.89578095518565\\
5286	1.87909598015072\\
5287	1.80924484423009\\
5288	1.71893845482719\\
5289	1.67369711565072\\
5290	1.66643139323896\\
5291	1.67139625902726\\
5292	1.75214174896658\\
5293	1.8366759049439\\
5294	1.83611447246856\\
5295	1.83304899897076\\
5296	1.78551166138504\\
5297	1.7021761780697\\
5298	1.62347881545896\\
5299	1.55206499068132\\
5300	1.58311649278145\\
5301	1.54909513929099\\
5302	1.51462197643426\\
5303	1.51848816054046\\
5304	1.61108557304273\\
5305	1.67485941268947\\
5306	1.74688930847379\\
5307	1.80713574042204\\
5308	1.81725705781646\\
5309	1.81353199867003\\
5310	1.86318941114022\\
5311	1.8632856567444\\
5312	1.81294109976004\\
5313	1.75042213253978\\
5314	1.71030391194284\\
5315	1.66407519997308\\
5316	1.6832142130594\\
5317	1.818208098987\\
5318	1.88189849297426\\
5319	1.94131938085684\\
5320	1.91007458231898\\
5321	1.85254388037823\\
5322	1.79366052196478\\
5323	1.77941849353858\\
5324	1.78026288096084\\
5325	1.7292115625666\\
5326	1.69676847016518\\
5327	1.76486022132411\\
5328	1.95634149695228\\
5329	2.03692559561564\\
5330	2.11693667105648\\
5331	2.11710379948344\\
5332	2.06115219033984\\
5333	1.92594419867381\\
5334	1.78099130667032\\
5335	1.62646104925123\\
5336	1.53013562995122\\
5337	1.56810068217786\\
5338	1.60095733737027\\
5339	1.60538389946524\\
5340	1.61893464266262\\
5341	1.62766999330259\\
5342	1.6358129796159\\
5343	1.63065145225729\\
5344	1.641455943792\\
5345	1.62123280486859\\
5346	1.57084192021961\\
5347	1.53956801892375\\
5348	1.53658878147442\\
5349	1.55277474148025\\
5350	1.50659371302244\\
5351	1.58471345906682\\
5352	1.70167542495962\\
5353	1.79191357455511\\
5354	1.83644959050952\\
5355	1.85421281616043\\
5356	1.83217155420689\\
5357	1.74451211216487\\
5358	1.62358719994193\\
5359	1.51868096832261\\
5360	1.48814251124298\\
5361	1.50002527210344\\
5362	1.51471096795269\\
5363	1.4950917900688\\
5364	1.54378742563379\\
5365	1.5435824246366\\
5366	1.57118218572204\\
5367	1.55295935427411\\
5368	1.51834499558395\\
5369	1.47272744480672\\
5370	1.42406740199846\\
5371	1.448154765647\\
5372	1.44602714915264\\
5373	1.43174062849104\\
5374	1.4034338730743\\
5375	1.49903291678204\\
5376	1.59534023401427\\
5377	1.6644642841554\\
5378	1.72031217376165\\
5379	1.73916179486891\\
5380	1.71252589842983\\
5381	1.63332238379239\\
5382	1.52610772464493\\
5383	1.43963827649943\\
5384	1.35418647296709\\
5385	1.35053504745947\\
5386	1.3425303174889\\
5387	1.33048171111976\\
5388	1.36659515084737\\
5389	1.38706680115272\\
5390	1.40272747641675\\
5391	1.41256725429627\\
5392	1.4154640095923\\
5393	1.40744563683002\\
5394	1.42411605783239\\
5395	1.38339276404149\\
5396	1.39176403141773\\
5397	1.38364230892908\\
5398	1.3637329599118\\
5399	1.4525950794569\\
5400	1.53465429083845\\
5401	1.60691282498894\\
5402	1.63385598665713\\
5403	1.65241461495679\\
5404	1.63599171935179\\
5405	1.56187399504891\\
5406	1.45679454600768\\
5407	1.34574494252237\\
5408	1.28710181256362\\
5409	1.26521994767664\\
5410	1.22338231673277\\
5411	1.22249590992553\\
5412	1.25735839657224\\
5413	1.27444494492442\\
5414	1.29331828632878\\
5415	1.29473945187724\\
5416	1.2901334071982\\
5417	1.28962633886417\\
5418	1.26046160086775\\
5419	1.28576299015438\\
5420	1.30156549689527\\
5421	1.29427749513685\\
5422	1.30887320185752\\
5423	1.35016341010438\\
5424	1.46701707210354\\
5425	1.54338391829028\\
5426	1.59222482495475\\
5427	1.61039715570794\\
5428	1.63406339534216\\
5429	1.63303567945935\\
5430	1.65047474883555\\
5431	1.63233513503368\\
5432	1.60151940501311\\
5433	1.5723367716938\\
5434	1.55246457136426\\
5435	1.5712670201461\\
5436	1.60357181727302\\
5437	1.6563657720104\\
5438	1.71807550126526\\
5439	1.74880129061556\\
5440	1.76208201506812\\
5441	1.70082676582338\\
5442	1.63205192582561\\
5443	1.57817281266107\\
5444	1.53421070422567\\
5445	1.48389533148456\\
5446	1.44063424859629\\
5447	1.50749066291435\\
5448	1.6060610934891\\
5449	1.68114943678217\\
5450	1.7598631561066\\
5451	1.78507259088368\\
5452	1.77504040396323\\
5453	1.73319460741223\\
5454	1.70486587290205\\
5455	1.63330124887663\\
5456	1.55137835780258\\
5457	1.52906235274967\\
5458	1.52086478730684\\
5459	1.53587181314355\\
5460	1.56823956965305\\
5461	1.62380505048316\\
5462	1.67558406565666\\
5463	1.68965017226418\\
5464	1.66229588603983\\
5465	1.60881942893149\\
5466	1.53112434107205\\
5467	1.51043805895321\\
5468	1.5698492112396\\
5469	1.56864764387915\\
5470	1.54261893939668\\
5471	1.59834994022549\\
5472	1.73936954069078\\
5473	1.84381862162982\\
5474	1.9230401475829\\
5475	1.98490142772909\\
5476	2.0207098743116\\
5477	1.97574382470606\\
5478	2.00037331938698\\
5479	1.97304086330386\\
5480	1.92610628910919\\
5481	1.84615843393231\\
5482	1.83181542035584\\
5483	1.8263522298329\\
5484	1.82657875690754\\
5485	1.8873455631738\\
5486	1.92261359116176\\
5487	1.92863984331248\\
5488	1.89612435575911\\
5489	1.80872329126979\\
5490	1.72831106567526\\
5491	1.64314957112438\\
5492	1.6392542606851\\
5493	1.63015755757443\\
5494	1.64203660446223\\
5495	1.66339736132379\\
5496	1.8089730536639\\
5497	1.88690833254568\\
5498	1.929683763976\\
5499	1.9475791810379\\
5500	1.9270507122255\\
5501	1.79389649811207\\
5502	1.59881153854887\\
5503	1.47420432205644\\
5504	1.41173133137598\\
5505	1.4474236323701\\
5506	1.44228319495421\\
5507	1.43328493734761\\
5508	1.44439265379685\\
5509	1.44127383534997\\
5510	1.43935390793754\\
5511	1.40993935582571\\
5512	1.38713995579126\\
5513	1.32821012306132\\
5514	1.29955847955\\
5515	1.26806385462351\\
5516	1.26088106274925\\
5517	1.26858422294256\\
5518	1.26385353440064\\
5519	1.2980479566834\\
5520	1.39787354837957\\
5521	1.47625874299585\\
5522	1.52820040443716\\
5523	1.55524581819548\\
5524	1.55334182054214\\
5525	1.50031619220396\\
5526	1.38522649285232\\
5527	1.32629965749529\\
5528	1.2617783623159\\
5529	1.25078305474792\\
5530	1.2657068082433\\
5531	1.28207578567975\\
5532	1.31392872827748\\
5533	1.37707578332813\\
5534	1.35208132529964\\
5535	1.36415605853591\\
5536	1.35123203830619\\
5537	1.33983933643455\\
5538	1.26604563445237\\
5539	1.25547986011411\\
5540	1.23265588858431\\
5541	1.24827742220998\\
5542	1.24568815367616\\
5543	1.30241909461925\\
5544	1.36699408418879\\
5545	1.43292201656171\\
5546	1.48468065253771\\
5547	1.49748764087071\\
5548	1.493385466564\\
5549	1.41570133148861\\
5550	1.30405988640266\\
5551	1.22353343181764\\
5552	1.21859330773147\\
5553	1.21397605226448\\
5554	1.23706142686088\\
5555	1.24160134701325\\
5556	1.28637985789731\\
5557	1.28883063038037\\
5558	1.31048969653655\\
5559	1.28411528000547\\
5560	1.25989638148022\\
5561	1.21097944321349\\
5562	1.21921393275501\\
5563	1.21531651580521\\
5564	1.21357382965277\\
5565	1.18016151959555\\
5566	1.18756528926452\\
5567	1.25452761629584\\
5568	1.36015967609146\\
5569	1.42706333086089\\
5570	1.45831613942052\\
5571	1.48401446226914\\
5572	1.47145838993584\\
5573	1.42022403900495\\
5574	1.30662743064457\\
5575	1.22454813323655\\
5576	1.20045939304318\\
5577	1.25763646414477\\
5578	1.28241092387083\\
5579	1.28502988156895\\
5580	1.26523675758345\\
5581	1.29438533410845\\
5582	1.26653911991705\\
5583	1.27604981480897\\
5584	1.2751015275153\\
5585	1.28070628434447\\
5586	1.2282199219755\\
5587	1.25995562996603\\
5588	1.26609833675301\\
5589	1.29395654703754\\
5590	1.29534675930638\\
5591	1.33784159044315\\
5592	1.47212816508155\\
5593	1.54226359899258\\
5594	1.58749114269617\\
5595	1.62995292674642\\
5596	1.62095490684671\\
5597	1.53351055331166\\
5598	1.41372736192254\\
5599	1.35533900212133\\
5600	1.26603055245063\\
5601	1.30755719920688\\
5602	1.30603672996378\\
5603	1.29118846559368\\
5604	1.3167957260591\\
5605	1.28543304761577\\
5606	1.30418705045692\\
5607	1.3026572527409\\
5608	1.307465315595\\
5609	1.29866740907259\\
5610	1.29695946526879\\
5611	1.27567837986484\\
5612	1.24435184214656\\
5613	1.22886152215657\\
5614	1.22624536992362\\
5615	1.2840601358095\\
5616	1.39361771579369\\
5617	1.45982176849056\\
5618	1.39801177216908\\
5619	1.43449938925321\\
5620	1.46108772208603\\
5621	1.46265335738634\\
5622	1.44448252385334\\
5623	1.41296669666264\\
5624	1.38073775340474\\
5625	1.42095400934462\\
5626	1.42193549484815\\
5627	1.4514484720637\\
5628	1.47474605235213\\
5629	1.48453048761654\\
5630	1.55044464676665\\
5631	1.55640259780404\\
5632	1.50931312794429\\
5633	1.43630996667747\\
5634	1.37101325279547\\
5635	1.32010863107557\\
5636	1.27821193122439\\
5637	1.28696306696023\\
5638	1.31037389486524\\
5639	1.32563339576269\\
5640	1.40841548778639\\
5641	1.60504500409052\\
5642	1.71125782775489\\
5643	1.76004685608421\\
5644	1.76979131235766\\
5645	1.76470723138057\\
5646	1.72020998641363\\
5647	1.63358727834434\\
5648	1.61686176534449\\
5649	1.61289088773149\\
5650	1.62723555575142\\
5651	1.60721160973948\\
5652	1.62167374480168\\
5653	1.73456167743034\\
5654	1.81593543900601\\
5655	1.75317913874717\\
5656	1.72985374340393\\
5657	1.62205625260664\\
5658	1.49384782840842\\
5659	1.41770001662979\\
5660	1.37052119193058\\
5661	1.31681906463247\\
5662	1.34208518594155\\
5663	1.38182363289655\\
5664	1.48922218376806\\
5665	1.55407636243386\\
5666	1.59997772850005\\
5667	1.61082639187422\\
5668	1.59195972053319\\
5669	1.49558273931934\\
5670	1.35299787618642\\
5671	1.28770368716674\\
5672	1.2250055854912\\
5673	1.19620369386882\\
5674	1.18200286254244\\
5675	1.17113251289197\\
5676	1.15499172560057\\
5677	1.14842180188832\\
5678	1.14949266808446\\
5679	1.1372142454434\\
5680	1.12853806322252\\
5681	1.14061574324205\\
5682	1.14748550378589\\
5683	1.1823414532674\\
5684	1.20412657272146\\
5685	1.25465900420794\\
5686	1.26483047570847\\
5687	1.32516838618609\\
5688	1.41409906265847\\
5689	1.46775210072691\\
5690	1.51320459437839\\
5691	1.51054636219352\\
5692	1.48981826823772\\
5693	1.42817155634371\\
5694	1.29295243275999\\
5695	1.20641740825486\\
5696	1.16106747005629\\
5697	1.11434017174618\\
5698	1.10194914536474\\
5699	1.09736532930531\\
5700	1.12059801924251\\
5701	1.13034856795127\\
5702	1.14534234052611\\
5703	1.12469465654186\\
5704	1.1277398899236\\
5705	1.12483976674299\\
5706	1.15059214103803\\
5707	1.15933093324485\\
5708	1.18415526274676\\
5709	1.20216708595653\\
5710	1.22266215333283\\
5711	1.27557832292514\\
5712	1.39142884025304\\
5713	1.46998819146466\\
5714	1.51361466233164\\
5715	1.51879850114344\\
5716	1.49969661563402\\
5717	1.42502552974751\\
5718	1.29773198358998\\
5719	1.23467294420552\\
5720	1.17026404208852\\
5721	1.17698745942853\\
5722	1.20478908754291\\
5723	1.22449524960657\\
5724	1.24432587597307\\
5725	1.23982624892832\\
5726	1.24722414516958\\
5727	1.23753503882601\\
5728	1.22971316439363\\
5729	1.21919639367707\\
5730	1.17942736661767\\
5731	1.18415767407641\\
5732	1.20431160166579\\
5733	1.19604590154725\\
5734	1.19733361527127\\
5735	1.27759178524405\\
5736	1.39126004907849\\
5737	1.45686212090595\\
5738	1.51874624980784\\
5739	1.55252742222899\\
5740	1.54204021844437\\
5741	1.47441862799314\\
5742	1.3401970419681\\
5743	1.24826486651964\\
5744	1.19779157417811\\
5745	1.18454691491549\\
5746	1.17023765218167\\
5747	1.17110727012785\\
5748	1.19873527910845\\
5749	1.20296739703471\\
5750	1.21116334161797\\
5751	1.19375052617038\\
5752	1.20802076600923\\
5753	1.22297319114427\\
5754	1.20191039211255\\
5755	1.22139480617933\\
5756	1.23180192903231\\
5757	1.24622534193359\\
5758	1.27486434801798\\
5759	1.3376843091237\\
5760	1.44842229558077\\
5761	1.52453791509157\\
5762	1.5809775329307\\
5763	1.58897283172159\\
5764	1.57419868572254\\
5765	1.50264083035475\\
5766	1.36313985408534\\
5767	1.28041084101333\\
5768	1.23543293154148\\
5769	1.25706239198116\\
5770	1.28892914072334\\
5771	1.30699567796741\\
5772	1.33331000907527\\
5773	1.37344121539491\\
5774	1.38556095517996\\
5775	1.42977511474274\\
5776	1.42985517086575\\
5777	1.39787320229039\\
5778	1.35498583078723\\
5779	1.35588477449822\\
5780	1.35041598156146\\
5781	1.37608262429397\\
5782	1.37732730046673\\
5783	1.43087523554639\\
5784	1.57899557756904\\
5785	1.6517181921652\\
5786	1.59092565801811\\
5787	1.62774765936544\\
5788	1.64703012260402\\
5789	1.65405676786047\\
5790	1.60130870579257\\
5791	1.54582361190645\\
5792	1.46423660166304\\
5793	1.3781005850287\\
5794	1.32277298260219\\
5795	1.2926565201257\\
5796	1.30551433909178\\
5797	1.32308602621241\\
5798	1.33375316059592\\
5799	1.33424151896849\\
5800	1.32568880327332\\
5801	1.2820414833445\\
5802	1.25393490297106\\
5803	1.23974602084725\\
5804	1.26198972976862\\
5805	1.27866069066774\\
5806	1.31232566045382\\
5807	1.35040058887513\\
5808	1.42971626700813\\
5809	1.50579917981056\\
5810	1.55894276301035\\
5811	1.58576888105651\\
5812	1.5872827042239\\
5813	1.58438574239399\\
5814	1.56343109047493\\
5815	1.54560684339355\\
5816	1.46324900974792\\
5817	1.39234482821837\\
5818	1.35485851920503\\
5819	1.3287628335157\\
5820	1.35499376035937\\
5821	1.41532770671408\\
5822	1.50210036562348\\
5823	1.55670528481722\\
5824	1.54541576085679\\
5825	1.45307367049636\\
5826	1.36208460898171\\
5827	1.26443908826209\\
5828	1.267337748578\\
5829	1.26588035095846\\
5830	1.25296212696964\\
5831	1.29627469613704\\
5832	1.37954066203569\\
5833	1.43523422848389\\
5834	1.47195077473596\\
5835	1.47391719490485\\
5836	1.45448786215656\\
5837	1.36247730197042\\
5838	1.19656569126235\\
5839	1.11631788056405\\
5840	1.11136244123613\\
5841	1.13140099695223\\
5842	1.14470740689901\\
5843	1.15660102697161\\
5844	1.16539911701212\\
5845	1.1651216024041\\
5846	1.15654056786033\\
5847	1.14821710317504\\
5848	1.13468228610589\\
5849	1.11787106510119\\
5850	1.10391495610845\\
5851	1.08120069244967\\
5852	1.09948168633151\\
5853	1.11260845752486\\
5854	1.12036684355616\\
5855	1.18746772939355\\
5856	1.25693787343719\\
5857	1.31656775047032\\
5858	1.35822007813894\\
5859	1.37194552509346\\
5860	1.36056761068959\\
5861	1.29889106701967\\
5862	1.18203120086223\\
5863	1.08414306941227\\
5864	1.07700927025048\\
5865	1.09155187176332\\
5866	1.12648669596175\\
5867	1.13583658068826\\
5868	1.15758430339695\\
5869	1.16354850685126\\
5870	1.1596230287892\\
5871	1.16442807559557\\
5872	1.14733291422169\\
5873	1.12650849873158\\
5874	1.10960632103904\\
5875	1.11057383766464\\
5876	1.11044628641042\\
5877	1.10811693366603\\
5878	1.12187328085963\\
5879	1.18177823579072\\
5880	1.26391221428105\\
5881	1.31630228879579\\
5882	1.34893799697681\\
5883	1.3681620642827\\
5884	1.3458414812721\\
5885	1.28837921619505\\
5886	1.17473335316091\\
5887	1.08026358750502\\
5888	1.07559771877615\\
5889	1.09313862128104\\
5890	1.1191026130816\\
5891	1.14973327520784\\
5892	1.19156268073116\\
5893	1.21425800291035\\
5894	1.22991687507756\\
5895	1.22589865464954\\
5896	1.18745309985811\\
5897	1.14614640788419\\
5898	1.1348680929746\\
5899	1.13773304291437\\
5900	1.10131118578349\\
5901	1.10056741762111\\
5902	1.1384744124771\\
5903	1.24253504700937\\
5904	1.3058739313817\\
5905	1.35917322252037\\
5906	1.37024330973227\\
5907	1.37981721843537\\
5908	1.37247255899177\\
5909	1.30617275144677\\
5910	1.17930684913725\\
5911	1.07651353758528\\
5912	1.07456476284438\\
5913	1.08293092780262\\
5914	1.10162108974457\\
5915	1.11468061953916\\
5916	1.15836678216867\\
5917	1.16879001881938\\
5918	1.17871087902505\\
5919	1.16630708813726\\
5920	1.12636404805949\\
5921	1.10424215884954\\
5922	1.05847068200151\\
5923	1.05862476807818\\
5924	1.0435211169413\\
5925	1.06583050263021\\
5926	1.08775551475772\\
5927	1.16458799768259\\
5928	1.23547070162894\\
5929	1.28061690546641\\
5930	1.31796478025354\\
5931	1.32900004074685\\
5932	1.31582752334871\\
5933	1.27543839109679\\
5934	1.16273857711909\\
5935	1.06218355581919\\
5936	1.03022387452698\\
5937	1.02253738616966\\
5938	1.03970365329546\\
5939	1.05601974619787\\
5940	1.10180546878053\\
5941	1.1132680619926\\
5942	1.12398585528369\\
5943	1.13641874303321\\
5944	1.12207561363152\\
5945	1.10102923270691\\
5946	1.09174807632419\\
5947	1.11146812437987\\
5948	1.09617047926679\\
5949	1.10798960600717\\
5950	1.14577312650666\\
5951	1.15282352972269\\
5952	1.14937252999667\\
5953	1.22998859307845\\
5954	1.28636649569764\\
5955	1.30705328552898\\
5956	1.33260309517695\\
5957	1.32010918232318\\
5958	1.27044449245754\\
5959	1.24109621964143\\
5960	1.18015506863458\\
5961	1.14369354232299\\
5962	1.08634688933081\\
5963	1.06575715456617\\
5964	1.07898969503636\\
5965	1.11065768839045\\
5966	1.12391784592702\\
5967	1.137606959095\\
5968	1.12950818346217\\
5969	1.08163977827933\\
5970	1.09158872546733\\
5971	1.09060222864698\\
5972	1.06662591528974\\
5973	1.08727439935209\\
5974	1.09054434965721\\
5975	1.11805329530521\\
5976	1.15962480165253\\
5977	1.2006740370044\\
5978	1.21320624076844\\
5979	1.22438771886312\\
5980	1.2338737459194\\
5981	1.22629635890309\\
5982	1.19076560866288\\
5983	1.19855154216158\\
5984	1.16919392967301\\
5985	1.15702277685768\\
5986	1.14089629979018\\
5987	1.13640510453971\\
5988	1.16066070538113\\
5989	1.20587213577556\\
5990	1.23338370143521\\
5991	1.22357037280189\\
5992	1.26978755757116\\
5993	1.28673879213862\\
5994	1.21388307776201\\
5995	1.17066682182889\\
5996	1.1394785501454\\
5997	1.13889333366554\\
5998	1.15745863390384\\
5999	1.22926841923568\\
6000	1.32479479079691\\
6001	1.36478858247178\\
6002	1.37903759264649\\
6003	1.38747876493051\\
6004	1.34714189412515\\
6005	1.2717794789387\\
6006	1.12166962045505\\
6007	1.02835782266586\\
6008	1.01626684684203\\
6009	1.02271592357479\\
6010	1.02712649148635\\
6011	1.02616922826202\\
6012	1.06900017267024\\
6013	1.10174748485759\\
6014	1.11716098100827\\
6015	1.12415304688421\\
6016	1.09141509721763\\
6017	1.05000368320096\\
6018	1.0374995527732\\
6019	1.04046010329966\\
6020	1.06420219878623\\
6021	1.07885148383994\\
6022	1.1043960428336\\
6023	1.14735528107585\\
6024	1.22691682975854\\
6025	1.27956120900612\\
6026	1.31020875241832\\
6027	1.31661986529784\\
6028	1.31348174538791\\
6029	1.25032748965298\\
6030	1.12231456277701\\
6031	1.0573693651132\\
6032	1.05777487037285\\
6033	1.10030652177927\\
6034	1.07967636410607\\
6035	1.06651108704169\\
6036	1.11012940970004\\
6037	1.12126844307729\\
6038	1.11352020116095\\
6039	1.10954756534984\\
6040	1.08768542674534\\
6041	1.0524387552992\\
6042	1.03632052009409\\
6043	1.07342849763167\\
6044	1.06288498650172\\
6045	1.08745289994388\\
6046	1.09640671600792\\
6047	1.17134316839144\\
6048	1.24897424955856\\
6049	1.30015402267315\\
6050	1.33311624613131\\
6051	1.34558700865597\\
6052	1.336144079424\\
6053	1.27053655685253\\
6054	1.14257266731081\\
6055	1.04891996453425\\
6056	1.03364959432722\\
6057	1.06904519574322\\
6058	1.09213732937494\\
6059	1.10532546019099\\
6060	1.12869217742457\\
6061	1.14741905463951\\
6062	1.15889831361385\\
6063	1.14122796413349\\
6064	1.10993835110732\\
6065	1.11265603690942\\
6066	1.06254685589474\\
6067	1.08957638347858\\
6068	1.07864332670375\\
6069	1.06886800722926\\
6070	1.12384541352968\\
6071	1.17660654743931\\
6072	1.26057443816591\\
6073	1.31081636878636\\
6074	1.34745123278675\\
6075	1.35337381043442\\
6076	1.33039775506741\\
6077	1.26565224377484\\
6078	1.12236037875668\\
6079	1.06157945531746\\
6080	1.05790408010644\\
6081	1.04370457175235\\
6082	1.05654786350356\\
6083	1.04802636466725\\
6084	1.08616259191234\\
6085	1.12649112795149\\
6086	1.10957472410668\\
6087	1.12442563525481\\
6088	1.11733153015403\\
6089	1.07727398244711\\
6090	1.06963522818834\\
6091	1.09058190568092\\
6092	1.08735627929065\\
6093	1.08935939551936\\
6094	1.12812102392293\\
6095	1.18040528723248\\
6096	1.27665801441634\\
6097	1.33211691920725\\
6098	1.359730816202\\
6099	1.36545276805536\\
6100	1.35060214287512\\
6101	1.28682629096096\\
6102	1.14428221440036\\
6103	1.08637831144836\\
6104	1.06030982881213\\
6105	1.08772209753902\\
6106	1.11890876596549\\
6107	1.17206645635426\\
6108	1.20269346410765\\
6109	1.21802988990734\\
6110	1.22325818316481\\
6111	1.23861236771138\\
6112	1.20624846318948\\
6113	1.15878884757743\\
6114	1.12771533594868\\
6115	1.11505857643594\\
6116	1.08355480754121\\
6117	1.12216105028957\\
6118	1.13708045492727\\
6119	1.19129918860632\\
6120	1.28973000782803\\
6121	1.3608403105672\\
6122	1.39348387301045\\
6123	1.42825207811461\\
6124	1.45036297876256\\
6125	1.431637893475\\
6126	1.3730390242025\\
6127	1.3234361627931\\
6128	1.24492193264778\\
6129	1.22523807973216\\
6130	1.24271831778083\\
6131	1.26923015280247\\
6132	1.31857164737088\\
6133	1.36064065534514\\
6134	1.43042639634228\\
6135	1.44842752066435\\
6136	1.40980035645572\\
6137	1.33660313983074\\
6138	1.28756867835219\\
6139	1.26706012551224\\
6140	1.2630487370346\\
6141	1.28556742210903\\
6142	1.28667642124576\\
6143	1.35105852601948\\
6144	1.39920819426311\\
6145	1.43568656069422\\
6146	1.48350071541414\\
6147	1.5259872280348\\
6148	1.50820972395439\\
6149	1.49538219625831\\
6150	1.44014480100143\\
6151	1.40728367893047\\
6152	1.33824340026268\\
6153	1.26811371140451\\
6154	1.24427660659886\\
6155	1.23011889955567\\
6156	1.25714074859751\\
6157	1.32528508441351\\
6158	1.36704146857922\\
6159	1.40224131353645\\
6160	1.39237867810958\\
6161	1.33985175859981\\
6162	1.31494513930004\\
6163	1.26293439400911\\
6164	1.22185704867832\\
6165	1.24917837666145\\
6166	1.25336444089468\\
6167	1.31105004801059\\
6168	1.39272021035613\\
6169	1.44233253878802\\
6170	1.46347373189142\\
6171	1.47283428509874\\
6172	1.44929268097609\\
6173	1.35067698585689\\
6174	1.18066787221276\\
6175	1.08178084191956\\
6176	1.06098054869838\\
6177	1.05636294274117\\
6178	1.0589506559257\\
6179	1.07013294906244\\
6180	1.11663067339669\\
6181	1.12752656381411\\
6182	1.12450394608213\\
6183	1.11853091226587\\
6184	1.10041441623096\\
6185	1.07661140038863\\
6186	1.04879083077439\\
6187	1.0403406776939\\
6188	1.04296457362708\\
6189	1.07610584627451\\
6190	1.11059202419838\\
6191	1.17783596480805\\
6192	1.27601713496547\\
6193	1.31607136031382\\
6194	1.35341744603008\\
6195	1.35138055424374\\
6196	1.33716289316041\\
6197	1.26584526671437\\
6198	1.11999379658759\\
6199	1.04115272999355\\
6200	1.03111949267814\\
6201	1.03526393389279\\
6202	1.07010684022263\\
6203	1.07702100184578\\
6204	1.12602337834447\\
6205	1.12680613811287\\
6206	1.13943831036479\\
6207	1.13085972592705\\
6208	1.08869784246234\\
6209	1.05478517286334\\
6210	1.03445610089754\\
6211	1.04156224706206\\
6212	1.0267209985147\\
6213	1.0569340706636\\
6214	1.09015673144429\\
6215	1.15369247150126\\
6216	1.26406558668682\\
6217	1.30830238327794\\
6218	1.3574709687194\\
6219	1.3545359207062\\
6220	1.34529857530903\\
6221	1.28452809977606\\
6222	1.13505862095601\\
6223	1.06018121914627\\
6224	1.06079363438171\\
6225	1.08662937534408\\
6226	1.09069267867392\\
6227	1.09380175025575\\
6228	1.14698781157204\\
6229	1.16457028762464\\
6230	1.16752522613824\\
6231	1.15140659166339\\
6232	1.14360928618166\\
6233	1.11090849013535\\
6234	1.06752203085328\\
6235	1.0660478212801\\
6236	1.05177412379669\\
6237	1.08836229888804\\
6238	1.13117857282727\\
6239	1.19743101334089\\
6240	1.27233062058796\\
6241	1.3405762519598\\
6242	1.37287331299431\\
6243	1.37333943463659\\
6244	1.35523204941726\\
6245	1.28679679179803\\
6246	1.15210127989248\\
6247	1.05610422175811\\
6248	1.02328918272038\\
6249	1.0056835200973\\
6250	1.00059279973531\\
6251	0.97507528725633\\
6252	1.01539917642716\\
6253	1.02382828167901\\
6254	1.0204997673555\\
6255	1.02874798914277\\
6256	1.02842637889958\\
6257	1.003221943468\\
6258	1.00906794426239\\
6259	1.02481714065144\\
6260	1.01466166234438\\
6261	1.06512463445152\\
6262	1.10900063712369\\
6263	1.16571909896689\\
6264	1.23516422725939\\
6265	1.28348310828458\\
6266	1.33421539911658\\
6267	1.34086978635431\\
6268	1.32742756609383\\
6269	1.26240126683649\\
6270	1.12474082129364\\
6271	1.03164483894063\\
6272	1.01982461541432\\
6273	1.00652577682091\\
6274	1.01720748497982\\
6275	1.02243355366693\\
6276	1.04608035205603\\
6277	1.05603602531326\\
6278	1.04791627831746\\
6279	1.05401417447043\\
6280	1.04920638100652\\
6281	1.03987776459649\\
6282	1.04248504705885\\
6283	1.0687470613805\\
6284	1.04620568071051\\
6285	1.08482727601794\\
6286	1.09811387572476\\
6287	1.14341263381382\\
6288	1.2350023808362\\
6289	1.2981220981621\\
6290	1.35632714322208\\
6291	1.38652180526813\\
6292	1.40009152398466\\
6293	1.39508588806609\\
6294	1.35641050551636\\
6295	1.28204189592731\\
6296	1.23315566424165\\
6297	1.17475253730895\\
6298	1.19433609390827\\
6299	1.20553811036044\\
6300	1.22153658277807\\
6301	1.22526043163222\\
6302	1.2483741798257\\
6303	1.25022299978474\\
6304	1.24119786007172\\
6305	1.21581292871173\\
6306	1.18469865011385\\
6307	1.18963553267854\\
6308	1.15385926292798\\
6309	1.20946490421989\\
6310	1.22044079720181\\
6311	1.28311960131124\\
6312	1.35242260801567\\
6313	1.4152817383348\\
6314	1.47582541743518\\
6315	1.54356540675367\\
6316	1.58938054373749\\
6317	1.60679665978099\\
6318	1.58196283678221\\
6319	1.51826973181335\\
6320	1.45009900539192\\
6321	1.3595650200756\\
6322	1.32644423703973\\
6323	1.33172374843957\\
6324	1.4026596764631\\
6325	1.51880094354691\\
6326	1.5924808246851\\
6327	1.63002213589022\\
6328	1.5962092652692\\
6329	1.48964086800758\\
6330	1.35386540900063\\
6331	1.29824960134502\\
6332	1.2697692956939\\
6333	1.28024253432463\\
6334	1.28461633319719\\
6335	1.36597822794308\\
6336	1.40965434994522\\
6337	1.46216916702351\\
6338	1.48762450170886\\
6339	1.51114242431265\\
6340	1.5019743392863\\
6341	1.40224780567657\\
6342	1.20639935923502\\
6343	1.10677913582642\\
6344	1.07213273481938\\
6345	1.05625510620986\\
6346	1.06951219926787\\
6347	1.07585959519414\\
6348	1.12317265120748\\
6349	1.1296675248566\\
6350	1.13172392171373\\
6351	1.13086035869675\\
6352	1.11263081019853\\
6353	1.10071524707479\\
6354	1.0641804341889\\
6355	1.05974108240756\\
6356	1.0547183621253\\
6357	1.07390513505327\\
6358	1.1073096936615\\
6359	1.18368237389488\\
6360	1.24850245359162\\
6361	1.30888703433087\\
6362	1.34637111717395\\
6363	1.36293965426417\\
6364	1.33876862211356\\
6365	1.2691763502699\\
6366	1.11850237493455\\
6367	1.02360173869168\\
6368	1.02511567763761\\
6369	1.04849241440744\\
6370	1.05786192018167\\
6371	1.07619231696406\\
6372	1.11613846045216\\
6373	1.13095913344736\\
6374	1.13104783995809\\
6375	1.11642625522885\\
6376	1.0861961940058\\
6377	1.05612535736652\\
6378	1.0435754824804\\
6379	1.02429948747093\\
6380	1.01284185188271\\
6381	1.06091694920792\\
6382	1.09140962127799\\
6383	1.15993819666373\\
6384	1.25520216686525\\
6385	1.32424633704283\\
6386	1.38276173377805\\
6387	1.40383550218864\\
6388	1.396857166497\\
6389	1.33549590379938\\
6390	1.18054798560843\\
6391	1.0841891163432\\
6392	1.07062072550032\\
6393	1.0669681010496\\
6394	1.07894532467343\\
6395	1.05547985580713\\
6396	1.07076327667464\\
6397	1.07164606045432\\
6398	1.08509892424757\\
6399	1.08816544315345\\
6400	1.09056934656639\\
6401	1.06342602307418\\
6402	1.08169116356397\\
6403	1.08876736935036\\
6404	1.07665504715491\\
6405	1.12382761091477\\
6406	1.15058808546595\\
6407	1.22224361334871\\
6408	1.30763003257696\\
6409	1.36521056448212\\
6410	1.39345205910147\\
6411	1.39926903500698\\
6412	1.38669492789855\\
6413	1.30606873480122\\
6414	1.15066832965282\\
6415	1.04890706033846\\
6416	1.03919585150504\\
6417	1.04677761128502\\
6418	1.06645927095258\\
6419	1.08165657769298\\
6420	1.13945879419626\\
6421	1.15679044298635\\
6422	1.17216953095948\\
6423	1.17372075740324\\
6424	1.16192648818333\\
6425	1.13650926808378\\
6426	1.09767289110602\\
6427	1.0888806500235\\
6428	1.07068577027992\\
6429	1.12065392151705\\
6430	1.18027551976377\\
6431	1.23238399747165\\
6432	1.2866599207292\\
6433	1.35111434009088\\
6434	1.39727528562092\\
6435	1.4152538640466\\
6436	1.39784717918361\\
6437	1.32896363715623\\
6438	1.16729278572394\\
6439	1.0557261221014\\
6440	1.05652997107488\\
6441	1.03958069307119\\
6442	1.06183240848527\\
6443	1.05986732720205\\
6444	1.05915659132188\\
6445	1.06552326424243\\
6446	1.05564741488684\\
6447	1.04975711557905\\
6448	1.02390639130751\\
6449	1.00743468819342\\
6450	1.00520263341083\\
6451	1.00147274727354\\
6452	1.02189511073797\\
6453	1.06522029500804\\
6454	1.06954918322061\\
6455	1.11697818325123\\
6456	1.19068440665822\\
6457	1.19987207687246\\
6458	1.2516398162406\\
6459	1.27969996655668\\
6460	1.28952704442446\\
6461	1.28450521566483\\
6462	1.23614002651555\\
6463	1.17410814087044\\
6464	1.12168097536531\\
6465	1.08657262610254\\
6466	1.06985940802181\\
6467	1.0811698917824\\
6468	1.1347521662066\\
6469	1.16590469344863\\
6470	1.19535333595407\\
6471	1.2084325391962\\
6472	1.18359640864364\\
6473	1.12548897465758\\
6474	1.09947214097822\\
6475	1.08176567244242\\
6476	1.06680200457655\\
6477	1.09058277901997\\
6478	1.12580729712709\\
6479	1.17756294098773\\
6480	1.2608748935644\\
6481	1.37621991794089\\
6482	1.4332060190306\\
6483	1.46631424720739\\
6484	1.47553989091465\\
6485	1.46236134174885\\
6486	1.40439536029458\\
6487	1.36500440021416\\
6488	1.33844322115599\\
6489	1.30502103963503\\
6490	1.30812624201994\\
6491	1.32939135662171\\
6492	1.37115015196389\\
6493	1.45355714095344\\
6494	1.4886351809948\\
6495	1.48111452370008\\
6496	1.43969476954237\\
6497	1.31199470069318\\
6498	1.18413965442275\\
6499	1.07784686310774\\
6500	1.05907048229507\\
6501	1.14837224729582\\
6502	1.19482294068329\\
6503	1.26669129173711\\
6504	1.3557140970899\\
6505	1.40646378046396\\
6506	1.45277781216671\\
6507	1.45459835894429\\
6508	1.4307583616594\\
6509	1.32722070155798\\
6510	1.16354212517854\\
6511	1.05787684044037\\
6512	1.04619865312199\\
6513	1.06153238358974\\
6514	1.07569129643722\\
6515	1.07675027897413\\
6516	1.1123057228909\\
6517	1.10885745184698\\
6518	1.10091085672828\\
6519	1.03534829360777\\
6520	1.0146662921039\\
6521	0.968181360991986\\
6522	0.953340839431368\\
6523	0.966718322843213\\
6524	0.989141898021704\\
6525	1.02091593940525\\
6526	1.06306912545735\\
6527	1.12761281205554\\
6528	1.21857610042671\\
6529	1.29287428809378\\
6530	1.34872478889377\\
6531	1.3411028215546\\
6532	1.34122349366009\\
6533	1.27697486402254\\
6534	1.06171837896335\\
6535	0.972001227496914\\
6536	0.957372389217668\\
6537	0.951803328308159\\
6538	0.974665831623613\\
6539	0.994080137280165\\
6540	1.01028131838187\\
6541	1.01280003179309\\
6542	1.05316783730325\\
6543	1.0464016290494\\
6544	1.03401488664035\\
6545	0.9919435322977\\
6546	0.974019082196425\\
6547	0.962525427621652\\
6548	0.966396311433782\\
6549	0.997435296249056\\
6550	1.03752215402897\\
6551	1.10403313810877\\
6552	1.20991494825769\\
6553	1.26310349113423\\
6554	1.32077104046606\\
6555	1.34201045454009\\
6556	1.32154917291404\\
6557	1.25711820616351\\
6558	1.10681082470751\\
6559	1.00303856246414\\
6560	1.02707005678177\\
6561	1.03733824741815\\
6562	1.06804005753761\\
6563	1.05819545602494\\
6564	1.1028996244047\\
6565	1.1006121154901\\
6566	1.10327200925608\\
6567	1.09796082841257\\
6568	1.07828091286787\\
6569	1.0262524457742\\
6570	0.996889087741182\\
6571	1.00709860226146\\
6572	1.01949688616563\\
6573	1.01273060640143\\
6574	1.03566121558723\\
6575	1.10524254464352\\
6576	1.18205784854624\\
6577	1.23675427129827\\
6578	1.26390832129631\\
6579	1.2900256994354\\
6580	1.27516742144448\\
6581	1.20974111506368\\
6582	1.06453032497273\\
6583	0.9717649230554\\
6584	0.966932372127469\\
6585	0.965721249812264\\
6586	0.975361095887292\\
6587	0.974085724355514\\
6588	0.991157129573679\\
6589	0.998546382756248\\
6590	0.990945045985716\\
6591	0.982027036702531\\
6592	0.976214682038873\\
6593	0.965597683521814\\
6594	0.969216403666555\\
6595	0.972754029985539\\
6596	0.977966626348916\\
6597	0.995353790690302\\
6598	1.0289972906188\\
6599	1.10074297291895\\
6600	1.17611771804873\\
6601	1.22196575920732\\
6602	1.25571212957983\\
6603	1.27784807928109\\
6604	1.26292805954421\\
6605	1.203996115373\\
6606	1.06818789429423\\
6607	0.962063296549035\\
6608	0.977326963047017\\
6609	0.995620510095195\\
6610	1.02935597522665\\
6611	1.02061400902273\\
6612	1.07952338207332\\
6613	1.09257008437647\\
6614	1.09422362423743\\
6615	1.09479707971062\\
6616	1.06500122875222\\
6617	1.02936005442554\\
6618	1.01570387942216\\
6619	1.0156528248611\\
6620	0.999163839785109\\
6621	1.0299421730448\\
6622	1.06214803024598\\
6623	1.11491709384988\\
6624	1.22073734551062\\
6625	1.29499796668145\\
6626	1.36315224818882\\
6627	1.40455797039164\\
6628	1.4610325034006\\
6629	1.47842148317372\\
6630	1.4211420219806\\
6631	1.34687935451874\\
6632	1.32526732443771\\
6633	1.33523733085211\\
6634	1.35229751311211\\
6635	1.36378854590468\\
6636	1.40626169933279\\
6637	1.42566813840641\\
6638	1.43771422185611\\
6639	1.3865811321534\\
6640	1.33176549422743\\
6641	1.24584239115482\\
6642	1.21165192524454\\
6643	1.22295827150337\\
6644	1.20886470558974\\
6645	1.22859046759303\\
6646	1.27631999517119\\
6647	1.31994316516016\\
6648	1.4064365074846\\
6649	1.47637547049649\\
6650	1.51738956116994\\
6651	1.52421971262825\\
6652	1.52026886626543\\
6653	1.48474594519533\\
6654	1.43000408768385\\
6655	1.35703554121798\\
6656	1.31558004551559\\
6657	1.26120129361356\\
6658	1.22232426844006\\
6659	1.19725150667439\\
6660	1.19813698254526\\
6661	1.2285323397458\\
6662	1.26835856061023\\
6663	1.26550275754198\\
6664	1.23366153116343\\
6665	1.17765489586021\\
6666	1.10384797983668\\
6667	1.10720834450425\\
6668	1.11754591859181\\
6669	1.10522173023921\\
6670	1.14042736484063\\
6671	1.19736634260638\\
6672	1.28298096604447\\
6673	1.35006931068142\\
6674	1.39436881100193\\
6675	1.39545136445762\\
6676	1.37656072209548\\
6677	1.29861035441242\\
6678	1.11534421401455\\
6679	0.989740393677629\\
6680	0.996439112395265\\
6681	0.991134463374204\\
6682	1.01813670806303\\
6683	1.03172031083869\\
6684	1.05820127099352\\
6685	1.03415387762129\\
6686	1.03661115447253\\
6687	1.07045434793573\\
6688	1.05746018526789\\
6689	1.03421673059398\\
6690	1.03469560491722\\
6691	1.07826780902852\\
6692	1.11409373433527\\
6693	1.16327259088569\\
6694	1.20322366181909\\
6695	1.28097851957772\\
6696	1.38393179420443\\
6697	1.48593377113569\\
6698	1.53834970928921\\
6699	1.56558916206962\\
6700	1.53645185387728\\
6701	1.42889130852181\\
6702	1.23309110220794\\
6703	1.08352016234384\\
6704	1.1127055714725\\
6705	1.13392548025908\\
6706	1.16389088303529\\
6707	1.15661225978082\\
6708	1.24433275789985\\
6709	1.21557665195322\\
6710	1.24144727270521\\
6711	1.2475966186886\\
6712	1.24011880046443\\
6713	1.17393181241229\\
6714	1.10457197869373\\
6715	1.10847768914979\\
6716	1.10983348046282\\
6717	1.1393993600164\\
6718	1.18133739779865\\
6719	1.235507223003\\
6720	1.33126258717378\\
6721	1.39650505108989\\
6722	1.43994378449407\\
6723	1.44056023770341\\
6724	1.41372943364306\\
6725	1.33688301877372\\
6726	1.18270347729878\\
6727	1.08817262712472\\
6728	1.07567862042529\\
6729	1.04085748795842\\
6730	1.04607098835442\\
6731	1.03526269560239\\
6732	1.0294420172673\\
6733	1.05013222146517\\
6734	1.08024223923421\\
6735	1.1072310867184\\
6736	1.12905648175428\\
6737	1.10733714565771\\
6738	1.08782017910056\\
6739	1.10504124164922\\
6740	1.11157900284245\\
6741	1.15502953263868\\
6742	1.17198091377341\\
6743	1.24323410758501\\
6744	1.33661360689791\\
6745	1.40487173403863\\
6746	1.44115160086132\\
6747	1.46252645664046\\
6748	1.45062743644539\\
6749	1.36568564425024\\
6750	1.19449021941435\\
6751	1.08871494267826\\
6752	1.05662092797719\\
6753	1.09417255438799\\
6754	1.09296409702309\\
6755	1.11420920700696\\
6756	1.20140443986314\\
6757	1.19332758617604\\
6758	1.16522375894883\\
6759	1.14680374608857\\
6760	1.13650829841715\\
6761	1.09499575094029\\
6762	1.10776188516708\\
6763	1.08681959981613\\
6764	1.11242844480959\\
6765	1.1384449799993\\
6766	1.18297722484497\\
6767	1.23204060617058\\
6768	1.3409530198561\\
6769	1.41685419230562\\
6770	1.47929966747219\\
6771	1.4879489898328\\
6772	1.47003481546988\\
6773	1.36383820309061\\
6774	1.17628742414817\\
6775	1.04615382733413\\
6776	1.06810896178931\\
6777	1.0766482968258\\
6778	1.09301365261411\\
6779	1.06360550716379\\
6780	1.12454575229459\\
6781	1.13174936458881\\
6782	1.13038205275117\\
6783	1.14158321667999\\
6784	1.12463489047063\\
6785	1.08686408817342\\
6786	1.04904596189162\\
6787	1.03149061014403\\
6788	1.02147845839543\\
6789	1.06319031328911\\
6790	1.07594177844274\\
6791	1.1078125314193\\
6792	1.2128541373209\\
6793	1.30201435157995\\
6794	1.35358548160911\\
6795	1.38417977696096\\
6796	1.39255985051169\\
6797	1.36590750810118\\
6798	1.30223819084585\\
6799	1.22073479911604\\
6800	1.1716833749041\\
6801	1.14864986558153\\
6802	1.14152920743883\\
6803	1.14952527380694\\
6804	1.17455271985975\\
6805	1.18674163919165\\
6806	1.22681536083519\\
6807	1.23293206026219\\
6808	1.21746829889102\\
6809	1.185672205983\\
6810	1.1685146271354\\
6811	1.12583216730776\\
6812	1.15132432682927\\
6813	1.18434723790799\\
6814	1.23059473177806\\
6815	1.27475610249237\\
6816	1.32448151483397\\
6817	1.39200846167806\\
6818	1.42539448649117\\
6819	1.43392287391302\\
6820	1.43447275226822\\
6821	1.40449324407957\\
6822	1.36485168507107\\
6823	1.30323442787665\\
6824	1.26803978159475\\
6825	1.24232058772887\\
6826	1.23212433388416\\
6827	1.2546734352652\\
6828	1.27907713128063\\
6829	1.31906698045827\\
6830	1.35102753246507\\
6831	1.35337925982347\\
6832	1.33623788921047\\
6833	1.29221617639035\\
6834	1.25922631765158\\
6835	1.23010685601932\\
6836	1.28405405537422\\
6837	1.31118539330367\\
6838	1.30940430563317\\
6839	1.35040461803892\\
6840	1.41162934222784\\
6841	1.46826413480329\\
6842	1.50447551030341\\
6843	1.48603789873669\\
6844	1.4462016474513\\
6845	1.34531453481803\\
6846	1.18971406850158\\
6847	1.10628752075115\\
6848	1.12107124400362\\
6849	1.15341841040044\\
6850	1.15145286814253\\
6851	1.13813557787785\\
6852	1.16063061824991\\
6853	1.15468712140335\\
6854	1.1585751195495\\
6855	1.15579847897405\\
6856	1.14400480235947\\
6857	1.10576690878591\\
6858	1.07943299662117\\
6859	1.07002945794879\\
6860	1.10554883968955\\
6861	1.13577321315801\\
6862	1.14659581354189\\
6863	1.21337867464061\\
6864	1.31687066907833\\
6865	1.41213216127112\\
6866	1.46243282553604\\
6867	1.46033559666298\\
6868	1.45196885758168\\
6869	1.35656220369682\\
6870	1.19329837142654\\
6871	1.05621824963868\\
6872	1.08262425756697\\
6873	1.10922887085\\
6874	1.14650891007591\\
6875	1.14385133863084\\
6876	1.19310148384587\\
6877	1.16687713577619\\
6878	1.15848928243148\\
6879	1.13743961958386\\
6880	1.11129335403783\\
6881	1.05810327364324\\
6882	1.02876846069083\\
6883	1.00776107351858\\
6884	1.02885280347141\\
6885	1.06592686594645\\
6886	1.08137788075266\\
6887	1.14412944144124\\
6888	1.21736679236901\\
6889	1.22725581080623\\
6890	1.26210115514195\\
6891	1.26827516103396\\
6892	1.23879804930237\\
6893	1.17743677800395\\
6894	1.03784036751909\\
6895	0.94088416929177\\
6896	0.999429611507393\\
6897	1.01431655137818\\
6898	1.01266210450956\\
6899	1.01399080766388\\
6900	1.04199240055136\\
6901	1.04450627652745\\
6902	1.04842995397843\\
6903	1.04029991087836\\
6904	1.02665413714874\\
6905	0.994330478815297\\
6906	0.97010776729696\\
6907	0.950309371831047\\
6908	0.984427497075079\\
6909	1.0311092857701\\
6910	1.05456159736668\\
6911	1.11043089315896\\
6912	1.21560880139525\\
6913	1.28301193945745\\
6914	1.34264503710169\\
6915	1.35703329946491\\
6916	1.35717988984943\\
6917	1.30091964139476\\
6918	1.15161521146225\\
6919	1.02706240569151\\
6920	1.03736260182257\\
6921	1.03203931935318\\
6922	1.03404140340997\\
6923	1.04645906384557\\
6924	1.06512219752681\\
6925	1.0754379248127\\
6926	1.09042300223007\\
6927	1.08479861652835\\
6928	1.08217902966293\\
6929	1.02969001269196\\
6930	0.992431622124358\\
6931	1.00655407317896\\
6932	1.02346021408403\\
6933	1.04847029064283\\
6934	1.06516121859179\\
6935	1.10553919341309\\
6936	1.22267018536434\\
6937	1.29648243326967\\
6938	1.33725948585811\\
6939	1.34538768537787\\
6940	1.3197033411302\\
6941	1.25443787983769\\
6942	1.11955460878253\\
6943	1.00876635299961\\
6944	1.00088609972708\\
6945	1.01168624018989\\
6946	1.02718902799072\\
6947	1.05932214040482\\
6948	1.0982458112139\\
6949	1.10648409953122\\
6950	1.1202625618742\\
6951	1.11429458682805\\
6952	1.08728674435378\\
6953	1.06137644167798\\
6954	1.02179831326305\\
6955	1.01328224905639\\
6956	1.04088374727933\\
6957	1.10704093437705\\
6958	1.1237992440107\\
6959	1.17869015578986\\
6960	1.29541910390175\\
6961	1.37890615271704\\
6962	1.47367469224211\\
6963	1.51029997186562\\
6964	1.54548378909121\\
6965	1.55479422578487\\
6966	1.46492129408149\\
6967	1.35257623637463\\
6968	1.29809689644421\\
6969	1.29854143644691\\
6970	1.26002325934467\\
6971	1.25582135561944\\
6972	1.32297772260869\\
6973	1.36154790661459\\
6974	1.39003332474836\\
6975	1.37369642692658\\
6976	1.34498267289797\\
6977	1.24605494148905\\
6978	1.14844846882675\\
6979	1.15473713351984\\
6980	1.22119711096354\\
6981	1.27081307429859\\
6982	1.26633022186902\\
6983	1.31878229212258\\
6984	1.42835542741908\\
6985	1.5076104141312\\
6986	1.59748136727985\\
6987	1.64524638938996\\
6988	1.64042581999128\\
6989	1.62970565627301\\
6990	1.57130553741902\\
6991	1.49025491031799\\
6992	1.43839899187725\\
6993	1.42569185822938\\
6994	1.41692303544156\\
6995	1.42182753380738\\
6996	1.46190510273933\\
6997	1.56228161502629\\
6998	1.5813937892952\\
6999	1.57322422744615\\
7000	1.56550569113687\\
7001	1.44014498890027\\
7002	1.33914413522088\\
7003	1.25345411008865\\
7004	1.27199479311024\\
7005	1.30789686073201\\
7006	1.34851944709077\\
7007	1.39725787265236\\
7008	1.54354399129661\\
7009	1.61824252799252\\
7010	1.63565672887708\\
7011	1.64914052194\\
7012	1.60893962006132\\
7013	1.49274729780713\\
7014	1.26134037701063\\
7015	1.09230571420108\\
7016	1.07192570619372\\
7017	1.07528368022789\\
7018	1.08720826557139\\
7019	1.09236344249377\\
7020	1.13056517463954\\
7021	1.12479946470986\\
7022	1.12880503954555\\
7023	1.12880366691605\\
7024	1.11965364349117\\
7025	1.07950540228618\\
7026	1.10173163023126\\
7027	1.115917746875\\
7028	1.1341791044516\\
7029	1.22229376146049\\
7030	1.2328862264288\\
7031	1.28844126319324\\
7032	1.36478840117516\\
7033	1.46198557362463\\
7034	1.52117866264049\\
7035	1.55706191288974\\
7036	1.53271367090381\\
7037	1.45028916796151\\
7038	1.26586965080933\\
7039	1.14190837677352\\
7040	1.12465335123485\\
7041	1.1273659102006\\
7042	1.13018451629937\\
7043	1.0964068325515\\
7044	1.14104832447419\\
7045	1.14951308681747\\
7046	1.16722406680986\\
7047	1.19077010251335\\
7048	1.19392003566595\\
7049	1.14751343958625\\
7050	1.150830716494\\
7051	1.17085666462176\\
7052	1.25232667478089\\
7053	1.30870079073767\\
7054	1.33206553527051\\
7055	1.39684265958892\\
7056	1.47455881362571\\
7057	1.56783254945078\\
7058	1.62166012291419\\
7059	1.64160936151336\\
7060	1.60486193122236\\
7061	1.50908860104034\\
7062	1.28972387665833\\
7063	1.13890642271139\\
7064	1.14733250016793\\
7065	1.16984774504266\\
7066	1.19617812624426\\
7067	1.19737843566587\\
7068	1.21605177298939\\
7069	1.23804363635746\\
7070	1.23623052346014\\
7071	1.20999717857827\\
7072	1.19019014609573\\
7073	1.15622639618842\\
7074	1.06764745855317\\
7075	1.08885904081132\\
7076	1.08437365204746\\
7077	1.14770529467111\\
7078	1.18852065960063\\
7079	1.26938820956142\\
7080	1.36271647648119\\
7081	1.4283164943781\\
7082	1.48496129157021\\
7083	1.50252933454902\\
7084	1.48204719680036\\
7085	1.4021679961566\\
7086	1.23529427734295\\
7087	1.11219602509224\\
7088	1.12146100434064\\
7089	1.13667434906374\\
7090	1.14891715763596\\
7091	1.14377467177243\\
7092	1.17578284086238\\
7093	1.16391003475416\\
7094	1.17728573696249\\
7095	1.15453612543353\\
7096	1.15192540198619\\
7097	1.13451644717194\\
7098	1.14466888456913\\
7099	1.14036902141265\\
7100	1.19866590556623\\
7101	1.23948796957708\\
7102	1.26810551566091\\
7103	1.3143893651321\\
7104	1.40489980758166\\
7105	1.50563200105343\\
7106	1.57004892804628\\
7107	1.59682086346622\\
7108	1.58125895181316\\
7109	1.49280893594251\\
7110	1.31183327000214\\
7111	1.21192972233685\\
7112	1.18176335146459\\
7113	1.14679925067967\\
7114	1.15029777239556\\
7115	1.14196647196734\\
7116	1.13020773802295\\
7117	1.15835365515408\\
7118	1.18832270711159\\
7119	1.18496767818014\\
7120	1.18099564984477\\
7121	1.16829484397023\\
7122	1.11744601776078\\
7123	1.18219759526424\\
7124	1.20543229998669\\
7125	1.26252792223447\\
7126	1.25278692000894\\
7127	1.22324669353896\\
7128	1.30858807521286\\
7129	1.38090878147762\\
7130	1.42875828703677\\
7131	1.45684924323796\\
7132	1.45504415411736\\
7133	1.42322829819961\\
7134	1.35424716259617\\
7135	1.2856539525844\\
7136	1.20102787980958\\
7137	1.18226501364758\\
7138	1.18247701948999\\
7139	1.16849944340784\\
7140	1.16179786905739\\
7141	1.25165954439495\\
7142	1.29360645674658\\
7143	1.30618514385679\\
7144	1.27968649141031\\
7145	1.21661791373529\\
7146	1.16425065104958\\
7147	1.20505020695239\\
7148	1.22680771272074\\
7149	1.27888650552904\\
7150	1.2714643627725\\
7151	1.28653772982765\\
7152	1.40892117028263\\
7153	1.50990560355304\\
7154	1.57490158711406\\
7155	1.59641867872458\\
7156	1.6125041902525\\
7157	1.60612061593765\\
7158	1.61612958414921\\
7159	1.56410612815446\\
7160	1.51095107722316\\
7161	1.45705897862537\\
7162	1.45991103933224\\
7163	1.4904206829563\\
7164	1.49230245856183\\
7165	1.50237205826144\\
7166	1.54456520444981\\
7167	1.54910980535535\\
7168	1.4859662347439\\
7169	1.45815439900055\\
7170	1.36571789763384\\
7171	1.27163710271483\\
7172	1.30464337296201\\
7173	1.38243247272\\
7174	1.40998766867369\\
7175	1.43188219852521\\
7176	1.42386199637341\\
7177	1.54595280698592\\
7178	1.63495486515114\\
7179	1.68692601952083\\
7180	1.68248507056633\\
7181	1.63720349220878\\
7182	1.52252436056007\\
7183	1.33503335701666\\
7184	1.23144808692181\\
7185	1.22203074458404\\
7186	1.23201171515131\\
7187	1.25698904540119\\
7188	1.31130751287233\\
7189	1.30245654909858\\
7190	1.30017853485907\\
7191	1.29745844869234\\
7192	1.26116644693038\\
7193	1.20082142580462\\
7194	1.16389617602985\\
7195	1.10290795857398\\
7196	1.15113167337087\\
7197	1.17151663783608\\
7198	1.23753568468017\\
7199	1.27706643936252\\
7200	1.31837113758805\\
7201	1.41722099504143\\
7202	1.50709774477095\\
7203	1.55848654413372\\
7204	1.58324766603661\\
7205	1.55133698583864\\
7206	1.46064927158724\\
7207	1.2887071765976\\
7208	1.21026211169725\\
7209	1.21958233481824\\
7210	1.24627006555265\\
7211	1.25974197753527\\
7212	1.27202018754474\\
7213	1.29839540385934\\
7214	1.29215277205447\\
7215	1.28710362158937\\
7216	1.24281032182655\\
7217	1.17941221336937\\
7218	1.12548633946012\\
7219	1.10507767827764\\
7220	1.13522651810542\\
7221	1.18055767906335\\
7222	1.24580573421381\\
7223	1.25435693710732\\
7224	1.29055265800647\\
7225	1.40662603645394\\
7226	1.48733019071202\\
7227	1.5425725669344\\
7228	1.55084653359096\\
7229	1.52407074856935\\
7230	1.43015089387623\\
7231	1.26875486293798\\
7232	1.17795082217821\\
7233	1.12424499809694\\
7234	1.1207202928696\\
7235	1.09654300070716\\
7236	1.07386928638935\\
7237	1.06940698643026\\
7238	1.06805835714938\\
7239	1.07173507328946\\
7240	1.07136410129226\\
7241	1.06135289816503\\
7242	1.05133249864721\\
7243	1.04443994433303\\
7244	1.08935355767761\\
7245	1.11666876809462\\
7246	1.15746165539138\\
7247	1.17755755131115\\
7248	1.1995628284929\\
7249	1.28864368082486\\
7250	1.36993102059566\\
7251	1.43614808287578\\
7252	1.44573869883087\\
7253	1.42944238653414\\
7254	1.35290935253116\\
7255	1.21328207196265\\
7256	1.16534336193683\\
7257	1.13370399003733\\
7258	1.11364423317598\\
7259	1.1304697297083\\
7260	1.12948931878582\\
7261	1.16417766818839\\
7262	1.16110147722216\\
7263	1.16776763450796\\
7264	1.16246816617096\\
7265	1.15380812136816\\
7266	1.10720418758389\\
7267	1.05213079173931\\
7268	1.10335237509004\\
7269	1.18851917385072\\
7270	1.2370957846523\\
7271	1.25997778472207\\
7272	1.2818493321489\\
7273	1.39597654164446\\
7274	1.4975914283859\\
7275	1.56966990739023\\
7276	1.59187850442756\\
7277	1.5581663529297\\
7278	1.46578311042\\
7279	1.29275710227124\\
7280	1.20772599445982\\
7281	1.18890555476759\\
7282	1.19446968330768\\
7283	1.23296996912909\\
7284	1.24576891155082\\
7285	1.26532526462999\\
7286	1.28520541443232\\
7287	1.29325916148401\\
7288	1.28277812068838\\
7289	1.25796692579806\\
7290	1.20268297005137\\
7291	1.17982612296326\\
7292	1.21132467828415\\
7293	1.28231549112805\\
7294	1.37169613686359\\
7295	1.37186220431969\\
7296	1.3494025094415\\
7297	1.48585874367668\\
7298	1.58036042801981\\
7299	1.65850831842235\\
7300	1.69690336294527\\
7301	1.71633406805881\\
7302	1.70064447503923\\
7303	1.63480002445063\\
7304	1.56785629722909\\
7305	1.55039692015983\\
7306	1.57515333913234\\
7307	1.59922363444031\\
7308	1.60360642238278\\
7309	1.58057979105424\\
7310	1.59191055582433\\
7311	1.60337748712402\\
7312	1.5806833249041\\
7313	1.48217664062599\\
7314	1.36572668449007\\
7315	1.34759493146839\\
7316	1.37270528856187\\
7317	1.42038373701682\\
7318	1.41700233550744\\
7319	1.38677709682858\\
7320	1.39817165255664\\
7321	1.54923863078182\\
7322	1.66786631602759\\
7323	1.74508426582008\\
7324	1.82106305470517\\
7325	1.84013817665236\\
7326	1.82551715756503\\
7327	1.74649698445822\\
7328	1.72171660619636\\
7329	1.71268443786388\\
7330	1.72776746188214\\
7331	1.7410367907147\\
7332	1.74340015031653\\
7333	1.69690267172663\\
7334	1.71152487451682\\
7335	1.6837941885451\\
7336	1.64387234030824\\
7337	1.59445038391844\\
7338	1.49588791868314\\
7339	1.43373698836182\\
7340	1.45092536545902\\
7341	1.51083125751249\\
7342	1.49278159732246\\
7343	1.50352051146666\\
7344	1.52973820363999\\
7345	1.68041366329489\\
7346	1.79173805384269\\
7347	1.85829655215958\\
7348	1.88815315236732\\
7349	1.84011079844167\\
7350	1.69604410525833\\
7351	1.45658687523597\\
7352	1.29768710490835\\
7353	1.29699347183618\\
7354	1.30975322814131\\
7355	1.27680187876811\\
7356	1.28657282556883\\
7357	1.29142590779471\\
7358	1.27195484691779\\
7359	1.26241574599898\\
7360	1.25762244653707\\
7361	1.21707647244036\\
7362	1.16504246941153\\
7363	1.16603041177295\\
7364	1.17767291644263\\
7365	1.23441727583983\\
7366	1.28080416362077\\
7367	1.27536638774925\\
7368	1.28995510917611\\
7369	1.37561398532459\\
7370	1.43894380576157\\
7371	1.48269908783761\\
7372	1.47295279557277\\
7373	1.44749046634773\\
7374	1.36307481398959\\
7375	1.20964535787835\\
7376	1.13096130502688\\
7377	1.14049206803113\\
7378	1.10567684146393\\
7379	1.07259311668044\\
7380	1.0660872691596\\
7381	1.07078107942011\\
7382	1.06679335249707\\
7383	1.06452123697932\\
7384	1.04786165135552\\
7385	1.06658797357636\\
7386	1.00341314625985\\
7387	1.0026421812786\\
7388	1.04956516541799\\
7389	1.05686508478737\\
7390	1.11745431076727\\
7391	1.13510522118355\\
7392	1.1563814181477\\
7393	1.23686588083633\\
7394	1.30125513204759\\
7395	1.34461067502714\\
7396	1.35125798766251\\
7397	1.33675320597229\\
7398	1.27375141864752\\
7399	1.13185698462149\\
7400	1.03063665229302\\
7401	1.03219434529899\\
7402	1.0446457214945\\
7403	1.0363380780069\\
7404	1.05276097770104\\
7405	1.05262931785551\\
7406	1.05266150985609\\
7407	1.05650792706577\\
7408	1.03978210320453\\
7409	1.02624769567328\\
7410	1.01482465683812\\
7411	1.0067272512266\\
7412	1.06277730703351\\
7413	1.05322154382293\\
7414	1.10756838881264\\
7415	1.12326998897477\\
7416	1.18764519825793\\
7417	1.25499416019032\\
7418	1.32845004042068\\
7419	1.37219754384602\\
7420	1.38939284642073\\
7421	1.38016812432982\\
7422	1.31600878386599\\
7423	1.18298398585468\\
7424	1.11290848555478\\
7425	1.1290864348999\\
7426	1.11459022114474\\
7427	1.10998349554398\\
7428	1.13438497806373\\
7429	1.13860936048646\\
7430	1.14293849745297\\
7431	1.15361182401636\\
7432	1.15360880670594\\
7433	1.14485284177677\\
7434	1.14055670982877\\
7435	1.11678370723856\\
7436	1.18719995861357\\
7437	1.20079706339676\\
7438	1.2819052948161\\
7439	1.27996025372244\\
7440	1.32243662074768\\
7441	1.45468773235101\\
7442	1.55840560736773\\
7443	1.63965021501366\\
7444	1.68396425975453\\
7445	1.67013483573619\\
7446	1.57162012980327\\
7447	1.37031627882753\\
7448	1.25755576054567\\
7449	1.23340041545807\\
7450	1.21467096137474\\
7451	1.21004209239358\\
7452	1.18576675337776\\
7453	1.19617104516924\\
7454	1.19959426413406\\
7455	1.21320186404023\\
7456	1.22647058757037\\
7457	1.255323785144\\
7458	1.22175035755911\\
7459	1.1921440127787\\
7460	1.21401175338454\\
7461	1.28479088212639\\
7462	1.34618798124034\\
7463	1.32080637866028\\
7464	1.30379193819666\\
7465	1.41760394696993\\
7466	1.52013214428761\\
7467	1.59008099081273\\
7468	1.66109546208216\\
7469	1.66780400699777\\
7470	1.60568720163971\\
7471	1.5370070294487\\
7472	1.4542050002841\\
7473	1.40739158847877\\
7474	1.39867274552525\\
7475	1.46541317274872\\
7476	1.46906789918025\\
7477	1.45942177014339\\
7478	1.47245748393445\\
7479	1.46161621600085\\
7480	1.44145495324817\\
7481	1.38802538434543\\
7482	1.32371165523007\\
7483	1.2755838605189\\
7484	1.34220098321629\\
7485	1.37301347804693\\
7486	1.39585272475636\\
7487	1.39056425626508\\
7488	1.40838271088115\\
7489	1.5005084824343\\
7490	1.60329105284511\\
7491	1.66678190067631\\
7492	1.69084373164267\\
7493	1.70365540100961\\
7494	1.67086103029063\\
7495	1.63075102506058\\
7496	1.59818651000119\\
7497	1.5528058345542\\
7498	1.49900901367461\\
7499	1.4637446916856\\
7500	1.40418187820297\\
7501	1.37353661600883\\
7502	1.36442309511314\\
7503	1.35969619548758\\
7504	1.34232061173291\\
7505	1.31893192543179\\
7506	1.29189719677127\\
7507	1.25442204560014\\
7508	1.2506011025794\\
7509	1.29101540488475\\
7510	1.3005921306706\\
7511	1.30918506391781\\
7512	1.33059310100441\\
7513	1.38621958503115\\
7514	1.47453053090137\\
7515	1.54105706170118\\
7516	1.56873011121501\\
7517	1.56082179129312\\
7518	1.50051113853253\\
7519	1.37882258188227\\
7520	1.32877943561405\\
7521	1.30540188487521\\
7522	1.3507825008099\\
7523	1.34659618466961\\
7524	1.36147129350223\\
7525	1.41027021147284\\
7526	1.40785688806561\\
7527	1.3941390292872\\
7528	1.35124199652387\\
7529	1.32359642464898\\
7530	1.22575821203442\\
7531	1.21351111275873\\
7532	1.26750000070935\\
7533	1.28959730824382\\
7534	1.34894342306033\\
7535	1.28823683450481\\
7536	1.32565186872595\\
7537	1.43065062478183\\
7538	1.53100563361599\\
7539	1.59071439069412\\
7540	1.63120480990362\\
7541	1.63700291426659\\
7542	1.61677165471014\\
7543	1.5530037443191\\
7544	1.59344302939817\\
7545	1.59136974072039\\
7546	1.5514322076895\\
7547	1.49002136546822\\
7548	1.45831697509721\\
7549	1.43243421318708\\
7550	1.46731502014876\\
7551	1.50088123923361\\
7552	1.47538501373738\\
7553	1.42660523137031\\
7554	1.38009578583954\\
7555	1.29449690608938\\
7556	1.30535486532593\\
7557	1.30490170283861\\
7558	1.36223472137256\\
7559	1.33164068469841\\
7560	1.35226617490796\\
7561	1.48171378342658\\
7562	1.59604116073204\\
7563	1.63775385094809\\
7564	1.63928258672474\\
7565	1.5913155071352\\
7566	1.47741897259339\\
7567	1.26117491957637\\
7568	1.12605243635897\\
7569	1.14512019356463\\
7570	1.13321601039193\\
7571	1.12124950548218\\
7572	1.11945455340099\\
7573	1.12142491767927\\
7574	1.10528985966913\\
7575	1.10517053831621\\
7576	1.10064008727927\\
7577	1.11311709758911\\
7578	1.03517626510538\\
7579	1.02437154386706\\
7580	1.06721021608131\\
7581	1.08908533692766\\
7582	1.14829209920267\\
7583	1.14173213993756\\
7584	1.18128092379569\\
7585	1.24833652856367\\
7586	1.3227713441275\\
7587	1.38070400662217\\
7588	1.39429687274772\\
7589	1.36818779085545\\
7590	1.30234107521328\\
7591	1.15737135318966\\
7592	1.09444762715177\\
7593	1.10557806307501\\
7594	1.1343311299089\\
7595	1.20137045606924\\
7596	1.19690719082193\\
7597	1.20444956406719\\
7598	1.19142229168113\\
7599	1.16203267650688\\
7600	1.16725996853544\\
7601	1.1310107567699\\
7602	1.04243267184555\\
7603	1.02938940467584\\
7604	1.08643636064711\\
7605	1.11768444806023\\
7606	1.18523569812686\\
7607	1.18595917931612\\
7608	1.21370284379914\\
7609	1.33374749886165\\
7610	1.4315139677899\\
7611	1.49902694979398\\
7612	1.51958198785278\\
7613	1.51424205351877\\
7614	1.41704125215453\\
7615	1.23828056051572\\
7616	1.12544788750909\\
7617	1.11353787798353\\
7618	1.09916448001672\\
7619	1.11577132451857\\
7620	1.08918157928777\\
7621	1.1117759264456\\
7622	1.08666427789288\\
7623	1.07670363911564\\
7624	1.07087010772358\\
7625	1.05298155961967\\
7626	1.05420635291141\\
7627	1.05284713728117\\
7628	1.0884174890969\\
7629	1.08452233896528\\
7630	1.14784226385094\\
7631	1.13036801004804\\
7632	1.18626912477169\\
7633	1.29813167214559\\
7634	1.36389847424077\\
7635	1.43367441109569\\
7636	1.47102102320681\\
7637	1.47871381953933\\
7638	1.45147674895472\\
7639	1.39666012860607\\
7640	1.32379561509785\\
7641	1.25314364158755\\
7642	1.20890191324126\\
7643	1.19307557732079\\
7644	1.17653519144751\\
7645	1.16964383052246\\
7646	1.18915077896739\\
7647	1.20979676555704\\
7648	1.22307719150447\\
7649	1.23665244971736\\
7650	1.21800858994917\\
7651	1.20045188190836\\
7652	1.17263602801937\\
7653	1.23160112334579\\
7654	1.29757750556523\\
7655	1.28071144717034\\
7656	1.28596391642507\\
7657	1.3674399057841\\
7658	1.4471660537761\\
7659	1.5129291598983\\
7660	1.54083611171858\\
7661	1.5419362368935\\
7662	1.54640362331748\\
7663	1.5040239048673\\
7664	1.45635952438758\\
7665	1.40199160597668\\
7666	1.3146484173776\\
7667	1.28530317884864\\
7668	1.25960639049589\\
7669	1.23550416383464\\
7670	1.24710479107221\\
7671	1.27204293312177\\
7672	1.26209399620845\\
7673	1.25906630489445\\
7674	1.20965949252367\\
7675	1.19069378772299\\
7676	1.18303745784681\\
7677	1.21495795283603\\
7678	1.24328286706077\\
7679	1.25387719283333\\
7680	1.27960862802315\\
7681	1.37137252953886\\
7682	1.441763021884\\
7683	1.48267704716921\\
7684	1.51211480495453\\
7685	1.49900015511564\\
7686	1.39277094084816\\
7687	1.22243709246244\\
7688	1.11029568243563\\
7689	1.0954600556972\\
7690	1.07789236049355\\
7691	1.08787309385529\\
7692	1.08634566246869\\
7693	1.110442494182\\
7694	1.09696358061795\\
7695	1.06516828066073\\
7696	1.04907254227579\\
7697	1.05954974368782\\
7698	1.00206114046926\\
7699	1.00360406684952\\
7700	1.06509626403912\\
7701	1.09425079790958\\
7702	1.15545085390075\\
7703	1.15015803959366\\
7704	1.1878444937259\\
7705	1.26375385253127\\
7706	1.33729077297691\\
7707	1.38984977145949\\
7708	1.40971698041716\\
7709	1.38919868445432\\
7710	1.30557127640296\\
7711	1.15442902866486\\
7712	1.0474645799112\\
7713	1.05615235942739\\
7714	1.06961146311399\\
7715	1.07725685385346\\
7716	1.07791132606382\\
7717	1.07764716390248\\
7718	1.0629695195443\\
7719	1.06385667051485\\
7720	1.03914281220779\\
7721	1.05869252689062\\
7722	1.0134096381461\\
7723	1.01743073301375\\
7724	1.03941560268363\\
7725	1.0632431683985\\
7726	1.12100278119352\\
7727	1.11989935053898\\
7728	1.1558531840246\\
7729	1.2471011577619\\
7730	1.31548813231623\\
7731	1.37006401423018\\
7732	1.39081287217043\\
7733	1.36868836089931\\
7734	1.30324978876611\\
7735	1.13707436974597\\
7736	1.02108955588854\\
7737	1.0387189220905\\
7738	1.02916605783834\\
7739	1.03256298470261\\
7740	1.03304347665922\\
7741	1.03870192258634\\
7742	1.02186813126664\\
7743	1.03325064125023\\
7744	1.0255193184383\\
7745	1.04857566132759\\
7746	1.02235649025934\\
7747	1.03060424069912\\
7748	1.06941358661043\\
7749	1.05969587448474\\
7750	1.12289442700254\\
7751	1.14871445429994\\
7752	1.18524655583246\\
7753	1.2343487650033\\
7754	1.3112801719613\\
7755	1.34432140809764\\
7756	1.36450659319946\\
7757	1.34465926349191\\
7758	1.27828702119909\\
7759	1.14594857663227\\
7760	1.04456556069701\\
7761	1.06313520487951\\
7762	1.07815414373704\\
7763	1.106534619516\\
7764	1.1212196616789\\
7765	1.12176918987848\\
7766	1.10884375290027\\
7767	1.10272654721681\\
7768	1.09024526735803\\
7769	1.0733347749563\\
7770	1.05604240177076\\
7771	1.05186400691022\\
7772	1.08682569733765\\
7773	1.09341669493135\\
7774	1.16044805396111\\
7775	1.17388620419271\\
7776	1.18143798982648\\
7777	1.25425629476532\\
7778	1.32501784943678\\
7779	1.36248271953444\\
7780	1.3811693840802\\
7781	1.35806542901804\\
7782	1.29893922465766\\
7783	1.15662789386553\\
7784	1.08420580676942\\
7785	1.05800822096083\\
7786	1.05854394207158\\
7787	1.05780571107523\\
7788	1.07176408035707\\
7789	1.08948508090298\\
7790	1.06699622155154\\
7791	1.08423553796223\\
7792	1.08524005301516\\
7793	1.06686457722924\\
7794	1.05552911669761\\
7795	1.05668341155985\\
7796	1.09002042413566\\
7797	1.13894232232077\\
7798	1.19861386564586\\
7799	1.18471523917354\\
7800	1.19768544044825\\
7801	1.29030398014069\\
7802	1.39021410894584\\
7803	1.4657974669166\\
7804	1.51038017421218\\
7805	1.53284982794108\\
7806	1.51635710934188\\
7807	1.47141365231208\\
7808	1.35599667361449\\
7809	1.30395699074751\\
7810	1.27258388216706\\
7811	1.33993902431494\\
7812	1.33597696875151\\
7813	1.32971636577593\\
7814	1.33343083217445\\
7815	1.34060820620004\\
7816	1.32041755754789\\
7817	1.26450818623\\
7818	1.24206757483882\\
7819	1.21095424393698\\
7820	1.20310062248499\\
7821	1.25437087307427\\
7822	1.30711440476669\\
7823	1.29670118293397\\
7824	1.29680050800515\\
7825	1.40956601501911\\
7826	1.49681929467887\\
7827	1.56608517900745\\
7828	1.60512195382431\\
7829	1.60407656373789\\
7830	1.58346029187581\\
7831	1.53672164738418\\
7832	1.50539922504569\\
7833	1.47359159162758\\
7834	1.44420652075804\\
7835	1.44063522149435\\
7836	1.40983983765842\\
7837	1.40745871013928\\
7838	1.43038374460364\\
7839	1.44040801401633\\
7840	1.42624637334458\\
7841	1.36531196889223\\
7842	1.34679147133657\\
7843	1.32630840792285\\
7844	1.28846365225749\\
7845	1.31864377157413\\
7846	1.39662952485392\\
7847	1.39480758791067\\
7848	1.40574249250964\\
7849	1.50504938184915\\
7850	1.56060218418323\\
7851	1.61003418793898\\
7852	1.61476856388513\\
7853	1.58462426942826\\
7854	1.49402332380897\\
7855	1.3099495143442\\
7856	1.19011064056304\\
7857	1.14632233792337\\
7858	1.14839240698732\\
7859	1.16861664819457\\
7860	1.15784407785445\\
7861	1.18914574474172\\
7862	1.21357025934924\\
7863	1.17750319582641\\
7864	1.17058524222801\\
7865	1.14346267740398\\
7866	1.09829877870926\\
7867	1.09303676128223\\
7868	1.11461960309142\\
7869	1.16118714131962\\
7870	1.23033644651135\\
7871	1.21367116680877\\
7872	1.24320936674117\\
7873	1.32750043110479\\
7874	1.39457232656152\\
7875	1.44582091450807\\
7876	1.46930176246939\\
7877	1.45574745931507\\
7878	1.38232155258887\\
7879	1.21895884746888\\
7880	1.10739831192362\\
7881	1.09755337263761\\
7882	1.11719706589047\\
7883	1.13464965755995\\
7884	1.14166392335622\\
7885	1.16912571678647\\
7886	1.15657824413073\\
7887	1.15013098687249\\
7888	1.1517920726982\\
7889	1.12993864826319\\
7890	1.08045829975623\\
7891	1.09658775263752\\
7892	1.10840922387191\\
7893	1.14902777838844\\
7894	1.21403746877215\\
7895	1.20385202063875\\
7896	1.22847860668493\\
7897	1.32028226929814\\
7898	1.39147293037083\\
7899	1.44499642740551\\
7900	1.46004739095762\\
7901	1.4354175257291\\
7902	1.3624413475812\\
7903	1.20642129075911\\
7904	1.1020507549484\\
7905	1.06786273830811\\
7906	1.05717644901794\\
7907	1.03809641179063\\
7908	1.02847482904692\\
7909	1.02757667233323\\
7910	1.02130045487999\\
7911	1.02015126240255\\
7912	1.01570533231853\\
7913	1.0073432300734\\
7914	1.01078808685437\\
7915	1.00517968245639\\
7916	1.0427019530266\\
7917	1.10526384259548\\
7918	1.14053482944537\\
7919	1.14317894287816\\
7920	1.16598108400041\\
7921	1.24691121497714\\
7922	1.32303470537658\\
7923	1.37929965594756\\
7924	1.36398755702996\\
7925	1.35195674500407\\
7926	1.29321904322495\\
7927	1.14290138098221\\
7928	1.04500005803602\\
7929	1.01444295546442\\
7930	1.04502060509009\\
7931	1.05440046184668\\
7932	1.02568138399726\\
7933	1.05079178914671\\
7934	1.04291396497014\\
7935	1.04977889115115\\
7936	1.04688701067462\\
7937	1.03604551578657\\
7938	1.03363182528548\\
7939	1.03964580491907\\
7940	1.04083065344723\\
7941	1.08011665538616\\
7942	1.1499983308061\\
7943	1.14576645051718\\
7944	1.17590264989348\\
7945	1.29982922962705\\
7946	1.3820575296193\\
7947	1.44528941049646\\
7948	1.48122922953787\\
7949	1.47380647500263\\
7950	1.38932478127899\\
7951	1.22825976182295\\
7952	1.16774644573713\\
7953	1.15688066730892\\
7954	1.16278554204201\\
7955	1.15955009856453\\
7956	1.15108188155975\\
7957	1.16794213076964\\
7958	1.1823108314222\\
7959	1.1891521991248\\
7960	1.18914964600505\\
7961	1.15113734856266\\
7962	1.16542283487871\\
7963	1.17852756061514\\
7964	1.18236279705618\\
7965	1.21962503818641\\
7966	1.27804156714253\\
7967	1.24915687458076\\
7968	1.25201229524308\\
7969	1.32090740410289\\
7970	1.40121201120829\\
7971	1.47561980941002\\
7972	1.52032852497847\\
7973	1.52515710571453\\
7974	1.51073501095657\\
7975	1.43689222466292\\
7976	1.35645315347259\\
7977	1.30908310632896\\
7978	1.27219072563634\\
7979	1.28891018923673\\
7980	1.30752247386622\\
7981	1.31043773421744\\
7982	1.32303043093813\\
7983	1.31518829183638\\
7984	1.2731492095661\\
7985	1.25411524724276\\
7986	1.2078148175986\\
7987	1.20643498070998\\
7988	1.26074342060625\\
7989	1.28746698168377\\
7990	1.2980901643087\\
7991	1.26454780341631\\
7992	1.24284822751565\\
7993	1.2853144847359\\
7994	1.35203760491879\\
7995	1.40877184751804\\
7996	1.44454498621928\\
7997	1.46351451227182\\
7998	1.45636412299606\\
7999	1.41636292089711\\
8000	1.38569379734371\\
8001	1.32714338990184\\
};
\addplot [color=mycolor1,line width=1.3pt,solid,forget plot]
  table[row sep=crcr]{%
8001	1.32714338990184\\
8002	1.2674075212338\\
8003	1.14362774899525\\
8004	1.13319750850523\\
8005	1.12447080811752\\
8006	1.15237038635345\\
8007	1.16015641768922\\
8008	1.1388931776476\\
8009	1.09150916218999\\
8010	1.0860738346635\\
8011	1.08138613468958\\
8012	1.06862009194893\\
8013	1.0804327780723\\
8014	1.14203677837624\\
8015	1.14193768572\\
8016	1.1680851209585\\
8017	1.24605609852322\\
8018	1.2976089673785\\
8019	1.36880748681882\\
8020	1.40818123806881\\
8021	1.39562889135193\\
8022	1.32937673970234\\
8023	1.16603230454341\\
8024	1.05427790767217\\
8025	1.00864756110756\\
8026	0.978957808151817\\
8027	0.968801154952644\\
8028	0.948475176961589\\
8029	0.948871398038586\\
8030	0.957600391326347\\
8031	0.944034951921649\\
8032	0.944337176684138\\
8033	0.945376561641493\\
8034	0.936550968501996\\
8035	0.947457915088902\\
8036	0.983822115177679\\
8037	0.982765590698779\\
8038	1.04284271141051\\
8039	1.04815171113407\\
8040	1.06503009210528\\
8041	1.10235603353149\\
8042	1.15213588622339\\
8043	1.1879960280464\\
8044	1.22055266170344\\
8045	1.33679125272692\\
8046	1.28660543185161\\
8047	1.13609739429769\\
8048	1.04523929503981\\
8049	1.03381907691649\\
8050	1.01870914209306\\
8051	1.01674731509505\\
8052	1.0240966384098\\
8053	1.01920201584531\\
8054	1.05626220069971\\
8055	1.04158817276111\\
8056	1.03006444666983\\
8057	1.0211111188244\\
8058	1.01424793275292\\
8059	1.02290240701807\\
8060	1.04327057337923\\
8061	1.08226376946989\\
8062	1.15813989231641\\
8063	1.159215648338\\
8064	1.15563858637609\\
8065	1.20893363087363\\
8066	1.26700888111112\\
8067	1.32741012006269\\
8068	1.35028200343963\\
8069	1.34938406688676\\
8070	1.28913265899389\\
8071	1.15878622002352\\
8072	1.05468381697732\\
8073	1.03795838810899\\
8074	1.03728350250654\\
8075	1.04520038241315\\
8076	1.04058198601095\\
8077	1.03928826158422\\
8078	1.02430520104429\\
8079	1.01997280945339\\
8080	0.994768047331192\\
8081	0.988564731027834\\
8082	0.990786051662645\\
8083	1.0017059501101\\
8084	1.01892099257461\\
8085	1.03216424814607\\
8086	1.0742264008316\\
8087	1.07728897467048\\
8088	1.09973077996003\\
8089	1.12288956837821\\
8090	1.17015407913711\\
8091	1.20996676400709\\
8092	1.24468255847432\\
8093	1.24007561745903\\
8094	1.19871446217658\\
8095	1.07766813704215\\
8096	0.997548267556559\\
8097	0.975779094581519\\
8098	0.975910833539629\\
8099	0.97715941708481\\
8100	0.959052621125101\\
8101	0.959024284284988\\
8102	0.958549032833736\\
8103	0.966191306550613\\
8104	0.962570188330324\\
8105	0.971775966357862\\
8106	0.943372835236049\\
8107	0.948769958911037\\
8108	0.961989949917675\\
8109	0.983741306653651\\
8110	1.04530761677083\\
8111	1.0400872316946\\
8112	1.04121526815226\\
8113	1.13955654336282\\
8114	1.16091790884612\\
8115	1.21321208664552\\
8116	1.24549972200872\\
8117	1.24962218733111\\
8118	1.20064984198152\\
8119	1.07799264454433\\
8120	1.00771601136051\\
8121	1.01790651164035\\
8122	0.998234475490073\\
8123	0.995316411288128\\
8124	0.995942184187268\\
8125	1.0067564788663\\
8126	1.01272572851109\\
8127	1.02714353228813\\
8128	1.030239263904\\
8129	1.01672414834098\\
8130	1.03114569653947\\
8131	1.05482100366943\\
8132	1.0627005401402\\
8133	1.12401527916026\\
8134	1.12315344344446\\
8135	1.11127057848047\\
8136	1.11536411846945\\
8137	1.17074092777347\\
8138	1.2381944951011\\
8139	1.31172244699663\\
8140	1.36015639969534\\
8141	1.39210556517492\\
8142	1.38400518319083\\
8143	1.31899715201476\\
8144	1.25744622202481\\
8145	1.19911255721701\\
8146	1.14612280374359\\
8147	1.13665084539821\\
8148	1.14152381787197\\
8149	1.14854800760858\\
8150	1.15060856031094\\
8151	1.16029829013551\\
8152	1.13074894134994\\
8153	1.09733511545445\\
8154	1.06285499208605\\
8155	1.05832928549834\\
8156	1.0585657608774\\
8157	1.08875361714719\\
8158	1.13217957623135\\
8159	1.12745231394739\\
8160	1.09258788344239\\
8161	1.14767427824545\\
8162	1.22348169054036\\
8163	1.28882118088677\\
8164	1.34814187825097\\
8165	1.39012747324226\\
8166	1.402245041811\\
8167	1.37583035980577\\
8168	1.37047522725175\\
8169	1.324938721044\\
8170	1.27518138566532\\
8171	1.22230717944232\\
8172	1.18893603171069\\
8173	1.17215879003556\\
8174	1.23004960575322\\
8175	1.25283774461116\\
8176	1.25834268306939\\
8177	1.2244318467072\\
8178	1.1942764558703\\
8179	1.19395429171517\\
8180	1.18401143359973\\
8181	1.17894483335176\\
8182	1.19597314372777\\
8183	1.1811631465486\\
8184	1.23686267241558\\
8185	1.34079306583217\\
8186	1.38818655366673\\
8187	1.45827111097811\\
8188	1.50924258077561\\
8189	1.50074630337809\\
8190	1.41490704168261\\
8191	1.23576563543977\\
8192	1.12832285638168\\
8193	1.06716449781852\\
8194	1.07212568982805\\
8195	1.08800158883547\\
8196	1.06363692176055\\
8197	1.04652860330476\\
8198	1.03977130059883\\
8199	1.02181097305879\\
8200	1.01649517240806\\
8201	1.00437072473902\\
8202	1.00651946246229\\
8203	0.988621238831433\\
8204	1.00898284020649\\
8205	1.0609416932194\\
8206	1.13114759457025\\
8207	1.11688143383646\\
8208	1.10710772603147\\
8209	1.19335839216978\\
8210	1.27490091079931\\
8211	1.32069316372277\\
8212	1.35812153779607\\
8213	1.33822121644125\\
8214	1.25465353509658\\
8215	1.10787211993941\\
8216	1.02853696673467\\
8217	0.992984851679174\\
8218	1.02547171963309\\
8219	1.01765998403277\\
8220	1.04302997615157\\
8221	1.03627457211846\\
8222	1.0273910585751\\
8223	1.03501838777361\\
8224	1.03107472764692\\
8225	1.02608232563071\\
8226	1.01913157797651\\
8227	1.04076470993463\\
8228	1.05591225568127\\
8229	1.13147724864141\\
8230	1.20050583128273\\
8231	1.21092004779116\\
8232	1.22727827958888\\
8233	1.3240600780942\\
8234	1.39244069774258\\
8235	1.46893212946179\\
8236	1.51602808039719\\
8237	1.52175378833141\\
8238	1.43381744711612\\
8239	1.25634251202815\\
8240	1.14411278300328\\
8241	1.07893922014659\\
8242	1.08249595157842\\
8243	1.10415301431682\\
8244	1.1113530615556\\
8245	1.14208344108835\\
8246	1.08988252774405\\
8247	1.08617933022818\\
8248	1.04891010714129\\
8249	1.01556956518918\\
8250	0.999107687409188\\
8251	1.01842916826334\\
8252	1.06919734801792\\
8253	1.08387837940913\\
8254	1.15665081923592\\
8255	1.15330301936905\\
8256	1.17803541395921\\
8257	1.26429577434067\\
8258	1.34052850892139\\
8259	1.41256064764692\\
8260	1.44716990994975\\
8261	1.43721383384624\\
8262	1.37121170446572\\
8263	1.23522503551443\\
8264	1.1133259236224\\
8265	1.08142880353185\\
8266	1.08747662709196\\
8267	1.07803823342975\\
8268	1.06843666002356\\
8269	1.07649942296848\\
8270	1.08007528732661\\
8271	1.0792310016126\\
8272	1.07066918070485\\
8273	1.06065849314233\\
8274	1.05005123990128\\
8275	1.06528723050508\\
8276	1.09501865839647\\
8277	1.11625793537462\\
8278	1.18815815615281\\
8279	1.19418736696983\\
8280	1.19066111681011\\
8281	1.28812568652554\\
8282	1.3974591587015\\
8283	1.45265425694484\\
8284	1.49036324833349\\
8285	1.48575315317391\\
8286	1.38683067106767\\
8287	1.22948229372524\\
8288	1.16061057236918\\
8289	1.15459011140171\\
8290	1.14218205517834\\
8291	1.11944930565554\\
8292	1.09814471223264\\
8293	1.10554553795501\\
8294	1.09699381947782\\
8295	1.08958299008165\\
8296	1.08903542471775\\
8297	1.05818312542576\\
8298	1.04918147400521\\
8299	1.030114273205\\
8300	1.07187759113221\\
8301	1.12385533098121\\
8302	1.16768367581928\\
8303	1.13128033532842\\
8304	1.13331632012856\\
8305	1.16607084424252\\
8306	1.23996021528096\\
8307	1.29753690410023\\
8308	1.33288953889236\\
8309	1.34051851350107\\
8310	1.32543300519182\\
8311	1.27855664308297\\
8312	1.2358580158805\\
8313	1.15269755251246\\
8314	1.12869765102761\\
8315	1.10875086506352\\
8316	1.11254384505299\\
8317	1.10585300144546\\
8318	1.11250514115532\\
8319	1.12213870815384\\
8320	1.10136397921569\\
8321	1.08226518655766\\
8322	1.07657003148479\\
8323	1.05820614973157\\
8324	1.09812647082653\\
8325	1.11773055954434\\
8326	1.16464990679266\\
8327	1.14966554450328\\
8328	1.14209261532454\\
8329	1.18590287644316\\
8330	1.24126338634169\\
8331	1.30121598283468\\
8332	1.34914177341647\\
8333	1.37740338956257\\
8334	1.38204674766572\\
8335	1.37089946391536\\
8336	1.36054424513417\\
8337	1.30026539428208\\
8338	1.25745577688457\\
8339	1.23938454639579\\
8340	1.22055288027389\\
8341	1.22601534039678\\
8342	1.2849899515783\\
8343	1.33184408450181\\
8344	1.29442679312928\\
8345	1.28112211739862\\
8346	1.21566747353943\\
8347	1.22138876627376\\
8348	1.26603493800594\\
8349	1.27025876091986\\
8350	1.3148742411285\\
8351	1.31747493451106\\
8352	1.33278817947977\\
8353	1.42557127688401\\
8354	1.51020999642886\\
8355	1.60622540199943\\
8356	1.68378323256844\\
8357	1.71338842621877\\
8358	1.67589858178971\\
8359	1.54792308216008\\
8360	1.45266439934437\\
8361	1.308101939105\\
8362	1.25410605809905\\
8363	1.19043037019656\\
8364	1.15923746168449\\
8365	1.17205581921283\\
8366	1.17146147575814\\
8367	1.18108884682878\\
8368	1.16070300319883\\
8369	1.13339031460694\\
8370	1.09649213410852\\
8371	1.10769519506942\\
8372	1.15531958448714\\
8373	1.24305770510821\\
8374	1.28411074585031\\
8375	1.26335544616121\\
8376	1.23779270026121\\
8377	1.32010525514882\\
8378	1.3996910418652\\
8379	1.46404775056346\\
8380	1.5084051681402\\
8381	1.50479530229249\\
8382	1.39752987611604\\
8383	1.18472360957616\\
8384	1.07249215929381\\
8385	1.03712495993264\\
8386	1.02894388101007\\
8387	1.04458459161933\\
8388	1.04251022364219\\
8389	1.05815952982376\\
8390	1.05453340453167\\
8391	1.04754112505741\\
8392	1.03317961007428\\
8393	1.02257155804158\\
8394	0.986993648029434\\
8395	0.984962378485203\\
8396	1.02362841987835\\
8397	1.03253805226395\\
8398	1.09600406978466\\
8399	1.0956979111307\\
8400	1.12439629125105\\
8401	1.23018345709125\\
8402	1.31264756102896\\
8403	1.40208533058254\\
8404	1.46212724082334\\
8405	1.49165601845845\\
8406	1.43656257889744\\
8407	1.27845727830626\\
8408	1.15322625014044\\
8409	1.0922551306894\\
8410	1.05534376284256\\
8411	1.04096593692628\\
8412	1.0232686110976\\
8413	1.02892795159319\\
8414	1.03048711374206\\
8415	1.02580990730217\\
8416	1.01886154516542\\
8417	0.992714311123493\\
8418	1.00162689959648\\
8419	1.00449952899897\\
8420	1.00270074324297\\
8421	1.06673401605805\\
8422	1.13652494192413\\
8423	1.177302564008\\
8424	1.17141882258827\\
8425	1.2511291782941\\
8426	1.34926432302792\\
8427	1.39853156336043\\
8428	1.43986736229519\\
8429	1.40144495491194\\
8430	1.3347464326993\\
8431	1.18287644649255\\
8432	1.11894665419411\\
8433	1.06826228576687\\
8434	1.16830444486746\\
8435	1.10876621003511\\
8436	1.09314059582075\\
8437	1.14521394076938\\
8438	1.11562659754462\\
8439	1.12927978444503\\
8440	1.13542358446419\\
8441	1.13299949651878\\
8442	1.137949720879\\
8443	1.10969009750241\\
8444	1.15358102137531\\
8445	1.2224167726239\\
8446	1.28541871278212\\
8447	1.27720183637854\\
8448	1.29077603760857\\
8449	1.39967049287388\\
8450	1.50875732661334\\
8451	1.60602079502665\\
8452	1.64724221348206\\
8453	1.64845068132474\\
8454	1.54502851175127\\
8455	1.35538279760793\\
8456	1.22728904570039\\
8457	1.1472869656789\\
8458	1.13966371185971\\
8459	1.1307546013623\\
8460	1.09404945033474\\
8461	1.07699413201875\\
8462	1.1154962028211\\
8463	1.14154793026088\\
8464	1.15660642498816\\
8465	1.15679095273787\\
8466	1.13392032562916\\
8467	1.1346093279584\\
8468	1.18951449088106\\
8469	1.24671693819689\\
8470	1.31108049510035\\
8471	1.27574426105955\\
8472	1.28452005859231\\
8473	1.45371625270884\\
8474	1.55444945752713\\
8475	1.64057248550187\\
8476	1.69454801532458\\
8477	1.72785420148592\\
8478	1.72842357489229\\
8479	1.66605015908381\\
8480	1.62697308854512\\
8481	1.50538761492827\\
8482	1.51401945432959\\
8483	1.49103287206764\\
8484	1.49173369002875\\
8485	1.49748666246559\\
8486	1.48120856615619\\
8487	1.4910369563797\\
8488	1.48429980518813\\
8489	1.44260824512219\\
8490	1.4244266050626\\
8491	1.37846846415113\\
8492	1.36404095639259\\
8493	1.40844480370983\\
8494	1.43482703637698\\
8495	1.40629640592344\\
8496	1.37247022689176\\
8497	1.45323711664685\\
8498	1.53183890857515\\
8499	1.60440030521016\\
8500	1.65373226866551\\
8501	1.66408306627363\\
8502	1.67121211674714\\
8503	1.6440584366952\\
8504	1.63037160358514\\
8505	1.56272482737773\\
8506	1.50352009382737\\
8507	1.46497699804384\\
8508	1.43831164789581\\
8509	1.42264660417006\\
8510	1.41628290800412\\
8511	1.46353933686541\\
8512	1.48018948358867\\
8513	1.46902353603382\\
8514	1.4937460865193\\
8515	1.48754359185873\\
8516	1.48282342416843\\
8517	1.47783415601633\\
8518	1.52278540880518\\
8519	1.49688531073387\\
8520	1.50751631022361\\
8521	1.64911409088117\\
8522	1.7380399270882\\
8523	1.84278219277516\\
8524	1.91102603910148\\
8525	1.88982056334449\\
8526	1.78308413693715\\
8527	1.58976825208901\\
8528	1.46459399453981\\
8529	1.33729922148427\\
8530	1.31338353420483\\
8531	1.29344127781202\\
8532	1.27974128008175\\
8533	1.33744107224697\\
8534	1.34986664334976\\
8535	1.38561570601464\\
8536	1.40619150985377\\
8537	1.40140447905447\\
8538	1.40487682906923\\
8539	1.37813571807168\\
8540	1.45959588515223\\
8541	1.5062553992216\\
8542	1.52724195454186\\
8543	1.49829696700234\\
8544	1.50436584696816\\
8545	1.64174260562922\\
8546	1.77467793906556\\
8547	1.87990479018297\\
8548	1.92268195856248\\
8549	1.90743694064729\\
8550	1.81309794258038\\
8551	1.57370382939131\\
8552	1.46080151982663\\
8553	1.35210966754684\\
8554	1.3371826231411\\
8555	1.33529123118354\\
8556	1.33106044673427\\
8557	1.30626440778397\\
8558	1.32526240118229\\
8559	1.34490157267694\\
8560	1.33478393415608\\
8561	1.34499941511857\\
8562	1.31686967045448\\
8563	1.34250411074167\\
8564	1.34247915014425\\
8565	1.43732234032891\\
8566	1.50367726988001\\
8567	1.46049385895492\\
8568	1.48084589004699\\
8569	1.62525815585464\\
8570	1.78623268844535\\
8571	1.90465573570297\\
8572	1.97750488782112\\
8573	1.98915052955196\\
8574	1.91493708276279\\
8575	1.76732578903348\\
8576	1.67432087898782\\
8577	1.54154792305925\\
8578	1.49003088867765\\
8579	1.44273791779057\\
8580	1.42963955181686\\
8581	1.426840692727\\
8582	1.36649313281802\\
8583	1.37435226912007\\
8584	1.37034179505831\\
8585	1.38464622224157\\
8586	1.4187142441262\\
8587	1.41363638995712\\
8588	1.44774574600538\\
8589	1.51733232224308\\
8590	1.54289111628408\\
8591	1.47906401168228\\
8592	1.3373111019866\\
8593	1.4181156898758\\
8594	1.51661824133831\\
8595	1.6331008785419\\
8596	1.72602632666105\\
8597	1.76685662112437\\
8598	1.76687434399964\\
8599	1.73689244027631\\
8600	1.76018739273318\\
8601	1.69901416154988\\
8602	1.65404534540226\\
8603	1.59499036599155\\
8604	1.55495011923529\\
8605	1.52346443941059\\
8606	1.52208670164402\\
8607	1.52343872800384\\
8608	1.51176478191814\\
8609	1.49477712082175\\
8610	1.47038786551795\\
8611	1.44654468840921\\
8612	1.40319965167553\\
8613	1.39319031460402\\
8614	1.36545934586638\\
8615	1.28848127907015\\
8616	1.27387531757695\\
8617	1.37441531619016\\
8618	1.44671595986635\\
8619	1.51160569489657\\
8620	1.54184993262004\\
8621	1.52687481016744\\
8622	1.46762806637376\\
8623	1.37037682837617\\
8624	1.32753517981743\\
8625	1.2539304156495\\
8626	1.20860842275527\\
8627	1.18510279314986\\
8628	1.18860509464683\\
8629	1.19615268691176\\
8630	1.19993024969369\\
8631	1.18405164532763\\
8632	1.18209855574803\\
8633	1.17345005485258\\
8634	1.19418586716758\\
8635	1.20054728178339\\
8636	1.25373724715499\\
8637	1.29167717473871\\
8638	1.37683124961439\\
8639	1.33296865343761\\
8640	1.33756531417263\\
8641	1.42181667740224\\
8642	1.52798157421027\\
8643	1.63215013860872\\
8644	1.70711197062328\\
8645	1.71713852234749\\
8646	1.69099616008878\\
8647	1.62720733793361\\
8648	1.56788242793241\\
8649	1.46159645706353\\
8650	1.3887735711199\\
8651	1.32469768234208\\
8652	1.2813917876019\\
8653	1.27992309397903\\
8654	1.27157889541129\\
8655	1.28813660915137\\
8656	1.29025021914606\\
8657	1.29645075301038\\
8658	1.28356367416061\\
8659	1.28464153738118\\
8660	1.28931907440499\\
8661	1.31981032823594\\
8662	1.3663827499791\\
8663	1.34595913322719\\
8664	1.32182417817365\\
8665	1.41534059051828\\
8666	1.50373159945767\\
8667	1.57557945415222\\
8668	1.64875183624293\\
8669	1.68894818416681\\
8670	1.68789467932532\\
8671	1.63702247323077\\
8672	1.62543211153994\\
8673	1.53699596934418\\
8674	1.45226720361628\\
8675	1.42473165753451\\
8676	1.40272487441901\\
8677	1.38947610209476\\
8678	1.39421554951891\\
8679	1.37161245835094\\
8680	1.30944101650075\\
8681	1.247635651852\\
8682	1.18096573031712\\
8683	1.16639149896853\\
8684	1.17607403372336\\
8685	1.16380512695592\\
8686	1.18021427257023\\
8687	1.18503415715792\\
8688	1.14404970588347\\
8689	1.21431908874673\\
8690	1.29065111632322\\
8691	1.37458449659015\\
8692	1.43538413224968\\
8693	1.46415401467346\\
8694	1.4461163806062\\
8695	1.35749888266681\\
8696	1.30812927646506\\
8697	1.21321622168228\\
8698	1.21657193783287\\
8699	1.21091471709704\\
8700	1.20401693677836\\
8701	1.21976546691653\\
8702	1.21608718284478\\
8703	1.22338064446681\\
8704	1.20445092203241\\
8705	1.18508735322561\\
8706	1.12926075635526\\
8707	1.10922701088948\\
8708	1.12026200956422\\
8709	1.14839566014159\\
8710	1.1845788772179\\
8711	1.14680187464326\\
8712	1.15168090420463\\
8713	1.23439092099838\\
8714	1.29764072001228\\
8715	1.37056622249755\\
8716	1.41740022872829\\
8717	1.42291367809524\\
8718	1.38408785617229\\
8719	1.29978002028562\\
8720	1.2508312619189\\
8721	1.18371747958621\\
8722	1.15673300303895\\
8723	1.15366871181385\\
8724	1.15643342164923\\
8725	1.14618583267885\\
8726	1.14457399673234\\
8727	1.15368258138542\\
8728	1.14743425371603\\
8729	1.12586706954808\\
8730	1.09176196233481\\
8731	1.10101955832463\\
8732	1.13645231798567\\
8733	1.15185910820279\\
8734	1.20781463026626\\
8735	1.16425463265425\\
8736	1.16208789058946\\
8737	1.2516302016378\\
8738	1.32688103504547\\
8739	1.39402901621521\\
8740	1.42823601179647\\
8741	1.44294004489756\\
8742	1.41427969104101\\
8743	1.33319658219332\\
8744	1.28331933389827\\
8745	1.20406101031944\\
8746	1.17747007448541\\
8747	1.20503401423872\\
8748	1.20825319533468\\
8749	1.22950298882453\\
8750	1.20746979830123\\
8751	1.22318878187661\\
8752	1.21691420791927\\
8753	1.19401951496614\\
8754	1.19550291861363\\
8755	1.21334587223175\\
8756	1.28311931989187\\
8757	1.2904425736041\\
8758	1.33421323961541\\
8759	1.28308881880662\\
8760	1.24069928656258\\
};
\end{axis}
\end{tikzpicture}%
    \caption{Ratio between the maximal Belgian production and the real demand level (the demand level minus the power coming from renewable energies)}
    \label{fig:ratio2}
\end{figure}

\paragraph{EDR - ImpExp} if we compare the two figures [\ref{fig:ratio2}] and [\ref{fig:EDR_R}], we observe that when the belgian ratio between production and demand level approaches the unit, the reserve price increases. The system is aware of the risk for the system reliability and is willing to pay more in exchange for reserves. For the ImpExp model, the behavior is similar, except that the demand level isn't reduced by the importation. Therefore, the pressure on the system is even stronger. However, it can rely on an external generator, if it's ready to pay the price.

\paragraph{ORDC} The reserve price is generally higher in the third model than in the two previous models. The third model attaches great value to reserves, even when the amount has exceeded the required level of the previous models. Meaning that a lot of reserves will be made in order to guarantee system reliability. Furthermore, the two peaks highlight the fact that the system must be very strained for the system operator to realize that : it's enjoyable to have reserves. But it's even better if the energy price isn't too high. Let's remember that operating reserve demand curves have been proposed as an approach for achieving high energy prices in conditions of scarcity through prices spikes in that are more frequent but less elevated. However, it clearly doesn't work in our case. Therefore, it may be interesting to review the parameters of the operating reserve demand curves. 

\begin{table}[H]
\centering
\begin{tabular}{l | c  c  c  c}
model & mean & variance & min & max \\
\hline
EDR & $0.26$ &  $1.91$ & $0$ &  $34.33$ \\
ImpExp & $0.52$ &  $1.14$ & $0$ &  $13.09$ \\
ORDC & $ 6.14$ & $154.2$ & $0$ & $718.04$ \\
\end{tabular}
\caption{Statistics on the predicted reserve prices}
\end{table}

\begin{figure}[H]
    \centering
    \setlength\fheight{0.3\textwidth}
    \setlength\fwidth{0.85\textwidth}
    \input{images/EDR_reserve.tikz}
    \caption{Predicted reserve prices for the EDR model}
    \label{fig:EDR_R}
\end{figure}

\begin{figure}[H]
    \centering
    \setlength\fheight{0.3\textwidth}
    \setlength\fwidth{0.85\textwidth}
    % This file was created by matlab2tikz.
% Minimal pgfplots version: 1.3
%
%The latest updates can be retrieved from
%  http://www.mathworks.com/matlabcentral/fileexchange/22022-matlab2tikz
%where you can also make suggestions and rate matlab2tikz.
%
\definecolor{mycolor1}{rgb}{0.04314,0.51765,0.78039}%
%
\begin{tikzpicture}

\begin{axis}[%
width=\fwidth,
height=\fheight,
at={(0\fwidth,0\fheight)},
scale only axis,
separate axis lines,
every outer x axis line/.append style={black},
every x tick label/.append style={font=\color{black}},
xmin=0,
xmax=8760,
xlabel={time [hour]},
xtick={0,1000,2000,3000,4000,5000,6000,7000,8000},
xmajorgrids,
every outer y axis line/.append style={black},
every y tick label/.append style={font=\color{black}},
ymin=0,
ymax=14,
ymajorgrids,
title style={font=\bfseries},
title={ImpExp - Reserve price [\euro/MWh]}
]
\addplot [color=mycolor1,solid,line width=1.0pt,forget plot]
  table[row sep=crcr]{%
1	0\\
2	0\\
3	0\\
4	0\\
5	0\\
6	0\\
7	0\\
8	0\\
9	0\\
10	0\\
11	0\\
12	0\\
13	0\\
14	0\\
15	0\\
16	0\\
17	0\\
18	0\\
19	0\\
20	0\\
21	0\\
22	0\\
23	0\\
24	0\\
25	0\\
26	0\\
27	0\\
28	0\\
29	0\\
30	0\\
31	0\\
32	0\\
33	0\\
34	0\\
35	0\\
36	0\\
37	0\\
38	0\\
39	0\\
40	0\\
41	0\\
42	0\\
43	0\\
44	0\\
45	0\\
46	0\\
47	0\\
48	0\\
49	0\\
50	0\\
51	0\\
52	0\\
53	0\\
54	0\\
55	0\\
56	0\\
57	0\\
58	0\\
59	0\\
60	0\\
61	0\\
62	0\\
63	0\\
64	0\\
65	0\\
66	0\\
67	0\\
68	0\\
69	0\\
70	0\\
71	0\\
72	0\\
73	0\\
74	0\\
75	0\\
76	0\\
77	0\\
78	0\\
79	0\\
80	0\\
81	0\\
82	0\\
83	0\\
84	0\\
85	0\\
86	0\\
87	0\\
88	0\\
89	0\\
90	0\\
91	0\\
92	0\\
93	0\\
94	0\\
95	0\\
96	0\\
97	0\\
98	0\\
99	0\\
100	0\\
101	0\\
102	0\\
103	0\\
104	0\\
105	0\\
106	0\\
107	0\\
108	0\\
109	0\\
110	0\\
111	0\\
112	0\\
113	0\\
114	0\\
115	0\\
116	0\\
117	0\\
118	0\\
119	0\\
120	0\\
121	0\\
122	0\\
123	0\\
124	0\\
125	0\\
126	0\\
127	0\\
128	0\\
129	0\\
130	0\\
131	0\\
132	0\\
133	0\\
134	0\\
135	0\\
136	0\\
137	0\\
138	0\\
139	0\\
140	0\\
141	0\\
142	0\\
143	0\\
144	0\\
145	0\\
146	0\\
147	0\\
148	0\\
149	0\\
150	0\\
151	0\\
152	0\\
153	0\\
154	0\\
155	0\\
156	0\\
157	0\\
158	0\\
159	0\\
160	0\\
161	0\\
162	0\\
163	0\\
164	0\\
165	0\\
166	0\\
167	0\\
168	0\\
169	0\\
170	0\\
171	0\\
172	0\\
173	0\\
174	0\\
175	0\\
176	0\\
177	0\\
178	0\\
179	0\\
180	0\\
181	0\\
182	0\\
183	0\\
184	0\\
185	0\\
186	0\\
187	0\\
188	0\\
189	0\\
190	0\\
191	0\\
192	0\\
193	0\\
194	0\\
195	0\\
196	0\\
197	0\\
198	0\\
199	0\\
200	0\\
201	0\\
202	0\\
203	0\\
204	0\\
205	0\\
206	0\\
207	0\\
208	0\\
209	0\\
210	0\\
211	0\\
212	0\\
213	0\\
214	0\\
215	0\\
216	0\\
217	0\\
218	0\\
219	0\\
220	0\\
221	0\\
222	0\\
223	0\\
224	0\\
225	0\\
226	0\\
227	0\\
228	0\\
229	0\\
230	0\\
231	0\\
232	0\\
233	0\\
234	0\\
235	0\\
236	0\\
237	0\\
238	0\\
239	0\\
240	0\\
241	0\\
242	0\\
243	0\\
244	0\\
245	0\\
246	0\\
247	0\\
248	0\\
249	0\\
250	0\\
251	0\\
252	0\\
253	0\\
254	0\\
255	0\\
256	0\\
257	0\\
258	0\\
259	0\\
260	0\\
261	0\\
262	0\\
263	0\\
264	0\\
265	0\\
266	0\\
267	0\\
268	0\\
269	0\\
270	0\\
271	0\\
272	0\\
273	0\\
274	0\\
275	0\\
276	0\\
277	0\\
278	0\\
279	0\\
280	0\\
281	0\\
282	0\\
283	0\\
284	0\\
285	0\\
286	0\\
287	0\\
288	0\\
289	0\\
290	0\\
291	0\\
292	0\\
293	0\\
294	0\\
295	0\\
296	0\\
297	0\\
298	0\\
299	0\\
300	0\\
301	0\\
302	0\\
303	0\\
304	0\\
305	0\\
306	0\\
307	0\\
308	0\\
309	0\\
310	0\\
311	0\\
312	0\\
313	0\\
314	0\\
315	0\\
316	0\\
317	0\\
318	0\\
319	0\\
320	0\\
321	0\\
322	0\\
323	0\\
324	0\\
325	0\\
326	0\\
327	0\\
328	0\\
329	0\\
330	0\\
331	0\\
332	0\\
333	0\\
334	0\\
335	0\\
336	0\\
337	0\\
338	0\\
339	0\\
340	0\\
341	0\\
342	0\\
343	0\\
344	0\\
345	0\\
346	0\\
347	0\\
348	0\\
349	0\\
350	0\\
351	0\\
352	0\\
353	0\\
354	0\\
355	0\\
356	0\\
357	0\\
358	0\\
359	0\\
360	0\\
361	0\\
362	0\\
363	0\\
364	0\\
365	0\\
366	0\\
367	0\\
368	0\\
369	0\\
370	0\\
371	0\\
372	0\\
373	0\\
374	0\\
375	0\\
376	0\\
377	0\\
378	0\\
379	0\\
380	0\\
381	0\\
382	0\\
383	0\\
384	0\\
385	0\\
386	0\\
387	0\\
388	0\\
389	0\\
390	0\\
391	0\\
392	0\\
393	0\\
394	0\\
395	0\\
396	0\\
397	0\\
398	0\\
399	0\\
400	0\\
401	0\\
402	0\\
403	0\\
404	0\\
405	0\\
406	0\\
407	0\\
408	0\\
409	0\\
410	0\\
411	0\\
412	0\\
413	0\\
414	0\\
415	0\\
416	0\\
417	0\\
418	0\\
419	0\\
420	0\\
421	0\\
422	0\\
423	0\\
424	0\\
425	0\\
426	0\\
427	0\\
428	0\\
429	0\\
430	0\\
431	0\\
432	0\\
433	0\\
434	0\\
435	0\\
436	0\\
437	0\\
438	0\\
439	0\\
440	0\\
441	0\\
442	0\\
443	0\\
444	0\\
445	0\\
446	0\\
447	0\\
448	0\\
449	0\\
450	0\\
451	0\\
452	0\\
453	0\\
454	0\\
455	0\\
456	0\\
457	0\\
458	0\\
459	0\\
460	0\\
461	0\\
462	0\\
463	0\\
464	0\\
465	0\\
466	0\\
467	0\\
468	0\\
469	0\\
470	0\\
471	0\\
472	0\\
473	0\\
474	0\\
475	0\\
476	0\\
477	0\\
478	0\\
479	0\\
480	0\\
481	0\\
482	0\\
483	0\\
484	0\\
485	0\\
486	0\\
487	0\\
488	0\\
489	0\\
490	0\\
491	0\\
492	0\\
493	0\\
494	0\\
495	0\\
496	0\\
497	0\\
498	0\\
499	0\\
500	0\\
501	0\\
502	0\\
503	0\\
504	0\\
505	0\\
506	0\\
507	0\\
508	0\\
509	0\\
510	0\\
511	0\\
512	0\\
513	0\\
514	0\\
515	0\\
516	0\\
517	0\\
518	0\\
519	0\\
520	0\\
521	0\\
522	0\\
523	0\\
524	0\\
525	0\\
526	0\\
527	0\\
528	0\\
529	0\\
530	0\\
531	0\\
532	0\\
533	0\\
534	0\\
535	0\\
536	0\\
537	0\\
538	0\\
539	0\\
540	0\\
541	0\\
542	0\\
543	0\\
544	0\\
545	0\\
546	0\\
547	0\\
548	0\\
549	0\\
550	0\\
551	0\\
552	0\\
553	0\\
554	0\\
555	0\\
556	0\\
557	0\\
558	0\\
559	0\\
560	0\\
561	0\\
562	0\\
563	0\\
564	0\\
565	0\\
566	0\\
567	0\\
568	0\\
569	0\\
570	0\\
571	0\\
572	0\\
573	0\\
574	0\\
575	0\\
576	0\\
577	0\\
578	0\\
579	0\\
580	0\\
581	0\\
582	0\\
583	0\\
584	0\\
585	0\\
586	0\\
587	0\\
588	0\\
589	0\\
590	0\\
591	0\\
592	0\\
593	0\\
594	0\\
595	0\\
596	0\\
597	0\\
598	0\\
599	0\\
600	0\\
601	0\\
602	0\\
603	0\\
604	0\\
605	0\\
606	0\\
607	0\\
608	0\\
609	0\\
610	0\\
611	0\\
612	0\\
613	0\\
614	0\\
615	0\\
616	0\\
617	0\\
618	0\\
619	0\\
620	0\\
621	0\\
622	0\\
623	0\\
624	0\\
625	0\\
626	0\\
627	0\\
628	0\\
629	0\\
630	0\\
631	0\\
632	0\\
633	0\\
634	0\\
635	0\\
636	0\\
637	0\\
638	0\\
639	0\\
640	0\\
641	0\\
642	0\\
643	0\\
644	0\\
645	0\\
646	0\\
647	0\\
648	0\\
649	0\\
650	0\\
651	0\\
652	0\\
653	0\\
654	0\\
655	0\\
656	0\\
657	0\\
658	0\\
659	0\\
660	0\\
661	0\\
662	0\\
663	0\\
664	0\\
665	0\\
666	0\\
667	0\\
668	0\\
669	0\\
670	0\\
671	0\\
672	0\\
673	0\\
674	0\\
675	0\\
676	0\\
677	0\\
678	0\\
679	0\\
680	0\\
681	0\\
682	0\\
683	0\\
684	0\\
685	0\\
686	0\\
687	0\\
688	0\\
689	0\\
690	0\\
691	0\\
692	0\\
693	0\\
694	0\\
695	0\\
696	0\\
697	0\\
698	0\\
699	0\\
700	0\\
701	0\\
702	0\\
703	0\\
704	0\\
705	0\\
706	0\\
707	0\\
708	0\\
709	0\\
710	0\\
711	0\\
712	0\\
713	0\\
714	0\\
715	0\\
716	0\\
717	0\\
718	0\\
719	0\\
720	0\\
721	0\\
722	0\\
723	0\\
724	0\\
725	0\\
726	0\\
727	0\\
728	0\\
729	0\\
730	0\\
731	0\\
732	0\\
733	0\\
734	0\\
735	0\\
736	0\\
737	0\\
738	0\\
739	0\\
740	0\\
741	0\\
742	0\\
743	0\\
744	0\\
745	0\\
746	0\\
747	0\\
748	0\\
749	0\\
750	0\\
751	0\\
752	0\\
753	0\\
754	0\\
755	0\\
756	0\\
757	0\\
758	0\\
759	0\\
760	0\\
761	0\\
762	0\\
763	0\\
764	0\\
765	0\\
766	0\\
767	0\\
768	0\\
769	0\\
770	0\\
771	0\\
772	0\\
773	0\\
774	0\\
775	0\\
776	0\\
777	0\\
778	0\\
779	0\\
780	0\\
781	0\\
782	0\\
783	0\\
784	0\\
785	0\\
786	0\\
787	0\\
788	0\\
789	0\\
790	0\\
791	0\\
792	0\\
793	0\\
794	0\\
795	0\\
796	0\\
797	0\\
798	0\\
799	0\\
800	0\\
801	0\\
802	0\\
803	0\\
804	0\\
805	0\\
806	0\\
807	0\\
808	0\\
809	0\\
810	0\\
811	0\\
812	0\\
813	0\\
814	0\\
815	0\\
816	0\\
817	0\\
818	0\\
819	0\\
820	0\\
821	0\\
822	0\\
823	0\\
824	0\\
825	0\\
826	0\\
827	0\\
828	0\\
829	0\\
830	0\\
831	0\\
832	0\\
833	0\\
834	0\\
835	0\\
836	0\\
837	0\\
838	0\\
839	0\\
840	0\\
841	0\\
842	0\\
843	0\\
844	0\\
845	0\\
846	0\\
847	0\\
848	0\\
849	0\\
850	0\\
851	0\\
852	0\\
853	0\\
854	0\\
855	0\\
856	0\\
857	0\\
858	0\\
859	0\\
860	0\\
861	0\\
862	0\\
863	0\\
864	0\\
865	0\\
866	0\\
867	0\\
868	0\\
869	0\\
870	0\\
871	0\\
872	0\\
873	0\\
874	0\\
875	0\\
876	0\\
877	0\\
878	0\\
879	0\\
880	0\\
881	0\\
882	0\\
883	0\\
884	0\\
885	0\\
886	0\\
887	0\\
888	0\\
889	0\\
890	0\\
891	0\\
892	0\\
893	0\\
894	0\\
895	0\\
896	0\\
897	0\\
898	0\\
899	0\\
900	0\\
901	0\\
902	0\\
903	0\\
904	0\\
905	0\\
906	0\\
907	0\\
908	0\\
909	0\\
910	0\\
911	0\\
912	0\\
913	0\\
914	0\\
915	0\\
916	0\\
917	0\\
918	0\\
919	0\\
920	0\\
921	0\\
922	0\\
923	0\\
924	0\\
925	0\\
926	0\\
927	0\\
928	0\\
929	0\\
930	0\\
931	0\\
932	0\\
933	0\\
934	0\\
935	0\\
936	0\\
937	0\\
938	0\\
939	0\\
940	0\\
941	0\\
942	0\\
943	0\\
944	0\\
945	0\\
946	0\\
947	0\\
948	0\\
949	0\\
950	0\\
951	0\\
952	0\\
953	0\\
954	0\\
955	0\\
956	0\\
957	0\\
958	0\\
959	0\\
960	0\\
961	0\\
962	0\\
963	0\\
964	0\\
965	0\\
966	0\\
967	0\\
968	0\\
969	0\\
970	0\\
971	0\\
972	0\\
973	0\\
974	0\\
975	0\\
976	0\\
977	0\\
978	0\\
979	0\\
980	0\\
981	0\\
982	0\\
983	0\\
984	0\\
985	0\\
986	0\\
987	0\\
988	0\\
989	0\\
990	0\\
991	0\\
992	0\\
993	0\\
994	0\\
995	0\\
996	0\\
997	0\\
998	0\\
999	0\\
1000	0\\
1001	0\\
1002	0\\
1003	0\\
1004	0\\
1005	0\\
1006	0\\
1007	0\\
1008	0\\
1009	0\\
1010	0\\
1011	0\\
1012	0\\
1013	0\\
1014	0\\
1015	0\\
1016	0\\
1017	0\\
1018	0\\
1019	0\\
1020	0\\
1021	0\\
1022	0\\
1023	0\\
1024	0\\
1025	0\\
1026	0\\
1027	0\\
1028	0\\
1029	0\\
1030	0\\
1031	0\\
1032	0\\
1033	0\\
1034	0\\
1035	0\\
1036	0\\
1037	0\\
1038	0\\
1039	0\\
1040	0\\
1041	0\\
1042	0\\
1043	0\\
1044	0\\
1045	0\\
1046	0\\
1047	0\\
1048	0\\
1049	0\\
1050	0\\
1051	0\\
1052	0\\
1053	0\\
1054	0\\
1055	0\\
1056	0\\
1057	0\\
1058	0\\
1059	0\\
1060	0\\
1061	0\\
1062	0\\
1063	0\\
1064	0\\
1065	0\\
1066	0\\
1067	0\\
1068	0\\
1069	0\\
1070	0\\
1071	0\\
1072	0\\
1073	0\\
1074	0\\
1075	0\\
1076	0\\
1077	0\\
1078	0\\
1079	0\\
1080	0\\
1081	0\\
1082	0\\
1083	0\\
1084	0\\
1085	0\\
1086	0\\
1087	0\\
1088	0\\
1089	0\\
1090	0\\
1091	0\\
1092	0\\
1093	0\\
1094	0\\
1095	0\\
1096	0\\
1097	0\\
1098	0\\
1099	0\\
1100	0\\
1101	0\\
1102	0\\
1103	0\\
1104	0\\
1105	0\\
1106	0\\
1107	0\\
1108	0\\
1109	0\\
1110	0\\
1111	0\\
1112	0\\
1113	0\\
1114	0\\
1115	0\\
1116	0\\
1117	0\\
1118	0\\
1119	0\\
1120	0\\
1121	0\\
1122	0\\
1123	0\\
1124	0\\
1125	0\\
1126	0\\
1127	0\\
1128	0\\
1129	0\\
1130	0\\
1131	0\\
1132	0\\
1133	0\\
1134	0\\
1135	0\\
1136	0\\
1137	0\\
1138	0\\
1139	0\\
1140	0\\
1141	0\\
1142	0\\
1143	0\\
1144	0\\
1145	0\\
1146	0\\
1147	0\\
1148	0\\
1149	0\\
1150	0\\
1151	0\\
1152	0\\
1153	0\\
1154	0\\
1155	0\\
1156	0\\
1157	0\\
1158	0\\
1159	0\\
1160	0\\
1161	0\\
1162	0\\
1163	0\\
1164	0\\
1165	0\\
1166	0\\
1167	0\\
1168	0\\
1169	0\\
1170	0\\
1171	0\\
1172	0\\
1173	0\\
1174	0\\
1175	0\\
1176	0\\
1177	0\\
1178	0\\
1179	0\\
1180	0\\
1181	0\\
1182	0\\
1183	0\\
1184	0\\
1185	0\\
1186	0\\
1187	0\\
1188	0\\
1189	0\\
1190	0\\
1191	0\\
1192	0\\
1193	0\\
1194	0\\
1195	0\\
1196	0\\
1197	0\\
1198	0\\
1199	0\\
1200	0\\
1201	0\\
1202	0\\
1203	0\\
1204	0\\
1205	0\\
1206	0\\
1207	0\\
1208	0\\
1209	0\\
1210	0\\
1211	0\\
1212	0\\
1213	0\\
1214	0\\
1215	0\\
1216	0\\
1217	0\\
1218	0\\
1219	0\\
1220	0\\
1221	0\\
1222	0\\
1223	0\\
1224	0\\
1225	0\\
1226	0\\
1227	0\\
1228	0\\
1229	0\\
1230	0\\
1231	0\\
1232	0\\
1233	0\\
1234	0\\
1235	0\\
1236	0\\
1237	0\\
1238	0\\
1239	0\\
1240	0\\
1241	0\\
1242	0\\
1243	0\\
1244	0\\
1245	0\\
1246	0\\
1247	0\\
1248	0\\
1249	0\\
1250	0\\
1251	0\\
1252	0\\
1253	0\\
1254	0\\
1255	0\\
1256	0\\
1257	0\\
1258	0\\
1259	0\\
1260	0\\
1261	0\\
1262	0\\
1263	0\\
1264	0\\
1265	0\\
1266	0\\
1267	0\\
1268	0\\
1269	0\\
1270	0\\
1271	0\\
1272	0\\
1273	0\\
1274	0\\
1275	0\\
1276	0\\
1277	0\\
1278	0\\
1279	0\\
1280	0\\
1281	0\\
1282	0\\
1283	0\\
1284	0\\
1285	0\\
1286	0\\
1287	0\\
1288	0\\
1289	0\\
1290	0\\
1291	0\\
1292	0\\
1293	0\\
1294	0\\
1295	0\\
1296	0\\
1297	0\\
1298	0\\
1299	0\\
1300	0\\
1301	0\\
1302	0\\
1303	0\\
1304	0\\
1305	0\\
1306	0\\
1307	0\\
1308	0\\
1309	0\\
1310	0\\
1311	0\\
1312	0\\
1313	0\\
1314	0\\
1315	0\\
1316	0\\
1317	0\\
1318	0\\
1319	0\\
1320	0\\
1321	0\\
1322	0\\
1323	0\\
1324	0\\
1325	0\\
1326	0\\
1327	0\\
1328	0\\
1329	0\\
1330	0\\
1331	0\\
1332	0\\
1333	0\\
1334	0\\
1335	0\\
1336	0\\
1337	0\\
1338	0\\
1339	0\\
1340	0\\
1341	0\\
1342	0\\
1343	0\\
1344	0\\
1345	0\\
1346	0\\
1347	0\\
1348	0\\
1349	0\\
1350	0\\
1351	0\\
1352	0\\
1353	0\\
1354	0\\
1355	0\\
1356	0\\
1357	0\\
1358	0\\
1359	0\\
1360	0\\
1361	0\\
1362	0\\
1363	0\\
1364	0\\
1365	0\\
1366	0\\
1367	0\\
1368	0\\
1369	0\\
1370	0\\
1371	0\\
1372	0\\
1373	0\\
1374	0\\
1375	0\\
1376	0\\
1377	0\\
1378	0\\
1379	0\\
1380	0\\
1381	0\\
1382	0\\
1383	0\\
1384	0\\
1385	0\\
1386	0\\
1387	0\\
1388	0\\
1389	0\\
1390	0\\
1391	0\\
1392	0\\
1393	0\\
1394	0\\
1395	0\\
1396	0\\
1397	0\\
1398	0\\
1399	0\\
1400	0\\
1401	0\\
1402	0\\
1403	0\\
1404	0\\
1405	0\\
1406	0\\
1407	0\\
1408	0\\
1409	0\\
1410	0\\
1411	0\\
1412	0\\
1413	0\\
1414	0\\
1415	0\\
1416	0\\
1417	0\\
1418	0\\
1419	0\\
1420	0\\
1421	0\\
1422	0\\
1423	0\\
1424	0\\
1425	0\\
1426	0\\
1427	0\\
1428	0\\
1429	0\\
1430	0\\
1431	0\\
1432	0\\
1433	0\\
1434	0\\
1435	0\\
1436	0\\
1437	0\\
1438	0\\
1439	0\\
1440	0\\
1441	0\\
1442	0\\
1443	0\\
1444	0\\
1445	0\\
1446	0\\
1447	0\\
1448	0\\
1449	0\\
1450	0\\
1451	0\\
1452	0\\
1453	0\\
1454	0\\
1455	0\\
1456	0\\
1457	0\\
1458	0\\
1459	0\\
1460	0\\
1461	0\\
1462	0\\
1463	0\\
1464	0\\
1465	0\\
1466	0\\
1467	0\\
1468	0\\
1469	0\\
1470	0\\
1471	0\\
1472	0\\
1473	0\\
1474	0\\
1475	0\\
1476	0\\
1477	0\\
1478	0\\
1479	0\\
1480	0\\
1481	0\\
1482	0\\
1483	0\\
1484	0\\
1485	0\\
1486	0\\
1487	0\\
1488	0\\
1489	0\\
1490	0\\
1491	0\\
1492	0\\
1493	0\\
1494	0\\
1495	0\\
1496	0\\
1497	0\\
1498	0\\
1499	0\\
1500	0\\
1501	0\\
1502	0\\
1503	0\\
1504	0\\
1505	0\\
1506	0\\
1507	0\\
1508	0\\
1509	0\\
1510	0\\
1511	0\\
1512	0\\
1513	0\\
1514	0\\
1515	0\\
1516	0\\
1517	0\\
1518	0\\
1519	0\\
1520	0\\
1521	0\\
1522	0\\
1523	0\\
1524	0\\
1525	0\\
1526	0\\
1527	0\\
1528	0\\
1529	0\\
1530	0\\
1531	0\\
1532	0\\
1533	0\\
1534	0\\
1535	0\\
1536	0\\
1537	0\\
1538	0\\
1539	0\\
1540	0\\
1541	0\\
1542	0\\
1543	0\\
1544	0\\
1545	0\\
1546	0\\
1547	0\\
1548	0\\
1549	0\\
1550	0\\
1551	0\\
1552	0\\
1553	0\\
1554	0\\
1555	0\\
1556	0\\
1557	0\\
1558	0\\
1559	0\\
1560	0\\
1561	0\\
1562	0\\
1563	0\\
1564	0\\
1565	0\\
1566	0\\
1567	0\\
1568	0\\
1569	0\\
1570	0\\
1571	0\\
1572	0\\
1573	0\\
1574	0\\
1575	0\\
1576	0\\
1577	0\\
1578	0\\
1579	0\\
1580	0\\
1581	0\\
1582	0\\
1583	0\\
1584	0\\
1585	0\\
1586	0\\
1587	0\\
1588	0\\
1589	0\\
1590	0\\
1591	0\\
1592	0\\
1593	0\\
1594	0\\
1595	0\\
1596	0\\
1597	0\\
1598	0\\
1599	0\\
1600	0\\
1601	0\\
1602	0\\
1603	0\\
1604	0\\
1605	0\\
1606	0\\
1607	0\\
1608	0\\
1609	0\\
1610	0\\
1611	0\\
1612	0\\
1613	0\\
1614	0\\
1615	0\\
1616	0\\
1617	0\\
1618	0\\
1619	0\\
1620	0\\
1621	0\\
1622	0\\
1623	0\\
1624	0\\
1625	0\\
1626	0\\
1627	0\\
1628	0\\
1629	0\\
1630	0\\
1631	0\\
1632	0\\
1633	0\\
1634	0\\
1635	0\\
1636	0\\
1637	0\\
1638	0\\
1639	0\\
1640	0\\
1641	0\\
1642	0\\
1643	0\\
1644	0\\
1645	0\\
1646	0\\
1647	0\\
1648	0\\
1649	0\\
1650	0\\
1651	0\\
1652	0\\
1653	0\\
1654	0\\
1655	0\\
1656	0\\
1657	0\\
1658	0\\
1659	0\\
1660	0\\
1661	0\\
1662	0\\
1663	0\\
1664	0\\
1665	0\\
1666	0\\
1667	0\\
1668	0\\
1669	0\\
1670	0\\
1671	0\\
1672	0\\
1673	0\\
1674	0\\
1675	0\\
1676	0\\
1677	0\\
1678	0\\
1679	0\\
1680	0\\
1681	0\\
1682	0\\
1683	0\\
1684	0\\
1685	0\\
1686	0\\
1687	0\\
1688	0\\
1689	0\\
1690	0\\
1691	0\\
1692	0\\
1693	0\\
1694	0\\
1695	0\\
1696	0\\
1697	0\\
1698	0\\
1699	0\\
1700	0\\
1701	0\\
1702	0\\
1703	0\\
1704	0\\
1705	0\\
1706	0\\
1707	0\\
1708	0\\
1709	0\\
1710	0\\
1711	0\\
1712	0\\
1713	0\\
1714	0\\
1715	0\\
1716	0\\
1717	0\\
1718	0\\
1719	0\\
1720	0\\
1721	0\\
1722	0\\
1723	0\\
1724	0\\
1725	0\\
1726	0\\
1727	0\\
1728	0\\
1729	0\\
1730	0\\
1731	0\\
1732	0\\
1733	0\\
1734	0\\
1735	0\\
1736	0\\
1737	0\\
1738	0\\
1739	0\\
1740	0\\
1741	0\\
1742	0\\
1743	0\\
1744	0\\
1745	0\\
1746	0\\
1747	0\\
1748	0\\
1749	0\\
1750	0\\
1751	0\\
1752	0\\
1753	0\\
1754	0\\
1755	0\\
1756	0\\
1757	0\\
1758	0\\
1759	0\\
1760	0\\
1761	0\\
1762	0\\
1763	0\\
1764	0\\
1765	0\\
1766	0\\
1767	0\\
1768	0\\
1769	0\\
1770	0\\
1771	0\\
1772	0\\
1773	0\\
1774	0\\
1775	0\\
1776	0\\
1777	0\\
1778	0\\
1779	0\\
1780	0\\
1781	0\\
1782	0\\
1783	0\\
1784	0\\
1785	0\\
1786	0\\
1787	0\\
1788	0\\
1789	0\\
1790	0\\
1791	0\\
1792	0\\
1793	0\\
1794	0\\
1795	0\\
1796	0\\
1797	0\\
1798	0\\
1799	0\\
1800	0\\
1801	0\\
1802	0\\
1803	0\\
1804	0\\
1805	0\\
1806	0\\
1807	0\\
1808	0\\
1809	0\\
1810	0\\
1811	0\\
1812	0\\
1813	0\\
1814	0\\
1815	0\\
1816	0\\
1817	0\\
1818	0\\
1819	0\\
1820	0\\
1821	0\\
1822	0\\
1823	0\\
1824	0\\
1825	0\\
1826	0\\
1827	0\\
1828	0\\
1829	0\\
1830	0\\
1831	0\\
1832	0\\
1833	0\\
1834	0\\
1835	0\\
1836	0\\
1837	0\\
1838	0\\
1839	0\\
1840	0\\
1841	0\\
1842	0\\
1843	0\\
1844	0\\
1845	0\\
1846	0\\
1847	0\\
1848	0\\
1849	0\\
1850	0\\
1851	0\\
1852	0\\
1853	0\\
1854	0\\
1855	0\\
1856	0\\
1857	0\\
1858	0\\
1859	0\\
1860	0\\
1861	0\\
1862	0\\
1863	0\\
1864	0\\
1865	0\\
1866	0\\
1867	0\\
1868	0\\
1869	0\\
1870	0\\
1871	0\\
1872	0\\
1873	0\\
1874	0\\
1875	0\\
1876	0\\
1877	0\\
1878	0\\
1879	0\\
1880	0\\
1881	0\\
1882	0\\
1883	0\\
1884	0\\
1885	0\\
1886	0\\
1887	0\\
1888	0\\
1889	0\\
1890	0\\
1891	0\\
1892	0\\
1893	0\\
1894	0\\
1895	0\\
1896	0\\
1897	0\\
1898	0\\
1899	0\\
1900	0\\
1901	0\\
1902	0\\
1903	0\\
1904	0\\
1905	0\\
1906	0\\
1907	0\\
1908	0\\
1909	0\\
1910	0\\
1911	0\\
1912	0\\
1913	0\\
1914	0\\
1915	0\\
1916	0\\
1917	0\\
1918	0\\
1919	0\\
1920	0\\
1921	0\\
1922	0\\
1923	0\\
1924	0\\
1925	0\\
1926	0\\
1927	0\\
1928	0\\
1929	0\\
1930	0\\
1931	0\\
1932	0\\
1933	0\\
1934	0\\
1935	0\\
1936	0\\
1937	0\\
1938	0\\
1939	0\\
1940	0\\
1941	0\\
1942	0\\
1943	0\\
1944	0\\
1945	0\\
1946	0\\
1947	0\\
1948	0\\
1949	0\\
1950	0\\
1951	0\\
1952	0\\
1953	0\\
1954	0\\
1955	0\\
1956	0\\
1957	0\\
1958	0\\
1959	0\\
1960	0\\
1961	0\\
1962	0\\
1963	0\\
1964	0\\
1965	0\\
1966	0\\
1967	0\\
1968	0\\
1969	0\\
1970	0\\
1971	0\\
1972	0\\
1973	0\\
1974	0\\
1975	0\\
1976	0\\
1977	0\\
1978	0\\
1979	0\\
1980	0\\
1981	0\\
1982	0\\
1983	0\\
1984	0\\
1985	0\\
1986	0\\
1987	0\\
1988	0\\
1989	0\\
1990	0\\
1991	0\\
1992	0\\
1993	0\\
1994	0\\
1995	0\\
1996	0\\
1997	0\\
1998	0\\
1999	0\\
2000	0\\
2001	0\\
2002	0\\
2003	0\\
2004	0\\
2005	0\\
2006	0\\
2007	0\\
2008	0\\
2009	0\\
2010	0\\
2011	0\\
2012	0\\
2013	0\\
2014	0\\
2015	0\\
2016	0\\
2017	0\\
2018	0\\
2019	0\\
2020	0\\
2021	0\\
2022	0\\
2023	0\\
2024	0\\
2025	0\\
2026	0\\
2027	0\\
2028	0\\
2029	0\\
2030	0\\
2031	0\\
2032	0\\
2033	0\\
2034	0\\
2035	0\\
2036	1.469982\\
2037	1.019151\\
2038	0.460767\\
2039	0.460837\\
2040	0.192769\\
2041	0\\
2042	0\\
2043	0\\
2044	0\\
2045	0\\
2046	0\\
2047	0\\
2048	0.297745\\
2049	0.072028\\
2050	0\\
2051	2e-05\\
2052	0\\
2053	0\\
2054	0\\
2055	0\\
2056	0\\
2057	5e-06\\
2058	1.620671\\
2059	2.583097\\
2060	2.583097\\
2061	1.622425\\
2062	0.361106\\
2063	0.7112\\
2064	4e-06\\
2065	0\\
2066	0\\
2067	0\\
2068	0\\
2069	0\\
2070	0\\
2071	1e-06\\
2072	0\\
2073	0\\
2074	0\\
2075	0\\
2076	0\\
2077	0\\
2078	0\\
2079	0\\
2080	0\\
2081	0\\
2082	0\\
2083	0.399228\\
2084	0\\
2085	0\\
2086	0\\
2087	0\\
2088	0\\
2089	0\\
2090	0\\
2091	0\\
2092	0\\
2093	0\\
2094	0\\
2095	0\\
2096	0\\
2097	0\\
2098	0\\
2099	0\\
2100	0\\
2101	0\\
2102	0\\
2103	0\\
2104	0\\
2105	0\\
2106	0\\
2107	0\\
2108	0\\
2109	0\\
2110	0\\
2111	0\\
2112	0\\
2113	0\\
2114	0\\
2115	0\\
2116	0\\
2117	0\\
2118	0\\
2119	0\\
2120	0\\
2121	0\\
2122	0\\
2123	0\\
2124	0\\
2125	0\\
2126	0\\
2127	0\\
2128	0\\
2129	0\\
2130	0\\
2131	0\\
2132	0\\
2133	0\\
2134	0\\
2135	0\\
2136	0\\
2137	0\\
2138	0\\
2139	0\\
2140	0\\
2141	0\\
2142	0\\
2143	0.314531\\
2144	0\\
2145	0\\
2146	0\\
2147	0\\
2148	0\\
2149	0\\
2150	0\\
2151	0\\
2152	0\\
2153	0\\
2154	0\\
2155	0\\
2156	0.905372\\
2157	0\\
2158	0\\
2159	0\\
2160	0\\
2161	0\\
2162	0\\
2163	0\\
2164	0\\
2165	0\\
2166	2.277406\\
2167	2.314827\\
2168	2.314827\\
2169	2.133942\\
2170	1.985004\\
2171	1.830616\\
2172	0.794485\\
2173	0.812134\\
2174	0.7955\\
2175	1.087827\\
2176	1.499705\\
2177	2.314827\\
2178	2.478995\\
2179	2.314827\\
2180	2.314827\\
2181	2.314827\\
2182	2.314827\\
2183	2.298741\\
2184	0\\
2185	0\\
2186	0\\
2187	0\\
2188	0\\
2189	0\\
2190	1.317321\\
2191	2.314827\\
2192	2.314827\\
2193	2.249228\\
2194	2.279606\\
2195	1.720862\\
2196	1.317321\\
2197	1.316786\\
2198	0.954991\\
2199	0.447443\\
2200	0.912811\\
2201	1.489009\\
2202	2.314827\\
2203	2.314827\\
2204	2.314827\\
2205	2.314827\\
2206	2.314827\\
2207	1.087827\\
2208	0\\
2209	0\\
2210	0\\
2211	0\\
2212	0\\
2213	0\\
2214	0.00972\\
2215	2.07283\\
2216	1.677531\\
2217	1.317321\\
2218	1.168145\\
2219	1.317321\\
2220	0.412976\\
2221	0.412977\\
2222	0.412976\\
2223	0.238992\\
2224	1.087826\\
2225	2.314826\\
2226	2.34395\\
2227	2.813579\\
2228	2.54323\\
2229	2.314827\\
2230	2.314827\\
2231	1.500393\\
2232	0\\
2233	0\\
2234	0\\
2235	0\\
2236	0\\
2237	0\\
2238	1.317321\\
2239	2.316032\\
2240	2.813579\\
2241	2.8969\\
2242	2.980422\\
2243	3.596576\\
2244	2.314827\\
2245	2.314827\\
2246	2.095076\\
2247	1.317321\\
2248	1.393765\\
2249	2.314827\\
2250	2.314827\\
2251	2.082137\\
2252	2.314827\\
2253	1.430236\\
2254	2.314827\\
2255	1.657748\\
2256	0\\
2257	0\\
2258	1e-06\\
2259	0\\
2260	0\\
2261	0\\
2262	0\\
2263	0\\
2264	0\\
2265	0\\
2266	0\\
2267	0\\
2268	0\\
2269	0\\
2270	0\\
2271	0\\
2272	0\\
2273	0\\
2274	0\\
2275	0\\
2276	0\\
2277	0\\
2278	0\\
2279	0\\
2280	0\\
2281	0\\
2282	0\\
2283	0\\
2284	0\\
2285	0\\
2286	0\\
2287	0\\
2288	0\\
2289	0\\
2290	0\\
2291	0\\
2292	0\\
2293	0\\
2294	0\\
2295	0\\
2296	0\\
2297	0\\
2298	0\\
2299	0\\
2300	0\\
2301	0\\
2302	0\\
2303	0\\
2304	0\\
2305	0\\
2306	0\\
2307	0\\
2308	0\\
2309	0\\
2310	0\\
2311	0\\
2312	1.317321\\
2313	1.734169\\
2314	2.162625\\
2315	2.01612\\
2316	0.823287\\
2317	0.975175\\
2318	0.510521\\
2319	0.725372\\
2320	0.594864\\
2321	1.317318\\
2322	0.626717\\
2323	0.423494\\
2324	0.985288\\
2325	1.51255\\
2326	1.317321\\
2327	4e-06\\
2328	0\\
2329	0\\
2330	0\\
2331	0\\
2332	0\\
2333	0\\
2334	0\\
2335	0.396392\\
2336	1.317321\\
2337	1.087827\\
2338	0.699051\\
2339	1.115002\\
2340	1e-06\\
2341	0\\
2342	0\\
2343	0\\
2344	7.6e-05\\
2345	1.31732\\
2346	0.100865\\
2347	1.00517\\
2348	2.314827\\
2349	2.314827\\
2350	2.314827\\
2351	2.314618\\
2352	0\\
2353	0\\
2354	0\\
2355	1e-06\\
2356	0\\
2357	0\\
2358	2.314827\\
2359	2.314827\\
2360	2.315994\\
2361	2.349212\\
2362	2.314827\\
2363	2.314827\\
2364	0.552453\\
2365	0.345095\\
2366	0\\
2367	0\\
2368	0.135736\\
2369	1.322791\\
2370	2.314827\\
2371	2.314827\\
2372	2.16202\\
2373	2.314827\\
2374	2.314827\\
2375	2.314827\\
2376	0.13483\\
2377	0\\
2378	0\\
2379	0\\
2380	0\\
2381	0\\
2382	2.314827\\
2383	2.718571\\
2384	2.770039\\
2385	2.346802\\
2386	2.314827\\
2387	2.314827\\
2388	1.570747\\
2389	1.744193\\
2390	1.448217\\
2391	1.566297\\
2392	1.896264\\
2393	2.314827\\
2394	2.595316\\
2395	2.314827\\
2396	2.314827\\
2397	2.314827\\
2398	2.314827\\
2399	2.314827\\
2400	0.100625\\
2401	0\\
2402	0\\
2403	0\\
2404	0\\
2405	0\\
2406	1.622285\\
2407	2.314827\\
2408	2.314906\\
2409	2.314827\\
2410	2.314827\\
2411	2.314827\\
2412	1.809712\\
2413	1.212978\\
2414	0.609824\\
2415	1.055435\\
2416	1.087772\\
2417	1.317321\\
2418	1.924041\\
2419	1.40707\\
2420	1.725015\\
2421	2.249432\\
2422	2.314819\\
2423	1.317317\\
2424	0\\
2425	0\\
2426	0\\
2427	0\\
2428	0\\
2429	0\\
2430	0\\
2431	0\\
2432	0\\
2433	1e-05\\
2434	0\\
2435	0\\
2436	0\\
2437	0\\
2438	0\\
2439	0\\
2440	0\\
2441	0\\
2442	0\\
2443	0\\
2444	0\\
2445	0\\
2446	0\\
2447	0\\
2448	0\\
2449	0\\
2450	0\\
2451	0\\
2452	0\\
2453	0\\
2454	0\\
2455	0\\
2456	0\\
2457	0\\
2458	0\\
2459	0\\
2460	0\\
2461	0\\
2462	0\\
2463	0\\
2464	0\\
2465	0\\
2466	0\\
2467	0\\
2468	0\\
2469	1e-06\\
2470	0\\
2471	0\\
2472	0\\
2473	0\\
2474	0\\
2475	0\\
2476	0\\
2477	0\\
2478	0\\
2479	0\\
2480	0\\
2481	0\\
2482	0\\
2483	0\\
2484	0\\
2485	0\\
2486	0\\
2487	0\\
2488	0\\
2489	0\\
2490	0\\
2491	0\\
2492	0\\
2493	0\\
2494	0\\
2495	0\\
2496	0\\
2497	0\\
2498	0\\
2499	0\\
2500	0\\
2501	0\\
2502	0\\
2503	1e-06\\
2504	0.732802\\
2505	0.412974\\
2506	0\\
2507	0\\
2508	0\\
2509	0\\
2510	0\\
2511	0\\
2512	0\\
2513	0\\
2514	0\\
2515	0\\
2516	0\\
2517	3e-06\\
2518	0.412976\\
2519	0\\
2520	0\\
2521	0\\
2522	0\\
2523	0\\
2524	0\\
2525	0\\
2526	0\\
2527	2.3e-05\\
2528	0\\
2529	0\\
2530	0\\
2531	0\\
2532	0\\
2533	0\\
2534	1e-06\\
2535	0\\
2536	0\\
2537	0\\
2538	0\\
2539	3e-06\\
2540	3.8e-05\\
2541	0\\
2542	0.412788\\
2543	1e-06\\
2544	0\\
2545	0\\
2546	0\\
2547	0\\
2548	0\\
2549	0\\
2550	0\\
2551	0\\
2552	0\\
2553	0\\
2554	0\\
2555	0\\
2556	0\\
2557	0\\
2558	0\\
2559	0\\
2560	0\\
2561	0\\
2562	0\\
2563	0\\
2564	0\\
2565	0\\
2566	0\\
2567	0\\
2568	0\\
2569	0\\
2570	0\\
2571	0\\
2572	0\\
2573	0\\
2574	0\\
2575	0\\
2576	0\\
2577	0\\
2578	0\\
2579	0\\
2580	0\\
2581	0\\
2582	0\\
2583	0\\
2584	0\\
2585	0\\
2586	0\\
2587	0\\
2588	0\\
2589	0\\
2590	0\\
2591	0\\
2592	0\\
2593	0\\
2594	0\\
2595	0\\
2596	0\\
2597	0\\
2598	0\\
2599	0\\
2600	0\\
2601	0\\
2602	0\\
2603	0\\
2604	0\\
2605	0\\
2606	0\\
2607	0\\
2608	0\\
2609	0\\
2610	0\\
2611	0\\
2612	0\\
2613	0\\
2614	0\\
2615	0\\
2616	0\\
2617	0\\
2618	0\\
2619	0\\
2620	0\\
2621	0\\
2622	0\\
2623	0\\
2624	0\\
2625	0\\
2626	0\\
2627	0\\
2628	0\\
2629	0\\
2630	0\\
2631	0\\
2632	0\\
2633	0\\
2634	0\\
2635	0\\
2636	0\\
2637	0\\
2638	0\\
2639	0\\
2640	0\\
2641	0\\
2642	0\\
2643	0\\
2644	0\\
2645	0\\
2646	0\\
2647	0\\
2648	0\\
2649	0\\
2650	0\\
2651	0\\
2652	0\\
2653	0\\
2654	0\\
2655	0\\
2656	0\\
2657	0\\
2658	0\\
2659	0\\
2660	0\\
2661	0\\
2662	0\\
2663	0\\
2664	0\\
2665	0\\
2666	0\\
2667	0\\
2668	0\\
2669	0\\
2670	0\\
2671	1e-06\\
2672	0\\
2673	0\\
2674	0\\
2675	0\\
2676	0\\
2677	0\\
2678	0\\
2679	0\\
2680	0\\
2681	0\\
2682	6e-06\\
2683	0.227885\\
2684	0\\
2685	0\\
2686	0.039218\\
2687	0\\
2688	1e-06\\
2689	0\\
2690	0\\
2691	0\\
2692	0\\
2693	0\\
2694	0\\
2695	0\\
2696	0\\
2697	0\\
2698	0\\
2699	0\\
2700	0\\
2701	0\\
2702	0\\
2703	0\\
2704	0\\
2705	0\\
2706	0.412976\\
2707	0.412976\\
2708	0\\
2709	1.087826\\
2710	0.702618\\
2711	0\\
2712	0\\
2713	0\\
2714	0\\
2715	0\\
2716	0\\
2717	0\\
2718	0\\
2719	0.412977\\
2720	1.155222\\
2721	1.317321\\
2722	1.034625\\
2723	1.317321\\
2724	0.588913\\
2725	0.412976\\
2726	0\\
2727	0.0655\\
2728	0.164454\\
2729	0.445883\\
2730	0.829395\\
2731	0.249396\\
2732	0\\
2733	0\\
2734	0\\
2735	0\\
2736	0\\
2737	0\\
2738	0\\
2739	0\\
2740	0\\
2741	0\\
2742	0\\
2743	0\\
2744	0\\
2745	0\\
2746	1e-06\\
2747	0\\
2748	0\\
2749	0\\
2750	0\\
2751	0\\
2752	0\\
2753	0\\
2754	0\\
2755	0\\
2756	0\\
2757	0\\
2758	0\\
2759	0\\
2760	0\\
2761	0\\
2762	0\\
2763	0\\
2764	0\\
2765	0\\
2766	0\\
2767	0\\
2768	0\\
2769	0\\
2770	0\\
2771	0\\
2772	0\\
2773	0\\
2774	0\\
2775	0\\
2776	0\\
2777	0\\
2778	0\\
2779	0\\
2780	0\\
2781	0\\
2782	0\\
2783	0\\
2784	0\\
2785	0\\
2786	0\\
2787	0\\
2788	0\\
2789	0\\
2790	0\\
2791	0\\
2792	0\\
2793	0\\
2794	0\\
2795	0\\
2796	0\\
2797	0\\
2798	0\\
2799	0\\
2800	0\\
2801	0\\
2802	0\\
2803	0\\
2804	0\\
2805	0\\
2806	0\\
2807	0\\
2808	0\\
2809	0\\
2810	0\\
2811	0\\
2812	0\\
2813	0\\
2814	1e-06\\
2815	0\\
2816	0\\
2817	0\\
2818	0\\
2819	0\\
2820	0\\
2821	0\\
2822	0\\
2823	0\\
2824	0\\
2825	0\\
2826	5e-06\\
2827	0.138669\\
2828	0\\
2829	0\\
2830	0\\
2831	0\\
2832	0\\
2833	0\\
2834	0\\
2835	0\\
2836	0\\
2837	0\\
2838	0\\
2839	0\\
2840	3e-06\\
2841	0.588161\\
2842	0.241214\\
2843	1.087826\\
2844	0.412975\\
2845	1.087826\\
2846	1.089396\\
2847	0.649225\\
2848	0.413334\\
2849	0\\
2850	9e-06\\
2851	0\\
2852	0\\
2853	0\\
2854	0\\
2855	0\\
2856	0\\
2857	0\\
2858	0\\
2859	0\\
2860	0\\
2861	0\\
2862	0\\
2863	0\\
2864	0\\
2865	1.1e-05\\
2866	2e-06\\
2867	0\\
2868	0\\
2869	0\\
2870	0\\
2871	0\\
2872	0\\
2873	0\\
2874	0\\
2875	0\\
2876	0\\
2877	0\\
2878	0\\
2879	0\\
2880	0\\
2881	0\\
2882	0\\
2883	0\\
2884	0\\
2885	0\\
2886	0\\
2887	0\\
2888	0\\
2889	0\\
2890	0\\
2891	0\\
2892	0\\
2893	0\\
2894	0\\
2895	0\\
2896	0\\
2897	0\\
2898	0\\
2899	0\\
2900	0\\
2901	0\\
2902	0\\
2903	0\\
2904	0\\
2905	0\\
2906	0\\
2907	0\\
2908	0\\
2909	0\\
2910	0\\
2911	0\\
2912	0\\
2913	0\\
2914	0\\
2915	0\\
2916	0\\
2917	0\\
2918	0\\
2919	0\\
2920	0\\
2921	0\\
2922	0\\
2923	0\\
2924	0\\
2925	0\\
2926	0\\
2927	0\\
2928	0\\
2929	0\\
2930	0\\
2931	0\\
2932	0\\
2933	0\\
2934	0\\
2935	0\\
2936	0\\
2937	0\\
2938	0\\
2939	0\\
2940	0\\
2941	0\\
2942	0\\
2943	0\\
2944	0\\
2945	0\\
2946	0\\
2947	0\\
2948	0\\
2949	0\\
2950	0\\
2951	0\\
2952	0\\
2953	0\\
2954	0\\
2955	0\\
2956	0\\
2957	0\\
2958	0\\
2959	0\\
2960	0\\
2961	0\\
2962	0\\
2963	0\\
2964	0\\
2965	0\\
2966	0\\
2967	0\\
2968	0\\
2969	0\\
2970	0\\
2971	0\\
2972	0\\
2973	0\\
2974	0\\
2975	0\\
2976	0\\
2977	0\\
2978	0\\
2979	0\\
2980	0\\
2981	0\\
2982	0\\
2983	0\\
2984	0\\
2985	1e-06\\
2986	0\\
2987	0\\
2988	0\\
2989	0\\
2990	0\\
2991	0\\
2992	0\\
2993	0\\
2994	0\\
2995	0\\
2996	0\\
2997	0\\
2998	0\\
2999	0\\
3000	0\\
3001	0\\
3002	0\\
3003	0\\
3004	0\\
3005	0\\
3006	0\\
3007	0\\
3008	0\\
3009	0.167402\\
3010	8e-06\\
3011	0\\
3012	0\\
3013	0\\
3014	0\\
3015	0\\
3016	0\\
3017	0\\
3018	0\\
3019	0\\
3020	0\\
3021	0\\
3022	0\\
3023	0\\
3024	0\\
3025	0\\
3026	0\\
3027	0\\
3028	0\\
3029	0\\
3030	0\\
3031	0\\
3032	0.113309\\
3033	0\\
3034	0\\
3035	1e-06\\
3036	0\\
3037	0\\
3038	0\\
3039	0\\
3040	0\\
3041	0\\
3042	0\\
3043	0\\
3044	0\\
3045	0\\
3046	3e-06\\
3047	0\\
3048	0\\
3049	0\\
3050	0\\
3051	0\\
3052	0\\
3053	0\\
3054	0\\
3055	0\\
3056	0\\
3057	0\\
3058	0.000424\\
3059	0.783973\\
3060	1.1e-05\\
3061	0.386724\\
3062	0.386722\\
3063	0.452283\\
3064	0.472219\\
3065	0.919021\\
3066	0.934026\\
3067	0\\
3068	0\\
3069	0\\
3070	0\\
3071	0\\
3072	0\\
3073	0\\
3074	0\\
3075	0\\
3076	0\\
3077	0\\
3078	0\\
3079	0\\
3080	0\\
3081	0\\
3082	0\\
3083	0\\
3084	0\\
3085	0\\
3086	0\\
3087	0\\
3088	0\\
3089	0\\
3090	0\\
3091	0\\
3092	0\\
3093	0\\
3094	0\\
3095	0\\
3096	0\\
3097	0\\
3098	0\\
3099	0\\
3100	0\\
3101	0\\
3102	0\\
3103	0\\
3104	0\\
3105	0\\
3106	0\\
3107	0\\
3108	0\\
3109	0\\
3110	0\\
3111	0\\
3112	0\\
3113	0\\
3114	0\\
3115	0\\
3116	0\\
3117	0\\
3118	0\\
3119	0\\
3120	0\\
3121	0\\
3122	0\\
3123	0\\
3124	0\\
3125	0\\
3126	0\\
3127	0\\
3128	0\\
3129	0\\
3130	0\\
3131	0\\
3132	0\\
3133	0\\
3134	0\\
3135	0\\
3136	0\\
3137	0\\
3138	0\\
3139	0\\
3140	0\\
3141	0\\
3142	0\\
3143	0\\
3144	0\\
3145	0\\
3146	0\\
3147	0\\
3148	0\\
3149	0\\
3150	0\\
3151	0\\
3152	0\\
3153	0\\
3154	0\\
3155	0\\
3156	0\\
3157	0\\
3158	0\\
3159	0\\
3160	0\\
3161	0\\
3162	0\\
3163	0\\
3164	0.1535\\
3165	0\\
3166	0.386718\\
3167	0\\
3168	0\\
3169	0\\
3170	0\\
3171	0\\
3172	0\\
3173	0\\
3174	0\\
3175	0\\
3176	0.806984\\
3177	1.018673\\
3178	0\\
3179	1e-06\\
3180	0.877499\\
3181	0\\
3182	0\\
3183	0\\
3184	0\\
3185	0\\
3186	0.027219\\
3187	0\\
3188	0\\
3189	0\\
3190	0.571726\\
3191	0\\
3192	0\\
3193	0\\
3194	0\\
3195	0\\
3196	0\\
3197	0\\
3198	0\\
3199	0.482925\\
3200	0.264463\\
3201	0\\
3202	1e-06\\
3203	0\\
3204	0\\
3205	0\\
3206	0\\
3207	0\\
3208	0\\
3209	0\\
3210	0\\
3211	0\\
3212	0\\
3213	0\\
3214	2e-06\\
3215	0\\
3216	0\\
3217	0\\
3218	0\\
3219	0\\
3220	0\\
3221	0\\
3222	0\\
3223	1e-06\\
3224	0.099292\\
3225	1.230007\\
3226	6e-06\\
3227	0\\
3228	0\\
3229	0\\
3230	0\\
3231	0\\
3232	0\\
3233	0\\
3234	0\\
3235	0\\
3236	0\\
3237	0\\
3238	5e-06\\
3239	0\\
3240	0\\
3241	0\\
3242	0\\
3243	0\\
3244	0\\
3245	0\\
3246	0\\
3247	0\\
3248	1e-06\\
3249	0\\
3250	0\\
3251	0\\
3252	0\\
3253	0\\
3254	0\\
3255	0\\
3256	0\\
3257	0\\
3258	0\\
3259	0\\
3260	0\\
3261	0\\
3262	0\\
3263	0\\
3264	0\\
3265	0\\
3266	0\\
3267	0\\
3268	0\\
3269	0\\
3270	0\\
3271	0\\
3272	0\\
3273	0\\
3274	0\\
3275	0\\
3276	0\\
3277	0\\
3278	0\\
3279	0\\
3280	0\\
3281	0\\
3282	0\\
3283	0\\
3284	0\\
3285	0\\
3286	0\\
3287	0\\
3288	0\\
3289	0\\
3290	0\\
3291	0\\
3292	0\\
3293	0\\
3294	0\\
3295	0\\
3296	0\\
3297	0\\
3298	0\\
3299	0\\
3300	0\\
3301	0\\
3302	0\\
3303	0\\
3304	0\\
3305	0\\
3306	0\\
3307	0\\
3308	0\\
3309	0\\
3310	0\\
3311	0\\
3312	0\\
3313	0\\
3314	0\\
3315	0\\
3316	0\\
3317	0\\
3318	0\\
3319	0\\
3320	0\\
3321	0\\
3322	0\\
3323	0\\
3324	0\\
3325	0\\
3326	0\\
3327	0\\
3328	0\\
3329	0\\
3330	0\\
3331	0\\
3332	0\\
3333	0\\
3334	0\\
3335	0\\
3336	0\\
3337	0\\
3338	0\\
3339	0\\
3340	0\\
3341	0\\
3342	0\\
3343	0\\
3344	0\\
3345	0\\
3346	0\\
3347	0\\
3348	0\\
3349	0\\
3350	0\\
3351	0\\
3352	0\\
3353	4e-06\\
3354	0.386724\\
3355	0.894831\\
3356	1e-06\\
3357	0.391413\\
3358	0\\
3359	0\\
3360	0\\
3361	0\\
3362	0\\
3363	0\\
3364	0\\
3365	0\\
3366	0\\
3367	0\\
3368	0\\
3369	0.603811\\
3370	0\\
3371	1.187502\\
3372	0.780669\\
3373	0.258023\\
3374	0.386724\\
3375	0.850344\\
3376	0.547766\\
3377	0.419534\\
3378	0.386723\\
3379	2e-06\\
3380	0\\
3381	0\\
3382	1e-06\\
3383	0\\
3384	0\\
3385	0\\
3386	0\\
3387	0\\
3388	0\\
3389	0\\
3390	0\\
3391	0\\
3392	0\\
3393	0\\
3394	1e-06\\
3395	0\\
3396	0\\
3397	0\\
3398	0\\
3399	0\\
3400	0\\
3401	1e-06\\
3402	0\\
3403	0\\
3404	0\\
3405	0\\
3406	0\\
3407	0\\
3408	0\\
3409	0\\
3410	0\\
3411	0\\
3412	0\\
3413	0\\
3414	0\\
3415	0\\
3416	0\\
3417	0\\
3418	0\\
3419	0\\
3420	0\\
3421	0\\
3422	0\\
3423	0\\
3424	0\\
3425	0\\
3426	0\\
3427	1e-06\\
3428	0\\
3429	0\\
3430	0\\
3431	0\\
3432	0\\
3433	0\\
3434	0\\
3435	0\\
3436	0\\
3437	0\\
3438	0\\
3439	0\\
3440	0\\
3441	0\\
3442	0\\
3443	0\\
3444	0\\
3445	0\\
3446	0\\
3447	0\\
3448	0\\
3449	0\\
3450	0\\
3451	0\\
3452	0\\
3453	0\\
3454	0\\
3455	0\\
3456	0\\
3457	0\\
3458	0\\
3459	0\\
3460	0\\
3461	0\\
3462	0\\
3463	0\\
3464	0\\
3465	0\\
3466	0\\
3467	0\\
3468	0\\
3469	0\\
3470	0\\
3471	0\\
3472	0\\
3473	0\\
3474	0\\
3475	0\\
3476	0\\
3477	0\\
3478	0\\
3479	0\\
3480	0\\
3481	0\\
3482	0\\
3483	0\\
3484	0\\
3485	0\\
3486	0\\
3487	1e-06\\
3488	2e-06\\
3489	0.838923\\
3490	1.034809\\
3491	1.233577\\
3492	0.209656\\
3493	0\\
3494	0\\
3495	0\\
3496	0\\
3497	1e-06\\
3498	0\\
3499	0\\
3500	0\\
3501	0\\
3502	0\\
3503	0\\
3504	0\\
3505	0\\
3506	0\\
3507	0\\
3508	0\\
3509	0\\
3510	0\\
3511	0\\
3512	2e-06\\
3513	0.741001\\
3514	1.233579\\
3515	1.018925\\
3516	0\\
3517	0\\
3518	0\\
3519	0.386724\\
3520	0.386724\\
3521	0.386724\\
3522	0.386724\\
3523	0\\
3524	0\\
3525	0\\
3526	0\\
3527	0\\
3528	0\\
3529	0\\
3530	0\\
3531	0\\
3532	0\\
3533	0\\
3534	0\\
3535	0\\
3536	0\\
3537	0.612381\\
3538	1.233579\\
3539	1.180816\\
3540	1.233579\\
3541	1.018674\\
3542	0.337421\\
3543	0\\
3544	0\\
3545	4.1e-05\\
3546	1e-05\\
3547	0\\
3548	0\\
3549	0\\
3550	0\\
3551	0\\
3552	0\\
3553	0\\
3554	0\\
3555	0\\
3556	0\\
3557	0\\
3558	0\\
3559	0\\
3560	0\\
3561	0\\
3562	0\\
3563	0\\
3564	0\\
3565	0\\
3566	0\\
3567	0\\
3568	0\\
3569	0\\
3570	0\\
3571	0\\
3572	0\\
3573	0\\
3574	0\\
3575	0\\
3576	0\\
3577	0\\
3578	0\\
3579	0\\
3580	0\\
3581	0\\
3582	0\\
3583	0\\
3584	0\\
3585	0\\
3586	0\\
3587	0\\
3588	0\\
3589	0\\
3590	0\\
3591	0\\
3592	0\\
3593	0\\
3594	0\\
3595	0\\
3596	0\\
3597	0\\
3598	0\\
3599	0\\
3600	0\\
3601	0\\
3602	0\\
3603	0\\
3604	0\\
3605	0\\
3606	0\\
3607	0\\
3608	0\\
3609	0\\
3610	0\\
3611	0\\
3612	0\\
3613	0\\
3614	0\\
3615	0\\
3616	0\\
3617	0\\
3618	0\\
3619	0\\
3620	0\\
3621	0\\
3622	0\\
3623	0\\
3624	0\\
3625	0\\
3626	0\\
3627	0\\
3628	0\\
3629	0\\
3630	0\\
3631	0\\
3632	0\\
3633	0\\
3634	0\\
3635	0\\
3636	0\\
3637	0\\
3638	0\\
3639	0\\
3640	0\\
3641	0\\
3642	0\\
3643	0\\
3644	0\\
3645	0\\
3646	0\\
3647	1e-06\\
3648	0\\
3649	0\\
3650	0\\
3651	0\\
3652	0\\
3653	0\\
3654	0\\
3655	1.223764\\
3656	1.693382\\
3657	1.693379\\
3658	1.115053\\
3659	1.115053\\
3660	0.727875\\
3661	0.764794\\
3662	0.764794\\
3663	0.764794\\
3664	1.105196\\
3665	1.693382\\
3666	1.120149\\
3667	1.502566\\
3668	1.079317\\
3669	1.11505\\
3670	1.865811\\
3671	0.920796\\
3672	0\\
3673	0\\
3674	0\\
3675	0\\
3676	0\\
3677	0\\
3678	0\\
3679	1.693382\\
3680	1.959397\\
3681	1.959397\\
3682	1.808165\\
3683	1.693382\\
3684	0.764794\\
3685	0.056884\\
3686	0.488255\\
3687	0.920796\\
3688	1.360236\\
3689	1.342407\\
3690	1.959397\\
3691	1.959397\\
3692	1.640509\\
3693	1.115052\\
3694	1.693382\\
3695	0.806083\\
3696	1e-06\\
3697	0\\
3698	0\\
3699	0\\
3700	0\\
3701	0\\
3702	0.920796\\
3703	1.959397\\
3704	3.205717\\
3705	3.205717\\
3706	3.401897\\
3707	3.332898\\
3708	3.401897\\
3709	3.401897\\
3710	2.381569\\
3711	2.351588\\
3712	2.170991\\
3713	2.381565\\
3714	3.199947\\
3715	2.273837\\
3716	1.959397\\
3717	1.280202\\
3718	0.764794\\
3719	1e-06\\
3720	0\\
3721	0\\
3722	0\\
3723	0\\
3724	0\\
3725	0\\
3726	0\\
3727	0\\
3728	0.764794\\
3729	0.764793\\
3730	0\\
3731	1e-06\\
3732	0\\
3733	0\\
3734	1e-06\\
3735	0\\
3736	0\\
3737	0\\
3738	0.375035\\
3739	0.764794\\
3740	0.393489\\
3741	0.088506\\
3742	1.115053\\
3743	0.764794\\
3744	0\\
3745	0\\
3746	0\\
3747	0\\
3748	0\\
3749	0\\
3750	0.049936\\
3751	1.959397\\
3752	1.959397\\
3753	1.693382\\
3754	1.115053\\
3755	0.824298\\
3756	0.764793\\
3757	0.300332\\
3758	1e-06\\
3759	1e-06\\
3760	0\\
3761	0.764793\\
3762	0.831261\\
3763	0.920796\\
3764	0.349566\\
3765	1e-06\\
3766	0.685846\\
3767	1e-06\\
3768	0\\
3769	0\\
3770	0\\
3771	0\\
3772	0\\
3773	0\\
3774	0\\
3775	0\\
3776	0\\
3777	0\\
3778	0\\
3779	0\\
3780	0\\
3781	0\\
3782	0\\
3783	0\\
3784	0\\
3785	0\\
3786	0\\
3787	0\\
3788	0\\
3789	0\\
3790	0\\
3791	0\\
3792	0\\
3793	0\\
3794	0\\
3795	0\\
3796	0\\
3797	0\\
3798	0\\
3799	0\\
3800	0\\
3801	0\\
3802	0\\
3803	0\\
3804	0\\
3805	0\\
3806	0\\
3807	0\\
3808	0\\
3809	0\\
3810	0\\
3811	0\\
3812	0\\
3813	0\\
3814	0\\
3815	0\\
3816	0\\
3817	0\\
3818	0\\
3819	0\\
3820	0\\
3821	0\\
3822	0\\
3823	0\\
3824	0\\
3825	0\\
3826	0\\
3827	0\\
3828	0\\
3829	0\\
3830	0\\
3831	0\\
3832	0\\
3833	0\\
3834	0\\
3835	0\\
3836	0\\
3837	0\\
3838	0\\
3839	0\\
3840	0\\
3841	0\\
3842	0\\
3843	0\\
3844	0\\
3845	0\\
3846	0\\
3847	1.693382\\
3848	2.098371\\
3849	2.495989\\
3850	2.251793\\
3851	2.381569\\
3852	1.959397\\
3853	1.959397\\
3854	1.959397\\
3855	1.959397\\
3856	1.959397\\
3857	1.959397\\
3858	2.493576\\
3859	2.381568\\
3860	1.959397\\
3861	1.959397\\
3862	1.959397\\
3863	1.959395\\
3864	0.697048\\
3865	0.187077\\
3866	0\\
3867	0\\
3868	0\\
3869	0.045635\\
3870	2.93933\\
3871	3.361507\\
3872	4.185656\\
3873	4.185656\\
3874	3.361506\\
3875	3.09178\\
3876	2.939335\\
3877	2.939335\\
3878	2.939335\\
3879	2.939335\\
3880	2.939335\\
3881	3.475943\\
3882	1.976398\\
3883	2.664675\\
3884	2.537748\\
3885	1.959397\\
3886	2.381572\\
3887	1.959397\\
3888	1.115053\\
3889	0.337629\\
3890	0\\
3891	0\\
3892	0\\
3893	3e-06\\
3894	1.507442\\
3895	2.591301\\
3896	3.205717\\
3897	2.225265\\
3898	2.123857\\
3899	1.959397\\
3900	1.693382\\
3901	1.829554\\
3902	1.693382\\
3903	1.693382\\
3904	1.959397\\
3905	1.959397\\
3906	2.795974\\
3907	2.725953\\
3908	2.381568\\
3909	2.044977\\
3910	2.787278\\
3911	1.959397\\
3912	1.959397\\
3913	1.115053\\
3914	0.742499\\
3915	0.349566\\
3916	0\\
3917	1e-05\\
3918	1.693383\\
3919	3.124015\\
3920	3.205717\\
3921	2.913634\\
3922	2.403187\\
3923	2.376236\\
3924	1.959397\\
3925	1.959397\\
3926	1.959397\\
3927	1.693383\\
3928	1.959397\\
3929	1.959397\\
3930	1.959397\\
3931	1.959397\\
3932	1.959397\\
3933	1.959397\\
3934	1.983972\\
3935	1.959397\\
3936	1.115053\\
3937	0.308663\\
3938	0.649688\\
3939	0.096245\\
3940	0\\
3941	0\\
3942	0\\
3943	0\\
3944	0.764794\\
3945	0.975156\\
3946	0.764792\\
3947	0.679183\\
3948	0\\
3949	0\\
3950	0\\
3951	0\\
3952	0\\
3953	0\\
3954	0\\
3955	0.764791\\
3956	0.402316\\
3957	0.242758\\
3958	1.115053\\
3959	0.764794\\
3960	0\\
3961	0\\
3962	0\\
3963	0\\
3964	0\\
3965	0\\
3966	0\\
3967	0\\
3968	0\\
3969	0\\
3970	0\\
3971	0\\
3972	0\\
3973	0\\
3974	0\\
3975	0\\
3976	0\\
3977	0\\
3978	0\\
3979	0\\
3980	2e-06\\
3981	0\\
3982	0.58917\\
3983	1e-06\\
3984	0\\
3985	0\\
3986	0\\
3987	0\\
3988	0\\
3989	1e-06\\
3990	0\\
3991	1.757027\\
3992	1.959397\\
3993	1.959397\\
3994	1.844944\\
3995	1.959397\\
3996	1.952889\\
3997	1.693382\\
3998	1.149902\\
3999	1.115053\\
4000	1.147688\\
4001	1.54023\\
};
\addplot [color=mycolor1,solid,line width=1.0pt,forget plot]
  table[row sep=crcr]{%
4001	1.54023\\
4002	1.722436\\
4003	1.543806\\
4004	1.115053\\
4005	0.934458\\
4006	1.294275\\
4007	0.764794\\
4008	0\\
4009	0\\
4010	0\\
4011	0\\
4012	0\\
4013	1e-06\\
4014	0.764794\\
4015	1.959396\\
4016	2.24662\\
4017	3.199947\\
4018	1.959397\\
4019	2.098358\\
4020	1.115053\\
4021	0.205704\\
4022	0.303727\\
4023	0.409268\\
4024	0.508245\\
4025	0.764794\\
4026	0.126219\\
4027	0.497841\\
4028	0.764794\\
4029	0.764794\\
4030	1.158209\\
4031	0.873228\\
4032	0\\
4033	0\\
4034	0\\
4035	0\\
4036	0\\
4037	0\\
4038	0.04578\\
4039	1.693382\\
4040	1.693382\\
4041	1.847006\\
4042	1.531936\\
4043	1.524752\\
4044	1.537616\\
4045	0.764794\\
4046	0.349566\\
4047	0.128196\\
4048	0.581163\\
4049	1.044599\\
4050	1.115053\\
4051	1.115053\\
4052	0.920902\\
4053	0.75197\\
4054	1.115053\\
4055	0.916125\\
4056	0\\
4057	0\\
4058	0\\
4059	0\\
4060	0\\
4061	0\\
4062	0.518798\\
4063	1.693382\\
4064	1.959397\\
4065	1.959397\\
4066	1.959397\\
4067	1.692586\\
4068	0.920406\\
4069	0.764794\\
4070	1.031998\\
4071	0.764796\\
4072	1.115053\\
4073	1.922381\\
4074	1.959397\\
4075	1.693382\\
4076	1.033323\\
4077	0.704635\\
4078	1.115053\\
4079	0.449269\\
4080	0\\
4081	0\\
4082	0\\
4083	0\\
4084	0\\
4085	0\\
4086	0\\
4087	0.764794\\
4088	1.305821\\
4089	1.513515\\
4090	1.693382\\
4091	1.514135\\
4092	0.92052\\
4093	0.643998\\
4094	1.115053\\
4095	0.920795\\
4096	1.115053\\
4097	1.507683\\
4098	1.509178\\
4099	1.693382\\
4100	1.531936\\
4101	1.115053\\
4102	1.830052\\
4103	1.526869\\
4104	0.264975\\
4105	0\\
4106	0\\
4107	0\\
4108	0\\
4109	0\\
4110	0\\
4111	0\\
4112	0.71664\\
4113	0.622102\\
4114	0.764794\\
4115	2.4e-05\\
4116	0\\
4117	0\\
4118	0\\
4119	0\\
4120	0\\
4121	0\\
4122	0\\
4123	0.154191\\
4124	0.254102\\
4125	0.134281\\
4126	0.686156\\
4127	0.638047\\
4128	0\\
4129	0\\
4130	0\\
4131	0\\
4132	0\\
4133	0\\
4134	0\\
4135	0\\
4136	0\\
4137	0\\
4138	0\\
4139	0\\
4140	0\\
4141	0\\
4142	0\\
4143	0\\
4144	0\\
4145	0\\
4146	0\\
4147	0\\
4148	1e-06\\
4149	2e-06\\
4150	0.764794\\
4151	0.323345\\
4152	0\\
4153	0\\
4154	0\\
4155	0\\
4156	0\\
4157	0\\
4158	0.331956\\
4159	1.959397\\
4160	1.959397\\
4161	1.959397\\
4162	1.937936\\
4163	1.744797\\
4164	1.551784\\
4165	1.437292\\
4166	1.115053\\
4167	1.115053\\
4168	1.693382\\
4169	1.959397\\
4170	1.959397\\
4171	1.959397\\
4172	1.959397\\
4173	1.115053\\
4174	1.959397\\
4175	1.959397\\
4176	0.764794\\
4177	0\\
4178	0\\
4179	0\\
4180	0\\
4181	0\\
4182	0.687973\\
4183	1.693382\\
4184	1.959397\\
4185	1.959397\\
4186	1.959397\\
4187	2.411319\\
4188	1.959397\\
4189	2.320135\\
4190	1.959397\\
4191	1.959397\\
4192	1.959397\\
4193	1.959397\\
4194	1.453721\\
4195	1.693382\\
4196	1.959397\\
4197	1.693382\\
4198	1.959397\\
4199	1.871152\\
4200	0.815156\\
4201	2e-06\\
4202	0\\
4203	0\\
4204	0\\
4205	0\\
4206	1e-06\\
4207	1.678017\\
4208	1.532155\\
4209	1.52172\\
4210	1.115053\\
4211	1.115053\\
4212	0.764794\\
4213	1.115053\\
4214	1.108899\\
4215	1.036932\\
4216	1.115053\\
4217	1.652166\\
4218	1.693382\\
4219	1.693382\\
4220	1.634843\\
4221	1.115053\\
4222	1.959397\\
4223	1.531936\\
4224	0.209002\\
4225	0\\
4226	0\\
4227	0\\
4228	0\\
4229	0\\
4230	0.783172\\
4231	1.453523\\
4232	1.693382\\
4233	1.959397\\
4234	1.959397\\
4235	1.830053\\
4236	1.693382\\
4237	1.472621\\
4238	1.22745\\
4239	1.243957\\
4240	1.693382\\
4241	1.959397\\
4242	1.959397\\
4243	1.288621\\
4244	1.115053\\
4245	1.115053\\
4246	1.959397\\
4247	1.693383\\
4248	0.75505\\
4249	0\\
4250	0\\
4251	0\\
4252	0\\
4253	0\\
4254	0.349567\\
4255	1.282711\\
4256	1.693382\\
4257	1.959397\\
4258	2.386478\\
4259	1.959397\\
4260	1.959397\\
4261	2.286503\\
4262	1.959397\\
4263	0.764794\\
4264	0.920796\\
4265	1.115053\\
4266	0.840136\\
4267	0.764794\\
4268	0.53073\\
4269	0.464068\\
4270	0.764794\\
4271	0.258006\\
4272	0\\
4273	0\\
4274	0\\
4275	0\\
4276	0\\
4277	0\\
4278	0\\
4279	0\\
4280	0\\
4281	0\\
4282	0\\
4283	0\\
4284	0\\
4285	0\\
4286	0\\
4287	0\\
4288	0\\
4289	0\\
4290	0.192817\\
4291	0\\
4292	0\\
4293	0\\
4294	0\\
4295	0\\
4296	0\\
4297	0\\
4298	0\\
4299	0\\
4300	0\\
4301	0\\
4302	0\\
4303	0\\
4304	0\\
4305	0\\
4306	0\\
4307	0\\
4308	0\\
4309	0\\
4310	0\\
4311	0\\
4312	0\\
4313	0\\
4314	0\\
4315	0\\
4316	0\\
4317	0\\
4318	0\\
4319	0\\
4320	0\\
4321	0\\
4322	0\\
4323	0\\
4324	0\\
4325	0\\
4326	0\\
4327	0.349565\\
4328	0.764794\\
4329	0.985842\\
4330	0.920781\\
4331	1.115053\\
4332	1.0289\\
4333	0.764794\\
4334	0.289984\\
4335	0.151122\\
4336	0.694309\\
4337	0.901202\\
4338	1.693382\\
4339	1.693382\\
4340	1.693382\\
4341	1.120958\\
4342	1.719329\\
4343	1.115053\\
4344	0.871858\\
4345	1e-06\\
4346	0\\
4347	0\\
4348	0\\
4349	0\\
4350	1.05579\\
4351	2.254992\\
4352	2.50001\\
4353	2.199245\\
4354	1.855258\\
4355	1.855258\\
4356	1.163586\\
4357	1.05612\\
4358	0.871858\\
4359	0.888148\\
4360	1.05579\\
4361	1.855258\\
4362	1.855258\\
4363	1.654423\\
4364	1.855258\\
4365	0\\
4366	0.330989\\
4367	1e-06\\
4368	0\\
4369	0\\
4370	0\\
4371	0\\
4372	0\\
4373	0\\
4374	0\\
4375	0.469224\\
4376	1.05579\\
4377	1.05579\\
4378	0.871855\\
4379	0.34343\\
4380	0\\
4381	0\\
4382	0\\
4383	0\\
4384	0\\
4385	0.330987\\
4386	1.05579\\
4387	1.05579\\
4388	1.731488\\
4389	1.056011\\
4390	1.716538\\
4391	0.47674\\
4392	0\\
4393	0\\
4394	0\\
4395	0\\
4396	0\\
4397	0\\
4398	0\\
4399	3e-06\\
4400	1.05579\\
4401	1.055789\\
4402	0.330986\\
4403	5e-06\\
4404	0\\
4405	0\\
4406	0\\
4407	0\\
4408	0\\
4409	1e-06\\
4410	0.170228\\
4411	0\\
4412	1e-06\\
4413	1e-06\\
4414	0.000109\\
4415	0\\
4416	0\\
4417	0\\
4418	0\\
4419	0\\
4420	0\\
4421	0\\
4422	0\\
4423	0.550942\\
4424	1.05579\\
4425	1.069568\\
4426	1.09967\\
4427	1.450549\\
4428	1.05579\\
4429	1.05579\\
4430	1.05579\\
4431	0.892253\\
4432	0.330987\\
4433	0.871858\\
4434	1.05579\\
4435	0.871857\\
4436	0.330989\\
4437	1e-06\\
4438	3.8e-05\\
4439	0\\
4440	0\\
4441	0\\
4442	0\\
4443	0\\
4444	0\\
4445	0\\
4446	0\\
4447	0\\
4448	0\\
4449	0\\
4450	0\\
4451	0\\
4452	0\\
4453	0\\
4454	0\\
4455	0\\
4456	0\\
4457	0\\
4458	0\\
4459	0\\
4460	0\\
4461	0\\
4462	0\\
4463	0\\
4464	0\\
4465	0\\
4466	0\\
4467	0\\
4468	0\\
4469	0\\
4470	0\\
4471	0\\
4472	0\\
4473	0\\
4474	0\\
4475	0\\
4476	0\\
4477	0\\
4478	0\\
4479	0\\
4480	0\\
4481	0\\
4482	0\\
4483	0\\
4484	0\\
4485	0\\
4486	0\\
4487	0\\
4488	0\\
4489	0\\
4490	0\\
4491	0\\
4492	0\\
4493	0\\
4494	0\\
4495	0\\
4496	1e-06\\
4497	1e-06\\
4498	1e-06\\
4499	1e-06\\
4500	0\\
4501	0\\
4502	1e-06\\
4503	0.043052\\
4504	0.871858\\
4505	0.348338\\
4506	1.05579\\
4507	1.016108\\
4508	1.303517\\
4509	1.05579\\
4510	1.739843\\
4511	0.777643\\
4512	0\\
4513	0\\
4514	0\\
4515	0\\
4516	0\\
4517	0\\
4518	0\\
4519	0.903378\\
4520	1.855258\\
4521	2.500648\\
4522	2.72168\\
4523	3.035339\\
4524	2.500648\\
4525	2.500648\\
4526	2.500648\\
4527	2.254992\\
4528	1.982721\\
4529	2.254992\\
4530	1.986799\\
4531	1.855258\\
4532	1.855258\\
4533	1.267088\\
4534	1.855258\\
4535	1.05579\\
4536	0\\
4537	0\\
4538	0\\
4539	0\\
4540	0\\
4541	0\\
4542	0\\
4543	0\\
4544	0.799357\\
4545	1.05579\\
4546	1.83179\\
4547	1.855258\\
4548	1.855258\\
4549	1.855258\\
4550	1.855258\\
4551	1.773967\\
4552	1.618517\\
4553	1.450514\\
4554	1.05579\\
4555	8e-06\\
4556	0\\
4557	0\\
4558	0\\
4559	0\\
4560	0\\
4561	0\\
4562	0\\
4563	0\\
4564	0\\
4565	0\\
4566	0\\
4567	0\\
4568	0.892154\\
4569	1.05579\\
4570	0.871858\\
4571	0.874419\\
4572	1.561745\\
4573	1.05579\\
4574	0.872628\\
4575	0.623119\\
4576	0.270931\\
4577	0.871857\\
4578	0.871858\\
4579	0.629801\\
4580	0\\
4581	0.070293\\
4582	0.617442\\
4583	0\\
4584	0\\
4585	0\\
4586	0\\
4587	0\\
4588	0\\
4589	0\\
4590	0\\
4591	4e-06\\
4592	1.1e-05\\
4593	0\\
4594	0\\
4595	1.84699\\
4596	1.633631\\
4597	1.055789\\
4598	0.330989\\
4599	3e-06\\
4600	0\\
4601	0.443463\\
4602	0.491984\\
4603	2e-06\\
4604	0.247621\\
4605	2e-05\\
4606	0.491956\\
4607	0.871858\\
4608	0\\
4609	0\\
4610	0\\
4611	0\\
4612	0\\
4613	0\\
4614	0\\
4615	0\\
4616	0\\
4617	0\\
4618	3e-06\\
4619	4e-06\\
4620	0\\
4621	0\\
4622	0\\
4623	0\\
4624	0\\
4625	0\\
4626	0\\
4627	1e-06\\
4628	0\\
4629	0\\
4630	0\\
4631	0\\
4632	0\\
4633	0\\
4634	0\\
4635	0\\
4636	0\\
4637	0\\
4638	0\\
4639	0\\
4640	0\\
4641	0\\
4642	0\\
4643	0\\
4644	0\\
4645	0\\
4646	0\\
4647	0\\
4648	0\\
4649	0\\
4650	0\\
4651	0\\
4652	0\\
4653	0\\
4654	0\\
4655	0\\
4656	0\\
4657	0\\
4658	0\\
4659	0\\
4660	0\\
4661	0\\
4662	0\\
4663	0\\
4664	0.331895\\
4665	2.1e-05\\
4666	1.05579\\
4667	1.181338\\
4668	1.240851\\
4669	0.837936\\
4670	0.26434\\
4671	0\\
4672	0.140584\\
4673	0.871857\\
4674	1.05579\\
4675	0.741083\\
4676	0.384091\\
4677	0.000132\\
4678	0.878884\\
4679	0.1587\\
4680	0\\
4681	0\\
4682	0\\
4683	0\\
4684	0\\
4685	0\\
4686	0\\
4687	0\\
4688	0\\
4689	8e-06\\
4690	1.05579\\
4691	1.737425\\
4692	1.855258\\
4693	1.607822\\
4694	1.317898\\
4695	1.494735\\
4696	1.421172\\
4697	1.855258\\
4698	1.855258\\
4699	1.05579\\
4700	1.05579\\
4701	1.055791\\
4702	1.855258\\
4703	1.855258\\
4704	1.05579\\
4705	0.01169\\
4706	1e-06\\
4707	0\\
4708	0\\
4709	0\\
4710	1.05579\\
4711	1.940049\\
4712	2.690348\\
4713	2.851895\\
4714	2.500648\\
4715	2.254992\\
4716	1.855258\\
4717	1.772686\\
4718	1.105199\\
4719	1.764582\\
4720	1.542092\\
4721	1.855258\\
4722	1.9243\\
4723	1.855258\\
4724	1.05579\\
4725	1.055797\\
4726	1.855258\\
4727	1.500168\\
4728	0.000203\\
4729	0\\
4730	0\\
4731	0\\
4732	0\\
4733	0\\
4734	1e-06\\
4735	1.05579\\
4736	1.855258\\
4737	1.855258\\
4738	1.493651\\
4739	1.110026\\
4740	1.05579\\
4741	0.666688\\
4742	0.392913\\
4743	0.871858\\
4744	1.05579\\
4745	1.732789\\
4746	1.855258\\
4747	1.732787\\
4748	1.015946\\
4749	1.055786\\
4750	1.520588\\
4751	0.330984\\
4752	0\\
4753	0\\
4754	0\\
4755	0\\
4756	0\\
4757	0\\
4758	0\\
4759	0\\
4760	0.871858\\
4761	0.594858\\
4762	0.384694\\
4763	1.05579\\
4764	0\\
4765	0\\
4766	0\\
4767	0\\
4768	0\\
4769	1e-06\\
4770	0.330987\\
4771	0.877881\\
4772	0.818447\\
4773	0.492437\\
4774	1.05579\\
4775	7e-06\\
4776	0\\
4777	0\\
4778	0\\
4779	0\\
4780	0\\
4781	0\\
4782	0\\
4783	0\\
4784	0\\
4785	0\\
4786	0\\
4787	0\\
4788	0\\
4789	0\\
4790	0\\
4791	0\\
4792	0\\
4793	0\\
4794	0\\
4795	0\\
4796	0\\
4797	0\\
4798	0\\
4799	0\\
4800	0\\
4801	0\\
4802	0\\
4803	0\\
4804	0\\
4805	0\\
4806	0\\
4807	0\\
4808	0\\
4809	0\\
4810	0\\
4811	0\\
4812	0\\
4813	0\\
4814	0\\
4815	0\\
4816	0\\
4817	0\\
4818	0\\
4819	0\\
4820	0\\
4821	0\\
4822	0\\
4823	0\\
4824	0\\
4825	0\\
4826	0\\
4827	0\\
4828	0\\
4829	0\\
4830	0\\
4831	0\\
4832	0\\
4833	0\\
4834	0\\
4835	0\\
4836	0\\
4837	0\\
4838	0\\
4839	0\\
4840	0\\
4841	0\\
4842	0\\
4843	0\\
4844	0\\
4845	0\\
4846	0\\
4847	0\\
4848	0\\
4849	0\\
4850	0\\
4851	0\\
4852	0\\
4853	0\\
4854	0\\
4855	0\\
4856	0\\
4857	0\\
4858	0\\
4859	0\\
4860	0\\
4861	0\\
4862	0\\
4863	0\\
4864	0\\
4865	0\\
4866	0\\
4867	0\\
4868	0\\
4869	0\\
4870	0\\
4871	0\\
4872	0\\
4873	0\\
4874	0\\
4875	0\\
4876	0\\
4877	0\\
4878	0\\
4879	0\\
4880	0\\
4881	0\\
4882	0\\
4883	0\\
4884	0\\
4885	0\\
4886	0\\
4887	0\\
4888	0\\
4889	0\\
4890	0\\
4891	0\\
4892	0\\
4893	0\\
4894	0\\
4895	0\\
4896	0\\
4897	0\\
4898	0\\
4899	0\\
4900	0\\
4901	0\\
4902	0\\
4903	0\\
4904	0\\
4905	0\\
4906	0\\
4907	0\\
4908	0\\
4909	0\\
4910	0\\
4911	0\\
4912	0\\
4913	0\\
4914	0\\
4915	0\\
4916	0\\
4917	0\\
4918	0\\
4919	0\\
4920	0\\
4921	0\\
4922	0\\
4923	0\\
4924	0\\
4925	0\\
4926	0\\
4927	0\\
4928	0\\
4929	0\\
4930	0\\
4931	0\\
4932	0\\
4933	0\\
4934	0\\
4935	0\\
4936	0\\
4937	0\\
4938	0\\
4939	0\\
4940	0\\
4941	0\\
4942	0\\
4943	0\\
4944	0\\
4945	0\\
4946	0\\
4947	0\\
4948	0\\
4949	0\\
4950	0\\
4951	0\\
4952	0\\
4953	0\\
4954	0\\
4955	0\\
4956	0\\
4957	0\\
4958	0\\
4959	0\\
4960	0\\
4961	0\\
4962	0\\
4963	0\\
4964	0\\
4965	0\\
4966	0\\
4967	0\\
4968	0\\
4969	0\\
4970	0\\
4971	0\\
4972	0\\
4973	0\\
4974	0\\
4975	0\\
4976	0\\
4977	0\\
4978	0\\
4979	0\\
4980	0\\
4981	0\\
4982	0\\
4983	0\\
4984	0\\
4985	0\\
4986	0\\
4987	0\\
4988	0\\
4989	0\\
4990	0\\
4991	0\\
4992	0\\
4993	0\\
4994	0\\
4995	0\\
4996	0\\
4997	0\\
4998	0\\
4999	0\\
5000	0\\
5001	0\\
5002	0\\
5003	0\\
5004	0\\
5005	0\\
5006	0\\
5007	0\\
5008	0\\
5009	0\\
5010	0\\
5011	0\\
5012	0\\
5013	0\\
5014	0\\
5015	0\\
5016	0\\
5017	0\\
5018	0\\
5019	0\\
5020	0\\
5021	0\\
5022	0\\
5023	0\\
5024	0\\
5025	0\\
5026	0\\
5027	0\\
5028	0\\
5029	0\\
5030	0\\
5031	0\\
5032	0\\
5033	0\\
5034	0\\
5035	0\\
5036	0\\
5037	0\\
5038	0\\
5039	0\\
5040	0\\
5041	0\\
5042	0\\
5043	0\\
5044	0\\
5045	0\\
5046	0\\
5047	0\\
5048	0\\
5049	0\\
5050	0\\
5051	0\\
5052	0\\
5053	0\\
5054	0\\
5055	0\\
5056	0\\
5057	0\\
5058	0\\
5059	0\\
5060	0\\
5061	0\\
5062	0\\
5063	0\\
5064	0\\
5065	0\\
5066	0\\
5067	0\\
5068	0\\
5069	0\\
5070	0\\
5071	0\\
5072	0\\
5073	0\\
5074	0\\
5075	0\\
5076	0\\
5077	0\\
5078	0\\
5079	0\\
5080	0\\
5081	0\\
5082	0\\
5083	0\\
5084	0\\
5085	0\\
5086	0\\
5087	0\\
5088	0\\
5089	0\\
5090	0\\
5091	0\\
5092	0\\
5093	0\\
5094	0\\
5095	0\\
5096	0\\
5097	0\\
5098	0\\
5099	0\\
5100	0\\
5101	0\\
5102	0\\
5103	0\\
5104	0\\
5105	0\\
5106	0\\
5107	0\\
5108	0\\
5109	0\\
5110	0\\
5111	0\\
5112	0\\
5113	0\\
5114	0\\
5115	0\\
5116	0\\
5117	0\\
5118	0\\
5119	0\\
5120	0\\
5121	0\\
5122	0\\
5123	0\\
5124	0\\
5125	0\\
5126	0\\
5127	0\\
5128	0\\
5129	0\\
5130	0\\
5131	0\\
5132	0\\
5133	0\\
5134	0\\
5135	0\\
5136	0\\
5137	0\\
5138	0\\
5139	0\\
5140	0\\
5141	0\\
5142	0\\
5143	0\\
5144	0\\
5145	0\\
5146	0\\
5147	0\\
5148	0\\
5149	0\\
5150	0\\
5151	0\\
5152	0\\
5153	0\\
5154	0\\
5155	0\\
5156	0\\
5157	0\\
5158	0\\
5159	0\\
5160	0\\
5161	0\\
5162	0\\
5163	0\\
5164	0\\
5165	0\\
5166	0\\
5167	0\\
5168	0\\
5169	1e-06\\
5170	1e-06\\
5171	0\\
5172	0\\
5173	0\\
5174	0\\
5175	0\\
5176	0\\
5177	0\\
5178	0\\
5179	0\\
5180	0\\
5181	0\\
5182	0\\
5183	0\\
5184	0\\
5185	0\\
5186	0\\
5187	0\\
5188	0\\
5189	0\\
5190	0\\
5191	0\\
5192	0\\
5193	0\\
5194	0\\
5195	0\\
5196	0\\
5197	0\\
5198	0\\
5199	0\\
5200	0\\
5201	0.576113\\
5202	1.102671\\
5203	1.121495\\
5204	1.121495\\
5205	1.288501\\
5206	1.840625\\
5207	0.576113\\
5208	0\\
5209	0\\
5210	0\\
5211	0\\
5212	0\\
5213	0\\
5214	0\\
5215	0\\
5216	0.21691\\
5217	0.329982\\
5218	0.237978\\
5219	0.273179\\
5220	0\\
5221	0\\
5222	0\\
5223	0\\
5224	0.321821\\
5225	0.576113\\
5226	0.576113\\
5227	0.338927\\
5228	2.4e-05\\
5229	3e-06\\
5230	0.54237\\
5231	0\\
5232	0\\
5233	0\\
5234	0\\
5235	0\\
5236	0\\
5237	0\\
5238	0\\
5239	0.576113\\
5240	1.121495\\
5241	1.970716\\
5242	1.864093\\
5243	1.504702\\
5244	1.504702\\
5245	1.288531\\
5246	1.121495\\
5247	1.067415\\
5248	0.845356\\
5249	0.931816\\
5250	1.121495\\
5251	0.773259\\
5252	1.152338\\
5253	1.484478\\
5254	1.504702\\
5255	0.576113\\
5256	0\\
5257	0\\
5258	0\\
5259	0\\
5260	0\\
5261	0\\
5262	0\\
5263	1.064417\\
5264	1.593422\\
5265	1.970716\\
5266	1.504721\\
5267	1.504702\\
5268	1.401801\\
5269	1.504702\\
5270	1.121495\\
5271	1.504702\\
5272	1.504702\\
5273	1.248369\\
5274	1.121495\\
5275	1.121495\\
5276	0.576113\\
5277	0.288254\\
5278	0.082968\\
5279	0\\
5280	0\\
5281	0\\
5282	0\\
5283	0\\
5284	0\\
5285	0\\
5286	0\\
5287	0\\
5288	0\\
5289	0\\
5290	0\\
5291	0\\
5292	0\\
5293	0\\
5294	0\\
5295	0\\
5296	0\\
5297	0\\
5298	0\\
5299	0\\
5300	0\\
5301	0\\
5302	0\\
5303	0\\
5304	0\\
5305	0\\
5306	0\\
5307	0\\
5308	0\\
5309	0\\
5310	0\\
5311	0\\
5312	0\\
5313	0\\
5314	0\\
5315	0\\
5316	0\\
5317	0\\
5318	0\\
5319	0\\
5320	0\\
5321	0\\
5322	0\\
5323	0\\
5324	0\\
5325	0\\
5326	0\\
5327	0\\
5328	0\\
5329	0\\
5330	0\\
5331	0\\
5332	0\\
5333	0\\
5334	0\\
5335	0\\
5336	0\\
5337	0\\
5338	0\\
5339	0\\
5340	0\\
5341	0\\
5342	0\\
5343	0\\
5344	0\\
5345	0\\
5346	0\\
5347	0\\
5348	0\\
5349	0\\
5350	0\\
5351	0\\
5352	0\\
5353	0\\
5354	0\\
5355	0\\
5356	0\\
5357	0\\
5358	0\\
5359	0\\
5360	0\\
5361	0\\
5362	0\\
5363	0\\
5364	0\\
5365	0\\
5366	0\\
5367	0\\
5368	0\\
5369	0\\
5370	0\\
5371	0\\
5372	0\\
5373	0\\
5374	0.351585\\
5375	0\\
5376	0\\
5377	0\\
5378	0\\
5379	0\\
5380	0\\
5381	0\\
5382	0\\
5383	2e-06\\
5384	1.121139\\
5385	1.121495\\
5386	1.121495\\
5387	1.268879\\
5388	0.903267\\
5389	0.576113\\
5390	0.351586\\
5391	0.226428\\
5392	0.139887\\
5393	0.351565\\
5394	0\\
5395	0.576113\\
5396	0.576113\\
5397	0.576113\\
5398	0.926116\\
5399	0\\
5400	0\\
5401	0\\
5402	0\\
5403	0\\
5404	0\\
5405	0\\
5406	0\\
5407	1.121495\\
5408	1.970716\\
5409	1.970716\\
5410	2.087288\\
5411	2.110298\\
5412	1.970716\\
5413	1.970716\\
5414	1.530774\\
5415	1.504702\\
5416	1.588953\\
5417	1.606825\\
5418	1.970716\\
5419	1.743497\\
5420	1.504702\\
5421	1.504732\\
5422	1.504637\\
5423	0.926115\\
5424	0\\
5425	0\\
5426	0\\
5427	0\\
5428	0\\
5429	0\\
5430	0\\
5431	0\\
5432	0\\
5433	0\\
5434	0\\
5435	0\\
5436	0\\
5437	0\\
5438	0\\
5439	0\\
5440	0\\
5441	0\\
5442	0\\
5443	0\\
5444	0\\
5445	0\\
5446	0\\
5447	0\\
5448	0\\
5449	0\\
5450	0\\
5451	0\\
5452	0\\
5453	0\\
5454	0\\
5455	0\\
5456	0\\
5457	0\\
5458	0\\
5459	0\\
5460	0\\
5461	0\\
5462	0\\
5463	0\\
5464	0\\
5465	0\\
5466	0\\
5467	0\\
5468	0\\
5469	0\\
5470	0\\
5471	0\\
5472	0\\
5473	0\\
5474	0\\
5475	0\\
5476	0\\
5477	0\\
5478	0\\
5479	0\\
5480	0\\
5481	0\\
5482	0\\
5483	0\\
5484	0\\
5485	0\\
5486	0\\
5487	0\\
5488	0\\
5489	0\\
5490	0\\
5491	0\\
5492	0\\
5493	0\\
5494	0\\
5495	0\\
5496	0\\
5497	0\\
5498	0\\
5499	0\\
5500	0\\
5501	0\\
5502	0\\
5503	0\\
5504	4e-06\\
5505	0\\
5506	0\\
5507	0\\
5508	0\\
5509	0\\
5510	0\\
5511	1.2e-05\\
5512	0.356395\\
5513	1.121495\\
5514	1.504702\\
5515	1.970716\\
5516	1.970716\\
5517	1.970716\\
5518	1.970716\\
5519	1.504702\\
5520	0.223969\\
5521	0\\
5522	0\\
5523	0\\
5524	0\\
5525	0\\
5526	0.4144\\
5527	1.121495\\
5528	1.970716\\
5529	1.970716\\
5530	1.970716\\
5531	1.840625\\
5532	1.338348\\
5533	0.576113\\
5534	0.899436\\
5535	0.576113\\
5536	0.924883\\
5537	1.119321\\
5538	1.970716\\
5539	1.970716\\
5540	1.970716\\
5541	1.970716\\
5542	1.970716\\
5543	1.504702\\
5544	0.576113\\
5545	0\\
5546	0\\
5547	0\\
5548	0\\
5549	0\\
5550	1.504702\\
5551	2.08137\\
5552	2.135591\\
5553	2.31866\\
5554	1.970716\\
5555	1.970716\\
5556	1.721619\\
5557	1.634908\\
5558	1.455506\\
5559	1.802038\\
5560	1.970716\\
5561	2.358365\\
5562	2.112947\\
5563	2.265378\\
5564	2.33206\\
5565	3.21295\\
5566	2.903523\\
5567	1.970716\\
5568	0.641762\\
5569	0\\
5570	0\\
5571	0\\
5572	0\\
5573	0\\
5574	1.504702\\
5575	2.041644\\
5576	2.460061\\
5577	1.970716\\
5578	1.840624\\
5579	1.769525\\
5580	1.970716\\
5581	1.504706\\
5582	1.970716\\
5583	1.970716\\
5584	1.970716\\
5585	1.844117\\
5586	1.970716\\
5587	1.970716\\
5588	1.970716\\
5589	1.508601\\
5590	1.504702\\
5591	1.12149\\
5592	0\\
5593	0\\
5594	0\\
5595	0\\
5596	0\\
5597	0\\
5598	1e-06\\
5599	0.795157\\
5600	1.970716\\
5601	1.504702\\
5602	1.504702\\
5603	1.5518\\
5604	1.241146\\
5605	1.755208\\
5606	1.504702\\
5607	1.504702\\
5608	1.504702\\
5609	1.504702\\
5610	1.504702\\
5611	1.970716\\
5612	1.970716\\
5613	1.970716\\
5614	1.976021\\
5615	1.803999\\
5616	0.3506\\
5617	0\\
5618	0.679275\\
5619	0.351585\\
5620	1e-06\\
5621	0\\
5622	0.17256\\
5623	0.576113\\
5624	0.991159\\
5625	0.576111\\
5626	0.574795\\
5627	0.000203\\
5628	1e-06\\
5629	0\\
5630	0\\
5631	0\\
5632	0\\
5633	0.351578\\
5634	1.121495\\
5635	1.583036\\
5636	1.970716\\
5637	1.970716\\
5638	1.840626\\
5639	1.504702\\
5640	0.576113\\
5641	0\\
5642	0\\
5643	0\\
5644	0\\
5645	0\\
5646	0\\
5647	0\\
5648	0\\
5649	0\\
5650	0\\
5651	0\\
5652	0\\
5653	0\\
5654	0\\
5655	0\\
5656	0\\
5657	0\\
5658	0\\
5659	0\\
5660	0.576113\\
5661	1.240356\\
5662	1.047399\\
5663	0.518683\\
5664	0\\
5665	0\\
5666	0\\
5667	0\\
5668	0\\
5669	0\\
5670	0.870047\\
5671	1.674739\\
5672	2.110481\\
5673	2.758742\\
5674	3.218433\\
5675	3.224237\\
5676	3.42155\\
5677	3.42155\\
5678	3.42155\\
5679	3.782072\\
5680	4.173354\\
5681	3.630294\\
5682	3.42155\\
5683	3.218433\\
5684	2.510508\\
5685	1.970716\\
5686	1.970716\\
5687	1.073973\\
5688	0\\
5689	0\\
5690	0\\
5691	0\\
5692	0\\
5693	0\\
5694	1.540439\\
5695	2.489969\\
5696	3.329814\\
5697	4.794764\\
5698	5.378813\\
5699	6.014299\\
5700	4.907937\\
5701	4.457156\\
5702	3.810988\\
5703	4.717592\\
5704	4.576995\\
5705	4.710875\\
5706	3.577708\\
5707	3.42155\\
5708	3.224237\\
5709	2.944523\\
5710	2.294953\\
5711	1.970716\\
5712	0.199994\\
5713	0\\
5714	0\\
5715	0\\
5716	0\\
5717	0\\
5718	1.647015\\
5719	1.972288\\
5720	3.315514\\
5721	3.224237\\
5722	2.838511\\
5723	2.223253\\
5724	1.970716\\
5725	1.970716\\
5726	1.970716\\
5727	1.970716\\
5728	2.11048\\
5729	2.351238\\
5730	3.224237\\
5731	3.224237\\
5732	2.857782\\
5733	3.193821\\
5734	3.141166\\
5735	1.970716\\
5736	0.2051\\
5737	0\\
5738	0\\
5739	0\\
5740	0\\
5741	0\\
5742	0.926116\\
5743	1.970716\\
5744	3.122465\\
5745	3.224237\\
5746	3.316643\\
5747	3.279485\\
5748	3.083978\\
5749	2.912116\\
5750	2.582703\\
5751	3.218433\\
5752	2.708482\\
5753	2.282772\\
5754	2.886469\\
5755	2.395327\\
5756	2.170174\\
5757	1.970716\\
5758	1.970716\\
5759	1.121495\\
5760	0\\
5761	0\\
5762	0\\
5763	0\\
5764	0\\
5765	0\\
5766	0.926116\\
5767	1.970716\\
5768	2.11048\\
5769	1.970716\\
5770	1.970716\\
5771	1.540775\\
5772	1.173639\\
5773	0.685383\\
5774	0.576113\\
5775	1e-06\\
5776	1e-06\\
5777	0.478262\\
5778	1.10258\\
5779	1.073375\\
5780	1.121495\\
5781	0.601897\\
5782	0.576114\\
5783	0\\
5784	0\\
5785	0\\
5786	0\\
5787	0\\
5788	0\\
5789	0\\
5790	0\\
5791	0\\
5792	0.260711\\
5793	1.121495\\
5794	1.970716\\
5795	1.970716\\
5796	1.970716\\
5797	1.970716\\
5798	1.790777\\
5799	1.775705\\
5800	1.961007\\
5801	1.970716\\
5802	2.289263\\
5803	2.395328\\
5804	2.11048\\
5805	1.970716\\
5806	1.970716\\
5807	1.504702\\
5808	0.576113\\
5809	0\\
5810	0\\
5811	0\\
5812	0\\
5813	0\\
5814	0\\
5815	0\\
5816	0.286027\\
5817	1.121495\\
5818	1.504702\\
5819	1.865336\\
5820	1.504702\\
5821	0.811116\\
5822	1e-06\\
5823	0\\
5824	0\\
5825	0.356876\\
5826	1.504126\\
5827	2.06555\\
5828	1.971063\\
5829	2.016098\\
5830	2.32324\\
5831	1.970716\\
5832	0\\
5833	0\\
5834	0\\
5835	0\\
5836	0\\
5837	0\\
5838	1.093288\\
5839	2.347652\\
5840	2.347652\\
5841	2.347652\\
5842	2.218066\\
5843	1.835477\\
5844	1.500101\\
5845	1.511326\\
5846	1.835483\\
5847	2.151498\\
5848	2.347652\\
5849	2.347652\\
5850	2.347652\\
5851	2.853476\\
5852	2.347652\\
5853	2.347652\\
5854	2.347652\\
5855	1.269953\\
5856	0\\
5857	0\\
5858	0\\
5859	0\\
5860	0\\
5861	0\\
5862	1.336001\\
5863	2.780283\\
5864	2.853477\\
5865	2.514901\\
5866	2.347652\\
5867	2.347652\\
5868	1.818247\\
5869	1.575055\\
5870	1.734837\\
5871	1.539401\\
5872	2.1878\\
5873	2.347652\\
5874	2.347652\\
5875	2.347652\\
5876	2.347652\\
5877	2.347652\\
5878	2.347652\\
5879	1.336001\\
5880	0\\
5881	0\\
5882	0\\
5883	0\\
5884	0\\
5885	0\\
5886	1.336001\\
5887	2.853477\\
5888	2.857254\\
5889	2.514147\\
5890	2.347652\\
5891	2.088423\\
5892	1.111782\\
5893	0.42498\\
5894	0.04156\\
5895	0.187913\\
5896	1.270523\\
5897	2.192678\\
5898	2.347652\\
5899	2.347652\\
5900	2.347652\\
5901	2.347652\\
5902	2.347652\\
5903	6e-06\\
5904	0\\
5905	0\\
5906	0\\
5907	0\\
5908	0\\
5909	0\\
5910	1.336001\\
5911	2.870813\\
5912	2.961918\\
5913	2.853477\\
5914	2.347652\\
5915	2.347652\\
5916	1.845418\\
5917	1.48264\\
5918	1.336001\\
5919	1.581171\\
5920	2.347652\\
5921	2.347652\\
5922	3.650284\\
5923	3.642845\\
5924	3.840931\\
5925	3.297378\\
5926	2.754128\\
5927	1.649637\\
5928	0.001551\\
5929	0\\
5930	0\\
5931	0\\
5932	0\\
5933	0\\
5934	1.607934\\
5935	3.423292\\
5936	4.070265\\
5937	4.075984\\
5938	3.840931\\
5939	3.72509\\
5940	2.347652\\
5941	2.347652\\
5942	2.347652\\
5943	2.347652\\
5944	2.347652\\
5945	2.347652\\
5946	2.514155\\
5947	2.347652\\
5948	2.440048\\
5949	2.347652\\
5950	2.19268\\
5951	2.190864\\
5952	2.347652\\
5953	0.604296\\
5954	0\\
5955	0\\
5956	0\\
5957	0\\
5958	1e-06\\
5959	0.418423\\
5960	1.529734\\
5961	2.347652\\
5962	2.853477\\
5963	3.500055\\
5964	2.990014\\
5965	2.347652\\
5966	2.347652\\
5967	2.347652\\
5968	2.347652\\
5969	2.876297\\
5970	2.779736\\
5971	2.812629\\
5972	3.46118\\
5973	2.853477\\
5974	2.815104\\
5975	2.347652\\
5976	2.192678\\
5977	1.336001\\
5978	1.207896\\
5979	1.009029\\
5980	0.69702\\
5981	0.945863\\
5982	1.336792\\
5983	1.336001\\
5984	2.050214\\
5985	2.347651\\
5986	2.347652\\
5987	2.347652\\
5988	2.282586\\
5989	1.336001\\
5990	0.713021\\
5991	1.036139\\
5992	1e-06\\
5993	0\\
5994	1.051577\\
5995	1.795148\\
5996	2.347652\\
5997	2.347652\\
5998	2.192678\\
5999	0.521174\\
6000	0\\
6001	0\\
6002	0\\
6003	0\\
6004	0\\
6005	0\\
6006	2.347652\\
6007	4.075984\\
6008	4.264819\\
6009	4.075984\\
6010	4.075985\\
6011	4.075985\\
6012	3.31811\\
6013	2.507155\\
6014	2.347652\\
6015	2.347652\\
6016	2.779735\\
6017	3.840931\\
6018	3.840949\\
6019	3.840931\\
6020	3.464111\\
6021	2.869841\\
6022	2.347652\\
6023	2.34765\\
6024	0.418833\\
6025	1e-06\\
6026	0\\
6027	0\\
6028	0\\
6029	0\\
6030	2.347652\\
6031	3.783624\\
6032	3.764547\\
6033	2.461598\\
6034	2.853481\\
6035	3.357069\\
6036	2.347652\\
6037	2.347652\\
6038	2.347652\\
6039	2.347652\\
6040	2.796483\\
6041	3.834018\\
6042	3.841771\\
6043	3.039129\\
6044	3.525386\\
6045	2.806827\\
6046	2.514149\\
6047	1.597738\\
6048	1e-06\\
6049	1e-06\\
6050	0\\
6051	0\\
6052	0\\
6053	0\\
6054	2.347652\\
6055	3.840931\\
6056	3.968623\\
6057	3.240118\\
6058	2.679164\\
6059	2.347652\\
6060	2.347652\\
6061	2.347651\\
6062	2.026326\\
6063	2.347652\\
6064	2.347652\\
6065	2.347652\\
6066	3.54114\\
6067	2.779725\\
6068	2.879223\\
6069	3.248277\\
6070	2.347652\\
6071	1.396789\\
6072	0\\
6073	0\\
6074	0\\
6075	0\\
6076	0\\
6077	0\\
6078	2.347652\\
6079	3.586268\\
6080	3.758472\\
6081	3.840931\\
6082	3.82198\\
6083	3.840931\\
6084	2.8532\\
6085	2.347652\\
6086	2.347652\\
6087	2.347652\\
6088	2.347652\\
6089	2.941226\\
6090	3.212967\\
6091	2.747879\\
6092	2.811126\\
6093	2.779734\\
6094	2.347652\\
6095	1.336001\\
6096	0\\
6097	0\\
6098	0\\
6099	0\\
6100	0\\
6101	0\\
6102	2.347652\\
6103	2.85111\\
6104	3.688322\\
6105	2.851399\\
6106	2.347652\\
6107	1.570017\\
6108	1.103261\\
6109	0.696759\\
6110	0.512119\\
6111	0.171593\\
6112	1.10325\\
6113	2.030615\\
6114	2.347652\\
6115	2.347652\\
6116	2.853477\\
6117	2.347652\\
6118	2.347652\\
6119	1.336001\\
6120	0\\
6121	0\\
6122	0\\
6123	0\\
6124	0\\
6125	0\\
6126	0\\
6127	0\\
6128	0.300914\\
6129	0.764642\\
6130	0.373367\\
6131	1e-06\\
6132	0\\
6133	0\\
6134	0\\
6135	0\\
6136	0\\
6137	0\\
6138	0\\
6139	2e-06\\
6140	0\\
6141	0\\
6142	0\\
6143	0\\
6144	0\\
6145	0\\
6146	0\\
6147	0\\
6148	0\\
6149	0\\
6150	0\\
6151	0\\
6152	0\\
6153	1e-06\\
6154	0.322105\\
6155	0.599919\\
6156	4e-06\\
6157	0\\
6158	0\\
6159	0\\
6160	0\\
6161	0\\
6162	0\\
6163	0\\
6164	0.5978\\
6165	1.4e-05\\
6166	0\\
6167	0\\
6168	0\\
6169	0\\
6170	0\\
6171	0\\
6172	0\\
6173	0\\
6174	1.336001\\
6175	2.853477\\
6176	3.625754\\
6177	3.834016\\
6178	3.720456\\
6179	3.203208\\
6180	2.347652\\
6181	2.347652\\
6182	2.347652\\
6183	2.347652\\
6184	2.475178\\
6185	2.985426\\
6186	3.840931\\
6187	3.840931\\
6188	3.840931\\
6189	2.990591\\
6190	2.347652\\
6191	1.380351\\
6192	0\\
6193	0\\
6194	0\\
6195	0\\
6196	0\\
6197	0\\
6198	2.347652\\
6199	3.840931\\
6200	4.075984\\
6201	3.896063\\
6202	3.204403\\
6203	2.967015\\
6204	2.347652\\
6205	2.347652\\
6206	2.347652\\
6207	2.347652\\
6208	2.779751\\
6209	3.834018\\
6210	3.935603\\
6211	3.840931\\
6212	4.075984\\
6213	3.814817\\
6214	2.779046\\
6215	2.192701\\
6216	0\\
6217	0\\
6218	0\\
6219	0\\
6220	0\\
6221	0\\
6222	2.347652\\
6223	3.663003\\
6224	3.634459\\
6225	2.852694\\
6226	2.759623\\
6227	2.623019\\
6228	2.347652\\
6229	1.836612\\
6230	1.773363\\
6231	2.267987\\
6232	2.347652\\
6233	2.347652\\
6234	3.32301\\
6235	3.390913\\
6236	3.83699\\
6237	2.783404\\
6238	2.347652\\
6239	1.297735\\
6240	0\\
6241	0\\
6242	0\\
6243	0\\
6244	0\\
6245	0\\
6246	2.192678\\
6247	3.81392\\
6248	4.075984\\
6249	4.78775\\
6250	5.024695\\
6251	6.415296\\
6252	4.281886\\
6253	4.075985\\
6254	4.075984\\
6255	3.840931\\
6256	3.840931\\
6257	4.471077\\
6258	4.191207\\
6259	3.961086\\
6260	4.075984\\
6261	3.387628\\
6262	2.347652\\
6263	1.736196\\
6264	0.185365\\
6265	0\\
6266	0\\
6267	0\\
6268	0\\
6269	0\\
6270	2.347652\\
6271	4.050658\\
6272	4.108465\\
6273	4.74351\\
6274	4.191609\\
6275	4.075984\\
6276	3.840931\\
6277	3.435163\\
6278	3.820696\\
6279	3.530669\\
6280	3.759254\\
6281	3.840931\\
6282	3.840931\\
6283	2.920084\\
6284	3.834018\\
6285	2.593533\\
6286	2.347652\\
6287	2.192678\\
6288	7.5e-05\\
6289	0\\
6290	0\\
6291	0\\
6292	0\\
6293	0\\
6294	0\\
6295	0\\
6296	1.1e-05\\
6297	1.336001\\
6298	1.103252\\
6299	0.737804\\
6300	0.359284\\
6301	0.228466\\
6302	1e-06\\
6303	0\\
6304	0\\
6305	0.418833\\
6306	1.329695\\
6307	1.145665\\
6308	1.835489\\
6309	0.596203\\
6310	0.397927\\
6311	0\\
6312	0\\
6313	0\\
6314	0\\
6315	0\\
6316	0\\
6317	0\\
6318	0\\
6319	0\\
6320	0\\
6321	0\\
6322	0\\
6323	0\\
6324	0\\
6325	0\\
6326	0\\
6327	0\\
6328	0\\
6329	0\\
6330	0\\
6331	0\\
6332	0\\
6333	0\\
6334	0\\
6335	0\\
6336	0\\
6337	0\\
6338	0\\
6339	0\\
6340	0\\
6341	0\\
6342	1.006372\\
6343	2.347652\\
6344	2.882589\\
6345	3.518134\\
6346	2.990552\\
6347	2.853477\\
6348	2.347652\\
6349	2.347652\\
6350	2.347652\\
6351	2.347652\\
6352	2.347652\\
6353	2.347652\\
6354	3.054119\\
6355	3.261092\\
6356	3.497364\\
6357	2.853477\\
6358	2.347652\\
6359	1.336001\\
6360	0\\
6361	0\\
6362	0\\
6363	0\\
6364	0\\
6365	0\\
6366	2.347652\\
6367	4.075984\\
6368	4.075984\\
6369	3.840931\\
6370	3.730336\\
6371	2.95134\\
6372	2.347652\\
6373	2.347652\\
6374	2.347652\\
6375	2.347652\\
6376	2.81912\\
6377	3.812974\\
6378	3.840931\\
6379	4.075985\\
6380	4.414097\\
6381	3.585608\\
6382	2.665177\\
6383	1.909708\\
6384	1e-06\\
6385	0\\
6386	0\\
6387	0\\
6388	0\\
6389	0\\
6390	1.336001\\
6391	2.853477\\
6392	3.13137\\
6393	3.301381\\
6394	2.853478\\
6395	3.834008\\
6396	3.124758\\
6397	3.083853\\
6398	2.853382\\
6399	2.779735\\
6400	2.702713\\
6401	3.467363\\
6402	2.853477\\
6403	2.778957\\
6404	2.930106\\
6405	2.347652\\
6406	2.202159\\
6407	0.447718\\
6408	0\\
6409	0\\
6410	0\\
6411	0\\
6412	0\\
6413	0\\
6414	2.199097\\
6415	3.840931\\
6416	3.840931\\
6417	3.840931\\
6418	3.325157\\
6419	2.853477\\
6420	2.347652\\
6421	2.034438\\
6422	1.485203\\
6423	1.425254\\
6424	1.835759\\
6425	2.347652\\
6426	2.514084\\
6427	2.776798\\
6428	3.128353\\
6429	2.347652\\
6430	1.336001\\
6431	0.282478\\
6432	0\\
6433	0\\
6434	0\\
6435	0\\
6436	0\\
6437	0\\
6438	1.934367\\
6439	3.834018\\
6440	3.834018\\
6441	3.840931\\
6442	3.669765\\
6443	3.758773\\
6444	3.791064\\
6445	3.503453\\
6446	3.834018\\
6447	3.840931\\
6448	4.075984\\
6449	4.711993\\
6450	4.824363\\
6451	5.012128\\
6452	4.075985\\
6453	3.517061\\
6454	3.323351\\
6455	2.347652\\
6456	1.336001\\
6457	1.336001\\
6458	0.418833\\
6459	1e-05\\
6460	0\\
6461	1e-06\\
6462	0.840224\\
6463	2.04081\\
6464	2.347652\\
6465	2.853574\\
6466	3.468036\\
6467	2.993511\\
6468	2.347652\\
6469	2.251695\\
6470	1.359728\\
6471	1.336001\\
6472	1.763095\\
6473	2.347652\\
6474	2.725439\\
6475	2.990595\\
6476	3.598105\\
6477	2.853477\\
6478	2.347652\\
6479	1.919483\\
6480	0.26237\\
6481	0\\
6482	0\\
6483	0\\
6484	0\\
6485	0\\
6486	0\\
6487	0\\
6488	0\\
6489	0\\
6490	0\\
6491	0\\
6492	0\\
6493	0\\
6494	0\\
6495	0\\
6496	0\\
6497	0\\
6498	1.404158\\
6499	3.131715\\
6500	3.834017\\
6501	2.347652\\
6502	1.336001\\
6503	6e-06\\
6504	0\\
6505	0\\
6506	0\\
6507	0\\
6508	0\\
6509	0\\
6510	2.108922\\
6511	3.833974\\
6512	3.840931\\
6513	3.697261\\
6514	3.068082\\
6515	3.021689\\
6516	2.347652\\
6517	2.347652\\
6518	2.514148\\
6519	3.949182\\
6520	4.354794\\
6521	6.723788\\
6522	7.539637\\
6523	6.803105\\
6524	5.6132\\
6525	4.102467\\
6526	3.628162\\
6527	2.347652\\
6528	1.014053\\
6529	0\\
6530	0\\
6531	0\\
6532	0\\
6533	0\\
6534	3.82241\\
6535	6.451505\\
6536	7.21299\\
6537	7.509032\\
6538	6.315264\\
6539	5.344655\\
6540	4.595278\\
6541	4.47604\\
6542	3.84093\\
6543	3.840931\\
6544	4.012381\\
6545	5.468312\\
6546	6.409675\\
6547	7.031753\\
6548	6.820594\\
6549	5.186665\\
6550	3.846923\\
6551	2.477037\\
6552	0.334971\\
6553	0\\
6554	0\\
6555	0\\
6556	0\\
6557	0\\
6558	2.417835\\
6559	4.197836\\
6560	3.948638\\
6561	3.642268\\
6562	2.715909\\
6563	2.938782\\
6564	2.417835\\
6565	2.417835\\
6566	2.417835\\
6567	2.417835\\
6568	2.417871\\
6569	3.949671\\
6570	4.197836\\
6571	4.154065\\
6572	3.955756\\
6573	3.955756\\
6574	3.721476\\
6575	2.417835\\
6576	1.130122\\
6577	1e-06\\
6578	0\\
6579	0\\
6580	0\\
6581	0.340997\\
6582	2.862731\\
6583	5.360661\\
6584	5.621627\\
6585	5.687439\\
6586	5.168222\\
6587	5.236255\\
6588	4.37108\\
6589	4.197836\\
6590	4.382036\\
6591	4.847044\\
6592	5.154687\\
6593	5.694163\\
6594	5.497961\\
6595	5.307567\\
6596	5.061574\\
6597	4.209098\\
6598	3.948635\\
6599	2.417835\\
6600	1.172254\\
6601	2e-06\\
6602	0\\
6603	0\\
6604	0\\
6605	0.431353\\
6606	2.709334\\
6607	5.887218\\
6608	5.095533\\
6609	4.198951\\
6610	3.948635\\
6611	3.955756\\
6612	2.417835\\
6613	2.417835\\
6614	2.417835\\
6615	2.417835\\
6616	2.851418\\
6617	3.948635\\
6618	3.955756\\
6619	3.955756\\
6620	4.197836\\
6621	3.948635\\
6622	2.899575\\
6623	2.417835\\
6624	0.041922\\
6625	0\\
6626	1e-06\\
6627	0\\
6628	0\\
6629	0\\
6630	0\\
6631	0\\
6632	0\\
6633	0\\
6634	0\\
6635	0\\
6636	0\\
6637	1e-06\\
6638	0\\
6639	0\\
6640	0\\
6641	0\\
6642	0.350915\\
6643	6.6e-05\\
6644	0.431264\\
6645	0\\
6646	0\\
6647	0\\
6648	0\\
6649	0\\
6650	0\\
6651	0\\
6652	0\\
6653	0\\
6654	0\\
6655	0\\
6656	0\\
6657	0\\
6658	0.000436\\
6659	0.657602\\
6660	0.626552\\
6661	1.4e-05\\
6662	0\\
6663	0\\
6664	0\\
6665	1.181627\\
6666	2.417835\\
6667	2.417835\\
6668	2.417835\\
6669	2.417835\\
6670	1.890365\\
6671	0.653572\\
6672	0\\
6673	0\\
6674	0\\
6675	0\\
6676	0\\
6677	0\\
6678	2.417835\\
6679	4.439847\\
6680	4.197836\\
6681	4.36155\\
6682	3.955756\\
6683	3.945145\\
6684	2.938782\\
6685	3.807269\\
6686	3.691957\\
6687	2.589308\\
6688	2.938782\\
6689	3.750082\\
6690	3.726836\\
6691	2.417835\\
6692	2.417835\\
6693	1.37594\\
6694	0.36489\\
6695	0\\
6696	0\\
6697	0\\
6698	0\\
6699	0\\
6700	0\\
6701	0\\
6702	0\\
6703	2.417835\\
6704	2.417835\\
6705	1.890261\\
6706	1.344956\\
6707	1.375941\\
6708	0\\
6709	2e-06\\
6710	0\\
6711	0\\
6712	0\\
6713	1.132416\\
6714	2.417835\\
6715	2.417835\\
6716	2.417835\\
6717	1.673204\\
6718	0.851022\\
6719	0\\
6720	0\\
6721	0\\
6722	0\\
6723	0\\
6724	0\\
6725	0\\
6726	0.799463\\
6727	2.417835\\
6728	2.417835\\
6729	3.398687\\
6730	3.146186\\
6731	3.672485\\
6732	3.94862\\
6733	3.028304\\
6734	2.417835\\
6735	2.417835\\
6736	2.043374\\
6737	2.417835\\
6738	2.417835\\
6739	2.417835\\
6740	2.417835\\
6741	1.375941\\
6742	1.136231\\
6743	0\\
6744	0\\
6745	0\\
6746	0\\
6747	0\\
6748	0\\
6749	0\\
6750	0.431354\\
6751	2.417835\\
6752	2.938782\\
6753	2.417835\\
6754	2.417835\\
6755	2.417835\\
6756	0.298111\\
6757	0.431354\\
6758	1.293149\\
6759	1.376917\\
6760	1.790937\\
6761	2.417835\\
6762	2.417835\\
6763	2.417835\\
6764	2.417835\\
6765	1.712004\\
6766	0.789145\\
6767	0\\
6768	0\\
6769	0\\
6770	0\\
6771	0\\
6772	0\\
6773	0\\
6774	1.042664\\
6775	3.142196\\
6776	2.589991\\
6777	2.417835\\
6778	2.417835\\
6779	2.794866\\
6780	2.230703\\
6781	1.932243\\
6782	1.988603\\
6783	1.584667\\
6784	2.226994\\
6785	2.417835\\
6786	3.078947\\
6787	3.85876\\
6788	3.955756\\
6789	2.814228\\
6790	2.417835\\
6791	2.417835\\
6792	1e-05\\
6793	0\\
6794	0\\
6795	0\\
6796	0\\
6797	0\\
6798	0\\
6799	1e-06\\
6800	1.136234\\
6801	1.375941\\
6802	1.586852\\
6803	1.375941\\
6804	1.108873\\
6805	0.647742\\
6806	0\\
6807	0\\
6808	1.3e-05\\
6809	0.687824\\
6810	1.165696\\
6811	2.177137\\
6812	1.375941\\
6813	0.73758\\
6814	0\\
6815	0\\
6816	0\\
6817	0\\
6818	0\\
6819	0\\
6820	0\\
6821	0\\
6822	0\\
6823	0\\
6824	0\\
6825	0\\
6826	0\\
6827	0\\
6828	0\\
6829	0\\
6830	0\\
6831	0\\
6832	0\\
6833	0\\
6834	0\\
6835	1e-06\\
6836	0\\
6837	0\\
6838	0\\
6839	0\\
6840	0\\
6841	0\\
6842	0\\
6843	0\\
6844	0\\
6845	0\\
6846	0.656441\\
6847	2.417835\\
6848	2.377043\\
6849	1.375941\\
6850	1.375941\\
6851	1.815715\\
6852	1.375941\\
6853	1.375941\\
6854	1.375941\\
6855	1.375941\\
6856	1.581389\\
6857	2.417835\\
6858	2.417835\\
6859	2.589308\\
6860	2.417835\\
6861	1.890349\\
6862	1.478708\\
6863	5e-06\\
6864	0\\
6865	0\\
6866	0\\
6867	0\\
6868	0\\
6869	0\\
6870	0.431354\\
6871	2.938782\\
6872	2.417835\\
6873	2.417835\\
6874	1.386328\\
6875	1.49307\\
6876	0.431354\\
6877	1.229024\\
6878	1.375941\\
6879	1.752946\\
6880	2.417835\\
6881	2.938781\\
6882	3.948633\\
6883	4.105603\\
6884	3.948636\\
6885	2.686891\\
6886	2.417835\\
6887	1.481864\\
6888	1.6e-05\\
6889	0\\
6890	0\\
6891	0\\
6892	1e-06\\
6893	1.189537\\
6894	3.634479\\
6895	13.097841\\
6896	4.197836\\
6897	3.955756\\
6898	3.955756\\
6899	3.955756\\
6900	3.343505\\
6901	3.221702\\
6902	3.079992\\
6903	3.425842\\
6904	3.948636\\
6905	4.272857\\
6906	5.511238\\
6907	12.438647\\
6908	4.75256\\
6909	3.877659\\
6910	2.938782\\
6911	2.417835\\
6912	0\\
6913	0\\
6914	0\\
6915	0\\
6916	0\\
6917	0\\
6918	1.375941\\
6919	3.948636\\
6920	3.569374\\
6921	3.831578\\
6922	3.732647\\
6923	3.127489\\
6924	2.724266\\
6925	2.417835\\
6926	2.417835\\
6927	2.417835\\
6928	2.417835\\
6929	3.945684\\
6930	4.320704\\
6931	4.16835\\
6932	3.955756\\
6933	3.07999\\
6934	2.722452\\
6935	2.417835\\
6936	1e-06\\
6937	0\\
6938	0\\
6939	0\\
6940	0\\
6941	0\\
6942	2.359687\\
6943	4.053457\\
6944	4.197836\\
6945	3.955756\\
6946	3.948636\\
6947	2.915463\\
6948	2.417835\\
6949	2.417835\\
6950	2.32992\\
6951	2.417835\\
6952	2.417835\\
6953	2.862834\\
6954	3.955756\\
6955	3.955756\\
6956	3.397409\\
6957	2.417835\\
6958	2.257904\\
6959	0.951278\\
6960	0\\
6961	0\\
6962	0\\
6963	0\\
6964	0\\
6965	0\\
6966	0\\
6967	0\\
6968	0\\
6969	0\\
6970	0\\
6971	0\\
6972	0\\
6973	0\\
6974	0\\
6975	0\\
6976	0\\
6977	0\\
6978	1.653031\\
6979	1.375941\\
6980	0\\
6981	0\\
6982	0\\
6983	0\\
6984	0\\
6985	0\\
6986	0\\
6987	0\\
6988	0\\
6989	0\\
6990	0\\
6991	0\\
6992	0\\
6993	0\\
6994	0\\
6995	0\\
6996	0\\
6997	0\\
6998	0\\
6999	0\\
7000	0\\
7001	0\\
7002	0\\
7003	0\\
7004	0\\
7005	0\\
7006	0\\
7007	0\\
7008	0\\
7009	0\\
7010	0\\
7011	0\\
7012	0\\
7013	0\\
7014	0\\
7015	2.417835\\
7016	2.550169\\
7017	2.417839\\
7018	2.417835\\
7019	2.417835\\
7020	1.981046\\
7021	2.220141\\
7022	2.053777\\
7023	2.053834\\
7024	2.355521\\
7025	2.417835\\
7026	2.417835\\
7027	2.116012\\
7028	1.375941\\
7029	0\\
7030	0\\
7031	0\\
7032	0\\
7033	0\\
7034	0\\
7035	0\\
7036	0\\
7037	0\\
7038	0\\
7039	1.375941\\
7040	1.9594\\
7041	1.890027\\
7042	1.773251\\
7043	2.417835\\
7044	1.375941\\
7045	1.375941\\
7046	1.045865\\
7047	0.304419\\
7048	0.182246\\
7049	1.375941\\
7050	1.25697\\
7051	0.580131\\
7052	0\\
7053	0\\
7054	0\\
7055	0\\
7056	0\\
7057	0\\
7058	0\\
7059	0\\
7060	0\\
7061	0\\
7062	0\\
7063	1.375942\\
7064	1.37594\\
7065	0.622358\\
7066	3e-06\\
7067	2e-06\\
7068	0\\
7069	0\\
7070	0\\
7071	0\\
7072	0.000534\\
7073	1.136234\\
7074	2.456183\\
7075	2.417835\\
7076	2.417835\\
7077	1.375935\\
7078	0.045161\\
7079	0\\
7080	0\\
7081	0\\
7082	0\\
7083	0\\
7084	0\\
7085	0\\
7086	0\\
7087	2.045065\\
7088	1.655144\\
7089	1.375941\\
7090	1.136232\\
7091	1.261663\\
7092	0.207771\\
7093	0.533326\\
7094	0.142925\\
7095	0.950032\\
7096	1.067295\\
7097	1.375941\\
7098	1.220863\\
7099	1.37593\\
7100	0\\
7101	0\\
7102	0\\
7103	0\\
7104	0\\
7105	0\\
7106	0\\
7107	0\\
7108	0\\
7109	0\\
7110	0\\
7111	0\\
7112	5e-06\\
7113	1.136381\\
7114	1.136106\\
7115	1.344343\\
7116	1.494897\\
7117	1.078518\\
7118	0.018726\\
7119	0.155394\\
7120	0.318729\\
7121	0.655772\\
7122	2.022605\\
7123	0.26918\\
7124	0\\
7125	0\\
7126	0\\
7127	0\\
7128	0\\
7129	0\\
7130	0\\
7131	0\\
7132	0\\
7133	0\\
7134	0\\
7135	0\\
7136	3e-06\\
7137	0.431354\\
7138	0.431354\\
7139	0.959298\\
7140	1.136234\\
7141	0\\
7142	0\\
7143	0\\
7144	0\\
7145	0\\
7146	0.823702\\
7147	0\\
7148	0\\
7149	0\\
7150	0\\
7151	0\\
7152	0\\
7153	0\\
7154	0\\
7155	0\\
7156	0\\
7157	0\\
7158	0\\
7159	0\\
7160	0\\
7161	0\\
7162	0\\
7163	0\\
7164	0\\
7165	0\\
7166	0\\
7167	0\\
7168	0\\
7169	0\\
7170	0\\
7171	0\\
7172	0\\
7173	0\\
7174	0\\
7175	0\\
7176	0\\
7177	0\\
7178	0\\
7179	0\\
7180	0\\
7181	0\\
7182	0\\
7183	0\\
7184	0\\
7185	0\\
7186	0\\
7187	0\\
7188	0\\
7189	0\\
7190	0\\
7191	0\\
7192	0\\
7193	1e-06\\
7194	0.533937\\
7195	2.414969\\
7196	1.103051\\
7197	0.392781\\
7198	0\\
7199	0\\
7200	0\\
7201	0\\
7202	0\\
7203	0\\
7204	0\\
7205	0\\
7206	0\\
7207	0\\
7208	0\\
7209	0\\
7210	0\\
7211	0\\
7212	0\\
7213	0\\
7214	0\\
7215	0\\
7216	0\\
7217	0.051437\\
7218	1.464637\\
7219	2.310946\\
7220	1.375941\\
7221	0.003137\\
7222	0\\
7223	0\\
7224	0\\
7225	0\\
7226	0\\
7227	0\\
7228	0\\
7229	0\\
7230	0\\
7231	0\\
7232	0.114299\\
7233	1.52324\\
7234	1.690348\\
7235	2.417835\\
7236	2.417835\\
7237	2.417835\\
7238	2.417835\\
7239	2.417835\\
7240	2.417835\\
7241	2.589308\\
7242	2.93876\\
7243	3.004055\\
7244	2.417835\\
7245	1.883558\\
7246	0.819257\\
7247	0.131212\\
7248	0\\
7249	0\\
7250	0\\
7251	0\\
7252	0\\
7253	0\\
7254	0\\
7255	0\\
7256	0.470201\\
7257	1.375941\\
7258	1.975219\\
7259	1.375941\\
7260	1.375941\\
7261	0.521527\\
7262	0.657471\\
7263	0.431354\\
7264	0.596985\\
7265	0.982677\\
7266	2.258228\\
7267	2.938782\\
7268	2.417835\\
7269	9e-06\\
7270	0\\
7271	0\\
7272	0\\
7273	0\\
7274	0\\
7275	0\\
7276	0\\
7277	0\\
7278	0\\
7279	0\\
7280	0\\
7281	0\\
7282	3e-06\\
7283	0\\
7284	0\\
7285	0\\
7286	0\\
7287	0\\
7288	0\\
7289	0\\
7290	0\\
7291	0.033676\\
7292	0\\
7293	0\\
7294	0\\
7295	0\\
7296	0\\
7297	0\\
7298	0\\
7299	0\\
7300	0\\
7301	0\\
7302	0\\
7303	0\\
7304	0\\
7305	0\\
7306	0\\
7307	0\\
7308	0\\
7309	0\\
7310	0\\
7311	0\\
7312	0\\
7313	0\\
7314	0\\
7315	0\\
7316	0\\
7317	0\\
7318	0\\
7319	0\\
7320	0\\
7321	0\\
7322	0\\
7323	0\\
7324	0\\
7325	0\\
7326	0\\
7327	0\\
7328	0\\
7329	0\\
7330	0\\
7331	0\\
7332	0\\
7333	0\\
7334	0\\
7335	0\\
7336	0\\
7337	0\\
7338	0\\
7339	0\\
7340	0\\
7341	0\\
7342	0\\
7343	0\\
7344	0\\
7345	0\\
7346	0\\
7347	0\\
7348	0\\
7349	0\\
7350	0\\
7351	0\\
7352	0\\
7353	0\\
7354	0\\
7355	0\\
7356	1e-06\\
7357	0\\
7358	0\\
7359	0\\
7360	0\\
7361	0\\
7362	0\\
7363	0\\
7364	0\\
7365	0\\
7366	0\\
7367	0\\
7368	1e-06\\
7369	0\\
7370	0\\
7371	0\\
7372	0\\
7373	0\\
7374	0\\
7375	0\\
7376	0.174467\\
7377	1e-05\\
7378	1.151667\\
7379	2.009662\\
7380	2.32886\\
7381	2.098158\\
7382	2.294026\\
7383	2.385092\\
7384	2.553665\\
7385	2.194789\\
7386	3.103887\\
7387	3.129176\\
7388	2.553665\\
7389	2.553665\\
7390	0.600147\\
7391	3.1e-05\\
7392	1e-06\\
7393	1e-06\\
7394	0\\
7395	0\\
7396	0\\
7397	0\\
7398	0\\
7399	0.134011\\
7400	2.553665\\
7401	2.553665\\
7402	2.553665\\
7403	2.553665\\
7404	2.553665\\
7405	2.553665\\
7406	2.553665\\
7407	2.553665\\
7408	2.553665\\
7409	2.553665\\
7410	2.839109\\
7411	3.103878\\
7412	2.3849\\
7413	2.553665\\
7414	1.062297\\
7415	0.455586\\
7416	0\\
7417	0\\
7418	0\\
7419	0\\
7420	0\\
7421	0\\
7422	0\\
7423	0\\
7424	0.811636\\
7425	0.04488\\
7426	0.540229\\
7427	0.762662\\
7428	5e-06\\
7429	7e-06\\
7430	1e-06\\
7431	0\\
7432	0\\
7433	0\\
7434	2e-06\\
7435	0.455586\\
7436	0\\
7437	0\\
7438	0\\
7439	0\\
7440	0\\
7441	0\\
7442	0\\
7443	0\\
7444	0\\
7445	0\\
7446	0\\
7447	0\\
7448	0\\
7449	0\\
7450	0\\
7451	0\\
7452	0\\
7453	0\\
7454	0\\
7455	0\\
7456	0\\
7457	0\\
7458	0\\
7459	0\\
7460	0\\
7461	0\\
7462	0\\
7463	0\\
7464	0\\
7465	0\\
7466	0\\
7467	0\\
7468	0\\
7469	0\\
7470	0\\
7471	0\\
7472	0\\
7473	0\\
7474	0\\
7475	0\\
7476	0\\
7477	0\\
7478	0\\
7479	0\\
7480	0\\
7481	0\\
7482	0\\
7483	0\\
7484	0\\
7485	0\\
7486	0\\
7487	0\\
7488	0\\
7489	0\\
7490	0\\
7491	0\\
7492	0\\
7493	0\\
7494	0\\
7495	0\\
7496	0\\
7497	0\\
7498	0\\
7499	0\\
7500	0\\
7501	0\\
7502	0\\
7503	0\\
7504	0\\
7505	0\\
7506	0\\
7507	0\\
7508	0\\
7509	0\\
7510	0\\
7511	0\\
7512	0\\
7513	0\\
7514	0\\
7515	0\\
7516	0\\
7517	0\\
7518	0\\
7519	0\\
7520	0\\
7521	0\\
7522	0\\
7523	0\\
7524	0\\
7525	0\\
7526	0\\
7527	0\\
7528	0\\
7529	0\\
7530	0\\
7531	0\\
7532	0\\
7533	0\\
7534	0\\
7535	0\\
7536	0\\
7537	0\\
7538	0\\
7539	0\\
7540	0\\
7541	0\\
7542	0\\
7543	0\\
7544	0\\
7545	0\\
7546	0\\
7547	0\\
7548	0\\
7549	0\\
7550	0\\
7551	0\\
7552	0\\
7553	0\\
7554	0\\
7555	0\\
7556	0\\
7557	0\\
7558	0\\
7559	0\\
7560	0\\
7561	0\\
7562	0\\
7563	0\\
7564	0\\
7565	0\\
7566	0\\
7567	0\\
7568	0.417094\\
7569	5e-06\\
7570	0.093268\\
7571	0.455588\\
7572	0.526319\\
7573	0.455586\\
7574	1.186601\\
7575	1.191874\\
7576	1.231977\\
7577	0.819693\\
7578	2.553665\\
7579	2.553665\\
7580	2.174193\\
7581	1.453239\\
7582	1e-06\\
7583	0\\
7584	0\\
7585	0\\
7586	0\\
7587	0\\
7588	0\\
7589	0\\
7590	0\\
7591	0\\
7592	1.453234\\
7593	1.173069\\
7594	0.043288\\
7595	0\\
7596	0\\
7597	0\\
7598	0\\
7599	0\\
7600	0\\
7601	0.192531\\
7602	2.553665\\
7603	2.553665\\
7604	1.453239\\
7605	0.607926\\
7606	0\\
7607	0\\
7608	0\\
7609	0\\
7610	0\\
7611	0\\
7612	0\\
7613	0\\
7614	0\\
7615	0\\
7616	0.444444\\
7617	0.800112\\
7618	1.302336\\
7619	0.696418\\
7620	1.453239\\
7621	0.882208\\
7622	1.453239\\
7623	1.751423\\
7624	1.996908\\
7625	2.553665\\
7626	2.553665\\
7627	2.553665\\
7628	1.453239\\
7629	1.453239\\
7630	1e-06\\
7631	0.221519\\
7632	0\\
7633	0\\
7634	0\\
7635	0\\
7636	0\\
7637	0\\
7638	0\\
7639	0\\
7640	0\\
7641	0\\
7642	0\\
7643	0\\
7644	0\\
7645	0\\
7646	0\\
7647	0\\
7648	0\\
7649	0\\
7650	0\\
7651	0\\
7652	0\\
7653	0\\
7654	0\\
7655	0\\
7656	0\\
7657	0\\
7658	0\\
7659	0\\
7660	0\\
7661	0\\
7662	0\\
7663	0\\
7664	0\\
7665	0\\
7666	0\\
7667	0\\
7668	0\\
7669	0\\
7670	0\\
7671	0\\
7672	0\\
7673	0\\
7674	0\\
7675	0\\
7676	0\\
7677	0\\
7678	0\\
7679	0\\
7680	0\\
7681	0\\
7682	0\\
7683	0\\
7684	0\\
7685	0\\
7686	0\\
7687	0\\
7688	0.74753\\
7689	1.300982\\
7690	1.54405\\
7691	1.453239\\
7692	1.453239\\
7693	0.740417\\
7694	1.226241\\
7695	2.202786\\
7696	2.553665\\
7697	2.41597\\
7698	3.164687\\
7699	3.103879\\
7700	2.28147\\
7701	1.361242\\
7702	0\\
7703	0\\
7704	0\\
7705	0\\
7706	0\\
7707	0\\
7708	0\\
7709	0\\
7710	0\\
7711	0\\
7712	2.553665\\
7713	2.549678\\
7714	1.973095\\
7715	1.576743\\
7716	1.543075\\
7717	1.55666\\
7718	2.268265\\
7719	2.221402\\
7720	2.553665\\
7721	2.415205\\
7722	2.893882\\
7723	2.734772\\
7724	2.553665\\
7725	2.253801\\
7726	0.426344\\
7727	0.455579\\
7728	0\\
7729	0\\
7730	0\\
7731	0\\
7732	0\\
7733	0\\
7734	0\\
7735	4.2e-05\\
7736	2.626861\\
7737	2.553665\\
7738	2.553665\\
7739	2.553665\\
7740	2.553665\\
7741	2.553665\\
7742	2.55408\\
7743	2.553665\\
7744	2.553665\\
7745	2.553665\\
7746	2.553666\\
7747	2.553665\\
7748	1.983303\\
7749	2.385093\\
7750	0.336574\\
7751	0\\
7752	0\\
7753	0\\
7754	0\\
7755	0\\
7756	0\\
7757	0\\
7758	0\\
7759	0\\
7760	2.553665\\
7761	2.259507\\
7762	1.530595\\
7763	0.930399\\
7764	0.416036\\
7765	0.389939\\
7766	0.817979\\
7767	1.116825\\
7768	1.453237\\
7769	1.77937\\
7770	2.552543\\
7771	2.553665\\
7772	1.453239\\
7773	1.402884\\
7774	0\\
7775	0\\
7776	0\\
7777	0\\
7778	0\\
7779	0\\
7780	0\\
7781	0\\
7782	0\\
7783	0\\
7784	1.453239\\
7785	2.451699\\
7786	2.423125\\
7787	2.462508\\
7788	1.860932\\
7789	1.453238\\
7790	2.114323\\
7791	1.453239\\
7792	1.453239\\
7793	2.121124\\
7794	2.553665\\
7795	2.553656\\
7796	1.453238\\
7797	7e-06\\
7798	0\\
7799	0\\
7800	0\\
7801	0\\
7802	0\\
7803	0\\
7804	0\\
7805	0\\
7806	0\\
7807	0\\
7808	0\\
7809	0\\
7810	0\\
7811	0\\
7812	0\\
7813	0\\
7814	0\\
7815	0\\
7816	0\\
7817	0\\
7818	0\\
7819	0\\
7820	0\\
7821	0\\
7822	0\\
7823	0\\
7824	0\\
7825	0\\
7826	0\\
7827	0\\
7828	0\\
7829	0\\
7830	0\\
7831	0\\
7832	0\\
7833	0\\
7834	0\\
7835	0\\
7836	0\\
7837	0\\
7838	0\\
7839	0\\
7840	0\\
7841	0\\
7842	0\\
7843	0\\
7844	0\\
7845	0\\
7846	0\\
7847	0\\
7848	0\\
7849	0\\
7850	0\\
7851	0\\
7852	0\\
7853	0\\
7854	0\\
7855	0\\
7856	0\\
7857	0\\
7858	0\\
7859	0\\
7860	0\\
7861	0\\
7862	0\\
7863	0\\
7864	0\\
7865	1e-06\\
7866	1.200066\\
7867	1.421857\\
7868	0.538816\\
7869	0\\
7870	0\\
7871	0\\
7872	0\\
7873	0\\
7874	0\\
7875	0\\
7876	0\\
7877	0\\
7878	0\\
7879	0\\
7880	0.888295\\
7881	1.20092\\
7882	0.455587\\
7883	3e-06\\
7884	1e-06\\
7885	0\\
7886	0\\
7887	0\\
7888	1e-06\\
7889	0.005546\\
7890	1.45324\\
7891	1.244902\\
7892	0.839098\\
7893	0\\
7894	0\\
7895	0\\
7896	0\\
7897	0\\
7898	0\\
7899	0\\
7900	0\\
7901	0\\
7902	0\\
7903	0\\
7904	1.150043\\
7905	2.010762\\
7906	2.496121\\
7907	2.553665\\
7908	2.553665\\
7909	2.553665\\
7910	2.578598\\
7911	2.644486\\
7912	2.760655\\
7913	3.103878\\
7914	3.023665\\
7915	3.103878\\
7916	2.553665\\
7917	0.992468\\
7918	1e-06\\
7919	0\\
7920	0\\
7921	0\\
7922	0\\
7923	0\\
7924	0\\
7925	0\\
7926	0\\
7927	1e-05\\
7928	2.553665\\
7929	2.85878\\
7930	2.553665\\
7931	2.553665\\
7932	2.553665\\
7933	2.553665\\
7934	2.553665\\
7935	2.553665\\
7936	2.553665\\
7937	2.553665\\
7938	2.553665\\
7939	2.553665\\
7940	2.553665\\
7941	1.453259\\
7942	0\\
7943	0\\
7944	0\\
7945	0\\
7946	0\\
7947	0\\
7948	0\\
7949	0\\
7950	0\\
7951	0\\
7952	0\\
7953	0\\
7954	0\\
7955	0\\
7956	0\\
7957	0\\
7958	0\\
7959	0\\
7960	0\\
7961	0\\
7962	0\\
7963	0\\
7964	0\\
7965	0\\
7966	0\\
7967	0\\
7968	0\\
7969	0\\
7970	0\\
7971	0\\
7972	0\\
7973	0\\
7974	0\\
7975	0\\
7976	0\\
7977	0\\
7978	0\\
7979	0\\
7980	0\\
7981	0\\
7982	0\\
7983	0\\
7984	0\\
7985	0\\
7986	0\\
7987	0\\
7988	0\\
7989	0\\
7990	0\\
7991	0\\
7992	0\\
7993	0\\
7994	0\\
7995	0\\
7996	0\\
7997	0\\
7998	0\\
7999	0\\
8000	0\\
8001	0\\
};
\addplot [color=mycolor1,solid,line width=1.0pt,forget plot]
  table[row sep=crcr]{%
8001	0\\
8002	0\\
8003	0.029654\\
8004	0.455413\\
8005	0.648703\\
8006	0\\
8007	0\\
8008	0.227616\\
8009	1.453239\\
8010	1.600902\\
8011	1.81544\\
8012	2.355411\\
8013	1.859299\\
8014	0.189265\\
8015	0.193349\\
8016	0\\
8017	0\\
8018	0\\
8019	0\\
8020	0\\
8021	0\\
8022	0\\
8023	0\\
8024	2.450128\\
8025	2.767504\\
8026	3.850431\\
8027	4.220381\\
8028	4.486731\\
8029	4.486731\\
8030	4.227991\\
8031	4.486731\\
8032	4.486731\\
8033	4.486731\\
8034	4.786262\\
8035	4.486731\\
8036	3.578959\\
8037	3.637694\\
8038	2.58423\\
8039	2.58423\\
8040	2.019953\\
8041	1.111572\\
8042	0\\
8043	0\\
8044	0\\
8045	0\\
8046	0\\
8047	0\\
8048	2.58423\\
8049	2.58423\\
8050	2.58423\\
8051	2.58423\\
8052	2.58423\\
8053	2.58423\\
8054	2.115644\\
8055	2.58423\\
8056	2.58423\\
8057	2.58423\\
8058	2.58423\\
8059	2.58423\\
8060	2.58423\\
8061	1.456402\\
8062	0\\
8063	0\\
8064	0\\
8065	0\\
8066	0\\
8067	0\\
8068	0\\
8069	0\\
8070	0\\
8071	0\\
8072	2.20028\\
8073	2.58423\\
8074	2.58423\\
8075	2.58423\\
8076	2.58423\\
8077	2.58423\\
8078	2.58423\\
8079	2.58423\\
8080	3.141029\\
8081	3.386083\\
8082	3.29196\\
8083	3.059854\\
8084	2.58423\\
8085	2.58423\\
8086	1.470633\\
8087	1.470633\\
8088	0.754919\\
8089	3e-06\\
8090	0\\
8091	0\\
8092	0\\
8093	0\\
8094	0\\
8095	1.470633\\
8096	3.141029\\
8097	4.177853\\
8098	4.169591\\
8099	4.09138\\
8100	4.288594\\
8101	4.290434\\
8102	4.321311\\
8103	4.227991\\
8104	4.227991\\
8105	4.220381\\
8106	4.730499\\
8107	4.486731\\
8108	4.227991\\
8109	3.682365\\
8110	2.58423\\
8111	2.58423\\
8112	2.58423\\
8113	0\\
8114	0\\
8115	0\\
8116	0\\
8117	0\\
8118	0\\
8119	1.470633\\
8120	2.767504\\
8121	2.58423\\
8122	3.141028\\
8123	3.141029\\
8124	3.141029\\
8125	2.774505\\
8126	2.584234\\
8127	2.58423\\
8128	2.58423\\
8129	2.58423\\
8130	2.58423\\
8131	2.192914\\
8132	1.826818\\
8133	2e-06\\
8134	0.054671\\
8135	0.46104\\
8136	0.412491\\
8137	0\\
8138	0\\
8139	0\\
8140	0\\
8141	0\\
8142	0\\
8143	0\\
8144	0\\
8145	0\\
8146	0\\
8147	1e-06\\
8148	0\\
8149	0\\
8150	0\\
8151	0\\
8152	0\\
8153	1.067512\\
8154	1.947996\\
8155	2.125732\\
8156	2.113585\\
8157	1.305699\\
8158	1.3e-05\\
8159	2.2e-05\\
8160	1.315305\\
8161	0\\
8162	1e-06\\
8163	0\\
8164	0\\
8165	0\\
8166	0\\
8167	0\\
8168	0\\
8169	0\\
8170	0\\
8171	0\\
8172	0\\
8173	0\\
8174	0\\
8175	0\\
8176	0\\
8177	0\\
8178	0\\
8179	0\\
8180	0\\
8181	0\\
8182	0\\
8183	0\\
8184	0\\
8185	0\\
8186	0\\
8187	0\\
8188	0\\
8189	0\\
8190	0\\
8191	0\\
8192	6e-06\\
8193	1.591686\\
8194	1.470633\\
8195	1.21443\\
8196	1.777332\\
8197	2.561552\\
8198	2.58423\\
8199	2.58423\\
8200	2.58423\\
8201	2.914891\\
8202	2.787961\\
8203	3.382629\\
8204	2.767477\\
8205	1.920006\\
8206	1e-06\\
8207	0.113503\\
8208	0.46104\\
8209	0\\
8210	0\\
8211	0\\
8212	0\\
8213	0\\
8214	0\\
8215	0.461039\\
8216	2.58423\\
8217	3.19418\\
8218	2.58423\\
8219	2.58423\\
8220	2.58423\\
8221	2.58423\\
8222	2.58423\\
8223	2.58423\\
8224	2.58423\\
8225	2.58423\\
8226	2.58423\\
8227	2.58423\\
8228	2.134387\\
8229	1e-06\\
8230	0\\
8231	0\\
8232	0\\
8233	0\\
8234	0\\
8235	0\\
8236	0\\
8237	0\\
8238	0\\
8239	0\\
8240	0\\
8241	1.470632\\
8242	1.444567\\
8243	0.537365\\
8244	0.379563\\
8245	0\\
8246	0.461039\\
8247	0.619491\\
8248	2.169805\\
8249	2.58423\\
8250	2.774466\\
8251	2.58423\\
8252	1.283418\\
8253	0.731644\\
8254	0\\
8255	0\\
8256	0\\
8257	0\\
8258	0\\
8259	0\\
8260	0\\
8261	0\\
8262	0\\
8263	0\\
8264	0\\
8265	0.851568\\
8266	0.556467\\
8267	1.018457\\
8268	1.32163\\
8269	1.094546\\
8270	0.918063\\
8271	0.959626\\
8272	1.214838\\
8273	1.617639\\
8274	2.110352\\
8275	1.470424\\
8276	0.385723\\
8277	1e-06\\
8278	0\\
8279	0\\
8280	0\\
8281	0\\
8282	0\\
8283	0\\
8284	0\\
8285	0\\
8286	0\\
8287	0\\
8288	0\\
8289	0\\
8290	0\\
8291	0\\
8292	0.236546\\
8293	4.5e-05\\
8294	0.291366\\
8295	0.461102\\
8296	0.480944\\
8297	1.744197\\
8298	2.155655\\
8299	2.58423\\
8300	1.21443\\
8301	0\\
8302	0\\
8303	0\\
8304	0\\
8305	0\\
8306	0\\
8307	0\\
8308	0\\
8309	0\\
8310	0\\
8311	0\\
8312	0\\
8313	0\\
8314	0\\
8315	1e-06\\
8316	7e-06\\
8317	2.3e-05\\
8318	7e-06\\
8319	0\\
8320	0.083805\\
8321	0.81056\\
8322	1.091049\\
8323	1.743017\\
8324	0.237414\\
8325	1e-06\\
8326	0\\
8327	0\\
8328	0\\
8329	0\\
8330	0\\
8331	0\\
8332	0\\
8333	0\\
8334	0\\
8335	0\\
8336	0\\
8337	0\\
8338	0\\
8339	0\\
8340	0\\
8341	0\\
8342	0\\
8343	0\\
8344	0\\
8345	0\\
8346	0\\
8347	0\\
8348	0\\
8349	0\\
8350	0\\
8351	0\\
8352	0\\
8353	0\\
8354	0\\
8355	0\\
8356	0\\
8357	0\\
8358	0\\
8359	0\\
8360	0\\
8361	0\\
8362	0\\
8363	0\\
8364	0\\
8365	0\\
8366	0\\
8367	0\\
8368	0\\
8369	0\\
8370	0.315301\\
8371	2e-06\\
8372	0\\
8373	0\\
8374	0\\
8375	0\\
8376	0\\
8377	0\\
8378	0\\
8379	0\\
8380	0\\
8381	0\\
8382	0\\
8383	0\\
8384	1.21443\\
8385	2.58423\\
8386	2.58423\\
8387	2.396343\\
8388	2.425566\\
8389	1.745406\\
8390	1.93188\\
8391	2.2413\\
8392	2.58423\\
8393	2.58423\\
8394	3.322144\\
8395	3.442039\\
8396	2.58423\\
8397	2.58423\\
8398	0.338607\\
8399	0.353238\\
8400	0\\
8401	0\\
8402	0\\
8403	0\\
8404	0\\
8405	0\\
8406	0\\
8407	0\\
8408	0\\
8409	0.461039\\
8410	1.890096\\
8411	2.507219\\
8412	2.58423\\
8413	2.58423\\
8414	2.58423\\
8415	2.58423\\
8416	2.58423\\
8417	3.065261\\
8418	2.765555\\
8419	2.606016\\
8420	2.708492\\
8421	1.407354\\
8422	0\\
8423	0\\
8424	0\\
8425	0\\
8426	0\\
8427	0\\
8428	0\\
8429	0\\
8430	0\\
8431	0\\
8432	1e-06\\
8433	1.429666\\
8434	0\\
8435	1e-06\\
8436	0.46089\\
8437	0\\
8438	1e-06\\
8439	0\\
8440	0\\
8441	0\\
8442	0\\
8443	0\\
8444	0\\
8445	0\\
8446	0\\
8447	0\\
8448	0\\
8449	0\\
8450	0\\
8451	0\\
8452	0\\
8453	0\\
8454	0\\
8455	0\\
8456	0\\
8457	0\\
8458	0\\
8459	0\\
8460	0.432145\\
8461	1.072206\\
8462	0\\
8463	0\\
8464	0\\
8465	0\\
8466	0\\
8467	0\\
8468	0\\
8469	0\\
8470	0\\
8471	0\\
8472	0\\
8473	0\\
8474	0\\
8475	0\\
8476	0\\
8477	0\\
8478	0\\
8479	0\\
8480	0\\
8481	0\\
8482	0\\
8483	0\\
8484	0\\
8485	0\\
8486	0\\
8487	0\\
8488	0\\
8489	0\\
8490	0\\
8491	0\\
8492	0\\
8493	0\\
8494	0\\
8495	0\\
8496	0\\
8497	0\\
8498	0\\
8499	0\\
8500	0\\
8501	0\\
8502	0\\
8503	0\\
8504	0\\
8505	0\\
8506	0\\
8507	0\\
8508	0\\
8509	0\\
8510	0\\
8511	0\\
8512	0\\
8513	0\\
8514	0\\
8515	0\\
8516	0\\
8517	0\\
8518	0\\
8519	0\\
8520	0\\
8521	0\\
8522	0\\
8523	0\\
8524	0\\
8525	0\\
8526	0\\
8527	0\\
8528	0\\
8529	0\\
8530	0\\
8531	0\\
8532	0\\
8533	0\\
8534	0\\
8535	0\\
8536	0\\
8537	0\\
8538	0\\
8539	0\\
8540	0\\
8541	0\\
8542	0\\
8543	0\\
8544	0\\
8545	0\\
8546	0\\
8547	0\\
8548	0\\
8549	0\\
8550	0\\
8551	0\\
8552	0\\
8553	0\\
8554	0\\
8555	0\\
8556	0\\
8557	0\\
8558	0\\
8559	0\\
8560	0\\
8561	0\\
8562	0\\
8563	0\\
8564	0\\
8565	0\\
8566	0\\
8567	0\\
8568	0\\
8569	0\\
8570	0\\
8571	0\\
8572	0\\
8573	0\\
8574	0\\
8575	0\\
8576	0\\
8577	0\\
8578	0\\
8579	0\\
8580	0\\
8581	0\\
8582	0\\
8583	0\\
8584	0\\
8585	0\\
8586	0\\
8587	0\\
8588	0\\
8589	0\\
8590	0\\
8591	0\\
8592	0\\
8593	0\\
8594	0\\
8595	0\\
8596	0\\
8597	0\\
8598	0\\
8599	0\\
8600	0\\
8601	0\\
8602	0\\
8603	0\\
8604	0\\
8605	0\\
8606	0\\
8607	0\\
8608	0\\
8609	0\\
8610	0\\
8611	0\\
8612	0\\
8613	0\\
8614	0\\
8615	0\\
8616	0\\
8617	0\\
8618	0\\
8619	0\\
8620	0\\
8621	0\\
8622	0\\
8623	0\\
8624	0\\
8625	0\\
8626	0\\
8627	0\\
8628	0\\
8629	0\\
8630	0\\
8631	0\\
8632	0\\
8633	0\\
8634	0\\
8635	0\\
8636	0\\
8637	0\\
8638	0\\
8639	0\\
8640	0\\
8641	0\\
8642	0\\
8643	0\\
8644	0\\
8645	0\\
8646	0\\
8647	0\\
8648	0\\
8649	0\\
8650	0\\
8651	0\\
8652	0\\
8653	0\\
8654	0\\
8655	0\\
8656	0\\
8657	0\\
8658	0\\
8659	0\\
8660	0\\
8661	0\\
8662	0\\
8663	0\\
8664	0\\
8665	0\\
8666	0\\
8667	0\\
8668	0\\
8669	0\\
8670	0\\
8671	0\\
8672	0\\
8673	0\\
8674	0\\
8675	0\\
8676	0\\
8677	0\\
8678	0\\
8679	0\\
8680	0\\
8681	0\\
8682	0\\
8683	0\\
8684	0\\
8685	0\\
8686	0\\
8687	0\\
8688	0\\
8689	0\\
8690	0\\
8691	0\\
8692	0\\
8693	0\\
8694	0\\
8695	0\\
8696	0\\
8697	0\\
8698	0\\
8699	0\\
8700	0\\
8701	0\\
8702	0\\
8703	0\\
8704	0\\
8705	0\\
8706	0\\
8707	7e-06\\
8708	0\\
8709	0\\
8710	0\\
8711	0\\
8712	0\\
8713	0\\
8714	0\\
8715	0\\
8716	0\\
8717	0\\
8718	0\\
8719	0\\
8720	0\\
8721	0\\
8722	0\\
8723	0\\
8724	0\\
8725	0\\
8726	0\\
8727	0\\
8728	0\\
8729	0\\
8730	0.491856\\
8731	0.233086\\
8732	0\\
8733	0\\
8734	0\\
8735	0\\
8736	0\\
8737	0\\
8738	0\\
8739	0\\
8740	0\\
8741	0\\
8742	0\\
8743	0\\
8744	0\\
8745	0\\
8746	0\\
8747	0\\
8748	0\\
8749	0\\
8750	0\\
8751	0\\
8752	0\\
8753	0\\
8754	0\\
8755	0\\
8756	0\\
8757	0\\
8758	0\\
8759	0\\
8760	0\\
};
\end{axis}
\end{tikzpicture}%
    \caption{Predicted reserve prices for the ImpExp model}
    \label{fig:ImpExp_R}
\end{figure}

\begin{figure}[H]
    \centering
    \setlength\fheight{0.3\textwidth}
    \setlength\fwidth{0.85\textwidth}
    \input{images/ORDC_reserve.tikz}
    \caption{Predicted reserve prices for the ORDC model}
    \label{fig:ORDC_R1}
\end{figure}

\begin{figure}[H]
    \centering
    \setlength\fheight{0.3\textwidth}
    \setlength\fwidth{0.85\textwidth}
    % This file was created by matlab2tikz.
% Minimal pgfplots version: 1.3
%
%The latest updates can be retrieved from
%  http://www.mathworks.com/matlabcentral/fileexchange/22022-matlab2tikz
%where you can also make suggestions and rate matlab2tikz.
%
\definecolor{mycolor1}{rgb}{0.87059,0.49020,0.00000}%
%
\begin{tikzpicture}

\begin{axis}[%
width=\fwidth,
height=\fheight,
at={(0\fwidth,0\fheight)},
scale only axis,
separate axis lines,
every outer x axis line/.append style={black},
every x tick label/.append style={font=\color{black}},
xmin=0,
xmax=8760,
xlabel={time [hour]},
xtick={0,1000,2000,3000,4000,5000,6000,7000,8000},
xmajorgrids,
every outer y axis line/.append style={black},
every y tick label/.append style={font=\color{black}},
ymin=0,
ymax=70,
ymajorgrids,
title style={font=\bfseries},
title={ORDC - Reserve price [\euro/MWh]}
]
\addplot [color=mycolor1,solid,line width=1.0pt,forget plot]
  table[row sep=crcr]{%
1	2e-06\\
2	2e-06\\
3	6e-06\\
4	1e-06\\
5	0\\
6	0\\
7	0\\
8	2e-06\\
9	2e-06\\
10	2e-06\\
11	5e-06\\
12	4e-06\\
13	0\\
14	0\\
15	0\\
16	0\\
17	0\\
18	0\\
19	1e-06\\
20	1e-06\\
21	1e-06\\
22	1e-06\\
23	0\\
24	2e-06\\
25	1e-06\\
26	0\\
27	1e-06\\
28	0\\
29	1e-06\\
30	0\\
31	2e-06\\
32	1e-06\\
33	2e-06\\
34	3e-06\\
35	9e-06\\
36	2e-06\\
37	2e-06\\
38	2e-06\\
39	2e-06\\
40	2e-06\\
41	1e-06\\
42	2e-06\\
43	1e-06\\
44	4e-06\\
45	3e-06\\
46	1e-06\\
47	2e-06\\
48	6e-06\\
49	3e-06\\
50	0\\
51	4e-06\\
52	0\\
53	0\\
54	2e-06\\
55	1e-06\\
56	8e-06\\
57	3e-06\\
58	9e-06\\
59	3e-06\\
60	3e-06\\
61	3e-06\\
62	2e-06\\
63	4e-06\\
64	6e-06\\
65	3e-06\\
66	4e-06\\
67	4e-06\\
68	2e-06\\
69	3e-06\\
70	5e-06\\
71	7e-06\\
72	5e-06\\
73	9e-06\\
74	3e-06\\
75	1e-06\\
76	2e-06\\
77	0\\
78	6e-06\\
79	1e-06\\
80	4e-06\\
81	3e-06\\
82	1e-05\\
83	0\\
84	4e-06\\
85	9e-06\\
86	4e-06\\
87	1e-05\\
88	0\\
89	3e-06\\
90	1e-06\\
91	3e-06\\
92	2e-06\\
93	5e-06\\
94	2e-06\\
95	2e-06\\
96	4e-06\\
97	4e-06\\
98	4e-06\\
99	0\\
100	3e-06\\
101	5e-06\\
102	3e-06\\
103	5e-06\\
104	3e-06\\
105	2e-06\\
106	4e-06\\
107	0\\
108	0\\
109	0\\
110	1e-05\\
111	8e-06\\
112	0\\
113	4e-06\\
114	3e-06\\
115	1.1e-05\\
116	4e-06\\
117	2e-06\\
118	4e-06\\
119	2e-06\\
120	2e-06\\
121	2e-06\\
122	0\\
123	2e-06\\
124	1e-06\\
125	1e-06\\
126	0\\
127	2e-06\\
128	0\\
129	3e-06\\
130	1e-06\\
131	1.3e-05\\
132	0\\
133	6e-06\\
134	3e-06\\
135	5e-06\\
136	0\\
137	0\\
138	0.011258\\
139	1e-06\\
140	1e-06\\
141	1e-06\\
142	5e-06\\
143	0\\
144	2e-06\\
145	2e-06\\
146	5e-06\\
147	2e-06\\
148	1e-06\\
149	1e-06\\
150	0\\
151	2e-06\\
152	0\\
153	0.224931\\
154	1e-06\\
155	1e-05\\
156	8.2e-05\\
157	7e-06\\
158	0\\
159	2.8e-05\\
160	1.688574\\
161	1.808459\\
162	4.547575\\
163	4.547572\\
164	2.160853\\
165	0.70194\\
166	2.6e-05\\
167	9e-06\\
168	11.017886\\
169	8.19841\\
170	6.073945\\
171	9e-06\\
172	5e-06\\
173	5e-06\\
174	1.4e-05\\
175	2e-06\\
176	1.983045\\
177	4.547572\\
178	3.570045\\
179	2.098037\\
180	1.408328\\
181	3.266787\\
182	6.819754\\
183	6.821725\\
184	6.821726\\
185	6.821726\\
186	7.563612\\
187	8.141963\\
188	6.821726\\
189	7.764469\\
190	7.073953\\
191	7.226651\\
192	5.982189\\
193	0.10759\\
194	3.4e-05\\
195	8.9e-05\\
196	1.4e-05\\
197	3.2e-05\\
198	1e-06\\
199	0.107596\\
200	3.431216\\
201	5.273166\\
202	3.108903\\
203	3.07087\\
204	3.383761\\
205	4.834447\\
206	5.158967\\
207	4.670445\\
208	4.915935\\
209	3.767686\\
210	4.131445\\
211	4.547564\\
212	2.751694\\
213	0.701945\\
214	0.803057\\
215	1.596805\\
216	1.352639\\
217	2e-06\\
218	0\\
219	7e-06\\
220	1e-06\\
221	2e-06\\
222	1.1e-05\\
223	2.303371\\
224	5.687139\\
225	6.684602\\
226	4.987116\\
227	4.547572\\
228	4.956\\
229	3.709624\\
230	4.547548\\
231	4.547576\\
232	4.666527\\
233	5.061395\\
234	6.446448\\
235	5.8579\\
236	4.547573\\
237	3.32127\\
238	2.247857\\
239	2.96533\\
240	3.238165\\
241	1.4e-05\\
242	2e-06\\
243	2e-06\\
244	1.5e-05\\
245	1e-06\\
246	3e-06\\
247	1.4e-05\\
248	0\\
249	8.2e-05\\
250	0.70175\\
251	0.922854\\
252	1.022764\\
253	1.20278\\
254	1.331871\\
255	0.516584\\
256	0.709078\\
257	1.352633\\
258	3.978581\\
259	3.01312\\
260	1.829029\\
261	2.7e-05\\
262	0.48637\\
263	1.688592\\
264	0.13505\\
265	0.591038\\
266	0.107596\\
267	7e-06\\
268	1.1e-05\\
269	2.9e-05\\
270	1.5e-05\\
271	2.9e-05\\
272	1e-06\\
273	0.000577\\
274	0.1076\\
275	0.107599\\
276	2e-05\\
277	8e-06\\
278	3e-06\\
279	1e-06\\
280	2e-06\\
281	0\\
282	0.000108\\
283	1.457164\\
284	0.224879\\
285	2e-06\\
286	2.6e-05\\
287	6e-06\\
288	0\\
289	7e-06\\
290	2e-06\\
291	4e-06\\
292	1e-06\\
293	1e-06\\
294	5e-06\\
295	1.2e-05\\
296	8e-06\\
297	3.3e-05\\
298	1.6e-05\\
299	1.055009\\
300	1.466137\\
301	0.520482\\
302	0.724583\\
303	0.343273\\
304	0.885351\\
305	1.48607\\
306	2.517327\\
307	3.194939\\
308	0.637698\\
309	1e-06\\
310	9e-06\\
311	2e-06\\
312	3e-06\\
313	3e-06\\
314	1e-06\\
315	3e-06\\
316	2e-06\\
317	1e-06\\
318	3e-06\\
319	5e-06\\
320	1.997841\\
321	2.383914\\
322	3.194939\\
323	3.194939\\
324	3.194939\\
325	3.194938\\
326	3.018529\\
327	3.194939\\
328	3.195245\\
329	3.646261\\
330	4.872207\\
331	3.708764\\
332	3.226687\\
333	3.194939\\
334	0.550494\\
335	1.042986\\
336	0.335973\\
337	0\\
338	2e-06\\
339	2e-06\\
340	3e-06\\
341	3e-06\\
342	2e-06\\
343	0\\
344	0.894438\\
345	0.988691\\
346	0.891175\\
347	1.575733\\
348	1.773645\\
349	1.031291\\
350	1.150245\\
351	0.477475\\
352	0.591114\\
353	0.891077\\
354	2.014155\\
355	1.70528\\
356	1.173415\\
357	5e-06\\
358	0\\
359	5e-06\\
360	3e-06\\
361	4e-06\\
362	2e-06\\
363	1e-06\\
364	1e-06\\
365	1e-06\\
366	5e-06\\
367	1e-06\\
368	7e-06\\
369	5e-06\\
370	1e-06\\
371	3e-06\\
372	3e-06\\
373	0.750127\\
374	2.892947\\
375	2.342836\\
376	1.693536\\
377	1.17774\\
378	3.194939\\
379	1.89341\\
380	0.894432\\
381	1e-06\\
382	1.3e-05\\
383	0\\
384	9e-06\\
385	2e-06\\
386	4e-06\\
387	3e-06\\
388	1e-06\\
389	2e-06\\
390	1e-06\\
391	4e-06\\
392	1e-06\\
393	1e-06\\
394	7e-06\\
395	1e-06\\
396	3.7e-05\\
397	1.7e-05\\
398	4e-06\\
399	2e-06\\
400	5.6e-05\\
401	0.262853\\
402	2.108915\\
403	0.667017\\
404	1e-06\\
405	5.5e-05\\
406	2e-06\\
407	4e-06\\
408	3e-06\\
409	2e-06\\
410	4e-06\\
411	3e-06\\
412	1e-06\\
413	1e-06\\
414	1e-06\\
415	1e-06\\
416	1e-06\\
417	2e-06\\
418	2e-06\\
419	1e-06\\
420	3e-06\\
421	5e-06\\
422	7e-06\\
423	6e-06\\
424	2e-06\\
425	4e-06\\
426	0\\
427	2e-06\\
428	2e-06\\
429	1e-06\\
430	0\\
431	4e-06\\
432	5e-06\\
433	3e-06\\
434	1e-06\\
435	0\\
436	3e-06\\
437	3e-06\\
438	3e-06\\
439	4e-06\\
440	3e-06\\
441	4e-06\\
442	3e-06\\
443	2e-06\\
444	1e-06\\
445	0\\
446	1e-06\\
447	0\\
448	0\\
449	6e-06\\
450	8e-06\\
451	0\\
452	1e-05\\
453	0\\
454	0\\
455	1e-06\\
456	6e-06\\
457	4e-06\\
458	3e-06\\
459	5e-06\\
460	1e-06\\
461	8e-06\\
462	2e-06\\
463	3e-05\\
464	2.938922\\
465	2.513503\\
466	3.195058\\
467	3.603385\\
468	3.686806\\
469	3.19494\\
470	3.708764\\
471	3.686806\\
472	3.708764\\
473	3.765364\\
474	4.670758\\
475	3.719842\\
476	3.084152\\
477	3.190196\\
478	2e-06\\
479	4.3e-05\\
480	0\\
481	7e-06\\
482	4e-06\\
483	0\\
484	2e-06\\
485	5e-06\\
486	4e-06\\
487	4e-06\\
488	3.194939\\
489	2.568523\\
490	0.894438\\
491	0.894425\\
492	2.314475\\
493	1.594926\\
494	2.761125\\
495	2.902721\\
496	3.194824\\
497	3.194939\\
498	4.338004\\
499	3.19494\\
500	3.194939\\
501	1.865849\\
502	8e-06\\
503	5e-06\\
504	2.6e-05\\
505	4e-06\\
506	2e-06\\
507	5e-06\\
508	1e-06\\
509	1e-06\\
510	3e-06\\
511	1e-06\\
512	0.335863\\
513	1e-06\\
514	4e-05\\
515	1.9e-05\\
516	0\\
517	1e-06\\
518	3e-06\\
519	2.5e-05\\
520	0.712887\\
521	1.606437\\
522	3.195044\\
523	3.194939\\
524	3.021405\\
525	1.293913\\
526	4e-06\\
527	0.257127\\
528	1e-06\\
529	7e-06\\
530	2e-06\\
531	3e-06\\
532	3e-06\\
533	8e-06\\
534	4e-06\\
535	3e-06\\
536	2.443548\\
537	1.672179\\
538	1.846317\\
539	1.624857\\
540	1.296079\\
541	0.756864\\
542	1.573246\\
543	1.074152\\
544	1.271382\\
545	1.233731\\
546	3.194929\\
547	1.395702\\
548	0.057361\\
549	2.8e-05\\
550	1.4e-05\\
551	4e-06\\
552	0\\
553	2e-06\\
554	1e-06\\
555	3e-06\\
556	2e-06\\
557	3e-06\\
558	3e-06\\
559	2.6e-05\\
560	2.665624\\
561	1.484651\\
562	0.894425\\
563	0.063348\\
564	1e-06\\
565	1e-06\\
566	4.3e-05\\
567	6e-06\\
568	5e-06\\
569	0.015638\\
570	1.912096\\
571	0.894436\\
572	0.006999\\
573	1.3e-05\\
574	6e-06\\
575	0\\
576	2e-05\\
577	1.1e-05\\
578	3e-06\\
579	1e-06\\
580	5e-06\\
581	6e-06\\
582	2e-06\\
583	2e-06\\
584	5e-06\\
585	3e-06\\
586	5e-06\\
587	0\\
588	0\\
589	0\\
590	1e-06\\
591	4e-06\\
592	3e-06\\
593	3e-06\\
594	4e-06\\
595	6e-06\\
596	3e-06\\
597	1e-06\\
598	3e-06\\
599	3e-06\\
600	7e-06\\
601	2e-06\\
602	0\\
603	0\\
604	0\\
605	0\\
606	4e-06\\
607	1e-06\\
608	3e-06\\
609	1e-06\\
610	1e-06\\
611	2e-06\\
612	2e-06\\
613	2e-06\\
614	2e-06\\
615	1e-06\\
616	1e-06\\
617	1e-06\\
618	1e-06\\
619	1e-06\\
620	1e-06\\
621	1e-06\\
622	2e-06\\
623	1e-06\\
624	1e-06\\
625	4e-06\\
626	1e-06\\
627	0\\
628	2e-06\\
629	2e-06\\
630	1e-06\\
631	3e-06\\
632	0\\
633	1e-06\\
634	2e-06\\
635	4e-06\\
636	1e-06\\
637	1e-06\\
638	6e-06\\
639	7e-06\\
640	1e-06\\
641	0.000251\\
642	2e-06\\
643	0.169579\\
644	7e-06\\
645	1.1e-05\\
646	1e-06\\
647	5e-06\\
648	4e-06\\
649	2e-06\\
650	3e-06\\
651	1e-06\\
652	1e-06\\
653	1e-06\\
654	2e-06\\
655	1e-06\\
656	7e-06\\
657	1.2e-05\\
658	2e-06\\
659	0\\
660	3e-06\\
661	2e-06\\
662	0\\
663	6e-06\\
664	6e-06\\
665	2.6e-05\\
666	2e-06\\
667	3e-06\\
668	3e-06\\
669	0\\
670	1e-06\\
671	2e-05\\
672	2e-06\\
673	1e-06\\
674	6e-06\\
675	2e-06\\
676	1e-06\\
677	3e-06\\
678	1e-06\\
679	1e-06\\
680	1e-06\\
681	1e-05\\
682	1e-06\\
683	0\\
684	0\\
685	5e-06\\
686	0\\
687	8e-06\\
688	1e-06\\
689	1e-05\\
690	5e-06\\
691	0.894434\\
692	7e-06\\
693	9e-06\\
694	4e-06\\
695	3e-06\\
696	3e-06\\
697	6e-06\\
698	2e-06\\
699	3e-06\\
700	5e-06\\
701	7e-06\\
702	2e-06\\
703	2e-06\\
704	2.503988\\
705	0.393345\\
706	1.897277\\
707	0.396437\\
708	0.89425\\
709	5.4e-05\\
710	0.078146\\
711	0.262768\\
712	0.821091\\
713	2.370327\\
714	3.275833\\
715	3.194939\\
716	2.05159\\
717	1.876045\\
718	1e-06\\
719	0.772518\\
720	0.447602\\
721	8e-06\\
722	0\\
723	2e-06\\
724	2e-06\\
725	3e-06\\
726	3e-06\\
727	0.000111\\
728	3.194939\\
729	0.583524\\
730	1e-06\\
731	4.9e-05\\
732	0\\
733	7e-06\\
734	1e-06\\
735	3e-06\\
736	0\\
737	5e-06\\
738	3e-06\\
739	7.3e-05\\
740	0\\
741	0\\
742	0\\
743	2e-06\\
744	3e-06\\
745	1e-06\\
746	9e-06\\
747	0\\
748	2e-06\\
749	0\\
750	0\\
751	0\\
752	0\\
753	1e-06\\
754	3e-06\\
755	0\\
756	1e-06\\
757	5e-06\\
758	3e-06\\
759	1e-06\\
760	1e-06\\
761	1e-06\\
762	0\\
763	0\\
764	2e-06\\
765	1e-06\\
766	1e-06\\
767	1e-06\\
768	4e-06\\
769	1e-06\\
770	0\\
771	0\\
772	0\\
773	3e-06\\
774	1e-06\\
775	0\\
776	1e-06\\
777	0\\
778	0\\
779	5e-06\\
780	8e-06\\
781	0\\
782	5e-06\\
783	0\\
784	6e-06\\
785	1e-06\\
786	2e-06\\
787	0\\
788	0\\
789	6e-06\\
790	1e-06\\
791	5e-06\\
792	1e-05\\
793	0\\
794	1e-06\\
795	1e-06\\
796	4e-06\\
797	1e-06\\
798	1e-06\\
799	1.8e-05\\
800	2.910001\\
801	1.643808\\
802	0.381114\\
803	1e-05\\
804	6e-06\\
805	0\\
806	3.3e-05\\
807	1e-05\\
808	0.306032\\
809	1.740324\\
810	3.281994\\
811	4.110747\\
812	2.920389\\
813	2.413893\\
814	0.00829\\
815	0.530802\\
816	0.022366\\
817	6e-06\\
818	0\\
819	0\\
820	3e-06\\
821	3e-06\\
822	1e-06\\
823	1.8e-05\\
824	2.516684\\
825	0.746623\\
826	0.700138\\
827	0.418578\\
828	0.658305\\
829	3.9e-05\\
830	1.6e-05\\
831	1.8e-05\\
832	2.9e-05\\
833	0.362787\\
834	2.151352\\
835	2.910001\\
836	1.315299\\
837	0.632901\\
838	8e-06\\
839	3.2e-05\\
840	6e-06\\
841	1e-06\\
842	1e-06\\
843	0\\
844	0\\
845	0\\
846	9e-06\\
847	8e-06\\
848	1.3e-05\\
849	0.775314\\
850	0.306029\\
851	0.816052\\
852	0.749947\\
853	2e-05\\
854	9.2e-05\\
855	1.2e-05\\
856	4e-06\\
857	0.306055\\
858	2.893872\\
859	2.62514\\
860	0.787793\\
861	5e-06\\
862	1e-06\\
863	0\\
864	3e-06\\
865	1e-06\\
866	0\\
867	1e-06\\
868	0\\
869	1e-06\\
870	1e-06\\
871	0\\
872	1.7e-05\\
873	1e-06\\
874	2.7e-05\\
875	7e-06\\
876	3.1e-05\\
877	3.9e-05\\
878	0.456056\\
879	0.567914\\
880	0.538624\\
881	1.051859\\
882	2.909999\\
883	1.955781\\
884	0.635856\\
885	1e-06\\
886	2e-06\\
887	6e-06\\
888	0\\
889	1e-06\\
890	1e-06\\
891	9e-06\\
892	0\\
893	0\\
894	0\\
895	2e-06\\
896	0\\
897	2.2e-05\\
898	6e-06\\
899	0.30604\\
900	2.512183\\
901	1.301385\\
902	1.540771\\
903	1.136297\\
904	0.486668\\
905	9e-06\\
906	3e-06\\
907	0.515569\\
908	3e-06\\
909	1e-06\\
910	4e-06\\
911	2e-06\\
912	0\\
913	1e-06\\
914	1e-06\\
915	1e-06\\
916	2e-06\\
917	7e-06\\
918	2e-06\\
919	2e-06\\
920	0\\
921	0\\
922	1e-06\\
923	1e-06\\
924	9e-06\\
925	2e-06\\
926	6e-06\\
927	1e-06\\
928	1e-06\\
929	1e-06\\
930	1e-06\\
931	1e-06\\
932	1e-06\\
933	9e-06\\
934	6e-06\\
935	0\\
936	1e-06\\
937	0\\
938	6e-06\\
939	0\\
940	1e-06\\
941	1e-06\\
942	1e-06\\
943	2e-06\\
944	1e-06\\
945	0\\
946	0\\
947	4e-06\\
948	0\\
949	7e-06\\
950	1e-06\\
951	0\\
952	0\\
953	2e-06\\
954	1e-06\\
955	1e-06\\
956	1.2e-05\\
957	1e-06\\
958	1e-06\\
959	1e-06\\
960	1e-06\\
961	1e-06\\
962	1e-06\\
963	0\\
964	0\\
965	6e-06\\
966	1e-06\\
967	3e-06\\
968	3.248924\\
969	3.282019\\
970	3.282043\\
971	2.910001\\
972	3.086409\\
973	2.91\\
974	3.357998\\
975	3.358001\\
976	3.615126\\
977	3.98693\\
978	4.981335\\
979	4.981336\\
980	4.219446\\
981	2.910005\\
982	1.720482\\
983	2.046779\\
984	1.35606\\
985	2e-06\\
986	8e-06\\
987	1e-06\\
988	1e-06\\
989	0\\
990	3e-06\\
991	1.2e-05\\
992	0.595757\\
993	2.2e-05\\
994	5.9e-05\\
995	2e-06\\
996	4.3e-05\\
997	1e-05\\
998	2e-06\\
999	3.8e-05\\
1000	1.1e-05\\
1001	0.306596\\
1002	2.629336\\
1003	2.910001\\
1004	2.195944\\
1005	0.783765\\
1006	6e-05\\
1007	0.305987\\
1008	0.005497\\
1009	4e-06\\
1010	1e-06\\
1011	1e-06\\
1012	3e-06\\
1013	5e-06\\
1014	1e-06\\
1015	1.3e-05\\
1016	3.053448\\
1017	1.723285\\
1018	0.011443\\
1019	3e-06\\
1020	0\\
1021	1e-06\\
1022	0\\
1023	3e-06\\
1024	2e-05\\
1025	2e-06\\
1026	1.291911\\
1027	1.718756\\
1028	0.068331\\
1029	5e-05\\
1030	1e-06\\
1031	7e-06\\
1032	3e-06\\
1033	1e-06\\
1034	0\\
1035	0\\
1036	0\\
1037	0\\
1038	1e-06\\
1039	0\\
1040	4e-06\\
1041	0.002862\\
1042	2.901425\\
1043	2.910001\\
1044	3.378017\\
1045	3.358001\\
1046	3.630718\\
1047	3.358002\\
1048	3.358001\\
1049	3.536167\\
1050	4.158963\\
1051	3.664101\\
1052	3.358001\\
1053	2.3772\\
1054	0.765745\\
1055	1.109988\\
1056	0.509128\\
1057	2e-06\\
1058	0\\
1059	0\\
1060	0\\
1061	8e-06\\
1062	1e-06\\
1063	1.798535\\
1064	2.379916\\
1065	0.823262\\
1066	0.308136\\
1067	3.7e-05\\
1068	0.023844\\
1069	0.305983\\
1070	1.083575\\
1071	1.220152\\
1072	0.819437\\
1073	0.809976\\
1074	2.049086\\
1075	0.291721\\
1076	1e-05\\
1077	5e-06\\
1078	5e-06\\
1079	0\\
1080	0\\
1081	1e-06\\
1082	0\\
1083	0\\
1084	4e-06\\
1085	2e-06\\
1086	2e-06\\
1087	0\\
1088	0\\
1089	0\\
1090	9e-06\\
1091	1e-06\\
1092	1e-06\\
1093	0\\
1094	0\\
1095	7e-06\\
1096	0\\
1097	8e-06\\
1098	0\\
1099	1e-06\\
1100	0\\
1101	0\\
1102	0\\
1103	1e-06\\
1104	1e-06\\
1105	0\\
1106	0\\
1107	0\\
1108	2e-06\\
1109	3e-06\\
1110	2e-06\\
1111	4e-06\\
1112	0\\
1113	0\\
1114	0\\
1115	0\\
1116	7e-06\\
1117	0\\
1118	0\\
1119	0\\
1120	0\\
1121	8e-06\\
1122	1e-06\\
1123	4e-06\\
1124	1e-06\\
1125	1e-06\\
1126	1e-06\\
1127	1e-06\\
1128	1e-06\\
1129	0\\
1130	0\\
1131	8e-06\\
1132	0\\
1133	0\\
1134	0\\
1135	1.7e-05\\
1136	2.908612\\
1137	2.197431\\
1138	0.757512\\
1139	0.306036\\
1140	0.767711\\
1141	5e-06\\
1142	1e-06\\
1143	1.3e-05\\
1144	1.2e-05\\
1145	2e-05\\
1146	0.148303\\
1147	2.910001\\
1148	2.910001\\
1149	2.382677\\
1150	0.294341\\
1151	0.125237\\
1152	3e-06\\
1153	0\\
1154	1e-06\\
1155	5e-06\\
1156	0\\
1157	1.1e-05\\
1158	1e-06\\
1159	3.6e-05\\
1160	0.987208\\
1161	0.423428\\
1162	1.059701\\
1163	2.05043\\
1164	2.910001\\
1165	2.91\\
1166	2.910001\\
1167	2.910001\\
1168	2.910001\\
1169	2.910001\\
1170	3.530949\\
1171	3.98308\\
1172	2.991439\\
1173	2.887698\\
1174	0.008352\\
1175	0.647297\\
1176	0.001832\\
1177	3e-06\\
1178	5e-06\\
1179	1e-06\\
1180	1e-06\\
1181	1e-06\\
1182	1e-06\\
1183	0.30624\\
1184	2.910001\\
1185	2.55182\\
1186	2.874642\\
1187	2.931737\\
1188	3.377989\\
1189	2.910001\\
1190	2.910004\\
1191	2.910001\\
1192	2.910003\\
1193	2.910001\\
1194	3.357995\\
1195	3.743931\\
1196	3.358005\\
1197	3.117159\\
1198	1.870867\\
1199	2.264529\\
1200	0.972408\\
1201	3e-06\\
1202	0\\
1203	1e-06\\
1204	8e-06\\
1205	0\\
1206	1e-06\\
1207	4e-06\\
1208	2.8e-05\\
1209	4e-06\\
1210	8e-06\\
1211	1.8e-05\\
1212	0.604556\\
1213	0.175843\\
1214	0.348222\\
1215	0.306021\\
1216	0.306001\\
1217	0.809518\\
1218	1.660652\\
1219	2.764377\\
1220	1.349112\\
1221	0.116587\\
1222	1.7e-05\\
1223	2.1e-05\\
1224	2e-06\\
1225	0\\
1226	0\\
1227	1e-06\\
1228	2e-06\\
1229	1e-06\\
1230	0\\
1231	6e-06\\
1232	0.780758\\
1233	0.073225\\
1234	8.8e-05\\
1235	0.306027\\
1236	0.305985\\
1237	1e-06\\
1238	8e-06\\
1239	3e-06\\
1240	3e-06\\
1241	1e-06\\
1242	5e-06\\
1243	2.281154\\
1244	0.772019\\
1245	3e-06\\
1246	0\\
1247	2e-06\\
1248	3e-06\\
1249	0\\
1250	2e-06\\
1251	0\\
1252	0\\
1253	0\\
1254	0\\
1255	2e-06\\
1256	0\\
1257	1e-06\\
1258	1e-06\\
1259	8e-06\\
1260	1e-06\\
1261	0\\
1262	1e-06\\
1263	1e-06\\
1264	1e-06\\
1265	1e-06\\
1266	9e-06\\
1267	0\\
1268	3e-06\\
1269	3e-06\\
1270	1e-06\\
1271	1e-06\\
1272	2e-06\\
1273	0\\
1274	0\\
1275	0\\
1276	2e-06\\
1277	1e-06\\
1278	3e-06\\
1279	4e-06\\
1280	2e-06\\
1281	1e-06\\
1282	2e-06\\
1283	2e-06\\
1284	0\\
1285	1e-06\\
1286	2e-06\\
1287	1e-06\\
1288	0\\
1289	0\\
1290	4e-06\\
1291	1e-06\\
1292	0\\
1293	0\\
1294	1e-06\\
1295	1e-06\\
1296	1e-06\\
1297	9e-06\\
1298	0\\
1299	3e-06\\
1300	0\\
1301	3e-06\\
1302	1e-06\\
1303	1.2e-05\\
1304	0\\
1305	1e-06\\
1306	0\\
1307	1.1e-05\\
1308	0\\
1309	0\\
1310	1e-06\\
1311	1e-06\\
1312	2e-06\\
1313	0\\
1314	1e-06\\
1315	0.305734\\
1316	0.312512\\
1317	5e-06\\
1318	2e-06\\
1319	0\\
1320	0\\
1321	1e-06\\
1322	6e-06\\
1323	0\\
1324	0\\
1325	0\\
1326	0\\
1327	8e-06\\
1328	1.7e-05\\
1329	1e-06\\
1330	3e-06\\
1331	6e-06\\
1332	1e-06\\
1333	1e-06\\
1334	1e-06\\
1335	4e-06\\
1336	0.000109\\
1337	6e-06\\
1338	0.498101\\
1339	1.74823\\
1340	1.455472\\
1341	1e-06\\
1342	2e-06\\
1343	0\\
1344	2e-06\\
1345	1e-05\\
1346	1e-06\\
1347	1e-06\\
1348	1e-06\\
1349	1e-06\\
1350	1e-06\\
1351	2.9e-05\\
1352	1.146255\\
1353	1.04349\\
1354	2.17768\\
1355	1.376336\\
1356	0.814649\\
1357	3e-06\\
1358	4e-05\\
1359	2e-06\\
1360	0.038946\\
1361	4e-06\\
1362	0.158223\\
1363	2.910001\\
1364	2.910001\\
1365	1.131842\\
1366	0.013413\\
1367	0.306018\\
1368	1.2e-05\\
1369	6e-06\\
1370	1e-06\\
1371	1e-06\\
1372	0\\
1373	0\\
1374	1e-06\\
1375	1e-06\\
1376	1e-06\\
1377	3e-06\\
1378	0\\
1379	3e-06\\
1380	2e-06\\
1381	3e-06\\
1382	0.302492\\
1383	0.410332\\
1384	0.578779\\
1385	0.814663\\
1386	1.920933\\
1387	2.91\\
1388	2.910001\\
1389	2.836604\\
1390	1.78486\\
1391	2.484364\\
1392	1.684186\\
1393	0\\
1394	1.3e-05\\
1395	1e-06\\
1396	1e-06\\
1397	1e-06\\
1398	2e-06\\
1399	1.436188\\
1400	2.459968\\
1401	1.596261\\
1402	2.540869\\
1403	2.910001\\
1404	3.328104\\
1405	3.378001\\
1406	3.578228\\
1407	3.358042\\
1408	3.357997\\
1409	3.072901\\
1410	3.283204\\
1411	3.281978\\
1412	2.964777\\
1413	2.910002\\
1414	1.194519\\
1415	2.909733\\
1416	3.120666\\
1417	0.000835\\
1418	1e-06\\
1419	2e-06\\
1420	1e-06\\
1421	0\\
1422	0\\
1423	6e-06\\
1424	1e-06\\
1425	2e-06\\
1426	3e-06\\
1427	1.1e-05\\
1428	3.2e-05\\
1429	0.437303\\
1430	1e-06\\
1431	6.4e-05\\
1432	3.3e-05\\
1433	1.7e-05\\
1434	0\\
1435	0.774585\\
1436	0.774602\\
1437	1.5e-05\\
1438	3e-06\\
1439	4e-06\\
1440	3e-06\\
1441	1e-06\\
1442	0\\
1443	0\\
1444	0\\
1445	1e-06\\
1446	1e-06\\
1447	1e-06\\
1448	2e-06\\
1449	0\\
1450	0\\
1451	0\\
1452	4e-06\\
1453	1e-06\\
1454	0\\
1455	0\\
1456	1e-06\\
1457	4e-06\\
1458	5e-06\\
1459	0\\
1460	4e-06\\
1461	0\\
1462	5e-06\\
1463	4e-06\\
1464	3e-06\\
1465	0\\
1466	6e-06\\
1467	5e-06\\
1468	0\\
1469	1e-06\\
1470	2e-06\\
1471	1.2e-05\\
1472	2.6e-05\\
1473	2.2e-05\\
1474	4e-05\\
1475	2.1e-05\\
1476	7e-05\\
1477	0\\
1478	3.6e-05\\
1479	1.4e-05\\
1480	2e-06\\
1481	5.3e-05\\
1482	1.478165\\
1483	1.737808\\
1484	2.384167\\
1485	0.883219\\
1486	0.000111\\
1487	0.980438\\
1488	1.091235\\
1489	1.4e-05\\
1490	1e-06\\
1491	0\\
1492	0\\
1493	4e-06\\
1494	2.9e-05\\
1495	2.683771\\
1496	3.12066\\
1497	2.766925\\
1498	3.017474\\
1499	4.050909\\
1500	4.051174\\
1501	3.697474\\
1502	3.697474\\
1503	3.697481\\
1504	4.070875\\
1505	4.123447\\
1506	4.736418\\
1507	4.736418\\
1508	4.736418\\
1509	4.166542\\
1510	3.214073\\
1511	4.185822\\
1512	3.877884\\
1513	2.5877\\
1514	0.343442\\
1515	2e-06\\
1516	1.8e-05\\
1517	6.6e-05\\
1518	0.117646\\
1519	3.120641\\
1520	4.206064\\
1521	3.702315\\
1522	3.120664\\
1523	3.055142\\
1524	2.766926\\
1525	2.227478\\
1526	1.960536\\
1527	1.977953\\
1528	2.766925\\
1529	2.766926\\
1530	3.476226\\
1531	3.120635\\
1532	4.413129\\
1533	4.736415\\
1534	3.211831\\
1535	3.912693\\
1536	3.192876\\
1537	1.881464\\
1538	2e-06\\
1539	1e-06\\
1540	1e-06\\
1541	0\\
1542	1e-06\\
1543	1.727188\\
1544	0.914534\\
1545	0.731007\\
1546	9.9e-05\\
1547	0.29096\\
1548	4e-06\\
1549	1e-06\\
1550	0\\
1551	1e-06\\
1552	1e-06\\
1553	0.000103\\
1554	2.011314\\
1555	2.495534\\
1556	2.869362\\
1557	3.120597\\
1558	1.08908\\
1559	1.465024\\
1560	0.625508\\
1561	2e-06\\
1562	0\\
1563	1e-06\\
1564	1e-06\\
1565	1e-06\\
1566	2e-06\\
1567	7e-06\\
1568	3e-06\\
1569	1.3e-05\\
1570	1e-06\\
1571	5.3e-05\\
1572	2e-06\\
1573	2e-06\\
1574	1e-06\\
1575	2.3e-05\\
1576	6.3e-05\\
1577	1e-06\\
1578	1.023692\\
1579	0.99249\\
1580	2.19495\\
1581	2.766925\\
1582	0.367735\\
1583	1.101701\\
1584	1.14167\\
1585	3.7e-05\\
1586	1e-06\\
1587	1e-06\\
1588	1e-06\\
1589	1e-06\\
1590	1e-06\\
1591	1e-06\\
1592	1e-06\\
1593	0\\
1594	0\\
1595	1e-06\\
1596	0\\
1597	1e-06\\
1598	0\\
1599	0\\
1600	0\\
1601	0\\
1602	0\\
1603	8e-06\\
1604	0\\
1605	0\\
1606	1e-06\\
1607	0\\
1608	9e-06\\
1609	1e-06\\
1610	0\\
1611	1e-06\\
1612	1e-06\\
1613	1e-06\\
1614	1e-06\\
1615	7e-06\\
1616	0\\
1617	6e-06\\
1618	7e-06\\
1619	1e-06\\
1620	1e-06\\
1621	1e-06\\
1622	1e-06\\
1623	4e-06\\
1624	1e-06\\
1625	1e-06\\
1626	0\\
1627	1e-06\\
1628	7e-06\\
1629	1e-06\\
1630	0\\
1631	1.1e-05\\
1632	1e-06\\
1633	2e-06\\
1634	0\\
1635	2e-06\\
1636	2e-06\\
1637	5e-06\\
1638	1e-06\\
1639	2.234604\\
1640	2.85992\\
1641	2.179978\\
1642	1.332821\\
1643	1.430584\\
1644	1.134551\\
1645	2e-06\\
1646	8e-06\\
1647	2e-06\\
1648	0.204573\\
1649	1.171384\\
1650	3.743967\\
1651	4.364325\\
1652	3.938352\\
1653	3.360317\\
1654	1.715918\\
1655	1.864914\\
1656	1.171424\\
1657	0\\
1658	0\\
1659	4e-06\\
1660	6e-06\\
1661	2e-06\\
1662	0\\
1663	6.1e-05\\
1664	0.094664\\
1665	0.772766\\
1666	1.167866\\
1667	3.229675\\
1668	2.654749\\
1669	0.222262\\
1670	1.9e-05\\
1671	0\\
1672	1e-06\\
1673	2.2e-05\\
1674	0.942998\\
1675	3.138931\\
1676	2.770973\\
1677	2.780109\\
1678	0.34391\\
1679	0.462832\\
1680	0.271862\\
1681	9e-06\\
1682	1e-05\\
1683	1e-06\\
1684	1e-06\\
1685	0\\
1686	0\\
1687	1.807783\\
1688	1.43971\\
1689	0.753695\\
1690	2.452803\\
1691	0.462838\\
1692	0.000879\\
1693	1e-06\\
1694	2e-06\\
1695	7e-06\\
1696	1.9e-05\\
1697	0.100722\\
1698	2.477711\\
1699	3.195315\\
1700	3.655672\\
1701	3.229665\\
1702	2.9026\\
1703	3.22968\\
1704	1.262968\\
1705	6e-06\\
1706	1e-06\\
1707	1e-06\\
1708	1.1e-05\\
1709	0\\
1710	0\\
1711	3.229651\\
1712	3.583404\\
1713	3.229666\\
1714	2.77903\\
1715	2.25964\\
1716	1.030485\\
1717	0.00086\\
1718	0.000188\\
1719	0.000173\\
1720	0.000148\\
1721	0.358324\\
1722	1.73109\\
1723	2.174436\\
1724	3.445358\\
1725	3.229666\\
1726	2.337951\\
1727	2.683367\\
1728	1.237358\\
1729	4e-06\\
1730	1e-06\\
1731	2e-06\\
1732	0\\
1733	2e-06\\
1734	1e-06\\
1735	1.906736\\
1736	2.104039\\
1737	1.396853\\
1738	0.254908\\
1739	1.237358\\
1740	0.506797\\
1741	5.3e-05\\
1742	1e-06\\
1743	1e-05\\
1744	2e-06\\
1745	2.9e-05\\
1746	1.237231\\
1747	2.621203\\
1748	3.22968\\
1749	2.046035\\
1750	3.229692\\
1751	3.229664\\
1752	2.837721\\
1753	1.2e-05\\
1754	1e-06\\
1755	0\\
1756	0\\
1757	0\\
1758	0\\
1759	3e-06\\
1760	0\\
1761	6e-06\\
1762	3e-05\\
1763	2.7e-05\\
1764	0.000639\\
1765	5.4e-05\\
1766	4e-06\\
1767	1e-06\\
1768	0\\
1769	2.1e-05\\
1770	2e-06\\
1771	0.290908\\
1772	1.251716\\
1773	0.291049\\
1774	5.8e-05\\
1775	1e-06\\
1776	1.1e-05\\
1777	0\\
1778	6e-06\\
1779	8e-06\\
1780	3e-06\\
1781	4e-06\\
1782	1e-06\\
1783	1e-06\\
1784	1e-06\\
1785	7e-06\\
1786	1e-06\\
1787	0\\
1788	0\\
1789	3e-06\\
1790	0\\
1791	1e-06\\
1792	1e-06\\
1793	1e-06\\
1794	3e-06\\
1795	3e-06\\
1796	0.777622\\
1797	0.774615\\
1798	1e-06\\
1799	5.1e-05\\
1800	4.2e-05\\
1801	0\\
1802	1e-06\\
1803	0\\
1804	2e-06\\
1805	1e-06\\
1806	0\\
1807	2.766925\\
1808	3.120815\\
1809	3.073706\\
1810	3.292861\\
1811	4.736418\\
1812	4.736418\\
1813	3.766922\\
1814	3.311323\\
1815	3.229304\\
1816	3.873433\\
1817	4.348276\\
1818	4.736418\\
1819	4.736418\\
1820	4.736418\\
1821	4.736418\\
1822	4.288881\\
1823	4.044247\\
1824	3.192899\\
1825	1.297269\\
1826	1.2e-05\\
1827	0\\
1828	2e-06\\
1829	0\\
1830	3e-06\\
1831	3.120649\\
1832	4.736417\\
1833	4.247956\\
1834	4.736416\\
1835	3.988804\\
1836	4.736417\\
1837	4.312165\\
1838	4.247681\\
1839	3.241846\\
1840	3.155934\\
1841	3.150678\\
1842	3.823715\\
1843	4.044211\\
1844	3.316512\\
1845	2.766929\\
1846	2.109406\\
1847	2.620474\\
1848	1.202422\\
1849	0\\
1850	0\\
1851	0\\
1852	6e-06\\
1853	7e-06\\
1854	4e-06\\
1855	2.766929\\
1856	4.377601\\
1857	2.766925\\
1858	2.652362\\
1859	4.466358\\
1860	4.244193\\
1861	3.211941\\
1862	3.092966\\
1863	2.766925\\
1864	2.843634\\
1865	3.192899\\
1866	4.736418\\
1867	4.25431\\
1868	4.736418\\
1869	4.736418\\
1870	4.736419\\
1871	3.597719\\
1872	2.852355\\
1873	1.576119\\
1874	4e-06\\
1875	2e-06\\
1876	4e-06\\
1877	1.5e-05\\
1878	1.7e-05\\
1879	0.040341\\
1880	2.127386\\
1881	4.3e-05\\
1882	3e-06\\
1883	3e-05\\
1884	1.1e-05\\
1885	0\\
1886	8e-06\\
1887	0\\
1888	9e-06\\
1889	7e-06\\
1890	1.8e-05\\
1891	6e-05\\
1892	0.774615\\
1893	2.304263\\
1894	0.77461\\
1895	0.517176\\
1896	7.6e-05\\
1897	0\\
1898	8e-06\\
1899	5e-06\\
1900	1e-06\\
1901	1e-06\\
1902	7e-06\\
1903	1.63212\\
1904	3.211928\\
1905	3.633837\\
1906	4.23914\\
1907	4.618266\\
1908	4.736418\\
1909	4.736418\\
1910	4.04396\\
1911	2.864875\\
1912	2.766925\\
1913	2.766925\\
1914	2.767138\\
1915	2.209592\\
1916	2.766935\\
1917	2.766924\\
1918	2.766925\\
1919	2.766925\\
1920	2.40766\\
1921	1e-05\\
1922	0\\
1923	0\\
1924	1e-06\\
1925	1e-06\\
1926	1e-06\\
1927	1e-06\\
1928	1e-06\\
1929	0\\
1930	1e-06\\
1931	1e-06\\
1932	1.8e-05\\
1933	0\\
1934	0\\
1935	0\\
1936	1e-06\\
1937	1.2e-05\\
1938	3e-06\\
1939	0\\
1940	6.6e-05\\
1941	1.7e-05\\
1942	2.7e-05\\
1943	2.3e-05\\
1944	1e-06\\
1945	0\\
1946	1e-06\\
1947	0\\
1948	1e-06\\
1949	0\\
1950	0\\
1951	0\\
1952	0\\
1953	1e-05\\
1954	1e-06\\
1955	0\\
1956	1e-06\\
1957	1e-06\\
1958	2e-06\\
1959	7e-06\\
1960	0\\
1961	0\\
1962	8e-06\\
1963	0.427517\\
1964	0.290959\\
1965	0.323756\\
1966	5e-06\\
1967	1e-06\\
1968	0.759362\\
1969	7e-06\\
1970	0\\
1971	1e-06\\
1972	1e-06\\
1973	0\\
1974	3e-06\\
1975	2.766935\\
1976	4.64358\\
1977	2.892299\\
1978	2.766925\\
1979	2.766925\\
1980	2.822834\\
1981	2.766926\\
1982	2.766926\\
1983	2.766907\\
1984	2.867983\\
1985	3.192898\\
1986	4.736414\\
1987	4.335996\\
1988	4.736419\\
1989	4.736418\\
1990	3.801943\\
1991	4.603886\\
1992	3.302205\\
1993	1.491037\\
1994	1.3e-05\\
1995	3e-06\\
1996	2.7e-05\\
1997	1e-06\\
1998	0.423996\\
1999	3.19292\\
2000	2.975045\\
2001	3.812145\\
2002	3.192897\\
2003	2.766925\\
2004	2.557748\\
2005	1.559786\\
2006	1.734382\\
2007	2.09597\\
2008	2.766925\\
2009	2.951325\\
2010	3.476758\\
2011	4.734325\\
2012	4.736408\\
2013	4.473075\\
2014	3.120619\\
2015	2.766925\\
2016	2.766926\\
2017	0.727059\\
2018	2.7e-05\\
2019	2e-06\\
2020	1e-06\\
2021	6e-06\\
2022	0.290949\\
2023	3.211881\\
2024	3.701726\\
2025	3.192886\\
2026	3.120639\\
2027	3.192898\\
2028	3.192997\\
2029	3.099167\\
2030	4.736419\\
2031	4.736419\\
2032	4.783027\\
2033	5.595124\\
2034	6.735232\\
2035	6.856049\\
2036	8.962555\\
2037	8.952663\\
2038	8.236019\\
2039	8.382214\\
2040	8.955037\\
2041	7.797634\\
2042	5.865759\\
2043	5.666968\\
2044	5.666966\\
2045	4.736418\\
2046	5.115018\\
2047	7.319515\\
2048	8.059062\\
2049	7.910881\\
2050	7.851227\\
2051	7.87607\\
2052	7.347405\\
2053	6.88949\\
2054	7.263303\\
2055	7.319512\\
2056	7.319515\\
2057	7.791801\\
2058	9.221182\\
2059	12.111623\\
2060	10.979403\\
2061	10.151731\\
2062	8.122239\\
2063	8.663571\\
2064	7.87607\\
2065	6.796293\\
2066	5.197275\\
2067	4.736418\\
2068	4.736418\\
2069	4.736418\\
2070	5.739515\\
2071	7.66187\\
2072	7.869636\\
2073	7.477012\\
2074	7.454123\\
2075	7.319515\\
2076	8.250063\\
2077	8.250061\\
2078	8.250064\\
2079	7.17343\\
2080	7.148995\\
2081	7.296012\\
2082	7.319515\\
2083	8.160263\\
2084	7.319524\\
2085	7.509736\\
2086	7.319515\\
2087	7.319515\\
2088	6.871417\\
2089	4.736418\\
2090	4.631078\\
2091	3.211915\\
2092	3.120648\\
2093	3.187032\\
2094	3.211974\\
2095	3.948583\\
2096	4.454762\\
2097	4.736418\\
2098	4.736418\\
2099	4.171027\\
2100	3.296123\\
2101	3.120635\\
2102	2.766925\\
2103	2.603943\\
2104	2.766925\\
2105	3.211902\\
2106	4.736418\\
2107	4.736418\\
2108	4.736418\\
2109	5.066392\\
2110	4.736418\\
2111	4.736418\\
2112	4.736418\\
2113	3.539816\\
2114	2.980858\\
2115	2.766925\\
2116	2.363928\\
2117	2.766914\\
2118	2.766925\\
2119	2.766925\\
2120	3.120638\\
2121	3.299113\\
2122	3.192898\\
2123	2.874625\\
2124	2.766925\\
2125	1.437761\\
2126	0.516268\\
2127	0.719873\\
2128	1.364898\\
2129	2.766927\\
2130	3.224712\\
2131	4.736418\\
2132	4.736418\\
2133	4.736418\\
2134	4.736418\\
2135	4.736418\\
2136	4.044966\\
2137	3.211979\\
2138	3.098977\\
2139	3.04731\\
2140	3.211913\\
2141	4.736418\\
2142	6.539734\\
2143	8.96254\\
2144	7.319515\\
2145	7.319519\\
2146	7.319519\\
2147	6.379967\\
2148	4.73643\\
2149	4.736418\\
2150	4.736418\\
2151	4.736917\\
2152	5.567349\\
2153	6.983056\\
2154	7.319515\\
2155	7.319516\\
2156	9.221182\\
2157	7.332759\\
2158	7.319517\\
2159	6.206406\\
2160	7.973345\\
2161	7.474611\\
2162	8.129961\\
2163	7.7875\\
2164	8.130012\\
2165	7.474593\\
2166	10.053835\\
2167	10.822495\\
2168	9.245077\\
2169	8.918103\\
2170	9.682344\\
2171	9.528343\\
2172	8.945967\\
2173	8.94698\\
2174	8.946202\\
2175	8.946997\\
2176	9.198503\\
2177	10.971625\\
2178	12.19851\\
2179	10.952136\\
2180	11.868064\\
2181	11.901236\\
2182	11.85788\\
2183	10.075113\\
2184	7.900633\\
2185	7.179101\\
2186	6.093014\\
2187	5.594659\\
2188	5.587513\\
2189	6.96548\\
2190	9.17878\\
2191	10.922583\\
2192	10.247325\\
2193	10.025746\\
2194	10.056024\\
2195	9.365257\\
2196	9.17877\\
2197	8.946992\\
2198	8.946996\\
2199	8.822735\\
2200	8.946996\\
2201	9.189145\\
2202	10.488523\\
2203	10.730427\\
2204	10.898295\\
2205	11.154769\\
2206	10.341382\\
2207	8.946999\\
2208	7.474595\\
2209	6.475757\\
2210	5.170483\\
2211	5.159796\\
2212	5.159809\\
2213	6.270765\\
2214	8.193929\\
2215	9.769932\\
2216	9.322064\\
2217	9.178765\\
2218	8.947032\\
2219	9.177426\\
2220	8.688332\\
2221	8.735857\\
2222	8.659662\\
2223	8.422666\\
2224	8.946996\\
2225	10.121121\\
2226	11.985571\\
2227	13.71412\\
2228	12.324093\\
2229	11.433752\\
2230	10.270278\\
2231	9.199329\\
2232	7.474593\\
2233	6.783195\\
2234	5.719895\\
2235	5.702534\\
2236	6.247607\\
2237	7.474602\\
2238	9.17878\\
2239	11.95536\\
2240	13.714117\\
2241	13.714131\\
2242	14.773528\\
2243	17.699593\\
2244	11.719869\\
2245	10.559524\\
2246	9.792125\\
2247	9.178777\\
2248	9.178765\\
2249	10.493844\\
2250	11.167941\\
2251	9.779215\\
2252	10.278418\\
2253	9.178767\\
2254	10.256929\\
2255	9.302962\\
2256	7.474594\\
2257	6.477089\\
2258	5.555496\\
2259	5.159771\\
2260	5.160017\\
2261	5.572781\\
2262	6.477162\\
2263	7.28297\\
2264	7.4746\\
2265	7.474599\\
2266	7.474598\\
2267	7.638656\\
2268	7.474593\\
2269	6.775641\\
2270	6.219779\\
2271	5.944769\\
2272	6.106729\\
2273	6.477087\\
2274	7.474594\\
2275	7.474597\\
2276	7.474593\\
2277	7.900636\\
2278	7.474593\\
2279	7.474593\\
2280	6.379508\\
2281	5.159784\\
2282	5.159794\\
2283	4.771994\\
2284	4.312289\\
2285	4.647964\\
2286	5.159792\\
2287	4.822592\\
2288	5.159794\\
2289	5.159793\\
2290	5.15979\\
2291	5.159927\\
2292	5.745486\\
2293	5.159772\\
2294	5.159782\\
2295	5.159788\\
2296	5.159787\\
2297	5.189724\\
2298	6.477087\\
2299	7.032287\\
2300	7.474594\\
2301	7.474593\\
2302	7.21339\\
2303	6.477088\\
2304	5.159919\\
2305	5.159791\\
2306	5.158872\\
2307	4.984638\\
2308	5.159794\\
2309	5.159771\\
2310	7.474593\\
2311	7.97311\\
2312	9.178785\\
2313	9.378528\\
2314	9.93342\\
2315	9.713376\\
2316	8.946963\\
2317	8.946994\\
2318	8.885535\\
2319	8.940177\\
2320	8.940136\\
2321	8.965064\\
2322	8.940164\\
2323	8.798845\\
2324	8.946994\\
2325	9.211215\\
2326	9.178622\\
2327	7.973423\\
2328	6.477149\\
2329	5.159777\\
2330	5.15978\\
2331	5.159781\\
2332	5.159778\\
2333	6.247657\\
2334	7.870852\\
2335	8.579695\\
2336	9.08729\\
2337	8.947048\\
2338	8.940177\\
2339	8.947049\\
2340	7.973383\\
2341	7.474656\\
2342	7.474654\\
2343	7.639158\\
2344	8.108558\\
2345	8.998629\\
2346	8.28505\\
2347	8.946994\\
2348	11.742396\\
2349	11.930724\\
2350	12.149589\\
2351	10.523778\\
2352	8.834697\\
2353	7.609105\\
2354	7.501845\\
2355	7.104364\\
2356	7.580045\\
2357	8.274854\\
2358	10.983621\\
2359	11.867426\\
2360	12.410987\\
2361	12.446904\\
2362	11.28143\\
2363	10.931555\\
2364	9.081505\\
2365	9.081504\\
2366	8.337301\\
2367	8.400857\\
2368	9.081434\\
2369	9.42778\\
2370	10.987837\\
2371	10.889569\\
2372	10.333269\\
2373	11.594555\\
2374	11.180922\\
2375	10.907469\\
2376	9.081426\\
2377	7.609105\\
2378	7.609105\\
2379	7.411295\\
2380	7.609104\\
2381	8.087956\\
2382	11.136562\\
2383	13.848627\\
2384	13.848627\\
2385	12.444501\\
2386	11.856335\\
2387	12.223453\\
2388	9.671617\\
2389	9.844599\\
2390	9.549415\\
2391	9.667179\\
2392	10.049922\\
2393	11.636503\\
2394	12.832116\\
2395	11.128998\\
2396	11.774664\\
2397	12.316384\\
2398	11.85309\\
2399	11.009543\\
2400	8.284803\\
2401	7.474593\\
2402	6.922947\\
2403	6.808551\\
2404	7.016935\\
2405	7.496596\\
2406	9.292349\\
2407	11.067072\\
2408	11.948898\\
2409	11.593132\\
2410	11.292228\\
2411	10.765686\\
2412	9.468356\\
2413	8.947\\
2414	8.940174\\
2415	8.947005\\
2416	8.947002\\
2417	9.178768\\
2418	9.621545\\
2419	9.178768\\
2420	9.369398\\
2421	10.025947\\
2422	10.101548\\
2423	8.963708\\
2424	7.553584\\
2425	7.474593\\
2426	6.477108\\
2427	6.277154\\
2428	6.417149\\
2429	6.477104\\
2430	6.969409\\
2431	7.474595\\
2432	7.474595\\
2433	8.002284\\
2434	7.810061\\
2435	7.474596\\
2436	6.477088\\
2437	5.390733\\
2438	5.1598\\
2439	5.159813\\
2440	5.704022\\
2441	6.719911\\
2442	7.474595\\
2443	7.474593\\
2444	7.474593\\
2445	7.474593\\
2446	7.474592\\
2447	7.474597\\
2448	4.780425\\
2449	3.829476\\
2450	3.432931\\
2451	2.970977\\
2452	2.820556\\
2453	3.419645\\
2454	3.529084\\
2455	3.52901\\
2456	4.164958\\
2457	4.101374\\
2458	3.910767\\
2459	4.093421\\
2460	3.911014\\
2461	3.528957\\
2462	1.975595\\
2463	1.743849\\
2464	2.182976\\
2465	3.529005\\
2466	4.782102\\
2467	5.159798\\
2468	5.159919\\
2469	5.505218\\
2470	5.159859\\
2471	5.159794\\
2472	3.846286\\
2473	3.366205\\
2474	2.447841\\
2475	1.800534\\
2476	2.156054\\
2477	3.529346\\
2478	5.159789\\
2479	5.698795\\
2480	6.477087\\
2481	6.477095\\
2482	6.477095\\
2483	6.474087\\
2484	5.591095\\
2485	5.572773\\
2486	5.159831\\
2487	5.159827\\
2488	5.15983\\
2489	6.24758\\
2490	6.83884\\
2491	7.4747\\
2492	7.352001\\
2493	7.474719\\
2494	7.47472\\
2495	7.474593\\
2496	5.568752\\
2497	5.159769\\
2498	5.159799\\
2499	5.159814\\
2500	5.159774\\
2501	5.416695\\
2502	7.474709\\
2503	8.065196\\
2504	8.940191\\
2505	8.62616\\
2506	7.879397\\
2507	7.474593\\
2508	7.474593\\
2509	7.321941\\
2510	6.840744\\
2511	6.543581\\
2512	6.477129\\
2513	7.321788\\
2514	7.474592\\
2515	7.865317\\
2516	7.888431\\
2517	7.973353\\
2518	8.712692\\
2519	7.972762\\
2520	6.96928\\
2521	5.579549\\
2522	5.159777\\
2523	5.159774\\
2524	5.159902\\
2525	6.301206\\
2526	7.708042\\
2527	8.10855\\
2528	7.487672\\
2529	7.474702\\
2530	7.474713\\
2531	7.047706\\
2532	6.477104\\
2533	5.430083\\
2534	5.458207\\
2535	5.625287\\
2536	6.354957\\
2537	7.34972\\
2538	7.695688\\
2539	7.973353\\
2540	8.148821\\
2541	7.973345\\
2542	8.607623\\
2543	7.973348\\
2544	6.582385\\
2545	5.159787\\
2546	5.159825\\
2547	5.159813\\
2548	5.159789\\
2549	5.159773\\
2550	6.694029\\
2551	6.95705\\
2552	7.474686\\
2553	7.474669\\
2554	6.969593\\
2555	7.20907\\
2556	5.680408\\
2557	5.159777\\
2558	5.159807\\
2559	5.159773\\
2560	5.159819\\
2561	6.312878\\
2562	7.010063\\
2563	7.474729\\
2564	7.474702\\
2565	7.474709\\
2566	7.474592\\
2567	7.474715\\
2568	5.843899\\
2569	5.159817\\
2570	5.159758\\
2571	4.718661\\
2572	4.78323\\
2573	5.159782\\
2574	6.262244\\
2575	6.477139\\
2576	7.103542\\
2577	6.709515\\
2578	6.477562\\
2579	6.477092\\
2580	6.104135\\
2581	5.60831\\
2582	5.159816\\
2583	5.159772\\
2584	5.15977\\
2585	5.159863\\
2586	6.221983\\
2587	6.247761\\
2588	5.572758\\
2589	6.477087\\
2590	7.464257\\
2591	7.321787\\
2592	5.159808\\
2593	5.159797\\
2594	4.306693\\
2595	3.923026\\
2596	4.299473\\
2597	4.610871\\
2598	4.671048\\
2599	5.034754\\
2600	5.15982\\
2601	5.159781\\
2602	5.15978\\
2603	5.159835\\
2604	5.159805\\
2605	4.238774\\
2606	3.529255\\
2607	3.274663\\
2608	3.529197\\
2609	3.911006\\
2610	5.128325\\
2611	5.159781\\
2612	5.159767\\
2613	4.992715\\
2614	4.777149\\
2615	4.783217\\
2616	3.846256\\
2617	2.678757\\
2618	1.310196\\
2619	1.062689\\
2620	0.823751\\
2621	1.062689\\
2622	1.493245\\
2623	1.559363\\
2624	3.304469\\
2625	3.529209\\
2626	3.529223\\
2627	3.529075\\
2628	3.529097\\
2629	3.443469\\
2630	2.380925\\
2631	1.743915\\
2632	1.743921\\
2633	3.529236\\
2634	4.168936\\
2635	5.159797\\
2636	5.159818\\
2637	5.159905\\
2638	5.1598\\
2639	5.159816\\
2640	5.096732\\
2641	3.910954\\
2642	3.529292\\
2643	3.072768\\
2644	3.09329\\
2645	3.529325\\
2646	3.528998\\
2647	3.528222\\
2648	3.529\\
2649	3.528973\\
2650	3.635745\\
2651	3.529227\\
2652	3.508631\\
2653	1.860648\\
2654	2.252122\\
2655	2.456721\\
2656	3.529302\\
2657	4.672582\\
2658	5.159884\\
2659	5.159795\\
2660	5.159903\\
2661	5.159797\\
2662	5.159792\\
2663	5.159952\\
2664	5.159779\\
2665	4.31392\\
2666	3.82871\\
2667	3.624408\\
2668	3.910297\\
2669	5.159792\\
2670	6.678662\\
2671	7.973357\\
2672	7.953136\\
2673	7.474606\\
2674	7.4746\\
2675	7.474695\\
2676	6.955327\\
2677	6.86826\\
2678	6.969549\\
2679	6.882028\\
2680	7.474612\\
2681	7.4761\\
2682	8.104195\\
2683	8.411595\\
2684	7.781323\\
2685	7.936699\\
2686	8.223408\\
2687	7.474596\\
2688	6.440197\\
2689	5.159832\\
2690	5.159791\\
2691	5.159792\\
2692	5.159784\\
2693	5.289869\\
2694	7.474592\\
2695	7.798865\\
2696	7.476689\\
2697	7.4746\\
2698	7.474694\\
2699	7.474594\\
2700	7.47462\\
2701	7.474617\\
2702	7.474701\\
2703	7.474594\\
2704	7.474619\\
2705	7.656785\\
2706	8.649962\\
2707	8.690628\\
2708	7.886934\\
2709	8.946997\\
2710	8.940176\\
2711	7.474595\\
2712	6.193821\\
2713	5.159831\\
2714	5.159798\\
2715	5.159813\\
2716	5.159806\\
2717	5.15979\\
2718	7.474729\\
2719	8.707427\\
2720	8.947006\\
2721	9.178808\\
2722	8.947072\\
2723	9.178771\\
2724	8.940171\\
2725	8.66654\\
2726	7.973598\\
2727	8.249706\\
2728	8.348402\\
2729	8.821179\\
2730	8.947005\\
2731	8.433132\\
2732	7.906816\\
2733	7.475235\\
2734	7.474658\\
2735	6.4771\\
2736	5.159798\\
2737	4.191186\\
2738	3.765113\\
2739	3.5292\\
2740	3.559739\\
2741	4.666575\\
2742	5.159794\\
2743	6.806059\\
2744	6.477067\\
2745	6.696496\\
2746	5.614896\\
2747	5.1598\\
2748	5.159806\\
2749	5.159784\\
2750	5.159861\\
2751	5.159909\\
2752	5.321027\\
2753	6.360502\\
2754	7.474697\\
2755	7.474636\\
2756	7.148756\\
2757	7.322411\\
2758	6.736143\\
2759	6.247601\\
2760	5.159806\\
2761	4.1411\\
2762	3.528963\\
2763	3.528626\\
2764	3.529203\\
2765	3.529081\\
2766	3.528931\\
2767	3.603413\\
2768	4.193073\\
2769	4.605946\\
2770	4.006687\\
2771	3.909849\\
2772	3.528337\\
2773	3.528103\\
2774	3.44172\\
2775	3.528936\\
2776	3.910915\\
2777	5.159297\\
2778	5.159801\\
2779	5.159927\\
2780	5.159814\\
2781	5.159914\\
2782	5.159794\\
2783	5.159769\\
2784	3.799668\\
2785	2.850044\\
2786	1.432795\\
2787	1.148549\\
2788	1.063086\\
2789	1.17152\\
2790	1.097702\\
2791	1.310482\\
2792	2.107\\
2793	3.137138\\
2794	3.272159\\
2795	3.529184\\
2796	3.529192\\
2797	2.445956\\
2798	2.170719\\
2799	2.158647\\
2800	2.596685\\
2801	3.529202\\
2802	5.194881\\
2803	5.815186\\
2804	5.815185\\
2805	5.815187\\
2806	5.815187\\
2807	5.815187\\
2808	4.469705\\
2809	4.050239\\
2810	3.862168\\
2811	3.508919\\
2812	4.050236\\
2813	5.273307\\
2814	6.31159\\
2815	7.976688\\
2816	8.130012\\
2817	7.311792\\
2818	7.331655\\
2819	8.130012\\
2820	8.130012\\
2821	8.130012\\
2822	8.130012\\
2823	8.130012\\
2824	8.130014\\
2825	8.271963\\
2826	8.628769\\
2827	8.976253\\
2828	8.611862\\
2829	8.130012\\
2830	8.355127\\
2831	8.130011\\
2832	6.065355\\
2833	5.815185\\
2834	5.815185\\
2835	5.811802\\
2836	5.815185\\
2837	5.815185\\
2838	7.132506\\
2839	8.130012\\
2840	8.773402\\
2841	9.59558\\
2842	9.078544\\
2843	9.602412\\
2844	9.289149\\
2845	9.602412\\
2846	9.602412\\
2847	9.595589\\
2848	9.438169\\
2849	7.973353\\
2850	8.108444\\
2851	7.474593\\
2852	7.474614\\
2853	7.474611\\
2854	7.474594\\
2855	6.477118\\
2856	5.159773\\
2857	5.159792\\
2858	4.782885\\
2859	4.668217\\
2860	4.782809\\
2861	5.159786\\
2862	6.477118\\
2863	7.474598\\
2864	7.638771\\
2865	7.973352\\
2866	7.973455\\
2867	7.620454\\
2868	7.474599\\
2869	7.474598\\
2870	6.915391\\
2871	6.477117\\
2872	6.477116\\
2873	7.034606\\
2874	7.41093\\
2875	7.299585\\
2876	6.891236\\
2877	7.474602\\
2878	7.474612\\
2879	6.791847\\
2880	5.608819\\
2881	4.804204\\
2882	4.802731\\
2883	4.714414\\
2884	4.376137\\
2885	4.376054\\
2886	3.946058\\
2887	3.662366\\
2888	3.716928\\
2889	4.196369\\
2890	4.426675\\
2891	4.802526\\
2892	4.803659\\
2893	4.544061\\
2894	4.801918\\
2895	4.801767\\
2896	4.782466\\
2897	4.789108\\
2898	5.774321\\
2899	5.989181\\
2900	5.931232\\
2901	6.123433\\
2902	6.923258\\
2903	6.442838\\
2904	4.789588\\
2905	4.655829\\
2906	3.671467\\
2907	3.601786\\
2908	3.601787\\
2909	4.015909\\
2910	4.785885\\
2911	4.802985\\
2912	4.803765\\
2913	4.804249\\
2914	4.803563\\
2915	4.802538\\
2916	4.583698\\
2917	3.503987\\
2918	3.304814\\
2919	3.304839\\
2920	3.395265\\
2921	4.384949\\
2922	4.800864\\
2923	4.803368\\
2924	4.803835\\
2925	4.804056\\
2926	4.802152\\
2927	4.804227\\
2928	4.543778\\
2929	3.304799\\
2930	3.29208\\
2931	3.179269\\
2932	3.179218\\
2933	3.304826\\
2934	3.304815\\
2935	3.662409\\
2936	4.544379\\
2937	4.801259\\
2938	4.543367\\
2939	3.678379\\
2940	3.601784\\
2941	3.30482\\
2942	3.30485\\
2943	3.30485\\
2944	3.3048\\
2945	4.395805\\
2946	4.803864\\
2947	4.804213\\
2948	4.80305\\
2949	4.803637\\
2950	4.803993\\
2951	4.803667\\
2952	4.804227\\
2953	4.552819\\
2954	3.725727\\
2955	3.662418\\
2956	3.662417\\
2957	3.756822\\
2958	3.662415\\
2959	4.00547\\
2960	4.033527\\
2961	3.773348\\
2962	3.662346\\
2963	3.304819\\
2964	3.179232\\
2965	2.251599\\
2966	1.465566\\
2967	1.632872\\
2968	2.939892\\
2969	3.66236\\
2970	4.803921\\
2971	4.787457\\
2972	4.914112\\
2973	4.787166\\
2974	5.246982\\
2975	4.788831\\
2976	4.803254\\
2977	4.035988\\
2978	3.413443\\
2979	3.304802\\
2980	3.662404\\
2981	4.790701\\
2982	5.989222\\
2983	7.319737\\
2984	6.923258\\
2985	7.125398\\
2986	6.484172\\
2987	5.709287\\
2988	4.79096\\
2989	4.790554\\
2990	4.812198\\
2991	4.803021\\
2992	5.757757\\
2993	6.923278\\
2994	6.954151\\
2995	7.390306\\
2996	6.923258\\
2997	6.923258\\
2998	6.923259\\
2999	5.989229\\
3000	4.786617\\
3001	3.758676\\
3002	3.431251\\
3003	3.304798\\
3004	3.601787\\
3005	4.764043\\
3006	4.78901\\
3007	6.923258\\
3008	7.076994\\
3009	7.948162\\
3010	7.621828\\
3011	7.503799\\
3012	6.923327\\
3013	6.923283\\
3014	6.3294\\
3015	6.435615\\
3016	6.782568\\
3017	6.923258\\
3018	6.923323\\
3019	7.233145\\
3020	6.923259\\
3021	6.923261\\
3022	6.92326\\
3023	6.893131\\
3024	4.809893\\
3025	4.814544\\
3026	4.781173\\
3027	4.78496\\
3028	4.814651\\
3029	4.811028\\
3030	5.992333\\
3031	7.32754\\
3032	7.894191\\
3033	7.517588\\
3034	7.322222\\
3035	7.39046\\
3036	7.100845\\
3037	6.976686\\
3038	6.923261\\
3039	6.923259\\
3040	6.780274\\
3041	6.923259\\
3042	6.923259\\
3043	7.1636\\
3044	6.923259\\
3045	6.923262\\
3046	7.441749\\
3047	6.923262\\
3048	4.811174\\
3049	4.811736\\
3050	4.777111\\
3051	4.476251\\
3052	4.782767\\
3053	4.815171\\
3054	6.172817\\
3055	6.92326\\
3056	6.92326\\
3057	6.923271\\
3058	7.775666\\
3059	8.302057\\
3060	7.637353\\
3061	8.291518\\
3062	8.200122\\
3063	8.295677\\
3064	8.295679\\
3065	8.302113\\
3066	8.302112\\
3067	7.362089\\
3068	6.92326\\
3069	6.909911\\
3070	6.78853\\
3071	5.989164\\
3072	4.812617\\
3073	4.781261\\
3074	4.089512\\
3075	4.149898\\
3076	4.150313\\
3077	4.811845\\
3078	5.593978\\
3079	6.958129\\
3080	7.04574\\
3081	6.923276\\
3082	6.92326\\
3083	6.92326\\
3084	5.989166\\
3085	4.862361\\
3086	4.817002\\
3087	4.817713\\
3088	4.817438\\
3089	4.789141\\
3090	5.142548\\
3091	5.293617\\
3092	4.803995\\
3093	4.804409\\
3094	6.009025\\
3095	5.796029\\
3096	4.798834\\
3097	4.816348\\
3098	4.767413\\
3099	4.165825\\
3100	4.105654\\
3101	4.150142\\
3102	4.165809\\
3103	4.781289\\
3104	4.813448\\
3105	5.142625\\
3106	5.774948\\
3107	5.10174\\
3108	4.806228\\
3109	4.805317\\
3110	4.805046\\
3111	4.806129\\
3112	4.014282\\
3113	4.250115\\
3114	4.614943\\
3115	4.543752\\
3116	3.662418\\
3117	3.677684\\
3118	4.30482\\
3119	3.723353\\
3120	3.304835\\
3121	1.633016\\
3122	0.301087\\
3123	9e-06\\
3124	0\\
3125	4.5e-05\\
3126	2e-06\\
3127	6e-06\\
3128	0.510113\\
3129	0.881688\\
3130	0.768894\\
3131	1.198699\\
3132	0.6202\\
3133	0.145743\\
3134	6e-05\\
3135	0\\
3136	1e-06\\
3137	1.200232\\
3138	3.179395\\
3139	3.300934\\
3140	3.30497\\
3141	3.304971\\
3142	3.644564\\
3143	3.304971\\
3144	2.386719\\
3145	2.058271\\
3146	1.214216\\
3147	1.092974\\
3148	2.090614\\
3149	3.304967\\
3150	4.957722\\
3151	4.957722\\
3152	5.962266\\
3153	5.976398\\
3154	6.604028\\
3155	6.648254\\
3156	5.557838\\
3157	6.691207\\
3158	6.380725\\
3159	6.661685\\
3160	7.125394\\
3161	7.125404\\
3162	7.27957\\
3163	7.838782\\
3164	8.504195\\
3165	8.497812\\
3166	8.504196\\
3167	7.718616\\
3168	6.164203\\
3169	4.804663\\
3170	4.803579\\
3171	4.803278\\
3172	4.803584\\
3173	4.802875\\
3174	6.62434\\
3175	6.923261\\
3176	8.302061\\
3177	8.302108\\
3178	7.082409\\
3179	7.39031\\
3180	8.30206\\
3181	7.228062\\
3182	6.923258\\
3183	7.104306\\
3184	7.076706\\
3185	6.923258\\
3186	7.808202\\
3187	7.516906\\
3188	7.322217\\
3189	7.318822\\
3190	8.302048\\
3191	7.334515\\
3192	6.296827\\
3193	4.786386\\
3194	4.804396\\
3195	4.803626\\
3196	4.804229\\
3197	5.62562\\
3198	6.923258\\
3199	8.295684\\
3200	8.044965\\
3201	7.076992\\
3202	7.390316\\
3203	7.322219\\
3204	7.076976\\
3205	6.923258\\
3206	6.923269\\
3207	6.790328\\
3208	6.470656\\
3209	6.923353\\
3210	6.923258\\
3211	6.923258\\
3212	6.923258\\
3213	6.923258\\
3214	7.70948\\
3215	6.994835\\
3216	5.232307\\
3217	4.804935\\
3218	4.803722\\
3219	4.803669\\
3220	4.804099\\
3221	4.924247\\
3222	6.923261\\
3223	7.390317\\
3224	8.081892\\
3225	8.622382\\
3226	7.810362\\
3227	7.379793\\
3228	6.596843\\
3229	5.500469\\
3230	5.15733\\
3231	5.142602\\
3232	5.989245\\
3233	6.923264\\
3234	6.923459\\
3235	7.249767\\
3236	6.923264\\
3237	6.923342\\
3238	7.762575\\
3239	6.923265\\
3240	5.69694\\
3241	4.804802\\
3242	4.803589\\
3243	4.803435\\
3244	4.803916\\
3245	4.786159\\
3246	6.644795\\
3247	7.48288\\
3248	7.390306\\
3249	6.977362\\
3250	6.923258\\
3251	6.780167\\
3252	5.552891\\
3253	4.843745\\
3254	4.786202\\
3255	4.804907\\
3256	4.785518\\
3257	5.958899\\
3258	6.780157\\
3259	6.923258\\
3260	6.911617\\
3261	6.646666\\
3262	6.92327\\
3263	6.923268\\
3264	6.192057\\
3265	4.957726\\
3266	4.957923\\
3267	4.95784\\
3268	4.957905\\
3269	4.95791\\
3270	4.957897\\
3271	4.957733\\
3272	4.995506\\
3273	5.344463\\
3274	4.957741\\
3275	4.957837\\
3276	4.957874\\
3277	3.986982\\
3278	3.599647\\
3279	3.6018\\
3280	4.547561\\
3281	4.957839\\
3282	4.960404\\
3283	5.976435\\
3284	6.055622\\
3285	6.191302\\
3286	5.976394\\
3287	6.191304\\
3288	4.957726\\
3289	4.357132\\
3290	3.601791\\
3291	3.305006\\
3292	3.304815\\
3293	3.304826\\
3294	3.304841\\
3295	3.30481\\
3296	3.379591\\
3297	3.304799\\
3298	3.304811\\
3299	3.304843\\
3300	3.08763\\
3301	2.250743\\
3302	1.788647\\
3303	1.633053\\
3304	2.45136\\
3305	3.304831\\
3306	4.326681\\
3307	4.802145\\
3308	4.803205\\
3309	4.803772\\
3310	4.804448\\
3311	4.803825\\
3312	3.678375\\
3313	3.304818\\
3314	3.179283\\
3315	3.09823\\
3316	3.304823\\
3317	3.678377\\
3318	4.804446\\
3319	5.957691\\
3320	6.780171\\
3321	6.785425\\
3322	6.923258\\
3323	6.923258\\
3324	5.989166\\
3325	5.989173\\
3326	5.989166\\
3327	5.989333\\
3328	6.0451\\
3329	6.923342\\
3330	6.923258\\
3331	6.923258\\
3332	6.773169\\
3333	5.989185\\
3334	6.781624\\
3335	5.774338\\
3336	4.957726\\
3337	4.957845\\
3338	3.895739\\
3339	3.675119\\
3340	4.212114\\
3341	4.957749\\
3342	5.219405\\
3343	7.059987\\
3344	7.125397\\
3345	7.125398\\
3346	7.125399\\
3347	7.125396\\
3348	6.585484\\
3349	6.654419\\
3350	6.915926\\
3351	7.125396\\
3352	7.125398\\
3353	7.634723\\
3354	8.485259\\
3355	8.504196\\
3356	7.753183\\
3357	8.497811\\
3358	7.699381\\
3359	7.1254\\
3360	6.651889\\
3361	5.367864\\
3362	4.957743\\
3363	4.957722\\
3364	4.957722\\
3365	4.957727\\
3366	6.981281\\
3367	7.125397\\
3368	7.682677\\
3369	8.504196\\
3370	7.69043\\
3371	8.579625\\
3372	8.504196\\
3373	8.240146\\
3374	8.439461\\
3375	8.504196\\
3376	8.500617\\
3377	8.497814\\
3378	8.42084\\
3379	7.906003\\
3380	7.200235\\
3381	7.279129\\
3382	7.592444\\
3383	7.125396\\
3384	4.957902\\
3385	3.179239\\
3386	2.250743\\
3387	2.250743\\
3388	2.825472\\
3389	3.304816\\
3390	4.629264\\
3391	4.852061\\
3392	5.928901\\
3393	5.989166\\
3394	5.989039\\
3395	5.233641\\
3396	4.785974\\
3397	4.963777\\
3398	5.147721\\
3399	5.390226\\
3400	5.989182\\
3401	6.481809\\
3402	6.92326\\
3403	6.899832\\
3404	6.065181\\
3405	6.913465\\
3406	6.923258\\
3407	5.989348\\
3408	4.804118\\
3409	4.406978\\
3410	3.601785\\
3411	3.545169\\
3412	3.678337\\
3413	4.544353\\
3414	4.786714\\
3415	6.923269\\
3416	6.859377\\
3417	6.923261\\
3418	6.923262\\
3419	6.923263\\
3420	6.748286\\
3421	6.872406\\
3422	6.194608\\
3423	5.989186\\
3424	5.989166\\
3425	6.92327\\
3426	6.747462\\
3427	6.394532\\
3428	5.777075\\
3429	4.853098\\
3430	5.924741\\
3431	5.514718\\
3432	4.79513\\
3433	4.798583\\
3434	3.828218\\
3435	3.311588\\
3436	3.30481\\
3437	3.304694\\
3438	3.304689\\
3439	3.452679\\
3440	4.545312\\
3441	4.545288\\
3442	4.375994\\
3443	3.718811\\
3444	3.662416\\
3445	3.3047\\
3446	2.250757\\
3447	2.250754\\
3448	3.304694\\
3449	3.678518\\
3450	4.376188\\
3451	4.795301\\
3452	4.795007\\
3453	4.271663\\
3454	4.795659\\
3455	3.678369\\
3456	3.678377\\
3457	3.304792\\
3458	3.10775\\
3459	2.892088\\
3460	3.005818\\
3461	2.663278\\
3462	2.884155\\
3463	3.304711\\
3464	3.306948\\
3465	3.600667\\
3466	3.304788\\
3467	3.282566\\
3468	2.91092\\
3469	2.250649\\
3470	1.661159\\
3471	2.250708\\
3472	3.236447\\
3473	3.678352\\
3474	4.795364\\
3475	4.800448\\
3476	4.795785\\
3477	4.798205\\
3478	4.80104\\
3479	4.79734\\
3480	3.96762\\
3481	3.304776\\
3482	3.179232\\
3483	3.179243\\
3484	3.304707\\
3485	3.640237\\
3486	4.801899\\
3487	7.125396\\
3488	7.783678\\
3489	8.504197\\
3490	8.504197\\
3491	8.678527\\
3492	8.191905\\
3493	7.122946\\
3494	6.979987\\
3495	7.034168\\
3496	7.100978\\
3497	7.390305\\
3498	7.322217\\
3499	6.923258\\
3500	6.518234\\
3501	5.989183\\
3502	6.273591\\
3503	5.749805\\
3504	4.801576\\
3505	4.795621\\
3506	4.37618\\
3507	4.2758\\
3508	4.546156\\
3509	4.797207\\
3510	5.142361\\
3511	6.923271\\
3512	7.390307\\
3513	8.302061\\
3514	8.519108\\
3515	8.302113\\
3516	6.92326\\
3517	7.345575\\
3518	7.322306\\
3519	8.269399\\
3520	8.295673\\
3521	8.295672\\
3522	8.295642\\
3523	7.319166\\
3524	6.923261\\
3525	6.923265\\
3526	6.459754\\
3527	5.989299\\
3528	4.799729\\
3529	4.800684\\
3530	4.780874\\
3531	4.651578\\
3532	4.72792\\
3533	4.800572\\
3534	5.774463\\
3535	6.977511\\
3536	7.516924\\
3537	8.302062\\
3538	8.51908\\
3539	8.371363\\
3540	8.519101\\
3541	8.302063\\
3542	8.117751\\
3543	7.541503\\
3544	7.390305\\
3545	7.745416\\
3546	7.735617\\
3547	7.306627\\
3548	6.923258\\
3549	6.923266\\
3550	7.127118\\
3551	6.923258\\
3552	5.14236\\
3553	4.801843\\
3554	4.794971\\
3555	4.311215\\
3556	3.678342\\
3557	3.541819\\
3558	3.304804\\
3559	3.601786\\
3560	3.761004\\
3561	4.546555\\
3562	4.795063\\
3563	4.798704\\
3564	4.797196\\
3565	4.376247\\
3566	3.601787\\
3567	3.306046\\
3568	3.304809\\
3569	3.601786\\
3570	4.624504\\
3571	4.801289\\
3572	4.801161\\
3573	4.802632\\
3574	4.800892\\
3575	4.800462\\
3576	3.91595\\
3577	3.304801\\
3578	3.179199\\
3579	3.179214\\
3580	3.179234\\
3581	3.30472\\
3582	3.662419\\
3583	4.79562\\
3584	4.795346\\
3585	4.796409\\
3586	4.79523\\
3587	4.799688\\
3588	4.036323\\
3589	3.601787\\
3590	3.612015\\
3591	3.304807\\
3592	3.607284\\
3593	3.662415\\
3594	4.797966\\
3595	4.802456\\
3596	4.801609\\
3597	4.799564\\
3598	4.802202\\
3599	4.801577\\
3600	4.802221\\
3601	3.678319\\
3602	3.30484\\
3603	3.304744\\
3604	3.304784\\
3605	3.304816\\
3606	3.304795\\
3607	3.963003\\
3608	4.548082\\
3609	4.800939\\
3610	4.019911\\
3611	3.976586\\
3612	3.927536\\
3613	3.662417\\
3614	3.381111\\
3615	3.360525\\
3616	3.69741\\
3617	4.797545\\
3618	4.802726\\
3619	4.810658\\
3620	4.81145\\
3621	4.801655\\
3622	4.787254\\
3623	4.788169\\
3624	4.940134\\
3625	4.47772\\
3626	4.479638\\
3627	4.922293\\
3628	4.922293\\
3629	4.922293\\
3630	4.922293\\
3631	4.922293\\
3632	4.922293\\
3633	5.271747\\
3634	5.38384\\
3635	5.272429\\
3636	4.922293\\
3637	4.922293\\
3638	4.922293\\
3639	4.660322\\
3640	4.922293\\
3641	5.110125\\
3642	6.52151\\
3643	6.752432\\
3644	6.881689\\
3645	6.881692\\
3646	6.881689\\
3647	6.881692\\
3648	6.037346\\
3649	5.27186\\
3650	4.922293\\
3651	4.922293\\
3652	4.966136\\
3653	5.819333\\
3654	6.881691\\
3655	8.32419\\
3656	8.918808\\
3657	8.644668\\
3658	8.32419\\
3659	8.168888\\
3660	7.786387\\
3661	8.064058\\
3662	8.12224\\
3663	8.12224\\
3664	8.128012\\
3665	8.985563\\
3666	8.32419\\
3667	8.420175\\
3668	8.128012\\
3669	8.128018\\
3670	9.364566\\
3671	8.128009\\
3672	6.881692\\
3673	6.615675\\
3674	6.037346\\
3675	5.865886\\
3676	5.84297\\
3677	6.037353\\
3678	6.881691\\
3679	8.854804\\
3680	9.875465\\
3681	10.165934\\
3682	9.2272\\
3683	9.075084\\
3684	8.121935\\
3685	7.303639\\
3686	7.557392\\
3687	8.128009\\
3688	8.324231\\
3689	8.324196\\
3690	9.838646\\
3691	9.815219\\
3692	8.574311\\
3693	8.128022\\
3694	9.077744\\
3695	8.128002\\
3696	6.881692\\
3697	13.522295\\
3698	13.266926\\
3699	13.256288\\
3700	13.522293\\
3701	13.522294\\
3702	14.760149\\
3703	14.909328\\
3704	15.964878\\
3705	16.091113\\
3706	16.907074\\
3707	16.462464\\
3708	16.907019\\
3709	16.872605\\
3710	12.163165\\
3711	11.965799\\
3712	11.70602\\
3713	12.018917\\
3714	15.762363\\
3715	14.914432\\
3716	14.909266\\
3717	14.837069\\
3718	7.910721\\
3719	6.881692\\
3720	5.686808\\
3721	4.922293\\
3722	4.922293\\
3723	4.922293\\
3724	4.922293\\
3725	4.922293\\
3726	6.03735\\
3727	7.059421\\
3728	8.122188\\
3729	7.836641\\
3730	7.020657\\
3731	6.881692\\
3732	6.881689\\
3733	6.615676\\
3734	6.394005\\
3735	6.037346\\
3736	6.615675\\
3737	6.881692\\
3738	7.444609\\
3739	8.122237\\
3740	7.46288\\
3741	7.303847\\
3742	8.129129\\
3743	7.876838\\
3744	6.881691\\
3745	6.454234\\
3746	5.98508\\
3747	5.864398\\
3748	6.037346\\
3749	6.615675\\
3750	7.303444\\
3751	9.602122\\
3752	10.200849\\
3753	8.911155\\
3754	8.277728\\
3755	8.128003\\
3756	7.850268\\
3757	7.303873\\
3758	6.881692\\
3759	6.881692\\
3760	7.020694\\
3761	7.85851\\
3762	8.128004\\
3763	8.12801\\
3764	7.41822\\
3765	6.881692\\
3766	7.744471\\
3767	6.881692\\
3768	6.264454\\
3769	5.533759\\
3770	4.922293\\
3771	4.922293\\
3772	4.922293\\
3773	4.322566\\
3774	4.551552\\
3775	4.922293\\
3776	5.457805\\
3777	6.037345\\
3778	6.19049\\
3779	6.118857\\
3780	6.013033\\
3781	5.687085\\
3782	4.922294\\
3783	5.088646\\
3784	5.687092\\
3785	6.037347\\
3786	6.615678\\
3787	6.88169\\
3788	6.881648\\
3789	6.88169\\
3790	6.88169\\
3791	6.88169\\
3792	6.067996\\
3793	5.340139\\
3794	4.922293\\
3795	4.922293\\
3796	4.922293\\
3797	4.922293\\
3798	4.922038\\
3799	4.922293\\
3800	5.260072\\
3801	5.687107\\
3802	5.84309\\
3803	5.687087\\
3804	5.271857\\
3805	4.922293\\
3806	4.551564\\
3807	4.285255\\
3808	4.922223\\
3809	4.922293\\
3810	5.687087\\
3811	5.843085\\
3812	6.037346\\
3813	6.14928\\
3814	6.037346\\
3815	6.037346\\
3816	5.271864\\
3817	4.922293\\
3818	4.548158\\
3819	4.413465\\
3820	4.08264\\
3821	3.751463\\
3822	3.765908\\
3823	4.770781\\
3824	4.922293\\
3825	4.922293\\
3826	4.922293\\
3827	4.92246\\
3828	4.922293\\
3829	4.922293\\
3830	4.922293\\
3831	4.922293\\
3832	4.922293\\
3833	5.610983\\
3834	6.037369\\
3835	6.615688\\
3836	6.881689\\
3837	6.87914\\
3838	6.881692\\
3839	6.881684\\
3840	9.447867\\
3841	8.842052\\
3842	8.647599\\
3843	8.061733\\
3844	8.076538\\
3845	8.731891\\
3846	9.786342\\
3847	11.579579\\
3848	14.417049\\
3849	14.96789\\
3850	14.583797\\
3851	14.967861\\
3852	13.987893\\
3853	13.729711\\
3854	13.810376\\
3855	13.065034\\
3856	13.008994\\
3857	13.859589\\
3858	14.967885\\
3859	14.944808\\
3860	13.510643\\
3861	12.506579\\
3862	13.769715\\
3863	12.290619\\
3864	10.552843\\
3865	13.944478\\
3866	13.522306\\
3867	13.522306\\
3868	13.651604\\
3869	13.941989\\
3870	23.430258\\
3871	26.142084\\
3872	27.237366\\
3873	27.123103\\
3874	25.995432\\
3875	25.600756\\
3876	25.131344\\
3877	24.885417\\
3878	24.04857\\
3879	23.998645\\
3880	25.106948\\
3881	26.154451\\
3882	14.169368\\
3883	15.108618\\
3884	14.980736\\
3885	13.829421\\
3886	14.967869\\
3887	13.719262\\
3888	10.932731\\
3889	10.108799\\
3890	9.686398\\
3891	9.686417\\
3892	9.686399\\
3893	9.970381\\
3894	11.224919\\
3895	15.033911\\
3896	15.889805\\
3897	14.55734\\
3898	14.456199\\
3899	13.566721\\
3900	11.757358\\
3901	12.05091\\
3902	11.561737\\
3903	11.80811\\
3904	13.173668\\
3905	13.916034\\
3906	15.248018\\
3907	15.178185\\
3908	14.915364\\
3909	14.23777\\
3910	15.239346\\
3911	13.599626\\
3912	23.494407\\
3913	22.119313\\
3914	21.754771\\
3915	21.35414\\
3916	9.686392\\
3917	9.987853\\
3918	15.521763\\
3919	19.40082\\
3920	19.79145\\
3921	19.191003\\
3922	18.803785\\
3923	18.61322\\
3924	17.619743\\
3925	17.34481\\
3926	16.435327\\
3927	15.652748\\
3928	16.599205\\
3929	17.002507\\
3930	17.621636\\
3931	17.755736\\
3932	17.255387\\
3933	16.598714\\
3934	18.002565\\
3935	16.459263\\
3936	14.964807\\
3937	13.944478\\
3938	14.33126\\
3939	13.944476\\
3940	13.522306\\
3941	13.522306\\
3942	13.522307\\
3943	13.812533\\
3944	14.66578\\
3945	14.768627\\
3946	14.508328\\
3947	14.360672\\
3948	13.624155\\
3949	13.522307\\
3950	13.094849\\
3951	12.677963\\
3952	13.070543\\
3953	13.522307\\
3954	13.522306\\
3955	14.492731\\
3956	14.09453\\
3957	13.944481\\
3958	14.768627\\
3959	14.598331\\
3960	13.661267\\
3961	9.686391\\
3962	8.984165\\
3963	8.842019\\
3964	8.841564\\
3965	8.491816\\
3966	8.491803\\
3967	8.842051\\
3968	9.304478\\
3969	9.347928\\
3970	9.42039\\
3971	9.375116\\
3972	8.737267\\
3973	8.452943\\
3974	7.726996\\
3975	7.726997\\
3976	7.726996\\
3977	8.491795\\
3978	9.480009\\
3979	9.686397\\
3980	10.045868\\
3981	9.829182\\
3982	10.445271\\
3983	9.871057\\
3984	13.522306\\
3985	5.865475\\
3986	5.450251\\
3987	5.531166\\
3988	5.865485\\
3989	6.506394\\
3990	7.344935\\
3991	9.354144\\
3992	10.516988\\
3993	9.794109\\
3994	9.521644\\
3995	9.775636\\
3996	9.629735\\
3997	8.992188\\
3998	8.502573\\
3999	8.502572\\
4000	8.502573\\
4001	8.652208\\
};
\addplot [color=mycolor1,solid,line width=1.0pt,forget plot]
  table[row sep=crcr]{%
4001	8.652208\\
4002	9.319868\\
4003	8.655777\\
4004	8.502579\\
4005	8.306399\\
4006	8.502578\\
4007	8.30063\\
4008	7.060074\\
4009	7.044386\\
4010	6.794064\\
4011	6.794065\\
4012	6.973845\\
4013	7.060079\\
4014	8.300628\\
4015	9.675213\\
4016	11.959344\\
4017	13.041261\\
4018	9.714491\\
4019	11.156391\\
4020	7.62205\\
4021	6.68022\\
4022	6.68022\\
4023	6.856645\\
4024	6.955348\\
4025	7.291945\\
4026	6.68022\\
4027	7.127208\\
4028	7.478162\\
4029	7.467503\\
4030	7.883266\\
4031	7.687083\\
4032	6.440765\\
4033	5.596427\\
4034	5.440382\\
4035	5.386403\\
4036	5.596421\\
4037	6.058943\\
4038	6.862876\\
4039	8.416522\\
4040	8.623723\\
4041	8.906068\\
4042	8.004982\\
4043	7.979676\\
4044	8.031715\\
4045	6.722465\\
4046	6.085206\\
4047	5.974356\\
4048	6.314148\\
4049	6.798511\\
4050	6.994686\\
4051	6.994678\\
4052	6.798507\\
4053	6.48448\\
4054	6.994686\\
4055	6.798503\\
4056	5.552185\\
4057	5.286176\\
4058	4.707841\\
4059	4.707162\\
4060	4.780435\\
4061	5.332568\\
4062	6.257618\\
4063	7.565078\\
4064	9.138534\\
4065	8.748774\\
4066	8.465545\\
4067	7.302004\\
4068	6.798503\\
4069	6.650084\\
4070	6.798511\\
4071	6.798393\\
4072	6.994687\\
4073	8.095042\\
4074	9.119055\\
4075	7.738487\\
4076	6.798511\\
4077	6.43727\\
4078	6.875573\\
4079	6.192571\\
4080	5.552185\\
4081	5.286171\\
4082	4.614301\\
4083	4.513609\\
4084	4.513604\\
4085	4.707841\\
4086	5.552185\\
4087	6.792735\\
4088	6.994688\\
4089	7.090825\\
4090	7.415985\\
4091	7.090834\\
4092	6.798503\\
4093	6.376802\\
4094	6.798511\\
4095	6.798505\\
4096	6.994567\\
4097	7.090601\\
4098	7.090686\\
4099	7.337405\\
4100	7.124086\\
4101	6.994685\\
4102	7.96932\\
4103	7.091641\\
4104	5.974357\\
4105	5.552188\\
4106	5.284661\\
4107	4.70785\\
4108	4.513584\\
4109	4.357581\\
4110	4.357582\\
4111	5.124672\\
4112	7.335438\\
4113	7.241154\\
4114	7.681236\\
4115	6.584526\\
4116	6.258048\\
4117	6.258048\\
4118	5.992037\\
4119	5.992034\\
4120	6.128705\\
4121	6.258048\\
4122	6.258048\\
4123	6.68022\\
4124	6.68022\\
4125	6.68022\\
4126	7.122812\\
4127	7.074831\\
4128	6.258048\\
4129	5.413705\\
4130	5.063446\\
4131	4.648219\\
4132	4.79201\\
4133	4.298666\\
4134	4.298653\\
4135	4.931021\\
4136	5.092967\\
4137	5.413705\\
4138	5.413705\\
4139	5.413702\\
4140	5.063445\\
4141	4.298663\\
4142	4.298652\\
4143	4.298652\\
4144	4.596854\\
4145	5.413705\\
4146	5.992034\\
4147	6.258048\\
4148	6.258048\\
4149	6.618674\\
4150	7.498599\\
4151	6.68023\\
4152	6.257827\\
4153	5.413705\\
4154	5.063445\\
4155	4.978212\\
4156	5.063445\\
4157	5.413789\\
4158	6.680316\\
4159	8.967248\\
4160	9.724116\\
4161	10.03097\\
4162	8.81453\\
4163	8.542032\\
4164	7.863859\\
4165	7.749671\\
4166	7.611963\\
4167	7.700548\\
4168	8.164338\\
4169	9.097425\\
4170	10.388897\\
4171	9.935379\\
4172	8.928712\\
4173	7.70054\\
4174	9.151076\\
4175	9.207234\\
4176	7.458742\\
4177	6.258048\\
4178	6.253827\\
4179	5.995811\\
4180	6.209042\\
4181	6.258048\\
4182	7.124624\\
4183	8.419739\\
4184	9.017897\\
4185	9.357394\\
4186	10.397369\\
4187	11.722241\\
4188	10.536494\\
4189	11.486641\\
4190	10.539195\\
4191	9.647144\\
4192	9.406598\\
4193	10.119395\\
4194	7.94827\\
4195	8.478049\\
4196	9.628607\\
4197	8.468252\\
4198	10.494104\\
4199	8.93015\\
4200	7.68708\\
4201	6.57936\\
4202	5.721833\\
4203	5.596423\\
4204	5.958271\\
4205	6.174749\\
4206	6.80139\\
4207	8.171987\\
4208	8.021106\\
4209	7.979465\\
4210	7.883266\\
4211	7.687088\\
4212	7.664775\\
4213	7.882995\\
4214	7.687086\\
4215	7.687093\\
4216	7.834813\\
4217	8.146201\\
4218	8.355078\\
4219	8.526728\\
4220	8.128924\\
4221	7.883266\\
4222	9.075263\\
4223	8.003427\\
4224	6.862937\\
4225	6.440765\\
4226	6.174751\\
4227	6.14306\\
4228	6.440765\\
4229	6.440772\\
4230	7.687078\\
4231	7.948073\\
4232	8.440216\\
4233	9.453878\\
4234	9.851829\\
4235	8.869018\\
4236	8.357305\\
4237	7.966721\\
4238	7.883273\\
4239	7.883274\\
4240	8.534796\\
4241	9.460677\\
4242	9.613316\\
4243	7.883271\\
4244	7.883266\\
4245	7.8832\\
4246	9.161323\\
4247	8.666539\\
4248	7.373765\\
4249	6.440767\\
4250	6.291031\\
4251	6.013422\\
4252	6.145466\\
4253	6.380267\\
4254	6.977376\\
4255	7.883271\\
4256	8.306754\\
4257	9.981902\\
4258	12.341553\\
4259	11.036113\\
4260	11.290821\\
4261	11.999134\\
4262	9.146733\\
4263	6.792706\\
4264	6.798505\\
4265	6.823347\\
4266	6.798501\\
4267	6.605141\\
4268	6.264066\\
4269	6.207332\\
4270	6.732354\\
4271	5.974361\\
4272	5.124727\\
4273	4.357381\\
4274	3.592789\\
4275	3.592789\\
4276	3.592789\\
4277	3.592789\\
4278	3.592789\\
4279	3.655227\\
4280	4.638083\\
4281	5.286171\\
4282	5.552185\\
4283	5.552186\\
4284	5.552186\\
4285	5.549989\\
4286	5.286171\\
4287	5.49612\\
4288	5.552185\\
4289	5.552185\\
4290	5.974358\\
4291	5.552185\\
4292	5.413874\\
4293	5.552185\\
4294	5.552185\\
4295	5.552185\\
4296	4.681332\\
4297	3.795576\\
4298	3.592789\\
4299	3.592789\\
4300	3.592789\\
4301	4.481369\\
4302	4.481369\\
4303	4.481369\\
4304	3.592789\\
4305	3.592789\\
4306	3.592791\\
4307	4.322201\\
4308	5.860464\\
4309	5.402166\\
4310	5.107458\\
4311	4.482616\\
4312	4.481369\\
4313	5.221779\\
4314	5.596422\\
4315	6.174751\\
4316	5.603651\\
4317	6.440765\\
4318	6.066398\\
4319	5.625582\\
4320	5.401726\\
4321	4.830939\\
4322	4.481369\\
4323	4.481369\\
4324	4.481369\\
4325	4.857175\\
4326	6.247568\\
4327	6.919206\\
4328	7.526756\\
4329	7.687086\\
4330	7.687085\\
4331	7.883266\\
4332	7.687086\\
4333	7.49337\\
4334	6.862938\\
4335	6.862937\\
4336	7.313183\\
4337	7.687085\\
4338	8.317003\\
4339	8.11071\\
4340	8.475823\\
4341	7.700565\\
4342	8.516633\\
4343	7.700568\\
4344	7.105561\\
4345	6.570913\\
4346	6.090327\\
4347	5.925472\\
4348	6.057161\\
4349	6.32519\\
4350	7.291318\\
4351	10.050875\\
4352	10.596487\\
4353	9.893623\\
4354	8.549774\\
4355	7.638434\\
4356	7.291327\\
4357	7.291341\\
4358	7.105561\\
4359	7.105562\\
4360	7.245211\\
4361	7.615526\\
4362	8.671907\\
4363	6.27473\\
4364	6.452114\\
4365	5.269887\\
4366	5.728813\\
4367	5.230635\\
4368	4.700753\\
4369	4.579014\\
4370	3.726046\\
4371	3.637747\\
4372	3.637686\\
4373	3.901855\\
4374	4.700753\\
4375	5.875372\\
4376	5.930591\\
4377	6.000531\\
4378	5.880844\\
4379	5.859285\\
4380	5.04221\\
4381	4.930696\\
4382	4.700753\\
4383	4.832327\\
4384	5.100486\\
4385	5.76915\\
4386	6.036384\\
4387	6.066517\\
4388	6.274729\\
4389	6.066589\\
4390	6.27473\\
4391	5.875377\\
4392	4.700753\\
4393	4.600869\\
4394	3.896983\\
4395	3.63866\\
4396	3.648133\\
4397	3.788312\\
4398	4.700752\\
4399	5.346143\\
4400	6.06172\\
4401	5.880895\\
4402	5.746547\\
4403	5.346146\\
4404	4.769535\\
4405	4.700753\\
4406	4.700753\\
4407	4.700753\\
4408	4.700753\\
4409	5.247743\\
4410	5.494435\\
4411	5.346147\\
4412	5.345444\\
4413	5.277288\\
4414	5.346141\\
4415	5.100487\\
4416	4.659807\\
4417	3.755845\\
4418	3.637334\\
4419	3.466979\\
4420	3.637734\\
4421	3.806731\\
4422	4.700753\\
4423	5.875387\\
4424	6.066589\\
4425	6.066588\\
4426	6.066588\\
4427	6.274368\\
4428	6.066588\\
4429	6.066588\\
4430	6.022107\\
4431	5.880862\\
4432	5.727294\\
4433	5.880837\\
4434	6.066586\\
4435	5.880837\\
4436	5.812405\\
4437	5.235347\\
4438	5.346172\\
4439	4.700797\\
4440	4.137039\\
4441	3.607933\\
4442	3.400026\\
4443	3.39722\\
4444	3.392536\\
4445	3.401723\\
4446	3.28189\\
4447	3.400004\\
4448	3.392537\\
4449	3.401828\\
4450	3.401828\\
4451	3.401829\\
4452	3.401626\\
4453	3.401832\\
4454	3.399937\\
4455	3.40183\\
4456	3.399958\\
4457	3.39971\\
4458	3.394924\\
4459	3.40178\\
4460	3.40183\\
4461	3.401721\\
4462	3.706104\\
4463	3.401785\\
4464	3.40183\\
4465	3.401773\\
4466	3.303951\\
4467	2.404658\\
4468	1.987286\\
4469	1.987287\\
4470	1.987287\\
4471	2.262661\\
4472	3.050671\\
4473	3.401746\\
4474	3.397062\\
4475	3.39316\\
4476	3.398766\\
4477	3.399163\\
4478	3.401767\\
4479	3.401767\\
4480	3.399401\\
4481	3.40183\\
4482	3.753891\\
4483	3.901499\\
4484	3.718819\\
4485	3.749437\\
4486	3.665229\\
4487	3.432995\\
4488	3.401764\\
4489	3.39745\\
4490	2.904645\\
4491	2.856435\\
4492	3.238555\\
4493	3.397873\\
4494	3.637735\\
4495	4.832331\\
4496	5.193992\\
4497	5.346137\\
4498	5.174127\\
4499	5.34481\\
4500	4.832364\\
4501	5.100495\\
4502	5.346141\\
4503	5.367586\\
4504	5.880843\\
4505	5.864024\\
4506	6.043673\\
4507	5.880842\\
4508	6.143016\\
4509	6.066587\\
4510	6.274727\\
4511	5.880854\\
4512	4.700753\\
4513	4.519965\\
4514	3.789925\\
4515	3.638283\\
4516	3.638522\\
4517	3.901415\\
4518	4.700753\\
4519	5.880843\\
4520	7.79162\\
4521	9.436613\\
4522	9.777886\\
4523	10.414756\\
4524	9.633488\\
4525	9.555402\\
4526	9.464808\\
4527	8.804728\\
4528	8.239028\\
4529	8.807256\\
4530	8.254876\\
4531	7.217275\\
4532	6.420372\\
4533	6.106693\\
4534	6.380362\\
4535	5.888611\\
4536	4.700754\\
4537	3.901452\\
4538	3.646329\\
4539	3.401761\\
4540	3.401669\\
4541	3.401757\\
4542	3.901462\\
4543	4.930245\\
4544	5.880854\\
4545	6.009954\\
4546	6.274728\\
4547	7.610028\\
4548	7.116499\\
4549	6.875075\\
4550	6.66185\\
4551	6.274877\\
4552	6.274727\\
4553	6.258514\\
4554	6.066587\\
4555	5.346172\\
4556	4.832325\\
4557	4.701187\\
4558	5.04221\\
4559	4.700754\\
4560	3.637737\\
4561	3.401652\\
4562	3.399671\\
4563	3.397172\\
4564	3.399869\\
4565	3.401662\\
4566	3.901406\\
4567	4.700788\\
4568	5.880837\\
4569	6.066587\\
4570	5.880837\\
4571	5.880837\\
4572	6.274727\\
4573	6.066587\\
4574	5.880841\\
4575	5.880864\\
4576	5.594875\\
4577	5.880837\\
4578	5.880837\\
4579	5.880838\\
4580	5.100487\\
4581	5.3948\\
4582	5.880843\\
4583	5.102247\\
4584	4.486848\\
4585	3.64271\\
4586	3.492976\\
4587	3.401763\\
4588	3.401754\\
4589	3.672221\\
4590	4.700754\\
4591	5.346141\\
4592	5.346141\\
4593	5.208677\\
4594	5.346141\\
4595	6.274728\\
4596	6.274732\\
4597	5.880869\\
4598	5.806319\\
4599	5.346141\\
4600	5.346141\\
4601	5.875375\\
4602	5.875377\\
4603	5.346141\\
4604	5.571622\\
4605	5.346141\\
4606	5.875377\\
4607	5.881006\\
4608	5.100744\\
4609	4.70076\\
4610	3.901304\\
4611	3.865364\\
4612	3.638099\\
4613	3.592351\\
4614	3.540995\\
4615	3.901307\\
4616	4.700758\\
4617	5.100292\\
4618	5.346313\\
4619	5.346339\\
4620	5.13095\\
4621	4.701126\\
4622	4.700761\\
4623	4.296001\\
4624	4.624126\\
4625	4.701173\\
4626	4.81544\\
4627	5.209062\\
4628	5.042604\\
4629	5.042504\\
4630	5.346386\\
4631	4.701179\\
4632	4.578779\\
4633	3.645209\\
4634	3.401716\\
4635	3.401631\\
4636	3.400826\\
4637	3.392012\\
4638	3.399587\\
4639	3.398545\\
4640	3.401693\\
4641	3.638239\\
4642	3.901305\\
4643	4.368588\\
4644	4.320678\\
4645	3.637535\\
4646	3.401705\\
4647	3.401681\\
4648	3.401695\\
4649	3.437381\\
4650	3.901303\\
4651	4.701178\\
4652	4.701174\\
4653	4.701173\\
4654	4.701174\\
4655	4.701173\\
4656	3.926403\\
4657	3.399217\\
4658	3.401746\\
4659	3.401732\\
4660	3.401785\\
4661	3.401828\\
4662	3.756801\\
4663	4.700753\\
4664	5.84567\\
4665	5.346144\\
4666	7.136295\\
4667	7.291324\\
4668	7.301962\\
4669	7.105551\\
4670	6.809747\\
4671	6.433833\\
4672	6.686325\\
4673	7.105553\\
4674	7.291303\\
4675	7.105548\\
4676	7.100041\\
4677	6.57089\\
4678	7.105554\\
4679	6.704389\\
4680	5.925467\\
4681	5.126031\\
4682	4.243186\\
4683	4.243183\\
4684	4.243184\\
4685	4.269168\\
4686	5.63216\\
4687	5.925468\\
4688	6.394396\\
4689	6.570887\\
4690	7.287332\\
4691	7.499438\\
4692	7.597688\\
4693	7.499444\\
4694	7.375621\\
4695	7.499435\\
4696	7.428146\\
4697	7.948939\\
4698	8.16154\\
4699	7.291314\\
4700	7.288432\\
4701	7.291333\\
4702	7.596423\\
4703	9.631975\\
4704	7.881774\\
4705	7.161352\\
4706	6.647609\\
4707	6.51595\\
4708	6.515994\\
4709	7.157761\\
4710	7.806103\\
4711	10.005811\\
4712	11.557026\\
4713	11.718085\\
4714	11.516718\\
4715	10.61072\\
4716	8.709721\\
4717	8.089921\\
4718	7.881812\\
4719	8.08992\\
4720	8.089942\\
4721	8.520437\\
4722	9.990104\\
4723	9.442928\\
4724	7.291281\\
4725	7.291281\\
4726	8.367443\\
4727	7.499421\\
4728	6.570834\\
4729	5.925444\\
4730	5.925445\\
4731	5.731859\\
4732	5.756867\\
4733	5.925444\\
4734	6.433546\\
4735	7.291278\\
4736	8.355035\\
4737	7.537178\\
4738	7.499421\\
4739	7.291282\\
4740	7.291279\\
4741	7.105522\\
4742	7.100056\\
4743	7.105525\\
4744	7.124106\\
4745	7.499435\\
4746	7.934627\\
4747	7.499435\\
4748	7.105528\\
4749	7.105534\\
4750	7.499422\\
4751	6.9112\\
4752	5.925445\\
4753	5.125976\\
4754	4.401176\\
4755	4.22263\\
4756	4.264758\\
4757	5.07245\\
4758	5.925444\\
4759	6.325179\\
4760	7.105525\\
4761	7.105066\\
4762	7.100046\\
4763	7.105533\\
4764	6.325159\\
4765	5.925444\\
4766	5.925444\\
4767	5.925444\\
4768	5.968336\\
4769	6.570834\\
4770	6.995076\\
4771	7.105525\\
4772	7.105526\\
4773	7.100062\\
4774	7.291279\\
4775	6.570834\\
4776	5.925444\\
4777	3.401755\\
4778	3.399482\\
4779	3.401677\\
4780	3.294977\\
4781	2.590708\\
4782	2.306808\\
4783	2.90464\\
4784	3.401664\\
4785	3.401677\\
4786	3.401776\\
4787	2.42578\\
4788	2.271485\\
4789	2.293215\\
4790	2.241321\\
4791	2.362624\\
4792	3.401736\\
4793	3.40183\\
4794	3.669285\\
4795	4.219549\\
4796	4.589194\\
4797	3.901458\\
4798	3.901822\\
4799	3.901415\\
4800	3.401798\\
4801	3.401829\\
4802	3.395963\\
4803	3.392137\\
4804	3.401737\\
4805	3.401729\\
4806	3.218412\\
4807	3.401751\\
4808	3.391926\\
4809	3.401829\\
4810	3.401687\\
4811	3.401832\\
4812	3.395213\\
4813	3.396767\\
4814	3.401756\\
4815	3.132563\\
4816	3.272985\\
4817	3.401649\\
4818	3.401829\\
4819	3.401758\\
4820	3.473933\\
4821	3.629283\\
4822	3.648198\\
4823	3.401769\\
4824	3.40183\\
4825	3.401751\\
4826	3.059085\\
4827	2.529041\\
4828	2.306818\\
4829	2.258311\\
4830	2.238613\\
4831	2.512398\\
4832	2.632808\\
4833	3.401697\\
4834	3.392781\\
4835	3.401745\\
4836	3.401732\\
4837	3.394367\\
4838	3.401739\\
4839	3.358409\\
4840	3.246768\\
4841	3.401729\\
4842	3.396896\\
4843	3.398263\\
4844	3.399383\\
4845	3.399689\\
4846	3.401829\\
4847	3.397881\\
4848	3.159764\\
4849	2.293182\\
4850	1.987288\\
4851	1.987286\\
4852	1.987286\\
4853	2.243115\\
4854	3.401728\\
4855	3.401759\\
4856	3.901504\\
4857	4.700761\\
4858	4.700761\\
4859	4.578305\\
4860	3.845699\\
4861	3.63769\\
4862	3.401762\\
4863	3.401673\\
4864	3.401831\\
4865	3.710841\\
4866	3.90142\\
4867	3.954892\\
4868	4.387637\\
4869	3.901491\\
4870	4.503123\\
4871	3.915439\\
4872	3.401673\\
4873	3.394675\\
4874	3.401671\\
4875	3.401666\\
4876	3.401664\\
4877	3.401786\\
4878	3.40183\\
4879	3.821993\\
4880	4.648449\\
4881	4.545211\\
4882	4.040347\\
4883	3.901421\\
4884	3.401762\\
4885	3.401758\\
4886	3.401716\\
4887	3.401762\\
4888	3.428102\\
4889	3.647177\\
4890	3.847251\\
4891	3.90142\\
4892	3.901427\\
4893	3.901421\\
4894	4.700758\\
4895	3.901452\\
4896	3.401718\\
4897	3.394323\\
4898	3.401736\\
4899	3.40171\\
4900	3.40176\\
4901	3.396922\\
4902	3.401744\\
4903	4.058479\\
4904	4.365225\\
4905	3.901512\\
4906	3.650758\\
4907	3.637745\\
4908	3.637733\\
4909	3.401762\\
4910	3.401755\\
4911	3.401757\\
4912	3.402705\\
4913	3.720484\\
4914	3.901468\\
4915	3.901483\\
4916	4.083889\\
4917	4.300035\\
4918	4.700758\\
4919	4.422085\\
4920	3.576677\\
4921	3.401829\\
4922	3.397461\\
4923	3.396866\\
4924	3.398554\\
4925	3.40183\\
4926	3.401765\\
4927	3.82106\\
4928	4.700753\\
4929	4.700758\\
4930	4.674117\\
4931	4.387654\\
4932	3.994403\\
4933	3.901503\\
4934	3.901504\\
4935	3.901423\\
4936	3.89938\\
4937	3.901421\\
4938	3.94778\\
4939	4.289596\\
4940	3.974527\\
4941	4.218404\\
4942	4.700758\\
4943	3.901479\\
4944	3.648142\\
4945	3.40165\\
4946	3.398153\\
4947	3.394237\\
4948	3.401772\\
4949	3.401764\\
4950	3.40177\\
4951	3.399815\\
4952	3.401712\\
4953	3.638335\\
4954	3.650625\\
4955	3.65566\\
4956	3.53429\\
4957	3.40183\\
4958	3.398044\\
4959	3.397391\\
4960	3.399755\\
4961	3.401761\\
4962	3.637852\\
4963	3.887965\\
4964	3.901447\\
4965	3.901448\\
4966	4.249381\\
4967	3.863936\\
4968	3.401626\\
4969	3.39893\\
4970	3.401761\\
4971	3.401645\\
4972	3.093648\\
4973	3.275514\\
4974	2.90464\\
4975	3.401661\\
4976	3.40176\\
4977	3.391642\\
4978	3.401677\\
4979	3.062113\\
4980	2.483642\\
4981	2.241402\\
4982	2.127882\\
4983	2.241329\\
4984	2.530767\\
4985	3.401753\\
4986	3.401829\\
4987	3.637742\\
4988	3.719123\\
4989	3.901425\\
4990	3.901427\\
4991	3.647372\\
4992	3.401829\\
4993	3.399582\\
4994	3.401775\\
4995	3.401755\\
4996	3.40177\\
4997	3.399722\\
4998	3.401672\\
4999	4.229829\\
5000	4.700755\\
5001	4.832223\\
5002	4.873525\\
5003	4.832329\\
5004	4.832323\\
5005	4.700755\\
5006	4.700766\\
5007	4.700764\\
5008	4.700768\\
5009	4.700755\\
5010	4.700755\\
5011	4.553886\\
5012	3.901536\\
5013	3.901438\\
5014	4.700765\\
5015	3.758422\\
5016	3.401826\\
5017	3.396874\\
5018	3.401651\\
5019	3.24122\\
5020	3.351518\\
5021	3.391712\\
5022	3.400249\\
5023	3.640593\\
5024	4.205082\\
5025	3.901439\\
5026	3.901874\\
5027	3.901878\\
5028	3.901526\\
5029	3.80462\\
5030	3.637847\\
5031	3.55722\\
5032	3.639089\\
5033	3.901479\\
5034	4.199411\\
5035	3.901957\\
5036	4.271281\\
5037	4.217469\\
5038	4.700754\\
5039	4.417744\\
5040	3.637131\\
5041	3.401826\\
5042	3.398939\\
5043	3.396669\\
5044	3.396505\\
5045	3.399956\\
5046	3.401703\\
5047	4.30391\\
5048	4.700766\\
5049	4.700764\\
5050	4.408553\\
5051	4.013749\\
5052	3.758506\\
5053	3.642237\\
5054	3.734336\\
5055	3.782101\\
5056	3.901537\\
5057	4.700767\\
5058	4.70076\\
5059	4.700762\\
5060	4.700768\\
5061	4.700764\\
5062	4.849781\\
5063	4.320864\\
5064	3.638157\\
5065	3.401828\\
5066	3.396764\\
5067	3.40178\\
5068	3.394091\\
5069	3.3997\\
5070	3.401625\\
5071	4.063099\\
5072	4.700764\\
5073	4.70075\\
5074	4.700769\\
5075	4.295795\\
5076	4.298813\\
5077	3.901526\\
5078	3.901448\\
5079	3.901443\\
5080	3.901536\\
5081	3.901442\\
5082	4.294942\\
5083	4.235922\\
5084	4.700766\\
5085	4.700768\\
5086	4.963807\\
5087	4.578294\\
5088	3.593077\\
5089	3.590901\\
5090	3.38284\\
5091	3.244578\\
5092	3.221713\\
5093	3.591468\\
5094	3.593157\\
5095	3.685162\\
5096	4.04592\\
5097	3.686039\\
5098	3.639572\\
5099	3.600612\\
5100	3.60689\\
5101	3.600193\\
5102	3.596687\\
5103	3.59965\\
5104	3.6019\\
5105	3.639598\\
5106	3.77979\\
5107	4.144073\\
5108	4.144153\\
5109	4.144098\\
5110	4.563398\\
5111	3.948885\\
5112	3.60094\\
5113	3.593585\\
5114	3.575799\\
5115	2.823054\\
5116	2.450157\\
5117	2.43592\\
5118	2.380813\\
5119	2.951397\\
5120	3.59099\\
5121	3.592538\\
5122	3.593415\\
5123	3.593083\\
5124	3.592457\\
5125	3.591901\\
5126	3.582579\\
5127	3.591816\\
5128	3.594706\\
5129	3.627458\\
5130	3.642633\\
5131	3.644563\\
5132	3.692025\\
5133	3.773056\\
5134	3.928628\\
5135	3.639224\\
5136	3.592114\\
5137	3.598818\\
5138	2.497923\\
5139	2.435893\\
5140	2.425664\\
5141	2.435887\\
5142	2.429772\\
5143	2.6042\\
5144	3.240583\\
5145	2.435927\\
5146	2.435892\\
5147	2.382247\\
5148	2.110821\\
5149	1.905611\\
5150	1.004416\\
5151	0.921281\\
5152	1.819505\\
5153	2.290383\\
5154	2.46727\\
5155	3.591292\\
5156	3.591563\\
5157	3.590971\\
5158	3.592635\\
5159	3.578402\\
5160	3.577693\\
5161	2.948517\\
5162	2.381084\\
5163	2.380813\\
5164	2.435901\\
5165	3.280446\\
5166	4.323521\\
5167	5.44498\\
5168	6.294201\\
5169	6.294201\\
5170	6.294201\\
5171	4.863131\\
5172	4.527327\\
5173	4.144147\\
5174	3.685404\\
5175	3.688225\\
5176	3.685576\\
5177	4.144102\\
5178	4.144145\\
5179	4.144138\\
5180	4.527329\\
5181	4.527327\\
5182	4.993332\\
5183	4.144082\\
5184	3.604105\\
5185	3.593618\\
5186	3.590796\\
5187	3.578687\\
5188	3.576858\\
5189	3.59277\\
5190	3.607935\\
5191	4.144159\\
5192	4.527328\\
5193	4.240682\\
5194	4.144134\\
5195	3.686707\\
5196	4.993519\\
5197	4.993511\\
5198	4.99349\\
5199	4.993492\\
5200	4.993532\\
5201	5.844163\\
5202	6.246813\\
5203	6.444127\\
5204	6.444128\\
5205	6.444123\\
5206	7.417046\\
5207	6.195414\\
5208	4.993545\\
5209	4.145357\\
5210	3.651265\\
5211	3.637956\\
5212	3.570991\\
5213	3.666075\\
5214	4.145004\\
5215	4.99352\\
5216	5.41794\\
5217	5.417958\\
5218	5.417941\\
5219	5.417943\\
5220	4.993789\\
5221	4.99353\\
5222	4.993522\\
5223	5.006718\\
5224	5.417951\\
5225	6.165175\\
5226	6.241124\\
5227	5.417981\\
5228	5.292297\\
5229	5.356013\\
5230	5.706755\\
5231	4.993521\\
5232	4.527279\\
5233	3.82792\\
5234	3.653539\\
5235	3.638289\\
5236	3.666026\\
5237	4.145729\\
5238	4.993511\\
5239	5.916918\\
5240	6.259416\\
5241	8.067782\\
5242	7.468279\\
5243	6.777769\\
5244	6.775688\\
5245	6.444126\\
5246	6.428319\\
5247	6.246815\\
5248	6.246822\\
5249	6.246815\\
5250	6.353568\\
5251	6.246801\\
5252	6.444126\\
5253	6.524056\\
5254	6.744883\\
5255	6.14709\\
5256	4.99353\\
5257	4.527559\\
5258	4.145453\\
5259	3.930927\\
5260	3.948707\\
5261	4.527587\\
5262	4.993522\\
5263	6.246813\\
5264	7.150386\\
5265	7.763583\\
5266	7.000482\\
5267	6.985798\\
5268	6.446228\\
5269	6.55755\\
5270	6.444127\\
5271	6.822204\\
5272	6.986621\\
5273	6.444123\\
5274	6.444128\\
5275	6.444129\\
5276	6.21125\\
5277	5.417944\\
5278	5.417966\\
5279	4.993528\\
5280	4.042419\\
5281	3.602912\\
5282	3.57254\\
5283	3.574374\\
5284	3.604002\\
5285	3.600239\\
5286	3.602033\\
5287	3.574333\\
5288	3.572009\\
5289	3.638238\\
5290	3.638345\\
5291	3.638358\\
5292	3.572155\\
5293	3.574164\\
5294	3.574209\\
5295	3.574452\\
5296	3.573158\\
5297	3.570353\\
5298	3.722475\\
5299	4.528059\\
5300	4.144079\\
5301	4.527976\\
5302	4.883575\\
5303	4.860677\\
5304	3.948719\\
5305	3.63805\\
5306	3.572075\\
5307	3.57421\\
5308	3.57485\\
5309	3.5746\\
5310	3.603544\\
5311	3.60355\\
5312	3.574563\\
5313	3.572125\\
5314	3.571338\\
5315	3.638298\\
5316	3.635636\\
5317	3.574917\\
5318	3.601806\\
5319	3.343214\\
5320	3.598415\\
5321	3.604194\\
5322	3.573513\\
5323	3.57292\\
5324	3.572952\\
5325	3.572114\\
5326	3.602982\\
5327	3.572447\\
5328	3.240468\\
5329	2.435968\\
5330	2.11085\\
5331	2.11085\\
5332	2.399928\\
5333	3.579859\\
5334	3.571188\\
5335	3.684984\\
5336	4.565012\\
5337	4.1947\\
5338	4.057221\\
5339	3.954761\\
5340	3.818784\\
5341	3.67348\\
5342	3.665106\\
5343	3.665229\\
5344	3.666126\\
5345	3.766999\\
5346	4.14599\\
5347	4.52728\\
5348	4.52728\\
5349	4.527281\\
5350	4.993503\\
5351	4.14408\\
5352	3.569486\\
5353	3.571274\\
5354	3.57224\\
5355	3.573387\\
5356	3.57207\\
5357	3.571167\\
5358	3.72165\\
5359	4.857306\\
5360	4.993518\\
5361	4.993516\\
5362	4.881303\\
5363	4.993519\\
5364	4.52728\\
5365	4.52728\\
5366	4.145912\\
5367	4.527281\\
5368	4.86214\\
5369	4.993528\\
5370	5.347188\\
5371	4.993521\\
5372	4.99352\\
5373	5.133427\\
5374	5.533001\\
5375	4.993517\\
5376	4.144082\\
5377	3.638295\\
5378	3.570816\\
5379	3.571226\\
5380	3.570464\\
5381	3.665177\\
5382	4.667093\\
5383	5.034826\\
5384	6.246815\\
5385	6.246825\\
5386	6.442658\\
5387	6.444126\\
5388	6.246817\\
5389	5.974854\\
5390	5.533005\\
5391	5.417944\\
5392	5.417945\\
5393	5.439923\\
5394	5.345813\\
5395	6.088924\\
5396	5.829894\\
5397	6.081157\\
5398	6.246815\\
5399	4.993526\\
5400	4.52728\\
5401	3.948725\\
5402	3.665166\\
5403	3.666024\\
5404	3.665119\\
5405	4.344444\\
5406	4.993529\\
5407	6.339293\\
5408	7.732743\\
5409	8.533131\\
5410	10.177541\\
5411	10.212214\\
5412	8.885355\\
5413	8.261539\\
5414	7.623612\\
5415	7.57396\\
5416	7.735286\\
5417	7.753115\\
5418	8.770807\\
5419	7.857467\\
5420	7.336989\\
5421	7.590087\\
5422	7.130274\\
5423	6.83778\\
5424	5.584262\\
5425	5.017386\\
5426	4.539722\\
5427	4.258708\\
5428	4.189654\\
5429	4.189655\\
5430	3.965245\\
5431	4.189656\\
5432	4.460072\\
5433	4.735038\\
5434	4.795693\\
5435	4.735038\\
5436	4.413323\\
5437	3.964352\\
5438	3.613548\\
5439	3.613549\\
5440	3.613549\\
5441	3.613548\\
5442	4.189656\\
5443	4.735028\\
5444	5.118245\\
5445	5.584261\\
5446	5.584264\\
5447	5.348629\\
5448	4.356779\\
5449	3.613566\\
5450	3.613549\\
5451	3.61355\\
5452	3.61355\\
5453	3.613548\\
5454	3.613547\\
5455	4.189655\\
5456	4.822071\\
5457	5.118245\\
5458	5.118247\\
5459	5.118245\\
5460	4.735039\\
5461	4.189657\\
5462	3.618171\\
5463	3.613549\\
5464	3.86824\\
5465	4.294328\\
5466	5.118245\\
5467	5.27291\\
5468	4.735039\\
5469	4.735039\\
5470	5.036176\\
5471	4.529379\\
5472	3.613549\\
5473	3.613544\\
5474	3.400379\\
5475	2.762244\\
5476	2.440824\\
5477	2.898832\\
5478	2.53455\\
5479	2.939391\\
5480	3.351957\\
5481	3.613544\\
5482	3.613546\\
5483	3.613547\\
5484	3.613547\\
5485	3.613543\\
5486	3.407128\\
5487	3.312072\\
5488	3.613544\\
5489	3.613549\\
5490	3.613548\\
5491	4.086161\\
5492	4.170219\\
5493	4.189657\\
5494	4.110294\\
5495	3.844945\\
5496	3.613548\\
5497	3.613543\\
5498	3.295651\\
5499	3.167316\\
5500	3.337066\\
5501	3.613548\\
5502	4.520895\\
5503	5.584262\\
5504	5.864046\\
5505	5.584263\\
5506	5.584264\\
5507	5.584265\\
5508	5.584263\\
5509	5.584264\\
5510	5.584264\\
5511	5.916499\\
5512	6.124321\\
5513	7.025958\\
5514	7.406406\\
5515	8.492537\\
5516	8.755354\\
5517	8.473613\\
5518	8.64623\\
5519	7.458792\\
5520	6.008871\\
5521	5.584262\\
5522	5.118245\\
5523	4.738199\\
5524	4.774463\\
5525	5.456209\\
5526	6.168575\\
5527	7.035094\\
5528	8.722359\\
5529	9.12996\\
5530	8.578452\\
5531	7.988319\\
5532	7.035098\\
5533	6.418481\\
5534	6.837778\\
5535	6.817879\\
5536	6.837779\\
5537	6.837787\\
5538	8.566082\\
5539	8.954982\\
5540	9.817785\\
5541	9.223844\\
5542	9.32126\\
5543	7.307531\\
5544	6.731793\\
5545	5.584265\\
5546	5.58426\\
5547	5.527985\\
5548	5.584224\\
5549	5.747844\\
5550	7.251014\\
5551	10.171642\\
5552	10.36548\\
5553	10.548074\\
5554	9.648765\\
5555	9.475844\\
5556	7.835765\\
5557	7.781119\\
5558	7.084561\\
5559	7.915846\\
5560	8.791617\\
5561	10.667323\\
5562	10.341042\\
5563	10.494921\\
5564	10.564046\\
5565	11.561904\\
5566	11.252857\\
5567	8.990352\\
5568	6.831978\\
5569	5.584622\\
5570	5.584263\\
5571	5.584261\\
5572	5.584262\\
5573	5.723908\\
5574	7.162873\\
5575	10.132022\\
5576	10.896257\\
5577	8.875077\\
5578	7.976394\\
5579	7.883421\\
5580	8.595624\\
5581	7.586322\\
5582	8.548078\\
5583	8.203806\\
5584	8.237902\\
5585	8.037114\\
5586	9.989198\\
5587	8.789434\\
5588	8.564159\\
5589	7.601299\\
5590	7.552776\\
5591	6.837794\\
5592	5.584262\\
5593	5.044906\\
5594	4.608176\\
5595	4.189657\\
5596	4.189658\\
5597	5.118245\\
5598	5.80556\\
5599	6.837773\\
5600	8.566633\\
5601	7.134753\\
5602	7.183112\\
5603	7.69823\\
5604	7.035096\\
5605	7.869144\\
5606	7.246639\\
5607	7.299318\\
5608	7.136565\\
5609	7.437294\\
5610	7.496618\\
5611	8.217155\\
5612	9.371694\\
5613	9.964329\\
5614	10.065897\\
5615	7.917803\\
5616	6.009451\\
5617	5.584263\\
5618	6.831979\\
5619	6.083462\\
5620	5.796522\\
5621	5.756758\\
5622	6.00887\\
5623	6.573303\\
5624	6.837784\\
5625	6.357363\\
5626	6.330995\\
5627	5.963581\\
5628	5.592583\\
5629	5.58426\\
5630	5.58426\\
5631	5.558469\\
5632	5.58426\\
5633	6.035842\\
5634	6.837782\\
5635	7.729391\\
5636	9.04519\\
5637	8.756428\\
5638	8.002898\\
5639	7.558017\\
5640	6.697439\\
5641	4.379838\\
5642	3.613548\\
5643	3.61355\\
5644	3.613551\\
5645	3.61355\\
5646	3.613548\\
5647	4.189655\\
5648	4.189659\\
5649	4.203941\\
5650	4.189657\\
5651	4.330704\\
5652	4.189658\\
5653	3.613549\\
5654	3.613551\\
5655	3.61355\\
5656	3.613549\\
5657	4.189658\\
5658	5.584155\\
5659	5.724113\\
5660	6.621659\\
5661	7.035096\\
5662	6.837783\\
5663	6.272516\\
5664	5.584258\\
5665	4.757167\\
5666	4.49523\\
5667	4.249031\\
5668	4.539792\\
5669	5.577276\\
5670	6.837778\\
5671	7.819376\\
5672	10.317778\\
5673	11.108457\\
5674	11.699993\\
5675	11.953902\\
5676	12.747116\\
5677	12.846652\\
5678	12.846649\\
5679	13.238993\\
5680	13.629227\\
5681	13.087643\\
5682	12.846659\\
5683	11.737808\\
5684	10.896571\\
5685	9.290769\\
5686	8.920773\\
5687	7.035095\\
5688	5.848009\\
5689	5.584265\\
5690	5.291501\\
5691	5.358635\\
5692	5.584262\\
5693	5.587328\\
5694	7.928145\\
5695	10.896254\\
5696	12.485702\\
5697	14.227259\\
5698	14.289264\\
5699	14.289264\\
5700	14.28924\\
5701	13.890562\\
5702	13.267831\\
5703	14.150299\\
5704	14.010079\\
5705	14.1436\\
5706	13.035173\\
5707	12.846601\\
5708	11.886005\\
5709	11.29374\\
5710	10.846315\\
5711	8.864783\\
5712	6.630793\\
5713	5.58426\\
5714	5.454398\\
5715	5.394317\\
5716	5.58426\\
5717	5.807746\\
5718	8.083173\\
5719	10.381671\\
5720	12.471441\\
5721	12.18636\\
5722	11.188012\\
5723	10.774834\\
5724	10.014722\\
5725	10.185061\\
5726	9.905654\\
5727	10.272274\\
5728	10.572454\\
5729	10.896245\\
5730	12.083709\\
5731	11.885904\\
5732	11.207232\\
5733	11.542372\\
5734	11.489856\\
5735	8.792627\\
5736	6.635886\\
5737	5.58426\\
5738	5.39564\\
5739	5.118232\\
5740	5.118246\\
5741	5.58426\\
5742	7.035094\\
5743	9.866614\\
5744	11.471207\\
5745	11.8697\\
5746	12.472566\\
5747	12.435507\\
5748	11.432821\\
5749	11.26142\\
5750	10.932989\\
5751	11.636263\\
5752	11.058331\\
5753	10.834186\\
5754	10.646462\\
5755	10.222683\\
5756	9.810584\\
5757	9.250866\\
5758	8.176922\\
5759	6.444129\\
5760	4.993493\\
5761	4.70776\\
5762	4.144081\\
5763	4.144083\\
5764	4.144079\\
5765	4.993534\\
5766	6.246815\\
5767	7.974494\\
5768	9.668441\\
5769	8.838711\\
5770	7.667029\\
5771	7.060232\\
5772	6.444127\\
5773	6.241027\\
5774	6.021534\\
5775	5.180399\\
5776	5.178069\\
5777	5.642819\\
5778	6.246815\\
5779	6.246815\\
5780	6.246826\\
5781	6.241028\\
5782	6.241021\\
5783	5.148496\\
5784	4.145786\\
5785	3.66598\\
5786	4.859623\\
5787	4.527308\\
5788	4.258773\\
5789	4.144118\\
5790	4.637008\\
5791	4.993294\\
5792	5.417969\\
5793	6.44413\\
5794	7.656123\\
5795	8.620964\\
5796	8.203671\\
5797	7.646326\\
5798	7.315228\\
5799	7.300196\\
5800	7.565046\\
5801	8.971778\\
5802	9.929349\\
5803	10.305282\\
5804	9.650569\\
5805	9.084733\\
5806	7.985772\\
5807	6.840913\\
5808	6.136878\\
5809	4.993292\\
5810	4.993302\\
5811	4.89232\\
5812	4.863587\\
5813	4.922471\\
5814	4.993301\\
5815	4.993294\\
5816	5.417999\\
5817	6.24683\\
5818	6.707455\\
5819	7.469461\\
5820	6.703419\\
5821	6.24683\\
5822	4.993291\\
5823	4.993301\\
5824	4.993294\\
5825	5.533034\\
5826	6.540163\\
5827	9.5665\\
5828	9.467429\\
5829	9.517184\\
5830	9.963264\\
5831	8.502701\\
5832	5.640702\\
5833	4.662008\\
5834	4.304711\\
5835	4.30471\\
5836	4.304701\\
5837	5.689137\\
5838	8.145631\\
5839	10.32681\\
5840	10.545585\\
5841	9.672679\\
5842	9.141871\\
5843	8.649833\\
5844	8.380825\\
5845	8.381246\\
5846	8.652308\\
5847	8.995612\\
5848	9.532679\\
5849	10.258634\\
5850	10.878093\\
5851	11.920475\\
5852	11.078155\\
5853	10.49039\\
5854	10.149479\\
5855	8.145632\\
5856	6.894069\\
5857	6.652352\\
5858	5.815145\\
5859	5.640702\\
5860	5.745563\\
5861	6.652353\\
5862	8.222732\\
5863	12.980354\\
5864	14.054218\\
5865	11.440064\\
5866	9.883878\\
5867	9.483622\\
5868	8.609595\\
5869	8.420714\\
5870	8.526411\\
5871	8.386668\\
5872	9.032374\\
5873	9.882937\\
5874	10.623586\\
5875	10.580582\\
5876	10.586247\\
5877	10.689934\\
5878	10.08383\\
5879	8.23265\\
5880	6.792904\\
5881	6.652352\\
5882	6.092636\\
5883	5.640702\\
5884	6.140543\\
5885	6.652353\\
5886	8.380685\\
5887	12.980359\\
5888	14.054226\\
5889	11.367225\\
5890	10.204707\\
5891	8.932704\\
5892	8.145632\\
5893	7.947282\\
5894	7.372976\\
5895	7.518946\\
5896	8.145634\\
5897	9.081795\\
5898	9.524776\\
5899	9.40324\\
5900	10.995394\\
5901	11.029005\\
5902	9.371891\\
5903	7.158178\\
5904	6.652353\\
5905	5.786864\\
5906	5.683882\\
5907	5.640702\\
5908	5.641349\\
5909	6.652354\\
5910	8.380685\\
5911	14.054228\\
5912	14.240307\\
5913	12.980354\\
5914	11.057379\\
5915	10.483608\\
5916	8.690233\\
5917	8.380686\\
5918	8.380685\\
5919	8.426812\\
5920	9.981573\\
5921	10.941139\\
5922	17.022347\\
5923	17.022346\\
5924	17.022348\\
5925	16.38229\\
5926	11.681643\\
5927	8.49184\\
5928	7.329782\\
5929	6.652354\\
5930	6.652353\\
5931	6.652349\\
5932	6.652353\\
5933	6.652354\\
5934	8.453497\\
5935	17.022348\\
5936	21.629869\\
5937	21.684814\\
5938	17.731991\\
5939	17.022348\\
5940	10.973088\\
5941	10.461222\\
5942	9.992064\\
5943	9.458918\\
5944	10.075026\\
5945	11.00813\\
5946	11.431046\\
5947	10.540899\\
5948	11.228636\\
5949	10.695615\\
5950	9.097365\\
5951	9.880065\\
5952	11.029842\\
5953	9.712767\\
5954	8.411833\\
5955	8.245299\\
5956	8.245299\\
5957	8.245299\\
5958	8.751087\\
5959	9.345939\\
5960	9.975518\\
5961	11.249007\\
5962	13.590539\\
5963	14.492719\\
5964	13.908956\\
5965	12.568371\\
5966	12.02948\\
5967	11.486331\\
5968	11.806079\\
5969	13.793762\\
5970	13.366292\\
5971	13.40833\\
5972	14.45395\\
5973	13.550703\\
5974	13.410799\\
5975	12.266239\\
5976	10.671572\\
5977	9.767924\\
5978	9.738578\\
5979	9.738572\\
5980	9.731664\\
5981	9.738568\\
5982	9.973632\\
5983	9.973631\\
5984	10.483277\\
5985	10.912072\\
5986	11.502626\\
5987	11.674296\\
5988	10.794903\\
5989	9.785841\\
5990	9.731664\\
5991	9.738573\\
5992	8.755037\\
5993	7.657907\\
5994	8.99142\\
5995	9.430075\\
5996	10.597634\\
5997	10.62077\\
5998	9.93017\\
5999	8.886986\\
6000	7.579261\\
6001	7.529694\\
6002	7.20441\\
6003	6.995208\\
6004	7.579664\\
6005	7.930258\\
6006	11.312488\\
6007	17.868118\\
6008	22.475634\\
6009	17.868085\\
6010	17.868098\\
6011	17.86808\\
6012	13.826045\\
6013	12.139561\\
6014	11.497083\\
6015	11.211443\\
6016	12.580404\\
6017	16.149548\\
6018	17.86808\\
6019	17.86808\\
6020	13.826111\\
6021	13.042083\\
6022	11.917003\\
6023	10.163438\\
6024	8.750078\\
6025	7.633668\\
6026	7.579395\\
6027	7.579423\\
6028	7.579409\\
6029	8.026183\\
6030	11.15836\\
6031	14.899957\\
6032	14.899956\\
6033	12.093608\\
6034	13.004918\\
6035	13.826085\\
6036	11.671596\\
6037	11.201984\\
6038	11.527648\\
6039	11.696385\\
6040	12.647015\\
6041	16.14954\\
6042	17.868079\\
6043	13.287837\\
6044	13.826086\\
6045	12.657342\\
6046	12.263242\\
6047	9.286801\\
6048	8.071606\\
6049	7.579009\\
6050	7.579664\\
6051	7.578814\\
6052	7.57978\\
6053	7.760177\\
6054	10.329326\\
6055	17.868082\\
6056	17.868082\\
6057	14.899957\\
6058	12.450343\\
6059	11.877044\\
6060	10.894154\\
6061	10.162713\\
6062	9.714267\\
6063	10.383444\\
6064	11.679722\\
6065	11.564251\\
6066	16.149559\\
6067	12.563282\\
6068	13.826096\\
6069	14.899957\\
6070	11.094669\\
6071	9.226459\\
6072	8.001655\\
6073	7.579361\\
6074	7.574503\\
6075	7.533949\\
6076	7.579592\\
6077	7.918994\\
6078	11.156452\\
6079	16.149556\\
6080	17.82816\\
6081	17.86808\\
6082	17.868079\\
6083	17.868079\\
6084	12.714696\\
6085	10.984999\\
6086	11.695215\\
6087	11.070573\\
6088	11.366894\\
6089	13.826085\\
6090	14.899956\\
6091	12.518875\\
6092	12.661619\\
6093	12.572875\\
6094	10.917694\\
6095	9.226469\\
6096	7.678587\\
6097	7.579635\\
6098	7.382229\\
6099	7.222794\\
6100	7.536544\\
6101	7.580744\\
6102	10.26071\\
6103	12.705064\\
6104	16.706039\\
6105	13.260807\\
6106	11.935255\\
6107	9.259417\\
6108	8.991395\\
6109	8.984452\\
6110	8.877986\\
6111	8.346142\\
6112	8.991402\\
6113	9.718545\\
6114	10.934428\\
6115	11.462633\\
6116	12.830937\\
6117	11.164758\\
6118	11.506979\\
6119	9.973631\\
6120	8.410033\\
6121	8.244757\\
6122	7.524987\\
6123	7.233607\\
6124	6.915917\\
6125	7.198463\\
6126	8.013413\\
6127	8.245299\\
6128	9.220326\\
6129	9.731666\\
6130	9.292585\\
6131	8.751116\\
6132	8.245299\\
6133	8.24507\\
6134	7.223903\\
6135	6.962617\\
6136	7.233648\\
6137	8.245299\\
6138	8.411795\\
6139	8.751123\\
6140	8.751136\\
6141	8.416414\\
6142	8.411819\\
6143	8.245299\\
6144	7.376048\\
6145	7.098775\\
6146	6.314774\\
6147	5.897647\\
6148	5.897648\\
6149	6.055071\\
6150	7.002168\\
6151	7.23365\\
6152	8.245299\\
6153	8.751122\\
6154	9.241461\\
6155	9.709127\\
6156	8.222831\\
6157	7.717005\\
6158	7.23317\\
6159	6.705354\\
6160	6.705356\\
6161	7.717005\\
6162	7.717005\\
6163	8.188898\\
6164	9.179315\\
6165	8.322799\\
6166	8.222832\\
6167	7.717005\\
6168	6.705355\\
6169	6.109227\\
6170	5.772571\\
6171	5.550802\\
6172	5.935161\\
6173	7.616618\\
6174	9.445338\\
6175	13.143787\\
6176	16.368457\\
6177	18.086996\\
6178	17.446941\\
6179	15.118877\\
6180	11.635534\\
6181	11.183117\\
6182	11.307744\\
6183	11.555999\\
6184	12.325469\\
6185	14.045007\\
6186	18.087\\
6187	18.086993\\
6188	18.087\\
6189	14.045007\\
6190	11.890107\\
6191	9.445338\\
6192	7.883577\\
6193	7.717005\\
6194	7.562049\\
6195	7.59647\\
6196	7.717005\\
6197	8.148939\\
6198	11.494953\\
6199	18.086997\\
6200	22.694506\\
6201	18.086997\\
6202	15.118877\\
6203	14.045007\\
6204	11.245012\\
6205	11.212761\\
6206	10.698417\\
6207	11.046459\\
6208	12.83675\\
6209	18.086978\\
6210	18.087001\\
6211	18.086998\\
6212	22.694521\\
6213	18.08699\\
6214	12.772489\\
6215	10.163502\\
6216	8.153868\\
6217	7.717005\\
6218	7.502779\\
6219	7.562029\\
6220	7.717003\\
6221	7.722373\\
6222	10.875447\\
6223	16.368467\\
6224	16.368456\\
6225	12.928157\\
6226	12.748925\\
6227	12.612683\\
6228	10.39648\\
6229	9.740252\\
6230	9.626629\\
6231	10.253463\\
6232	10.53108\\
6233	11.876698\\
6234	15.304961\\
6235	15.304962\\
6236	18.086992\\
6237	12.851554\\
6238	11.033428\\
6239	9.210285\\
6240	7.962883\\
6241	7.495402\\
6242	6.702686\\
6243	6.689577\\
6244	7.151806\\
6245	7.498083\\
6246	9.9089\\
6247	17.868072\\
6248	22.4756\\
6249	22.530544\\
6250	22.530545\\
6251	38.484788\\
6252	22.530545\\
6253	22.4756\\
6254	22.475607\\
6255	17.868079\\
6256	17.868079\\
6257	22.530547\\
6258	22.53056\\
6259	17.86808\\
6260	22.530544\\
6261	15.08604\\
6262	11.665727\\
6263	9.371231\\
6264	8.359867\\
6265	7.498085\\
6266	7.498083\\
6267	7.490008\\
6268	7.498083\\
6269	7.911336\\
6270	10.996583\\
6271	22.47553\\
6272	22.530543\\
6273	22.530544\\
6274	22.530544\\
6275	22.475601\\
6276	17.868079\\
6277	16.149537\\
6278	17.868071\\
6279	16.149541\\
6280	17.867783\\
6281	17.868083\\
6282	17.868081\\
6283	13.826085\\
6284	17.868081\\
6285	12.364942\\
6286	11.780094\\
6287	9.920214\\
6288	8.137849\\
6289	7.498083\\
6290	6.916303\\
6291	6.486433\\
6292	6.351762\\
6293	6.480194\\
6294	6.913928\\
6295	7.498083\\
6296	8.142128\\
6297	9.226416\\
6298	8.991363\\
6299	8.98445\\
6300	8.53333\\
6301	8.402857\\
6302	8.003366\\
6303	7.94716\\
6304	8.003922\\
6305	8.735414\\
6306	8.991368\\
6307	8.991363\\
6308	9.505007\\
6309	8.961548\\
6310	8.57187\\
6311	7.498083\\
6312	6.986202\\
6313	6.124486\\
6314	5.150434\\
6315	5.150432\\
6316	5.150438\\
6317	5.150434\\
6318	5.150438\\
6319	5.150432\\
6320	5.427057\\
6321	6.824233\\
6322	7.498083\\
6323	7.493623\\
6324	6.283228\\
6325	5.150432\\
6326	5.150432\\
6327	5.150447\\
6328	5.150432\\
6329	5.150432\\
6330	6.981718\\
6331	7.498083\\
6332	7.520965\\
6333	7.498083\\
6334	7.498083\\
6335	6.643151\\
6336	7.233645\\
6337	6.31648\\
6338	5.897683\\
6339	5.897647\\
6340	5.897647\\
6341	7.233647\\
6342	9.738564\\
6343	12.319983\\
6344	14.5733\\
6345	16.896755\\
6346	14.5733\\
6347	13.636252\\
6348	11.651496\\
6349	11.392017\\
6350	11.310482\\
6351	11.344686\\
6352	12.079105\\
6353	11.667235\\
6354	13.826087\\
6355	14.89996\\
6356	16.149535\\
6357	12.856558\\
6358	11.738744\\
6359	9.226415\\
6360	8.004062\\
6361	7.498083\\
6362	7.34339\\
6363	6.981334\\
6364	7.498082\\
6365	7.688141\\
6366	11.259538\\
6367	22.4756\\
6368	22.475602\\
6369	17.868079\\
6370	17.228023\\
6371	13.826085\\
6372	11.359947\\
6373	10.737366\\
6374	10.733689\\
6375	11.3477\\
6376	12.66959\\
6377	17.868072\\
6378	17.868079\\
6379	22.4756\\
6380	22.530545\\
6381	16.149542\\
6382	12.436393\\
6383	9.597942\\
6384	8.003901\\
6385	7.498083\\
6386	6.486527\\
6387	6.484417\\
6388	6.486432\\
6389	7.498083\\
6390	9.226416\\
6391	12.759965\\
6392	14.899944\\
6393	15.086038\\
6394	12.997684\\
6395	17.868077\\
6396	13.826222\\
6397	13.826086\\
6398	12.718957\\
6399	12.581248\\
6400	12.473828\\
6401	16.14954\\
6402	12.872912\\
6403	12.554302\\
6404	13.826084\\
6405	11.034893\\
6406	9.969431\\
6407	8.813688\\
6408	7.498083\\
6409	6.919475\\
6410	6.486435\\
6411	6.486432\\
6412	6.486441\\
6413	7.498083\\
6414	9.966218\\
6415	17.868079\\
6416	17.868082\\
6417	17.868079\\
6418	15.086039\\
6419	12.87448\\
6420	10.387615\\
6421	9.722329\\
6422	9.226417\\
6423	9.226416\\
6424	9.519726\\
6425	10.508403\\
6426	12.159167\\
6427	12.549235\\
6428	14.275516\\
6429	11.168518\\
6430	9.226416\\
6431	8.456719\\
6432	7.664582\\
6433	7.498083\\
6434	6.648348\\
6435	6.486432\\
6436	6.63346\\
6437	7.498083\\
6438	9.622529\\
6439	17.868077\\
6440	17.868079\\
6441	17.868079\\
6442	16.149613\\
6443	17.840724\\
6444	17.868079\\
6445	16.14954\\
6446	17.868077\\
6447	17.868079\\
6448	22.475608\\
6449	22.530544\\
6450	22.530546\\
6451	22.530548\\
6452	22.530533\\
6453	16.14954\\
6454	15.086038\\
6455	11.552676\\
6456	9.226416\\
6457	13.334126\\
6458	12.816716\\
6459	12.11159\\
6460	12.078845\\
6461	12.111585\\
6462	13.092211\\
6463	13.825364\\
6464	15.721159\\
6465	17.816904\\
6466	20.257216\\
6467	17.933757\\
6468	15.223861\\
6469	14.115699\\
6470	13.334121\\
6471	13.245628\\
6472	13.494805\\
6473	15.575092\\
6474	16.593204\\
6475	17.933756\\
6476	20.257211\\
6477	16.95232\\
6478	15.562914\\
6479	13.704355\\
6480	12.533354\\
6481	7.962131\\
6482	7.19751\\
6483	6.575098\\
6484	6.359315\\
6485	6.668388\\
6486	7.278072\\
6487	8.184388\\
6488	8.245299\\
6489	8.245299\\
6490	8.245299\\
6491	8.245299\\
6492	8.089679\\
6493	6.877994\\
6494	6.249789\\
6495	6.31648\\
6496	7.038521\\
6497	8.245299\\
6498	9.973632\\
6499	17.93376\\
6500	21.975872\\
6501	11.084411\\
6502	9.973575\\
6503	8.751124\\
6504	8.245299\\
6505	7.235043\\
6506	6.896669\\
6507	6.853073\\
6508	7.233153\\
6509	8.245299\\
6510	10.541828\\
6511	18.615295\\
6512	18.615295\\
6513	16.896765\\
6514	14.573301\\
6515	14.573301\\
6516	12.513299\\
6517	12.654781\\
6518	12.984204\\
6519	18.615295\\
6520	23.27776\\
6521	145.217096\\
6522	189.961798\\
6523	145.217116\\
6524	24.244632\\
6525	23.277759\\
6526	16.896759\\
6527	11.895696\\
6528	9.738574\\
6529	8.355426\\
6530	8.245299\\
6531	8.245299\\
6532	8.245299\\
6533	8.677358\\
6534	21.975749\\
6535	49.338843\\
6536	190.7038\\
6537	190.703828\\
6538	39.243108\\
6539	27.12977\\
6540	26.638213\\
6541	26.638214\\
6542	18.615295\\
6543	18.615295\\
6544	18.615296\\
6545	24.244459\\
6546	39.232\\
6547	159.131475\\
6548	145.217116\\
6549	23.277767\\
6550	18.615295\\
6551	12.854206\\
6552	9.43013\\
6553	8.491792\\
6554	8.435123\\
6555	7.959598\\
6556	8.412563\\
6557	8.492555\\
6558	11.748285\\
6559	23.973647\\
6560	22.229267\\
6561	19.171798\\
6562	16.306602\\
6563	19.171797\\
6564	11.910437\\
6565	12.005797\\
6566	11.894952\\
6567	12.116825\\
6568	15.008969\\
6569	22.229267\\
6570	23.973647\\
6571	23.973647\\
6572	22.775767\\
6573	23.973647\\
6574	19.171798\\
6575	11.813162\\
6576	10.029715\\
6577	8.936803\\
6578	8.491792\\
6579	8.491797\\
6580	8.491798\\
6581	9.43614\\
6582	17.401883\\
6583	78.749884\\
6584	144.626093\\
6585	144.626071\\
6586	39.690301\\
6587	40.404839\\
6588	24.913421\\
6589	23.973647\\
6590	24.913427\\
6591	28.516773\\
6592	36.360082\\
6593	144.626081\\
6594	129.856998\\
6595	58.986872\\
6596	36.360081\\
6597	23.973647\\
6598	22.229267\\
6599	12.000331\\
6600	10.029713\\
6601	9.094527\\
6602	8.522976\\
6603	8.491798\\
6604	8.491792\\
6605	9.635745\\
6606	16.306591\\
6607	164.342673\\
6608	36.360081\\
6609	23.973647\\
6610	22.229267\\
6611	22.775766\\
6612	15.008969\\
6613	12.344236\\
6614	12.274242\\
6615	12.250017\\
6616	17.401882\\
6617	22.229267\\
6618	23.973647\\
6619	23.973647\\
6620	23.973647\\
6621	22.229267\\
6622	17.401911\\
6623	11.460607\\
6624	9.155058\\
6625	8.491785\\
6626	7.493216\\
6627	7.209271\\
6628	6.07586\\
6629	6.073941\\
6630	6.809894\\
6631	7.937764\\
6632	8.411696\\
6633	8.2089\\
6634	7.788518\\
6635	7.476668\\
6636	7.182996\\
6637	6.697816\\
6638	6.505292\\
6639	7.449782\\
6640	8.306548\\
6641	8.788126\\
6642	9.446028\\
6643	9.139923\\
6644	9.541477\\
6645	9.012738\\
6646	8.491786\\
6647	8.491771\\
6648	7.179034\\
6649	6.073947\\
6650	6.073939\\
6651	6.073938\\
6652	6.073938\\
6653	6.073944\\
6654	6.591118\\
6655	7.658999\\
6656	8.49179\\
6657	8.491792\\
6658	9.153165\\
6659	9.94387\\
6660	9.913031\\
6661	9.012738\\
6662	8.491791\\
6663	8.491791\\
6664	9.012705\\
6665	10.029676\\
6666	11.91186\\
6667	11.773916\\
6668	11.354754\\
6669	11.855366\\
6670	10.485951\\
6671	9.939884\\
6672	8.49179\\
6673	7.849768\\
6674	7.313034\\
6675	7.285128\\
6676	7.449897\\
6677	8.49179\\
6678	11.443374\\
6679	22.229184\\
6680	19.513736\\
6681	22.229268\\
6682	19.171798\\
6683	16.306589\\
6684	13.872458\\
6685	16.114884\\
6686	15.008968\\
6687	12.017993\\
6688	12.612699\\
6689	13.801599\\
6690	13.801569\\
6691	11.667345\\
6692	10.122262\\
6693	8.822768\\
6694	8.349818\\
6695	7.378683\\
6696	6.097095\\
6697	4.43339\\
6698	4.433392\\
6699	4.433392\\
6700	4.433392\\
6701	4.452686\\
6702	7.110764\\
6703	10.941791\\
6704	9.668663\\
6705	8.816106\\
6706	8.389146\\
6707	8.498111\\
6708	6.866088\\
6709	7.372172\\
6710	6.963445\\
6711	6.851226\\
6712	7.008034\\
6713	8.389142\\
6714	10.016704\\
6715	9.848938\\
6716	9.790978\\
6717	8.647731\\
6718	8.382025\\
6719	7.030142\\
6720	6.136175\\
6721	5.109187\\
6722	4.43339\\
6723	4.43339\\
6724	4.841901\\
6725	5.970089\\
6726	8.381955\\
6727	10.734265\\
6728	11.29563\\
6729	12.9313\\
6730	12.679464\\
6731	13.368405\\
6732	14.666004\\
6733	12.485028\\
6734	11.089078\\
6735	9.902356\\
6736	9.01612\\
6737	9.897807\\
6738	10.749924\\
6739	9.996485\\
6740	9.716565\\
6741	8.560417\\
6742	8.389144\\
6743	6.902563\\
6744	5.978018\\
6745	4.884945\\
6746	4.43339\\
6747	4.433389\\
6748	4.433389\\
6749	5.783558\\
6750	8.010589\\
6751	10.71019\\
6752	12.177476\\
6753	10.46924\\
6754	10.522385\\
6755	9.605565\\
6756	7.757196\\
6757	8.053486\\
6758	8.389145\\
6759	8.631225\\
6760	8.749599\\
6761	10.433105\\
6762	9.879597\\
6763	10.794434\\
6764	9.680437\\
6765	8.68564\\
6766	8.381946\\
6767	7.14713\\
6768	5.850695\\
6769	4.760482\\
6770	4.43339\\
6771	4.43339\\
6772	4.433389\\
6773	5.80866\\
6774	8.389137\\
6775	12.675483\\
6776	11.642129\\
6777	11.251595\\
6778	10.520204\\
6779	11.850614\\
6780	9.202963\\
6781	8.905286\\
6782	8.961496\\
6783	8.631228\\
6784	9.199256\\
6785	10.792454\\
6786	12.536886\\
6787	14.474385\\
6788	16.872033\\
6789	11.869924\\
6790	11.283672\\
6791	9.877426\\
6792	7.420861\\
6793	6.851167\\
6794	5.80933\\
6795	5.444526\\
6796	5.21588\\
6797	5.777659\\
6798	6.851083\\
6799	7.372169\\
6800	8.389145\\
6801	8.631225\\
6802	8.631229\\
6803	8.631225\\
6804	8.389142\\
6805	8.298061\\
6806	7.296223\\
6807	7.116267\\
6808	7.372171\\
6809	8.33792\\
6810	8.389145\\
6811	9.149528\\
6812	8.631224\\
6813	8.378653\\
6814	7.197284\\
6815	6.851224\\
6816	6.323323\\
6817	5.23084\\
6818	4.538094\\
6819	4.433394\\
6820	4.433393\\
6821	4.894997\\
6822	5.80124\\
6823	6.845755\\
6824	6.851224\\
6825	6.933666\\
6826	7.144229\\
6827	6.851224\\
6828	6.851224\\
6829	6.447771\\
6830	5.80933\\
6831	5.80933\\
6832	5.989082\\
6833	6.851224\\
6834	6.851224\\
6835	7.214234\\
6836	6.851224\\
6837	6.684683\\
6838	6.691663\\
6839	5.80933\\
6840	5.35619\\
6841	4.866599\\
6842	4.866524\\
6843	4.866696\\
6844	4.866959\\
6845	6.347744\\
6846	8.738718\\
6847	10.446826\\
6848	9.860751\\
6849	9.064385\\
6850	9.064392\\
6851	9.183043\\
6852	8.876331\\
6853	9.064368\\
6854	8.955529\\
6855	9.058708\\
6856	9.064492\\
6857	10.468854\\
6858	17.96436\\
6859	17.964362\\
6860	10.478104\\
6861	9.26214\\
6862	9.064501\\
6863	7.402087\\
6864	6.514269\\
6865	4.863523\\
6866	4.433391\\
6867	4.43339\\
6868	4.43339\\
6869	5.80933\\
6870	8.054565\\
6871	17.53123\\
6872	15.761316\\
6873	9.816815\\
6874	8.631225\\
6875	8.631226\\
6876	8.061838\\
6877	8.389145\\
6878	8.424564\\
6879	8.726231\\
6880	9.728742\\
6881	17.53123\\
6882	22.333079\\
6883	23.328693\\
6884	22.333079\\
6885	17.53123\\
6886	15.761365\\
6887	8.631227\\
6888	7.372171\\
6889	9.01279\\
6890	8.491791\\
6891	8.49179\\
6892	8.940893\\
6893	10.029672\\
6894	23.973433\\
6895	718.036701\\
6896	33.248531\\
6897	22.333085\\
6898	22.333107\\
6899	22.333086\\
6900	20.588701\\
6901	20.5887\\
6902	20.5887\\
6903	21.135199\\
6904	22.33308\\
6905	38.764268\\
6906	187.130922\\
6907	567.89465\\
6908	142.386241\\
6909	22.333079\\
6910	17.53123\\
6911	9.76549\\
6912	7.372172\\
6913	6.851224\\
6914	5.810037\\
6915	5.80933\\
6916	5.80933\\
6917	6.851223\\
6918	8.631223\\
6919	22.33308\\
6920	21.1352\\
6921	22.333079\\
6922	22.333079\\
6923	20.5887\\
6924	17.53123\\
6925	17.53123\\
6926	14.474377\\
6927	15.761315\\
6928	15.761316\\
6929	22.333079\\
6930	52.261719\\
6931	23.328694\\
6932	22.33308\\
6933	20.5887\\
6934	17.53123\\
6935	9.975048\\
6936	7.372145\\
6937	6.851224\\
6938	5.959021\\
6939	5.809347\\
6940	6.428546\\
6941	6.851224\\
6942	9.411461\\
6943	23.328693\\
6944	26.876236\\
6945	22.358989\\
6946	22.33308\\
6947	17.53123\\
6948	13.368401\\
6949	9.934424\\
6950	9.381774\\
6951	9.603427\\
6952	14.858628\\
6953	17.53123\\
6954	14.474382\\
6955	15.761371\\
6956	12.930016\\
6957	9.910515\\
6958	9.234031\\
6959	8.382499\\
6960	6.851224\\
6961	6.157225\\
6962	4.960861\\
6963	4.960856\\
6964	4.960861\\
6965	4.960853\\
6966	4.960861\\
6967	6.336802\\
6968	7.378684\\
6969	7.378684\\
6970	8.523602\\
6971	8.654729\\
6972	8.49179\\
6973	7.599434\\
6974	7.449684\\
6975	7.449898\\
6976	7.997477\\
6977	8.828078\\
6978	10.271966\\
6979	9.064449\\
6980	7.899593\\
6981	7.378683\\
6982	7.378683\\
6983	7.172467\\
6984	5.223428\\
6985	4.960857\\
6986	4.960862\\
6987	4.463902\\
6988	4.503402\\
6989	4.613925\\
6990	4.960862\\
6991	4.960861\\
6992	4.970925\\
6993	5.291189\\
6994	5.392216\\
6995	5.388665\\
6996	4.960861\\
6997	4.43339\\
6998	4.433391\\
6999	4.433391\\
7000	4.43339\\
7001	4.43339\\
7002	5.903665\\
7003	6.851224\\
7004	6.851224\\
7005	6.708651\\
7006	5.80933\\
7007	5.088897\\
7008	4.433391\\
7009	4.431111\\
7010	4.099082\\
7011	3.97486\\
7012	4.433384\\
7013	4.433389\\
7014	6.851224\\
7015	10.551389\\
7016	11.466807\\
7017	11.313588\\
7018	10.777135\\
7019	10.548845\\
7020	8.95396\\
7021	9.192415\\
7022	9.026496\\
7023	9.026553\\
7024	9.407308\\
7025	11.122309\\
7026	9.93804\\
7027	9.088564\\
7028	8.631226\\
7029	6.851225\\
7030	6.851322\\
7031	6.778106\\
7032	5.34263\\
7033	4.433391\\
7034	4.433396\\
7035	4.43339\\
7036	4.433389\\
7037	4.433392\\
7038	6.851226\\
7039	8.631223\\
7040	8.932392\\
7041	8.814777\\
7042	8.743862\\
7043	10.16018\\
7044	8.631225\\
7045	8.466691\\
7046	8.389143\\
7047	7.763487\\
7048	7.641642\\
7049	8.550012\\
7050	8.389145\\
7051	8.230638\\
7052	6.851224\\
7053	5.809371\\
7054	5.613998\\
7055	4.433401\\
7056	5.304405\\
7057	5.304255\\
7058	4.468355\\
7059	4.052996\\
7060	4.667247\\
7061	5.304403\\
7062	7.562625\\
7063	9.502247\\
7064	8.389146\\
7065	8.272752\\
7066	7.372171\\
7067	7.37217\\
7068	7.022551\\
7069	6.851229\\
7070	6.851229\\
7071	7.129525\\
7072	7.511567\\
7073	8.389145\\
7074	11.37308\\
7075	10.329844\\
7076	10.547042\\
7077	8.389146\\
7078	7.51362\\
7079	6.52617\\
7080	4.433604\\
7081	4.433389\\
7082	4.433389\\
7083	4.433389\\
7084	4.433389\\
7085	4.433389\\
7086	6.851225\\
7087	9.017808\\
7088	8.63698\\
7089	8.456044\\
7090	8.389145\\
7091	8.38915\\
7092	7.667096\\
7093	8.183957\\
7094	7.602416\\
7095	8.382439\\
7096	8.389142\\
7097	8.555712\\
7098	8.38915\\
7099	8.38916\\
7100	7.178273\\
7101	6.851224\\
7102	6.573643\\
7103	5.616935\\
7104	4.433389\\
7105	4.433384\\
7106	3.217174\\
7107	2.988622\\
7108	3.00643\\
7109	4.433389\\
7110	5.705244\\
7111	6.851247\\
7112	7.487522\\
7113	8.389145\\
7114	8.389144\\
7115	8.389145\\
7116	8.631225\\
7117	8.38914\\
7118	7.513384\\
7119	7.614856\\
7120	7.777754\\
7121	8.30595\\
7122	8.995407\\
7123	7.728343\\
7124	7.271018\\
7125	6.851224\\
7126	6.851224\\
7127	7.022699\\
7128	6.197421\\
7129	4.864747\\
7130	4.433389\\
7131	4.433389\\
7132	4.433389\\
7133	4.433389\\
7134	5.5692\\
7135	6.816851\\
7136	7.405456\\
7137	8.058403\\
7138	8.050027\\
7139	8.386923\\
7140	8.389145\\
7141	6.851224\\
7142	6.191175\\
7143	5.809337\\
7144	6.620322\\
7145	6.96902\\
7146	8.382024\\
7147	7.282192\\
7148	6.851228\\
7149	6.648384\\
7150	6.830332\\
7151	6.381387\\
7152	4.43339\\
7153	4.433389\\
7154	3.785472\\
7155	3.396949\\
7156	3.038769\\
7157	3.179694\\
7158	3.006443\\
7159	3.972636\\
7160	4.433389\\
7161	4.43339\\
7162	4.43339\\
7163	4.433389\\
7164	4.433389\\
7165	4.433389\\
7166	4.29442\\
7167	4.185365\\
7168	5.304403\\
7169	5.304403\\
7170	5.61683\\
7171	7.605171\\
7172	6.680345\\
7173	4.43339\\
7174	4.43339\\
7175	4.43339\\
7176	4.43339\\
7177	3.785298\\
7178	2.589927\\
7179	2.589901\\
7180	2.589901\\
7181	2.589905\\
7182	4.091642\\
7183	5.09075\\
7184	6.851224\\
7185	6.851227\\
7186	6.851224\\
7187	6.85122\\
7188	5.72345\\
7189	5.80933\\
7190	5.80933\\
7191	5.80933\\
7192	6.752257\\
7193	7.089118\\
7194	8.184568\\
7195	9.468916\\
7196	8.389143\\
7197	7.851611\\
7198	6.851224\\
7199	6.297307\\
7200	5.569624\\
7201	4.43339\\
7202	4.433325\\
7203	3.498689\\
7204	3.00642\\
7205	3.674861\\
7206	4.433389\\
7207	5.876266\\
7208	6.853926\\
7209	6.851228\\
7210	6.851224\\
7211	6.805663\\
7212	6.429061\\
7213	5.80933\\
7214	5.809333\\
7215	5.933859\\
7216	6.851224\\
7217	7.516748\\
7218	8.631225\\
7219	9.362855\\
7220	8.522868\\
7221	7.513223\\
7222	6.851227\\
7223	6.851224\\
7224	5.814456\\
7225	4.43339\\
7226	4.433385\\
7227	3.785472\\
7228	3.687005\\
7229	4.05194\\
7230	4.43339\\
7231	6.549598\\
7232	7.573853\\
7233	8.631226\\
7234	8.664043\\
7235	9.75051\\
7236	10.89764\\
7237	11.129131\\
7238	11.199479\\
7239	11.008119\\
7240	11.027368\\
7241	11.551881\\
7242	13.368401\\
7243	13.368401\\
7244	10.109069\\
7245	8.803309\\
7246	8.381977\\
7247	7.590732\\
7248	7.141119\\
7249	5.878543\\
7250	4.43339\\
7251	4.43339\\
7252	4.433389\\
7253	4.43339\\
7254	4.693298\\
7255	6.851231\\
7256	8.121002\\
7257	8.59332\\
7258	8.948153\\
7259	8.631224\\
7260	8.631227\\
7261	8.172191\\
7262	8.307748\\
7263	8.014872\\
7264	8.247445\\
7265	8.389072\\
7266	9.259314\\
7267	12.151969\\
7268	9.619563\\
7269	7.372171\\
7270	7.722238\\
7271	7.722235\\
7272	7.156565\\
7273	5.304406\\
7274	5.304403\\
7275	4.366754\\
7276	3.877443\\
7277	4.642399\\
7278	5.304403\\
7279	6.769681\\
7280	7.893711\\
7281	8.243192\\
7282	8.243174\\
7283	7.722238\\
7284	7.722239\\
7285	7.62268\\
7286	7.037056\\
7287	6.75202\\
7288	7.12363\\
7289	6.851204\\
7290	7.023014\\
7291	7.514257\\
7292	6.851231\\
7293	6.106692\\
7294	4.43339\\
7295	4.43339\\
7296	5.840531\\
7297	4.377665\\
7298	3.893476\\
7299	2.460283\\
7300	1.856514\\
7301	1.472052\\
7302	1.78175\\
7303	2.964855\\
7304	3.893478\\
7305	3.893481\\
7306	3.893477\\
7307	3.750026\\
7308	3.651463\\
7309	3.893476\\
7310	3.893319\\
7311	3.656582\\
7312	3.893476\\
7313	4.472373\\
7314	5.840531\\
7315	5.840531\\
7316	5.840531\\
7317	5.83523\\
7318	5.840514\\
7319	5.84053\\
7320	5.84053\\
7321	3.893478\\
7322	2.26507\\
7323	1.15808\\
7324	0.170531\\
7325	0.000326\\
7326	0.093339\\
7327	1.158078\\
7328	1.445718\\
7329	1.543226\\
7330	1.445373\\
7331	1.206841\\
7332	1.166334\\
7333	1.856526\\
7334	1.566052\\
7335	1.941131\\
7336	2.770054\\
7337	3.858059\\
7338	4.31727\\
7339	5.477684\\
7340	5.160498\\
7341	4.263288\\
7342	4.333392\\
7343	4.314598\\
7344	4.078718\\
7345	2.006772\\
7346	0.600973\\
7347	1e-06\\
7348	7.8e-05\\
7349	0.000321\\
7350	1.873582\\
7351	5.1474\\
7352	6.141911\\
7353	6.165691\\
7354	5.840545\\
7355	6.677097\\
7356	6.333899\\
7357	6.296089\\
7358	6.849316\\
7359	7.041813\\
7360	7.191604\\
7361	8.022586\\
7362	8.394198\\
7363	8.394196\\
7364	8.394196\\
7365	7.41022\\
7366	6.535874\\
7367	6.727965\\
7368	6.296121\\
7369	5.84053\\
7370	5.357579\\
7371	4.458721\\
7372	4.71611\\
7373	5.245517\\
7374	5.84053\\
7375	8.234305\\
7376	9.093537\\
7377	8.944403\\
7378	10.011007\\
7379	10.974187\\
7380	11.277326\\
7381	11.047243\\
7382	11.242586\\
7383	11.354542\\
7384	12.158307\\
7385	10.570511\\
7386	19.67419\\
7387	19.67419\\
7388	11.41604\\
7389	11.06822\\
7390	9.653967\\
7391	8.944629\\
7392	8.533891\\
7393	7.318989\\
7394	6.019987\\
7395	5.840531\\
7396	5.840531\\
7397	5.840531\\
7398	6.78533\\
7399	9.073661\\
7400	12.42575\\
7401	12.341261\\
7402	11.674924\\
7403	12.117727\\
7404	11.281049\\
7405	11.287655\\
7406	11.286098\\
7407	11.086678\\
7408	11.9333\\
7409	15.277499\\
7410	17.804843\\
7411	19.67419\\
7412	10.764494\\
7413	11.257102\\
7414	10.010993\\
7415	9.38661\\
7416	8.394196\\
7417	7.280767\\
7418	5.840531\\
7419	5.84053\\
7420	5.84053\\
7421	5.84053\\
7422	5.840531\\
7423	8.394196\\
7424	9.864881\\
7425	7.830089\\
7426	8.439239\\
7427	8.661069\\
7428	7.786329\\
7429	7.712651\\
7430	7.591971\\
7431	7.249631\\
7432	7.24976\\
7433	7.504812\\
7434	7.698559\\
7435	8.334255\\
7436	7.236117\\
7437	7.117494\\
7438	5.120843\\
7439	5.138\\
7440	4.682451\\
7441	4.056969\\
7442	2.735399\\
7443	1.723973\\
7444	0.772378\\
7445	1.060768\\
7446	2.735399\\
7447	4.682451\\
7448	5.830183\\
7449	6.135691\\
7450	6.677073\\
7451	6.818309\\
7452	7.236117\\
7453	7.236116\\
7454	7.167235\\
7455	6.690288\\
7456	6.212499\\
7457	5.882501\\
7458	6.400144\\
7459	7.236116\\
7460	6.679013\\
7461	5.01835\\
7462	4.682455\\
7463	4.682451\\
7464	4.819444\\
7465	4.682452\\
7466	3.175332\\
7467	2.735404\\
7468	2.069396\\
7469	1.930087\\
7470	3.893478\\
7471	4.332852\\
7472	5.840389\\
7473	5.840531\\
7474	5.840531\\
7475	5.649724\\
7476	5.553342\\
7477	5.808636\\
7478	5.464375\\
7479	5.750039\\
7480	4.682446\\
7481	4.682452\\
7482	4.682451\\
7483	5.60302\\
7484	4.682453\\
7485	4.682452\\
7486	4.682452\\
7487	4.682452\\
7488	4.682451\\
7489	3.648722\\
7490	2.735399\\
7491	1.951239\\
7492	1.460069\\
7493	1.204212\\
7494	1.866976\\
7495	2.713388\\
7496	2.735402\\
7497	3.085078\\
7498	3.687073\\
7499	4.301709\\
7500	4.682453\\
7501	4.682451\\
7502	4.682451\\
7503	4.682451\\
7504	4.682451\\
7505	4.682452\\
7506	5.13803\\
7507	6.135434\\
7508	6.13569\\
7509	5.138039\\
7510	4.927993\\
7511	4.683532\\
7512	4.682451\\
7513	4.682452\\
7514	4.164415\\
7515	3.156493\\
7516	2.798268\\
7517	2.984069\\
7518	3.648654\\
7519	4.682452\\
7520	4.682451\\
7521	4.765122\\
7522	4.682451\\
7523	4.682451\\
7524	4.682451\\
7525	4.682452\\
7526	4.682452\\
7527	4.682453\\
7528	4.682451\\
7529	4.682451\\
7530	6.617561\\
7531	7.037371\\
7532	5.877942\\
7533	5.138439\\
7534	4.682451\\
7535	5.160237\\
7536	4.682451\\
7537	4.68245\\
7538	3.156519\\
7539	2.735405\\
7540	2.703927\\
7541	2.579075\\
7542	2.735399\\
7543	3.085078\\
7544	2.735403\\
7545	2.735404\\
7546	3.085079\\
7547	3.918553\\
7548	4.44799\\
7549	4.682449\\
7550	4.208699\\
7551	3.6392\\
7552	4.141865\\
7553	4.682451\\
7554	4.682452\\
7555	5.128669\\
7556	4.766646\\
7557	4.781928\\
7558	4.682452\\
7559	4.682455\\
7560	4.682453\\
7561	3.998204\\
7562	2.735402\\
7563	2.562969\\
7564	2.530222\\
7565	2.735404\\
7566	4.08819\\
7567	5.943622\\
7568	8.124254\\
7569	7.70616\\
7570	7.878188\\
7571	8.343097\\
7572	8.425366\\
7573	8.335072\\
7574	8.859382\\
7575	8.860047\\
7576	8.860435\\
7577	8.71795\\
7578	11.031181\\
7579	14.11942\\
7580	9.394985\\
7581	8.90417\\
7582	7.613246\\
7583	7.786327\\
7584	7.236122\\
7585	6.13569\\
7586	4.682451\\
7587	4.682452\\
7588	4.682453\\
7589	4.682453\\
7590	4.868593\\
7591	7.359983\\
7592	8.860439\\
7593	8.853191\\
7594	7.828261\\
7595	7.236116\\
7596	7.236117\\
7597	7.23611\\
7598	7.236117\\
7599	7.236136\\
7600	7.236117\\
7601	7.935476\\
7602	10.644325\\
7603	11.343618\\
7604	9.03279\\
7605	8.506754\\
7606	7.236123\\
7607	7.236117\\
7608	7.030138\\
7609	4.682452\\
7610	4.682451\\
7611	3.686614\\
7612	3.177174\\
7613	3.300996\\
7614	4.682453\\
7615	6.144596\\
7616	8.151697\\
7617	8.698423\\
7618	8.860436\\
7619	8.595007\\
7620	8.899509\\
7621	8.780282\\
7622	9.021684\\
7623	9.116117\\
7624	9.241446\\
7625	10.123264\\
7626	10.05993\\
7627	10.129643\\
7628	8.936417\\
7629	9.113695\\
7630	7.632881\\
7631	7.937419\\
7632	7.236116\\
7633	4.682457\\
7634	4.682451\\
7635	4.50695\\
7636	3.761952\\
7637	3.551259\\
7638	4.147561\\
7639	4.682451\\
7640	4.682451\\
7641	5.882521\\
7642	6.864744\\
7643	7.236117\\
7644	7.236117\\
7645	7.236117\\
7646	7.236116\\
7647	6.828295\\
7648	6.347245\\
7649	6.135691\\
7650	6.549943\\
7651	7.131758\\
7652	7.236117\\
7653	6.135692\\
7654	4.682457\\
7655	5.137572\\
7656	4.976053\\
7657	4.682451\\
7658	4.199291\\
7659	3.156508\\
7660	2.748362\\
7661	2.737563\\
7662	2.735404\\
7663	3.156519\\
7664	4.010885\\
7665	4.682451\\
7666	4.682451\\
7667	4.999869\\
7668	5.753284\\
7669	6.135691\\
7670	6.052645\\
7671	5.290876\\
7672	5.660068\\
7673	5.773567\\
7674	6.833882\\
7675	7.236117\\
7676	7.236117\\
7677	6.66946\\
7678	6.135513\\
7679	5.882529\\
7680	5.138032\\
7681	4.682451\\
7682	4.273855\\
7683	3.443566\\
7684	3.156518\\
7685	3.175316\\
7686	4.682451\\
7687	6.372751\\
7688	8.645978\\
7689	8.860436\\
7690	9.116116\\
7691	8.860439\\
7692	8.875136\\
7693	8.638883\\
7694	8.860436\\
7695	10.340997\\
7696	11.155313\\
7697	10.633473\\
7698	17.804834\\
7699	16.802447\\
7700	9.501979\\
7701	8.860436\\
7702	7.236117\\
7703	7.403753\\
7704	7.236116\\
7705	5.59807\\
7706	4.682451\\
7707	4.682451\\
7708	4.682451\\
7709	4.682451\\
7710	4.682452\\
7711	7.236162\\
7712	10.286491\\
7713	9.850568\\
7714	9.19442\\
7715	9.116116\\
7716	9.116116\\
7717	9.116116\\
7718	9.488807\\
7719	9.442071\\
7720	10.741954\\
7721	9.715229\\
7722	14.119419\\
7723	11.965419\\
7724	10.726908\\
7725	9.474383\\
7726	8.133485\\
7727	8.185843\\
7728	7.236117\\
7729	6.052817\\
7730	4.682452\\
7731	4.682451\\
7732	4.682451\\
7733	4.682451\\
7734	4.682452\\
7735	8.706276\\
7736	12.709175\\
7737	10.765351\\
7738	11.297729\\
7739	11.10729\\
7740	11.080454\\
7741	10.766289\\
7742	11.711149\\
7743	11.068891\\
7744	11.503576\\
7745	10.226205\\
7746	11.683304\\
7747	11.216948\\
7748	9.204695\\
7749	9.661947\\
7750	8.04396\\
7751	7.417222\\
7752	7.236117\\
7753	6.135691\\
7754	4.682452\\
7755	4.682451\\
7756	4.682451\\
7757	4.682451\\
7758	5.138038\\
7759	7.455065\\
7760	10.444328\\
7761	9.480073\\
7762	9.116116\\
7763	8.826014\\
7764	8.123208\\
7765	8.09718\\
7766	8.71623\\
7767	8.852919\\
7768	8.860436\\
7769	9.116122\\
7770	9.856441\\
7771	10.080539\\
7772	8.860791\\
7773	8.860436\\
7774	7.236117\\
7775	7.236117\\
7776	7.236117\\
7777	5.882517\\
7778	4.682452\\
7779	4.682451\\
7780	4.682451\\
7781	4.682451\\
7782	4.682458\\
7783	7.236117\\
7784	8.982788\\
7785	9.751626\\
7786	9.723128\\
7787	9.762406\\
7788	9.116119\\
7789	8.860436\\
7790	9.335275\\
7791	9.053014\\
7792	9.002951\\
7793	9.342058\\
7794	9.932455\\
7795	9.871755\\
7796	8.860436\\
7797	7.707035\\
7798	7.207821\\
7799	7.236117\\
7800	7.234962\\
7801	4.820239\\
7802	4.682451\\
7803	3.906271\\
7804	3.15652\\
7805	2.945239\\
7806	3.08644\\
7807	3.751145\\
7808	5.602398\\
7809	5.662313\\
7810	6.562668\\
7811	4.682451\\
7812	4.682451\\
7813	4.682451\\
7814	4.682451\\
7815	4.682451\\
7816	4.682451\\
7817	5.56995\\
7818	6.135676\\
7819	6.781225\\
7820	7.067542\\
7821	5.882518\\
7822	4.682451\\
7823	4.68247\\
7824	4.682467\\
7825	4.682451\\
7826	3.175318\\
7827	2.735399\\
7828	2.505574\\
7829	2.529763\\
7830	2.735399\\
7831	2.847323\\
7832	3.156519\\
7833	3.691303\\
7834	4.207362\\
7835	4.306203\\
7836	4.682451\\
7837	4.682451\\
7838	4.602561\\
7839	4.312726\\
7840	4.68199\\
7841	4.682451\\
7842	4.682451\\
7843	4.682451\\
7844	4.886174\\
7845	4.682451\\
7846	4.682451\\
7847	4.682451\\
7848	4.68245\\
7849	3.156519\\
7850	2.735399\\
7851	2.392333\\
7852	2.283846\\
7853	2.735399\\
7854	3.175334\\
7855	4.682451\\
7856	7.236117\\
7857	7.438305\\
7858	7.417223\\
7859	7.236117\\
7860	7.236117\\
7861	7.236116\\
7862	6.680025\\
7863	7.236116\\
7864	7.236118\\
7865	7.568076\\
7866	8.860432\\
7867	8.860436\\
7868	8.43783\\
7869	7.236117\\
7870	6.135694\\
7871	6.679323\\
7872	6.135629\\
7873	4.682451\\
7874	4.682451\\
7875	4.199333\\
7876	3.809341\\
7877	4.027153\\
7878	4.682451\\
7879	6.511812\\
7880	8.786261\\
7881	8.860435\\
7882	8.314517\\
7883	7.786328\\
7884	7.650159\\
7885	7.236117\\
7886	7.236117\\
7887	7.404719\\
7888	7.331062\\
7889	7.793549\\
7890	9.1161\\
7891	8.860436\\
7892	8.737293\\
7893	7.417215\\
7894	6.67901\\
7895	7.065224\\
7896	6.141017\\
7897	4.682451\\
7898	4.682451\\
7899	4.199494\\
7900	3.998126\\
7901	4.456521\\
7902	4.682451\\
7903	6.966073\\
7904	8.852928\\
7905	9.24291\\
7906	9.795925\\
7907	10.799746\\
7908	15.287525\\
7909	15.287526\\
7910	16.646765\\
7911	16.646767\\
7912	18.516111\\
7913	18.516111\\
7914	18.516111\\
7915	18.516111\\
7916	10.546259\\
7917	8.852897\\
7918	7.697788\\
7919	7.581007\\
7920	7.236117\\
7921	6.060048\\
7922	4.682452\\
7923	4.682451\\
7924	5.602397\\
7925	5.602397\\
7926	6.029401\\
7927	8.649955\\
7928	11.413324\\
7929	15.039368\\
7930	10.419488\\
7931	9.944291\\
7932	11.494397\\
7933	10.132934\\
7934	10.534644\\
7935	10.161089\\
7936	10.317861\\
7937	10.91335\\
7938	11.047628\\
7939	10.714217\\
7940	10.648983\\
7941	9.116115\\
7942	7.409207\\
7943	7.463325\\
7944	7.236117\\
7945	4.682452\\
7946	4.682451\\
7947	4.199427\\
7948	3.48284\\
7949	3.685409\\
7950	4.682451\\
7951	6.146253\\
7952	7.236117\\
7953	7.236117\\
7954	7.236117\\
7955	7.236117\\
7956	7.362971\\
7957	7.236117\\
7958	7.236117\\
7959	7.236116\\
7960	7.236116\\
7961	7.41722\\
7962	7.236117\\
7963	7.236117\\
7964	7.236117\\
7965	6.485118\\
7966	5.138038\\
7967	5.974247\\
7968	5.883141\\
7969	4.682451\\
7970	4.682451\\
7971	3.635735\\
7972	3.085092\\
7973	3.085074\\
7974	3.156516\\
7975	4.413925\\
7976	4.682451\\
7977	4.682452\\
7978	5.285436\\
7979	4.87016\\
7980	4.682452\\
7981	4.682452\\
7982	4.682452\\
7983	4.682452\\
7984	5.250181\\
7985	5.882517\\
7986	6.909099\\
7987	6.965513\\
7988	5.710634\\
7989	4.921968\\
7990	4.682455\\
7991	5.568472\\
7992	6.13563\\
7993	4.999461\\
7994	4.682451\\
7995	4.682451\\
7996	4.204267\\
7997	3.969439\\
7998	4.010153\\
7999	4.68245\\
8000	4.682451\\
8001	4.682451\\
};
\addplot [color=mycolor1,solid,line width=1.0pt,forget plot]
  table[row sep=crcr]{%
8001	4.682451\\
8002	5.46217\\
8003	7.815303\\
8004	8.175164\\
8005	8.547418\\
8006	7.774924\\
8007	7.539024\\
8008	7.939906\\
8009	10.027498\\
8010	10.036062\\
8011	10.036064\\
8012	10.49322\\
8013	10.036105\\
8014	8.855394\\
8015	8.85542\\
8016	8.253683\\
8017	7.140098\\
8018	6.213373\\
8019	5.669452\\
8020	5.669452\\
8021	5.669452\\
8022	5.669453\\
8023	8.253683\\
8024	10.734413\\
8025	16.606313\\
8026	19.668687\\
8027	19.668688\\
8028	24.801001\\
8029	24.801001\\
8030	24.801\\
8031	24.801\\
8032	24.801\\
8033	24.801\\
8034	25.865132\\
8035	24.801001\\
8036	19.668687\\
8037	19.668687\\
8038	10.300796\\
8039	10.068739\\
8040	9.300552\\
8041	8.958876\\
8042	7.505992\\
8043	7.322728\\
8044	6.885416\\
8045	4.738496\\
8046	4.738498\\
8047	7.506\\
8048	9.987296\\
8049	10.584976\\
8050	11.43972\\
8051	11.552561\\
8052	11.132064\\
8053	11.411439\\
8054	9.39249\\
8055	10.187088\\
8056	10.795025\\
8057	11.302152\\
8058	11.696954\\
8059	11.199982\\
8060	10.094854\\
8061	8.966486\\
8062	7.322726\\
8063	7.322725\\
8064	7.322726\\
8065	6.464895\\
8066	5.197337\\
8067	4.738495\\
8068	4.738495\\
8069	4.738495\\
8070	4.738496\\
8071	7.322725\\
8072	9.476894\\
8073	10.35519\\
8074	10.392513\\
8075	9.989418\\
8076	10.242377\\
8077	10.281719\\
8078	11.120219\\
8079	11.367265\\
8080	16.846011\\
8081	18.120207\\
8082	18.120207\\
8083	15.470506\\
8084	11.42756\\
8085	10.677363\\
8086	9.041058\\
8087	8.966486\\
8088	8.709244\\
8089	7.879523\\
8090	7.322725\\
8091	6.422817\\
8092	5.868171\\
8093	5.952926\\
8094	6.831351\\
8095	8.966517\\
8096	18.73773\\
8097	23.870043\\
8098	23.870043\\
8099	23.870043\\
8100	24.934175\\
8101	24.934175\\
8102	24.934175\\
8103	23.870043\\
8104	23.870043\\
8105	18.737718\\
8106	23.870043\\
8107	22.257683\\
8108	18.73773\\
8109	18.120202\\
8110	9.983576\\
8111	10.257344\\
8112	10.207576\\
8113	7.434216\\
8114	7.322725\\
8115	6.29112\\
8116	5.836769\\
8117	5.67897\\
8118	6.759711\\
8119	8.966486\\
8120	14.288415\\
8121	11.485833\\
8122	16.84601\\
8123	16.846013\\
8124	16.846011\\
8125	14.288415\\
8126	11.785243\\
8127	10.959498\\
8128	10.785211\\
8129	11.553896\\
8130	10.734378\\
8131	9.469548\\
8132	9.225227\\
8133	7.879513\\
8134	7.895678\\
8135	8.36511\\
8136	8.175556\\
8137	7.322726\\
8138	6.209126\\
8139	4.738495\\
8140	4.738496\\
8141	4.738495\\
8142	4.738495\\
8143	4.738495\\
8144	5.82893\\
8145	7.152113\\
8146	7.412809\\
8147	7.690344\\
8148	7.506492\\
8149	7.323447\\
8150	7.322729\\
8151	7.322725\\
8152	7.873324\\
8153	8.958703\\
8154	9.231187\\
8155	9.402561\\
8156	9.390468\\
8157	8.966485\\
8158	7.810187\\
8159	7.879524\\
8160	10.719954\\
8161	9.261512\\
8162	8.445541\\
8163	6.981146\\
8164	6.491967\\
8165	6.491966\\
8166	6.491966\\
8167	6.491967\\
8168	6.491967\\
8169	6.491967\\
8170	7.4384\\
8171	8.487758\\
8172	9.076195\\
8173	9.076197\\
8174	6.209129\\
8175	5.952903\\
8176	5.796414\\
8177	6.241108\\
8178	7.290824\\
8179	7.303133\\
8180	7.322726\\
8181	7.322728\\
8182	7.222828\\
8183	7.322726\\
8184	5.996087\\
8185	4.738496\\
8186	4.738495\\
8187	3.663831\\
8188	3.097117\\
8189	3.122032\\
8190	4.605949\\
8191	6.03884\\
8192	7.798237\\
8193	9.225226\\
8194	9.149696\\
8195	8.966482\\
8196	9.225226\\
8197	9.917076\\
8198	10.260553\\
8199	11.262191\\
8200	14.288415\\
8201	16.846011\\
8202	16.84601\\
8203	18.737729\\
8204	15.675338\\
8205	9.225323\\
8206	7.683098\\
8207	7.954299\\
8208	8.348275\\
8209	7.054412\\
8210	4.925853\\
8211	4.738496\\
8212	4.738495\\
8213	4.738496\\
8214	5.487786\\
8215	8.311147\\
8216	10.880919\\
8217	18.120206\\
8218	11.054057\\
8219	11.500011\\
8220	10.108026\\
8221	10.448424\\
8222	10.945523\\
8223	10.518198\\
8224	10.738354\\
8225	11.019485\\
8226	11.415479\\
8227	10.232345\\
8228	9.411179\\
8229	7.667754\\
8230	6.761594\\
8231	6.384058\\
8232	6.209125\\
8233	4.738495\\
8234	4.738495\\
8235	3.367334\\
8236	2.924284\\
8237	2.781129\\
8238	4.200863\\
8239	5.423951\\
8240	7.322726\\
8241	8.966486\\
8242	8.966486\\
8243	8.492277\\
8244	8.142721\\
8245	7.324468\\
8246	8.966482\\
8247	8.966486\\
8248	9.76845\\
8249	14.288415\\
8250	18.120207\\
8251	11.369485\\
8252	9.225226\\
8253	8.966486\\
8254	7.322726\\
8255	7.322726\\
8256	7.322726\\
8257	4.811915\\
8258	4.738497\\
8259	4.364027\\
8260	3.705065\\
8261	3.979011\\
8262	4.738497\\
8263	5.684331\\
8264	8.157763\\
8265	8.966486\\
8266	8.966487\\
8267	8.966487\\
8268	9.225226\\
8269	8.967809\\
8270	8.966486\\
8271	8.966513\\
8272	9.225219\\
8273	9.227158\\
8274	9.709156\\
8275	9.225227\\
8276	8.958871\\
8277	8.03104\\
8278	7.152145\\
8279	6.950533\\
8280	7.092446\\
8281	4.738496\\
8282	4.737379\\
8283	3.555764\\
8284	3.122031\\
8285	3.194289\\
8286	4.738496\\
8287	5.900659\\
8288	7.322726\\
8289	7.322726\\
8290	7.322726\\
8291	7.953433\\
8292	8.86831\\
8293	8.519507\\
8294	8.921419\\
8295	8.966482\\
8296	8.966484\\
8297	9.342149\\
8298	9.754324\\
8299	10.732048\\
8300	9.191086\\
8301	7.79835\\
8302	7.322726\\
8303	7.506986\\
8304	7.505996\\
8305	7.322726\\
8306	5.507399\\
8307	4.738496\\
8308	4.738497\\
8309	4.738497\\
8310	4.738498\\
8311	4.738496\\
8312	5.660601\\
8313	7.322726\\
8314	7.614207\\
8315	8.369858\\
8316	8.193879\\
8317	8.505114\\
8318	8.195668\\
8319	7.830922\\
8320	8.716046\\
8321	8.966486\\
8322	8.967111\\
8323	9.341201\\
8324	8.869173\\
8325	8.028347\\
8326	7.322726\\
8327	7.322726\\
8328	7.322726\\
8329	7.205452\\
8330	5.458946\\
8331	4.738498\\
8332	4.7385\\
8333	4.738499\\
8334	4.738498\\
8335	4.738499\\
8336	4.7385\\
8337	4.738498\\
8338	5.057407\\
8339	5.528841\\
8340	6.066567\\
8341	5.952932\\
8342	4.738496\\
8343	4.738496\\
8344	4.738496\\
8345	4.738496\\
8346	6.209125\\
8347	6.034498\\
8348	4.759382\\
8349	4.738517\\
8350	4.738501\\
8351	4.738501\\
8352	4.738497\\
8353	4.144504\\
8354	2.8712\\
8355	1.916247\\
8356	0.458607\\
8357	0.083471\\
8358	0.618532\\
8359	2.768142\\
8360	3.555489\\
8361	4.738497\\
8362	5.17598\\
8363	7.101779\\
8364	7.322726\\
8365	7.322726\\
8366	7.322726\\
8367	7.322725\\
8368	7.322726\\
8369	7.505995\\
8370	8.940979\\
8371	8.419047\\
8372	7.322726\\
8373	5.392392\\
8374	4.738496\\
8375	4.845066\\
8376	5.588221\\
8377	4.738496\\
8378	4.731872\\
8379	3.249229\\
8380	2.916537\\
8381	3.007528\\
8382	4.737418\\
8383	8.253663\\
8384	10.139513\\
8385	11.311742\\
8386	11.744745\\
8387	9.994369\\
8388	10.103383\\
8389	9.343091\\
8390	9.47749\\
8391	9.839739\\
8392	10.567195\\
8393	11.141831\\
8394	18.73773\\
8395	18.73773\\
8396	11.084041\\
8397	10.601612\\
8398	8.953963\\
8399	8.958354\\
8400	7.797744\\
8401	5.874179\\
8402	4.738496\\
8403	4.666098\\
8404	3.300517\\
8405	3.122071\\
8406	3.996915\\
8407	4.738496\\
8408	7.322726\\
8409	8.958887\\
8410	9.435819\\
8411	10.184827\\
8412	11.1037\\
8413	10.796081\\
8414	10.711927\\
8415	10.965146\\
8416	11.345617\\
8417	18.120208\\
8418	18.120202\\
8419	16.846011\\
8420	18.033063\\
8421	9.225227\\
8422	7.40477\\
8423	7.322726\\
8424	7.322726\\
8425	5.199532\\
8426	4.738496\\
8427	4.738495\\
8428	4.219194\\
8429	6.491947\\
8430	6.491966\\
8431	9.076196\\
8432	8.064241\\
8433	9.225227\\
8434	7.322726\\
8435	8.369143\\
8436	8.958879\\
8437	7.322726\\
8438	8.051742\\
8439	7.588069\\
8440	7.453612\\
8441	7.505999\\
8442	7.341738\\
8443	8.326174\\
8444	7.322726\\
8445	5.995077\\
8446	4.738496\\
8447	4.738497\\
8448	4.738496\\
8449	4.732199\\
8450	2.907682\\
8451	1.920786\\
8452	1.029048\\
8453	1.00358\\
8454	2.768142\\
8455	4.738496\\
8456	5.952745\\
8457	7.322726\\
8458	7.322726\\
8459	7.522209\\
8460	8.958876\\
8461	8.966517\\
8462	7.979556\\
8463	7.322726\\
8464	7.322726\\
8465	7.322726\\
8466	7.343582\\
8467	7.322822\\
8468	6.900213\\
8469	5.152626\\
8470	4.738496\\
8471	4.738496\\
8472	4.738497\\
8473	2.768545\\
8474	1.25491\\
8475	3.4e-05\\
8476	5e-05\\
8477	1.2e-05\\
8478	1.1e-05\\
8479	2.3e-05\\
8480	0.00017\\
8481	2.563427\\
8482	2.327059\\
8483	2.768139\\
8484	2.768139\\
8485	2.767191\\
8486	2.768143\\
8487	2.768139\\
8488	2.768141\\
8489	3.056915\\
8490	3.194299\\
8491	4.093708\\
8492	4.422341\\
8493	3.290011\\
8494	3.122022\\
8495	3.357547\\
8496	4.249488\\
8497	2.769495\\
8498	1.847536\\
8499	0.291084\\
8500	9e-06\\
8501	2e-06\\
8502	4e-06\\
8503	3e-05\\
8504	2.4e-05\\
8505	1.042292\\
8506	2.614923\\
8507	2.768141\\
8508	3.122004\\
8509	3.194299\\
8510	3.213205\\
8511	2.768139\\
8512	2.768141\\
8513	2.76814\\
8514	2.768139\\
8515	2.768139\\
8516	2.768139\\
8517	2.768139\\
8518	2.089764\\
8519	2.766914\\
8520	2.504916\\
8521	0.000143\\
8522	2e-06\\
8523	0\\
8524	1.3e-05\\
8525	3e-06\\
8526	1e-06\\
8527	0.546432\\
8528	2.768141\\
8529	4.738495\\
8530	4.7385\\
8531	4.738497\\
8532	4.738499\\
8533	4.738495\\
8534	4.738495\\
8535	4.204226\\
8536	3.680212\\
8537	3.837516\\
8538	3.723306\\
8539	4.307782\\
8540	2.81222\\
8541	2.747057\\
8542	2.157327\\
8543	2.768139\\
8544	2.767985\\
8545	1e-06\\
8546	8e-06\\
8547	5e-06\\
8548	0\\
8549	0\\
8550	0\\
8551	1.831974\\
8552	3.940077\\
8553	5.716072\\
8554	5.910432\\
8555	5.910433\\
8556	5.910434\\
8557	5.910434\\
8558	5.910434\\
8559	5.9093\\
8560	5.910433\\
8561	5.909065\\
8562	5.910434\\
8563	4.738495\\
8564	4.738495\\
8565	3.194294\\
8566	2.768124\\
8567	2.786626\\
8568	2.768145\\
8569	0.008868\\
8570	1e-06\\
8571	0\\
8572	0\\
8573	0\\
8574	0\\
8575	0\\
8576	6.9e-05\\
8577	1.763831\\
8578	2.768139\\
8579	4.125256\\
8580	4.144292\\
8581	4.189022\\
8582	5.66945\\
8583	5.589851\\
8584	5.669188\\
8585	5.244081\\
8586	3.657249\\
8587	3.66308\\
8588	3.231953\\
8589	2.510614\\
8590	1.815348\\
8591	3.231081\\
8592	5.201437\\
8593	4.698879\\
8594	3.231089\\
8595	1.559086\\
8596	0.147215\\
8597	3e-05\\
8598	3e-05\\
8599	0.000407\\
8600	0.000135\\
8601	0.565899\\
8602	1.237894\\
8603	2.414453\\
8604	3.231088\\
8605	3.231089\\
8606	3.231089\\
8607	3.231089\\
8608	3.231092\\
8609	3.560722\\
8610	3.657241\\
8611	4.048794\\
8612	4.987283\\
8613	5.201432\\
8614	5.201438\\
8615	5.201492\\
8616	5.201493\\
8617	5.201438\\
8618	4.044009\\
8619	3.231098\\
8620	3.231087\\
8621	3.231082\\
8622	3.668813\\
8623	5.201439\\
8624	5.20144\\
8625	5.641262\\
8626	6.672104\\
8627	7.2864\\
8628	7.194218\\
8629	6.884305\\
8630	6.730469\\
8631	7.330143\\
8632	7.411723\\
8633	7.696357\\
8634	6.964808\\
8635	6.705537\\
8636	5.646964\\
8637	5.201492\\
8638	5.201438\\
8639	5.201437\\
8640	5.201437\\
8641	4.591984\\
8642	3.231092\\
8643	1.579939\\
8644	0.463073\\
8645	0.322495\\
8646	0.729053\\
8647	1.688744\\
8648	3.048246\\
8649	3.676275\\
8650	5.201438\\
8651	5.201468\\
8652	5.201473\\
8653	5.201477\\
8654	5.201482\\
8655	5.201456\\
8656	5.201457\\
8657	5.201465\\
8658	5.201453\\
8659	5.201454\\
8660	5.201457\\
8661	5.201463\\
8662	5.20147\\
8663	5.201466\\
8664	5.201465\\
8665	4.712496\\
8666	3.328653\\
8667	2.866031\\
8668	1.23791\\
8669	0.753266\\
8670	0.753949\\
8671	1.473314\\
8672	1.727983\\
8673	3.23111\\
8674	3.889616\\
8675	4.511943\\
8676	5.001383\\
8677	5.201438\\
8678	5.201414\\
8679	5.201481\\
8680	5.201463\\
8681	5.68621\\
8682	7.459211\\
8683	7.785712\\
8684	7.615265\\
8685	7.7857\\
8686	7.490724\\
8687	7.289244\\
8688	8.494671\\
8689	7.38107\\
8690	5.910437\\
8691	5.910437\\
8692	4.889169\\
8693	4.366241\\
8694	4.596575\\
8695	5.910436\\
8696	5.910437\\
8697	7.38107\\
8698	6.672065\\
8699	6.672072\\
8700	6.672075\\
8701	6.582059\\
8702	6.671947\\
8703	6.440441\\
8704	6.672074\\
8705	7.287043\\
8706	7.785699\\
8707	8.342479\\
8708	8.02475\\
8709	7.785669\\
8710	7.308166\\
8711	7.785668\\
8712	7.785668\\
8713	6.188912\\
8714	5.201438\\
8715	5.201439\\
8716	4.711394\\
8717	4.560286\\
8718	5.201439\\
8719	5.201438\\
8720	5.662478\\
8721	7.344082\\
8722	7.785668\\
8723	7.785668\\
8724	7.785668\\
8725	7.785668\\
8726	7.785671\\
8727	7.785668\\
8728	7.785671\\
8729	7.904682\\
8730	8.90859\\
8731	8.493398\\
8732	7.785667\\
8733	7.785668\\
8734	6.672085\\
8735	7.785668\\
8736	7.785668\\
8737	5.66249\\
8738	5.20144\\
8739	5.20141\\
8740	4.508931\\
8741	4.149758\\
8742	4.712501\\
8743	5.20144\\
8744	5.201483\\
8745	6.672085\\
8746	7.604462\\
8747	6.672086\\
8748	6.672102\\
8749	6.377036\\
8750	6.672086\\
8751	6.447919\\
8752	6.672009\\
8753	6.971629\\
8754	6.910869\\
8755	6.672098\\
8756	5.201483\\
8757	5.201489\\
8758	5.201437\\
8759	5.201473\\
8760	5.948071\\
};
\end{axis}
\end{tikzpicture}%
    \caption{Predicted reserve prices for the ORDC model}
    \label{fig:ORDC_R2}
\end{figure}


\end{document}